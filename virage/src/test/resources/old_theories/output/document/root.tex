\documentclass[11pt,a4paper,notitlepage]{report}
\usepackage[T1]{fontenc}
\usepackage{lmodern}
\usepackage{sectsty}
\usepackage{xpatch}
\usepackage{upquote}

\makeatletter
\xpatchcmd{\@maketitle}
  {\@title}
{%\vspace{-2.45\baselineskip}
\bfseries\@title\vspace{.25\baselineskip}}
  {}{}
\makeatother

\makeatletter
\providecommand{\institute}[1]{%
    \apptocmd{\@author}{\end{tabular}%
    \vspace{.5\baselineskip}\par
    \begin{tabular}[t]{c}
    #1}{}{}
}
\makeatother

\newcommand{\email}[1]{%
    \texttt{\href{mailto:#1}{\color{black}{#1}}}%
}

\pretocmd{\tableofcontents}{\sectionfont{\Huge}}{}{}
%\let\clearpage\relax
\apptocmd{\tableofcontents}{\sectionfont{\Large}}{}{}

\usepackage{isabelle,isabellesym}
\usepackage{amssymb}
\usepackage{pdfsetup}

\urlstyle{rm}
\isabellestyle{it}

% for uniform font size
%\renewcommand{\isastyle}{\isastyleminor}

% This is required due to some unusual behaviour by Isabelle.
% It should be included in isabellesym.sty automatically, but it is not.
\newcommand{\isasymcirclearrowleft}{\isamath{\circlearrowleft}}

\begin{document}

\title{Verified Construction of Fair Voting Rules}
\author{Michael Kirsten}
\institute{Karlsruhe Institute of Technology (KIT), Karlsruhe, Germany\\
\email{kirsten@kit.edu}}
\maketitle

\begin{abstract}
Voting rules aggregate multiple individual preferences in order to make a
collective decision.
Commonly, these mechanisms are expected to respect a multitude of different
notions of fairness and reliability, which must be carefully balanced to avoid
inconsistencies.

This article contains a formalisation of a framework for the construction of
such fair voting rules using composable modules~\cite{adt2019,lopstr2019}.
The framework is a formal and systematic approach for the flexible and verified
construction of voting rules from individual composable modules to respect such
social-choice properties by construction.
Formal composition rules guarantee resulting social-choice properties from
properties of the individual components which are of generic nature to be
reused for various voting rules.
We provide proofs for a selected set of structures and composition rules.
The approach can be readily extended in order to support more voting rules,
e.g., from the literature by extending the sets of modules and composition
rules.
\end{abstract}

\tableofcontents

\parindent 0pt\parskip 0.5ex

%
\begin{isabellebody}%
\setisabellecontext{Preference{\isacharunderscore}{\kern0pt}Relation}%
%
\isadelimdocument
\isanewline
%
\endisadelimdocument
%
\isatagdocument
\isanewline
\isanewline
%
\isamarkupchapter{Social-Choice Types%
}
\isamarkuptrue%
%
\isamarkupsection{Preference Relation%
}
\isamarkuptrue%
%
\endisatagdocument
{\isafolddocument}%
%
\isadelimdocument
%
\endisadelimdocument
%
\isadelimtheory
%
\endisadelimtheory
%
\isatagtheory
\isacommand{theory}\isamarkupfalse%
\ Preference{\isacharunderscore}{\kern0pt}Relation\isanewline
\ \ \isakeyword{imports}\ Main\isanewline
\isakeyword{begin}%
\endisatagtheory
{\isafoldtheory}%
%
\isadelimtheory
%
\endisadelimtheory
%
\begin{isamarkuptext}%
The very core of the composable modules voting framework: types and
functions, derivations, lemmata, operations on preference relations, etc.%
\end{isamarkuptext}\isamarkuptrue%
%
\isadelimdocument
%
\endisadelimdocument
%
\isatagdocument
%
\isamarkupsubsection{Definition%
}
\isamarkuptrue%
%
\endisatagdocument
{\isafolddocument}%
%
\isadelimdocument
%
\endisadelimdocument
\isacommand{type{\isacharunderscore}{\kern0pt}synonym}\isamarkupfalse%
\ {\isacharprime}{\kern0pt}a\ Preference{\isacharunderscore}{\kern0pt}Relation\ {\isacharequal}{\kern0pt}\ {\isachardoublequoteopen}{\isacharprime}{\kern0pt}a\ rel{\isachardoublequoteclose}\isanewline
\isanewline
\isacommand{fun}\isamarkupfalse%
\ is{\isacharunderscore}{\kern0pt}less{\isacharunderscore}{\kern0pt}preferred{\isacharunderscore}{\kern0pt}than\ {\isacharcolon}{\kern0pt}{\isacharcolon}{\kern0pt}\isanewline
\ \ {\isachardoublequoteopen}{\isacharprime}{\kern0pt}a\ {\isasymRightarrow}\ {\isacharprime}{\kern0pt}a\ Preference{\isacharunderscore}{\kern0pt}Relation\ {\isasymRightarrow}\ {\isacharprime}{\kern0pt}a\ {\isasymRightarrow}\ bool{\isachardoublequoteclose}\ {\isacharparenleft}{\kern0pt}{\isachardoublequoteopen}{\isacharunderscore}{\kern0pt}\ {\isasympreceq}\isactrlsub {\isacharunderscore}{\kern0pt}\ {\isacharunderscore}{\kern0pt}{\isachardoublequoteclose}\ {\isacharbrackleft}{\kern0pt}{\isadigit{5}}{\isadigit{0}}{\isacharcomma}{\kern0pt}\ {\isadigit{1}}{\isadigit{0}}{\isadigit{0}}{\isadigit{0}}{\isacharcomma}{\kern0pt}\ {\isadigit{5}}{\isadigit{1}}{\isacharbrackright}{\kern0pt}\ {\isadigit{5}}{\isadigit{0}}{\isacharparenright}{\kern0pt}\ \isakeyword{where}\isanewline
\ \ \ \ {\isachardoublequoteopen}x\ {\isasympreceq}\isactrlsub r\ y\ {\isacharequal}{\kern0pt}\ {\isacharparenleft}{\kern0pt}{\isacharparenleft}{\kern0pt}x{\isacharcomma}{\kern0pt}\ y{\isacharparenright}{\kern0pt}\ {\isasymin}\ r{\isacharparenright}{\kern0pt}{\isachardoublequoteclose}\isanewline
\isanewline
\isacommand{lemma}\isamarkupfalse%
\ lin{\isacharunderscore}{\kern0pt}imp{\isacharunderscore}{\kern0pt}antisym{\isacharcolon}{\kern0pt}\isanewline
\ \ \isakeyword{assumes}\ {\isachardoublequoteopen}linear{\isacharunderscore}{\kern0pt}order{\isacharunderscore}{\kern0pt}on\ A\ r{\isachardoublequoteclose}\isanewline
\ \ \isakeyword{shows}\ {\isachardoublequoteopen}antisym\ r{\isachardoublequoteclose}\isanewline
%
\isadelimproof
\ \ %
\endisadelimproof
%
\isatagproof
\isacommand{using}\isamarkupfalse%
\ assms\ linear{\isacharunderscore}{\kern0pt}order{\isacharunderscore}{\kern0pt}on{\isacharunderscore}{\kern0pt}def\ partial{\isacharunderscore}{\kern0pt}order{\isacharunderscore}{\kern0pt}on{\isacharunderscore}{\kern0pt}def\isanewline
\ \ \isacommand{by}\isamarkupfalse%
\ auto%
\endisatagproof
{\isafoldproof}%
%
\isadelimproof
\isanewline
%
\endisadelimproof
\isanewline
\isacommand{lemma}\isamarkupfalse%
\ lin{\isacharunderscore}{\kern0pt}imp{\isacharunderscore}{\kern0pt}trans{\isacharcolon}{\kern0pt}\isanewline
\ \ \isakeyword{assumes}\ {\isachardoublequoteopen}linear{\isacharunderscore}{\kern0pt}order{\isacharunderscore}{\kern0pt}on\ A\ r{\isachardoublequoteclose}\isanewline
\ \ \isakeyword{shows}\ {\isachardoublequoteopen}trans\ r{\isachardoublequoteclose}\isanewline
%
\isadelimproof
\ \ %
\endisadelimproof
%
\isatagproof
\isacommand{using}\isamarkupfalse%
\ assms\ order{\isacharunderscore}{\kern0pt}on{\isacharunderscore}{\kern0pt}defs\isanewline
\ \ \isacommand{by}\isamarkupfalse%
\ blast%
\endisatagproof
{\isafoldproof}%
%
\isadelimproof
%
\endisadelimproof
%
\isadelimdocument
%
\endisadelimdocument
%
\isatagdocument
%
\isamarkupsubsection{Ranking%
}
\isamarkuptrue%
%
\endisatagdocument
{\isafolddocument}%
%
\isadelimdocument
%
\endisadelimdocument
\isacommand{fun}\isamarkupfalse%
\ rank\ {\isacharcolon}{\kern0pt}{\isacharcolon}{\kern0pt}\ {\isachardoublequoteopen}{\isacharprime}{\kern0pt}a\ Preference{\isacharunderscore}{\kern0pt}Relation\ {\isasymRightarrow}\ {\isacharprime}{\kern0pt}a\ {\isasymRightarrow}\ nat{\isachardoublequoteclose}\ \isakeyword{where}\isanewline
\ \ {\isachardoublequoteopen}rank\ r\ x\ {\isacharequal}{\kern0pt}\ card\ {\isacharparenleft}{\kern0pt}above\ r\ x{\isacharparenright}{\kern0pt}{\isachardoublequoteclose}\isanewline
\isanewline
\isacommand{lemma}\isamarkupfalse%
\ rank{\isacharunderscore}{\kern0pt}gt{\isacharunderscore}{\kern0pt}zero{\isacharcolon}{\kern0pt}\isanewline
\ \ \isakeyword{assumes}\isanewline
\ \ \ \ refl{\isacharcolon}{\kern0pt}\ {\isachardoublequoteopen}x\ {\isasympreceq}\isactrlsub r\ x{\isachardoublequoteclose}\ \isakeyword{and}\isanewline
\ \ \ \ fin{\isacharcolon}{\kern0pt}\ \ {\isachardoublequoteopen}finite\ r{\isachardoublequoteclose}\isanewline
\ \ \isakeyword{shows}\ {\isachardoublequoteopen}rank\ r\ x\ {\isasymge}\ {\isadigit{1}}{\isachardoublequoteclose}\isanewline
%
\isadelimproof
%
\endisadelimproof
%
\isatagproof
\isacommand{proof}\isamarkupfalse%
\ {\isacharminus}{\kern0pt}\isanewline
\ \ \isacommand{have}\isamarkupfalse%
\ {\isachardoublequoteopen}x\ {\isasymin}\ {\isacharbraceleft}{\kern0pt}y\ {\isasymin}\ Field\ r{\isachardot}{\kern0pt}\ {\isacharparenleft}{\kern0pt}x{\isacharcomma}{\kern0pt}\ y{\isacharparenright}{\kern0pt}\ {\isasymin}\ r{\isacharbraceright}{\kern0pt}{\isachardoublequoteclose}\isanewline
\ \ \ \ \isacommand{using}\isamarkupfalse%
\ FieldI{\isadigit{2}}\ refl\isanewline
\ \ \ \ \isacommand{by}\isamarkupfalse%
\ fastforce\isanewline
\ \ \isacommand{hence}\isamarkupfalse%
\ {\isachardoublequoteopen}{\isacharbraceleft}{\kern0pt}y\ {\isasymin}\ Field\ r{\isachardot}{\kern0pt}\ {\isacharparenleft}{\kern0pt}x{\isacharcomma}{\kern0pt}\ y{\isacharparenright}{\kern0pt}\ {\isasymin}\ r{\isacharbraceright}{\kern0pt}\ {\isasymnoteq}\ {\isacharbraceleft}{\kern0pt}{\isacharbraceright}{\kern0pt}{\isachardoublequoteclose}\isanewline
\ \ \ \ \isacommand{by}\isamarkupfalse%
\ blast\isanewline
\ \ \isacommand{hence}\isamarkupfalse%
\ {\isachardoublequoteopen}card\ {\isacharbraceleft}{\kern0pt}y\ {\isasymin}\ Field\ r{\isachardot}{\kern0pt}\ {\isacharparenleft}{\kern0pt}x{\isacharcomma}{\kern0pt}\ y{\isacharparenright}{\kern0pt}\ {\isasymin}\ r{\isacharbraceright}{\kern0pt}\ {\isasymnoteq}\ {\isadigit{0}}{\isachardoublequoteclose}\isanewline
\ \ \ \ \isacommand{by}\isamarkupfalse%
\ {\isacharparenleft}{\kern0pt}simp\ add{\isacharcolon}{\kern0pt}\ fin\ finite{\isacharunderscore}{\kern0pt}Field{\isacharparenright}{\kern0pt}\isanewline
\ \ \isacommand{moreover}\isamarkupfalse%
\ \isacommand{have}\isamarkupfalse%
\ {\isachardoublequoteopen}card{\isacharbraceleft}{\kern0pt}y\ {\isasymin}\ Field\ r{\isachardot}{\kern0pt}\ {\isacharparenleft}{\kern0pt}x{\isacharcomma}{\kern0pt}\ y{\isacharparenright}{\kern0pt}\ {\isasymin}\ r{\isacharbraceright}{\kern0pt}\ {\isasymge}\ {\isadigit{0}}{\isachardoublequoteclose}\isanewline
\ \ \ \ \isacommand{using}\isamarkupfalse%
\ fin\isanewline
\ \ \ \ \isacommand{by}\isamarkupfalse%
\ auto\isanewline
\ \ \isacommand{ultimately}\isamarkupfalse%
\ \isacommand{show}\isamarkupfalse%
\ {\isacharquery}{\kern0pt}thesis\isanewline
\ \ \ \ \isacommand{using}\isamarkupfalse%
\ Collect{\isacharunderscore}{\kern0pt}cong\ FieldI{\isadigit{2}}\ above{\isacharunderscore}{\kern0pt}def\isanewline
\ \ \ \ \ \ \ \ \ \ less{\isacharunderscore}{\kern0pt}one\ not{\isacharunderscore}{\kern0pt}le{\isacharunderscore}{\kern0pt}imp{\isacharunderscore}{\kern0pt}less\ rank{\isachardot}{\kern0pt}elims\isanewline
\ \ \ \ \isacommand{by}\isamarkupfalse%
\ {\isacharparenleft}{\kern0pt}metis\ {\isacharparenleft}{\kern0pt}no{\isacharunderscore}{\kern0pt}types{\isacharcomma}{\kern0pt}\ lifting{\isacharparenright}{\kern0pt}{\isacharparenright}{\kern0pt}\isanewline
\isacommand{qed}\isamarkupfalse%
%
\endisatagproof
{\isafoldproof}%
%
\isadelimproof
%
\endisadelimproof
%
\isadelimdocument
%
\endisadelimdocument
%
\isatagdocument
%
\isamarkupsubsection{Limited Preference%
}
\isamarkuptrue%
%
\endisatagdocument
{\isafolddocument}%
%
\isadelimdocument
%
\endisadelimdocument
\isacommand{definition}\isamarkupfalse%
\ limited\ {\isacharcolon}{\kern0pt}{\isacharcolon}{\kern0pt}\ {\isachardoublequoteopen}{\isacharprime}{\kern0pt}a\ set\ {\isasymRightarrow}\ {\isacharprime}{\kern0pt}a\ Preference{\isacharunderscore}{\kern0pt}Relation\ {\isasymRightarrow}\ bool{\isachardoublequoteclose}\ \isakeyword{where}\isanewline
\ \ {\isachardoublequoteopen}limited\ A\ r\ {\isasymequiv}\ r\ {\isasymsubseteq}\ A\ {\isasymtimes}\ A{\isachardoublequoteclose}\isanewline
\isanewline
\isacommand{lemma}\isamarkupfalse%
\ limitedI{\isacharcolon}{\kern0pt}\isanewline
\ \ {\isachardoublequoteopen}{\isacharparenleft}{\kern0pt}{\isasymAnd}x\ y{\isachardot}{\kern0pt}\ {\isasymlbrakk}\ x\ {\isasympreceq}\isactrlsub r\ y\ {\isasymrbrakk}\ {\isasymLongrightarrow}\ \ x\ {\isasymin}\ A\ {\isasymand}\ y\ {\isasymin}\ A{\isacharparenright}{\kern0pt}\ {\isasymLongrightarrow}\ limited\ A\ r{\isachardoublequoteclose}\isanewline
%
\isadelimproof
\ \ %
\endisadelimproof
%
\isatagproof
\isacommand{unfolding}\isamarkupfalse%
\ limited{\isacharunderscore}{\kern0pt}def\isanewline
\ \ \isacommand{by}\isamarkupfalse%
\ auto%
\endisatagproof
{\isafoldproof}%
%
\isadelimproof
\isanewline
%
\endisadelimproof
\isanewline
\isacommand{lemma}\isamarkupfalse%
\ limited{\isacharunderscore}{\kern0pt}dest{\isacharcolon}{\kern0pt}\isanewline
\ \ {\isachardoublequoteopen}{\isacharparenleft}{\kern0pt}{\isasymAnd}x\ y{\isachardot}{\kern0pt}\ {\isasymlbrakk}\ x\ {\isasympreceq}\isactrlsub r\ y{\isacharsemicolon}{\kern0pt}\ limited\ A\ r\ {\isasymrbrakk}\ {\isasymLongrightarrow}\ \ x\ {\isasymin}\ A\ {\isasymand}\ y\ {\isasymin}\ A{\isacharparenright}{\kern0pt}{\isachardoublequoteclose}\isanewline
%
\isadelimproof
\ \ %
\endisadelimproof
%
\isatagproof
\isacommand{unfolding}\isamarkupfalse%
\ limited{\isacharunderscore}{\kern0pt}def\isanewline
\ \ \isacommand{by}\isamarkupfalse%
\ auto%
\endisatagproof
{\isafoldproof}%
%
\isadelimproof
\isanewline
%
\endisadelimproof
\isanewline
\isacommand{fun}\isamarkupfalse%
\ limit\ {\isacharcolon}{\kern0pt}{\isacharcolon}{\kern0pt}\ {\isachardoublequoteopen}{\isacharprime}{\kern0pt}a\ set\ {\isasymRightarrow}\ {\isacharprime}{\kern0pt}a\ Preference{\isacharunderscore}{\kern0pt}Relation\ {\isasymRightarrow}\ {\isacharprime}{\kern0pt}a\ Preference{\isacharunderscore}{\kern0pt}Relation{\isachardoublequoteclose}\ \isakeyword{where}\isanewline
\ \ {\isachardoublequoteopen}limit\ A\ r\ {\isacharequal}{\kern0pt}\ {\isacharbraceleft}{\kern0pt}{\isacharparenleft}{\kern0pt}a{\isacharcomma}{\kern0pt}\ b{\isacharparenright}{\kern0pt}\ {\isasymin}\ r{\isachardot}{\kern0pt}\ a\ {\isasymin}\ A\ {\isasymand}\ b\ {\isasymin}\ A{\isacharbraceright}{\kern0pt}{\isachardoublequoteclose}\isanewline
\isanewline
\isacommand{definition}\isamarkupfalse%
\ connex\ {\isacharcolon}{\kern0pt}{\isacharcolon}{\kern0pt}\ {\isachardoublequoteopen}{\isacharprime}{\kern0pt}a\ set\ {\isasymRightarrow}\ {\isacharprime}{\kern0pt}a\ Preference{\isacharunderscore}{\kern0pt}Relation\ {\isasymRightarrow}\ bool{\isachardoublequoteclose}\ \isakeyword{where}\isanewline
\ \ {\isachardoublequoteopen}connex\ A\ r\ {\isasymequiv}\ limited\ A\ r\ {\isasymand}\ {\isacharparenleft}{\kern0pt}{\isasymforall}x\ {\isasymin}\ A{\isachardot}{\kern0pt}\ {\isasymforall}y\ {\isasymin}\ A{\isachardot}{\kern0pt}\ x\ {\isasympreceq}\isactrlsub r\ y\ {\isasymor}\ y\ {\isasympreceq}\isactrlsub r\ x{\isacharparenright}{\kern0pt}{\isachardoublequoteclose}\isanewline
\isanewline
\isacommand{lemma}\isamarkupfalse%
\ connex{\isacharunderscore}{\kern0pt}imp{\isacharunderscore}{\kern0pt}refl{\isacharcolon}{\kern0pt}\isanewline
\ \ \isakeyword{assumes}\ {\isachardoublequoteopen}connex\ A\ r{\isachardoublequoteclose}\isanewline
\ \ \isakeyword{shows}\ {\isachardoublequoteopen}refl{\isacharunderscore}{\kern0pt}on\ A\ r{\isachardoublequoteclose}\isanewline
%
\isadelimproof
%
\endisadelimproof
%
\isatagproof
\isacommand{proof}\isamarkupfalse%
\isanewline
\ \ \isacommand{show}\isamarkupfalse%
\ {\isachardoublequoteopen}r\ {\isasymsubseteq}\ A\ {\isasymtimes}\ A{\isachardoublequoteclose}\isanewline
\ \ \ \ \isacommand{using}\isamarkupfalse%
\ assms\ connex{\isacharunderscore}{\kern0pt}def\ limited{\isacharunderscore}{\kern0pt}def\isanewline
\ \ \ \ \isacommand{by}\isamarkupfalse%
\ metis\isanewline
\isacommand{next}\isamarkupfalse%
\isanewline
\ \ \isacommand{fix}\isamarkupfalse%
\isanewline
\ \ \ \ x\ {\isacharcolon}{\kern0pt}{\isacharcolon}{\kern0pt}\ {\isachardoublequoteopen}{\isacharprime}{\kern0pt}a{\isachardoublequoteclose}\isanewline
\ \ \isacommand{assume}\isamarkupfalse%
\isanewline
\ \ \ \ x{\isacharunderscore}{\kern0pt}in{\isacharunderscore}{\kern0pt}A{\isacharcolon}{\kern0pt}\ {\isachardoublequoteopen}x\ {\isasymin}\ A{\isachardoublequoteclose}\isanewline
\ \ \isacommand{have}\isamarkupfalse%
\ {\isachardoublequoteopen}x\ {\isasympreceq}\isactrlsub r\ x{\isachardoublequoteclose}\isanewline
\ \ \ \ \isacommand{using}\isamarkupfalse%
\ assms\ connex{\isacharunderscore}{\kern0pt}def\ x{\isacharunderscore}{\kern0pt}in{\isacharunderscore}{\kern0pt}A\isanewline
\ \ \ \ \isacommand{by}\isamarkupfalse%
\ metis\isanewline
\ \ \isacommand{thus}\isamarkupfalse%
\ {\isachardoublequoteopen}{\isacharparenleft}{\kern0pt}x{\isacharcomma}{\kern0pt}\ x{\isacharparenright}{\kern0pt}\ {\isasymin}\ r{\isachardoublequoteclose}\isanewline
\ \ \ \ \isacommand{by}\isamarkupfalse%
\ simp\isanewline
\isacommand{qed}\isamarkupfalse%
%
\endisatagproof
{\isafoldproof}%
%
\isadelimproof
\isanewline
%
\endisadelimproof
\isanewline
\isacommand{lemma}\isamarkupfalse%
\ lin{\isacharunderscore}{\kern0pt}ord{\isacharunderscore}{\kern0pt}imp{\isacharunderscore}{\kern0pt}connex{\isacharcolon}{\kern0pt}\isanewline
\ \ \isakeyword{assumes}\ {\isachardoublequoteopen}linear{\isacharunderscore}{\kern0pt}order{\isacharunderscore}{\kern0pt}on\ A\ r{\isachardoublequoteclose}\isanewline
\ \ \isakeyword{shows}\ {\isachardoublequoteopen}connex\ A\ r{\isachardoublequoteclose}\isanewline
%
\isadelimproof
\ \ %
\endisadelimproof
%
\isatagproof
\isacommand{unfolding}\isamarkupfalse%
\ connex{\isacharunderscore}{\kern0pt}def\ limited{\isacharunderscore}{\kern0pt}def\isanewline
\isacommand{proof}\isamarkupfalse%
\ {\isacharparenleft}{\kern0pt}safe{\isacharparenright}{\kern0pt}\isanewline
\ \ \isacommand{fix}\isamarkupfalse%
\isanewline
\ \ \ \ a\ {\isacharcolon}{\kern0pt}{\isacharcolon}{\kern0pt}\ {\isachardoublequoteopen}{\isacharprime}{\kern0pt}a{\isachardoublequoteclose}\ \isakeyword{and}\isanewline
\ \ \ \ b\ {\isacharcolon}{\kern0pt}{\isacharcolon}{\kern0pt}\ {\isachardoublequoteopen}{\isacharprime}{\kern0pt}a{\isachardoublequoteclose}\isanewline
\ \ \isacommand{assume}\isamarkupfalse%
\isanewline
\ \ \ \ asm{\isadigit{1}}{\isacharcolon}{\kern0pt}\ {\isachardoublequoteopen}{\isacharparenleft}{\kern0pt}a{\isacharcomma}{\kern0pt}\ b{\isacharparenright}{\kern0pt}\ {\isasymin}\ r{\isachardoublequoteclose}\isanewline
\ \ \isacommand{show}\isamarkupfalse%
\ {\isachardoublequoteopen}a\ {\isasymin}\ A{\isachardoublequoteclose}\isanewline
\ \ \ \ \isacommand{using}\isamarkupfalse%
\ asm{\isadigit{1}}\ assms\ partial{\isacharunderscore}{\kern0pt}order{\isacharunderscore}{\kern0pt}onD{\isacharparenleft}{\kern0pt}{\isadigit{1}}{\isacharparenright}{\kern0pt}\isanewline
\ \ \ \ \ \ \ \ \ \ order{\isacharunderscore}{\kern0pt}on{\isacharunderscore}{\kern0pt}defs{\isacharparenleft}{\kern0pt}{\isadigit{3}}{\isacharparenright}{\kern0pt}\ refl{\isacharunderscore}{\kern0pt}on{\isacharunderscore}{\kern0pt}domain\isanewline
\ \ \ \ \isacommand{by}\isamarkupfalse%
\ metis\isanewline
\isacommand{next}\isamarkupfalse%
\isanewline
\ \ \isacommand{fix}\isamarkupfalse%
\isanewline
\ \ \ \ a\ {\isacharcolon}{\kern0pt}{\isacharcolon}{\kern0pt}\ {\isachardoublequoteopen}{\isacharprime}{\kern0pt}a{\isachardoublequoteclose}\ \isakeyword{and}\isanewline
\ \ \ \ b\ {\isacharcolon}{\kern0pt}{\isacharcolon}{\kern0pt}\ {\isachardoublequoteopen}{\isacharprime}{\kern0pt}a{\isachardoublequoteclose}\isanewline
\ \ \isacommand{assume}\isamarkupfalse%
\isanewline
\ \ \ \ asm{\isadigit{1}}{\isacharcolon}{\kern0pt}\ {\isachardoublequoteopen}{\isacharparenleft}{\kern0pt}a{\isacharcomma}{\kern0pt}\ b{\isacharparenright}{\kern0pt}\ {\isasymin}\ r{\isachardoublequoteclose}\isanewline
\ \ \isacommand{show}\isamarkupfalse%
\ {\isachardoublequoteopen}b\ {\isasymin}\ A{\isachardoublequoteclose}\isanewline
\ \ \ \ \isacommand{using}\isamarkupfalse%
\ asm{\isadigit{1}}\ assms\ partial{\isacharunderscore}{\kern0pt}order{\isacharunderscore}{\kern0pt}onD{\isacharparenleft}{\kern0pt}{\isadigit{1}}{\isacharparenright}{\kern0pt}\isanewline
\ \ \ \ \ \ \ \ \ \ order{\isacharunderscore}{\kern0pt}on{\isacharunderscore}{\kern0pt}defs{\isacharparenleft}{\kern0pt}{\isadigit{3}}{\isacharparenright}{\kern0pt}\ refl{\isacharunderscore}{\kern0pt}on{\isacharunderscore}{\kern0pt}domain\isanewline
\ \ \ \ \isacommand{by}\isamarkupfalse%
\ metis\isanewline
\isacommand{next}\isamarkupfalse%
\isanewline
\ \ \isacommand{fix}\isamarkupfalse%
\isanewline
\ \ \ \ x\ {\isacharcolon}{\kern0pt}{\isacharcolon}{\kern0pt}\ {\isachardoublequoteopen}{\isacharprime}{\kern0pt}a{\isachardoublequoteclose}\ \isakeyword{and}\isanewline
\ \ \ \ y\ {\isacharcolon}{\kern0pt}{\isacharcolon}{\kern0pt}\ {\isachardoublequoteopen}{\isacharprime}{\kern0pt}a{\isachardoublequoteclose}\isanewline
\ \ \isacommand{assume}\isamarkupfalse%
\isanewline
\ \ \ \ asm{\isadigit{1}}{\isacharcolon}{\kern0pt}\ {\isachardoublequoteopen}x\ {\isasymin}\ A{\isachardoublequoteclose}\ \isakeyword{and}\isanewline
\ \ \ \ asm{\isadigit{2}}{\isacharcolon}{\kern0pt}\ {\isachardoublequoteopen}y\ {\isasymin}\ A{\isachardoublequoteclose}\ \isakeyword{and}\isanewline
\ \ \ \ asm{\isadigit{3}}{\isacharcolon}{\kern0pt}\ {\isachardoublequoteopen}{\isasymnot}\ y\ {\isasympreceq}\isactrlsub r\ x{\isachardoublequoteclose}\isanewline
\ \ \isacommand{have}\isamarkupfalse%
\ {\isachardoublequoteopen}{\isacharparenleft}{\kern0pt}y{\isacharcomma}{\kern0pt}\ x{\isacharparenright}{\kern0pt}\ {\isasymnotin}\ r{\isachardoublequoteclose}\isanewline
\ \ \ \ \isacommand{using}\isamarkupfalse%
\ asm{\isadigit{3}}\isanewline
\ \ \ \ \isacommand{by}\isamarkupfalse%
\ simp\isanewline
\ \ \isacommand{hence}\isamarkupfalse%
\ {\isachardoublequoteopen}{\isacharparenleft}{\kern0pt}x{\isacharcomma}{\kern0pt}\ y{\isacharparenright}{\kern0pt}\ {\isasymin}\ r{\isachardoublequoteclose}\isanewline
\ \ \ \ \isacommand{using}\isamarkupfalse%
\ asm{\isadigit{1}}\ asm{\isadigit{2}}\ assms\ partial{\isacharunderscore}{\kern0pt}order{\isacharunderscore}{\kern0pt}onD{\isacharparenleft}{\kern0pt}{\isadigit{1}}{\isacharparenright}{\kern0pt}\isanewline
\ \ \ \ \ \ \ \ \ \ linear{\isacharunderscore}{\kern0pt}order{\isacharunderscore}{\kern0pt}on{\isacharunderscore}{\kern0pt}def\ refl{\isacharunderscore}{\kern0pt}onD\ total{\isacharunderscore}{\kern0pt}on{\isacharunderscore}{\kern0pt}def\isanewline
\ \ \ \ \isacommand{by}\isamarkupfalse%
\ metis\isanewline
\ \ \isacommand{thus}\isamarkupfalse%
\ {\isachardoublequoteopen}x\ {\isasympreceq}\isactrlsub r\ y{\isachardoublequoteclose}\isanewline
\ \ \ \ \isacommand{by}\isamarkupfalse%
\ simp\isanewline
\isacommand{qed}\isamarkupfalse%
%
\endisatagproof
{\isafoldproof}%
%
\isadelimproof
\isanewline
%
\endisadelimproof
\isanewline
\isacommand{lemma}\isamarkupfalse%
\ connex{\isacharunderscore}{\kern0pt}antsym{\isacharunderscore}{\kern0pt}and{\isacharunderscore}{\kern0pt}trans{\isacharunderscore}{\kern0pt}imp{\isacharunderscore}{\kern0pt}lin{\isacharunderscore}{\kern0pt}ord{\isacharcolon}{\kern0pt}\isanewline
\ \ \isakeyword{assumes}\isanewline
\ \ \ \ connex{\isacharunderscore}{\kern0pt}r{\isacharcolon}{\kern0pt}\ {\isachardoublequoteopen}connex\ A\ r{\isachardoublequoteclose}\ \isakeyword{and}\isanewline
\ \ \ \ antisym{\isacharunderscore}{\kern0pt}r{\isacharcolon}{\kern0pt}\ {\isachardoublequoteopen}antisym\ r{\isachardoublequoteclose}\ \isakeyword{and}\isanewline
\ \ \ \ trans{\isacharunderscore}{\kern0pt}r{\isacharcolon}{\kern0pt}\ {\isachardoublequoteopen}trans\ r{\isachardoublequoteclose}\isanewline
\ \ \isakeyword{shows}\ {\isachardoublequoteopen}linear{\isacharunderscore}{\kern0pt}order{\isacharunderscore}{\kern0pt}on\ A\ r{\isachardoublequoteclose}\isanewline
%
\isadelimproof
\ \ %
\endisadelimproof
%
\isatagproof
\isacommand{unfolding}\isamarkupfalse%
\ connex{\isacharunderscore}{\kern0pt}def\ linear{\isacharunderscore}{\kern0pt}order{\isacharunderscore}{\kern0pt}on{\isacharunderscore}{\kern0pt}def\ partial{\isacharunderscore}{\kern0pt}order{\isacharunderscore}{\kern0pt}on{\isacharunderscore}{\kern0pt}def\isanewline
\ \ \ \ \ \ \ \ \ \ \ \ preorder{\isacharunderscore}{\kern0pt}on{\isacharunderscore}{\kern0pt}def\ refl{\isacharunderscore}{\kern0pt}on{\isacharunderscore}{\kern0pt}def\ total{\isacharunderscore}{\kern0pt}on{\isacharunderscore}{\kern0pt}def\isanewline
\isacommand{proof}\isamarkupfalse%
\ {\isacharparenleft}{\kern0pt}safe{\isacharparenright}{\kern0pt}\isanewline
\ \ \isacommand{fix}\isamarkupfalse%
\isanewline
\ \ \ \ a\ {\isacharcolon}{\kern0pt}{\isacharcolon}{\kern0pt}\ {\isachardoublequoteopen}{\isacharprime}{\kern0pt}a{\isachardoublequoteclose}\ \isakeyword{and}\isanewline
\ \ \ \ b\ {\isacharcolon}{\kern0pt}{\isacharcolon}{\kern0pt}\ {\isachardoublequoteopen}{\isacharprime}{\kern0pt}a{\isachardoublequoteclose}\isanewline
\ \ \isacommand{assume}\isamarkupfalse%
\isanewline
\ \ \ \ asm{\isadigit{1}}{\isacharcolon}{\kern0pt}\ {\isachardoublequoteopen}{\isacharparenleft}{\kern0pt}a{\isacharcomma}{\kern0pt}\ b{\isacharparenright}{\kern0pt}\ {\isasymin}\ r{\isachardoublequoteclose}\isanewline
\ \ \isacommand{show}\isamarkupfalse%
\ {\isachardoublequoteopen}a\ {\isasymin}\ A{\isachardoublequoteclose}\isanewline
\ \ \ \ \isacommand{using}\isamarkupfalse%
\ asm{\isadigit{1}}\ connex{\isacharunderscore}{\kern0pt}r\ refl{\isacharunderscore}{\kern0pt}on{\isacharunderscore}{\kern0pt}domain\ connex{\isacharunderscore}{\kern0pt}imp{\isacharunderscore}{\kern0pt}refl\isanewline
\ \ \ \ \isacommand{by}\isamarkupfalse%
\ metis\isanewline
\isacommand{next}\isamarkupfalse%
\isanewline
\ \ \isacommand{fix}\isamarkupfalse%
\isanewline
\ \ \ \ a\ {\isacharcolon}{\kern0pt}{\isacharcolon}{\kern0pt}\ {\isachardoublequoteopen}{\isacharprime}{\kern0pt}a{\isachardoublequoteclose}\ \isakeyword{and}\isanewline
\ \ \ \ b\ {\isacharcolon}{\kern0pt}{\isacharcolon}{\kern0pt}\ {\isachardoublequoteopen}{\isacharprime}{\kern0pt}a{\isachardoublequoteclose}\isanewline
\ \ \isacommand{assume}\isamarkupfalse%
\isanewline
\ \ \ \ asm{\isadigit{1}}{\isacharcolon}{\kern0pt}\ {\isachardoublequoteopen}{\isacharparenleft}{\kern0pt}a{\isacharcomma}{\kern0pt}\ b{\isacharparenright}{\kern0pt}\ {\isasymin}\ r{\isachardoublequoteclose}\isanewline
\ \ \isacommand{show}\isamarkupfalse%
\ {\isachardoublequoteopen}b\ {\isasymin}\ A{\isachardoublequoteclose}\isanewline
\ \ \ \ \isacommand{using}\isamarkupfalse%
\ asm{\isadigit{1}}\ connex{\isacharunderscore}{\kern0pt}r\ refl{\isacharunderscore}{\kern0pt}on{\isacharunderscore}{\kern0pt}domain\ connex{\isacharunderscore}{\kern0pt}imp{\isacharunderscore}{\kern0pt}refl\isanewline
\ \ \ \ \isacommand{by}\isamarkupfalse%
\ metis\isanewline
\isacommand{next}\isamarkupfalse%
\isanewline
\ \ \isacommand{fix}\isamarkupfalse%
\isanewline
\ \ \ \ x\ {\isacharcolon}{\kern0pt}{\isacharcolon}{\kern0pt}\ {\isachardoublequoteopen}{\isacharprime}{\kern0pt}a{\isachardoublequoteclose}\isanewline
\ \ \isacommand{assume}\isamarkupfalse%
\isanewline
\ \ \ \ asm{\isadigit{1}}{\isacharcolon}{\kern0pt}\ {\isachardoublequoteopen}x\ {\isasymin}\ A{\isachardoublequoteclose}\isanewline
\ \ \isacommand{show}\isamarkupfalse%
\ {\isachardoublequoteopen}{\isacharparenleft}{\kern0pt}x{\isacharcomma}{\kern0pt}\ x{\isacharparenright}{\kern0pt}\ {\isasymin}\ r{\isachardoublequoteclose}\isanewline
\ \ \ \ \isacommand{using}\isamarkupfalse%
\ asm{\isadigit{1}}\ connex{\isacharunderscore}{\kern0pt}r\ connex{\isacharunderscore}{\kern0pt}imp{\isacharunderscore}{\kern0pt}refl\ refl{\isacharunderscore}{\kern0pt}onD\isanewline
\ \ \isacommand{by}\isamarkupfalse%
\ metis\isanewline
\isacommand{next}\isamarkupfalse%
\isanewline
\ \ \isacommand{show}\isamarkupfalse%
\ {\isachardoublequoteopen}trans\ r{\isachardoublequoteclose}\isanewline
\ \ \ \ \isacommand{using}\isamarkupfalse%
\ trans{\isacharunderscore}{\kern0pt}r\isanewline
\ \ \ \ \isacommand{by}\isamarkupfalse%
\ simp\isanewline
\isacommand{next}\isamarkupfalse%
\isanewline
\ \ \isacommand{show}\isamarkupfalse%
\ {\isachardoublequoteopen}antisym\ r{\isachardoublequoteclose}\isanewline
\ \ \ \ \isacommand{using}\isamarkupfalse%
\ antisym{\isacharunderscore}{\kern0pt}r\isanewline
\ \ \ \ \isacommand{by}\isamarkupfalse%
\ simp\isanewline
\isacommand{next}\isamarkupfalse%
\isanewline
\ \ \isacommand{fix}\isamarkupfalse%
\isanewline
\ \ \ \ x\ {\isacharcolon}{\kern0pt}{\isacharcolon}{\kern0pt}\ {\isachardoublequoteopen}{\isacharprime}{\kern0pt}a{\isachardoublequoteclose}\ \isakeyword{and}\isanewline
\ \ \ \ y\ {\isacharcolon}{\kern0pt}{\isacharcolon}{\kern0pt}\ {\isachardoublequoteopen}{\isacharprime}{\kern0pt}a{\isachardoublequoteclose}\isanewline
\ \ \isacommand{assume}\isamarkupfalse%
\isanewline
\ \ \ \ asm{\isadigit{1}}{\isacharcolon}{\kern0pt}\ {\isachardoublequoteopen}x\ {\isasymin}\ A{\isachardoublequoteclose}\ \isakeyword{and}\isanewline
\ \ \ \ asm{\isadigit{2}}{\isacharcolon}{\kern0pt}\ {\isachardoublequoteopen}y\ {\isasymin}\ A{\isachardoublequoteclose}\ \isakeyword{and}\isanewline
\ \ \ \ asm{\isadigit{3}}{\isacharcolon}{\kern0pt}\ {\isachardoublequoteopen}x\ {\isasymnoteq}\ y{\isachardoublequoteclose}\ \isakeyword{and}\isanewline
\ \ \ \ asm{\isadigit{4}}{\isacharcolon}{\kern0pt}\ {\isachardoublequoteopen}{\isacharparenleft}{\kern0pt}y{\isacharcomma}{\kern0pt}\ x{\isacharparenright}{\kern0pt}\ {\isasymnotin}\ r{\isachardoublequoteclose}\isanewline
\ \ \isacommand{have}\isamarkupfalse%
\ {\isachardoublequoteopen}x\ {\isasympreceq}\isactrlsub r\ y\ {\isasymor}\ y\ {\isasympreceq}\isactrlsub r\ x{\isachardoublequoteclose}\isanewline
\ \ \ \ \isacommand{using}\isamarkupfalse%
\ asm{\isadigit{1}}\ asm{\isadigit{2}}\ connex{\isacharunderscore}{\kern0pt}r\ connex{\isacharunderscore}{\kern0pt}def\isanewline
\ \ \ \ \isacommand{by}\isamarkupfalse%
\ metis\isanewline
\ \ \isacommand{hence}\isamarkupfalse%
\ {\isachardoublequoteopen}{\isacharparenleft}{\kern0pt}x{\isacharcomma}{\kern0pt}\ y{\isacharparenright}{\kern0pt}\ {\isasymin}\ r\ {\isasymor}\ {\isacharparenleft}{\kern0pt}y{\isacharcomma}{\kern0pt}\ x{\isacharparenright}{\kern0pt}\ {\isasymin}\ r{\isachardoublequoteclose}\isanewline
\ \ \ \ \isacommand{by}\isamarkupfalse%
\ simp\isanewline
\ \ \isacommand{thus}\isamarkupfalse%
\ {\isachardoublequoteopen}{\isacharparenleft}{\kern0pt}x{\isacharcomma}{\kern0pt}\ y{\isacharparenright}{\kern0pt}\ {\isasymin}\ r{\isachardoublequoteclose}\isanewline
\ \ \ \ \isacommand{using}\isamarkupfalse%
\ asm{\isadigit{4}}\isanewline
\ \ \ \ \isacommand{by}\isamarkupfalse%
\ metis\isanewline
\isacommand{qed}\isamarkupfalse%
%
\endisatagproof
{\isafoldproof}%
%
\isadelimproof
\isanewline
%
\endisadelimproof
\isanewline
\isacommand{lemma}\isamarkupfalse%
\ limit{\isacharunderscore}{\kern0pt}to{\isacharunderscore}{\kern0pt}limits{\isacharcolon}{\kern0pt}\ {\isachardoublequoteopen}limited\ A\ {\isacharparenleft}{\kern0pt}limit\ A\ r{\isacharparenright}{\kern0pt}{\isachardoublequoteclose}\isanewline
%
\isadelimproof
\ \ %
\endisadelimproof
%
\isatagproof
\isacommand{unfolding}\isamarkupfalse%
\ limited{\isacharunderscore}{\kern0pt}def\isanewline
\ \ \isacommand{by}\isamarkupfalse%
\ auto%
\endisatagproof
{\isafoldproof}%
%
\isadelimproof
\isanewline
%
\endisadelimproof
\isanewline
\isacommand{lemma}\isamarkupfalse%
\ limit{\isacharunderscore}{\kern0pt}presv{\isacharunderscore}{\kern0pt}connex{\isacharcolon}{\kern0pt}\isanewline
\ \ \isakeyword{assumes}\isanewline
\ \ \ \ connex{\isacharcolon}{\kern0pt}\ {\isachardoublequoteopen}connex\ S\ r{\isachardoublequoteclose}\ \isakeyword{and}\isanewline
\ \ \ \ subset{\isacharcolon}{\kern0pt}\ {\isachardoublequoteopen}A\ {\isasymsubseteq}\ S{\isachardoublequoteclose}\isanewline
\ \ \isakeyword{shows}\ {\isachardoublequoteopen}connex\ A\ {\isacharparenleft}{\kern0pt}limit\ A\ r{\isacharparenright}{\kern0pt}{\isachardoublequoteclose}\isanewline
%
\isadelimproof
\ \ %
\endisadelimproof
%
\isatagproof
\isacommand{unfolding}\isamarkupfalse%
\ connex{\isacharunderscore}{\kern0pt}def\ limited{\isacharunderscore}{\kern0pt}def\isanewline
\isacommand{proof}\isamarkupfalse%
\ {\isacharparenleft}{\kern0pt}simp{\isacharcomma}{\kern0pt}\ safe{\isacharparenright}{\kern0pt}\isanewline
\ \ \isacommand{let}\isamarkupfalse%
\ {\isacharquery}{\kern0pt}s\ {\isacharequal}{\kern0pt}\ {\isachardoublequoteopen}{\isacharbraceleft}{\kern0pt}{\isacharparenleft}{\kern0pt}a{\isacharcomma}{\kern0pt}\ b{\isacharparenright}{\kern0pt}{\isachardot}{\kern0pt}\ {\isacharparenleft}{\kern0pt}a{\isacharcomma}{\kern0pt}\ b{\isacharparenright}{\kern0pt}\ {\isasymin}\ r\ {\isasymand}\ a\ {\isasymin}\ A\ {\isasymand}\ b\ {\isasymin}\ A{\isacharbraceright}{\kern0pt}{\isachardoublequoteclose}\isanewline
\ \ \isacommand{fix}\isamarkupfalse%
\isanewline
\ \ \ \ x\ {\isacharcolon}{\kern0pt}{\isacharcolon}{\kern0pt}\ {\isachardoublequoteopen}{\isacharprime}{\kern0pt}a{\isachardoublequoteclose}\ \isakeyword{and}\isanewline
\ \ \ \ y\ {\isacharcolon}{\kern0pt}{\isacharcolon}{\kern0pt}\ {\isachardoublequoteopen}{\isacharprime}{\kern0pt}a{\isachardoublequoteclose}\ \isakeyword{and}\isanewline
\ \ \ \ a\ {\isacharcolon}{\kern0pt}{\isacharcolon}{\kern0pt}\ {\isachardoublequoteopen}{\isacharprime}{\kern0pt}a{\isachardoublequoteclose}\ \isakeyword{and}\isanewline
\ \ \ \ b\ {\isacharcolon}{\kern0pt}{\isacharcolon}{\kern0pt}\ {\isachardoublequoteopen}{\isacharprime}{\kern0pt}a{\isachardoublequoteclose}\isanewline
\ \ \isacommand{assume}\isamarkupfalse%
\isanewline
\ \ \ \ asm{\isadigit{1}}{\isacharcolon}{\kern0pt}\ {\isachardoublequoteopen}x\ {\isasymin}\ A{\isachardoublequoteclose}\ \isakeyword{and}\isanewline
\ \ \ \ asm{\isadigit{2}}{\isacharcolon}{\kern0pt}\ {\isachardoublequoteopen}y\ {\isasymin}\ A{\isachardoublequoteclose}\ \isakeyword{and}\isanewline
\ \ \ \ asm{\isadigit{3}}{\isacharcolon}{\kern0pt}\ {\isachardoublequoteopen}{\isacharparenleft}{\kern0pt}y{\isacharcomma}{\kern0pt}\ x{\isacharparenright}{\kern0pt}\ {\isasymnotin}\ r{\isachardoublequoteclose}\isanewline
\ \ \isacommand{have}\isamarkupfalse%
\ {\isachardoublequoteopen}y\ {\isasympreceq}\isactrlsub r\ x\ {\isasymor}\ x\ {\isasympreceq}\isactrlsub r\ y{\isachardoublequoteclose}\isanewline
\ \ \ \ \isacommand{using}\isamarkupfalse%
\ asm{\isadigit{1}}\ asm{\isadigit{2}}\ connex\ connex{\isacharunderscore}{\kern0pt}def\ in{\isacharunderscore}{\kern0pt}mono\ subset\isanewline
\ \ \ \ \isacommand{by}\isamarkupfalse%
\ metis\isanewline
\ \ \isacommand{hence}\isamarkupfalse%
\isanewline
\ \ \ \ {\isachardoublequoteopen}x\ {\isasympreceq}\isactrlsub {\isacharquery}{\kern0pt}s\ y\ {\isasymor}\ y\ {\isasympreceq}\isactrlsub {\isacharquery}{\kern0pt}s\ x{\isachardoublequoteclose}\isanewline
\ \ \ \ \isacommand{using}\isamarkupfalse%
\ asm{\isadigit{1}}\ asm{\isadigit{2}}\isanewline
\ \ \ \ \isacommand{by}\isamarkupfalse%
\ auto\isanewline
\ \ \isacommand{hence}\isamarkupfalse%
\ {\isachardoublequoteopen}x\ {\isasympreceq}\isactrlsub {\isacharquery}{\kern0pt}s\ y{\isachardoublequoteclose}\isanewline
\ \ \ \ \isacommand{using}\isamarkupfalse%
\ asm{\isadigit{3}}\isanewline
\ \ \ \ \isacommand{by}\isamarkupfalse%
\ simp\isanewline
\ \ \isacommand{thus}\isamarkupfalse%
\ {\isachardoublequoteopen}{\isacharparenleft}{\kern0pt}x{\isacharcomma}{\kern0pt}\ y{\isacharparenright}{\kern0pt}\ {\isasymin}\ r{\isachardoublequoteclose}\isanewline
\ \ \ \ \isacommand{by}\isamarkupfalse%
\ simp\isanewline
\isacommand{qed}\isamarkupfalse%
%
\endisatagproof
{\isafoldproof}%
%
\isadelimproof
\isanewline
%
\endisadelimproof
\isanewline
\isacommand{lemma}\isamarkupfalse%
\ limit{\isacharunderscore}{\kern0pt}presv{\isacharunderscore}{\kern0pt}antisym{\isacharcolon}{\kern0pt}\isanewline
\ \ \isakeyword{assumes}\isanewline
\ \ \ \ antisymmetric{\isacharcolon}{\kern0pt}\ {\isachardoublequoteopen}antisym\ r{\isachardoublequoteclose}\ \isakeyword{and}\isanewline
\ \ \ \ subset{\isacharcolon}{\kern0pt}\ {\isachardoublequoteopen}A\ {\isasymsubseteq}\ S{\isachardoublequoteclose}\isanewline
\ \ \isakeyword{shows}\ {\isachardoublequoteopen}antisym\ {\isacharparenleft}{\kern0pt}limit\ A\ r{\isacharparenright}{\kern0pt}{\isachardoublequoteclose}\isanewline
%
\isadelimproof
\ \ %
\endisadelimproof
%
\isatagproof
\isacommand{using}\isamarkupfalse%
\ antisym{\isacharunderscore}{\kern0pt}def\ antisymmetric\isanewline
\ \ \isacommand{by}\isamarkupfalse%
\ auto%
\endisatagproof
{\isafoldproof}%
%
\isadelimproof
\isanewline
%
\endisadelimproof
\isanewline
\isacommand{lemma}\isamarkupfalse%
\ limit{\isacharunderscore}{\kern0pt}presv{\isacharunderscore}{\kern0pt}trans{\isacharcolon}{\kern0pt}\isanewline
\ \ \isakeyword{assumes}\isanewline
\ \ \ \ transitive{\isacharcolon}{\kern0pt}\ {\isachardoublequoteopen}trans\ r{\isachardoublequoteclose}\ \isakeyword{and}\isanewline
\ \ \ \ subset{\isacharcolon}{\kern0pt}\ \ \ \ \ {\isachardoublequoteopen}A\ {\isasymsubseteq}\ S{\isachardoublequoteclose}\isanewline
\ \ \isakeyword{shows}\ {\isachardoublequoteopen}trans\ {\isacharparenleft}{\kern0pt}limit\ A\ r{\isacharparenright}{\kern0pt}{\isachardoublequoteclose}\isanewline
%
\isadelimproof
\ \ %
\endisadelimproof
%
\isatagproof
\isacommand{unfolding}\isamarkupfalse%
\ trans{\isacharunderscore}{\kern0pt}def\isanewline
\isacommand{proof}\isamarkupfalse%
\ {\isacharparenleft}{\kern0pt}simp{\isacharcomma}{\kern0pt}\ safe{\isacharparenright}{\kern0pt}\isanewline
\ \ \isacommand{fix}\isamarkupfalse%
\isanewline
\ \ \ \ x\ {\isacharcolon}{\kern0pt}{\isacharcolon}{\kern0pt}\ {\isachardoublequoteopen}{\isacharprime}{\kern0pt}a{\isachardoublequoteclose}\ \isakeyword{and}\isanewline
\ \ \ \ y\ {\isacharcolon}{\kern0pt}{\isacharcolon}{\kern0pt}\ {\isachardoublequoteopen}{\isacharprime}{\kern0pt}a{\isachardoublequoteclose}\ \isakeyword{and}\isanewline
\ \ \ \ z\ {\isacharcolon}{\kern0pt}{\isacharcolon}{\kern0pt}\ {\isachardoublequoteopen}{\isacharprime}{\kern0pt}a{\isachardoublequoteclose}\isanewline
\ \ \isacommand{assume}\isamarkupfalse%
\isanewline
\ \ \ \ asm{\isadigit{1}}{\isacharcolon}{\kern0pt}\ {\isachardoublequoteopen}{\isacharparenleft}{\kern0pt}x{\isacharcomma}{\kern0pt}\ y{\isacharparenright}{\kern0pt}\ {\isasymin}\ r{\isachardoublequoteclose}\ \isakeyword{and}\isanewline
\ \ \ \ asm{\isadigit{2}}{\isacharcolon}{\kern0pt}\ {\isachardoublequoteopen}x\ {\isasymin}\ A{\isachardoublequoteclose}\ \isakeyword{and}\isanewline
\ \ \ \ asm{\isadigit{3}}{\isacharcolon}{\kern0pt}\ {\isachardoublequoteopen}y\ {\isasymin}\ A{\isachardoublequoteclose}\ \isakeyword{and}\isanewline
\ \ \ \ asm{\isadigit{4}}{\isacharcolon}{\kern0pt}\ {\isachardoublequoteopen}{\isacharparenleft}{\kern0pt}y{\isacharcomma}{\kern0pt}\ z{\isacharparenright}{\kern0pt}\ {\isasymin}\ r{\isachardoublequoteclose}\ \isakeyword{and}\isanewline
\ \ \ \ asm{\isadigit{5}}{\isacharcolon}{\kern0pt}\ {\isachardoublequoteopen}z\ {\isasymin}\ A{\isachardoublequoteclose}\isanewline
\ \ \isacommand{show}\isamarkupfalse%
\ \ {\isachardoublequoteopen}{\isacharparenleft}{\kern0pt}x{\isacharcomma}{\kern0pt}\ z{\isacharparenright}{\kern0pt}\ {\isasymin}\ r{\isachardoublequoteclose}\isanewline
\ \ \ \ \isacommand{using}\isamarkupfalse%
\ asm{\isadigit{1}}\ asm{\isadigit{4}}\ transE\ transitive\isanewline
\ \ \ \ \isacommand{by}\isamarkupfalse%
\ metis\isanewline
\isacommand{qed}\isamarkupfalse%
%
\endisatagproof
{\isafoldproof}%
%
\isadelimproof
\isanewline
%
\endisadelimproof
\isanewline
\isacommand{lemma}\isamarkupfalse%
\ limit{\isacharunderscore}{\kern0pt}presv{\isacharunderscore}{\kern0pt}lin{\isacharunderscore}{\kern0pt}ord{\isacharcolon}{\kern0pt}\isanewline
\ \ \isakeyword{assumes}\isanewline
\ \ \ \ {\isachardoublequoteopen}linear{\isacharunderscore}{\kern0pt}order{\isacharunderscore}{\kern0pt}on\ S\ r{\isachardoublequoteclose}\ \isakeyword{and}\isanewline
\ \ \ \ \ \ {\isachardoublequoteopen}A\ {\isasymsubseteq}\ S{\isachardoublequoteclose}\isanewline
\ \ \ \ \isakeyword{shows}\ {\isachardoublequoteopen}linear{\isacharunderscore}{\kern0pt}order{\isacharunderscore}{\kern0pt}on\ A\ {\isacharparenleft}{\kern0pt}limit\ A\ r{\isacharparenright}{\kern0pt}{\isachardoublequoteclose}\isanewline
%
\isadelimproof
\ \ %
\endisadelimproof
%
\isatagproof
\isacommand{using}\isamarkupfalse%
\ assms\ connex{\isacharunderscore}{\kern0pt}antsym{\isacharunderscore}{\kern0pt}and{\isacharunderscore}{\kern0pt}trans{\isacharunderscore}{\kern0pt}imp{\isacharunderscore}{\kern0pt}lin{\isacharunderscore}{\kern0pt}ord\isanewline
\ \ \ \ \ \ \ \ \ \ \ \ limit{\isacharunderscore}{\kern0pt}presv{\isacharunderscore}{\kern0pt}antisym\ limit{\isacharunderscore}{\kern0pt}presv{\isacharunderscore}{\kern0pt}connex\isanewline
\ \ \ \ \ \ \ \ \ \ \ \ limit{\isacharunderscore}{\kern0pt}presv{\isacharunderscore}{\kern0pt}trans\ lin{\isacharunderscore}{\kern0pt}ord{\isacharunderscore}{\kern0pt}imp{\isacharunderscore}{\kern0pt}connex\isanewline
\ \ \ \ \ \ \ \ \ \ \ \ order{\isacharunderscore}{\kern0pt}on{\isacharunderscore}{\kern0pt}defs{\isacharparenleft}{\kern0pt}{\isadigit{1}}{\isacharparenright}{\kern0pt}\ order{\isacharunderscore}{\kern0pt}on{\isacharunderscore}{\kern0pt}defs{\isacharparenleft}{\kern0pt}{\isadigit{2}}{\isacharparenright}{\kern0pt}\isanewline
\ \ \ \ \ \ \ \ \ \ \ \ order{\isacharunderscore}{\kern0pt}on{\isacharunderscore}{\kern0pt}defs{\isacharparenleft}{\kern0pt}{\isadigit{3}}{\isacharparenright}{\kern0pt}\isanewline
\ \ \isacommand{by}\isamarkupfalse%
\ metis%
\endisatagproof
{\isafoldproof}%
%
\isadelimproof
\isanewline
%
\endisadelimproof
\isanewline
\isacommand{lemma}\isamarkupfalse%
\ limit{\isacharunderscore}{\kern0pt}presv{\isacharunderscore}{\kern0pt}prefs{\isadigit{1}}{\isacharcolon}{\kern0pt}\isanewline
\ \ \isakeyword{assumes}\isanewline
\ \ \ \ x{\isacharunderscore}{\kern0pt}less{\isacharunderscore}{\kern0pt}y{\isacharcolon}{\kern0pt}\ {\isachardoublequoteopen}x\ {\isasympreceq}\isactrlsub r\ y{\isachardoublequoteclose}\ \isakeyword{and}\isanewline
\ \ \ \ x{\isacharunderscore}{\kern0pt}in{\isacharunderscore}{\kern0pt}A{\isacharcolon}{\kern0pt}\ {\isachardoublequoteopen}x\ {\isasymin}\ A{\isachardoublequoteclose}\ \isakeyword{and}\isanewline
\ \ \ \ y{\isacharunderscore}{\kern0pt}in{\isacharunderscore}{\kern0pt}A{\isacharcolon}{\kern0pt}\ {\isachardoublequoteopen}y\ {\isasymin}\ A{\isachardoublequoteclose}\isanewline
\ \ \isakeyword{shows}\ {\isachardoublequoteopen}let\ s\ {\isacharequal}{\kern0pt}\ limit\ A\ r\ in\ x\ {\isasympreceq}\isactrlsub s\ y{\isachardoublequoteclose}\isanewline
%
\isadelimproof
\ \ %
\endisadelimproof
%
\isatagproof
\isacommand{using}\isamarkupfalse%
\ x{\isacharunderscore}{\kern0pt}in{\isacharunderscore}{\kern0pt}A\ x{\isacharunderscore}{\kern0pt}less{\isacharunderscore}{\kern0pt}y\ y{\isacharunderscore}{\kern0pt}in{\isacharunderscore}{\kern0pt}A\isanewline
\ \ \isacommand{by}\isamarkupfalse%
\ simp%
\endisatagproof
{\isafoldproof}%
%
\isadelimproof
\isanewline
%
\endisadelimproof
\isanewline
\isacommand{lemma}\isamarkupfalse%
\ limit{\isacharunderscore}{\kern0pt}presv{\isacharunderscore}{\kern0pt}prefs{\isadigit{2}}{\isacharcolon}{\kern0pt}\isanewline
\ \ \isakeyword{assumes}\ x{\isacharunderscore}{\kern0pt}less{\isacharunderscore}{\kern0pt}y{\isacharcolon}{\kern0pt}\ {\isachardoublequoteopen}{\isacharparenleft}{\kern0pt}x{\isacharcomma}{\kern0pt}\ y{\isacharparenright}{\kern0pt}\ {\isasymin}\ limit\ A\ r{\isachardoublequoteclose}\isanewline
\ \ \isakeyword{shows}\ {\isachardoublequoteopen}x\ {\isasympreceq}\isactrlsub r\ y{\isachardoublequoteclose}\isanewline
%
\isadelimproof
\ \ %
\endisadelimproof
%
\isatagproof
\isacommand{using}\isamarkupfalse%
\ mem{\isacharunderscore}{\kern0pt}Collect{\isacharunderscore}{\kern0pt}eq\ x{\isacharunderscore}{\kern0pt}less{\isacharunderscore}{\kern0pt}y\isanewline
\ \ \isacommand{by}\isamarkupfalse%
\ auto%
\endisatagproof
{\isafoldproof}%
%
\isadelimproof
\isanewline
%
\endisadelimproof
\isanewline
\isacommand{lemma}\isamarkupfalse%
\ limit{\isacharunderscore}{\kern0pt}trans{\isacharcolon}{\kern0pt}\isanewline
\ \ \isakeyword{assumes}\isanewline
\ \ \ \ {\isachardoublequoteopen}B\ {\isasymsubseteq}\ A{\isachardoublequoteclose}\ \isakeyword{and}\isanewline
\ \ \ \ {\isachardoublequoteopen}C\ {\isasymsubseteq}\ B{\isachardoublequoteclose}\ \isakeyword{and}\isanewline
\ \ \ \ {\isachardoublequoteopen}linear{\isacharunderscore}{\kern0pt}order{\isacharunderscore}{\kern0pt}on\ A\ r{\isachardoublequoteclose}\isanewline
\ \ \isakeyword{shows}\ {\isachardoublequoteopen}limit\ C\ r\ {\isacharequal}{\kern0pt}\ limit\ C\ {\isacharparenleft}{\kern0pt}limit\ B\ r{\isacharparenright}{\kern0pt}{\isachardoublequoteclose}\isanewline
%
\isadelimproof
\ \ %
\endisadelimproof
%
\isatagproof
\isacommand{using}\isamarkupfalse%
\ assms\isanewline
\ \ \isacommand{by}\isamarkupfalse%
\ auto%
\endisatagproof
{\isafoldproof}%
%
\isadelimproof
\isanewline
%
\endisadelimproof
\isanewline
\isacommand{lemma}\isamarkupfalse%
\ lin{\isacharunderscore}{\kern0pt}ord{\isacharunderscore}{\kern0pt}not{\isacharunderscore}{\kern0pt}empty{\isacharcolon}{\kern0pt}\isanewline
\ \ \isakeyword{assumes}\ {\isachardoublequoteopen}r\ {\isasymnoteq}\ {\isacharbraceleft}{\kern0pt}{\isacharbraceright}{\kern0pt}{\isachardoublequoteclose}\isanewline
\ \ \isakeyword{shows}\ {\isachardoublequoteopen}{\isasymnot}\ linear{\isacharunderscore}{\kern0pt}order{\isacharunderscore}{\kern0pt}on\ {\isacharbraceleft}{\kern0pt}{\isacharbraceright}{\kern0pt}\ r{\isachardoublequoteclose}\isanewline
%
\isadelimproof
\ \ %
\endisadelimproof
%
\isatagproof
\isacommand{using}\isamarkupfalse%
\ assms\ connex{\isacharunderscore}{\kern0pt}imp{\isacharunderscore}{\kern0pt}refl\ lin{\isacharunderscore}{\kern0pt}ord{\isacharunderscore}{\kern0pt}imp{\isacharunderscore}{\kern0pt}connex\isanewline
\ \ \ \ \ \ \ \ refl{\isacharunderscore}{\kern0pt}on{\isacharunderscore}{\kern0pt}domain\ subrelI\isanewline
\ \ \isacommand{by}\isamarkupfalse%
\ fastforce%
\endisatagproof
{\isafoldproof}%
%
\isadelimproof
\isanewline
%
\endisadelimproof
\isanewline
\isacommand{lemma}\isamarkupfalse%
\ lin{\isacharunderscore}{\kern0pt}ord{\isacharunderscore}{\kern0pt}singleton{\isacharcolon}{\kern0pt}\isanewline
\ \ {\isachardoublequoteopen}{\isasymforall}r{\isachardot}{\kern0pt}\ linear{\isacharunderscore}{\kern0pt}order{\isacharunderscore}{\kern0pt}on\ {\isacharbraceleft}{\kern0pt}a{\isacharbraceright}{\kern0pt}\ r\ {\isasymlongrightarrow}\ r\ {\isacharequal}{\kern0pt}\ {\isacharbraceleft}{\kern0pt}{\isacharparenleft}{\kern0pt}a{\isacharcomma}{\kern0pt}\ a{\isacharparenright}{\kern0pt}{\isacharbraceright}{\kern0pt}{\isachardoublequoteclose}\isanewline
%
\isadelimproof
%
\endisadelimproof
%
\isatagproof
\isacommand{proof}\isamarkupfalse%
\isanewline
\ \ \isacommand{fix}\isamarkupfalse%
\ r\ {\isacharcolon}{\kern0pt}{\isacharcolon}{\kern0pt}\ {\isachardoublequoteopen}{\isacharprime}{\kern0pt}a\ Preference{\isacharunderscore}{\kern0pt}Relation{\isachardoublequoteclose}\isanewline
\ \ \isacommand{show}\isamarkupfalse%
\ {\isachardoublequoteopen}linear{\isacharunderscore}{\kern0pt}order{\isacharunderscore}{\kern0pt}on\ {\isacharbraceleft}{\kern0pt}a{\isacharbraceright}{\kern0pt}\ r\ {\isasymlongrightarrow}\ r\ {\isacharequal}{\kern0pt}\ {\isacharbraceleft}{\kern0pt}{\isacharparenleft}{\kern0pt}a{\isacharcomma}{\kern0pt}\ a{\isacharparenright}{\kern0pt}{\isacharbraceright}{\kern0pt}{\isachardoublequoteclose}\isanewline
\ \ \isacommand{proof}\isamarkupfalse%
\isanewline
\ \ \ \ \isacommand{assume}\isamarkupfalse%
\ asm{\isacharcolon}{\kern0pt}\ {\isachardoublequoteopen}linear{\isacharunderscore}{\kern0pt}order{\isacharunderscore}{\kern0pt}on\ {\isacharbraceleft}{\kern0pt}a{\isacharbraceright}{\kern0pt}\ r{\isachardoublequoteclose}\isanewline
\ \ \ \ \isacommand{hence}\isamarkupfalse%
\ {\isachardoublequoteopen}a\ {\isasympreceq}\isactrlsub r\ a{\isachardoublequoteclose}\isanewline
\ \ \ \ \ \ \isacommand{using}\isamarkupfalse%
\ connex{\isacharunderscore}{\kern0pt}def\ lin{\isacharunderscore}{\kern0pt}ord{\isacharunderscore}{\kern0pt}imp{\isacharunderscore}{\kern0pt}connex\ singletonI\isanewline
\ \ \ \ \ \ \isacommand{by}\isamarkupfalse%
\ metis\isanewline
\ \ \ \ \isacommand{moreover}\isamarkupfalse%
\ \isacommand{have}\isamarkupfalse%
\ {\isachardoublequoteopen}{\isasymforall}{\isacharparenleft}{\kern0pt}x{\isacharcomma}{\kern0pt}\ y{\isacharparenright}{\kern0pt}\ {\isasymin}\ r{\isachardot}{\kern0pt}\ x\ {\isacharequal}{\kern0pt}\ a\ {\isasymand}\ y\ {\isacharequal}{\kern0pt}\ a{\isachardoublequoteclose}\isanewline
\ \ \ \ \ \ \isacommand{using}\isamarkupfalse%
\ asm\ connex{\isacharunderscore}{\kern0pt}imp{\isacharunderscore}{\kern0pt}refl\ lin{\isacharunderscore}{\kern0pt}ord{\isacharunderscore}{\kern0pt}imp{\isacharunderscore}{\kern0pt}connex\isanewline
\ \ \ \ \ \ \ \ \ \ \ \ refl{\isacharunderscore}{\kern0pt}on{\isacharunderscore}{\kern0pt}domain\ split{\isacharunderscore}{\kern0pt}beta\isanewline
\ \ \ \ \ \ \isacommand{by}\isamarkupfalse%
\ fastforce\isanewline
\ \ \ \ \isacommand{ultimately}\isamarkupfalse%
\ \isacommand{show}\isamarkupfalse%
\ {\isachardoublequoteopen}r\ {\isacharequal}{\kern0pt}\ {\isacharbraceleft}{\kern0pt}{\isacharparenleft}{\kern0pt}a{\isacharcomma}{\kern0pt}\ a{\isacharparenright}{\kern0pt}{\isacharbraceright}{\kern0pt}{\isachardoublequoteclose}\isanewline
\ \ \ \ \ \ \isacommand{by}\isamarkupfalse%
\ auto\isanewline
\ \ \isacommand{qed}\isamarkupfalse%
\isanewline
\isacommand{qed}\isamarkupfalse%
%
\endisatagproof
{\isafoldproof}%
%
\isadelimproof
%
\endisadelimproof
%
\isadelimdocument
%
\endisadelimdocument
%
\isatagdocument
%
\isamarkupsubsection{Auxiliary Lemmata%
}
\isamarkuptrue%
%
\endisatagdocument
{\isafolddocument}%
%
\isadelimdocument
%
\endisadelimdocument
\isacommand{lemma}\isamarkupfalse%
\ above{\isacharunderscore}{\kern0pt}trans{\isacharcolon}{\kern0pt}\isanewline
\ \ \isakeyword{assumes}\isanewline
\ \ \ \ {\isachardoublequoteopen}trans\ r{\isachardoublequoteclose}\ \isakeyword{and}\isanewline
\ \ \ \ {\isachardoublequoteopen}{\isacharparenleft}{\kern0pt}a{\isacharcomma}{\kern0pt}\ b{\isacharparenright}{\kern0pt}\ {\isasymin}\ r{\isachardoublequoteclose}\isanewline
\ \ \isakeyword{shows}\ {\isachardoublequoteopen}above\ r\ b\ {\isasymsubseteq}\ above\ r\ a{\isachardoublequoteclose}\isanewline
%
\isadelimproof
\ \ %
\endisadelimproof
%
\isatagproof
\isacommand{using}\isamarkupfalse%
\ Collect{\isacharunderscore}{\kern0pt}mono\ above{\isacharunderscore}{\kern0pt}def\ assms\ transE\isanewline
\ \ \isacommand{by}\isamarkupfalse%
\ metis%
\endisatagproof
{\isafoldproof}%
%
\isadelimproof
\isanewline
%
\endisadelimproof
\isanewline
\isacommand{lemma}\isamarkupfalse%
\ above{\isacharunderscore}{\kern0pt}refl{\isacharcolon}{\kern0pt}\isanewline
\ \ \isakeyword{assumes}\isanewline
\ \ \ \ {\isachardoublequoteopen}refl{\isacharunderscore}{\kern0pt}on\ A\ r{\isachardoublequoteclose}\ \isakeyword{and}\isanewline
\ \ \ \ \ {\isachardoublequoteopen}a\ {\isasymin}\ A{\isachardoublequoteclose}\isanewline
\ \ \isakeyword{shows}\ {\isachardoublequoteopen}a\ {\isasymin}\ above\ r\ a{\isachardoublequoteclose}\isanewline
%
\isadelimproof
\ \ %
\endisadelimproof
%
\isatagproof
\isacommand{using}\isamarkupfalse%
\ above{\isacharunderscore}{\kern0pt}def\ assms\ refl{\isacharunderscore}{\kern0pt}onD\isanewline
\ \ \isacommand{by}\isamarkupfalse%
\ fastforce%
\endisatagproof
{\isafoldproof}%
%
\isadelimproof
\isanewline
%
\endisadelimproof
\isanewline
\isacommand{lemma}\isamarkupfalse%
\ above{\isacharunderscore}{\kern0pt}subset{\isacharunderscore}{\kern0pt}geq{\isacharunderscore}{\kern0pt}one{\isacharcolon}{\kern0pt}\isanewline
\ \ \isakeyword{assumes}\isanewline
\ \ \ \ {\isachardoublequoteopen}linear{\isacharunderscore}{\kern0pt}order{\isacharunderscore}{\kern0pt}on\ A\ r\ {\isasymand}\ linear{\isacharunderscore}{\kern0pt}order{\isacharunderscore}{\kern0pt}on\ A\ s{\isachardoublequoteclose}\ \isakeyword{and}\isanewline
\ \ \ \ {\isachardoublequoteopen}above\ r\ a\ {\isasymsubseteq}\ above\ s\ a{\isachardoublequoteclose}\ \isakeyword{and}\isanewline
\ \ \ \ {\isachardoublequoteopen}above\ s\ a\ {\isacharequal}{\kern0pt}\ {\isacharbraceleft}{\kern0pt}a{\isacharbraceright}{\kern0pt}{\isachardoublequoteclose}\isanewline
\ \ \isakeyword{shows}\ {\isachardoublequoteopen}above\ r\ a\ {\isacharequal}{\kern0pt}\ {\isacharbraceleft}{\kern0pt}a{\isacharbraceright}{\kern0pt}{\isachardoublequoteclose}\isanewline
%
\isadelimproof
\ \ %
\endisadelimproof
%
\isatagproof
\isacommand{using}\isamarkupfalse%
\ above{\isacharunderscore}{\kern0pt}def\ assms\ connex{\isacharunderscore}{\kern0pt}imp{\isacharunderscore}{\kern0pt}refl\ above{\isacharunderscore}{\kern0pt}refl\ insert{\isacharunderscore}{\kern0pt}absorb\isanewline
\ \ \ \ \ \ \ \ lin{\isacharunderscore}{\kern0pt}ord{\isacharunderscore}{\kern0pt}imp{\isacharunderscore}{\kern0pt}connex\ mem{\isacharunderscore}{\kern0pt}Collect{\isacharunderscore}{\kern0pt}eq\ refl{\isacharunderscore}{\kern0pt}on{\isacharunderscore}{\kern0pt}domain\isanewline
\ \ \ \ \ \ \ \ singletonI\ subset{\isacharunderscore}{\kern0pt}singletonD\isanewline
\ \ \isacommand{by}\isamarkupfalse%
\ metis%
\endisatagproof
{\isafoldproof}%
%
\isadelimproof
\isanewline
%
\endisadelimproof
\isanewline
\isacommand{lemma}\isamarkupfalse%
\ above{\isacharunderscore}{\kern0pt}connex{\isacharcolon}{\kern0pt}\isanewline
\ \ \isakeyword{assumes}\isanewline
\ \ \ \ {\isachardoublequoteopen}connex\ A\ r{\isachardoublequoteclose}\ \isakeyword{and}\isanewline
\ \ \ \ {\isachardoublequoteopen}a\ {\isasymin}\ A{\isachardoublequoteclose}\isanewline
\ \ \isakeyword{shows}\ {\isachardoublequoteopen}a\ {\isasymin}\ above\ r\ a{\isachardoublequoteclose}\isanewline
%
\isadelimproof
\ \ %
\endisadelimproof
%
\isatagproof
\isacommand{using}\isamarkupfalse%
\ assms\ connex{\isacharunderscore}{\kern0pt}imp{\isacharunderscore}{\kern0pt}refl\ above{\isacharunderscore}{\kern0pt}refl\isanewline
\ \ \isacommand{by}\isamarkupfalse%
\ metis%
\endisatagproof
{\isafoldproof}%
%
\isadelimproof
\isanewline
%
\endisadelimproof
\isanewline
\isacommand{lemma}\isamarkupfalse%
\ pref{\isacharunderscore}{\kern0pt}imp{\isacharunderscore}{\kern0pt}in{\isacharunderscore}{\kern0pt}above{\isacharcolon}{\kern0pt}\ {\isachardoublequoteopen}a\ {\isasympreceq}\isactrlsub r\ b\ {\isasymlongleftrightarrow}\ b\ {\isasymin}\ above\ r\ a{\isachardoublequoteclose}\isanewline
%
\isadelimproof
\ \ %
\endisadelimproof
%
\isatagproof
\isacommand{by}\isamarkupfalse%
\ {\isacharparenleft}{\kern0pt}simp\ add{\isacharcolon}{\kern0pt}\ above{\isacharunderscore}{\kern0pt}def{\isacharparenright}{\kern0pt}%
\endisatagproof
{\isafoldproof}%
%
\isadelimproof
\isanewline
%
\endisadelimproof
\isanewline
\isacommand{lemma}\isamarkupfalse%
\ limit{\isacharunderscore}{\kern0pt}presv{\isacharunderscore}{\kern0pt}above{\isacharcolon}{\kern0pt}\isanewline
\ \ \isakeyword{assumes}\isanewline
\ \ \ \ {\isachardoublequoteopen}b\ {\isasymin}\ above\ r\ a{\isachardoublequoteclose}\ \isakeyword{and}\isanewline
\ \ \ \ \isanewline
\ \ \ \ {\isachardoublequoteopen}a\ {\isasymin}\ B\ {\isasymand}\ b\ {\isasymin}\ B{\isachardoublequoteclose}\isanewline
\ \ \isakeyword{shows}\ {\isachardoublequoteopen}b\ {\isasymin}\ above\ {\isacharparenleft}{\kern0pt}limit\ B\ r{\isacharparenright}{\kern0pt}\ a{\isachardoublequoteclose}\isanewline
%
\isadelimproof
\ \ %
\endisadelimproof
%
\isatagproof
\isacommand{using}\isamarkupfalse%
\ pref{\isacharunderscore}{\kern0pt}imp{\isacharunderscore}{\kern0pt}in{\isacharunderscore}{\kern0pt}above\ assms\ limit{\isacharunderscore}{\kern0pt}presv{\isacharunderscore}{\kern0pt}prefs{\isadigit{1}}\isanewline
\ \ \isacommand{by}\isamarkupfalse%
\ metis%
\endisatagproof
{\isafoldproof}%
%
\isadelimproof
\isanewline
%
\endisadelimproof
\isanewline
\isacommand{lemma}\isamarkupfalse%
\ limit{\isacharunderscore}{\kern0pt}presv{\isacharunderscore}{\kern0pt}above{\isadigit{2}}{\isacharcolon}{\kern0pt}\isanewline
\ \ \isakeyword{assumes}\isanewline
\ \ \ \ {\isachardoublequoteopen}b\ {\isasymin}\ above\ {\isacharparenleft}{\kern0pt}limit\ B\ r{\isacharparenright}{\kern0pt}\ a{\isachardoublequoteclose}\ \isakeyword{and}\isanewline
\ \ \ \ {\isachardoublequoteopen}linear{\isacharunderscore}{\kern0pt}order{\isacharunderscore}{\kern0pt}on\ A\ r{\isachardoublequoteclose}\ \isakeyword{and}\isanewline
\ \ \ \ {\isachardoublequoteopen}B\ {\isasymsubseteq}\ A{\isachardoublequoteclose}\ \isakeyword{and}\isanewline
\ \ \ \ {\isachardoublequoteopen}a\ {\isasymin}\ B{\isachardoublequoteclose}\ \isakeyword{and}\isanewline
\ \ \ \ {\isachardoublequoteopen}b\ {\isasymin}\ B{\isachardoublequoteclose}\isanewline
\ \ \isakeyword{shows}\ {\isachardoublequoteopen}b\ {\isasymin}\ above\ r\ a{\isachardoublequoteclose}\isanewline
%
\isadelimproof
\ \ %
\endisadelimproof
%
\isatagproof
\isacommand{unfolding}\isamarkupfalse%
\ above{\isacharunderscore}{\kern0pt}def\isanewline
\ \ \isacommand{using}\isamarkupfalse%
\ above{\isacharunderscore}{\kern0pt}def\ assms{\isacharparenleft}{\kern0pt}{\isadigit{1}}{\isacharparenright}{\kern0pt}\ limit{\isacharunderscore}{\kern0pt}presv{\isacharunderscore}{\kern0pt}prefs{\isadigit{2}}\isanewline
\ \ \ \ \ \ \ \ mem{\isacharunderscore}{\kern0pt}Collect{\isacharunderscore}{\kern0pt}eq\ pref{\isacharunderscore}{\kern0pt}imp{\isacharunderscore}{\kern0pt}in{\isacharunderscore}{\kern0pt}above\isanewline
\ \ \isacommand{by}\isamarkupfalse%
\ metis%
\endisatagproof
{\isafoldproof}%
%
\isadelimproof
\isanewline
%
\endisadelimproof
\isanewline
\isacommand{lemma}\isamarkupfalse%
\ above{\isacharunderscore}{\kern0pt}one{\isacharcolon}{\kern0pt}\isanewline
\ \ \isakeyword{assumes}\isanewline
\ \ \ \ {\isachardoublequoteopen}linear{\isacharunderscore}{\kern0pt}order{\isacharunderscore}{\kern0pt}on\ A\ r{\isachardoublequoteclose}\ \isakeyword{and}\isanewline
\ \ \ \ {\isachardoublequoteopen}finite\ A\ {\isasymand}\ A\ {\isasymnoteq}\ {\isacharbraceleft}{\kern0pt}{\isacharbraceright}{\kern0pt}{\isachardoublequoteclose}\isanewline
\ \ \isakeyword{shows}\ {\isachardoublequoteopen}{\isasymexists}a{\isasymin}A{\isachardot}{\kern0pt}\ above\ r\ a\ {\isacharequal}{\kern0pt}\ {\isacharbraceleft}{\kern0pt}a{\isacharbraceright}{\kern0pt}\ {\isasymand}\ {\isacharparenleft}{\kern0pt}{\isasymforall}x{\isasymin}A{\isachardot}{\kern0pt}\ above\ r\ x\ {\isacharequal}{\kern0pt}\ {\isacharbraceleft}{\kern0pt}x{\isacharbraceright}{\kern0pt}\ {\isasymlongrightarrow}\ x\ {\isacharequal}{\kern0pt}\ a{\isacharparenright}{\kern0pt}{\isachardoublequoteclose}\isanewline
%
\isadelimproof
%
\endisadelimproof
%
\isatagproof
\isacommand{proof}\isamarkupfalse%
\ {\isacharminus}{\kern0pt}\isanewline
\ \ \isacommand{obtain}\isamarkupfalse%
\ n{\isacharcolon}{\kern0pt}{\isacharcolon}{\kern0pt}nat\ \isakeyword{where}\ n{\isacharcolon}{\kern0pt}\ {\isachardoublequoteopen}n{\isacharplus}{\kern0pt}{\isadigit{1}}\ {\isacharequal}{\kern0pt}\ card\ A{\isachardoublequoteclose}\isanewline
\ \ \ \ \isacommand{using}\isamarkupfalse%
\ Suc{\isacharunderscore}{\kern0pt}eq{\isacharunderscore}{\kern0pt}plus{\isadigit{1}}\ antisym{\isacharunderscore}{\kern0pt}conv{\isadigit{2}}\ assms{\isacharparenleft}{\kern0pt}{\isadigit{2}}{\isacharparenright}{\kern0pt}\ card{\isacharunderscore}{\kern0pt}eq{\isacharunderscore}{\kern0pt}{\isadigit{0}}{\isacharunderscore}{\kern0pt}iff\isanewline
\ \ \ \ \ \ \ \ \ \ gr{\isadigit{0}}{\isacharunderscore}{\kern0pt}implies{\isacharunderscore}{\kern0pt}Suc\ le{\isadigit{0}}\isanewline
\ \ \ \ \isacommand{by}\isamarkupfalse%
\ metis\isanewline
\ \ \isacommand{have}\isamarkupfalse%
\isanewline
\ \ \ \ {\isachardoublequoteopen}{\isacharparenleft}{\kern0pt}linear{\isacharunderscore}{\kern0pt}order{\isacharunderscore}{\kern0pt}on\ A\ r\ {\isasymand}\ finite\ A\ {\isasymand}\ A\ {\isasymnoteq}\ {\isacharbraceleft}{\kern0pt}{\isacharbraceright}{\kern0pt}\ {\isasymand}\ n{\isacharplus}{\kern0pt}{\isadigit{1}}\ {\isacharequal}{\kern0pt}\ card\ A{\isacharparenright}{\kern0pt}\isanewline
\ \ \ \ \ \ \ \ \ \ {\isasymlongrightarrow}\ {\isacharparenleft}{\kern0pt}{\isasymexists}a{\isachardot}{\kern0pt}\ a{\isasymin}A\ {\isasymand}\ above\ r\ a\ {\isacharequal}{\kern0pt}\ {\isacharbraceleft}{\kern0pt}a{\isacharbraceright}{\kern0pt}{\isacharparenright}{\kern0pt}{\isachardoublequoteclose}\isanewline
\ \ \isacommand{proof}\isamarkupfalse%
\ {\isacharparenleft}{\kern0pt}induction\ n\ arbitrary{\isacharcolon}{\kern0pt}\ A\ r{\isacharparenright}{\kern0pt}\isanewline
\ \ \ \ \isacommand{case}\isamarkupfalse%
\ {\isadigit{0}}\isanewline
\ \ \ \ \isacommand{show}\isamarkupfalse%
\ {\isacharquery}{\kern0pt}case\isanewline
\ \ \ \ \isacommand{proof}\isamarkupfalse%
\isanewline
\ \ \ \ \ \ \isacommand{assume}\isamarkupfalse%
\ asm{\isacharcolon}{\kern0pt}\ {\isachardoublequoteopen}linear{\isacharunderscore}{\kern0pt}order{\isacharunderscore}{\kern0pt}on\ A\ r\ {\isasymand}\ finite\ A\ {\isasymand}\ A\ {\isasymnoteq}\ {\isacharbraceleft}{\kern0pt}{\isacharbraceright}{\kern0pt}\ {\isasymand}\ {\isadigit{0}}{\isacharplus}{\kern0pt}{\isadigit{1}}\ {\isacharequal}{\kern0pt}\ card\ A{\isachardoublequoteclose}\isanewline
\ \ \ \ \ \ \isacommand{then}\isamarkupfalse%
\ \isacommand{obtain}\isamarkupfalse%
\ a\ \isakeyword{where}\ {\isachardoublequoteopen}{\isacharbraceleft}{\kern0pt}a{\isacharbraceright}{\kern0pt}\ {\isacharequal}{\kern0pt}\ A{\isachardoublequoteclose}\isanewline
\ \ \ \ \ \ \ \ \isacommand{using}\isamarkupfalse%
\ card{\isacharunderscore}{\kern0pt}{\isadigit{1}}{\isacharunderscore}{\kern0pt}singletonE\ add{\isachardot}{\kern0pt}left{\isacharunderscore}{\kern0pt}neutral\isanewline
\ \ \ \ \ \ \ \ \isacommand{by}\isamarkupfalse%
\ metis\isanewline
\ \ \ \ \ \ \isacommand{hence}\isamarkupfalse%
\ {\isachardoublequoteopen}a\ {\isasymin}\ A\ {\isasymand}\ above\ r\ a\ {\isacharequal}{\kern0pt}\ {\isacharbraceleft}{\kern0pt}a{\isacharbraceright}{\kern0pt}{\isachardoublequoteclose}\isanewline
\ \ \ \ \ \ \ \ \isacommand{using}\isamarkupfalse%
\ above{\isacharunderscore}{\kern0pt}def\ asm\ connex{\isacharunderscore}{\kern0pt}imp{\isacharunderscore}{\kern0pt}refl\ above{\isacharunderscore}{\kern0pt}refl\isanewline
\ \ \ \ \ \ \ \ \ \ \ \ \ \ lin{\isacharunderscore}{\kern0pt}ord{\isacharunderscore}{\kern0pt}imp{\isacharunderscore}{\kern0pt}connex\ refl{\isacharunderscore}{\kern0pt}on{\isacharunderscore}{\kern0pt}domain\isanewline
\ \ \ \ \ \ \ \ \isacommand{by}\isamarkupfalse%
\ fastforce\isanewline
\ \ \ \ \ \ \isacommand{thus}\isamarkupfalse%
\ {\isachardoublequoteopen}{\isasymexists}a{\isachardot}{\kern0pt}\ a{\isasymin}A\ {\isasymand}\ above\ r\ a\ {\isacharequal}{\kern0pt}\ {\isacharbraceleft}{\kern0pt}a{\isacharbraceright}{\kern0pt}{\isachardoublequoteclose}\isanewline
\ \ \ \ \ \ \ \ \isacommand{by}\isamarkupfalse%
\ auto\isanewline
\ \ \ \ \isacommand{qed}\isamarkupfalse%
\isanewline
\ \ \isacommand{next}\isamarkupfalse%
\isanewline
\ \ \ \ \isacommand{case}\isamarkupfalse%
\ {\isacharparenleft}{\kern0pt}Suc\ n{\isacharparenright}{\kern0pt}\isanewline
\ \ \ \ \isacommand{show}\isamarkupfalse%
\ {\isacharquery}{\kern0pt}case\isanewline
\ \ \ \ \isacommand{proof}\isamarkupfalse%
\isanewline
\ \ \ \ \ \ \isacommand{assume}\isamarkupfalse%
\ asm{\isacharcolon}{\kern0pt}\isanewline
\ \ \ \ \ \ \ \ {\isachardoublequoteopen}linear{\isacharunderscore}{\kern0pt}order{\isacharunderscore}{\kern0pt}on\ A\ r\ {\isasymand}\ finite\ A\ {\isasymand}\ A\ {\isasymnoteq}\ {\isacharbraceleft}{\kern0pt}{\isacharbraceright}{\kern0pt}\ {\isasymand}\ Suc\ n{\isacharplus}{\kern0pt}{\isadigit{1}}\ {\isacharequal}{\kern0pt}\ card\ A{\isachardoublequoteclose}\isanewline
\ \ \ \ \ \ \isacommand{then}\isamarkupfalse%
\ \isacommand{obtain}\isamarkupfalse%
\ B\ \isakeyword{where}\ B{\isacharcolon}{\kern0pt}\ {\isachardoublequoteopen}card\ B\ {\isacharequal}{\kern0pt}\ n{\isacharplus}{\kern0pt}{\isadigit{1}}\ {\isasymand}\ B\ {\isasymsubseteq}\ A{\isachardoublequoteclose}\isanewline
\ \ \ \ \ \ \ \ \isacommand{using}\isamarkupfalse%
\ Suc{\isacharunderscore}{\kern0pt}inject\ add{\isacharunderscore}{\kern0pt}Suc\ card{\isachardot}{\kern0pt}insert{\isacharunderscore}{\kern0pt}remove\ finite{\isachardot}{\kern0pt}cases\isanewline
\ \ \ \ \ \ \ \ \ \ \ \ \ \ insert{\isacharunderscore}{\kern0pt}Diff{\isacharunderscore}{\kern0pt}single\ subset{\isacharunderscore}{\kern0pt}insertI\isanewline
\ \ \ \ \ \ \ \ \isacommand{by}\isamarkupfalse%
\ {\isacharparenleft}{\kern0pt}metis\ {\isacharparenleft}{\kern0pt}mono{\isacharunderscore}{\kern0pt}tags{\isacharcomma}{\kern0pt}\ lifting{\isacharparenright}{\kern0pt}{\isacharparenright}{\kern0pt}\isanewline
\ \ \ \ \ \ \isacommand{then}\isamarkupfalse%
\ \isacommand{obtain}\isamarkupfalse%
\ a\ \isakeyword{where}\ a{\isacharcolon}{\kern0pt}\ {\isachardoublequoteopen}{\isacharbraceleft}{\kern0pt}a{\isacharbraceright}{\kern0pt}\ {\isacharequal}{\kern0pt}\ A\ {\isacharminus}{\kern0pt}\ B{\isachardoublequoteclose}\isanewline
\ \ \ \ \ \ \ \ \isacommand{using}\isamarkupfalse%
\ Suc{\isacharunderscore}{\kern0pt}eq{\isacharunderscore}{\kern0pt}plus{\isadigit{1}}\ add{\isacharunderscore}{\kern0pt}diff{\isacharunderscore}{\kern0pt}cancel{\isacharunderscore}{\kern0pt}left{\isacharprime}{\kern0pt}\ asm\ card{\isacharunderscore}{\kern0pt}{\isadigit{1}}{\isacharunderscore}{\kern0pt}singletonE\isanewline
\ \ \ \ \ \ \ \ \ \ \ \ \ \ card{\isacharunderscore}{\kern0pt}Diff{\isacharunderscore}{\kern0pt}subset\ finite{\isacharunderscore}{\kern0pt}subset\isanewline
\ \ \ \ \ \ \ \ \isacommand{by}\isamarkupfalse%
\ metis\isanewline
\ \ \ \ \ \ \isacommand{have}\isamarkupfalse%
\ {\isachardoublequoteopen}{\isasymexists}b{\isasymin}B{\isachardot}{\kern0pt}\ above\ {\isacharparenleft}{\kern0pt}limit\ B\ r{\isacharparenright}{\kern0pt}\ b\ {\isacharequal}{\kern0pt}\ {\isacharbraceleft}{\kern0pt}b{\isacharbraceright}{\kern0pt}{\isachardoublequoteclose}\isanewline
\ \ \ \ \ \ \ \ \isacommand{using}\isamarkupfalse%
\ B\ One{\isacharunderscore}{\kern0pt}nat{\isacharunderscore}{\kern0pt}def\ Suc{\isachardot}{\kern0pt}IH\ add{\isacharunderscore}{\kern0pt}diff{\isacharunderscore}{\kern0pt}cancel{\isacharunderscore}{\kern0pt}left{\isacharprime}{\kern0pt}\ asm\isanewline
\ \ \ \ \ \ \ \ \ \ \ \ \ \ card{\isacharunderscore}{\kern0pt}eq{\isacharunderscore}{\kern0pt}{\isadigit{0}}{\isacharunderscore}{\kern0pt}iff\ diff{\isacharunderscore}{\kern0pt}le{\isacharunderscore}{\kern0pt}self\ finite{\isacharunderscore}{\kern0pt}subset\ leD\ lessI\isanewline
\ \ \ \ \ \ \ \ \ \ \ \ \ \ limit{\isacharunderscore}{\kern0pt}presv{\isacharunderscore}{\kern0pt}lin{\isacharunderscore}{\kern0pt}ord\isanewline
\ \ \ \ \ \ \ \ \isacommand{by}\isamarkupfalse%
\ metis\isanewline
\ \ \ \ \ \ \isacommand{then}\isamarkupfalse%
\ \isacommand{obtain}\isamarkupfalse%
\ b\ \isakeyword{where}\ b{\isacharcolon}{\kern0pt}\ {\isachardoublequoteopen}above\ {\isacharparenleft}{\kern0pt}limit\ B\ r{\isacharparenright}{\kern0pt}\ b\ {\isacharequal}{\kern0pt}\ {\isacharbraceleft}{\kern0pt}b{\isacharbraceright}{\kern0pt}{\isachardoublequoteclose}\isanewline
\ \ \ \ \ \ \ \ \isacommand{by}\isamarkupfalse%
\ blast\isanewline
\ \ \ \ \ \ \isacommand{show}\isamarkupfalse%
\ {\isachardoublequoteopen}{\isasymexists}a{\isachardot}{\kern0pt}\ a\ {\isasymin}\ A\ {\isasymand}\ above\ r\ a\ {\isacharequal}{\kern0pt}\ {\isacharbraceleft}{\kern0pt}a{\isacharbraceright}{\kern0pt}{\isachardoublequoteclose}\isanewline
\ \ \ \ \ \ \isacommand{proof}\isamarkupfalse%
\ cases\isanewline
\ \ \ \ \ \ \ \ \isacommand{assume}\isamarkupfalse%
\isanewline
\ \ \ \ \ \ \ \ \ \ asm{\isadigit{1}}{\isacharcolon}{\kern0pt}\ {\isachardoublequoteopen}a\ {\isasympreceq}\isactrlsub r\ b{\isachardoublequoteclose}\isanewline
\ \ \ \ \ \ \ \ \isacommand{have}\isamarkupfalse%
\ f{\isadigit{1}}{\isacharcolon}{\kern0pt}\isanewline
\ \ \ \ \ \ \ \ \ \ {\isachardoublequoteopen}{\isasymforall}A\ r\ a\ aa{\isachardot}{\kern0pt}\isanewline
\ \ \ \ \ \ \ \ \ \ \ \ {\isasymnot}\ refl{\isacharunderscore}{\kern0pt}on\ A\ r\ {\isasymor}\ {\isacharparenleft}{\kern0pt}a{\isacharcolon}{\kern0pt}{\isacharcolon}{\kern0pt}{\isacharprime}{\kern0pt}a{\isacharcomma}{\kern0pt}\ aa{\isacharparenright}{\kern0pt}\ {\isasymnotin}\ r\ {\isasymor}\ a\ {\isasymin}\ A\ {\isasymand}\ aa\ {\isasymin}\ A{\isachardoublequoteclose}\isanewline
\ \ \ \ \ \ \ \ \ \ \isacommand{using}\isamarkupfalse%
\ refl{\isacharunderscore}{\kern0pt}on{\isacharunderscore}{\kern0pt}domain\isanewline
\ \ \ \ \ \ \ \ \ \ \isacommand{by}\isamarkupfalse%
\ metis\isanewline
\ \ \ \ \ \ \ \ \isacommand{have}\isamarkupfalse%
\ f{\isadigit{2}}{\isacharcolon}{\kern0pt}\isanewline
\ \ \ \ \ \ \ \ \ \ {\isachardoublequoteopen}{\isasymforall}A\ r{\isachardot}{\kern0pt}\ {\isasymnot}\ connex\ {\isacharparenleft}{\kern0pt}A{\isacharcolon}{\kern0pt}{\isacharcolon}{\kern0pt}{\isacharprime}{\kern0pt}a\ set{\isacharparenright}{\kern0pt}\ r\ {\isasymor}\ refl{\isacharunderscore}{\kern0pt}on\ A\ r{\isachardoublequoteclose}\isanewline
\ \ \ \ \ \ \ \ \ \ \isacommand{using}\isamarkupfalse%
\ connex{\isacharunderscore}{\kern0pt}imp{\isacharunderscore}{\kern0pt}refl\isanewline
\ \ \ \ \ \ \ \ \ \ \isacommand{by}\isamarkupfalse%
\ metis\isanewline
\ \ \ \ \ \ \ \ \isacommand{have}\isamarkupfalse%
\ f{\isadigit{3}}{\isacharcolon}{\kern0pt}\isanewline
\ \ \ \ \ \ \ \ \ \ {\isachardoublequoteopen}{\isasymforall}A\ r{\isachardot}{\kern0pt}\ {\isasymnot}\ linear{\isacharunderscore}{\kern0pt}order{\isacharunderscore}{\kern0pt}on\ {\isacharparenleft}{\kern0pt}A{\isacharcolon}{\kern0pt}{\isacharcolon}{\kern0pt}{\isacharprime}{\kern0pt}a\ set{\isacharparenright}{\kern0pt}\ r\ {\isasymor}\ connex\ A\ r{\isachardoublequoteclose}\isanewline
\ \ \ \ \ \ \ \ \ \ \isacommand{by}\isamarkupfalse%
\ {\isacharparenleft}{\kern0pt}simp\ add{\isacharcolon}{\kern0pt}\ lin{\isacharunderscore}{\kern0pt}ord{\isacharunderscore}{\kern0pt}imp{\isacharunderscore}{\kern0pt}connex{\isacharparenright}{\kern0pt}\isanewline
\ \ \ \ \ \ \ \ \isacommand{hence}\isamarkupfalse%
\ {\isachardoublequoteopen}refl{\isacharunderscore}{\kern0pt}on\ A\ r{\isachardoublequoteclose}\isanewline
\ \ \ \ \ \ \ \ \ \ \isacommand{using}\isamarkupfalse%
\ f{\isadigit{2}}\ asm\isanewline
\ \ \ \ \ \ \ \ \ \ \isacommand{by}\isamarkupfalse%
\ metis\isanewline
\ \ \ \ \ \ \ \ \isacommand{hence}\isamarkupfalse%
\ {\isachardoublequoteopen}a\ {\isasymin}\ A\ {\isasymand}\ b\ {\isasymin}\ A{\isachardoublequoteclose}\isanewline
\ \ \ \ \ \ \ \ \ \ \isacommand{using}\isamarkupfalse%
\ f{\isadigit{1}}\ asm{\isadigit{1}}\isanewline
\ \ \ \ \ \ \ \ \ \ \isacommand{by}\isamarkupfalse%
\ simp\isanewline
\ \ \ \ \ \ \ \ \isacommand{hence}\isamarkupfalse%
\ f{\isadigit{4}}{\isacharcolon}{\kern0pt}\isanewline
\ \ \ \ \ \ \ \ \ \ {\isachardoublequoteopen}{\isasymforall}a{\isachardot}{\kern0pt}\ a\ {\isasymnotin}\ A\ {\isasymor}\ b\ {\isacharequal}{\kern0pt}\ a\ {\isasymor}\ {\isacharparenleft}{\kern0pt}b{\isacharcomma}{\kern0pt}\ a{\isacharparenright}{\kern0pt}\ {\isasymin}\ r\ {\isasymor}\ {\isacharparenleft}{\kern0pt}a{\isacharcomma}{\kern0pt}\ b{\isacharparenright}{\kern0pt}\ {\isasymin}\ r{\isachardoublequoteclose}\isanewline
\ \ \ \ \ \ \ \ \ \ \isacommand{using}\isamarkupfalse%
\ asm\ order{\isacharunderscore}{\kern0pt}on{\isacharunderscore}{\kern0pt}defs{\isacharparenleft}{\kern0pt}{\isadigit{3}}{\isacharparenright}{\kern0pt}\ total{\isacharunderscore}{\kern0pt}on{\isacharunderscore}{\kern0pt}def\isanewline
\ \ \ \ \ \ \ \ \ \ \isacommand{by}\isamarkupfalse%
\ metis\isanewline
\ \ \ \ \ \ \ \ \isacommand{have}\isamarkupfalse%
\ f{\isadigit{5}}{\isacharcolon}{\kern0pt}\isanewline
\ \ \ \ \ \ \ \ \ \ {\isachardoublequoteopen}{\isacharparenleft}{\kern0pt}b{\isacharcomma}{\kern0pt}\ b{\isacharparenright}{\kern0pt}\ {\isasymin}\ limit\ B\ r{\isachardoublequoteclose}\isanewline
\ \ \ \ \ \ \ \ \ \ \isacommand{using}\isamarkupfalse%
\ above{\isacharunderscore}{\kern0pt}def\ b\ mem{\isacharunderscore}{\kern0pt}Collect{\isacharunderscore}{\kern0pt}eq\ singletonI\isanewline
\ \ \ \ \ \ \ \ \ \ \isacommand{by}\isamarkupfalse%
\ metis\isanewline
\ \ \ \ \ \ \ \ \isacommand{have}\isamarkupfalse%
\ f{\isadigit{6}}{\isacharcolon}{\kern0pt}\isanewline
\ \ \ \ \ \ \ \ \ \ {\isachardoublequoteopen}{\isasymforall}a\ A\ Aa{\isachardot}{\kern0pt}\ {\isacharparenleft}{\kern0pt}a{\isacharcolon}{\kern0pt}{\isacharcolon}{\kern0pt}{\isacharprime}{\kern0pt}a{\isacharparenright}{\kern0pt}\ {\isasymnotin}\ A\ {\isacharminus}{\kern0pt}\ Aa\ {\isasymor}\ a\ {\isasymin}\ A\ {\isasymand}\ a\ {\isasymnotin}\ Aa{\isachardoublequoteclose}\isanewline
\ \ \ \ \ \ \ \ \ \ \isacommand{by}\isamarkupfalse%
\ simp\isanewline
\ \ \ \ \ \ \ \ \isacommand{have}\isamarkupfalse%
\ ff{\isadigit{1}}{\isacharcolon}{\kern0pt}\isanewline
\ \ \ \ \ \ \ \ \ \ {\isachardoublequoteopen}{\isacharbraceleft}{\kern0pt}a{\isachardot}{\kern0pt}\ {\isacharparenleft}{\kern0pt}b{\isacharcomma}{\kern0pt}\ a{\isacharparenright}{\kern0pt}\ {\isasymin}\ limit\ B\ r{\isacharbraceright}{\kern0pt}\ {\isacharequal}{\kern0pt}\ {\isacharbraceleft}{\kern0pt}b{\isacharbraceright}{\kern0pt}{\isachardoublequoteclose}\isanewline
\ \ \ \ \ \ \ \ \ \ \isacommand{using}\isamarkupfalse%
\ above{\isacharunderscore}{\kern0pt}def\ b\isanewline
\ \ \ \ \ \ \ \ \ \ \isacommand{by}\isamarkupfalse%
\ {\isacharparenleft}{\kern0pt}metis\ {\isacharparenleft}{\kern0pt}no{\isacharunderscore}{\kern0pt}types{\isacharparenright}{\kern0pt}{\isacharparenright}{\kern0pt}\isanewline
\ \ \ \ \ \ \ \ \isacommand{have}\isamarkupfalse%
\ ff{\isadigit{2}}{\isacharcolon}{\kern0pt}\isanewline
\ \ \ \ \ \ \ \ \ \ {\isachardoublequoteopen}{\isacharparenleft}{\kern0pt}b{\isacharcomma}{\kern0pt}\ b{\isacharparenright}{\kern0pt}\ {\isasymin}\ {\isacharbraceleft}{\kern0pt}{\isacharparenleft}{\kern0pt}aa{\isacharcomma}{\kern0pt}\ a{\isacharparenright}{\kern0pt}{\isachardot}{\kern0pt}\ {\isacharparenleft}{\kern0pt}aa{\isacharcomma}{\kern0pt}\ a{\isacharparenright}{\kern0pt}\ {\isasymin}\ r\ {\isasymand}\ aa\ {\isasymin}\ B\ {\isasymand}\ a\ {\isasymin}\ B{\isacharbraceright}{\kern0pt}{\isachardoublequoteclose}\isanewline
\ \ \ \ \ \ \ \ \ \ \isacommand{using}\isamarkupfalse%
\ f{\isadigit{5}}\isanewline
\ \ \ \ \ \ \ \ \ \ \isacommand{by}\isamarkupfalse%
\ simp\isanewline
\ \ \ \ \ \ \ \ \isacommand{moreover}\isamarkupfalse%
\ \isacommand{have}\isamarkupfalse%
\ b{\isacharunderscore}{\kern0pt}wins{\isacharunderscore}{\kern0pt}B{\isacharcolon}{\kern0pt}\isanewline
\ \ \ \ \ \ \ \ \ \ {\isachardoublequoteopen}{\isasymforall}x\ {\isasymin}\ B{\isachardot}{\kern0pt}\ b\ {\isasymin}\ above\ r\ x{\isachardoublequoteclose}\isanewline
\ \ \ \ \ \ \ \ \ \ \isacommand{using}\isamarkupfalse%
\ B\ above{\isacharunderscore}{\kern0pt}def\ f{\isadigit{4}}\ ff{\isadigit{1}}\ ff{\isadigit{2}}\ CollectI\isanewline
\ \ \ \ \ \ \ \ \ \ \ \ \ \ \ \ Product{\isacharunderscore}{\kern0pt}Type{\isachardot}{\kern0pt}Collect{\isacharunderscore}{\kern0pt}case{\isacharunderscore}{\kern0pt}prodD\isanewline
\ \ \ \ \ \ \ \ \ \ \isacommand{by}\isamarkupfalse%
\ fastforce\isanewline
\ \ \ \ \ \ \ \ \isacommand{moreover}\isamarkupfalse%
\ \isacommand{have}\isamarkupfalse%
\ {\isachardoublequoteopen}b\ {\isasymin}\ above\ r\ a{\isachardoublequoteclose}\isanewline
\ \ \ \ \ \ \ \ \ \ \isacommand{using}\isamarkupfalse%
\ asm{\isadigit{1}}\ pref{\isacharunderscore}{\kern0pt}imp{\isacharunderscore}{\kern0pt}in{\isacharunderscore}{\kern0pt}above\isanewline
\ \ \ \ \ \ \ \ \ \ \isacommand{by}\isamarkupfalse%
\ metis\isanewline
\ \ \ \ \ \ \ \ \isacommand{ultimately}\isamarkupfalse%
\ \isacommand{have}\isamarkupfalse%
\ b{\isacharunderscore}{\kern0pt}wins{\isacharcolon}{\kern0pt}\isanewline
\ \ \ \ \ \ \ \ \ \ {\isachardoublequoteopen}{\isasymforall}x\ {\isasymin}\ A{\isachardot}{\kern0pt}\ b\ {\isasymin}\ above\ r\ x{\isachardoublequoteclose}\isanewline
\ \ \ \ \ \ \ \ \ \ \isacommand{using}\isamarkupfalse%
\ Diff{\isacharunderscore}{\kern0pt}iff\ a\ empty{\isacharunderscore}{\kern0pt}iff\ insert{\isacharunderscore}{\kern0pt}iff\isanewline
\ \ \ \ \ \ \ \ \ \ \isacommand{by}\isamarkupfalse%
\ {\isacharparenleft}{\kern0pt}metis\ {\isacharparenleft}{\kern0pt}no{\isacharunderscore}{\kern0pt}types{\isacharparenright}{\kern0pt}{\isacharparenright}{\kern0pt}\isanewline
\ \ \ \ \ \ \ \ \isacommand{hence}\isamarkupfalse%
\ {\isachardoublequoteopen}{\isasymforall}x\ {\isasymin}\ A{\isachardot}{\kern0pt}\ x\ {\isasymin}\ above\ r\ b\ {\isasymlongrightarrow}\ x\ {\isacharequal}{\kern0pt}\ b{\isachardoublequoteclose}\isanewline
\ \ \ \ \ \ \ \ \ \ \isacommand{using}\isamarkupfalse%
\ CollectD\ above{\isacharunderscore}{\kern0pt}def\ antisym{\isacharunderscore}{\kern0pt}def\ asm\ lin{\isacharunderscore}{\kern0pt}imp{\isacharunderscore}{\kern0pt}antisym\isanewline
\ \ \ \ \ \ \ \ \ \ \isacommand{by}\isamarkupfalse%
\ metis\isanewline
\ \ \ \ \ \ \ \ \isacommand{hence}\isamarkupfalse%
\ {\isachardoublequoteopen}{\isasymforall}x\ {\isasymin}\ A{\isachardot}{\kern0pt}\ x\ {\isasymin}\ above\ r\ b\ {\isasymlongleftrightarrow}\ x\ {\isacharequal}{\kern0pt}\ b{\isachardoublequoteclose}\isanewline
\ \ \ \ \ \ \ \ \ \ \isacommand{using}\isamarkupfalse%
\ b{\isacharunderscore}{\kern0pt}wins\isanewline
\ \ \ \ \ \ \ \ \ \ \isacommand{by}\isamarkupfalse%
\ blast\isanewline
\ \ \ \ \ \ \ \ \isacommand{moreover}\isamarkupfalse%
\ \isacommand{have}\isamarkupfalse%
\ above{\isacharunderscore}{\kern0pt}b{\isacharunderscore}{\kern0pt}in{\isacharunderscore}{\kern0pt}A{\isacharcolon}{\kern0pt}\ {\isachardoublequoteopen}above\ r\ b\ {\isasymsubseteq}\ A{\isachardoublequoteclose}\isanewline
\ \ \ \ \ \ \ \ \ \ \isacommand{using}\isamarkupfalse%
\ above{\isacharunderscore}{\kern0pt}def\ asm\ connex{\isacharunderscore}{\kern0pt}imp{\isacharunderscore}{\kern0pt}refl\ lin{\isacharunderscore}{\kern0pt}ord{\isacharunderscore}{\kern0pt}imp{\isacharunderscore}{\kern0pt}connex\isanewline
\ \ \ \ \ \ \ \ \ \ \ \ \ \ \ \ mem{\isacharunderscore}{\kern0pt}Collect{\isacharunderscore}{\kern0pt}eq\ refl{\isacharunderscore}{\kern0pt}on{\isacharunderscore}{\kern0pt}domain\ subsetI\isanewline
\ \ \ \ \ \ \ \ \ \ \isacommand{by}\isamarkupfalse%
\ metis\isanewline
\ \ \ \ \ \ \ \ \isacommand{ultimately}\isamarkupfalse%
\ \isacommand{have}\isamarkupfalse%
\ {\isachardoublequoteopen}above\ r\ b\ {\isacharequal}{\kern0pt}\ {\isacharbraceleft}{\kern0pt}b{\isacharbraceright}{\kern0pt}{\isachardoublequoteclose}\isanewline
\ \ \ \ \ \ \ \ \ \ \isacommand{using}\isamarkupfalse%
\ above{\isacharunderscore}{\kern0pt}def\ b\isanewline
\ \ \ \ \ \ \ \ \ \ \isacommand{by}\isamarkupfalse%
\ fastforce\isanewline
\ \ \ \ \ \ \ \ \isacommand{thus}\isamarkupfalse%
\ {\isacharquery}{\kern0pt}thesis\isanewline
\ \ \ \ \ \ \ \ \ \ \isacommand{using}\isamarkupfalse%
\ above{\isacharunderscore}{\kern0pt}b{\isacharunderscore}{\kern0pt}in{\isacharunderscore}{\kern0pt}A\isanewline
\ \ \ \ \ \ \ \ \ \ \isacommand{by}\isamarkupfalse%
\ blast\isanewline
\ \ \ \ \ \ \isacommand{next}\isamarkupfalse%
\isanewline
\ \ \ \ \ \ \ \ \isacommand{assume}\isamarkupfalse%
\ {\isachardoublequoteopen}{\isasymnot}a\ {\isasympreceq}\isactrlsub r\ b{\isachardoublequoteclose}\isanewline
\ \ \ \ \ \ \ \ \isacommand{hence}\isamarkupfalse%
\ b{\isacharunderscore}{\kern0pt}smaller{\isacharunderscore}{\kern0pt}a{\isacharcolon}{\kern0pt}\ {\isachardoublequoteopen}b\ {\isasympreceq}\isactrlsub r\ a{\isachardoublequoteclose}\isanewline
\ \ \ \ \ \ \ \ \ \ \isacommand{using}\isamarkupfalse%
\ B\ DiffE\ a\ asm\ b\ limit{\isacharunderscore}{\kern0pt}to{\isacharunderscore}{\kern0pt}limits\ connex{\isacharunderscore}{\kern0pt}def\isanewline
\ \ \ \ \ \ \ \ \ \ \ \ \ \ \ \ limited{\isacharunderscore}{\kern0pt}dest\ singletonI\ subset{\isacharunderscore}{\kern0pt}iff\isanewline
\ \ \ \ \ \ \ \ \ \ \ \ \ \ \ \ lin{\isacharunderscore}{\kern0pt}ord{\isacharunderscore}{\kern0pt}imp{\isacharunderscore}{\kern0pt}connex\ pref{\isacharunderscore}{\kern0pt}imp{\isacharunderscore}{\kern0pt}in{\isacharunderscore}{\kern0pt}above\isanewline
\ \ \ \ \ \ \ \ \ \ \isacommand{by}\isamarkupfalse%
\ metis\isanewline
\ \ \ \ \ \ \ \ \isacommand{hence}\isamarkupfalse%
\ b{\isacharunderscore}{\kern0pt}smaller{\isacharunderscore}{\kern0pt}a{\isacharunderscore}{\kern0pt}{\isadigit{0}}{\isacharcolon}{\kern0pt}\ {\isachardoublequoteopen}{\isacharparenleft}{\kern0pt}b{\isacharcomma}{\kern0pt}\ a{\isacharparenright}{\kern0pt}\ {\isasymin}\ r{\isachardoublequoteclose}\isanewline
\ \ \ \ \ \ \ \ \ \ \isacommand{by}\isamarkupfalse%
\ simp\isanewline
\ \ \ \ \ \ \ \ \isacommand{have}\isamarkupfalse%
\ g{\isadigit{1}}{\isacharcolon}{\kern0pt}\isanewline
\ \ \ \ \ \ \ \ \ \ {\isachardoublequoteopen}{\isasymforall}A\ r\ Aa{\isachardot}{\kern0pt}\isanewline
\ \ \ \ \ \ \ \ \ \ \ \ {\isasymnot}\ linear{\isacharunderscore}{\kern0pt}order{\isacharunderscore}{\kern0pt}on\ {\isacharparenleft}{\kern0pt}A{\isacharcolon}{\kern0pt}{\isacharcolon}{\kern0pt}{\isacharprime}{\kern0pt}a\ set{\isacharparenright}{\kern0pt}\ r\ {\isasymor}\isanewline
\ \ \ \ \ \ \ \ \ \ \ \ \ \ {\isasymnot}\ Aa\ {\isasymsubseteq}\ A\ {\isasymor}\isanewline
\ \ \ \ \ \ \ \ \ \ \ \ \ \ linear{\isacharunderscore}{\kern0pt}order{\isacharunderscore}{\kern0pt}on\ Aa\ {\isacharparenleft}{\kern0pt}limit\ Aa\ r{\isacharparenright}{\kern0pt}{\isachardoublequoteclose}\isanewline
\ \ \ \ \ \ \ \ \ \ \isacommand{using}\isamarkupfalse%
\ limit{\isacharunderscore}{\kern0pt}presv{\isacharunderscore}{\kern0pt}lin{\isacharunderscore}{\kern0pt}ord\isanewline
\ \ \ \ \ \ \ \ \ \ \isacommand{by}\isamarkupfalse%
\ metis\isanewline
\ \ \ \ \ \ \ \ \isacommand{have}\isamarkupfalse%
\isanewline
\ \ \ \ \ \ \ \ \ \ {\isachardoublequoteopen}{\isacharbraceleft}{\kern0pt}a{\isachardot}{\kern0pt}\ {\isacharparenleft}{\kern0pt}b{\isacharcomma}{\kern0pt}\ a{\isacharparenright}{\kern0pt}\ {\isasymin}\ limit\ B\ r{\isacharbraceright}{\kern0pt}\ {\isacharequal}{\kern0pt}\ {\isacharbraceleft}{\kern0pt}b{\isacharbraceright}{\kern0pt}{\isachardoublequoteclose}\isanewline
\ \ \ \ \ \ \ \ \ \ \isacommand{using}\isamarkupfalse%
\ above{\isacharunderscore}{\kern0pt}def\ b\isanewline
\ \ \ \ \ \ \ \ \ \ \isacommand{by}\isamarkupfalse%
\ metis\isanewline
\ \ \ \ \ \ \ \ \isacommand{hence}\isamarkupfalse%
\ g{\isadigit{2}}{\isacharcolon}{\kern0pt}\ {\isachardoublequoteopen}b\ {\isasymin}\ B{\isachardoublequoteclose}\isanewline
\ \ \ \ \ \ \ \ \ \ \isacommand{by}\isamarkupfalse%
\ auto\isanewline
\ \ \ \ \ \ \ \ \isacommand{have}\isamarkupfalse%
\ g{\isadigit{3}}{\isacharcolon}{\kern0pt}\isanewline
\ \ \ \ \ \ \ \ \ \ {\isachardoublequoteopen}partial{\isacharunderscore}{\kern0pt}order{\isacharunderscore}{\kern0pt}on\ B\ {\isacharparenleft}{\kern0pt}limit\ B\ r{\isacharparenright}{\kern0pt}\ {\isasymand}\ total{\isacharunderscore}{\kern0pt}on\ B\ {\isacharparenleft}{\kern0pt}limit\ B\ r{\isacharparenright}{\kern0pt}{\isachardoublequoteclose}\isanewline
\ \ \ \ \ \ \ \ \ \ \isacommand{using}\isamarkupfalse%
\ g{\isadigit{1}}\ B\ asm\ order{\isacharunderscore}{\kern0pt}on{\isacharunderscore}{\kern0pt}defs{\isacharparenleft}{\kern0pt}{\isadigit{3}}{\isacharparenright}{\kern0pt}\isanewline
\ \ \ \ \ \ \ \ \ \ \isacommand{by}\isamarkupfalse%
\ metis\isanewline
\ \ \ \ \ \ \ \ \isacommand{have}\isamarkupfalse%
\isanewline
\ \ \ \ \ \ \ \ \ \ {\isachardoublequoteopen}{\isasymforall}A\ r{\isachardot}{\kern0pt}\isanewline
\ \ \ \ \ \ \ \ \ \ \ \ total{\isacharunderscore}{\kern0pt}on\ A\ r\ {\isacharequal}{\kern0pt}\ {\isacharparenleft}{\kern0pt}{\isasymforall}a{\isachardot}{\kern0pt}\ {\isacharparenleft}{\kern0pt}a{\isacharcolon}{\kern0pt}{\isacharcolon}{\kern0pt}{\isacharprime}{\kern0pt}a{\isacharparenright}{\kern0pt}\ {\isasymnotin}\ A\ {\isasymor}\isanewline
\ \ \ \ \ \ \ \ \ \ \ \ \ \ {\isacharparenleft}{\kern0pt}{\isasymforall}aa{\isachardot}{\kern0pt}\ {\isacharparenleft}{\kern0pt}aa\ {\isasymnotin}\ A\ {\isasymor}\ a\ {\isacharequal}{\kern0pt}\ aa{\isacharparenright}{\kern0pt}\ {\isasymor}\ {\isacharparenleft}{\kern0pt}a{\isacharcomma}{\kern0pt}\ aa{\isacharparenright}{\kern0pt}\ {\isasymin}\ r\ {\isasymor}\ {\isacharparenleft}{\kern0pt}aa{\isacharcomma}{\kern0pt}\ a{\isacharparenright}{\kern0pt}\ {\isasymin}\ r{\isacharparenright}{\kern0pt}{\isacharparenright}{\kern0pt}{\isachardoublequoteclose}\isanewline
\ \ \ \ \ \ \ \ \ \ \isacommand{using}\isamarkupfalse%
\ total{\isacharunderscore}{\kern0pt}on{\isacharunderscore}{\kern0pt}def\isanewline
\ \ \ \ \ \ \ \ \ \ \isacommand{by}\isamarkupfalse%
\ metis\isanewline
\ \ \ \ \ \ \ \ \isacommand{hence}\isamarkupfalse%
\isanewline
\ \ \ \ \ \ \ \ \ \ {\isachardoublequoteopen}{\isasymforall}a{\isachardot}{\kern0pt}\ a\ {\isasymnotin}\ B\ {\isasymor}\isanewline
\ \ \ \ \ \ \ \ \ \ \ \ {\isacharparenleft}{\kern0pt}{\isasymforall}aa{\isachardot}{\kern0pt}\ aa\ {\isasymnotin}\ B\ {\isasymor}\ a\ {\isacharequal}{\kern0pt}\ aa\ {\isasymor}\isanewline
\ \ \ \ \ \ \ \ \ \ \ \ \ \ \ \ {\isacharparenleft}{\kern0pt}a{\isacharcomma}{\kern0pt}\ aa{\isacharparenright}{\kern0pt}\ {\isasymin}\ limit\ B\ r\ {\isasymor}\ {\isacharparenleft}{\kern0pt}aa{\isacharcomma}{\kern0pt}\ a{\isacharparenright}{\kern0pt}\ {\isasymin}\ limit\ B\ r{\isacharparenright}{\kern0pt}{\isachardoublequoteclose}\isanewline
\ \ \ \ \ \ \ \ \ \ \isacommand{using}\isamarkupfalse%
\ g{\isadigit{3}}\isanewline
\ \ \ \ \ \ \ \ \ \ \isacommand{by}\isamarkupfalse%
\ simp\isanewline
\ \ \ \ \ \ \ \ \isacommand{have}\isamarkupfalse%
\ {\isachardoublequoteopen}{\isasymforall}x\ {\isasymin}\ B{\isachardot}{\kern0pt}\ b\ {\isasymin}\ above\ r\ x{\isachardoublequoteclose}\isanewline
\ \ \ \ \ \ \ \ \ \ \isacommand{using}\isamarkupfalse%
\ limit{\isacharunderscore}{\kern0pt}presv{\isacharunderscore}{\kern0pt}above{\isadigit{2}}\ B\ pref{\isacharunderscore}{\kern0pt}imp{\isacharunderscore}{\kern0pt}in{\isacharunderscore}{\kern0pt}above\ asm\ b\ above{\isacharunderscore}{\kern0pt}def\isanewline
\ \ \ \ \ \ \ \ \ \ \ \ \ \ \ \ limit{\isacharunderscore}{\kern0pt}presv{\isacharunderscore}{\kern0pt}lin{\isacharunderscore}{\kern0pt}ord\ order{\isacharunderscore}{\kern0pt}on{\isacharunderscore}{\kern0pt}defs{\isacharparenleft}{\kern0pt}{\isadigit{3}}{\isacharparenright}{\kern0pt}\ singletonD\isanewline
\ \ \ \ \ \ \ \ \ \ \ \ \ \ \ \ singletonI\ total{\isacharunderscore}{\kern0pt}on{\isacharunderscore}{\kern0pt}def\ mem{\isacharunderscore}{\kern0pt}Collect{\isacharunderscore}{\kern0pt}eq\ g{\isadigit{2}}\isanewline
\ \ \ \ \ \ \ \ \ \ \isacommand{by}\isamarkupfalse%
\ {\isacharparenleft}{\kern0pt}smt\ {\isacharparenleft}{\kern0pt}verit{\isacharcomma}{\kern0pt}\ ccfv{\isacharunderscore}{\kern0pt}threshold{\isacharparenright}{\kern0pt}{\isacharparenright}{\kern0pt}\isanewline
\ \ \ \ \ \ \ \ \isacommand{hence}\isamarkupfalse%
\ b{\isacharunderscore}{\kern0pt}wins{\isadigit{2}}{\isacharcolon}{\kern0pt}\isanewline
\ \ \ \ \ \ \ \ \ \ {\isachardoublequoteopen}{\isasymforall}x\ {\isasymin}\ B{\isachardot}{\kern0pt}\ x\ {\isasympreceq}\isactrlsub r\ b{\isachardoublequoteclose}\isanewline
\ \ \ \ \ \ \ \ \ \ \isacommand{by}\isamarkupfalse%
\ {\isacharparenleft}{\kern0pt}simp\ add{\isacharcolon}{\kern0pt}\ above{\isacharunderscore}{\kern0pt}def{\isacharparenright}{\kern0pt}\isanewline
\ \ \ \ \ \ \ \ \isacommand{hence}\isamarkupfalse%
\ b{\isacharunderscore}{\kern0pt}wins{\isadigit{2}}{\isacharunderscore}{\kern0pt}{\isadigit{0}}{\isacharcolon}{\kern0pt}\isanewline
\ \ \ \ \ \ \ \ \ \ {\isachardoublequoteopen}{\isasymforall}x\ {\isasymin}\ B{\isachardot}{\kern0pt}\ {\isacharparenleft}{\kern0pt}x{\isacharcomma}{\kern0pt}\ b{\isacharparenright}{\kern0pt}\ {\isasymin}\ r{\isachardoublequoteclose}\isanewline
\ \ \ \ \ \ \ \ \ \ \isacommand{by}\isamarkupfalse%
\ simp\isanewline
\ \ \ \ \ \ \ \ \isacommand{have}\isamarkupfalse%
\ {\isachardoublequoteopen}trans\ r{\isachardoublequoteclose}\isanewline
\ \ \ \ \ \ \ \ \ \ \isacommand{using}\isamarkupfalse%
\ asm\ lin{\isacharunderscore}{\kern0pt}imp{\isacharunderscore}{\kern0pt}trans\isanewline
\ \ \ \ \ \ \ \ \ \ \isacommand{by}\isamarkupfalse%
\ metis\isanewline
\ \ \ \ \ \ \ \ \isacommand{hence}\isamarkupfalse%
\ {\isachardoublequoteopen}{\isasymforall}x\ {\isasymin}\ B{\isachardot}{\kern0pt}\ {\isacharparenleft}{\kern0pt}x{\isacharcomma}{\kern0pt}\ a{\isacharparenright}{\kern0pt}\ {\isasymin}\ r{\isachardoublequoteclose}\isanewline
\ \ \ \ \ \ \ \ \ \ \isacommand{using}\isamarkupfalse%
\ transE\ b{\isacharunderscore}{\kern0pt}smaller{\isacharunderscore}{\kern0pt}a{\isacharunderscore}{\kern0pt}{\isadigit{0}}\ b{\isacharunderscore}{\kern0pt}wins{\isadigit{2}}{\isacharunderscore}{\kern0pt}{\isadigit{0}}\isanewline
\ \ \ \ \ \ \ \ \ \ \isacommand{by}\isamarkupfalse%
\ metis\isanewline
\ \ \ \ \ \ \ \ \isacommand{hence}\isamarkupfalse%
\ {\isachardoublequoteopen}{\isasymforall}x\ {\isasymin}\ B{\isachardot}{\kern0pt}\ x\ {\isasympreceq}\isactrlsub r\ a{\isachardoublequoteclose}\isanewline
\ \ \ \ \ \ \ \ \ \ \isacommand{by}\isamarkupfalse%
\ simp\isanewline
\ \ \ \ \ \ \ \ \isacommand{hence}\isamarkupfalse%
\ nothing{\isacharunderscore}{\kern0pt}above{\isacharunderscore}{\kern0pt}a{\isacharcolon}{\kern0pt}\ {\isachardoublequoteopen}{\isasymforall}x\ {\isasymin}\ A{\isachardot}{\kern0pt}\ x\ {\isasympreceq}\isactrlsub r\ a{\isachardoublequoteclose}\isanewline
\ \ \ \ \ \ \ \ \ \ \isacommand{using}\isamarkupfalse%
\ a\ asm\ lin{\isacharunderscore}{\kern0pt}ord{\isacharunderscore}{\kern0pt}imp{\isacharunderscore}{\kern0pt}connex\ above{\isacharunderscore}{\kern0pt}connex\ Diff{\isacharunderscore}{\kern0pt}iff\isanewline
\ \ \ \ \ \ \ \ \ \ \ \ \ \ \ \ empty{\isacharunderscore}{\kern0pt}iff\ insert{\isacharunderscore}{\kern0pt}iff\ pref{\isacharunderscore}{\kern0pt}imp{\isacharunderscore}{\kern0pt}in{\isacharunderscore}{\kern0pt}above\isanewline
\ \ \ \ \ \ \ \ \ \ \isacommand{by}\isamarkupfalse%
\ metis\isanewline
\ \ \ \ \ \ \ \ \isacommand{have}\isamarkupfalse%
\ {\isachardoublequoteopen}{\isasymforall}x\ {\isasymin}\ A{\isachardot}{\kern0pt}\ x\ {\isasymin}\ above\ r\ a\ {\isasymlongleftrightarrow}\ x\ {\isacharequal}{\kern0pt}\ a{\isachardoublequoteclose}\isanewline
\ \ \ \ \ \ \ \ \ \ \isacommand{using}\isamarkupfalse%
\ antisym{\isacharunderscore}{\kern0pt}def\ asm\ lin{\isacharunderscore}{\kern0pt}imp{\isacharunderscore}{\kern0pt}antisym\isanewline
\ \ \ \ \ \ \ \ \ \ \ \ \ \ \ \ nothing{\isacharunderscore}{\kern0pt}above{\isacharunderscore}{\kern0pt}a\ pref{\isacharunderscore}{\kern0pt}imp{\isacharunderscore}{\kern0pt}in{\isacharunderscore}{\kern0pt}above\isanewline
\ \ \ \ \ \ \ \ \ \ \ \ \ \ \ \ CollectD\ above{\isacharunderscore}{\kern0pt}def\isanewline
\ \ \ \ \ \ \ \ \ \ \isacommand{by}\isamarkupfalse%
\ metis\isanewline
\ \ \ \ \ \ \ \ \isacommand{moreover}\isamarkupfalse%
\ \isacommand{have}\isamarkupfalse%
\ above{\isacharunderscore}{\kern0pt}a{\isacharunderscore}{\kern0pt}in{\isacharunderscore}{\kern0pt}A{\isacharcolon}{\kern0pt}\ {\isachardoublequoteopen}above\ r\ a\ {\isasymsubseteq}\ A{\isachardoublequoteclose}\isanewline
\ \ \ \ \ \ \ \ \ \ \isacommand{using}\isamarkupfalse%
\ above{\isacharunderscore}{\kern0pt}def\ asm\ connex{\isacharunderscore}{\kern0pt}imp{\isacharunderscore}{\kern0pt}refl\ lin{\isacharunderscore}{\kern0pt}ord{\isacharunderscore}{\kern0pt}imp{\isacharunderscore}{\kern0pt}connex\isanewline
\ \ \ \ \ \ \ \ \ \ \ \ \ \ \ \ mem{\isacharunderscore}{\kern0pt}Collect{\isacharunderscore}{\kern0pt}eq\ refl{\isacharunderscore}{\kern0pt}on{\isacharunderscore}{\kern0pt}domain\isanewline
\ \ \ \ \ \ \ \ \ \ \isacommand{by}\isamarkupfalse%
\ fastforce\isanewline
\ \ \ \ \ \ \ \ \isacommand{ultimately}\isamarkupfalse%
\ \isacommand{have}\isamarkupfalse%
\ {\isachardoublequoteopen}above\ r\ a\ {\isacharequal}{\kern0pt}\ {\isacharbraceleft}{\kern0pt}a{\isacharbraceright}{\kern0pt}{\isachardoublequoteclose}\isanewline
\ \ \ \ \ \ \ \ \ \ \isacommand{using}\isamarkupfalse%
\ above{\isacharunderscore}{\kern0pt}def\ a\isanewline
\ \ \ \ \ \ \ \ \ \ \isacommand{by}\isamarkupfalse%
\ auto\isanewline
\ \ \ \ \ \ \ \ \isacommand{thus}\isamarkupfalse%
\ {\isacharquery}{\kern0pt}thesis\isanewline
\ \ \ \ \ \ \ \ \ \ \isacommand{using}\isamarkupfalse%
\ above{\isacharunderscore}{\kern0pt}a{\isacharunderscore}{\kern0pt}in{\isacharunderscore}{\kern0pt}A\isanewline
\ \ \ \ \ \ \ \ \ \ \isacommand{by}\isamarkupfalse%
\ blast\isanewline
\ \ \ \ \ \ \isacommand{qed}\isamarkupfalse%
\isanewline
\ \ \ \ \isacommand{qed}\isamarkupfalse%
\isanewline
\ \ \isacommand{qed}\isamarkupfalse%
\isanewline
\ \ \isacommand{hence}\isamarkupfalse%
\ {\isachardoublequoteopen}{\isasymexists}a{\isachardot}{\kern0pt}\ a{\isasymin}A\ {\isasymand}\ above\ r\ a\ {\isacharequal}{\kern0pt}\ {\isacharbraceleft}{\kern0pt}a{\isacharbraceright}{\kern0pt}{\isachardoublequoteclose}\isanewline
\ \ \ \ \isacommand{using}\isamarkupfalse%
\ assms\ n\isanewline
\ \ \ \ \isacommand{by}\isamarkupfalse%
\ blast\isanewline
\ \ \isacommand{thus}\isamarkupfalse%
\ {\isacharquery}{\kern0pt}thesis\isanewline
\ \ \ \ \isacommand{using}\isamarkupfalse%
\ Diff{\isacharunderscore}{\kern0pt}eq{\isacharunderscore}{\kern0pt}empty{\isacharunderscore}{\kern0pt}iff\ above{\isacharunderscore}{\kern0pt}trans\ assms{\isacharparenleft}{\kern0pt}{\isadigit{1}}{\isacharparenright}{\kern0pt}\ empty{\isacharunderscore}{\kern0pt}Diff\ insertE\isanewline
\ \ \ \ \ \ \ \ \ \ insert{\isacharunderscore}{\kern0pt}Diff{\isacharunderscore}{\kern0pt}if\ insert{\isacharunderscore}{\kern0pt}absorb\ insert{\isacharunderscore}{\kern0pt}not{\isacharunderscore}{\kern0pt}empty\ order{\isacharunderscore}{\kern0pt}on{\isacharunderscore}{\kern0pt}defs{\isacharparenleft}{\kern0pt}{\isadigit{1}}{\isacharparenright}{\kern0pt}\isanewline
\ \ \ \ \ \ \ \ \ \ order{\isacharunderscore}{\kern0pt}on{\isacharunderscore}{\kern0pt}defs{\isacharparenleft}{\kern0pt}{\isadigit{2}}{\isacharparenright}{\kern0pt}\ order{\isacharunderscore}{\kern0pt}on{\isacharunderscore}{\kern0pt}defs{\isacharparenleft}{\kern0pt}{\isadigit{3}}{\isacharparenright}{\kern0pt}\ total{\isacharunderscore}{\kern0pt}on{\isacharunderscore}{\kern0pt}def\isanewline
\ \ \ \ \isacommand{by}\isamarkupfalse%
\ {\isacharparenleft}{\kern0pt}smt\ {\isacharparenleft}{\kern0pt}verit{\isacharcomma}{\kern0pt}\ ccfv{\isacharunderscore}{\kern0pt}SIG{\isacharparenright}{\kern0pt}{\isacharparenright}{\kern0pt}\isanewline
\isacommand{qed}\isamarkupfalse%
%
\endisatagproof
{\isafoldproof}%
%
\isadelimproof
\isanewline
%
\endisadelimproof
\isanewline
\isacommand{lemma}\isamarkupfalse%
\ above{\isacharunderscore}{\kern0pt}one{\isadigit{2}}{\isacharcolon}{\kern0pt}\isanewline
\ \ \isakeyword{assumes}\isanewline
\ \ \ \ lin{\isacharunderscore}{\kern0pt}ord{\isacharcolon}{\kern0pt}\ {\isachardoublequoteopen}linear{\isacharunderscore}{\kern0pt}order{\isacharunderscore}{\kern0pt}on\ A\ r{\isachardoublequoteclose}\ \isakeyword{and}\isanewline
\ \ \ \ fin{\isacharunderscore}{\kern0pt}not{\isacharunderscore}{\kern0pt}emp{\isacharcolon}{\kern0pt}\ {\isachardoublequoteopen}finite\ A\ {\isasymand}\ A\ {\isasymnoteq}\ {\isacharbraceleft}{\kern0pt}{\isacharbraceright}{\kern0pt}{\isachardoublequoteclose}\ \isakeyword{and}\isanewline
\ \ \ \ above{\isadigit{1}}{\isacharcolon}{\kern0pt}\ {\isachardoublequoteopen}above\ r\ a\ {\isacharequal}{\kern0pt}\ {\isacharbraceleft}{\kern0pt}a{\isacharbraceright}{\kern0pt}\ {\isasymand}\ above\ r\ b\ {\isacharequal}{\kern0pt}\ {\isacharbraceleft}{\kern0pt}b{\isacharbraceright}{\kern0pt}{\isachardoublequoteclose}\isanewline
\ \ \isakeyword{shows}\ {\isachardoublequoteopen}a\ {\isacharequal}{\kern0pt}\ b{\isachardoublequoteclose}\isanewline
%
\isadelimproof
%
\endisadelimproof
%
\isatagproof
\isacommand{proof}\isamarkupfalse%
\ {\isacharminus}{\kern0pt}\isanewline
\ \ \isacommand{have}\isamarkupfalse%
\ {\isachardoublequoteopen}a\ {\isasympreceq}\isactrlsub r\ a\ {\isasymand}\ b\ {\isasympreceq}\isactrlsub r\ b{\isachardoublequoteclose}\isanewline
\ \ \ \ \isacommand{using}\isamarkupfalse%
\ above{\isadigit{1}}\ singletonI\ pref{\isacharunderscore}{\kern0pt}imp{\isacharunderscore}{\kern0pt}in{\isacharunderscore}{\kern0pt}above\isanewline
\ \ \ \ \isacommand{by}\isamarkupfalse%
\ metis\isanewline
\ \ \isacommand{also}\isamarkupfalse%
\ \isacommand{have}\isamarkupfalse%
\isanewline
\ \ \ \ {\isachardoublequoteopen}{\isasymexists}a{\isasymin}A{\isachardot}{\kern0pt}\ above\ r\ a\ {\isacharequal}{\kern0pt}\ {\isacharbraceleft}{\kern0pt}a{\isacharbraceright}{\kern0pt}\ {\isasymand}\isanewline
\ \ \ \ \ \ {\isacharparenleft}{\kern0pt}{\isasymforall}x{\isasymin}A{\isachardot}{\kern0pt}\ above\ r\ x\ {\isacharequal}{\kern0pt}\ {\isacharbraceleft}{\kern0pt}x{\isacharbraceright}{\kern0pt}\ {\isasymlongrightarrow}\ x\ {\isacharequal}{\kern0pt}\ a{\isacharparenright}{\kern0pt}{\isachardoublequoteclose}\isanewline
\ \ \ \ \isacommand{using}\isamarkupfalse%
\ lin{\isacharunderscore}{\kern0pt}ord\ fin{\isacharunderscore}{\kern0pt}not{\isacharunderscore}{\kern0pt}emp\isanewline
\ \ \ \ \isacommand{by}\isamarkupfalse%
\ {\isacharparenleft}{\kern0pt}simp\ add{\isacharcolon}{\kern0pt}\ above{\isacharunderscore}{\kern0pt}one{\isacharparenright}{\kern0pt}\isanewline
\ \ \isacommand{moreover}\isamarkupfalse%
\ \isacommand{have}\isamarkupfalse%
\ {\isachardoublequoteopen}connex\ A\ r{\isachardoublequoteclose}\isanewline
\ \ \ \ \isacommand{using}\isamarkupfalse%
\ lin{\isacharunderscore}{\kern0pt}ord\isanewline
\ \ \ \ \isacommand{by}\isamarkupfalse%
\ {\isacharparenleft}{\kern0pt}simp\ add{\isacharcolon}{\kern0pt}\ lin{\isacharunderscore}{\kern0pt}ord{\isacharunderscore}{\kern0pt}imp{\isacharunderscore}{\kern0pt}connex{\isacharparenright}{\kern0pt}\isanewline
\ \ \isacommand{ultimately}\isamarkupfalse%
\ \isacommand{show}\isamarkupfalse%
\ {\isachardoublequoteopen}a\ {\isacharequal}{\kern0pt}\ b{\isachardoublequoteclose}\isanewline
\ \ \ \ \isacommand{using}\isamarkupfalse%
\ above{\isadigit{1}}\ connex{\isacharunderscore}{\kern0pt}def\ limited{\isacharunderscore}{\kern0pt}dest\isanewline
\ \ \ \ \isacommand{by}\isamarkupfalse%
\ metis\isanewline
\isacommand{qed}\isamarkupfalse%
%
\endisatagproof
{\isafoldproof}%
%
\isadelimproof
\isanewline
%
\endisadelimproof
\isanewline
\isacommand{lemma}\isamarkupfalse%
\ above{\isacharunderscore}{\kern0pt}presv{\isacharunderscore}{\kern0pt}limit{\isacharcolon}{\kern0pt}\isanewline
\ \ \isakeyword{assumes}\ {\isachardoublequoteopen}linear{\isacharunderscore}{\kern0pt}order\ r{\isachardoublequoteclose}\isanewline
\ \ \isakeyword{shows}\ {\isachardoublequoteopen}above\ {\isacharparenleft}{\kern0pt}limit\ A\ r{\isacharparenright}{\kern0pt}\ x\ {\isasymsubseteq}\ A{\isachardoublequoteclose}\isanewline
%
\isadelimproof
\ \ %
\endisadelimproof
%
\isatagproof
\isacommand{unfolding}\isamarkupfalse%
\ above{\isacharunderscore}{\kern0pt}def\isanewline
\ \ \isacommand{by}\isamarkupfalse%
\ auto%
\endisatagproof
{\isafoldproof}%
%
\isadelimproof
%
\endisadelimproof
%
\isadelimdocument
%
\endisadelimdocument
%
\isatagdocument
%
\isamarkupsubsection{Lifting Property%
}
\isamarkuptrue%
%
\endisatagdocument
{\isafolddocument}%
%
\isadelimdocument
%
\endisadelimdocument
\isacommand{definition}\isamarkupfalse%
\ equiv{\isacharunderscore}{\kern0pt}rel{\isacharunderscore}{\kern0pt}except{\isacharunderscore}{\kern0pt}a\ {\isacharcolon}{\kern0pt}{\isacharcolon}{\kern0pt}\ {\isachardoublequoteopen}{\isacharprime}{\kern0pt}a\ set\ {\isasymRightarrow}\ {\isacharprime}{\kern0pt}a\ Preference{\isacharunderscore}{\kern0pt}Relation\ {\isasymRightarrow}\isanewline
\ \ \ \ \ \ \ \ \ \ \ \ \ \ \ \ \ \ \ \ \ \ \ \ \ \ \ \ \ \ \ \ \ \ \ \ {\isacharprime}{\kern0pt}a\ Preference{\isacharunderscore}{\kern0pt}Relation\ {\isasymRightarrow}\ {\isacharprime}{\kern0pt}a\ {\isasymRightarrow}\ bool{\isachardoublequoteclose}\ \isakeyword{where}\isanewline
\ \ {\isachardoublequoteopen}equiv{\isacharunderscore}{\kern0pt}rel{\isacharunderscore}{\kern0pt}except{\isacharunderscore}{\kern0pt}a\ A\ r\ s\ a\ {\isasymequiv}\isanewline
\ \ \ \ linear{\isacharunderscore}{\kern0pt}order{\isacharunderscore}{\kern0pt}on\ A\ r\ {\isasymand}\ linear{\isacharunderscore}{\kern0pt}order{\isacharunderscore}{\kern0pt}on\ A\ s\ {\isasymand}\ a\ {\isasymin}\ A\ {\isasymand}\isanewline
\ \ \ \ {\isacharparenleft}{\kern0pt}{\isasymforall}x\ {\isasymin}\ A\ {\isacharminus}{\kern0pt}\ {\isacharbraceleft}{\kern0pt}a{\isacharbraceright}{\kern0pt}{\isachardot}{\kern0pt}\ {\isasymforall}y\ {\isasymin}\ A\ {\isacharminus}{\kern0pt}\ {\isacharbraceleft}{\kern0pt}a{\isacharbraceright}{\kern0pt}{\isachardot}{\kern0pt}\ x\ {\isasympreceq}\isactrlsub r\ y\ {\isasymlongleftrightarrow}\ x\ {\isasympreceq}\isactrlsub s\ y{\isacharparenright}{\kern0pt}{\isachardoublequoteclose}\isanewline
\isanewline
\isacommand{definition}\isamarkupfalse%
\ lifted\ {\isacharcolon}{\kern0pt}{\isacharcolon}{\kern0pt}\ {\isachardoublequoteopen}{\isacharprime}{\kern0pt}a\ set\ {\isasymRightarrow}\ {\isacharprime}{\kern0pt}a\ Preference{\isacharunderscore}{\kern0pt}Relation\ {\isasymRightarrow}\isanewline
\ \ \ \ \ \ \ \ \ \ \ \ \ \ \ \ \ \ \ \ \ \ \ \ {\isacharprime}{\kern0pt}a\ Preference{\isacharunderscore}{\kern0pt}Relation\ {\isasymRightarrow}\ {\isacharprime}{\kern0pt}a\ {\isasymRightarrow}\ bool{\isachardoublequoteclose}\ \isakeyword{where}\isanewline
\ \ {\isachardoublequoteopen}lifted\ A\ r\ s\ a\ {\isasymequiv}\isanewline
\ \ \ \ equiv{\isacharunderscore}{\kern0pt}rel{\isacharunderscore}{\kern0pt}except{\isacharunderscore}{\kern0pt}a\ A\ r\ s\ a\ {\isasymand}\ {\isacharparenleft}{\kern0pt}{\isasymexists}x\ {\isasymin}\ A\ {\isacharminus}{\kern0pt}\ {\isacharbraceleft}{\kern0pt}a{\isacharbraceright}{\kern0pt}{\isachardot}{\kern0pt}\ a\ {\isasympreceq}\isactrlsub r\ x\ {\isasymand}\ x\ {\isasympreceq}\isactrlsub s\ a{\isacharparenright}{\kern0pt}{\isachardoublequoteclose}\isanewline
\isanewline
\isacommand{lemma}\isamarkupfalse%
\ trivial{\isacharunderscore}{\kern0pt}equiv{\isacharunderscore}{\kern0pt}rel{\isacharcolon}{\kern0pt}\isanewline
\ \ \isakeyword{assumes}\ order{\isacharcolon}{\kern0pt}\ {\isachardoublequoteopen}linear{\isacharunderscore}{\kern0pt}order{\isacharunderscore}{\kern0pt}on\ A\ p{\isachardoublequoteclose}\isanewline
\ \ \isakeyword{shows}\ {\isachardoublequoteopen}{\isasymforall}a\ {\isasymin}\ A{\isachardot}{\kern0pt}\ equiv{\isacharunderscore}{\kern0pt}rel{\isacharunderscore}{\kern0pt}except{\isacharunderscore}{\kern0pt}a\ A\ p\ p\ a{\isachardoublequoteclose}\isanewline
%
\isadelimproof
\ \ %
\endisadelimproof
%
\isatagproof
\isacommand{by}\isamarkupfalse%
\ {\isacharparenleft}{\kern0pt}simp\ add{\isacharcolon}{\kern0pt}\ equiv{\isacharunderscore}{\kern0pt}rel{\isacharunderscore}{\kern0pt}except{\isacharunderscore}{\kern0pt}a{\isacharunderscore}{\kern0pt}def\ order{\isacharparenright}{\kern0pt}%
\endisatagproof
{\isafoldproof}%
%
\isadelimproof
\isanewline
%
\endisadelimproof
\isanewline
\isacommand{lemma}\isamarkupfalse%
\ lifted{\isacharunderscore}{\kern0pt}imp{\isacharunderscore}{\kern0pt}equiv{\isacharunderscore}{\kern0pt}rel{\isacharunderscore}{\kern0pt}except{\isacharunderscore}{\kern0pt}a{\isacharcolon}{\kern0pt}\isanewline
\ \ \isakeyword{assumes}\ lifted{\isacharcolon}{\kern0pt}\ {\isachardoublequoteopen}lifted\ A\ r\ s\ a{\isachardoublequoteclose}\isanewline
\ \ \isakeyword{shows}\ {\isachardoublequoteopen}equiv{\isacharunderscore}{\kern0pt}rel{\isacharunderscore}{\kern0pt}except{\isacharunderscore}{\kern0pt}a\ A\ r\ s\ a{\isachardoublequoteclose}\isanewline
%
\isadelimproof
%
\endisadelimproof
%
\isatagproof
\isacommand{proof}\isamarkupfalse%
\ {\isacharminus}{\kern0pt}\isanewline
\ \ \isacommand{from}\isamarkupfalse%
\ lifted\ \isacommand{have}\isamarkupfalse%
\isanewline
\ \ \ \ {\isachardoublequoteopen}linear{\isacharunderscore}{\kern0pt}order{\isacharunderscore}{\kern0pt}on\ A\ r\ {\isasymand}\ linear{\isacharunderscore}{\kern0pt}order{\isacharunderscore}{\kern0pt}on\ A\ s\ {\isasymand}\ a\ {\isasymin}\ A\ {\isasymand}\isanewline
\ \ \ \ \ \ {\isacharparenleft}{\kern0pt}{\isasymforall}x\ {\isasymin}\ A\ {\isacharminus}{\kern0pt}\ {\isacharbraceleft}{\kern0pt}a{\isacharbraceright}{\kern0pt}{\isachardot}{\kern0pt}\ {\isasymforall}y\ {\isasymin}\ A\ {\isacharminus}{\kern0pt}\ {\isacharbraceleft}{\kern0pt}a{\isacharbraceright}{\kern0pt}{\isachardot}{\kern0pt}\ x\ {\isasympreceq}\isactrlsub r\ y\ {\isasymlongleftrightarrow}\ x\ {\isasympreceq}\isactrlsub s\ y{\isacharparenright}{\kern0pt}{\isachardoublequoteclose}\isanewline
\ \ \ \ \isacommand{by}\isamarkupfalse%
\ {\isacharparenleft}{\kern0pt}simp\ add{\isacharcolon}{\kern0pt}\ lifted{\isacharunderscore}{\kern0pt}def\ equiv{\isacharunderscore}{\kern0pt}rel{\isacharunderscore}{\kern0pt}except{\isacharunderscore}{\kern0pt}a{\isacharunderscore}{\kern0pt}def{\isacharparenright}{\kern0pt}\isanewline
\ \ \isacommand{thus}\isamarkupfalse%
\ {\isacharquery}{\kern0pt}thesis\isanewline
\ \ \ \ \isacommand{by}\isamarkupfalse%
\ {\isacharparenleft}{\kern0pt}simp\ add{\isacharcolon}{\kern0pt}\ equiv{\isacharunderscore}{\kern0pt}rel{\isacharunderscore}{\kern0pt}except{\isacharunderscore}{\kern0pt}a{\isacharunderscore}{\kern0pt}def{\isacharparenright}{\kern0pt}\isanewline
\isacommand{qed}\isamarkupfalse%
%
\endisatagproof
{\isafoldproof}%
%
\isadelimproof
\isanewline
%
\endisadelimproof
\isanewline
\isacommand{lemma}\isamarkupfalse%
\ lifted{\isacharunderscore}{\kern0pt}mono{\isacharcolon}{\kern0pt}\isanewline
\ \ \isakeyword{assumes}\ lifted{\isacharcolon}{\kern0pt}\ {\isachardoublequoteopen}lifted\ A\ r\ s\ a{\isachardoublequoteclose}\isanewline
\ \ \isakeyword{shows}\ {\isachardoublequoteopen}{\isasymforall}x\ {\isasymin}\ A\ {\isacharminus}{\kern0pt}\ {\isacharbraceleft}{\kern0pt}a{\isacharbraceright}{\kern0pt}{\isachardot}{\kern0pt}\ {\isasymnot}{\isacharparenleft}{\kern0pt}x\ {\isasympreceq}\isactrlsub r\ a\ {\isasymand}\ a\ {\isasympreceq}\isactrlsub s\ x{\isacharparenright}{\kern0pt}{\isachardoublequoteclose}\isanewline
%
\isadelimproof
%
\endisadelimproof
%
\isatagproof
\isacommand{proof}\isamarkupfalse%
\ {\isacharparenleft}{\kern0pt}safe{\isacharparenright}{\kern0pt}\isanewline
\ \ \isacommand{fix}\isamarkupfalse%
\isanewline
\ \ \ \ x\ {\isacharcolon}{\kern0pt}{\isacharcolon}{\kern0pt}\ {\isachardoublequoteopen}{\isacharprime}{\kern0pt}a{\isachardoublequoteclose}\isanewline
\ \ \isacommand{assume}\isamarkupfalse%
\isanewline
\ \ \ \ x{\isacharunderscore}{\kern0pt}in{\isacharunderscore}{\kern0pt}A{\isacharcolon}{\kern0pt}\ \ \ {\isachardoublequoteopen}x\ {\isasymin}\ A{\isachardoublequoteclose}\ \isakeyword{and}\isanewline
\ \ \ \ x{\isacharunderscore}{\kern0pt}exist{\isacharcolon}{\kern0pt}\ \ {\isachardoublequoteopen}x\ {\isasymnotin}\ {\isacharbraceleft}{\kern0pt}{\isacharbraceright}{\kern0pt}{\isachardoublequoteclose}\ \isakeyword{and}\isanewline
\ \ \ \ x{\isacharunderscore}{\kern0pt}neq{\isacharunderscore}{\kern0pt}a{\isacharcolon}{\kern0pt}\ \ {\isachardoublequoteopen}x\ {\isasymnoteq}\ a{\isachardoublequoteclose}\ \isakeyword{and}\isanewline
\ \ \ \ x{\isacharunderscore}{\kern0pt}pref{\isacharunderscore}{\kern0pt}a{\isacharcolon}{\kern0pt}\ {\isachardoublequoteopen}x\ {\isasympreceq}\isactrlsub r\ a{\isachardoublequoteclose}\ \isakeyword{and}\isanewline
\ \ \ \ a{\isacharunderscore}{\kern0pt}pref{\isacharunderscore}{\kern0pt}x{\isacharcolon}{\kern0pt}\ {\isachardoublequoteopen}a\ {\isasympreceq}\isactrlsub s\ x{\isachardoublequoteclose}\isanewline
\ \ \isacommand{from}\isamarkupfalse%
\ x{\isacharunderscore}{\kern0pt}pref{\isacharunderscore}{\kern0pt}a\isanewline
\ \ \isacommand{have}\isamarkupfalse%
\ x{\isacharunderscore}{\kern0pt}pref{\isacharunderscore}{\kern0pt}a{\isacharunderscore}{\kern0pt}{\isadigit{0}}{\isacharcolon}{\kern0pt}\ {\isachardoublequoteopen}{\isacharparenleft}{\kern0pt}x{\isacharcomma}{\kern0pt}\ a{\isacharparenright}{\kern0pt}\ {\isasymin}\ r{\isachardoublequoteclose}\isanewline
\ \ \ \ \isacommand{by}\isamarkupfalse%
\ simp\isanewline
\ \ \isacommand{from}\isamarkupfalse%
\ a{\isacharunderscore}{\kern0pt}pref{\isacharunderscore}{\kern0pt}x\isanewline
\ \ \isacommand{have}\isamarkupfalse%
\ a{\isacharunderscore}{\kern0pt}pref{\isacharunderscore}{\kern0pt}x{\isacharunderscore}{\kern0pt}{\isadigit{0}}{\isacharcolon}{\kern0pt}\ {\isachardoublequoteopen}{\isacharparenleft}{\kern0pt}a{\isacharcomma}{\kern0pt}\ x{\isacharparenright}{\kern0pt}\ {\isasymin}\ s{\isachardoublequoteclose}\isanewline
\ \ \ \ \isacommand{by}\isamarkupfalse%
\ simp\isanewline
\ \ \isacommand{have}\isamarkupfalse%
\ {\isachardoublequoteopen}antisym\ r{\isachardoublequoteclose}\isanewline
\ \ \ \ \isacommand{using}\isamarkupfalse%
\ equiv{\isacharunderscore}{\kern0pt}rel{\isacharunderscore}{\kern0pt}except{\isacharunderscore}{\kern0pt}a{\isacharunderscore}{\kern0pt}def\ lifted\isanewline
\ \ \ \ \ \ \ \ \ \ lifted{\isacharunderscore}{\kern0pt}imp{\isacharunderscore}{\kern0pt}equiv{\isacharunderscore}{\kern0pt}rel{\isacharunderscore}{\kern0pt}except{\isacharunderscore}{\kern0pt}a\isanewline
\ \ \ \ \ \ \ \ \ \ lin{\isacharunderscore}{\kern0pt}imp{\isacharunderscore}{\kern0pt}antisym\isanewline
\ \ \ \ \isacommand{by}\isamarkupfalse%
\ metis\isanewline
\ \ \isacommand{hence}\isamarkupfalse%
\ antisym{\isacharunderscore}{\kern0pt}r{\isacharcolon}{\kern0pt}\isanewline
\ \ \ \ {\isachardoublequoteopen}{\isacharparenleft}{\kern0pt}{\isasymforall}x\ y{\isachardot}{\kern0pt}\ {\isacharparenleft}{\kern0pt}x{\isacharcomma}{\kern0pt}\ y{\isacharparenright}{\kern0pt}\ {\isasymin}\ r\ {\isasymlongrightarrow}\ {\isacharparenleft}{\kern0pt}y{\isacharcomma}{\kern0pt}\ x{\isacharparenright}{\kern0pt}\ {\isasymin}\ r\ {\isasymlongrightarrow}\ x\ {\isacharequal}{\kern0pt}\ y{\isacharparenright}{\kern0pt}{\isachardoublequoteclose}\isanewline
\ \ \ \ \isacommand{using}\isamarkupfalse%
\ antisym{\isacharunderscore}{\kern0pt}def\isanewline
\ \ \ \ \isacommand{by}\isamarkupfalse%
\ metis\isanewline
\ \ \isacommand{hence}\isamarkupfalse%
\ imp{\isacharunderscore}{\kern0pt}x{\isacharunderscore}{\kern0pt}eq{\isacharunderscore}{\kern0pt}a{\isacharunderscore}{\kern0pt}{\isadigit{0}}{\isacharcolon}{\kern0pt}\isanewline
\ \ \ \ {\isachardoublequoteopen}{\isasymlbrakk}{\isacharparenleft}{\kern0pt}x{\isacharcomma}{\kern0pt}\ a{\isacharparenright}{\kern0pt}\ {\isasymin}\ r{\isacharsemicolon}{\kern0pt}\ {\isacharparenleft}{\kern0pt}a{\isacharcomma}{\kern0pt}\ x{\isacharparenright}{\kern0pt}\ {\isasymin}\ r{\isasymrbrakk}\ {\isasymLongrightarrow}\ x\ {\isacharequal}{\kern0pt}\ a{\isachardoublequoteclose}\isanewline
\ \ \ \ \isacommand{by}\isamarkupfalse%
\ simp\isanewline
\ \ \isacommand{have}\isamarkupfalse%
\ lift{\isacharunderscore}{\kern0pt}ex{\isacharcolon}{\kern0pt}\ {\isachardoublequoteopen}{\isasymexists}x\ {\isasymin}\ A\ {\isacharminus}{\kern0pt}\ {\isacharbraceleft}{\kern0pt}a{\isacharbraceright}{\kern0pt}{\isachardot}{\kern0pt}\ a\ {\isasympreceq}\isactrlsub r\ x\ {\isasymand}\ x\ {\isasympreceq}\isactrlsub s\ a{\isachardoublequoteclose}\isanewline
\ \ \ \ \isacommand{using}\isamarkupfalse%
\ lifted\ lifted{\isacharunderscore}{\kern0pt}def\isanewline
\ \ \ \ \isacommand{by}\isamarkupfalse%
\ metis\isanewline
\ \ \isacommand{from}\isamarkupfalse%
\ lift{\isacharunderscore}{\kern0pt}ex\ \isacommand{obtain}\isamarkupfalse%
\ y\ {\isacharcolon}{\kern0pt}{\isacharcolon}{\kern0pt}\ {\isacharprime}{\kern0pt}a\ \isakeyword{where}\isanewline
\ \ \ \ f{\isadigit{1}}{\isacharcolon}{\kern0pt}\ {\isachardoublequoteopen}y\ {\isasymin}\ A\ {\isacharminus}{\kern0pt}\ {\isacharbraceleft}{\kern0pt}a{\isacharbraceright}{\kern0pt}\ {\isasymand}\ a\ {\isasympreceq}\isactrlsub r\ y\ {\isasymand}\ y\ {\isasympreceq}\isactrlsub s\ a{\isachardoublequoteclose}\isanewline
\ \ \ \ \isacommand{by}\isamarkupfalse%
\ metis\isanewline
\ \ \isacommand{hence}\isamarkupfalse%
\ f{\isadigit{1}}{\isacharunderscore}{\kern0pt}{\isadigit{0}}{\isacharcolon}{\kern0pt}\isanewline
\ \ \ \ {\isachardoublequoteopen}y\ {\isasymin}\ A\ {\isacharminus}{\kern0pt}\ {\isacharbraceleft}{\kern0pt}a{\isacharbraceright}{\kern0pt}\ {\isasymand}\ {\isacharparenleft}{\kern0pt}a{\isacharcomma}{\kern0pt}\ y{\isacharparenright}{\kern0pt}\ {\isasymin}\ r\ {\isasymand}\ {\isacharparenleft}{\kern0pt}y{\isacharcomma}{\kern0pt}\ a{\isacharparenright}{\kern0pt}\ {\isasymin}\ s{\isachardoublequoteclose}\isanewline
\ \ \ \ \isacommand{by}\isamarkupfalse%
\ simp\isanewline
\ \ \isacommand{have}\isamarkupfalse%
\ f{\isadigit{2}}{\isacharcolon}{\kern0pt}\isanewline
\ \ \ \ {\isachardoublequoteopen}equiv{\isacharunderscore}{\kern0pt}rel{\isacharunderscore}{\kern0pt}except{\isacharunderscore}{\kern0pt}a\ A\ r\ s\ a{\isachardoublequoteclose}\isanewline
\ \ \ \ \isacommand{using}\isamarkupfalse%
\ lifted\ lifted{\isacharunderscore}{\kern0pt}def\isanewline
\ \ \ \ \isacommand{by}\isamarkupfalse%
\ metis\isanewline
\ \ \isacommand{hence}\isamarkupfalse%
\ f{\isadigit{2}}{\isacharunderscore}{\kern0pt}{\isadigit{0}}{\isacharcolon}{\kern0pt}\isanewline
\ \ \ \ {\isachardoublequoteopen}{\isasymforall}x\ {\isasymin}\ A\ {\isacharminus}{\kern0pt}\ {\isacharbraceleft}{\kern0pt}a{\isacharbraceright}{\kern0pt}{\isachardot}{\kern0pt}\ {\isasymforall}y\ {\isasymin}\ A\ {\isacharminus}{\kern0pt}\ {\isacharbraceleft}{\kern0pt}a{\isacharbraceright}{\kern0pt}{\isachardot}{\kern0pt}\ x\ {\isasympreceq}\isactrlsub r\ y\ {\isasymlongleftrightarrow}\ x\ {\isasympreceq}\isactrlsub s\ y{\isachardoublequoteclose}\isanewline
\ \ \ \ \isacommand{using}\isamarkupfalse%
\ equiv{\isacharunderscore}{\kern0pt}rel{\isacharunderscore}{\kern0pt}except{\isacharunderscore}{\kern0pt}a{\isacharunderscore}{\kern0pt}def\isanewline
\ \ \ \ \isacommand{by}\isamarkupfalse%
\ metis\isanewline
\ \ \isacommand{hence}\isamarkupfalse%
\ f{\isadigit{2}}{\isacharunderscore}{\kern0pt}{\isadigit{1}}{\isacharcolon}{\kern0pt}\isanewline
\ \ \ \ {\isachardoublequoteopen}{\isasymforall}x\ {\isasymin}\ A\ {\isacharminus}{\kern0pt}\ {\isacharbraceleft}{\kern0pt}a{\isacharbraceright}{\kern0pt}{\isachardot}{\kern0pt}\ {\isasymforall}y\ {\isasymin}\ A\ {\isacharminus}{\kern0pt}\ {\isacharbraceleft}{\kern0pt}a{\isacharbraceright}{\kern0pt}{\isachardot}{\kern0pt}\ {\isacharparenleft}{\kern0pt}x{\isacharcomma}{\kern0pt}\ y{\isacharparenright}{\kern0pt}\ {\isasymin}\ r\ {\isasymlongleftrightarrow}\ {\isacharparenleft}{\kern0pt}x{\isacharcomma}{\kern0pt}\ y{\isacharparenright}{\kern0pt}\ {\isasymin}\ s{\isachardoublequoteclose}\isanewline
\ \ \ \ \isacommand{by}\isamarkupfalse%
\ simp\isanewline
\ \ \isacommand{have}\isamarkupfalse%
\ trans{\isacharcolon}{\kern0pt}\ {\isachardoublequoteopen}{\isasymforall}x\ y\ z{\isachardot}{\kern0pt}\ {\isacharparenleft}{\kern0pt}x{\isacharcomma}{\kern0pt}\ y{\isacharparenright}{\kern0pt}\ {\isasymin}\ r\ {\isasymlongrightarrow}\ {\isacharparenleft}{\kern0pt}y{\isacharcomma}{\kern0pt}\ z{\isacharparenright}{\kern0pt}\ {\isasymin}\ r\ {\isasymlongrightarrow}\ {\isacharparenleft}{\kern0pt}x{\isacharcomma}{\kern0pt}\ z{\isacharparenright}{\kern0pt}\ {\isasymin}\ r{\isachardoublequoteclose}\isanewline
\ \ \ \ \isacommand{using}\isamarkupfalse%
\ f{\isadigit{2}}\ equiv{\isacharunderscore}{\kern0pt}rel{\isacharunderscore}{\kern0pt}except{\isacharunderscore}{\kern0pt}a{\isacharunderscore}{\kern0pt}def\ linear{\isacharunderscore}{\kern0pt}order{\isacharunderscore}{\kern0pt}on{\isacharunderscore}{\kern0pt}def\isanewline
\ \ \ \ \ \ \ \ \ \ partial{\isacharunderscore}{\kern0pt}order{\isacharunderscore}{\kern0pt}on{\isacharunderscore}{\kern0pt}def\ preorder{\isacharunderscore}{\kern0pt}on{\isacharunderscore}{\kern0pt}def\ trans{\isacharunderscore}{\kern0pt}def\isanewline
\ \ \ \ \isacommand{by}\isamarkupfalse%
\ metis\isanewline
\ \ \isacommand{have}\isamarkupfalse%
\ x{\isacharunderscore}{\kern0pt}pref{\isacharunderscore}{\kern0pt}y{\isacharunderscore}{\kern0pt}{\isadigit{0}}{\isacharcolon}{\kern0pt}\ {\isachardoublequoteopen}{\isacharparenleft}{\kern0pt}x{\isacharcomma}{\kern0pt}\ y{\isacharparenright}{\kern0pt}\ {\isasymin}\ s{\isachardoublequoteclose}\isanewline
\ \ \ \ \isacommand{using}\isamarkupfalse%
\ equiv{\isacharunderscore}{\kern0pt}rel{\isacharunderscore}{\kern0pt}except{\isacharunderscore}{\kern0pt}a{\isacharunderscore}{\kern0pt}def\ f{\isadigit{1}}{\isacharunderscore}{\kern0pt}{\isadigit{0}}\ f{\isadigit{2}}\ f{\isadigit{2}}{\isacharunderscore}{\kern0pt}{\isadigit{1}}\ insertE\isanewline
\ \ \ \ \ \ \ \ \ \ insert{\isacharunderscore}{\kern0pt}Diff\ x{\isacharunderscore}{\kern0pt}in{\isacharunderscore}{\kern0pt}A\ x{\isacharunderscore}{\kern0pt}neq{\isacharunderscore}{\kern0pt}a\ x{\isacharunderscore}{\kern0pt}pref{\isacharunderscore}{\kern0pt}a{\isacharunderscore}{\kern0pt}{\isadigit{0}}\ trans\isanewline
\ \ \ \ \isacommand{by}\isamarkupfalse%
\ metis\isanewline
\ \ \isacommand{have}\isamarkupfalse%
\ a{\isacharunderscore}{\kern0pt}pref{\isacharunderscore}{\kern0pt}y{\isacharunderscore}{\kern0pt}{\isadigit{0}}{\isacharcolon}{\kern0pt}\ {\isachardoublequoteopen}{\isacharparenleft}{\kern0pt}a{\isacharcomma}{\kern0pt}\ y{\isacharparenright}{\kern0pt}\ {\isasymin}\ s{\isachardoublequoteclose}\isanewline
\ \ \ \ \isacommand{using}\isamarkupfalse%
\ a{\isacharunderscore}{\kern0pt}pref{\isacharunderscore}{\kern0pt}x{\isacharunderscore}{\kern0pt}{\isadigit{0}}\ imp{\isacharunderscore}{\kern0pt}x{\isacharunderscore}{\kern0pt}eq{\isacharunderscore}{\kern0pt}a{\isacharunderscore}{\kern0pt}{\isadigit{0}}\ x{\isacharunderscore}{\kern0pt}neq{\isacharunderscore}{\kern0pt}a\ x{\isacharunderscore}{\kern0pt}pref{\isacharunderscore}{\kern0pt}a{\isacharunderscore}{\kern0pt}{\isadigit{0}}\isanewline
\ \ \ \ \ \ \ \ \ \ equiv{\isacharunderscore}{\kern0pt}rel{\isacharunderscore}{\kern0pt}except{\isacharunderscore}{\kern0pt}a{\isacharunderscore}{\kern0pt}def\ f{\isadigit{2}}\ lin{\isacharunderscore}{\kern0pt}imp{\isacharunderscore}{\kern0pt}trans\isanewline
\ \ \ \ \ \ \ \ \ \ transE\ x{\isacharunderscore}{\kern0pt}pref{\isacharunderscore}{\kern0pt}y{\isacharunderscore}{\kern0pt}{\isadigit{0}}\isanewline
\ \ \ \ \isacommand{by}\isamarkupfalse%
\ metis\isanewline
\ \ \isacommand{show}\isamarkupfalse%
\ {\isachardoublequoteopen}False{\isachardoublequoteclose}\isanewline
\ \ \ \ \isacommand{using}\isamarkupfalse%
\ a{\isacharunderscore}{\kern0pt}pref{\isacharunderscore}{\kern0pt}y{\isacharunderscore}{\kern0pt}{\isadigit{0}}\ antisymD\ equiv{\isacharunderscore}{\kern0pt}rel{\isacharunderscore}{\kern0pt}except{\isacharunderscore}{\kern0pt}a{\isacharunderscore}{\kern0pt}def\isanewline
\ \ \ \ \ \ \ \ \ \ DiffD{\isadigit{2}}\ f{\isadigit{1}}{\isacharunderscore}{\kern0pt}{\isadigit{0}}\ f{\isadigit{2}}\ lin{\isacharunderscore}{\kern0pt}imp{\isacharunderscore}{\kern0pt}antisym\ singletonI\isanewline
\ \ \ \ \isacommand{by}\isamarkupfalse%
\ metis\isanewline
\isacommand{qed}\isamarkupfalse%
%
\endisatagproof
{\isafoldproof}%
%
\isadelimproof
\isanewline
%
\endisadelimproof
\isanewline
\isacommand{lemma}\isamarkupfalse%
\ lifted{\isacharunderscore}{\kern0pt}mono{\isadigit{2}}{\isacharcolon}{\kern0pt}\isanewline
\ \ \isakeyword{assumes}\isanewline
\ \ \ \ lifted{\isacharcolon}{\kern0pt}\ {\isachardoublequoteopen}lifted\ A\ r\ s\ a{\isachardoublequoteclose}\ \isakeyword{and}\isanewline
\ \ \ \ x{\isacharunderscore}{\kern0pt}pref{\isacharunderscore}{\kern0pt}a{\isacharcolon}{\kern0pt}\ {\isachardoublequoteopen}x\ {\isasympreceq}\isactrlsub r\ a{\isachardoublequoteclose}\isanewline
\ \ \isakeyword{shows}\ {\isachardoublequoteopen}x\ {\isasympreceq}\isactrlsub s\ a{\isachardoublequoteclose}\isanewline
%
\isadelimproof
%
\endisadelimproof
%
\isatagproof
\isacommand{proof}\isamarkupfalse%
\ {\isacharparenleft}{\kern0pt}simp{\isacharparenright}{\kern0pt}\isanewline
\ \ \isacommand{have}\isamarkupfalse%
\ x{\isacharunderscore}{\kern0pt}pref{\isacharunderscore}{\kern0pt}a{\isacharunderscore}{\kern0pt}{\isadigit{0}}{\isacharcolon}{\kern0pt}\ {\isachardoublequoteopen}{\isacharparenleft}{\kern0pt}x{\isacharcomma}{\kern0pt}\ a{\isacharparenright}{\kern0pt}\ {\isasymin}\ r{\isachardoublequoteclose}\isanewline
\ \ \ \ \isacommand{using}\isamarkupfalse%
\ x{\isacharunderscore}{\kern0pt}pref{\isacharunderscore}{\kern0pt}a\isanewline
\ \ \ \ \isacommand{by}\isamarkupfalse%
\ simp\isanewline
\ \ \isacommand{have}\isamarkupfalse%
\ x{\isacharunderscore}{\kern0pt}in{\isacharunderscore}{\kern0pt}A{\isacharcolon}{\kern0pt}\ {\isachardoublequoteopen}x\ {\isasymin}\ A{\isachardoublequoteclose}\isanewline
\ \ \ \ \isacommand{using}\isamarkupfalse%
\ connex{\isacharunderscore}{\kern0pt}imp{\isacharunderscore}{\kern0pt}refl\ equiv{\isacharunderscore}{\kern0pt}rel{\isacharunderscore}{\kern0pt}except{\isacharunderscore}{\kern0pt}a{\isacharunderscore}{\kern0pt}def\isanewline
\ \ \ \ \ \ \ \ \ \ lifted\ lifted{\isacharunderscore}{\kern0pt}def\ lin{\isacharunderscore}{\kern0pt}ord{\isacharunderscore}{\kern0pt}imp{\isacharunderscore}{\kern0pt}connex\isanewline
\ \ \ \ \ \ \ \ \ \ refl{\isacharunderscore}{\kern0pt}on{\isacharunderscore}{\kern0pt}domain\ x{\isacharunderscore}{\kern0pt}pref{\isacharunderscore}{\kern0pt}a{\isacharunderscore}{\kern0pt}{\isadigit{0}}\isanewline
\ \ \ \ \isacommand{by}\isamarkupfalse%
\ metis\isanewline
\ \ \isacommand{have}\isamarkupfalse%
\ {\isachardoublequoteopen}{\isasymforall}x\ {\isasymin}\ A\ {\isacharminus}{\kern0pt}\ {\isacharbraceleft}{\kern0pt}a{\isacharbraceright}{\kern0pt}{\isachardot}{\kern0pt}\ {\isasymforall}y\ {\isasymin}\ A\ {\isacharminus}{\kern0pt}\ {\isacharbraceleft}{\kern0pt}a{\isacharbraceright}{\kern0pt}{\isachardot}{\kern0pt}\ x\ {\isasympreceq}\isactrlsub r\ y\ {\isasymlongleftrightarrow}\ x\ {\isasympreceq}\isactrlsub s\ y{\isachardoublequoteclose}\isanewline
\ \ \ \ \isacommand{using}\isamarkupfalse%
\ lifted\ lifted{\isacharunderscore}{\kern0pt}def\ equiv{\isacharunderscore}{\kern0pt}rel{\isacharunderscore}{\kern0pt}except{\isacharunderscore}{\kern0pt}a{\isacharunderscore}{\kern0pt}def\isanewline
\ \ \ \ \isacommand{by}\isamarkupfalse%
\ metis\isanewline
\ \ \isacommand{hence}\isamarkupfalse%
\ rest{\isacharunderscore}{\kern0pt}eq{\isacharcolon}{\kern0pt}\isanewline
\ \ \ \ {\isachardoublequoteopen}{\isasymforall}x\ {\isasymin}\ A\ {\isacharminus}{\kern0pt}\ {\isacharbraceleft}{\kern0pt}a{\isacharbraceright}{\kern0pt}{\isachardot}{\kern0pt}\ {\isasymforall}y\ {\isasymin}\ A\ {\isacharminus}{\kern0pt}\ {\isacharbraceleft}{\kern0pt}a{\isacharbraceright}{\kern0pt}{\isachardot}{\kern0pt}\ {\isacharparenleft}{\kern0pt}x{\isacharcomma}{\kern0pt}\ y{\isacharparenright}{\kern0pt}\ {\isasymin}\ r\ {\isasymlongleftrightarrow}\ {\isacharparenleft}{\kern0pt}x{\isacharcomma}{\kern0pt}\ y{\isacharparenright}{\kern0pt}\ {\isasymin}\ s{\isachardoublequoteclose}\isanewline
\ \ \ \ \isacommand{by}\isamarkupfalse%
\ simp\isanewline
\ \ \isacommand{have}\isamarkupfalse%
\ {\isachardoublequoteopen}{\isasymexists}x\ {\isasymin}\ A\ {\isacharminus}{\kern0pt}\ {\isacharbraceleft}{\kern0pt}a{\isacharbraceright}{\kern0pt}{\isachardot}{\kern0pt}\ a\ {\isasympreceq}\isactrlsub r\ x\ {\isasymand}\ x\ {\isasympreceq}\isactrlsub s\ a{\isachardoublequoteclose}\isanewline
\ \ \ \ \isacommand{using}\isamarkupfalse%
\ lifted\ lifted{\isacharunderscore}{\kern0pt}def\isanewline
\ \ \ \ \isacommand{by}\isamarkupfalse%
\ metis\isanewline
\ \ \isacommand{hence}\isamarkupfalse%
\ ex{\isacharunderscore}{\kern0pt}lifted{\isacharcolon}{\kern0pt}\isanewline
\ \ \ \ {\isachardoublequoteopen}{\isasymexists}x\ {\isasymin}\ A\ {\isacharminus}{\kern0pt}\ {\isacharbraceleft}{\kern0pt}a{\isacharbraceright}{\kern0pt}{\isachardot}{\kern0pt}\ {\isacharparenleft}{\kern0pt}a{\isacharcomma}{\kern0pt}\ x{\isacharparenright}{\kern0pt}\ {\isasymin}\ r\ {\isasymand}\ {\isacharparenleft}{\kern0pt}x{\isacharcomma}{\kern0pt}\ a{\isacharparenright}{\kern0pt}\ {\isasymin}\ s{\isachardoublequoteclose}\isanewline
\ \ \ \ \isacommand{by}\isamarkupfalse%
\ simp\isanewline
\ \ \isacommand{show}\isamarkupfalse%
\ {\isachardoublequoteopen}{\isacharparenleft}{\kern0pt}x{\isacharcomma}{\kern0pt}\ a{\isacharparenright}{\kern0pt}\ {\isasymin}\ s{\isachardoublequoteclose}\isanewline
\ \ \isacommand{proof}\isamarkupfalse%
\ {\isacharparenleft}{\kern0pt}cases\ {\isachardoublequoteopen}x\ {\isacharequal}{\kern0pt}\ a{\isachardoublequoteclose}{\isacharparenright}{\kern0pt}\isanewline
\ \ \ \ \isacommand{case}\isamarkupfalse%
\ True\isanewline
\ \ \ \ \isacommand{thus}\isamarkupfalse%
\ {\isacharquery}{\kern0pt}thesis\isanewline
\ \ \ \ \ \ \isacommand{using}\isamarkupfalse%
\ connex{\isacharunderscore}{\kern0pt}imp{\isacharunderscore}{\kern0pt}refl\ equiv{\isacharunderscore}{\kern0pt}rel{\isacharunderscore}{\kern0pt}except{\isacharunderscore}{\kern0pt}a{\isacharunderscore}{\kern0pt}def\ refl{\isacharunderscore}{\kern0pt}onD\isanewline
\ \ \ \ \ \ \ \ \ \ \ \ lifted\ lifted{\isacharunderscore}{\kern0pt}def\ lin{\isacharunderscore}{\kern0pt}ord{\isacharunderscore}{\kern0pt}imp{\isacharunderscore}{\kern0pt}connex\isanewline
\ \ \ \ \ \ \isacommand{by}\isamarkupfalse%
\ metis\isanewline
\ \ \isacommand{next}\isamarkupfalse%
\isanewline
\ \ \ \ \isacommand{case}\isamarkupfalse%
\ False\isanewline
\ \ \ \ \isacommand{thus}\isamarkupfalse%
\ {\isacharquery}{\kern0pt}thesis\isanewline
\ \ \ \ \ \ \isacommand{using}\isamarkupfalse%
\ equiv{\isacharunderscore}{\kern0pt}rel{\isacharunderscore}{\kern0pt}except{\isacharunderscore}{\kern0pt}a{\isacharunderscore}{\kern0pt}def\ insertE\ insert{\isacharunderscore}{\kern0pt}Diff\isanewline
\ \ \ \ \ \ \ \ \ \ \ \ lifted\ lifted{\isacharunderscore}{\kern0pt}imp{\isacharunderscore}{\kern0pt}equiv{\isacharunderscore}{\kern0pt}rel{\isacharunderscore}{\kern0pt}except{\isacharunderscore}{\kern0pt}a\ x{\isacharunderscore}{\kern0pt}in{\isacharunderscore}{\kern0pt}A\isanewline
\ \ \ \ \ \ \ \ \ \ \ \ x{\isacharunderscore}{\kern0pt}pref{\isacharunderscore}{\kern0pt}a{\isacharunderscore}{\kern0pt}{\isadigit{0}}\ ex{\isacharunderscore}{\kern0pt}lifted\ lin{\isacharunderscore}{\kern0pt}imp{\isacharunderscore}{\kern0pt}trans\ rest{\isacharunderscore}{\kern0pt}eq\isanewline
\ \ \ \ \ \ \ \ \ \ \ \ trans{\isacharunderscore}{\kern0pt}def\isanewline
\ \ \ \ \ \ \isacommand{by}\isamarkupfalse%
\ metis\isanewline
\ \ \isacommand{qed}\isamarkupfalse%
\isanewline
\isacommand{qed}\isamarkupfalse%
%
\endisatagproof
{\isafoldproof}%
%
\isadelimproof
\isanewline
%
\endisadelimproof
\isanewline
\isacommand{lemma}\isamarkupfalse%
\ lifted{\isacharunderscore}{\kern0pt}above{\isacharcolon}{\kern0pt}\isanewline
\ \ \isakeyword{assumes}\ {\isachardoublequoteopen}lifted\ A\ r\ s\ a{\isachardoublequoteclose}\isanewline
\ \ \isakeyword{shows}\ {\isachardoublequoteopen}above\ s\ a\ {\isasymsubseteq}\ above\ r\ a{\isachardoublequoteclose}\isanewline
%
\isadelimproof
\ \ %
\endisadelimproof
%
\isatagproof
\isacommand{unfolding}\isamarkupfalse%
\ above{\isacharunderscore}{\kern0pt}def\isanewline
\isacommand{proof}\isamarkupfalse%
\ {\isacharparenleft}{\kern0pt}safe{\isacharparenright}{\kern0pt}\isanewline
\ \ \isacommand{fix}\isamarkupfalse%
\isanewline
\ \ \ \ x\ {\isacharcolon}{\kern0pt}{\isacharcolon}{\kern0pt}\ {\isachardoublequoteopen}{\isacharprime}{\kern0pt}a{\isachardoublequoteclose}\isanewline
\ \ \isacommand{assume}\isamarkupfalse%
\isanewline
\ \ \ \ a{\isacharunderscore}{\kern0pt}pref{\isacharunderscore}{\kern0pt}x{\isacharcolon}{\kern0pt}\ {\isachardoublequoteopen}{\isacharparenleft}{\kern0pt}a{\isacharcomma}{\kern0pt}\ x{\isacharparenright}{\kern0pt}\ {\isasymin}\ s{\isachardoublequoteclose}\isanewline
\ \ \isacommand{have}\isamarkupfalse%
\ {\isachardoublequoteopen}{\isasymexists}x\ {\isasymin}\ A\ {\isacharminus}{\kern0pt}\ {\isacharbraceleft}{\kern0pt}a{\isacharbraceright}{\kern0pt}{\isachardot}{\kern0pt}\ a\ {\isasympreceq}\isactrlsub r\ x\ {\isasymand}\ x\ {\isasympreceq}\isactrlsub s\ a{\isachardoublequoteclose}\isanewline
\ \ \ \ \isacommand{using}\isamarkupfalse%
\ assms\ lifted{\isacharunderscore}{\kern0pt}def\isanewline
\ \ \ \ \isacommand{by}\isamarkupfalse%
\ metis\isanewline
\ \ \isacommand{hence}\isamarkupfalse%
\ lifted{\isacharunderscore}{\kern0pt}r{\isacharcolon}{\kern0pt}\isanewline
\ \ \ \ {\isachardoublequoteopen}{\isasymexists}x\ {\isasymin}\ A\ {\isacharminus}{\kern0pt}\ {\isacharbraceleft}{\kern0pt}a{\isacharbraceright}{\kern0pt}{\isachardot}{\kern0pt}\ {\isacharparenleft}{\kern0pt}a{\isacharcomma}{\kern0pt}\ x{\isacharparenright}{\kern0pt}\ {\isasymin}\ r\ {\isasymand}\ {\isacharparenleft}{\kern0pt}x{\isacharcomma}{\kern0pt}\ a{\isacharparenright}{\kern0pt}\ {\isasymin}\ s{\isachardoublequoteclose}\isanewline
\ \ \ \ \isacommand{by}\isamarkupfalse%
\ simp\isanewline
\ \ \isacommand{have}\isamarkupfalse%
\ {\isachardoublequoteopen}{\isasymforall}x\ {\isasymin}\ A\ {\isacharminus}{\kern0pt}\ {\isacharbraceleft}{\kern0pt}a{\isacharbraceright}{\kern0pt}{\isachardot}{\kern0pt}\ {\isasymforall}y\ {\isasymin}\ A\ {\isacharminus}{\kern0pt}\ {\isacharbraceleft}{\kern0pt}a{\isacharbraceright}{\kern0pt}{\isachardot}{\kern0pt}\ x\ {\isasympreceq}\isactrlsub r\ y\ {\isasymlongleftrightarrow}\ x\ {\isasympreceq}\isactrlsub s\ y{\isachardoublequoteclose}\isanewline
\ \ \ \ \isacommand{using}\isamarkupfalse%
\ assms\ lifted{\isacharunderscore}{\kern0pt}def\ equiv{\isacharunderscore}{\kern0pt}rel{\isacharunderscore}{\kern0pt}except{\isacharunderscore}{\kern0pt}a{\isacharunderscore}{\kern0pt}def\isanewline
\ \ \ \ \isacommand{by}\isamarkupfalse%
\ metis\isanewline
\ \ \isacommand{hence}\isamarkupfalse%
\ rest{\isacharunderscore}{\kern0pt}eq{\isacharcolon}{\kern0pt}\isanewline
\ \ \ \ {\isachardoublequoteopen}{\isasymforall}x\ {\isasymin}\ A\ {\isacharminus}{\kern0pt}\ {\isacharbraceleft}{\kern0pt}a{\isacharbraceright}{\kern0pt}{\isachardot}{\kern0pt}\ {\isasymforall}y\ {\isasymin}\ A\ {\isacharminus}{\kern0pt}\ {\isacharbraceleft}{\kern0pt}a{\isacharbraceright}{\kern0pt}{\isachardot}{\kern0pt}\ {\isacharparenleft}{\kern0pt}x{\isacharcomma}{\kern0pt}\ y{\isacharparenright}{\kern0pt}\ {\isasymin}\ r\ {\isasymlongleftrightarrow}\ {\isacharparenleft}{\kern0pt}x{\isacharcomma}{\kern0pt}\ y{\isacharparenright}{\kern0pt}\ {\isasymin}\ s{\isachardoublequoteclose}\isanewline
\ \ \ \ \isacommand{by}\isamarkupfalse%
\ simp\isanewline
\ \ \isacommand{have}\isamarkupfalse%
\ trans{\isacharunderscore}{\kern0pt}r{\isacharcolon}{\kern0pt}\isanewline
\ \ \ \ {\isachardoublequoteopen}{\isasymforall}x\ y\ z{\isachardot}{\kern0pt}\ {\isacharparenleft}{\kern0pt}x{\isacharcomma}{\kern0pt}\ y{\isacharparenright}{\kern0pt}\ {\isasymin}\ r\ {\isasymlongrightarrow}\ {\isacharparenleft}{\kern0pt}y{\isacharcomma}{\kern0pt}\ z{\isacharparenright}{\kern0pt}\ {\isasymin}\ r\ {\isasymlongrightarrow}\ {\isacharparenleft}{\kern0pt}x{\isacharcomma}{\kern0pt}\ z{\isacharparenright}{\kern0pt}\ {\isasymin}\ r{\isachardoublequoteclose}\isanewline
\ \ \ \ \isacommand{using}\isamarkupfalse%
\ trans{\isacharunderscore}{\kern0pt}def\ lifted{\isacharunderscore}{\kern0pt}def\ lin{\isacharunderscore}{\kern0pt}imp{\isacharunderscore}{\kern0pt}trans\isanewline
\ \ \ \ \ \ \ \ \ \ equiv{\isacharunderscore}{\kern0pt}rel{\isacharunderscore}{\kern0pt}except{\isacharunderscore}{\kern0pt}a{\isacharunderscore}{\kern0pt}def\ assms\isanewline
\ \ \ \ \isacommand{by}\isamarkupfalse%
\ metis\isanewline
\ \ \isacommand{have}\isamarkupfalse%
\ trans{\isacharunderscore}{\kern0pt}s{\isacharcolon}{\kern0pt}\isanewline
\ \ \ \ {\isachardoublequoteopen}{\isasymforall}x\ y\ z{\isachardot}{\kern0pt}\ {\isacharparenleft}{\kern0pt}x{\isacharcomma}{\kern0pt}\ y{\isacharparenright}{\kern0pt}\ {\isasymin}\ s\ {\isasymlongrightarrow}\ {\isacharparenleft}{\kern0pt}y{\isacharcomma}{\kern0pt}\ z{\isacharparenright}{\kern0pt}\ {\isasymin}\ s\ {\isasymlongrightarrow}\ {\isacharparenleft}{\kern0pt}x{\isacharcomma}{\kern0pt}\ z{\isacharparenright}{\kern0pt}\ {\isasymin}\ s{\isachardoublequoteclose}\isanewline
\ \ \ \ \isacommand{using}\isamarkupfalse%
\ trans{\isacharunderscore}{\kern0pt}def\ lifted{\isacharunderscore}{\kern0pt}def\ lin{\isacharunderscore}{\kern0pt}imp{\isacharunderscore}{\kern0pt}trans\isanewline
\ \ \ \ \ \ \ \ \ \ equiv{\isacharunderscore}{\kern0pt}rel{\isacharunderscore}{\kern0pt}except{\isacharunderscore}{\kern0pt}a{\isacharunderscore}{\kern0pt}def\ assms\isanewline
\ \ \ \ \isacommand{by}\isamarkupfalse%
\ metis\isanewline
\ \ \isacommand{have}\isamarkupfalse%
\ refl{\isacharunderscore}{\kern0pt}r{\isacharcolon}{\kern0pt}\isanewline
\ \ \ \ {\isachardoublequoteopen}{\isacharparenleft}{\kern0pt}a{\isacharcomma}{\kern0pt}\ a{\isacharparenright}{\kern0pt}\ {\isasymin}\ r{\isachardoublequoteclose}\isanewline
\ \ \ \ \isacommand{using}\isamarkupfalse%
\ assms\ connex{\isacharunderscore}{\kern0pt}imp{\isacharunderscore}{\kern0pt}refl\ equiv{\isacharunderscore}{\kern0pt}rel{\isacharunderscore}{\kern0pt}except{\isacharunderscore}{\kern0pt}a{\isacharunderscore}{\kern0pt}def\isanewline
\ \ \ \ \ \ \ \ \ \ lifted{\isacharunderscore}{\kern0pt}def\ lin{\isacharunderscore}{\kern0pt}ord{\isacharunderscore}{\kern0pt}imp{\isacharunderscore}{\kern0pt}connex\ refl{\isacharunderscore}{\kern0pt}onD\isanewline
\ \ \ \ \isacommand{by}\isamarkupfalse%
\ metis\isanewline
\ \ \isacommand{have}\isamarkupfalse%
\ x{\isacharunderscore}{\kern0pt}in{\isacharunderscore}{\kern0pt}A{\isacharcolon}{\kern0pt}\ {\isachardoublequoteopen}x\ {\isasymin}\ A{\isachardoublequoteclose}\isanewline
\ \ \ \ \isacommand{using}\isamarkupfalse%
\ a{\isacharunderscore}{\kern0pt}pref{\isacharunderscore}{\kern0pt}x\ assms\ connex{\isacharunderscore}{\kern0pt}imp{\isacharunderscore}{\kern0pt}refl\ equiv{\isacharunderscore}{\kern0pt}rel{\isacharunderscore}{\kern0pt}except{\isacharunderscore}{\kern0pt}a{\isacharunderscore}{\kern0pt}def\isanewline
\ \ \ \ \ \ \ \ \ \ lifted{\isacharunderscore}{\kern0pt}def\ lin{\isacharunderscore}{\kern0pt}ord{\isacharunderscore}{\kern0pt}imp{\isacharunderscore}{\kern0pt}connex\ refl{\isacharunderscore}{\kern0pt}onD{\isadigit{2}}\isanewline
\ \ \ \ \isacommand{by}\isamarkupfalse%
\ metis\isanewline
\ \ \isacommand{show}\isamarkupfalse%
\ {\isachardoublequoteopen}{\isacharparenleft}{\kern0pt}a{\isacharcomma}{\kern0pt}\ x{\isacharparenright}{\kern0pt}\ {\isasymin}\ r{\isachardoublequoteclose}\isanewline
\ \ \ \ \isacommand{using}\isamarkupfalse%
\ Diff{\isacharunderscore}{\kern0pt}iff\ a{\isacharunderscore}{\kern0pt}pref{\isacharunderscore}{\kern0pt}x\ lifted{\isacharunderscore}{\kern0pt}r\ rest{\isacharunderscore}{\kern0pt}eq\ singletonD\isanewline
\ \ \ \ \ \ \ \ \ \ trans{\isacharunderscore}{\kern0pt}r\ trans{\isacharunderscore}{\kern0pt}s\ x{\isacharunderscore}{\kern0pt}in{\isacharunderscore}{\kern0pt}A\ refl{\isacharunderscore}{\kern0pt}r\isanewline
\ \ \ \ \isacommand{by}\isamarkupfalse%
\ {\isacharparenleft}{\kern0pt}metis\ {\isacharparenleft}{\kern0pt}full{\isacharunderscore}{\kern0pt}types{\isacharparenright}{\kern0pt}{\isacharparenright}{\kern0pt}\isanewline
\isacommand{qed}\isamarkupfalse%
%
\endisatagproof
{\isafoldproof}%
%
\isadelimproof
\isanewline
%
\endisadelimproof
\isanewline
\isacommand{lemma}\isamarkupfalse%
\ lifted{\isacharunderscore}{\kern0pt}above{\isadigit{2}}{\isacharcolon}{\kern0pt}\isanewline
\ \ \isakeyword{assumes}\isanewline
\ \ \ \ {\isachardoublequoteopen}lifted\ A\ r\ s\ a{\isachardoublequoteclose}\ \isakeyword{and}\isanewline
\ \ \ \ {\isachardoublequoteopen}x\ {\isasymin}\ A{\isacharminus}{\kern0pt}{\isacharbraceleft}{\kern0pt}a{\isacharbraceright}{\kern0pt}{\isachardoublequoteclose}\isanewline
\ \ \isakeyword{shows}\ {\isachardoublequoteopen}above\ r\ x\ {\isasymsubseteq}\ above\ s\ x\ {\isasymunion}\ {\isacharbraceleft}{\kern0pt}a{\isacharbraceright}{\kern0pt}{\isachardoublequoteclose}\isanewline
%
\isadelimproof
%
\endisadelimproof
%
\isatagproof
\isacommand{proof}\isamarkupfalse%
\ {\isacharparenleft}{\kern0pt}safe{\isacharcomma}{\kern0pt}\ simp{\isacharparenright}{\kern0pt}\isanewline
\ \ \isacommand{fix}\isamarkupfalse%
\ y\ {\isacharcolon}{\kern0pt}{\isacharcolon}{\kern0pt}\ {\isachardoublequoteopen}{\isacharprime}{\kern0pt}a{\isachardoublequoteclose}\isanewline
\ \ \isacommand{assume}\isamarkupfalse%
\isanewline
\ \ \ \ y{\isacharunderscore}{\kern0pt}in{\isacharunderscore}{\kern0pt}above{\isacharunderscore}{\kern0pt}r{\isacharcolon}{\kern0pt}\ {\isachardoublequoteopen}y\ {\isasymin}\ above\ r\ x{\isachardoublequoteclose}\ \isakeyword{and}\isanewline
\ \ \ \ y{\isacharunderscore}{\kern0pt}not{\isacharunderscore}{\kern0pt}in{\isacharunderscore}{\kern0pt}above{\isacharunderscore}{\kern0pt}s{\isacharcolon}{\kern0pt}\ {\isachardoublequoteopen}y\ {\isasymnotin}\ above\ s\ x{\isachardoublequoteclose}\isanewline
\ \ \isacommand{have}\isamarkupfalse%
\ {\isachardoublequoteopen}{\isasymforall}z\ {\isasymin}\ A{\isacharminus}{\kern0pt}{\isacharbraceleft}{\kern0pt}a{\isacharbraceright}{\kern0pt}{\isachardot}{\kern0pt}\ x\ {\isasympreceq}\isactrlsub r\ z\ {\isasymlongleftrightarrow}\ x\ {\isasympreceq}\isactrlsub s\ z{\isachardoublequoteclose}\isanewline
\ \ \ \ \isacommand{using}\isamarkupfalse%
\ assms\ lifted{\isacharunderscore}{\kern0pt}def\ equiv{\isacharunderscore}{\kern0pt}rel{\isacharunderscore}{\kern0pt}except{\isacharunderscore}{\kern0pt}a{\isacharunderscore}{\kern0pt}def\isanewline
\ \ \ \ \isacommand{by}\isamarkupfalse%
\ metis\isanewline
\ \ \isacommand{hence}\isamarkupfalse%
\ {\isachardoublequoteopen}{\isasymforall}z\ {\isasymin}\ A{\isacharminus}{\kern0pt}{\isacharbraceleft}{\kern0pt}a{\isacharbraceright}{\kern0pt}{\isachardot}{\kern0pt}\ {\isacharparenleft}{\kern0pt}x{\isacharcomma}{\kern0pt}\ z{\isacharparenright}{\kern0pt}\ {\isasymin}\ r\ {\isasymlongleftrightarrow}\ {\isacharparenleft}{\kern0pt}x{\isacharcomma}{\kern0pt}\ z{\isacharparenright}{\kern0pt}\ {\isasymin}\ s{\isachardoublequoteclose}\isanewline
\ \ \ \ \isacommand{by}\isamarkupfalse%
\ simp\isanewline
\ \ \isacommand{hence}\isamarkupfalse%
\ {\isachardoublequoteopen}{\isasymforall}z\ {\isasymin}\ A{\isacharminus}{\kern0pt}{\isacharbraceleft}{\kern0pt}a{\isacharbraceright}{\kern0pt}{\isachardot}{\kern0pt}\ z\ {\isasymin}\ above\ r\ x\ {\isasymlongleftrightarrow}\ z\ {\isasymin}\ above\ s\ x{\isachardoublequoteclose}\isanewline
\ \ \ \ \isacommand{by}\isamarkupfalse%
\ {\isacharparenleft}{\kern0pt}simp\ add{\isacharcolon}{\kern0pt}\ above{\isacharunderscore}{\kern0pt}def{\isacharparenright}{\kern0pt}\isanewline
\ \ \isacommand{hence}\isamarkupfalse%
\ {\isachardoublequoteopen}y\ {\isasymin}\ above\ r\ x\ {\isasymlongleftrightarrow}\ y\ {\isasymin}\ above\ s\ x{\isachardoublequoteclose}\isanewline
\ \ \ \ \isacommand{using}\isamarkupfalse%
\ y{\isacharunderscore}{\kern0pt}not{\isacharunderscore}{\kern0pt}in{\isacharunderscore}{\kern0pt}above{\isacharunderscore}{\kern0pt}s\ assms{\isacharparenleft}{\kern0pt}{\isadigit{1}}{\isacharparenright}{\kern0pt}\ connex{\isacharunderscore}{\kern0pt}def\isanewline
\ \ \ \ \ \ \ \ \ \ equiv{\isacharunderscore}{\kern0pt}rel{\isacharunderscore}{\kern0pt}except{\isacharunderscore}{\kern0pt}a{\isacharunderscore}{\kern0pt}def\ lifted{\isacharunderscore}{\kern0pt}def\ lifted{\isacharunderscore}{\kern0pt}mono{\isadigit{2}}\isanewline
\ \ \ \ \ \ \ \ \ \ limited{\isacharunderscore}{\kern0pt}dest\ lin{\isacharunderscore}{\kern0pt}ord{\isacharunderscore}{\kern0pt}imp{\isacharunderscore}{\kern0pt}connex\ member{\isacharunderscore}{\kern0pt}remove\isanewline
\ \ \ \ \ \ \ \ \ \ pref{\isacharunderscore}{\kern0pt}imp{\isacharunderscore}{\kern0pt}in{\isacharunderscore}{\kern0pt}above\ remove{\isacharunderscore}{\kern0pt}def\isanewline
\ \ \ \ \isacommand{by}\isamarkupfalse%
\ metis\isanewline
\ \ \isacommand{thus}\isamarkupfalse%
\ {\isachardoublequoteopen}y\ {\isacharequal}{\kern0pt}\ a{\isachardoublequoteclose}\isanewline
\ \ \ \ \isacommand{using}\isamarkupfalse%
\ y{\isacharunderscore}{\kern0pt}in{\isacharunderscore}{\kern0pt}above{\isacharunderscore}{\kern0pt}r\ y{\isacharunderscore}{\kern0pt}not{\isacharunderscore}{\kern0pt}in{\isacharunderscore}{\kern0pt}above{\isacharunderscore}{\kern0pt}s\isanewline
\ \ \ \ \isacommand{by}\isamarkupfalse%
\ simp\isanewline
\isacommand{qed}\isamarkupfalse%
%
\endisatagproof
{\isafoldproof}%
%
\isadelimproof
\isanewline
%
\endisadelimproof
\isanewline
\isacommand{lemma}\isamarkupfalse%
\ limit{\isacharunderscore}{\kern0pt}lifted{\isacharunderscore}{\kern0pt}imp{\isacharunderscore}{\kern0pt}eq{\isacharunderscore}{\kern0pt}or{\isacharunderscore}{\kern0pt}lifted{\isacharcolon}{\kern0pt}\isanewline
\ \ \isakeyword{assumes}\isanewline
\ \ \ \ lifted{\isacharcolon}{\kern0pt}\ {\isachardoublequoteopen}lifted\ S\ r\ s\ a{\isachardoublequoteclose}\ \isakeyword{and}\isanewline
\ \ \ \ subset{\isacharcolon}{\kern0pt}\ {\isachardoublequoteopen}A\ {\isasymsubseteq}\ S{\isachardoublequoteclose}\isanewline
\ \ \isakeyword{shows}\isanewline
\ \ \ \ {\isachardoublequoteopen}limit\ A\ r\ {\isacharequal}{\kern0pt}\ limit\ A\ s\ {\isasymor}\isanewline
\ \ \ \ \ \ lifted\ A\ {\isacharparenleft}{\kern0pt}limit\ A\ r{\isacharparenright}{\kern0pt}\ {\isacharparenleft}{\kern0pt}limit\ A\ s{\isacharparenright}{\kern0pt}\ a{\isachardoublequoteclose}\isanewline
%
\isadelimproof
%
\endisadelimproof
%
\isatagproof
\isacommand{proof}\isamarkupfalse%
\ {\isacharminus}{\kern0pt}\isanewline
\ \ \isacommand{from}\isamarkupfalse%
\ lifted\ \isacommand{have}\isamarkupfalse%
\isanewline
\ \ \ \ {\isachardoublequoteopen}{\isasymforall}x\ {\isasymin}\ S\ {\isacharminus}{\kern0pt}\ {\isacharbraceleft}{\kern0pt}a{\isacharbraceright}{\kern0pt}{\isachardot}{\kern0pt}\ {\isasymforall}y\ {\isasymin}\ S\ {\isacharminus}{\kern0pt}\ {\isacharbraceleft}{\kern0pt}a{\isacharbraceright}{\kern0pt}{\isachardot}{\kern0pt}\ x\ {\isasympreceq}\isactrlsub r\ y\ {\isasymlongleftrightarrow}\ x\ {\isasympreceq}\isactrlsub s\ y{\isachardoublequoteclose}\isanewline
\ \ \ \ \isacommand{by}\isamarkupfalse%
\ {\isacharparenleft}{\kern0pt}simp\ add{\isacharcolon}{\kern0pt}\ lifted{\isacharunderscore}{\kern0pt}def\ equiv{\isacharunderscore}{\kern0pt}rel{\isacharunderscore}{\kern0pt}except{\isacharunderscore}{\kern0pt}a{\isacharunderscore}{\kern0pt}def{\isacharparenright}{\kern0pt}\isanewline
\ \ \isacommand{with}\isamarkupfalse%
\ subset\ \isacommand{have}\isamarkupfalse%
\ temp{\isacharcolon}{\kern0pt}\isanewline
\ \ \ \ {\isachardoublequoteopen}{\isasymforall}x\ {\isasymin}\ A\ {\isacharminus}{\kern0pt}\ {\isacharbraceleft}{\kern0pt}a{\isacharbraceright}{\kern0pt}{\isachardot}{\kern0pt}\ {\isasymforall}y\ {\isasymin}\ A\ {\isacharminus}{\kern0pt}\ {\isacharbraceleft}{\kern0pt}a{\isacharbraceright}{\kern0pt}{\isachardot}{\kern0pt}\ x\ {\isasympreceq}\isactrlsub r\ y\ {\isasymlongleftrightarrow}\ x\ {\isasympreceq}\isactrlsub s\ y{\isachardoublequoteclose}\isanewline
\ \ \ \ \isacommand{by}\isamarkupfalse%
\ auto\isanewline
\ \ \isacommand{hence}\isamarkupfalse%
\ eql{\isacharunderscore}{\kern0pt}rs{\isacharcolon}{\kern0pt}\isanewline
\ \ \ \ \ \ {\isachardoublequoteopen}{\isasymforall}x\ {\isasymin}\ A\ {\isacharminus}{\kern0pt}\ {\isacharbraceleft}{\kern0pt}a{\isacharbraceright}{\kern0pt}{\isachardot}{\kern0pt}\ {\isasymforall}y\ {\isasymin}\ A\ {\isacharminus}{\kern0pt}\ {\isacharbraceleft}{\kern0pt}a{\isacharbraceright}{\kern0pt}{\isachardot}{\kern0pt}\isanewline
\ \ \ \ \ \ {\isacharparenleft}{\kern0pt}x{\isacharcomma}{\kern0pt}\ y{\isacharparenright}{\kern0pt}\ {\isasymin}\ {\isacharparenleft}{\kern0pt}limit\ A\ r{\isacharparenright}{\kern0pt}\ {\isasymlongleftrightarrow}\ {\isacharparenleft}{\kern0pt}x{\isacharcomma}{\kern0pt}\ y{\isacharparenright}{\kern0pt}\ {\isasymin}\ {\isacharparenleft}{\kern0pt}limit\ A\ s{\isacharparenright}{\kern0pt}{\isachardoublequoteclose}\isanewline
\ \ \ \ \isacommand{using}\isamarkupfalse%
\ DiffD{\isadigit{1}}\ limit{\isacharunderscore}{\kern0pt}presv{\isacharunderscore}{\kern0pt}prefs{\isadigit{1}}\ limit{\isacharunderscore}{\kern0pt}presv{\isacharunderscore}{\kern0pt}prefs{\isadigit{2}}\isanewline
\ \ \ \ \isacommand{by}\isamarkupfalse%
\ auto\isanewline
\ \ \isacommand{show}\isamarkupfalse%
\ {\isacharquery}{\kern0pt}thesis\isanewline
\ \ \isacommand{proof}\isamarkupfalse%
\ cases\isanewline
\ \ \ \ \isacommand{assume}\isamarkupfalse%
\ a{\isadigit{1}}{\isacharcolon}{\kern0pt}\ {\isachardoublequoteopen}a\ {\isasymin}\ A{\isachardoublequoteclose}\isanewline
\ \ \ \ \isacommand{thus}\isamarkupfalse%
\ {\isacharquery}{\kern0pt}thesis\isanewline
\ \ \ \ \isacommand{proof}\isamarkupfalse%
\ cases\isanewline
\ \ \ \ \ \ \isanewline
\ \ \ \ \ \ \isacommand{assume}\isamarkupfalse%
\ a{\isadigit{1}}{\isacharunderscore}{\kern0pt}{\isadigit{1}}{\isacharcolon}{\kern0pt}\ {\isachardoublequoteopen}{\isasymexists}x\ {\isasymin}\ A\ {\isacharminus}{\kern0pt}\ {\isacharbraceleft}{\kern0pt}a{\isacharbraceright}{\kern0pt}{\isachardot}{\kern0pt}\ a\ {\isasympreceq}\isactrlsub r\ x\ {\isasymand}\ x\ {\isasympreceq}\isactrlsub s\ a{\isachardoublequoteclose}\ \isanewline
\ \ \ \ \ \ \isacommand{from}\isamarkupfalse%
\ lifted\ subset\ \isacommand{have}\isamarkupfalse%
\isanewline
\ \ \ \ \ \ \ \ {\isachardoublequoteopen}linear{\isacharunderscore}{\kern0pt}order{\isacharunderscore}{\kern0pt}on\ A\ {\isacharparenleft}{\kern0pt}limit\ A\ r{\isacharparenright}{\kern0pt}\ {\isasymand}\ linear{\isacharunderscore}{\kern0pt}order{\isacharunderscore}{\kern0pt}on\ A\ {\isacharparenleft}{\kern0pt}limit\ A\ s{\isacharparenright}{\kern0pt}{\isachardoublequoteclose}\isanewline
\ \ \ \ \ \ \ \ \isacommand{using}\isamarkupfalse%
\ lifted{\isacharunderscore}{\kern0pt}def\ equiv{\isacharunderscore}{\kern0pt}rel{\isacharunderscore}{\kern0pt}except{\isacharunderscore}{\kern0pt}a{\isacharunderscore}{\kern0pt}def\ limit{\isacharunderscore}{\kern0pt}presv{\isacharunderscore}{\kern0pt}lin{\isacharunderscore}{\kern0pt}ord\isanewline
\ \ \ \ \ \ \ \ \isacommand{by}\isamarkupfalse%
\ metis\isanewline
\ \ \ \ \ \ \isacommand{moreover}\isamarkupfalse%
\ \isacommand{from}\isamarkupfalse%
\ a{\isadigit{1}}\ a{\isadigit{1}}{\isacharunderscore}{\kern0pt}{\isadigit{1}}\ \isacommand{have}\isamarkupfalse%
\ keep{\isacharunderscore}{\kern0pt}lift{\isacharcolon}{\kern0pt}\isanewline
\ \ \ \ \ \ \ \ {\isachardoublequoteopen}{\isasymexists}x\ {\isasymin}\ A\ {\isacharminus}{\kern0pt}\ {\isacharbraceleft}{\kern0pt}a{\isacharbraceright}{\kern0pt}{\isachardot}{\kern0pt}\ {\isacharparenleft}{\kern0pt}let\ q\ {\isacharequal}{\kern0pt}\ limit\ A\ r\ in\ a\ {\isasympreceq}\isactrlsub q\ x{\isacharparenright}{\kern0pt}\ {\isasymand}\isanewline
\ \ \ \ \ \ \ \ \ \ \ \ {\isacharparenleft}{\kern0pt}let\ u\ {\isacharequal}{\kern0pt}\ limit\ A\ s\ in\ x\ {\isasympreceq}\isactrlsub u\ a{\isacharparenright}{\kern0pt}{\isachardoublequoteclose}\isanewline
\ \ \ \ \ \ \ \ \isacommand{using}\isamarkupfalse%
\ DiffD{\isadigit{1}}\ limit{\isacharunderscore}{\kern0pt}presv{\isacharunderscore}{\kern0pt}prefs{\isadigit{1}}\isanewline
\ \ \ \ \ \ \ \ \isacommand{by}\isamarkupfalse%
\ simp\isanewline
\ \ \ \ \ \ \isacommand{ultimately}\isamarkupfalse%
\ \isacommand{show}\isamarkupfalse%
\ {\isacharquery}{\kern0pt}thesis\isanewline
\ \ \ \ \ \ \ \ \isacommand{using}\isamarkupfalse%
\ a{\isadigit{1}}\ temp\isanewline
\ \ \ \ \ \ \ \ \isacommand{by}\isamarkupfalse%
\ {\isacharparenleft}{\kern0pt}simp\ add{\isacharcolon}{\kern0pt}\ lifted{\isacharunderscore}{\kern0pt}def\ equiv{\isacharunderscore}{\kern0pt}rel{\isacharunderscore}{\kern0pt}except{\isacharunderscore}{\kern0pt}a{\isacharunderscore}{\kern0pt}def{\isacharparenright}{\kern0pt}\isanewline
\ \ \ \ \isacommand{next}\isamarkupfalse%
\isanewline
\ \ \ \ \ \ \isacommand{assume}\isamarkupfalse%
\isanewline
\ \ \ \ \ \ \ \ {\isachardoublequoteopen}{\isasymnot}{\isacharparenleft}{\kern0pt}{\isasymexists}x\ {\isasymin}\ A\ {\isacharminus}{\kern0pt}\ {\isacharbraceleft}{\kern0pt}a{\isacharbraceright}{\kern0pt}{\isachardot}{\kern0pt}\ a\ {\isasympreceq}\isactrlsub r\ x\ {\isasymand}\ x\ {\isasympreceq}\isactrlsub s\ a{\isacharparenright}{\kern0pt}{\isachardoublequoteclose}\ \isanewline
\ \ \ \ \ \ \isacommand{hence}\isamarkupfalse%
\ a{\isadigit{1}}{\isacharunderscore}{\kern0pt}{\isadigit{2}}{\isacharcolon}{\kern0pt}\isanewline
\ \ \ \ \ \ \ \ {\isachardoublequoteopen}{\isasymforall}x\ {\isasymin}\ A\ {\isacharminus}{\kern0pt}\ {\isacharbraceleft}{\kern0pt}a{\isacharbraceright}{\kern0pt}{\isachardot}{\kern0pt}\ {\isasymnot}{\isacharparenleft}{\kern0pt}a\ {\isasympreceq}\isactrlsub r\ x\ {\isasymand}\ x\ {\isasympreceq}\isactrlsub s\ a{\isacharparenright}{\kern0pt}{\isachardoublequoteclose}\isanewline
\ \ \ \ \ \ \ \ \isacommand{by}\isamarkupfalse%
\ auto\isanewline
\ \ \ \ \ \ \isacommand{moreover}\isamarkupfalse%
\ \isacommand{have}\isamarkupfalse%
\ not{\isacharunderscore}{\kern0pt}worse{\isacharcolon}{\kern0pt}\isanewline
\ \ \ \ \ \ \ \ {\isachardoublequoteopen}{\isasymforall}x\ {\isasymin}\ A\ {\isacharminus}{\kern0pt}\ {\isacharbraceleft}{\kern0pt}a{\isacharbraceright}{\kern0pt}{\isachardot}{\kern0pt}\ {\isasymnot}{\isacharparenleft}{\kern0pt}x\ {\isasympreceq}\isactrlsub r\ a\ {\isasymand}\ a\ {\isasympreceq}\isactrlsub s\ x{\isacharparenright}{\kern0pt}{\isachardoublequoteclose}\isanewline
\ \ \ \ \ \ \ \ \isacommand{using}\isamarkupfalse%
\ lifted\ subset\ lifted{\isacharunderscore}{\kern0pt}mono\isanewline
\ \ \ \ \ \ \ \ \isacommand{by}\isamarkupfalse%
\ fastforce\isanewline
\ \ \ \ \ \ \isacommand{moreover}\isamarkupfalse%
\ \isacommand{have}\isamarkupfalse%
\ connex{\isacharcolon}{\kern0pt}\isanewline
\ \ \ \ \ \ \ \ {\isachardoublequoteopen}connex\ A\ {\isacharparenleft}{\kern0pt}limit\ A\ r{\isacharparenright}{\kern0pt}\ {\isasymand}\ connex\ A\ {\isacharparenleft}{\kern0pt}limit\ A\ s{\isacharparenright}{\kern0pt}{\isachardoublequoteclose}\isanewline
\ \ \ \ \ \ \ \ \isacommand{using}\isamarkupfalse%
\ lifted\ subset\ lifted{\isacharunderscore}{\kern0pt}def\ equiv{\isacharunderscore}{\kern0pt}rel{\isacharunderscore}{\kern0pt}except{\isacharunderscore}{\kern0pt}a{\isacharunderscore}{\kern0pt}def\isanewline
\ \ \ \ \ \ \ \ \ \ \ \ \ \ limit{\isacharunderscore}{\kern0pt}presv{\isacharunderscore}{\kern0pt}lin{\isacharunderscore}{\kern0pt}ord\ lin{\isacharunderscore}{\kern0pt}ord{\isacharunderscore}{\kern0pt}imp{\isacharunderscore}{\kern0pt}connex\isanewline
\ \ \ \ \ \ \ \ \isacommand{by}\isamarkupfalse%
\ metis\isanewline
\ \ \ \ \ \ \isacommand{moreover}\isamarkupfalse%
\ \isacommand{have}\isamarkupfalse%
\ connex{\isadigit{1}}{\isacharcolon}{\kern0pt}\isanewline
\ \ \ \ \ \ \ \ {\isachardoublequoteopen}{\isasymforall}A\ r{\isachardot}{\kern0pt}\ connex\ A\ r\ {\isacharequal}{\kern0pt}\isanewline
\ \ \ \ \ \ \ \ \ \ {\isacharparenleft}{\kern0pt}limited\ A\ r\ {\isasymand}\ {\isacharparenleft}{\kern0pt}{\isasymforall}a{\isachardot}{\kern0pt}\ {\isacharparenleft}{\kern0pt}a{\isacharcolon}{\kern0pt}{\isacharcolon}{\kern0pt}{\isacharprime}{\kern0pt}a{\isacharparenright}{\kern0pt}\ {\isasymin}\ A\ {\isasymlongrightarrow}\isanewline
\ \ \ \ \ \ \ \ \ \ \ \ {\isacharparenleft}{\kern0pt}{\isasymforall}aa{\isachardot}{\kern0pt}\ aa\ {\isasymin}\ A\ {\isasymlongrightarrow}\ a\ {\isasympreceq}\isactrlsub r\ aa\ {\isasymor}\ aa\ {\isasympreceq}\isactrlsub r\ a{\isacharparenright}{\kern0pt}{\isacharparenright}{\kern0pt}{\isacharparenright}{\kern0pt}{\isachardoublequoteclose}\isanewline
\ \ \ \ \ \ \ \ \isacommand{by}\isamarkupfalse%
\ {\isacharparenleft}{\kern0pt}simp\ add{\isacharcolon}{\kern0pt}\ Ball{\isacharunderscore}{\kern0pt}def{\isacharunderscore}{\kern0pt}raw\ connex{\isacharunderscore}{\kern0pt}def{\isacharparenright}{\kern0pt}\isanewline
\ \ \ \ \ \ \isacommand{hence}\isamarkupfalse%
\ limit{\isadigit{1}}{\isacharcolon}{\kern0pt}\isanewline
\ \ \ \ \ \ \ \ {\isachardoublequoteopen}limited\ A\ {\isacharparenleft}{\kern0pt}limit\ A\ r{\isacharparenright}{\kern0pt}\ {\isasymand}\isanewline
\ \ \ \ \ \ \ \ \ \ {\isacharparenleft}{\kern0pt}{\isasymforall}a{\isachardot}{\kern0pt}\ a\ {\isasymnotin}\ A\ {\isasymor}\isanewline
\ \ \ \ \ \ \ \ \ \ \ \ {\isacharparenleft}{\kern0pt}{\isasymforall}aa{\isachardot}{\kern0pt}\isanewline
\ \ \ \ \ \ \ \ \ \ \ \ \ \ aa\ {\isasymnotin}\ A\ {\isasymor}\ {\isacharparenleft}{\kern0pt}a{\isacharcomma}{\kern0pt}\ aa{\isacharparenright}{\kern0pt}\ {\isasymin}\ limit\ A\ r\ {\isasymor}\isanewline
\ \ \ \ \ \ \ \ \ \ \ \ \ \ \ \ {\isacharparenleft}{\kern0pt}aa{\isacharcomma}{\kern0pt}\ a{\isacharparenright}{\kern0pt}\ {\isasymin}\ limit\ A\ r\ {\isacharparenright}{\kern0pt}{\isacharparenright}{\kern0pt}{\isachardoublequoteclose}\isanewline
\ \ \ \ \ \ \ \ \isacommand{using}\isamarkupfalse%
\ connex\ connex{\isadigit{1}}\isanewline
\ \ \ \ \ \ \ \ \isacommand{by}\isamarkupfalse%
\ simp\isanewline
\ \ \ \ \ \ \isacommand{have}\isamarkupfalse%
\ limit{\isadigit{2}}{\isacharcolon}{\kern0pt}\isanewline
\ \ \ \ \ \ \ \ {\isachardoublequoteopen}{\isasymforall}a\ aa\ A\ r{\isachardot}{\kern0pt}\ {\isacharparenleft}{\kern0pt}a{\isacharcolon}{\kern0pt}{\isacharcolon}{\kern0pt}{\isacharprime}{\kern0pt}a{\isacharcomma}{\kern0pt}\ aa{\isacharparenright}{\kern0pt}\ {\isasymnotin}\ limit\ A\ r\ {\isasymor}\ a\ {\isasympreceq}\isactrlsub r\ aa{\isachardoublequoteclose}\isanewline
\ \ \ \ \ \ \ \ \isacommand{using}\isamarkupfalse%
\ limit{\isacharunderscore}{\kern0pt}presv{\isacharunderscore}{\kern0pt}prefs{\isadigit{2}}\isanewline
\ \ \ \ \ \ \ \ \isacommand{by}\isamarkupfalse%
\ metis\isanewline
\ \ \ \ \ \ \isacommand{have}\isamarkupfalse%
\isanewline
\ \ \ \ \ \ \ \ {\isachardoublequoteopen}limited\ A\ {\isacharparenleft}{\kern0pt}limit\ A\ s{\isacharparenright}{\kern0pt}\ {\isasymand}\isanewline
\ \ \ \ \ \ \ \ \ \ {\isacharparenleft}{\kern0pt}{\isasymforall}a{\isachardot}{\kern0pt}\ a\ {\isasymnotin}\ A\ {\isasymor}\isanewline
\ \ \ \ \ \ \ \ \ \ \ \ {\isacharparenleft}{\kern0pt}{\isasymforall}aa{\isachardot}{\kern0pt}\ aa\ {\isasymnotin}\ A\ {\isasymor}\isanewline
\ \ \ \ \ \ \ \ \ \ \ \ \ \ {\isacharparenleft}{\kern0pt}let\ q\ {\isacharequal}{\kern0pt}\ limit\ A\ s\ in\ a\ {\isasympreceq}\isactrlsub q\ aa\ {\isasymor}\ aa\ {\isasympreceq}\isactrlsub q\ a{\isacharparenright}{\kern0pt}{\isacharparenright}{\kern0pt}{\isacharparenright}{\kern0pt}{\isachardoublequoteclose}\isanewline
\ \ \ \ \ \ \ \ \isacommand{using}\isamarkupfalse%
\ connex\ connex{\isacharunderscore}{\kern0pt}def\isanewline
\ \ \ \ \ \ \ \ \isacommand{by}\isamarkupfalse%
\ metis\isanewline
\ \ \ \ \ \ \isacommand{hence}\isamarkupfalse%
\ connex{\isadigit{2}}{\isacharcolon}{\kern0pt}\isanewline
\ \ \ \ \ \ \ \ {\isachardoublequoteopen}limited\ A\ {\isacharparenleft}{\kern0pt}limit\ A\ s{\isacharparenright}{\kern0pt}\ {\isasymand}\isanewline
\ \ \ \ \ \ \ \ \ \ {\isacharparenleft}{\kern0pt}{\isasymforall}a{\isachardot}{\kern0pt}\ a\ {\isasymnotin}\ A\ {\isasymor}\isanewline
\ \ \ \ \ \ \ \ \ \ \ \ {\isacharparenleft}{\kern0pt}{\isasymforall}aa{\isachardot}{\kern0pt}\ aa\ {\isasymnotin}\ A\ {\isasymor}\isanewline
\ \ \ \ \ \ \ \ \ \ \ \ \ \ {\isacharparenleft}{\kern0pt}{\isacharparenleft}{\kern0pt}a{\isacharcomma}{\kern0pt}\ aa{\isacharparenright}{\kern0pt}\ {\isasymin}\ limit\ A\ s\ {\isasymor}\ {\isacharparenleft}{\kern0pt}aa{\isacharcomma}{\kern0pt}\ a{\isacharparenright}{\kern0pt}\ {\isasymin}\ limit\ A\ s{\isacharparenright}{\kern0pt}{\isacharparenright}{\kern0pt}{\isacharparenright}{\kern0pt}{\isachardoublequoteclose}\isanewline
\ \ \ \ \ \ \ \ \isacommand{by}\isamarkupfalse%
\ simp\isanewline
\ \ \ \ \ \ \isacommand{ultimately}\isamarkupfalse%
\ \isacommand{have}\isamarkupfalse%
\isanewline
\ \ \ \ \ \ \ \ \ \ {\isachardoublequoteopen}{\isasymforall}x\ {\isasymin}\ A\ {\isacharminus}{\kern0pt}\ {\isacharbraceleft}{\kern0pt}a{\isacharbraceright}{\kern0pt}{\isachardot}{\kern0pt}\ {\isacharparenleft}{\kern0pt}a\ {\isasympreceq}\isactrlsub r\ x\ {\isasymand}\ a\ {\isasympreceq}\isactrlsub s\ x{\isacharparenright}{\kern0pt}\ {\isasymor}\ {\isacharparenleft}{\kern0pt}x\ {\isasympreceq}\isactrlsub r\ a\ {\isasymand}\ x\ {\isasympreceq}\isactrlsub s\ a{\isacharparenright}{\kern0pt}{\isachardoublequoteclose}\isanewline
\ \ \ \ \ \ \ \ \isacommand{using}\isamarkupfalse%
\ DiffD{\isadigit{1}}\ limit{\isadigit{1}}\ limit{\isacharunderscore}{\kern0pt}presv{\isacharunderscore}{\kern0pt}prefs{\isadigit{2}}\ a{\isadigit{1}}\isanewline
\ \ \ \ \ \ \ \ \isacommand{by}\isamarkupfalse%
\ metis\isanewline
\ \ \ \ \ \ \isacommand{hence}\isamarkupfalse%
\ r{\isacharunderscore}{\kern0pt}eq{\isacharunderscore}{\kern0pt}s{\isacharunderscore}{\kern0pt}on{\isacharunderscore}{\kern0pt}A{\isacharunderscore}{\kern0pt}{\isadigit{0}}{\isacharcolon}{\kern0pt}\isanewline
\ \ \ \ \ \ \ \ {\isachardoublequoteopen}{\isasymforall}x\ {\isasymin}\ A\ {\isacharminus}{\kern0pt}\ {\isacharbraceleft}{\kern0pt}a{\isacharbraceright}{\kern0pt}{\isachardot}{\kern0pt}\ {\isacharparenleft}{\kern0pt}{\isacharparenleft}{\kern0pt}a{\isacharcomma}{\kern0pt}\ x{\isacharparenright}{\kern0pt}\ {\isasymin}\ r\ {\isasymand}\ {\isacharparenleft}{\kern0pt}a{\isacharcomma}{\kern0pt}\ x{\isacharparenright}{\kern0pt}\ {\isasymin}\ s{\isacharparenright}{\kern0pt}\ {\isasymor}\ {\isacharparenleft}{\kern0pt}{\isacharparenleft}{\kern0pt}x{\isacharcomma}{\kern0pt}\ a{\isacharparenright}{\kern0pt}\ {\isasymin}\ r\ {\isasymand}\ {\isacharparenleft}{\kern0pt}x{\isacharcomma}{\kern0pt}\ a{\isacharparenright}{\kern0pt}\ {\isasymin}\ s{\isacharparenright}{\kern0pt}{\isachardoublequoteclose}\isanewline
\ \ \ \ \ \ \ \ \isacommand{by}\isamarkupfalse%
\ simp\isanewline
\ \ \ \ \ \ \isacommand{have}\isamarkupfalse%
\isanewline
\ \ \ \ \ \ \ \ {\isachardoublequoteopen}{\isasymforall}x\ {\isasymin}\ A\ {\isacharminus}{\kern0pt}\ {\isacharbraceleft}{\kern0pt}a{\isacharbraceright}{\kern0pt}{\isachardot}{\kern0pt}\ {\isacharparenleft}{\kern0pt}a{\isacharcomma}{\kern0pt}\ x{\isacharparenright}{\kern0pt}\ {\isasymin}\ {\isacharparenleft}{\kern0pt}limit\ A\ r{\isacharparenright}{\kern0pt}\ {\isasymlongleftrightarrow}\ {\isacharparenleft}{\kern0pt}a{\isacharcomma}{\kern0pt}\ x{\isacharparenright}{\kern0pt}\ {\isasymin}\ {\isacharparenleft}{\kern0pt}limit\ A\ s{\isacharparenright}{\kern0pt}{\isachardoublequoteclose}\isanewline
\ \ \ \ \ \ \ \ \isacommand{using}\isamarkupfalse%
\ DiffD{\isadigit{1}}\ limit{\isadigit{2}}\ limit{\isadigit{1}}\ connex{\isadigit{2}}\ a{\isadigit{1}}\ a{\isadigit{1}}{\isacharunderscore}{\kern0pt}{\isadigit{2}}\ not{\isacharunderscore}{\kern0pt}worse\isanewline
\ \ \ \ \ \ \ \ \isacommand{by}\isamarkupfalse%
\ metis\isanewline
\ \ \ \ \ \ \isacommand{hence}\isamarkupfalse%
\isanewline
\ \ \ \ \ \ \ \ {\isachardoublequoteopen}{\isasymforall}x\ {\isasymin}\ A\ {\isacharminus}{\kern0pt}\ {\isacharbraceleft}{\kern0pt}a{\isacharbraceright}{\kern0pt}{\isachardot}{\kern0pt}\isanewline
\ \ \ \ \ \ \ \ \ \ {\isacharparenleft}{\kern0pt}let\ q\ {\isacharequal}{\kern0pt}\ limit\ A\ r\ in\ a\ {\isasympreceq}\isactrlsub q\ x{\isacharparenright}{\kern0pt}\ {\isasymlongleftrightarrow}\ {\isacharparenleft}{\kern0pt}let\ q\ {\isacharequal}{\kern0pt}\ limit\ A\ s\ in\ a\ {\isasympreceq}\isactrlsub q\ x{\isacharparenright}{\kern0pt}{\isachardoublequoteclose}\isanewline
\ \ \ \ \ \ \ \ \isacommand{by}\isamarkupfalse%
\ simp\isanewline
\ \ \ \ \ \ \isacommand{moreover}\isamarkupfalse%
\ \isacommand{have}\isamarkupfalse%
\isanewline
\ \ \ \ \ \ \ \ {\isachardoublequoteopen}{\isasymforall}x\ {\isasymin}\ A\ {\isacharminus}{\kern0pt}\ {\isacharbraceleft}{\kern0pt}a{\isacharbraceright}{\kern0pt}{\isachardot}{\kern0pt}\ {\isacharparenleft}{\kern0pt}x{\isacharcomma}{\kern0pt}\ a{\isacharparenright}{\kern0pt}\ {\isasymin}\ {\isacharparenleft}{\kern0pt}limit\ A\ r{\isacharparenright}{\kern0pt}\ {\isasymlongleftrightarrow}\ {\isacharparenleft}{\kern0pt}x{\isacharcomma}{\kern0pt}\ a{\isacharparenright}{\kern0pt}\ {\isasymin}\ {\isacharparenleft}{\kern0pt}limit\ A\ s{\isacharparenright}{\kern0pt}{\isachardoublequoteclose}\isanewline
\ \ \ \ \ \ \ \ \isacommand{using}\isamarkupfalse%
\ a{\isadigit{1}}\ a{\isadigit{1}}{\isacharunderscore}{\kern0pt}{\isadigit{2}}\ not{\isacharunderscore}{\kern0pt}worse\ DiffD{\isadigit{1}}\ limit{\isacharunderscore}{\kern0pt}presv{\isacharunderscore}{\kern0pt}prefs{\isadigit{2}}\ connex{\isadigit{2}}\ limit{\isadigit{1}}\isanewline
\ \ \ \ \ \ \ \ \isacommand{by}\isamarkupfalse%
\ metis\isanewline
\ \ \ \ \ \ \isacommand{moreover}\isamarkupfalse%
\ \isacommand{have}\isamarkupfalse%
\isanewline
\ \ \ \ \ \ \ \ {\isachardoublequoteopen}{\isacharparenleft}{\kern0pt}a{\isacharcomma}{\kern0pt}\ a{\isacharparenright}{\kern0pt}\ {\isasymin}\ {\isacharparenleft}{\kern0pt}limit\ A\ r{\isacharparenright}{\kern0pt}\ {\isasymand}\ {\isacharparenleft}{\kern0pt}a{\isacharcomma}{\kern0pt}\ a{\isacharparenright}{\kern0pt}\ {\isasymin}\ {\isacharparenleft}{\kern0pt}limit\ A\ s{\isacharparenright}{\kern0pt}{\isachardoublequoteclose}\isanewline
\ \ \ \ \ \ \ \ \isacommand{using}\isamarkupfalse%
\ a{\isadigit{1}}\ connex\ connex{\isacharunderscore}{\kern0pt}imp{\isacharunderscore}{\kern0pt}refl\ refl{\isacharunderscore}{\kern0pt}onD\isanewline
\ \ \ \ \ \ \ \ \isacommand{by}\isamarkupfalse%
\ metis\isanewline
\ \ \ \ \ \ \isacommand{moreover}\isamarkupfalse%
\ \isacommand{have}\isamarkupfalse%
\isanewline
\ \ \ \ \ \ \ \ {\isachardoublequoteopen}limited\ A\ {\isacharparenleft}{\kern0pt}limit\ A\ r{\isacharparenright}{\kern0pt}\ {\isasymand}\ limited\ A\ {\isacharparenleft}{\kern0pt}limit\ A\ s{\isacharparenright}{\kern0pt}{\isachardoublequoteclose}\isanewline
\ \ \ \ \ \ \ \ \isacommand{using}\isamarkupfalse%
\ limit{\isacharunderscore}{\kern0pt}to{\isacharunderscore}{\kern0pt}limits\isanewline
\ \ \ \ \ \ \ \ \isacommand{by}\isamarkupfalse%
\ metis\isanewline
\ \ \ \ \ \ \isacommand{ultimately}\isamarkupfalse%
\ \isacommand{have}\isamarkupfalse%
\isanewline
\ \ \ \ \ \ \ \ {\isachardoublequoteopen}{\isasymforall}x\ y{\isachardot}{\kern0pt}\ {\isacharparenleft}{\kern0pt}x{\isacharcomma}{\kern0pt}\ y{\isacharparenright}{\kern0pt}\ {\isasymin}\ limit\ A\ r\ {\isasymlongleftrightarrow}\ {\isacharparenleft}{\kern0pt}x{\isacharcomma}{\kern0pt}\ y{\isacharparenright}{\kern0pt}\ {\isasymin}\ limit\ A\ s{\isachardoublequoteclose}\isanewline
\ \ \ \ \ \ \ \ \isacommand{using}\isamarkupfalse%
\ eql{\isacharunderscore}{\kern0pt}rs\isanewline
\ \ \ \ \ \ \ \ \isacommand{by}\isamarkupfalse%
\ auto\isanewline
\ \ \ \ \ \ \isacommand{thus}\isamarkupfalse%
\ {\isacharquery}{\kern0pt}thesis\isanewline
\ \ \ \ \ \ \ \ \isacommand{by}\isamarkupfalse%
\ simp\isanewline
\ \ \ \ \isacommand{qed}\isamarkupfalse%
\isanewline
\ \ \isacommand{next}\isamarkupfalse%
\isanewline
\ \ \ \ \isacommand{assume}\isamarkupfalse%
\ a{\isadigit{2}}{\isacharcolon}{\kern0pt}\ {\isachardoublequoteopen}a\ {\isasymnotin}\ A{\isachardoublequoteclose}\isanewline
\ \ \ \ \isacommand{with}\isamarkupfalse%
\ eql{\isacharunderscore}{\kern0pt}rs\ \isacommand{have}\isamarkupfalse%
\isanewline
\ \ \ \ \ \ {\isachardoublequoteopen}{\isasymforall}x\ {\isasymin}\ A{\isachardot}{\kern0pt}\ {\isasymforall}y\ {\isasymin}\ A{\isachardot}{\kern0pt}\ {\isacharparenleft}{\kern0pt}x{\isacharcomma}{\kern0pt}\ y{\isacharparenright}{\kern0pt}\ {\isasymin}\ {\isacharparenleft}{\kern0pt}limit\ A\ r{\isacharparenright}{\kern0pt}\ {\isasymlongleftrightarrow}\ {\isacharparenleft}{\kern0pt}x{\isacharcomma}{\kern0pt}\ y{\isacharparenright}{\kern0pt}\ {\isasymin}\ {\isacharparenleft}{\kern0pt}limit\ A\ s{\isacharparenright}{\kern0pt}{\isachardoublequoteclose}\isanewline
\ \ \ \ \ \ \isacommand{by}\isamarkupfalse%
\ simp\isanewline
\ \ \ \ \isacommand{thus}\isamarkupfalse%
\ {\isacharquery}{\kern0pt}thesis\isanewline
\ \ \ \ \ \ \isacommand{using}\isamarkupfalse%
\ limit{\isacharunderscore}{\kern0pt}to{\isacharunderscore}{\kern0pt}limits\ limited{\isacharunderscore}{\kern0pt}dest\ subrelI\ subset{\isacharunderscore}{\kern0pt}antisym\isanewline
\ \ \ \ \ \ \isacommand{by}\isamarkupfalse%
\ auto\isanewline
\ \ \isacommand{qed}\isamarkupfalse%
\isanewline
\isacommand{qed}\isamarkupfalse%
%
\endisatagproof
{\isafoldproof}%
%
\isadelimproof
\isanewline
%
\endisadelimproof
\isanewline
\isacommand{lemma}\isamarkupfalse%
\ negl{\isacharunderscore}{\kern0pt}diff{\isacharunderscore}{\kern0pt}imp{\isacharunderscore}{\kern0pt}eq{\isacharunderscore}{\kern0pt}limit{\isacharcolon}{\kern0pt}\isanewline
\ \ \isakeyword{assumes}\isanewline
\ \ \ \ change{\isacharcolon}{\kern0pt}\ {\isachardoublequoteopen}equiv{\isacharunderscore}{\kern0pt}rel{\isacharunderscore}{\kern0pt}except{\isacharunderscore}{\kern0pt}a\ S\ r\ s\ a{\isachardoublequoteclose}\ \isakeyword{and}\isanewline
\ \ \ \ subset{\isacharcolon}{\kern0pt}\ {\isachardoublequoteopen}A\ {\isasymsubseteq}\ S{\isachardoublequoteclose}\ \isakeyword{and}\isanewline
\ \ \ \ notInA{\isacharcolon}{\kern0pt}\ {\isachardoublequoteopen}a\ {\isasymnotin}\ A{\isachardoublequoteclose}\isanewline
\ \ \isakeyword{shows}\ {\isachardoublequoteopen}limit\ A\ r\ {\isacharequal}{\kern0pt}\ limit\ A\ s{\isachardoublequoteclose}\isanewline
%
\isadelimproof
%
\endisadelimproof
%
\isatagproof
\isacommand{proof}\isamarkupfalse%
\ {\isacharminus}{\kern0pt}\isanewline
\ \ \isacommand{have}\isamarkupfalse%
\ {\isachardoublequoteopen}A\ {\isasymsubseteq}\ S{\isacharminus}{\kern0pt}{\isacharbraceleft}{\kern0pt}a{\isacharbraceright}{\kern0pt}{\isachardoublequoteclose}\isanewline
\ \ \ \ \isacommand{by}\isamarkupfalse%
\ {\isacharparenleft}{\kern0pt}simp\ add{\isacharcolon}{\kern0pt}\ notInA\ subset\ subset{\isacharunderscore}{\kern0pt}Diff{\isacharunderscore}{\kern0pt}insert{\isacharparenright}{\kern0pt}\isanewline
\ \ \isacommand{hence}\isamarkupfalse%
\ {\isachardoublequoteopen}{\isasymforall}x\ {\isasymin}\ A{\isachardot}{\kern0pt}\ {\isasymforall}y\ {\isasymin}\ A{\isachardot}{\kern0pt}\ x\ {\isasympreceq}\isactrlsub r\ y\ {\isasymlongleftrightarrow}\ x\ {\isasympreceq}\isactrlsub s\ y{\isachardoublequoteclose}\isanewline
\ \ \ \ \isacommand{by}\isamarkupfalse%
\ {\isacharparenleft}{\kern0pt}meson\ change\ equiv{\isacharunderscore}{\kern0pt}rel{\isacharunderscore}{\kern0pt}except{\isacharunderscore}{\kern0pt}a{\isacharunderscore}{\kern0pt}def\ in{\isacharunderscore}{\kern0pt}mono{\isacharparenright}{\kern0pt}\isanewline
\ \ \isacommand{thus}\isamarkupfalse%
\ {\isacharquery}{\kern0pt}thesis\isanewline
\ \ \ \ \isacommand{by}\isamarkupfalse%
\ auto\isanewline
\isacommand{qed}\isamarkupfalse%
%
\endisatagproof
{\isafoldproof}%
%
\isadelimproof
\isanewline
%
\endisadelimproof
\isanewline
\isacommand{theorem}\isamarkupfalse%
\ lifted{\isacharunderscore}{\kern0pt}above{\isacharunderscore}{\kern0pt}winner{\isacharcolon}{\kern0pt}\isanewline
\ \ \isakeyword{assumes}\isanewline
\ \ \ \ lifted{\isacharunderscore}{\kern0pt}a{\isacharcolon}{\kern0pt}\ {\isachardoublequoteopen}lifted\ A\ r\ s\ a{\isachardoublequoteclose}\ \isakeyword{and}\isanewline
\ \ \ \ above{\isacharunderscore}{\kern0pt}x{\isacharcolon}{\kern0pt}\ {\isachardoublequoteopen}above\ r\ x\ {\isacharequal}{\kern0pt}\ {\isacharbraceleft}{\kern0pt}x{\isacharbraceright}{\kern0pt}{\isachardoublequoteclose}\ \isakeyword{and}\isanewline
\ \ \ \ fin{\isacharunderscore}{\kern0pt}A{\isacharcolon}{\kern0pt}\ {\isachardoublequoteopen}finite\ A{\isachardoublequoteclose}\isanewline
\ \ \isakeyword{shows}\ {\isachardoublequoteopen}above\ s\ x\ {\isacharequal}{\kern0pt}\ {\isacharbraceleft}{\kern0pt}x{\isacharbraceright}{\kern0pt}\ {\isasymor}\ above\ s\ a\ {\isacharequal}{\kern0pt}\ {\isacharbraceleft}{\kern0pt}a{\isacharbraceright}{\kern0pt}{\isachardoublequoteclose}\isanewline
%
\isadelimproof
%
\endisadelimproof
%
\isatagproof
\isacommand{proof}\isamarkupfalse%
\ cases\isanewline
\ \ \isacommand{assume}\isamarkupfalse%
\ {\isachardoublequoteopen}x\ {\isacharequal}{\kern0pt}\ a{\isachardoublequoteclose}\isanewline
\ \ \isacommand{thus}\isamarkupfalse%
\ {\isacharquery}{\kern0pt}thesis\isanewline
\ \ \ \ \isacommand{using}\isamarkupfalse%
\ above{\isacharunderscore}{\kern0pt}subset{\isacharunderscore}{\kern0pt}geq{\isacharunderscore}{\kern0pt}one\ lifted{\isacharunderscore}{\kern0pt}a\ above{\isacharunderscore}{\kern0pt}x\isanewline
\ \ \ \ \ \ \ \ \ \ lifted{\isacharunderscore}{\kern0pt}above\ lifted{\isacharunderscore}{\kern0pt}def\ equiv{\isacharunderscore}{\kern0pt}rel{\isacharunderscore}{\kern0pt}except{\isacharunderscore}{\kern0pt}a{\isacharunderscore}{\kern0pt}def\isanewline
\ \ \ \ \isacommand{by}\isamarkupfalse%
\ metis\isanewline
\isacommand{next}\isamarkupfalse%
\isanewline
\ \ \isacommand{assume}\isamarkupfalse%
\ asm{\isadigit{1}}{\isacharcolon}{\kern0pt}\ {\isachardoublequoteopen}x\ {\isasymnoteq}\ a{\isachardoublequoteclose}\isanewline
\ \ \isacommand{thus}\isamarkupfalse%
\ {\isacharquery}{\kern0pt}thesis\isanewline
\ \ \isacommand{proof}\isamarkupfalse%
\ cases\isanewline
\ \ \ \ \isacommand{assume}\isamarkupfalse%
\ {\isachardoublequoteopen}above\ s\ x\ {\isacharequal}{\kern0pt}\ {\isacharbraceleft}{\kern0pt}x{\isacharbraceright}{\kern0pt}{\isachardoublequoteclose}\isanewline
\ \ \ \ \isacommand{thus}\isamarkupfalse%
\ {\isacharquery}{\kern0pt}thesis\isanewline
\ \ \ \ \ \ \isacommand{by}\isamarkupfalse%
\ simp\isanewline
\ \ \isacommand{next}\isamarkupfalse%
\isanewline
\ \ \ \ \isacommand{assume}\isamarkupfalse%
\ asm{\isadigit{2}}{\isacharcolon}{\kern0pt}\ {\isachardoublequoteopen}above\ s\ x\ {\isasymnoteq}\ {\isacharbraceleft}{\kern0pt}x{\isacharbraceright}{\kern0pt}{\isachardoublequoteclose}\ \isanewline
\ \ \ \ \isacommand{have}\isamarkupfalse%
\ {\isachardoublequoteopen}{\isasymforall}y\ {\isasymin}\ A{\isachardot}{\kern0pt}\ y\ {\isasympreceq}\isactrlsub r\ x{\isachardoublequoteclose}\isanewline
\ \ \ \ \isacommand{proof}\isamarkupfalse%
\ {\isacharminus}{\kern0pt}\isanewline
\ \ \ \ \ \ \isacommand{fix}\isamarkupfalse%
\ aa\ {\isacharcolon}{\kern0pt}{\isacharcolon}{\kern0pt}\ {\isacharprime}{\kern0pt}a\isanewline
\ \ \ \ \ \ \isacommand{have}\isamarkupfalse%
\ imp{\isacharunderscore}{\kern0pt}a{\isacharcolon}{\kern0pt}\ {\isachardoublequoteopen}x\ {\isasympreceq}\isactrlsub r\ aa\ {\isasymlongrightarrow}\ aa\ {\isasymnotin}\ A\ {\isasymor}\ aa\ {\isasympreceq}\isactrlsub r\ x{\isachardoublequoteclose}\isanewline
\ \ \ \ \ \ \ \ \isacommand{using}\isamarkupfalse%
\ singletonD\ pref{\isacharunderscore}{\kern0pt}imp{\isacharunderscore}{\kern0pt}in{\isacharunderscore}{\kern0pt}above\ above{\isacharunderscore}{\kern0pt}x\isanewline
\ \ \ \ \ \ \ \ \isacommand{by}\isamarkupfalse%
\ metis\isanewline
\ \ \ \ \ \ \isacommand{also}\isamarkupfalse%
\ \isacommand{have}\isamarkupfalse%
\ f{\isadigit{1}}{\isacharcolon}{\kern0pt}\isanewline
\ \ \ \ \ \ \ \ {\isachardoublequoteopen}{\isasymforall}A\ r{\isachardot}{\kern0pt}\isanewline
\ \ \ \ \ \ \ \ \ \ {\isacharparenleft}{\kern0pt}connex\ A\ r\ {\isasymor}\isanewline
\ \ \ \ \ \ \ \ \ \ \ \ {\isacharparenleft}{\kern0pt}{\isasymexists}a{\isachardot}{\kern0pt}\ {\isacharparenleft}{\kern0pt}{\isasymexists}aa{\isachardot}{\kern0pt}\ {\isasymnot}\ {\isacharparenleft}{\kern0pt}aa{\isacharcolon}{\kern0pt}{\isacharcolon}{\kern0pt}{\isacharprime}{\kern0pt}a{\isacharparenright}{\kern0pt}\ {\isasympreceq}\isactrlsub r\ a\ {\isasymand}\ {\isasymnot}\ a\ {\isasympreceq}\isactrlsub r\ aa\ {\isasymand}\ aa\ {\isasymin}\ A{\isacharparenright}{\kern0pt}\ {\isasymand}\ a\ {\isasymin}\ A{\isacharparenright}{\kern0pt}\ {\isasymor}\isanewline
\ \ \ \ \ \ \ \ \ \ \ \ \ \ {\isasymnot}\ limited\ A\ r{\isacharparenright}{\kern0pt}\ {\isasymand}\isanewline
\ \ \ \ \ \ \ \ \ \ \ \ {\isacharparenleft}{\kern0pt}{\isacharparenleft}{\kern0pt}{\isasymforall}a{\isachardot}{\kern0pt}\ {\isacharparenleft}{\kern0pt}{\isasymforall}aa{\isachardot}{\kern0pt}\ aa\ {\isasympreceq}\isactrlsub r\ a\ {\isasymor}\ a\ {\isasympreceq}\isactrlsub r\ aa\ {\isasymor}\ aa\ {\isasymnotin}\ A{\isacharparenright}{\kern0pt}\ {\isasymor}\ a\ {\isasymnotin}\ A{\isacharparenright}{\kern0pt}\ {\isasymand}\ limited\ A\ r\ {\isasymor}\isanewline
\ \ \ \ \ \ \ \ \ \ \ \ \ \ {\isasymnot}\ connex\ A\ r{\isacharparenright}{\kern0pt}{\isachardoublequoteclose}\isanewline
\ \ \ \ \ \ \ \ \isacommand{using}\isamarkupfalse%
\ connex{\isacharunderscore}{\kern0pt}def\isanewline
\ \ \ \ \ \ \ \ \isacommand{by}\isamarkupfalse%
\ metis\isanewline
\ \ \ \ \ \ \isacommand{moreover}\isamarkupfalse%
\ \isacommand{have}\isamarkupfalse%
\ eq{\isacharunderscore}{\kern0pt}exc{\isacharunderscore}{\kern0pt}a{\isacharcolon}{\kern0pt}\isanewline
\ \ \ \ \ \ \ \ {\isachardoublequoteopen}equiv{\isacharunderscore}{\kern0pt}rel{\isacharunderscore}{\kern0pt}except{\isacharunderscore}{\kern0pt}a\ A\ r\ s\ a{\isachardoublequoteclose}\isanewline
\ \ \ \ \ \ \ \ \isacommand{using}\isamarkupfalse%
\ lifted{\isacharunderscore}{\kern0pt}def\ lifted{\isacharunderscore}{\kern0pt}a\isanewline
\ \ \ \ \ \ \ \ \isacommand{by}\isamarkupfalse%
\ metis\isanewline
\ \ \ \ \ \ \isacommand{ultimately}\isamarkupfalse%
\ \isacommand{have}\isamarkupfalse%
\ {\isachardoublequoteopen}aa\ {\isasymnotin}\ A\ {\isasymor}\ aa\ {\isasympreceq}\isactrlsub r\ x{\isachardoublequoteclose}\isanewline
\ \ \ \ \ \ \ \ \isacommand{using}\isamarkupfalse%
\ pref{\isacharunderscore}{\kern0pt}imp{\isacharunderscore}{\kern0pt}in{\isacharunderscore}{\kern0pt}above\ above{\isacharunderscore}{\kern0pt}x\ equiv{\isacharunderscore}{\kern0pt}rel{\isacharunderscore}{\kern0pt}except{\isacharunderscore}{\kern0pt}a{\isacharunderscore}{\kern0pt}def\isanewline
\ \ \ \ \ \ \ \ \ \ \ \ \ \ lin{\isacharunderscore}{\kern0pt}ord{\isacharunderscore}{\kern0pt}imp{\isacharunderscore}{\kern0pt}connex\ limited{\isacharunderscore}{\kern0pt}dest\ insertCI\isanewline
\ \ \ \ \ \ \ \ \isacommand{by}\isamarkupfalse%
\ metis\isanewline
\ \ \ \ \ \ \isacommand{thus}\isamarkupfalse%
\ {\isacharquery}{\kern0pt}thesis\isanewline
\ \ \ \ \ \ \ \ \isacommand{using}\isamarkupfalse%
\ f{\isadigit{1}}\ eq{\isacharunderscore}{\kern0pt}exc{\isacharunderscore}{\kern0pt}a\ above{\isacharunderscore}{\kern0pt}one\ above{\isacharunderscore}{\kern0pt}one{\isadigit{2}}\ above{\isacharunderscore}{\kern0pt}x\ fin{\isacharunderscore}{\kern0pt}A\isanewline
\ \ \ \ \ \ \ \ \ \ \ \ \ \ equiv{\isacharunderscore}{\kern0pt}rel{\isacharunderscore}{\kern0pt}except{\isacharunderscore}{\kern0pt}a{\isacharunderscore}{\kern0pt}def\ insert{\isacharunderscore}{\kern0pt}not{\isacharunderscore}{\kern0pt}empty\ pref{\isacharunderscore}{\kern0pt}imp{\isacharunderscore}{\kern0pt}in{\isacharunderscore}{\kern0pt}above\isanewline
\ \ \ \ \ \ \ \ \ \ \ \ \ \ lin{\isacharunderscore}{\kern0pt}ord{\isacharunderscore}{\kern0pt}imp{\isacharunderscore}{\kern0pt}connex\ mk{\isacharunderscore}{\kern0pt}disjoint{\isacharunderscore}{\kern0pt}insert\ insertE\isanewline
\ \ \ \ \ \ \ \ \isacommand{by}\isamarkupfalse%
\ metis\isanewline
\ \ \ \ \isacommand{qed}\isamarkupfalse%
\isanewline
\ \ \ \ \isacommand{moreover}\isamarkupfalse%
\ \isacommand{have}\isamarkupfalse%
\ {\isachardoublequoteopen}equiv{\isacharunderscore}{\kern0pt}rel{\isacharunderscore}{\kern0pt}except{\isacharunderscore}{\kern0pt}a\ A\ r\ s\ a{\isachardoublequoteclose}\isanewline
\ \ \ \ \ \ \isacommand{using}\isamarkupfalse%
\ lifted{\isacharunderscore}{\kern0pt}a\ lifted{\isacharunderscore}{\kern0pt}def\isanewline
\ \ \ \ \ \ \isacommand{by}\isamarkupfalse%
\ metis\isanewline
\ \ \ \ \isacommand{moreover}\isamarkupfalse%
\ \isacommand{have}\isamarkupfalse%
\ {\isachardoublequoteopen}x\ {\isasymin}\ A{\isacharminus}{\kern0pt}{\isacharbraceleft}{\kern0pt}a{\isacharbraceright}{\kern0pt}{\isachardoublequoteclose}\isanewline
\ \ \ \ \ \ \isacommand{using}\isamarkupfalse%
\ above{\isacharunderscore}{\kern0pt}one\ above{\isacharunderscore}{\kern0pt}one{\isadigit{2}}\ asm{\isadigit{1}}\ assms\ calculation\isanewline
\ \ \ \ \ \ \ \ \ \ \ \ equiv{\isacharunderscore}{\kern0pt}rel{\isacharunderscore}{\kern0pt}except{\isacharunderscore}{\kern0pt}a{\isacharunderscore}{\kern0pt}def\ insert{\isacharunderscore}{\kern0pt}not{\isacharunderscore}{\kern0pt}empty\isanewline
\ \ \ \ \ \ \ \ \ \ \ \ member{\isacharunderscore}{\kern0pt}remove\ remove{\isacharunderscore}{\kern0pt}def\ insert{\isacharunderscore}{\kern0pt}absorb\isanewline
\ \ \ \ \ \ \isacommand{by}\isamarkupfalse%
\ metis\isanewline
\ \ \ \ \isacommand{ultimately}\isamarkupfalse%
\ \isacommand{have}\isamarkupfalse%
\ {\isachardoublequoteopen}{\isasymforall}y\ {\isasymin}\ A{\isacharminus}{\kern0pt}{\isacharbraceleft}{\kern0pt}a{\isacharbraceright}{\kern0pt}{\isachardot}{\kern0pt}\ y\ {\isasympreceq}\isactrlsub s\ x{\isachardoublequoteclose}\isanewline
\ \ \ \ \ \ \isacommand{using}\isamarkupfalse%
\ DiffD{\isadigit{1}}\ lifted{\isacharunderscore}{\kern0pt}a\ equiv{\isacharunderscore}{\kern0pt}rel{\isacharunderscore}{\kern0pt}except{\isacharunderscore}{\kern0pt}a{\isacharunderscore}{\kern0pt}def\isanewline
\ \ \ \ \ \ \isacommand{by}\isamarkupfalse%
\ metis\isanewline
\ \ \ \ \isacommand{hence}\isamarkupfalse%
\ not{\isacharunderscore}{\kern0pt}others{\isacharcolon}{\kern0pt}\ {\isachardoublequoteopen}{\isasymforall}y\ {\isasymin}\ A{\isacharminus}{\kern0pt}{\isacharbraceleft}{\kern0pt}a{\isacharbraceright}{\kern0pt}{\isachardot}{\kern0pt}\ above\ s\ y\ {\isasymnoteq}\ {\isacharbraceleft}{\kern0pt}y{\isacharbraceright}{\kern0pt}{\isachardoublequoteclose}\isanewline
\ \ \ \ \ \ \isacommand{using}\isamarkupfalse%
\ asm{\isadigit{2}}\ empty{\isacharunderscore}{\kern0pt}iff\ insert{\isacharunderscore}{\kern0pt}iff\ pref{\isacharunderscore}{\kern0pt}imp{\isacharunderscore}{\kern0pt}in{\isacharunderscore}{\kern0pt}above\isanewline
\ \ \ \ \ \ \isacommand{by}\isamarkupfalse%
\ metis\isanewline
\ \ \ \ \isacommand{hence}\isamarkupfalse%
\ {\isachardoublequoteopen}above\ s\ a\ {\isacharequal}{\kern0pt}\ {\isacharbraceleft}{\kern0pt}a{\isacharbraceright}{\kern0pt}{\isachardoublequoteclose}\isanewline
\ \ \ \ \ \ \isacommand{using}\isamarkupfalse%
\ Diff{\isacharunderscore}{\kern0pt}iff\ all{\isacharunderscore}{\kern0pt}not{\isacharunderscore}{\kern0pt}in{\isacharunderscore}{\kern0pt}conv\ lifted{\isacharunderscore}{\kern0pt}a\ fin{\isacharunderscore}{\kern0pt}A\ lifted{\isacharunderscore}{\kern0pt}def\isanewline
\ \ \ \ \ \ \ \ \ \ \ \ equiv{\isacharunderscore}{\kern0pt}rel{\isacharunderscore}{\kern0pt}except{\isacharunderscore}{\kern0pt}a{\isacharunderscore}{\kern0pt}def\ above{\isacharunderscore}{\kern0pt}one\ singleton{\isacharunderscore}{\kern0pt}iff\isanewline
\ \ \ \ \ \ \isacommand{by}\isamarkupfalse%
\ metis\isanewline
\ \ \ \ \isacommand{thus}\isamarkupfalse%
\ {\isacharquery}{\kern0pt}thesis\isanewline
\ \ \ \ \ \ \isacommand{by}\isamarkupfalse%
\ simp\isanewline
\ \ \isacommand{qed}\isamarkupfalse%
\isanewline
\isacommand{qed}\isamarkupfalse%
%
\endisatagproof
{\isafoldproof}%
%
\isadelimproof
\isanewline
%
\endisadelimproof
\isanewline
\isacommand{theorem}\isamarkupfalse%
\ lifted{\isacharunderscore}{\kern0pt}above{\isacharunderscore}{\kern0pt}winner{\isadigit{2}}{\isacharcolon}{\kern0pt}\isanewline
\ \ \isakeyword{assumes}\isanewline
\ \ \ \ {\isachardoublequoteopen}lifted\ A\ r\ s\ a{\isachardoublequoteclose}\ \isakeyword{and}\isanewline
\ \ \ \ {\isachardoublequoteopen}above\ r\ a\ {\isacharequal}{\kern0pt}\ {\isacharbraceleft}{\kern0pt}a{\isacharbraceright}{\kern0pt}{\isachardoublequoteclose}\ \isakeyword{and}\isanewline
\ \ \ \ {\isachardoublequoteopen}finite\ A{\isachardoublequoteclose}\isanewline
\ \ \isakeyword{shows}\ {\isachardoublequoteopen}above\ s\ a\ {\isacharequal}{\kern0pt}\ {\isacharbraceleft}{\kern0pt}a{\isacharbraceright}{\kern0pt}{\isachardoublequoteclose}\isanewline
%
\isadelimproof
\ \ %
\endisadelimproof
%
\isatagproof
\isacommand{using}\isamarkupfalse%
\ assms\ lifted{\isacharunderscore}{\kern0pt}above{\isacharunderscore}{\kern0pt}winner\isanewline
\ \ \isacommand{by}\isamarkupfalse%
\ metis%
\endisatagproof
{\isafoldproof}%
%
\isadelimproof
\isanewline
%
\endisadelimproof
\isanewline
\isacommand{theorem}\isamarkupfalse%
\ lifted{\isacharunderscore}{\kern0pt}above{\isacharunderscore}{\kern0pt}winner{\isadigit{3}}{\isacharcolon}{\kern0pt}\isanewline
\ \ \isakeyword{assumes}\isanewline
\ \ \ \ lifted{\isacharunderscore}{\kern0pt}a{\isacharcolon}{\kern0pt}\ {\isachardoublequoteopen}lifted\ A\ r\ s\ a{\isachardoublequoteclose}\ \isakeyword{and}\isanewline
\ \ \ \ above{\isacharunderscore}{\kern0pt}x{\isacharcolon}{\kern0pt}\ {\isachardoublequoteopen}above\ s\ x\ {\isacharequal}{\kern0pt}\ {\isacharbraceleft}{\kern0pt}x{\isacharbraceright}{\kern0pt}{\isachardoublequoteclose}\ \isakeyword{and}\isanewline
\ \ \ \ fin{\isacharunderscore}{\kern0pt}A{\isacharcolon}{\kern0pt}\ {\isachardoublequoteopen}finite\ A{\isachardoublequoteclose}\ \isakeyword{and}\isanewline
\ \ \ \ x{\isacharunderscore}{\kern0pt}not{\isacharunderscore}{\kern0pt}a{\isacharcolon}{\kern0pt}\ {\isachardoublequoteopen}x\ {\isasymnoteq}\ a{\isachardoublequoteclose}\isanewline
\ \ \isakeyword{shows}\ {\isachardoublequoteopen}above\ r\ x\ {\isacharequal}{\kern0pt}\ {\isacharbraceleft}{\kern0pt}x{\isacharbraceright}{\kern0pt}{\isachardoublequoteclose}\isanewline
%
\isadelimproof
%
\endisadelimproof
%
\isatagproof
\isacommand{proof}\isamarkupfalse%
\ {\isacharparenleft}{\kern0pt}rule\ ccontr{\isacharparenright}{\kern0pt}\isanewline
\ \ \isacommand{assume}\isamarkupfalse%
\ asm{\isacharcolon}{\kern0pt}\ {\isachardoublequoteopen}above\ r\ x\ {\isasymnoteq}\ {\isacharbraceleft}{\kern0pt}x{\isacharbraceright}{\kern0pt}{\isachardoublequoteclose}\isanewline
\ \ \isacommand{then}\isamarkupfalse%
\ \isacommand{obtain}\isamarkupfalse%
\ y\ \isakeyword{where}\ y{\isacharcolon}{\kern0pt}\ {\isachardoublequoteopen}above\ r\ y\ {\isacharequal}{\kern0pt}\ {\isacharbraceleft}{\kern0pt}y{\isacharbraceright}{\kern0pt}{\isachardoublequoteclose}\isanewline
\ \ \ \ \isacommand{using}\isamarkupfalse%
\ lifted{\isacharunderscore}{\kern0pt}a\ fin{\isacharunderscore}{\kern0pt}A\ insert{\isacharunderscore}{\kern0pt}Diff\ insert{\isacharunderscore}{\kern0pt}not{\isacharunderscore}{\kern0pt}empty\isanewline
\ \ \ \ \ \ \ \ \ \ lifted{\isacharunderscore}{\kern0pt}def\ equiv{\isacharunderscore}{\kern0pt}rel{\isacharunderscore}{\kern0pt}except{\isacharunderscore}{\kern0pt}a{\isacharunderscore}{\kern0pt}def\ above{\isacharunderscore}{\kern0pt}one\isanewline
\ \ \ \ \isacommand{by}\isamarkupfalse%
\ metis\isanewline
\ \ \isacommand{hence}\isamarkupfalse%
\ {\isachardoublequoteopen}above\ s\ y\ {\isacharequal}{\kern0pt}\ {\isacharbraceleft}{\kern0pt}y{\isacharbraceright}{\kern0pt}\ {\isasymor}\ above\ s\ a\ {\isacharequal}{\kern0pt}\ {\isacharbraceleft}{\kern0pt}a{\isacharbraceright}{\kern0pt}{\isachardoublequoteclose}\isanewline
\ \ \ \ \isacommand{using}\isamarkupfalse%
\ lifted{\isacharunderscore}{\kern0pt}a\ fin{\isacharunderscore}{\kern0pt}A\ lifted{\isacharunderscore}{\kern0pt}above{\isacharunderscore}{\kern0pt}winner\isanewline
\ \ \ \ \isacommand{by}\isamarkupfalse%
\ metis\isanewline
\ \ \isacommand{moreover}\isamarkupfalse%
\ \isacommand{have}\isamarkupfalse%
\ {\isachardoublequoteopen}{\isasymforall}b{\isachardot}{\kern0pt}\ above\ s\ b\ {\isacharequal}{\kern0pt}\ {\isacharbraceleft}{\kern0pt}b{\isacharbraceright}{\kern0pt}\ {\isasymlongrightarrow}\ b\ {\isacharequal}{\kern0pt}\ x{\isachardoublequoteclose}\isanewline
\ \ \ \ \isacommand{using}\isamarkupfalse%
\ all{\isacharunderscore}{\kern0pt}not{\isacharunderscore}{\kern0pt}in{\isacharunderscore}{\kern0pt}conv\ lifted{\isacharunderscore}{\kern0pt}a\ above{\isacharunderscore}{\kern0pt}x\ lifted{\isacharunderscore}{\kern0pt}def\isanewline
\ \ \ \ \ \ \ \ \ \ fin{\isacharunderscore}{\kern0pt}A\ equiv{\isacharunderscore}{\kern0pt}rel{\isacharunderscore}{\kern0pt}except{\isacharunderscore}{\kern0pt}a{\isacharunderscore}{\kern0pt}def\ above{\isacharunderscore}{\kern0pt}one{\isadigit{2}}\isanewline
\ \ \ \ \isacommand{by}\isamarkupfalse%
\ metis\isanewline
\ \ \isacommand{ultimately}\isamarkupfalse%
\ \isacommand{have}\isamarkupfalse%
\ {\isachardoublequoteopen}y\ {\isacharequal}{\kern0pt}\ x{\isachardoublequoteclose}\isanewline
\ \ \ \ \isacommand{using}\isamarkupfalse%
\ x{\isacharunderscore}{\kern0pt}not{\isacharunderscore}{\kern0pt}a\isanewline
\ \ \ \ \isacommand{by}\isamarkupfalse%
\ presburger\isanewline
\ \ \isacommand{moreover}\isamarkupfalse%
\ \isacommand{have}\isamarkupfalse%
\ {\isachardoublequoteopen}y\ {\isasymnoteq}\ x{\isachardoublequoteclose}\isanewline
\ \ \ \ \isacommand{using}\isamarkupfalse%
\ asm\ y\isanewline
\ \ \ \ \isacommand{by}\isamarkupfalse%
\ blast\isanewline
\ \ \isacommand{ultimately}\isamarkupfalse%
\ \isacommand{show}\isamarkupfalse%
\ {\isachardoublequoteopen}False{\isachardoublequoteclose}\isanewline
\ \ \ \ \isacommand{by}\isamarkupfalse%
\ simp\isanewline
\isacommand{qed}\isamarkupfalse%
%
\endisatagproof
{\isafoldproof}%
%
\isadelimproof
\isanewline
%
\endisadelimproof
%
\isadelimtheory
\isanewline
%
\endisadelimtheory
%
\isatagtheory
\isacommand{end}\isamarkupfalse%
%
\endisatagtheory
{\isafoldtheory}%
%
\isadelimtheory
%
\endisadelimtheory
%
\end{isabellebody}%
\endinput
%:%file=~/Documents/Studies/VotingRuleGenerator/virage/src/test/resources/verifiedVotingRuleConstruction/theories/Social_Choice_Types/Preference_Relation.thy%:%
%:%6=3%:%
%:%11=4%:%
%:%12=5%:%
%:%14=8%:%
%:%18=10%:%
%:%34=12%:%
%:%35=12%:%
%:%36=13%:%
%:%37=14%:%
%:%46=17%:%
%:%47=18%:%
%:%56=20%:%
%:%66=26%:%
%:%67=26%:%
%:%68=27%:%
%:%69=28%:%
%:%70=28%:%
%:%71=29%:%
%:%72=30%:%
%:%73=31%:%
%:%74=32%:%
%:%75=32%:%
%:%76=33%:%
%:%77=34%:%
%:%80=35%:%
%:%84=35%:%
%:%85=35%:%
%:%86=36%:%
%:%87=36%:%
%:%92=36%:%
%:%95=37%:%
%:%96=38%:%
%:%97=38%:%
%:%98=39%:%
%:%99=40%:%
%:%102=41%:%
%:%106=41%:%
%:%107=41%:%
%:%108=42%:%
%:%109=42%:%
%:%123=44%:%
%:%133=46%:%
%:%134=46%:%
%:%135=47%:%
%:%136=48%:%
%:%137=49%:%
%:%138=49%:%
%:%139=50%:%
%:%140=51%:%
%:%141=52%:%
%:%142=53%:%
%:%149=54%:%
%:%150=54%:%
%:%151=55%:%
%:%152=55%:%
%:%153=56%:%
%:%154=56%:%
%:%155=57%:%
%:%156=57%:%
%:%157=58%:%
%:%158=58%:%
%:%159=59%:%
%:%160=59%:%
%:%161=60%:%
%:%162=60%:%
%:%163=61%:%
%:%164=61%:%
%:%165=62%:%
%:%166=62%:%
%:%167=62%:%
%:%168=63%:%
%:%169=63%:%
%:%170=64%:%
%:%171=64%:%
%:%172=65%:%
%:%173=65%:%
%:%174=65%:%
%:%175=66%:%
%:%176=66%:%
%:%177=67%:%
%:%178=68%:%
%:%179=68%:%
%:%180=69%:%
%:%195=71%:%
%:%205=73%:%
%:%206=73%:%
%:%207=74%:%
%:%208=75%:%
%:%209=76%:%
%:%210=76%:%
%:%211=77%:%
%:%214=78%:%
%:%218=78%:%
%:%219=78%:%
%:%220=79%:%
%:%221=79%:%
%:%226=79%:%
%:%229=80%:%
%:%230=81%:%
%:%231=81%:%
%:%232=82%:%
%:%235=83%:%
%:%239=83%:%
%:%240=83%:%
%:%241=84%:%
%:%242=84%:%
%:%247=84%:%
%:%250=85%:%
%:%251=86%:%
%:%252=86%:%
%:%253=87%:%
%:%254=88%:%
%:%255=89%:%
%:%256=89%:%
%:%257=90%:%
%:%258=91%:%
%:%259=92%:%
%:%260=92%:%
%:%261=93%:%
%:%262=94%:%
%:%269=95%:%
%:%270=95%:%
%:%271=96%:%
%:%272=96%:%
%:%273=97%:%
%:%274=97%:%
%:%275=98%:%
%:%276=98%:%
%:%277=99%:%
%:%278=99%:%
%:%279=100%:%
%:%280=100%:%
%:%281=101%:%
%:%282=102%:%
%:%283=102%:%
%:%284=103%:%
%:%285=104%:%
%:%286=104%:%
%:%287=105%:%
%:%288=105%:%
%:%289=106%:%
%:%290=106%:%
%:%291=107%:%
%:%292=107%:%
%:%293=108%:%
%:%294=108%:%
%:%295=109%:%
%:%301=109%:%
%:%304=110%:%
%:%305=111%:%
%:%306=111%:%
%:%307=112%:%
%:%308=113%:%
%:%311=114%:%
%:%315=114%:%
%:%316=114%:%
%:%317=115%:%
%:%318=115%:%
%:%319=116%:%
%:%320=116%:%
%:%321=117%:%
%:%322=118%:%
%:%323=119%:%
%:%324=119%:%
%:%325=120%:%
%:%326=121%:%
%:%327=121%:%
%:%328=122%:%
%:%329=122%:%
%:%330=123%:%
%:%331=124%:%
%:%332=124%:%
%:%333=125%:%
%:%334=125%:%
%:%335=126%:%
%:%336=126%:%
%:%337=127%:%
%:%338=128%:%
%:%339=129%:%
%:%340=129%:%
%:%341=130%:%
%:%342=131%:%
%:%343=131%:%
%:%344=132%:%
%:%345=132%:%
%:%346=133%:%
%:%347=134%:%
%:%348=134%:%
%:%349=135%:%
%:%350=135%:%
%:%351=136%:%
%:%352=136%:%
%:%353=137%:%
%:%354=138%:%
%:%355=139%:%
%:%356=139%:%
%:%357=140%:%
%:%358=141%:%
%:%359=142%:%
%:%360=143%:%
%:%361=143%:%
%:%362=144%:%
%:%363=144%:%
%:%364=145%:%
%:%365=145%:%
%:%366=146%:%
%:%367=146%:%
%:%368=147%:%
%:%369=147%:%
%:%370=148%:%
%:%371=149%:%
%:%372=149%:%
%:%373=150%:%
%:%374=150%:%
%:%375=151%:%
%:%376=151%:%
%:%377=152%:%
%:%383=152%:%
%:%386=153%:%
%:%387=154%:%
%:%388=154%:%
%:%389=155%:%
%:%390=156%:%
%:%391=157%:%
%:%392=158%:%
%:%393=159%:%
%:%396=160%:%
%:%400=160%:%
%:%401=160%:%
%:%402=161%:%
%:%403=162%:%
%:%404=162%:%
%:%405=163%:%
%:%406=163%:%
%:%407=164%:%
%:%408=165%:%
%:%409=166%:%
%:%410=166%:%
%:%411=167%:%
%:%412=168%:%
%:%413=168%:%
%:%414=169%:%
%:%415=169%:%
%:%416=170%:%
%:%417=170%:%
%:%418=171%:%
%:%419=171%:%
%:%420=172%:%
%:%421=172%:%
%:%422=173%:%
%:%423=174%:%
%:%424=175%:%
%:%425=175%:%
%:%426=176%:%
%:%427=177%:%
%:%428=177%:%
%:%429=178%:%
%:%430=178%:%
%:%431=179%:%
%:%432=179%:%
%:%433=180%:%
%:%434=180%:%
%:%435=181%:%
%:%436=181%:%
%:%437=182%:%
%:%438=183%:%
%:%439=183%:%
%:%440=184%:%
%:%441=185%:%
%:%442=185%:%
%:%443=186%:%
%:%444=186%:%
%:%445=187%:%
%:%446=187%:%
%:%447=188%:%
%:%448=188%:%
%:%449=189%:%
%:%450=189%:%
%:%451=190%:%
%:%452=190%:%
%:%453=191%:%
%:%454=191%:%
%:%455=192%:%
%:%456=192%:%
%:%457=193%:%
%:%458=193%:%
%:%459=194%:%
%:%460=194%:%
%:%461=195%:%
%:%462=195%:%
%:%463=196%:%
%:%464=196%:%
%:%465=197%:%
%:%466=197%:%
%:%467=198%:%
%:%468=199%:%
%:%469=200%:%
%:%470=200%:%
%:%471=201%:%
%:%472=202%:%
%:%473=203%:%
%:%474=204%:%
%:%475=205%:%
%:%476=205%:%
%:%477=206%:%
%:%478=206%:%
%:%479=207%:%
%:%480=207%:%
%:%481=208%:%
%:%482=208%:%
%:%483=209%:%
%:%484=209%:%
%:%485=210%:%
%:%486=210%:%
%:%487=211%:%
%:%488=211%:%
%:%489=212%:%
%:%490=212%:%
%:%491=213%:%
%:%497=213%:%
%:%500=214%:%
%:%501=215%:%
%:%502=215%:%
%:%505=216%:%
%:%509=216%:%
%:%510=216%:%
%:%511=217%:%
%:%512=217%:%
%:%517=217%:%
%:%520=218%:%
%:%521=219%:%
%:%522=219%:%
%:%523=220%:%
%:%524=221%:%
%:%525=222%:%
%:%526=223%:%
%:%529=224%:%
%:%533=224%:%
%:%534=224%:%
%:%535=225%:%
%:%536=225%:%
%:%537=226%:%
%:%538=226%:%
%:%539=227%:%
%:%540=227%:%
%:%541=228%:%
%:%542=229%:%
%:%543=230%:%
%:%544=231%:%
%:%545=232%:%
%:%546=232%:%
%:%547=233%:%
%:%548=234%:%
%:%549=235%:%
%:%550=236%:%
%:%551=236%:%
%:%552=237%:%
%:%553=237%:%
%:%554=238%:%
%:%555=238%:%
%:%556=239%:%
%:%557=239%:%
%:%558=240%:%
%:%559=241%:%
%:%560=241%:%
%:%561=242%:%
%:%562=242%:%
%:%563=243%:%
%:%564=243%:%
%:%565=244%:%
%:%566=244%:%
%:%567=245%:%
%:%568=245%:%
%:%569=246%:%
%:%570=246%:%
%:%571=247%:%
%:%572=247%:%
%:%573=248%:%
%:%579=248%:%
%:%582=249%:%
%:%583=250%:%
%:%584=250%:%
%:%585=251%:%
%:%586=252%:%
%:%587=253%:%
%:%588=254%:%
%:%591=255%:%
%:%595=255%:%
%:%596=255%:%
%:%597=256%:%
%:%598=256%:%
%:%603=256%:%
%:%606=257%:%
%:%607=258%:%
%:%608=258%:%
%:%609=259%:%
%:%610=260%:%
%:%611=261%:%
%:%612=262%:%
%:%615=263%:%
%:%619=263%:%
%:%620=263%:%
%:%621=264%:%
%:%622=264%:%
%:%623=265%:%
%:%624=265%:%
%:%625=266%:%
%:%626=267%:%
%:%627=268%:%
%:%628=269%:%
%:%629=269%:%
%:%630=270%:%
%:%631=271%:%
%:%632=272%:%
%:%633=273%:%
%:%634=274%:%
%:%635=275%:%
%:%636=275%:%
%:%637=276%:%
%:%638=276%:%
%:%639=277%:%
%:%640=277%:%
%:%641=278%:%
%:%647=278%:%
%:%650=279%:%
%:%651=280%:%
%:%652=280%:%
%:%653=281%:%
%:%654=282%:%
%:%655=283%:%
%:%656=284%:%
%:%659=285%:%
%:%663=285%:%
%:%664=285%:%
%:%665=286%:%
%:%666=287%:%
%:%667=288%:%
%:%668=289%:%
%:%669=290%:%
%:%670=290%:%
%:%675=290%:%
%:%678=291%:%
%:%679=292%:%
%:%680=292%:%
%:%681=293%:%
%:%682=294%:%
%:%683=295%:%
%:%684=296%:%
%:%685=297%:%
%:%688=298%:%
%:%692=298%:%
%:%693=298%:%
%:%694=299%:%
%:%695=299%:%
%:%700=299%:%
%:%703=300%:%
%:%704=301%:%
%:%705=301%:%
%:%706=302%:%
%:%707=303%:%
%:%710=304%:%
%:%714=304%:%
%:%715=304%:%
%:%716=305%:%
%:%717=305%:%
%:%722=305%:%
%:%725=306%:%
%:%726=307%:%
%:%727=307%:%
%:%728=308%:%
%:%729=309%:%
%:%730=310%:%
%:%731=311%:%
%:%732=312%:%
%:%735=313%:%
%:%739=313%:%
%:%740=313%:%
%:%741=314%:%
%:%742=314%:%
%:%747=314%:%
%:%750=315%:%
%:%751=316%:%
%:%752=316%:%
%:%753=317%:%
%:%754=318%:%
%:%757=319%:%
%:%761=319%:%
%:%762=319%:%
%:%763=320%:%
%:%764=321%:%
%:%765=321%:%
%:%770=321%:%
%:%773=322%:%
%:%774=323%:%
%:%775=323%:%
%:%776=324%:%
%:%783=325%:%
%:%784=325%:%
%:%785=326%:%
%:%786=326%:%
%:%787=327%:%
%:%788=327%:%
%:%789=328%:%
%:%790=328%:%
%:%791=329%:%
%:%792=329%:%
%:%793=330%:%
%:%794=330%:%
%:%795=331%:%
%:%796=331%:%
%:%797=332%:%
%:%798=332%:%
%:%799=333%:%
%:%800=333%:%
%:%801=333%:%
%:%802=334%:%
%:%803=334%:%
%:%804=335%:%
%:%805=336%:%
%:%806=336%:%
%:%807=337%:%
%:%808=337%:%
%:%809=337%:%
%:%810=338%:%
%:%811=338%:%
%:%812=339%:%
%:%813=339%:%
%:%814=340%:%
%:%829=342%:%
%:%839=344%:%
%:%840=344%:%
%:%841=345%:%
%:%842=346%:%
%:%843=347%:%
%:%844=348%:%
%:%847=349%:%
%:%851=349%:%
%:%852=349%:%
%:%853=350%:%
%:%854=350%:%
%:%859=350%:%
%:%862=351%:%
%:%863=352%:%
%:%864=352%:%
%:%865=353%:%
%:%866=354%:%
%:%867=355%:%
%:%868=356%:%
%:%871=357%:%
%:%875=357%:%
%:%876=357%:%
%:%877=358%:%
%:%878=358%:%
%:%883=358%:%
%:%886=359%:%
%:%887=360%:%
%:%888=360%:%
%:%889=361%:%
%:%890=362%:%
%:%891=363%:%
%:%892=364%:%
%:%893=365%:%
%:%896=366%:%
%:%900=366%:%
%:%901=366%:%
%:%902=367%:%
%:%903=368%:%
%:%904=369%:%
%:%905=369%:%
%:%910=369%:%
%:%913=370%:%
%:%914=371%:%
%:%915=371%:%
%:%916=372%:%
%:%917=373%:%
%:%918=374%:%
%:%919=375%:%
%:%922=376%:%
%:%926=376%:%
%:%927=376%:%
%:%928=377%:%
%:%929=377%:%
%:%934=377%:%
%:%937=378%:%
%:%938=379%:%
%:%939=379%:%
%:%942=380%:%
%:%946=380%:%
%:%947=380%:%
%:%952=380%:%
%:%955=381%:%
%:%956=382%:%
%:%957=382%:%
%:%958=383%:%
%:%959=384%:%
%:%960=385%:%
%:%960=386%:%
%:%961=387%:%
%:%962=388%:%
%:%965=389%:%
%:%969=389%:%
%:%970=389%:%
%:%971=390%:%
%:%972=390%:%
%:%977=390%:%
%:%980=391%:%
%:%981=392%:%
%:%982=392%:%
%:%983=393%:%
%:%984=394%:%
%:%985=395%:%
%:%986=396%:%
%:%987=397%:%
%:%988=398%:%
%:%989=399%:%
%:%992=400%:%
%:%996=400%:%
%:%997=400%:%
%:%998=401%:%
%:%999=401%:%
%:%1000=402%:%
%:%1001=403%:%
%:%1002=403%:%
%:%1007=403%:%
%:%1010=404%:%
%:%1011=405%:%
%:%1012=405%:%
%:%1013=406%:%
%:%1014=407%:%
%:%1015=408%:%
%:%1016=409%:%
%:%1023=410%:%
%:%1024=410%:%
%:%1025=411%:%
%:%1026=411%:%
%:%1027=412%:%
%:%1028=412%:%
%:%1029=413%:%
%:%1030=414%:%
%:%1031=414%:%
%:%1032=415%:%
%:%1033=415%:%
%:%1034=416%:%
%:%1035=417%:%
%:%1036=418%:%
%:%1037=418%:%
%:%1038=419%:%
%:%1039=419%:%
%:%1040=420%:%
%:%1041=420%:%
%:%1042=421%:%
%:%1043=421%:%
%:%1044=422%:%
%:%1045=422%:%
%:%1046=423%:%
%:%1047=423%:%
%:%1048=423%:%
%:%1049=424%:%
%:%1050=424%:%
%:%1051=425%:%
%:%1052=425%:%
%:%1053=426%:%
%:%1054=426%:%
%:%1055=427%:%
%:%1056=427%:%
%:%1057=428%:%
%:%1058=429%:%
%:%1059=429%:%
%:%1060=430%:%
%:%1061=430%:%
%:%1062=431%:%
%:%1063=431%:%
%:%1064=432%:%
%:%1065=432%:%
%:%1066=433%:%
%:%1067=433%:%
%:%1068=434%:%
%:%1069=434%:%
%:%1070=435%:%
%:%1071=435%:%
%:%1072=436%:%
%:%1073=436%:%
%:%1074=437%:%
%:%1075=437%:%
%:%1076=438%:%
%:%1077=439%:%
%:%1078=439%:%
%:%1079=439%:%
%:%1080=440%:%
%:%1081=440%:%
%:%1082=441%:%
%:%1083=442%:%
%:%1084=442%:%
%:%1085=443%:%
%:%1086=443%:%
%:%1087=443%:%
%:%1088=444%:%
%:%1089=444%:%
%:%1090=445%:%
%:%1091=446%:%
%:%1092=446%:%
%:%1093=447%:%
%:%1094=447%:%
%:%1095=448%:%
%:%1096=448%:%
%:%1097=449%:%
%:%1098=450%:%
%:%1099=451%:%
%:%1100=451%:%
%:%1101=452%:%
%:%1102=452%:%
%:%1103=452%:%
%:%1104=453%:%
%:%1105=453%:%
%:%1106=454%:%
%:%1107=454%:%
%:%1108=455%:%
%:%1109=455%:%
%:%1110=456%:%
%:%1111=456%:%
%:%1112=457%:%
%:%1113=458%:%
%:%1114=458%:%
%:%1115=459%:%
%:%1116=460%:%
%:%1117=461%:%
%:%1118=461%:%
%:%1119=462%:%
%:%1120=462%:%
%:%1121=463%:%
%:%1122=463%:%
%:%1123=464%:%
%:%1124=465%:%
%:%1125=465%:%
%:%1126=466%:%
%:%1127=466%:%
%:%1128=467%:%
%:%1129=467%:%
%:%1130=468%:%
%:%1131=469%:%
%:%1132=469%:%
%:%1133=470%:%
%:%1134=470%:%
%:%1135=471%:%
%:%1136=471%:%
%:%1137=472%:%
%:%1138=472%:%
%:%1139=473%:%
%:%1140=473%:%
%:%1141=474%:%
%:%1142=474%:%
%:%1143=475%:%
%:%1144=475%:%
%:%1145=476%:%
%:%1146=476%:%
%:%1147=477%:%
%:%1148=478%:%
%:%1149=478%:%
%:%1150=479%:%
%:%1151=479%:%
%:%1152=480%:%
%:%1153=480%:%
%:%1154=481%:%
%:%1155=482%:%
%:%1156=482%:%
%:%1157=483%:%
%:%1158=483%:%
%:%1159=484%:%
%:%1160=484%:%
%:%1161=485%:%
%:%1162=486%:%
%:%1163=486%:%
%:%1164=487%:%
%:%1165=487%:%
%:%1166=488%:%
%:%1167=489%:%
%:%1168=489%:%
%:%1169=490%:%
%:%1170=490%:%
%:%1171=491%:%
%:%1172=491%:%
%:%1173=492%:%
%:%1174=493%:%
%:%1175=493%:%
%:%1176=494%:%
%:%1177=494%:%
%:%1178=495%:%
%:%1179=495%:%
%:%1180=495%:%
%:%1181=496%:%
%:%1182=497%:%
%:%1183=497%:%
%:%1184=498%:%
%:%1185=499%:%
%:%1186=499%:%
%:%1187=500%:%
%:%1188=500%:%
%:%1189=500%:%
%:%1190=501%:%
%:%1191=501%:%
%:%1192=502%:%
%:%1193=502%:%
%:%1194=503%:%
%:%1195=503%:%
%:%1196=503%:%
%:%1197=504%:%
%:%1198=505%:%
%:%1199=505%:%
%:%1200=506%:%
%:%1201=506%:%
%:%1202=507%:%
%:%1203=507%:%
%:%1204=508%:%
%:%1205=508%:%
%:%1206=509%:%
%:%1207=509%:%
%:%1208=510%:%
%:%1209=510%:%
%:%1210=511%:%
%:%1211=511%:%
%:%1212=512%:%
%:%1213=512%:%
%:%1214=513%:%
%:%1215=513%:%
%:%1216=513%:%
%:%1217=514%:%
%:%1218=514%:%
%:%1219=515%:%
%:%1220=516%:%
%:%1221=516%:%
%:%1222=517%:%
%:%1223=517%:%
%:%1224=517%:%
%:%1225=518%:%
%:%1226=518%:%
%:%1227=519%:%
%:%1228=519%:%
%:%1229=520%:%
%:%1230=520%:%
%:%1231=521%:%
%:%1232=521%:%
%:%1233=522%:%
%:%1234=522%:%
%:%1235=523%:%
%:%1236=523%:%
%:%1237=524%:%
%:%1238=524%:%
%:%1239=525%:%
%:%1240=525%:%
%:%1241=526%:%
%:%1242=526%:%
%:%1243=527%:%
%:%1244=528%:%
%:%1245=529%:%
%:%1246=529%:%
%:%1247=530%:%
%:%1248=530%:%
%:%1249=531%:%
%:%1250=531%:%
%:%1251=532%:%
%:%1252=532%:%
%:%1253=533%:%
%:%1256=536%:%
%:%1257=537%:%
%:%1258=537%:%
%:%1259=538%:%
%:%1260=538%:%
%:%1261=539%:%
%:%1262=539%:%
%:%1263=540%:%
%:%1264=541%:%
%:%1265=541%:%
%:%1266=542%:%
%:%1267=542%:%
%:%1268=543%:%
%:%1269=543%:%
%:%1270=544%:%
%:%1271=544%:%
%:%1272=545%:%
%:%1273=545%:%
%:%1274=546%:%
%:%1275=547%:%
%:%1276=547%:%
%:%1277=548%:%
%:%1278=548%:%
%:%1279=549%:%
%:%1280=549%:%
%:%1281=550%:%
%:%1283=552%:%
%:%1284=553%:%
%:%1285=553%:%
%:%1286=554%:%
%:%1287=554%:%
%:%1288=555%:%
%:%1289=555%:%
%:%1290=556%:%
%:%1292=558%:%
%:%1293=559%:%
%:%1294=559%:%
%:%1295=560%:%
%:%1296=560%:%
%:%1297=561%:%
%:%1298=561%:%
%:%1299=562%:%
%:%1300=562%:%
%:%1301=563%:%
%:%1302=564%:%
%:%1303=565%:%
%:%1304=565%:%
%:%1305=566%:%
%:%1306=566%:%
%:%1307=567%:%
%:%1308=568%:%
%:%1309=568%:%
%:%1310=569%:%
%:%1311=569%:%
%:%1312=570%:%
%:%1313=571%:%
%:%1314=571%:%
%:%1315=572%:%
%:%1316=572%:%
%:%1317=573%:%
%:%1318=573%:%
%:%1319=574%:%
%:%1320=574%:%
%:%1321=575%:%
%:%1322=575%:%
%:%1323=576%:%
%:%1324=576%:%
%:%1325=577%:%
%:%1326=577%:%
%:%1327=578%:%
%:%1328=578%:%
%:%1329=579%:%
%:%1330=579%:%
%:%1331=580%:%
%:%1332=580%:%
%:%1333=581%:%
%:%1334=581%:%
%:%1335=582%:%
%:%1336=583%:%
%:%1337=583%:%
%:%1338=584%:%
%:%1339=584%:%
%:%1340=585%:%
%:%1341=585%:%
%:%1342=586%:%
%:%1343=587%:%
%:%1344=588%:%
%:%1345=588%:%
%:%1346=589%:%
%:%1347=589%:%
%:%1348=589%:%
%:%1349=590%:%
%:%1350=590%:%
%:%1351=591%:%
%:%1352=592%:%
%:%1353=592%:%
%:%1354=593%:%
%:%1355=593%:%
%:%1356=593%:%
%:%1357=594%:%
%:%1358=594%:%
%:%1359=595%:%
%:%1360=595%:%
%:%1361=596%:%
%:%1362=596%:%
%:%1363=597%:%
%:%1364=597%:%
%:%1365=598%:%
%:%1366=598%:%
%:%1367=599%:%
%:%1368=599%:%
%:%1369=600%:%
%:%1370=600%:%
%:%1371=601%:%
%:%1372=601%:%
%:%1373=602%:%
%:%1374=602%:%
%:%1375=603%:%
%:%1376=603%:%
%:%1377=604%:%
%:%1378=604%:%
%:%1379=605%:%
%:%1380=605%:%
%:%1381=606%:%
%:%1382=606%:%
%:%1383=607%:%
%:%1384=608%:%
%:%1385=609%:%
%:%1386=609%:%
%:%1387=610%:%
%:%1393=610%:%
%:%1396=611%:%
%:%1397=612%:%
%:%1398=612%:%
%:%1399=613%:%
%:%1400=614%:%
%:%1401=615%:%
%:%1402=616%:%
%:%1403=617%:%
%:%1410=618%:%
%:%1411=618%:%
%:%1412=619%:%
%:%1413=619%:%
%:%1414=620%:%
%:%1415=620%:%
%:%1416=621%:%
%:%1417=621%:%
%:%1418=622%:%
%:%1419=622%:%
%:%1420=622%:%
%:%1421=623%:%
%:%1422=624%:%
%:%1423=625%:%
%:%1424=625%:%
%:%1425=626%:%
%:%1426=626%:%
%:%1427=627%:%
%:%1428=627%:%
%:%1429=627%:%
%:%1430=628%:%
%:%1431=628%:%
%:%1432=629%:%
%:%1433=629%:%
%:%1434=630%:%
%:%1435=630%:%
%:%1436=630%:%
%:%1437=631%:%
%:%1438=631%:%
%:%1439=632%:%
%:%1440=632%:%
%:%1441=633%:%
%:%1447=633%:%
%:%1450=634%:%
%:%1451=635%:%
%:%1452=635%:%
%:%1453=636%:%
%:%1454=637%:%
%:%1457=638%:%
%:%1461=638%:%
%:%1462=638%:%
%:%1463=639%:%
%:%1464=639%:%
%:%1478=641%:%
%:%1488=643%:%
%:%1489=643%:%
%:%1490=644%:%
%:%1491=645%:%
%:%1493=647%:%
%:%1494=648%:%
%:%1495=649%:%
%:%1496=649%:%
%:%1497=650%:%
%:%1498=651%:%
%:%1499=652%:%
%:%1500=653%:%
%:%1501=654%:%
%:%1502=654%:%
%:%1503=655%:%
%:%1504=656%:%
%:%1507=657%:%
%:%1511=657%:%
%:%1512=657%:%
%:%1517=657%:%
%:%1520=658%:%
%:%1521=659%:%
%:%1522=659%:%
%:%1523=660%:%
%:%1524=661%:%
%:%1531=662%:%
%:%1532=662%:%
%:%1533=663%:%
%:%1534=663%:%
%:%1535=663%:%
%:%1536=664%:%
%:%1537=665%:%
%:%1538=666%:%
%:%1539=666%:%
%:%1540=667%:%
%:%1541=667%:%
%:%1542=668%:%
%:%1543=668%:%
%:%1544=669%:%
%:%1550=669%:%
%:%1553=670%:%
%:%1554=671%:%
%:%1555=671%:%
%:%1556=672%:%
%:%1557=673%:%
%:%1564=674%:%
%:%1565=674%:%
%:%1566=675%:%
%:%1567=675%:%
%:%1568=676%:%
%:%1569=677%:%
%:%1570=677%:%
%:%1571=678%:%
%:%1572=679%:%
%:%1573=680%:%
%:%1574=681%:%
%:%1575=682%:%
%:%1576=683%:%
%:%1577=683%:%
%:%1578=684%:%
%:%1579=684%:%
%:%1580=685%:%
%:%1581=685%:%
%:%1582=686%:%
%:%1583=686%:%
%:%1584=687%:%
%:%1585=687%:%
%:%1586=688%:%
%:%1587=688%:%
%:%1588=689%:%
%:%1589=689%:%
%:%1590=690%:%
%:%1591=690%:%
%:%1592=691%:%
%:%1593=692%:%
%:%1594=693%:%
%:%1595=693%:%
%:%1596=694%:%
%:%1597=694%:%
%:%1598=695%:%
%:%1599=696%:%
%:%1600=696%:%
%:%1601=697%:%
%:%1602=697%:%
%:%1603=698%:%
%:%1604=698%:%
%:%1605=699%:%
%:%1606=700%:%
%:%1607=700%:%
%:%1608=701%:%
%:%1609=701%:%
%:%1610=702%:%
%:%1611=702%:%
%:%1612=703%:%
%:%1613=703%:%
%:%1614=704%:%
%:%1615=704%:%
%:%1616=704%:%
%:%1617=705%:%
%:%1618=706%:%
%:%1619=706%:%
%:%1620=707%:%
%:%1621=707%:%
%:%1622=708%:%
%:%1623=709%:%
%:%1624=709%:%
%:%1625=710%:%
%:%1626=710%:%
%:%1627=711%:%
%:%1628=712%:%
%:%1629=712%:%
%:%1630=713%:%
%:%1631=713%:%
%:%1632=714%:%
%:%1633=714%:%
%:%1634=715%:%
%:%1635=716%:%
%:%1636=716%:%
%:%1637=717%:%
%:%1638=717%:%
%:%1639=718%:%
%:%1640=718%:%
%:%1641=719%:%
%:%1642=720%:%
%:%1643=720%:%
%:%1644=721%:%
%:%1645=721%:%
%:%1646=722%:%
%:%1647=722%:%
%:%1648=723%:%
%:%1649=724%:%
%:%1650=724%:%
%:%1651=725%:%
%:%1652=725%:%
%:%1653=726%:%
%:%1654=726%:%
%:%1655=727%:%
%:%1656=728%:%
%:%1657=728%:%
%:%1658=729%:%
%:%1659=729%:%
%:%1660=730%:%
%:%1661=730%:%
%:%1662=731%:%
%:%1663=732%:%
%:%1664=733%:%
%:%1665=733%:%
%:%1666=734%:%
%:%1667=734%:%
%:%1668=735%:%
%:%1669=735%:%
%:%1670=736%:%
%:%1671=737%:%
%:%1672=737%:%
%:%1673=738%:%
%:%1679=738%:%
%:%1682=739%:%
%:%1683=740%:%
%:%1684=740%:%
%:%1685=741%:%
%:%1686=742%:%
%:%1687=743%:%
%:%1688=744%:%
%:%1695=745%:%
%:%1696=745%:%
%:%1697=746%:%
%:%1698=746%:%
%:%1699=747%:%
%:%1700=747%:%
%:%1701=748%:%
%:%1702=748%:%
%:%1703=749%:%
%:%1704=749%:%
%:%1705=750%:%
%:%1706=750%:%
%:%1707=751%:%
%:%1708=752%:%
%:%1709=753%:%
%:%1710=753%:%
%:%1711=754%:%
%:%1712=754%:%
%:%1713=755%:%
%:%1714=755%:%
%:%1715=756%:%
%:%1716=756%:%
%:%1717=757%:%
%:%1718=757%:%
%:%1719=758%:%
%:%1720=759%:%
%:%1721=759%:%
%:%1722=760%:%
%:%1723=760%:%
%:%1724=761%:%
%:%1725=761%:%
%:%1726=762%:%
%:%1727=762%:%
%:%1728=763%:%
%:%1729=763%:%
%:%1730=764%:%
%:%1731=765%:%
%:%1732=765%:%
%:%1733=766%:%
%:%1734=766%:%
%:%1735=767%:%
%:%1736=767%:%
%:%1737=768%:%
%:%1738=768%:%
%:%1739=769%:%
%:%1740=769%:%
%:%1741=770%:%
%:%1742=770%:%
%:%1743=771%:%
%:%1744=772%:%
%:%1745=772%:%
%:%1746=773%:%
%:%1747=773%:%
%:%1748=774%:%
%:%1749=774%:%
%:%1750=775%:%
%:%1751=775%:%
%:%1752=776%:%
%:%1753=776%:%
%:%1754=777%:%
%:%1755=778%:%
%:%1756=779%:%
%:%1757=780%:%
%:%1758=780%:%
%:%1759=781%:%
%:%1760=781%:%
%:%1761=782%:%
%:%1767=782%:%
%:%1770=783%:%
%:%1771=784%:%
%:%1772=784%:%
%:%1773=785%:%
%:%1774=786%:%
%:%1777=787%:%
%:%1781=787%:%
%:%1782=787%:%
%:%1783=788%:%
%:%1784=788%:%
%:%1785=789%:%
%:%1786=789%:%
%:%1787=790%:%
%:%1788=791%:%
%:%1789=791%:%
%:%1790=792%:%
%:%1791=793%:%
%:%1792=793%:%
%:%1793=794%:%
%:%1794=794%:%
%:%1795=795%:%
%:%1796=795%:%
%:%1797=796%:%
%:%1798=796%:%
%:%1799=797%:%
%:%1800=798%:%
%:%1801=798%:%
%:%1802=799%:%
%:%1803=799%:%
%:%1804=800%:%
%:%1805=800%:%
%:%1806=801%:%
%:%1807=801%:%
%:%1808=802%:%
%:%1809=802%:%
%:%1810=803%:%
%:%1811=804%:%
%:%1812=804%:%
%:%1813=805%:%
%:%1814=805%:%
%:%1815=806%:%
%:%1816=807%:%
%:%1817=807%:%
%:%1818=808%:%
%:%1819=809%:%
%:%1820=809%:%
%:%1821=810%:%
%:%1822=810%:%
%:%1823=811%:%
%:%1824=812%:%
%:%1825=812%:%
%:%1826=813%:%
%:%1827=814%:%
%:%1828=814%:%
%:%1829=815%:%
%:%1830=815%:%
%:%1831=816%:%
%:%1832=817%:%
%:%1833=817%:%
%:%1834=818%:%
%:%1835=819%:%
%:%1836=819%:%
%:%1837=820%:%
%:%1838=820%:%
%:%1839=821%:%
%:%1840=821%:%
%:%1841=822%:%
%:%1842=823%:%
%:%1843=823%:%
%:%1844=824%:%
%:%1845=824%:%
%:%1846=825%:%
%:%1847=825%:%
%:%1848=826%:%
%:%1849=827%:%
%:%1850=827%:%
%:%1851=828%:%
%:%1857=828%:%
%:%1860=829%:%
%:%1861=830%:%
%:%1862=830%:%
%:%1863=831%:%
%:%1864=832%:%
%:%1865=833%:%
%:%1866=834%:%
%:%1873=835%:%
%:%1874=835%:%
%:%1875=836%:%
%:%1876=836%:%
%:%1877=837%:%
%:%1878=837%:%
%:%1879=838%:%
%:%1880=839%:%
%:%1881=840%:%
%:%1882=840%:%
%:%1883=841%:%
%:%1884=841%:%
%:%1885=842%:%
%:%1886=842%:%
%:%1887=843%:%
%:%1888=843%:%
%:%1889=844%:%
%:%1890=844%:%
%:%1891=845%:%
%:%1892=845%:%
%:%1893=846%:%
%:%1894=846%:%
%:%1895=847%:%
%:%1896=847%:%
%:%1897=848%:%
%:%1898=848%:%
%:%1899=849%:%
%:%1900=850%:%
%:%1901=851%:%
%:%1902=852%:%
%:%1903=852%:%
%:%1904=853%:%
%:%1905=853%:%
%:%1906=854%:%
%:%1907=854%:%
%:%1908=855%:%
%:%1909=855%:%
%:%1910=856%:%
%:%1916=856%:%
%:%1919=857%:%
%:%1920=858%:%
%:%1921=858%:%
%:%1922=859%:%
%:%1923=860%:%
%:%1924=861%:%
%:%1925=862%:%
%:%1926=863%:%
%:%1927=864%:%
%:%1934=865%:%
%:%1935=865%:%
%:%1936=866%:%
%:%1937=866%:%
%:%1938=866%:%
%:%1939=867%:%
%:%1940=868%:%
%:%1941=868%:%
%:%1942=869%:%
%:%1943=869%:%
%:%1944=869%:%
%:%1945=870%:%
%:%1946=871%:%
%:%1947=871%:%
%:%1948=872%:%
%:%1949=872%:%
%:%1950=873%:%
%:%1951=874%:%
%:%1952=875%:%
%:%1953=875%:%
%:%1954=876%:%
%:%1955=876%:%
%:%1956=877%:%
%:%1957=877%:%
%:%1958=878%:%
%:%1959=878%:%
%:%1960=879%:%
%:%1961=879%:%
%:%1962=880%:%
%:%1963=880%:%
%:%1964=881%:%
%:%1965=881%:%
%:%1966=882%:%
%:%1967=883%:%
%:%1968=883%:%
%:%1969=884%:%
%:%1970=884%:%
%:%1971=884%:%
%:%1972=885%:%
%:%1973=886%:%
%:%1974=886%:%
%:%1975=887%:%
%:%1976=887%:%
%:%1977=888%:%
%:%1978=888%:%
%:%1979=888%:%
%:%1980=888%:%
%:%1981=889%:%
%:%1982=890%:%
%:%1983=891%:%
%:%1984=891%:%
%:%1985=892%:%
%:%1986=892%:%
%:%1987=893%:%
%:%1988=893%:%
%:%1989=893%:%
%:%1990=894%:%
%:%1991=894%:%
%:%1992=895%:%
%:%1993=895%:%
%:%1994=896%:%
%:%1995=896%:%
%:%1996=897%:%
%:%1997=897%:%
%:%1998=898%:%
%:%1999=899%:%
%:%2000=899%:%
%:%2001=900%:%
%:%2002=901%:%
%:%2003=901%:%
%:%2004=902%:%
%:%2005=902%:%
%:%2006=902%:%
%:%2007=903%:%
%:%2008=904%:%
%:%2009=904%:%
%:%2010=905%:%
%:%2011=905%:%
%:%2012=906%:%
%:%2013=906%:%
%:%2014=906%:%
%:%2015=907%:%
%:%2016=908%:%
%:%2017=908%:%
%:%2018=909%:%
%:%2019=910%:%
%:%2020=910%:%
%:%2021=911%:%
%:%2022=911%:%
%:%2023=911%:%
%:%2024=912%:%
%:%2026=914%:%
%:%2027=915%:%
%:%2028=915%:%
%:%2029=916%:%
%:%2030=916%:%
%:%2031=917%:%
%:%2035=921%:%
%:%2036=922%:%
%:%2037=922%:%
%:%2038=923%:%
%:%2039=923%:%
%:%2040=924%:%
%:%2041=924%:%
%:%2042=925%:%
%:%2043=926%:%
%:%2044=926%:%
%:%2045=927%:%
%:%2046=927%:%
%:%2047=928%:%
%:%2048=928%:%
%:%2049=929%:%
%:%2052=932%:%
%:%2053=933%:%
%:%2054=933%:%
%:%2055=934%:%
%:%2056=934%:%
%:%2057=935%:%
%:%2058=935%:%
%:%2059=936%:%
%:%2062=939%:%
%:%2063=940%:%
%:%2064=940%:%
%:%2065=941%:%
%:%2066=941%:%
%:%2067=941%:%
%:%2068=942%:%
%:%2069=943%:%
%:%2070=943%:%
%:%2071=944%:%
%:%2072=944%:%
%:%2073=945%:%
%:%2074=945%:%
%:%2075=946%:%
%:%2076=947%:%
%:%2077=947%:%
%:%2078=948%:%
%:%2079=948%:%
%:%2080=949%:%
%:%2081=950%:%
%:%2082=950%:%
%:%2083=951%:%
%:%2084=951%:%
%:%2085=952%:%
%:%2086=952%:%
%:%2087=953%:%
%:%2088=954%:%
%:%2089=955%:%
%:%2090=955%:%
%:%2091=956%:%
%:%2092=956%:%
%:%2093=956%:%
%:%2094=957%:%
%:%2095=958%:%
%:%2096=958%:%
%:%2097=959%:%
%:%2098=959%:%
%:%2099=960%:%
%:%2100=960%:%
%:%2101=960%:%
%:%2102=961%:%
%:%2103=962%:%
%:%2104=962%:%
%:%2105=963%:%
%:%2106=963%:%
%:%2107=964%:%
%:%2108=964%:%
%:%2109=964%:%
%:%2110=965%:%
%:%2111=966%:%
%:%2112=966%:%
%:%2113=967%:%
%:%2114=967%:%
%:%2115=968%:%
%:%2116=968%:%
%:%2117=968%:%
%:%2118=969%:%
%:%2119=970%:%
%:%2120=970%:%
%:%2121=971%:%
%:%2122=971%:%
%:%2123=972%:%
%:%2124=972%:%
%:%2125=973%:%
%:%2126=973%:%
%:%2127=974%:%
%:%2128=974%:%
%:%2129=975%:%
%:%2130=975%:%
%:%2131=976%:%
%:%2132=976%:%
%:%2133=977%:%
%:%2134=977%:%
%:%2135=977%:%
%:%2136=978%:%
%:%2137=979%:%
%:%2138=979%:%
%:%2139=980%:%
%:%2140=980%:%
%:%2141=981%:%
%:%2142=981%:%
%:%2143=982%:%
%:%2144=982%:%
%:%2145=983%:%
%:%2146=983%:%
%:%2147=984%:%
%:%2153=984%:%
%:%2156=985%:%
%:%2157=986%:%
%:%2158=986%:%
%:%2159=987%:%
%:%2160=988%:%
%:%2161=989%:%
%:%2162=990%:%
%:%2163=991%:%
%:%2170=992%:%
%:%2171=992%:%
%:%2172=993%:%
%:%2173=993%:%
%:%2174=994%:%
%:%2175=994%:%
%:%2176=995%:%
%:%2177=995%:%
%:%2178=996%:%
%:%2179=996%:%
%:%2180=997%:%
%:%2181=997%:%
%:%2182=998%:%
%:%2183=998%:%
%:%2184=999%:%
%:%2190=999%:%
%:%2193=1000%:%
%:%2194=1001%:%
%:%2195=1001%:%
%:%2196=1002%:%
%:%2197=1003%:%
%:%2198=1004%:%
%:%2199=1005%:%
%:%2200=1006%:%
%:%2207=1007%:%
%:%2208=1007%:%
%:%2209=1008%:%
%:%2210=1008%:%
%:%2211=1009%:%
%:%2212=1009%:%
%:%2213=1010%:%
%:%2214=1010%:%
%:%2215=1011%:%
%:%2216=1012%:%
%:%2217=1012%:%
%:%2218=1013%:%
%:%2219=1013%:%
%:%2220=1014%:%
%:%2221=1014%:%
%:%2222=1015%:%
%:%2223=1015%:%
%:%2224=1016%:%
%:%2225=1016%:%
%:%2226=1017%:%
%:%2227=1017%:%
%:%2228=1018%:%
%:%2229=1018%:%
%:%2230=1019%:%
%:%2231=1019%:%
%:%2232=1020%:%
%:%2233=1020%:%
%:%2234=1021%:%
%:%2235=1021%:%
%:%2236=1022%:%
%:%2237=1022%:%
%:%2238=1023%:%
%:%2239=1023%:%
%:%2240=1024%:%
%:%2241=1024%:%
%:%2242=1025%:%
%:%2243=1025%:%
%:%2244=1026%:%
%:%2245=1026%:%
%:%2246=1027%:%
%:%2247=1027%:%
%:%2248=1028%:%
%:%2249=1028%:%
%:%2250=1028%:%
%:%2251=1029%:%
%:%2256=1034%:%
%:%2257=1035%:%
%:%2258=1035%:%
%:%2259=1036%:%
%:%2260=1036%:%
%:%2261=1037%:%
%:%2262=1037%:%
%:%2263=1037%:%
%:%2264=1038%:%
%:%2265=1039%:%
%:%2266=1039%:%
%:%2267=1040%:%
%:%2268=1040%:%
%:%2269=1041%:%
%:%2270=1041%:%
%:%2271=1041%:%
%:%2272=1042%:%
%:%2273=1042%:%
%:%2274=1043%:%
%:%2275=1044%:%
%:%2276=1044%:%
%:%2277=1045%:%
%:%2278=1045%:%
%:%2279=1046%:%
%:%2280=1046%:%
%:%2281=1047%:%
%:%2282=1048%:%
%:%2283=1049%:%
%:%2284=1049%:%
%:%2285=1050%:%
%:%2286=1050%:%
%:%2287=1051%:%
%:%2288=1051%:%
%:%2289=1051%:%
%:%2290=1052%:%
%:%2291=1052%:%
%:%2292=1053%:%
%:%2293=1053%:%
%:%2294=1054%:%
%:%2295=1054%:%
%:%2296=1054%:%
%:%2297=1055%:%
%:%2298=1055%:%
%:%2299=1056%:%
%:%2300=1057%:%
%:%2301=1058%:%
%:%2302=1058%:%
%:%2303=1059%:%
%:%2304=1059%:%
%:%2305=1059%:%
%:%2306=1060%:%
%:%2307=1060%:%
%:%2308=1061%:%
%:%2309=1061%:%
%:%2310=1062%:%
%:%2311=1062%:%
%:%2312=1063%:%
%:%2313=1063%:%
%:%2314=1064%:%
%:%2315=1064%:%
%:%2316=1065%:%
%:%2317=1065%:%
%:%2318=1066%:%
%:%2319=1066%:%
%:%2320=1067%:%
%:%2321=1068%:%
%:%2322=1068%:%
%:%2323=1069%:%
%:%2324=1069%:%
%:%2325=1070%:%
%:%2326=1070%:%
%:%2327=1071%:%
%:%2328=1071%:%
%:%2329=1072%:%
%:%2335=1072%:%
%:%2338=1073%:%
%:%2339=1074%:%
%:%2340=1074%:%
%:%2341=1075%:%
%:%2342=1076%:%
%:%2343=1077%:%
%:%2344=1078%:%
%:%2345=1079%:%
%:%2348=1080%:%
%:%2352=1080%:%
%:%2353=1080%:%
%:%2354=1081%:%
%:%2355=1081%:%
%:%2360=1081%:%
%:%2363=1082%:%
%:%2364=1083%:%
%:%2365=1083%:%
%:%2366=1084%:%
%:%2367=1085%:%
%:%2368=1086%:%
%:%2369=1087%:%
%:%2370=1088%:%
%:%2371=1089%:%
%:%2378=1090%:%
%:%2379=1090%:%
%:%2380=1091%:%
%:%2381=1091%:%
%:%2382=1092%:%
%:%2383=1092%:%
%:%2384=1092%:%
%:%2385=1093%:%
%:%2386=1093%:%
%:%2387=1094%:%
%:%2388=1095%:%
%:%2389=1095%:%
%:%2390=1096%:%
%:%2391=1096%:%
%:%2392=1097%:%
%:%2393=1097%:%
%:%2394=1098%:%
%:%2395=1098%:%
%:%2396=1099%:%
%:%2397=1099%:%
%:%2398=1099%:%
%:%2399=1100%:%
%:%2400=1100%:%
%:%2401=1101%:%
%:%2402=1102%:%
%:%2403=1102%:%
%:%2404=1103%:%
%:%2405=1103%:%
%:%2406=1103%:%
%:%2407=1104%:%
%:%2408=1104%:%
%:%2409=1105%:%
%:%2410=1105%:%
%:%2411=1106%:%
%:%2412=1106%:%
%:%2413=1106%:%
%:%2414=1107%:%
%:%2415=1107%:%
%:%2416=1108%:%
%:%2417=1108%:%
%:%2418=1109%:%
%:%2419=1109%:%
%:%2420=1109%:%
%:%2421=1110%:%
%:%2422=1110%:%
%:%2423=1111%:%
%:%2429=1111%:%
%:%2434=1112%:%
%:%2439=1113%:%
%
\begin{isabellebody}%
\setisabellecontext{Result}%
%
\isadelimdocument
\isanewline
%
\endisadelimdocument
%
\isatagdocument
\isanewline
\isanewline
%
\isamarkupsection{Electoral Result%
}
\isamarkuptrue%
%
\endisatagdocument
{\isafolddocument}%
%
\isadelimdocument
%
\endisadelimdocument
%
\isadelimtheory
%
\endisadelimtheory
%
\isatagtheory
\isacommand{theory}\isamarkupfalse%
\ Result\isanewline
\ \ \isakeyword{imports}\ Main\isanewline
\isakeyword{begin}%
\endisatagtheory
{\isafoldtheory}%
%
\isadelimtheory
%
\endisadelimtheory
%
\begin{isamarkuptext}%
An electoral result is the principal result type of the composable modules
voting framework, as it is a generalization of the set of winning alternatives
from social choice functions. Electoral results are selections of the received
(possibly empty) set of alternatives into the three disjoint groups of elected,
rejected and deferred alternatives.
Any of those sets, e.g., the set of winning (elected) alternatives, may also
be left empty, as long as they collectively still hold all the received
alternatives.%
\end{isamarkuptext}\isamarkuptrue%
%
\isadelimdocument
%
\endisadelimdocument
%
\isatagdocument
%
\isamarkupsubsection{Definition%
}
\isamarkuptrue%
%
\endisatagdocument
{\isafolddocument}%
%
\isadelimdocument
%
\endisadelimdocument
\isacommand{type{\isacharunderscore}{\kern0pt}synonym}\isamarkupfalse%
\ {\isacharprime}{\kern0pt}a\ Result\ {\isacharequal}{\kern0pt}\ {\isachardoublequoteopen}{\isacharprime}{\kern0pt}a\ set\ {\isacharasterisk}{\kern0pt}\ {\isacharprime}{\kern0pt}a\ set\ {\isacharasterisk}{\kern0pt}\ {\isacharprime}{\kern0pt}a\ set{\isachardoublequoteclose}%
\isadelimdocument
%
\endisadelimdocument
%
\isatagdocument
%
\isamarkupsubsection{Auxiliary Functions%
}
\isamarkuptrue%
%
\endisatagdocument
{\isafolddocument}%
%
\isadelimdocument
%
\endisadelimdocument
\isacommand{fun}\isamarkupfalse%
\ disjoint{\isadigit{3}}\ {\isacharcolon}{\kern0pt}{\isacharcolon}{\kern0pt}\ {\isachardoublequoteopen}{\isacharprime}{\kern0pt}a\ Result\ {\isasymRightarrow}\ bool{\isachardoublequoteclose}\ \isakeyword{where}\isanewline
\ \ {\isachardoublequoteopen}disjoint{\isadigit{3}}\ {\isacharparenleft}{\kern0pt}e{\isacharcomma}{\kern0pt}\ r{\isacharcomma}{\kern0pt}\ d{\isacharparenright}{\kern0pt}\ {\isacharequal}{\kern0pt}\isanewline
\ \ \ \ {\isacharparenleft}{\kern0pt}{\isacharparenleft}{\kern0pt}e\ {\isasyminter}\ r\ {\isacharequal}{\kern0pt}\ {\isacharbraceleft}{\kern0pt}{\isacharbraceright}{\kern0pt}{\isacharparenright}{\kern0pt}\ {\isasymand}\isanewline
\ \ \ \ \ \ {\isacharparenleft}{\kern0pt}e\ {\isasyminter}\ d\ {\isacharequal}{\kern0pt}\ {\isacharbraceleft}{\kern0pt}{\isacharbraceright}{\kern0pt}{\isacharparenright}{\kern0pt}\ {\isasymand}\isanewline
\ \ \ \ \ \ {\isacharparenleft}{\kern0pt}r\ {\isasyminter}\ d\ {\isacharequal}{\kern0pt}\ {\isacharbraceleft}{\kern0pt}{\isacharbraceright}{\kern0pt}{\isacharparenright}{\kern0pt}{\isacharparenright}{\kern0pt}{\isachardoublequoteclose}\isanewline
\isanewline
\isacommand{fun}\isamarkupfalse%
\ set{\isacharunderscore}{\kern0pt}equals{\isacharunderscore}{\kern0pt}partition\ {\isacharcolon}{\kern0pt}{\isacharcolon}{\kern0pt}\ {\isachardoublequoteopen}{\isacharprime}{\kern0pt}a\ set\ {\isasymRightarrow}{\isacharprime}{\kern0pt}a\ Result\ {\isasymRightarrow}\ bool{\isachardoublequoteclose}\ \isakeyword{where}\isanewline
\ \ {\isachardoublequoteopen}set{\isacharunderscore}{\kern0pt}equals{\isacharunderscore}{\kern0pt}partition\ A\ {\isacharparenleft}{\kern0pt}e{\isacharcomma}{\kern0pt}\ r{\isacharcomma}{\kern0pt}\ d{\isacharparenright}{\kern0pt}\ {\isacharequal}{\kern0pt}\ {\isacharparenleft}{\kern0pt}e\ {\isasymunion}\ r\ {\isasymunion}\ d\ {\isacharequal}{\kern0pt}\ A{\isacharparenright}{\kern0pt}{\isachardoublequoteclose}\isanewline
\isanewline
\isacommand{fun}\isamarkupfalse%
\ well{\isacharunderscore}{\kern0pt}formed\ {\isacharcolon}{\kern0pt}{\isacharcolon}{\kern0pt}\ {\isachardoublequoteopen}{\isacharprime}{\kern0pt}a\ set\ {\isasymRightarrow}\ {\isacharprime}{\kern0pt}a\ Result\ {\isasymRightarrow}\ bool{\isachardoublequoteclose}\ \isakeyword{where}\isanewline
\ \ {\isachardoublequoteopen}well{\isacharunderscore}{\kern0pt}formed\ A\ result\ {\isacharequal}{\kern0pt}\ {\isacharparenleft}{\kern0pt}disjoint{\isadigit{3}}\ result\ {\isasymand}\ set{\isacharunderscore}{\kern0pt}equals{\isacharunderscore}{\kern0pt}partition\ A\ result{\isacharparenright}{\kern0pt}{\isachardoublequoteclose}\isanewline
\isanewline
\isanewline
\isacommand{abbreviation}\isamarkupfalse%
\ elect{\isacharunderscore}{\kern0pt}r\ {\isacharcolon}{\kern0pt}{\isacharcolon}{\kern0pt}\ {\isachardoublequoteopen}{\isacharprime}{\kern0pt}a\ Result\ {\isasymRightarrow}\ {\isacharprime}{\kern0pt}a\ set{\isachardoublequoteclose}\ \isakeyword{where}\isanewline
\ \ {\isachardoublequoteopen}elect{\isacharunderscore}{\kern0pt}r\ r\ {\isasymequiv}\ fst\ r{\isachardoublequoteclose}\isanewline
\isanewline
\isacommand{abbreviation}\isamarkupfalse%
\ reject{\isacharunderscore}{\kern0pt}r\ {\isacharcolon}{\kern0pt}{\isacharcolon}{\kern0pt}\ {\isachardoublequoteopen}{\isacharprime}{\kern0pt}a\ Result\ {\isasymRightarrow}\ {\isacharprime}{\kern0pt}a\ set{\isachardoublequoteclose}\ \isakeyword{where}\isanewline
\ \ {\isachardoublequoteopen}reject{\isacharunderscore}{\kern0pt}r\ r\ {\isasymequiv}\ fst\ {\isacharparenleft}{\kern0pt}snd\ r{\isacharparenright}{\kern0pt}{\isachardoublequoteclose}\isanewline
\isanewline
\isacommand{abbreviation}\isamarkupfalse%
\ defer{\isacharunderscore}{\kern0pt}r\ {\isacharcolon}{\kern0pt}{\isacharcolon}{\kern0pt}\ {\isachardoublequoteopen}{\isacharprime}{\kern0pt}a\ Result\ {\isasymRightarrow}\ {\isacharprime}{\kern0pt}a\ set{\isachardoublequoteclose}\ \isakeyword{where}\isanewline
\ \ {\isachardoublequoteopen}defer{\isacharunderscore}{\kern0pt}r\ r\ {\isasymequiv}\ snd\ {\isacharparenleft}{\kern0pt}snd\ r{\isacharparenright}{\kern0pt}{\isachardoublequoteclose}%
\isadelimdocument
%
\endisadelimdocument
%
\isatagdocument
%
\isamarkupsubsection{Auxiliary Lemmata%
}
\isamarkuptrue%
%
\endisatagdocument
{\isafolddocument}%
%
\isadelimdocument
%
\endisadelimdocument
\isacommand{lemma}\isamarkupfalse%
\ result{\isacharunderscore}{\kern0pt}imp{\isacharunderscore}{\kern0pt}rej{\isacharcolon}{\kern0pt}\isanewline
\ \ \isakeyword{assumes}\ {\isachardoublequoteopen}well{\isacharunderscore}{\kern0pt}formed\ A\ {\isacharparenleft}{\kern0pt}e{\isacharcomma}{\kern0pt}\ r{\isacharcomma}{\kern0pt}\ d{\isacharparenright}{\kern0pt}{\isachardoublequoteclose}\isanewline
\ \ \isakeyword{shows}\ {\isachardoublequoteopen}A\ {\isacharminus}{\kern0pt}\ {\isacharparenleft}{\kern0pt}e\ {\isasymunion}\ d{\isacharparenright}{\kern0pt}\ {\isacharequal}{\kern0pt}\ r{\isachardoublequoteclose}\isanewline
%
\isadelimproof
%
\endisadelimproof
%
\isatagproof
\isacommand{proof}\isamarkupfalse%
\ {\isacharparenleft}{\kern0pt}safe{\isacharparenright}{\kern0pt}\isanewline
\ \ \isacommand{fix}\isamarkupfalse%
\isanewline
\ \ \ \ x\ {\isacharcolon}{\kern0pt}{\isacharcolon}{\kern0pt}\ {\isachardoublequoteopen}{\isacharprime}{\kern0pt}a{\isachardoublequoteclose}\isanewline
\ \ \isacommand{assume}\isamarkupfalse%
\isanewline
\ \ \ \ x{\isacharunderscore}{\kern0pt}in{\isacharunderscore}{\kern0pt}A{\isacharcolon}{\kern0pt}\ {\isachardoublequoteopen}x\ {\isasymin}\ A{\isachardoublequoteclose}\ \isakeyword{and}\isanewline
\ \ \ \ x{\isacharunderscore}{\kern0pt}not{\isacharunderscore}{\kern0pt}rej{\isacharcolon}{\kern0pt}\ \ \ {\isachardoublequoteopen}x\ {\isasymnotin}\ r{\isachardoublequoteclose}\ \isakeyword{and}\isanewline
\ \ \ \ x{\isacharunderscore}{\kern0pt}not{\isacharunderscore}{\kern0pt}def{\isacharcolon}{\kern0pt}\ \ \ {\isachardoublequoteopen}x\ {\isasymnotin}\ d{\isachardoublequoteclose}\isanewline
\ \ \isacommand{from}\isamarkupfalse%
\ assms\ \isacommand{have}\isamarkupfalse%
\isanewline
\ \ \ \ {\isachardoublequoteopen}{\isacharparenleft}{\kern0pt}e\ {\isasyminter}\ r\ {\isacharequal}{\kern0pt}\ {\isacharbraceleft}{\kern0pt}{\isacharbraceright}{\kern0pt}{\isacharparenright}{\kern0pt}\ {\isasymand}\ {\isacharparenleft}{\kern0pt}e\ {\isasyminter}\ d\ {\isacharequal}{\kern0pt}\ {\isacharbraceleft}{\kern0pt}{\isacharbraceright}{\kern0pt}{\isacharparenright}{\kern0pt}\ {\isasymand}\isanewline
\ \ \ \ {\isacharparenleft}{\kern0pt}r\ {\isasyminter}\ d\ {\isacharequal}{\kern0pt}\ {\isacharbraceleft}{\kern0pt}{\isacharbraceright}{\kern0pt}{\isacharparenright}{\kern0pt}\ {\isasymand}\ {\isacharparenleft}{\kern0pt}e\ {\isasymunion}\ r\ {\isasymunion}\ d\ {\isacharequal}{\kern0pt}\ A{\isacharparenright}{\kern0pt}{\isachardoublequoteclose}\isanewline
\ \ \ \ \isacommand{by}\isamarkupfalse%
\ simp\isanewline
\ \ \isacommand{thus}\isamarkupfalse%
\ {\isachardoublequoteopen}x\ {\isasymin}\ e{\isachardoublequoteclose}\isanewline
\ \ \ \ \isacommand{using}\isamarkupfalse%
\ x{\isacharunderscore}{\kern0pt}in{\isacharunderscore}{\kern0pt}A\ x{\isacharunderscore}{\kern0pt}not{\isacharunderscore}{\kern0pt}rej\ x{\isacharunderscore}{\kern0pt}not{\isacharunderscore}{\kern0pt}def\isanewline
\ \ \ \ \isacommand{by}\isamarkupfalse%
\ auto\isanewline
\isacommand{next}\isamarkupfalse%
\isanewline
\ \ \isacommand{fix}\isamarkupfalse%
\isanewline
\ \ \ \ x\ {\isacharcolon}{\kern0pt}{\isacharcolon}{\kern0pt}\ {\isachardoublequoteopen}{\isacharprime}{\kern0pt}a{\isachardoublequoteclose}\isanewline
\ \ \isacommand{assume}\isamarkupfalse%
\isanewline
\ \ \ \ x{\isacharunderscore}{\kern0pt}rej{\isacharcolon}{\kern0pt}\ \ \ {\isachardoublequoteopen}x\ {\isasymin}\ r{\isachardoublequoteclose}\isanewline
\ \ \isacommand{from}\isamarkupfalse%
\ assms\ \isacommand{have}\isamarkupfalse%
\isanewline
\ \ \ \ {\isachardoublequoteopen}{\isacharparenleft}{\kern0pt}e\ {\isasyminter}\ r\ {\isacharequal}{\kern0pt}\ {\isacharbraceleft}{\kern0pt}{\isacharbraceright}{\kern0pt}{\isacharparenright}{\kern0pt}\ {\isasymand}\ {\isacharparenleft}{\kern0pt}e\ {\isasyminter}\ d\ {\isacharequal}{\kern0pt}\ {\isacharbraceleft}{\kern0pt}{\isacharbraceright}{\kern0pt}{\isacharparenright}{\kern0pt}\ {\isasymand}\isanewline
\ \ \ \ {\isacharparenleft}{\kern0pt}r\ {\isasyminter}\ d\ {\isacharequal}{\kern0pt}\ {\isacharbraceleft}{\kern0pt}{\isacharbraceright}{\kern0pt}{\isacharparenright}{\kern0pt}\ {\isasymand}\ {\isacharparenleft}{\kern0pt}e\ {\isasymunion}\ r\ {\isasymunion}\ d\ {\isacharequal}{\kern0pt}\ A{\isacharparenright}{\kern0pt}{\isachardoublequoteclose}\isanewline
\ \ \ \ \isacommand{by}\isamarkupfalse%
\ simp\isanewline
\ \ \isacommand{thus}\isamarkupfalse%
\ {\isachardoublequoteopen}x\ {\isasymin}\ A{\isachardoublequoteclose}\isanewline
\ \ \ \ \isacommand{using}\isamarkupfalse%
\ x{\isacharunderscore}{\kern0pt}rej\isanewline
\ \ \ \ \isacommand{by}\isamarkupfalse%
\ auto\isanewline
\isacommand{next}\isamarkupfalse%
\isanewline
\ \ \isacommand{fix}\isamarkupfalse%
\isanewline
\ \ \ \ x\ {\isacharcolon}{\kern0pt}{\isacharcolon}{\kern0pt}\ {\isachardoublequoteopen}{\isacharprime}{\kern0pt}a{\isachardoublequoteclose}\isanewline
\ \ \isacommand{assume}\isamarkupfalse%
\isanewline
\ \ \ \ x{\isacharunderscore}{\kern0pt}rej{\isacharcolon}{\kern0pt}\ \ {\isachardoublequoteopen}x\ {\isasymin}\ r{\isachardoublequoteclose}\ \isakeyword{and}\isanewline
\ \ \ \ x{\isacharunderscore}{\kern0pt}elec{\isacharcolon}{\kern0pt}\ {\isachardoublequoteopen}x\ {\isasymin}\ e{\isachardoublequoteclose}\isanewline
\ \ \isacommand{from}\isamarkupfalse%
\ assms\ \isacommand{have}\isamarkupfalse%
\isanewline
\ \ \ \ {\isachardoublequoteopen}{\isacharparenleft}{\kern0pt}e\ {\isasyminter}\ r\ {\isacharequal}{\kern0pt}\ {\isacharbraceleft}{\kern0pt}{\isacharbraceright}{\kern0pt}{\isacharparenright}{\kern0pt}\ {\isasymand}\ {\isacharparenleft}{\kern0pt}e\ {\isasyminter}\ d\ {\isacharequal}{\kern0pt}\ {\isacharbraceleft}{\kern0pt}{\isacharbraceright}{\kern0pt}{\isacharparenright}{\kern0pt}\ {\isasymand}\isanewline
\ \ \ \ {\isacharparenleft}{\kern0pt}r\ {\isasyminter}\ d\ {\isacharequal}{\kern0pt}\ {\isacharbraceleft}{\kern0pt}{\isacharbraceright}{\kern0pt}{\isacharparenright}{\kern0pt}\ {\isasymand}\ {\isacharparenleft}{\kern0pt}e\ {\isasymunion}\ r\ {\isasymunion}\ d\ {\isacharequal}{\kern0pt}\ A{\isacharparenright}{\kern0pt}{\isachardoublequoteclose}\isanewline
\ \ \ \ \isacommand{by}\isamarkupfalse%
\ simp\isanewline
\ \ \isacommand{thus}\isamarkupfalse%
\ {\isachardoublequoteopen}False{\isachardoublequoteclose}\isanewline
\ \ \ \ \isacommand{using}\isamarkupfalse%
\ x{\isacharunderscore}{\kern0pt}rej\ x{\isacharunderscore}{\kern0pt}elec\isanewline
\ \ \ \ \isacommand{by}\isamarkupfalse%
\ auto\isanewline
\isacommand{next}\isamarkupfalse%
\isanewline
\ \ \isacommand{fix}\isamarkupfalse%
\isanewline
\ \ \ \ x\ {\isacharcolon}{\kern0pt}{\isacharcolon}{\kern0pt}\ {\isachardoublequoteopen}{\isacharprime}{\kern0pt}a{\isachardoublequoteclose}\isanewline
\ \ \isacommand{assume}\isamarkupfalse%
\isanewline
\ \ \ \ x{\isacharunderscore}{\kern0pt}rej{\isacharcolon}{\kern0pt}\ {\isachardoublequoteopen}x\ {\isasymin}\ r{\isachardoublequoteclose}\ \isakeyword{and}\isanewline
\ \ \ \ x{\isacharunderscore}{\kern0pt}def{\isacharcolon}{\kern0pt}\ {\isachardoublequoteopen}x\ {\isasymin}\ d{\isachardoublequoteclose}\isanewline
\ \ \isacommand{from}\isamarkupfalse%
\ assms\ \isacommand{have}\isamarkupfalse%
\isanewline
\ \ \ \ {\isachardoublequoteopen}{\isacharparenleft}{\kern0pt}e\ {\isasyminter}\ r\ {\isacharequal}{\kern0pt}\ {\isacharbraceleft}{\kern0pt}{\isacharbraceright}{\kern0pt}{\isacharparenright}{\kern0pt}\ {\isasymand}\ {\isacharparenleft}{\kern0pt}e\ {\isasyminter}\ d\ {\isacharequal}{\kern0pt}\ {\isacharbraceleft}{\kern0pt}{\isacharbraceright}{\kern0pt}{\isacharparenright}{\kern0pt}\ {\isasymand}\isanewline
\ \ \ \ {\isacharparenleft}{\kern0pt}r\ {\isasyminter}\ d\ {\isacharequal}{\kern0pt}\ {\isacharbraceleft}{\kern0pt}{\isacharbraceright}{\kern0pt}{\isacharparenright}{\kern0pt}\ {\isasymand}\ {\isacharparenleft}{\kern0pt}e\ {\isasymunion}\ r\ {\isasymunion}\ d\ {\isacharequal}{\kern0pt}\ A{\isacharparenright}{\kern0pt}{\isachardoublequoteclose}\isanewline
\ \ \ \ \isacommand{by}\isamarkupfalse%
\ simp\isanewline
\ \ \isacommand{thus}\isamarkupfalse%
\ {\isachardoublequoteopen}False{\isachardoublequoteclose}\isanewline
\ \ \ \ \isacommand{using}\isamarkupfalse%
\ x{\isacharunderscore}{\kern0pt}rej\ x{\isacharunderscore}{\kern0pt}def\isanewline
\ \ \ \ \isacommand{by}\isamarkupfalse%
\ auto\isanewline
\isacommand{qed}\isamarkupfalse%
%
\endisatagproof
{\isafoldproof}%
%
\isadelimproof
\isanewline
%
\endisadelimproof
\isanewline
\isacommand{lemma}\isamarkupfalse%
\ result{\isacharunderscore}{\kern0pt}count{\isacharcolon}{\kern0pt}\isanewline
\ \ \isakeyword{assumes}\isanewline
\ \ \ \ {\isachardoublequoteopen}well{\isacharunderscore}{\kern0pt}formed\ A\ {\isacharparenleft}{\kern0pt}e{\isacharcomma}{\kern0pt}\ r{\isacharcomma}{\kern0pt}\ d{\isacharparenright}{\kern0pt}{\isachardoublequoteclose}\ \isakeyword{and}\isanewline
\ \ \ \ {\isachardoublequoteopen}finite\ A{\isachardoublequoteclose}\isanewline
\ \ \isakeyword{shows}\ {\isachardoublequoteopen}card\ A\ {\isacharequal}{\kern0pt}\ card\ e\ {\isacharplus}{\kern0pt}\ card\ r\ {\isacharplus}{\kern0pt}\ card\ d{\isachardoublequoteclose}\isanewline
%
\isadelimproof
%
\endisadelimproof
%
\isatagproof
\isacommand{proof}\isamarkupfalse%
\ {\isacharminus}{\kern0pt}\isanewline
\ \ \isacommand{from}\isamarkupfalse%
\ assms{\isacharparenleft}{\kern0pt}{\isadigit{1}}{\isacharparenright}{\kern0pt}\ \isacommand{have}\isamarkupfalse%
\ disj{\isacharcolon}{\kern0pt}\isanewline
\ \ \ \ {\isachardoublequoteopen}{\isacharparenleft}{\kern0pt}e\ {\isasyminter}\ r\ {\isacharequal}{\kern0pt}\ {\isacharbraceleft}{\kern0pt}{\isacharbraceright}{\kern0pt}{\isacharparenright}{\kern0pt}\ {\isasymand}\ {\isacharparenleft}{\kern0pt}e\ {\isasyminter}\ d\ {\isacharequal}{\kern0pt}\ {\isacharbraceleft}{\kern0pt}{\isacharbraceright}{\kern0pt}{\isacharparenright}{\kern0pt}\ {\isasymand}\ {\isacharparenleft}{\kern0pt}r\ {\isasyminter}\ d\ {\isacharequal}{\kern0pt}\ {\isacharbraceleft}{\kern0pt}{\isacharbraceright}{\kern0pt}{\isacharparenright}{\kern0pt}{\isachardoublequoteclose}\isanewline
\ \ \ \ \isacommand{by}\isamarkupfalse%
\ simp\isanewline
\ \ \isacommand{from}\isamarkupfalse%
\ assms{\isacharparenleft}{\kern0pt}{\isadigit{1}}{\isacharparenright}{\kern0pt}\ \isacommand{have}\isamarkupfalse%
\ set{\isacharunderscore}{\kern0pt}partit{\isacharcolon}{\kern0pt}\isanewline
\ \ \ \ {\isachardoublequoteopen}e\ {\isasymunion}\ r\ {\isasymunion}\ d\ {\isacharequal}{\kern0pt}\ A{\isachardoublequoteclose}\isanewline
\ \ \ \ \isacommand{by}\isamarkupfalse%
\ simp\isanewline
\ \ \isacommand{show}\isamarkupfalse%
\ {\isacharquery}{\kern0pt}thesis\isanewline
\ \ \ \ \isacommand{using}\isamarkupfalse%
\ assms\ disj\ set{\isacharunderscore}{\kern0pt}partit\ Int{\isacharunderscore}{\kern0pt}Un{\isacharunderscore}{\kern0pt}distrib{\isadigit{2}}\ finite{\isacharunderscore}{\kern0pt}Un\isanewline
\ \ \ \ \ \ \ \ \ \ card{\isacharunderscore}{\kern0pt}Un{\isacharunderscore}{\kern0pt}disjoint\ sup{\isacharunderscore}{\kern0pt}bot{\isachardot}{\kern0pt}right{\isacharunderscore}{\kern0pt}neutral\isanewline
\ \ \ \ \isacommand{by}\isamarkupfalse%
\ metis\isanewline
\isacommand{qed}\isamarkupfalse%
%
\endisatagproof
{\isafoldproof}%
%
\isadelimproof
\isanewline
%
\endisadelimproof
%
\isadelimtheory
\isanewline
%
\endisadelimtheory
%
\isatagtheory
\isacommand{end}\isamarkupfalse%
%
\endisatagtheory
{\isafoldtheory}%
%
\isadelimtheory
%
\endisadelimtheory
%
\end{isabellebody}%
\endinput
%:%file=~/Documents/Studies/VotingRuleGenerator/virage/src/test/resources/old_theories/Compositional_Structures/Basic_Modules/Component_Types/Social_Choice_Types/Result.thy%:%
%:%6=3%:%
%:%11=4%:%
%:%12=5%:%
%:%14=8%:%
%:%30=10%:%
%:%31=10%:%
%:%32=11%:%
%:%33=12%:%
%:%42=15%:%
%:%43=16%:%
%:%44=17%:%
%:%45=18%:%
%:%46=19%:%
%:%47=20%:%
%:%48=21%:%
%:%49=22%:%
%:%58=24%:%
%:%68=30%:%
%:%69=30%:%
%:%76=32%:%
%:%86=38%:%
%:%87=38%:%
%:%88=39%:%
%:%91=42%:%
%:%92=43%:%
%:%93=44%:%
%:%94=44%:%
%:%95=45%:%
%:%96=46%:%
%:%97=47%:%
%:%98=47%:%
%:%99=48%:%
%:%100=49%:%
%:%101=50%:%
%:%102=51%:%
%:%103=51%:%
%:%104=52%:%
%:%105=53%:%
%:%106=54%:%
%:%107=54%:%
%:%108=55%:%
%:%109=56%:%
%:%110=57%:%
%:%111=57%:%
%:%112=58%:%
%:%119=60%:%
%:%129=62%:%
%:%130=62%:%
%:%131=63%:%
%:%132=64%:%
%:%139=65%:%
%:%140=65%:%
%:%141=66%:%
%:%142=66%:%
%:%143=67%:%
%:%144=68%:%
%:%145=68%:%
%:%146=69%:%
%:%147=70%:%
%:%148=71%:%
%:%149=72%:%
%:%150=72%:%
%:%151=72%:%
%:%152=73%:%
%:%153=74%:%
%:%154=75%:%
%:%155=75%:%
%:%156=76%:%
%:%157=76%:%
%:%158=77%:%
%:%159=77%:%
%:%160=78%:%
%:%161=78%:%
%:%162=79%:%
%:%163=79%:%
%:%164=80%:%
%:%165=80%:%
%:%166=81%:%
%:%167=82%:%
%:%168=82%:%
%:%169=83%:%
%:%170=84%:%
%:%171=84%:%
%:%172=84%:%
%:%173=85%:%
%:%174=86%:%
%:%175=87%:%
%:%176=87%:%
%:%177=88%:%
%:%178=88%:%
%:%179=89%:%
%:%180=89%:%
%:%181=90%:%
%:%182=90%:%
%:%183=91%:%
%:%184=91%:%
%:%185=92%:%
%:%186=92%:%
%:%187=93%:%
%:%188=94%:%
%:%189=94%:%
%:%190=95%:%
%:%191=96%:%
%:%192=97%:%
%:%193=97%:%
%:%194=97%:%
%:%195=98%:%
%:%196=99%:%
%:%197=100%:%
%:%198=100%:%
%:%199=101%:%
%:%200=101%:%
%:%201=102%:%
%:%202=102%:%
%:%203=103%:%
%:%204=103%:%
%:%205=104%:%
%:%206=104%:%
%:%207=105%:%
%:%208=105%:%
%:%209=106%:%
%:%210=107%:%
%:%211=107%:%
%:%212=108%:%
%:%213=109%:%
%:%214=110%:%
%:%215=110%:%
%:%216=110%:%
%:%217=111%:%
%:%218=112%:%
%:%219=113%:%
%:%220=113%:%
%:%221=114%:%
%:%222=114%:%
%:%223=115%:%
%:%224=115%:%
%:%225=116%:%
%:%226=116%:%
%:%227=117%:%
%:%233=117%:%
%:%236=118%:%
%:%237=119%:%
%:%238=119%:%
%:%239=120%:%
%:%240=121%:%
%:%241=122%:%
%:%242=123%:%
%:%249=124%:%
%:%250=124%:%
%:%251=125%:%
%:%252=125%:%
%:%253=125%:%
%:%254=126%:%
%:%255=127%:%
%:%256=127%:%
%:%257=128%:%
%:%258=128%:%
%:%259=128%:%
%:%260=129%:%
%:%261=130%:%
%:%262=130%:%
%:%263=131%:%
%:%264=131%:%
%:%265=132%:%
%:%266=132%:%
%:%267=133%:%
%:%268=134%:%
%:%269=134%:%
%:%270=135%:%
%:%276=135%:%
%:%281=136%:%
%:%286=137%:%
%
\begin{isabellebody}%
\setisabellecontext{Profile}%
%
\isadelimdocument
\isanewline
%
\endisadelimdocument
%
\isatagdocument
\isanewline
\isanewline
\isanewline
%
\isamarkupsection{Preference Profile%
}
\isamarkuptrue%
%
\endisatagdocument
{\isafolddocument}%
%
\isadelimdocument
%
\endisadelimdocument
%
\isadelimtheory
%
\endisadelimtheory
%
\isatagtheory
\isacommand{theory}\isamarkupfalse%
\ Profile\isanewline
\ \ \isakeyword{imports}\ Preference{\isacharunderscore}{\kern0pt}Relation\isanewline
\isakeyword{begin}%
\endisatagtheory
{\isafoldtheory}%
%
\isadelimtheory
%
\endisadelimtheory
%
\begin{isamarkuptext}%
Preference profiles denote the decisions made by the individual voters on
the eligible alternatives. They are represented in the form of one preference
relation (e.g., selected on a ballot) per voter, collectively captured in a
list of such preference relations.
Unlike a the common preference profiles in the social-choice sense, the
profiles described here considers only the (sub-)set of alternatives that are
received.%
\end{isamarkuptext}\isamarkuptrue%
%
\isadelimdocument
%
\endisadelimdocument
%
\isatagdocument
%
\isamarkupsubsection{Definition%
}
\isamarkuptrue%
%
\endisatagdocument
{\isafolddocument}%
%
\isadelimdocument
%
\endisadelimdocument
\isacommand{type{\isacharunderscore}{\kern0pt}synonym}\isamarkupfalse%
\ {\isacharprime}{\kern0pt}a\ Profile\ {\isacharequal}{\kern0pt}\ {\isachardoublequoteopen}{\isacharparenleft}{\kern0pt}{\isacharprime}{\kern0pt}a\ Preference{\isacharunderscore}{\kern0pt}Relation{\isacharparenright}{\kern0pt}\ list{\isachardoublequoteclose}\isanewline
\isanewline
\isanewline
\isacommand{definition}\isamarkupfalse%
\ profile\ {\isacharcolon}{\kern0pt}{\isacharcolon}{\kern0pt}\ {\isachardoublequoteopen}{\isacharprime}{\kern0pt}a\ set\ {\isasymRightarrow}\ {\isacharprime}{\kern0pt}a\ Profile\ {\isasymRightarrow}\ bool{\isachardoublequoteclose}\ \isakeyword{where}\isanewline
\ \ {\isachardoublequoteopen}profile\ A\ p\ {\isasymequiv}\ {\isasymforall}i{\isacharcolon}{\kern0pt}{\isacharcolon}{\kern0pt}nat{\isachardot}{\kern0pt}\ i\ {\isacharless}{\kern0pt}\ size\ p\ {\isasymlongrightarrow}\ linear{\isacharunderscore}{\kern0pt}order{\isacharunderscore}{\kern0pt}on\ A\ {\isacharparenleft}{\kern0pt}p{\isacharbang}{\kern0pt}i{\isacharparenright}{\kern0pt}{\isachardoublequoteclose}\isanewline
\isanewline
\isacommand{lemma}\isamarkupfalse%
\ profile{\isacharunderscore}{\kern0pt}set\ {\isacharcolon}{\kern0pt}\ {\isachardoublequoteopen}profile\ A\ p\ {\isasymequiv}\ {\isacharparenleft}{\kern0pt}{\isasymforall}b\ {\isasymin}\ {\isacharparenleft}{\kern0pt}set\ p{\isacharparenright}{\kern0pt}{\isachardot}{\kern0pt}\ linear{\isacharunderscore}{\kern0pt}order{\isacharunderscore}{\kern0pt}on\ A\ b{\isacharparenright}{\kern0pt}{\isachardoublequoteclose}\isanewline
%
\isadelimproof
\ \ %
\endisadelimproof
%
\isatagproof
\isacommand{by}\isamarkupfalse%
\ {\isacharparenleft}{\kern0pt}simp\ add{\isacharcolon}{\kern0pt}\ all{\isacharunderscore}{\kern0pt}set{\isacharunderscore}{\kern0pt}conv{\isacharunderscore}{\kern0pt}all{\isacharunderscore}{\kern0pt}nth\ profile{\isacharunderscore}{\kern0pt}def{\isacharparenright}{\kern0pt}%
\endisatagproof
{\isafoldproof}%
%
\isadelimproof
\isanewline
%
\endisadelimproof
\isanewline
\isacommand{abbreviation}\isamarkupfalse%
\ finite{\isacharunderscore}{\kern0pt}profile\ {\isacharcolon}{\kern0pt}{\isacharcolon}{\kern0pt}\ {\isachardoublequoteopen}{\isacharprime}{\kern0pt}a\ set\ {\isasymRightarrow}\ {\isacharprime}{\kern0pt}a\ Profile\ {\isasymRightarrow}\ bool{\isachardoublequoteclose}\ \isakeyword{where}\isanewline
\ \ {\isachardoublequoteopen}finite{\isacharunderscore}{\kern0pt}profile\ A\ p\ {\isasymequiv}\ finite\ A\ {\isasymand}\ profile\ A\ p{\isachardoublequoteclose}%
\isadelimdocument
%
\endisadelimdocument
%
\isatagdocument
%
\isamarkupsubsection{Preference Counts and Comparisons%
}
\isamarkuptrue%
%
\endisatagdocument
{\isafolddocument}%
%
\isadelimdocument
%
\endisadelimdocument
\isacommand{fun}\isamarkupfalse%
\ win{\isacharunderscore}{\kern0pt}count\ {\isacharcolon}{\kern0pt}{\isacharcolon}{\kern0pt}\ {\isachardoublequoteopen}{\isacharprime}{\kern0pt}a\ Profile\ {\isasymRightarrow}\ {\isacharprime}{\kern0pt}a\ {\isasymRightarrow}\ nat{\isachardoublequoteclose}\ \isakeyword{where}\isanewline
\ \ {\isachardoublequoteopen}win{\isacharunderscore}{\kern0pt}count\ p\ a\ {\isacharequal}{\kern0pt}\isanewline
\ \ \ \ card\ {\isacharbraceleft}{\kern0pt}i{\isacharcolon}{\kern0pt}{\isacharcolon}{\kern0pt}nat{\isachardot}{\kern0pt}\ i\ {\isacharless}{\kern0pt}\ size\ p\ {\isasymand}\ above\ {\isacharparenleft}{\kern0pt}p{\isacharbang}{\kern0pt}i{\isacharparenright}{\kern0pt}\ a\ {\isacharequal}{\kern0pt}\ {\isacharbraceleft}{\kern0pt}a{\isacharbraceright}{\kern0pt}{\isacharbraceright}{\kern0pt}{\isachardoublequoteclose}\isanewline
\isanewline
\isacommand{fun}\isamarkupfalse%
\ win{\isacharunderscore}{\kern0pt}count{\isacharunderscore}{\kern0pt}code\ {\isacharcolon}{\kern0pt}{\isacharcolon}{\kern0pt}\ {\isachardoublequoteopen}{\isacharprime}{\kern0pt}a\ Profile\ {\isasymRightarrow}\ {\isacharprime}{\kern0pt}a\ {\isasymRightarrow}\ nat{\isachardoublequoteclose}\ \isakeyword{where}\isanewline
\ \ {\isachardoublequoteopen}win{\isacharunderscore}{\kern0pt}count{\isacharunderscore}{\kern0pt}code\ Nil\ a\ {\isacharequal}{\kern0pt}\ {\isadigit{0}}{\isachardoublequoteclose}\ {\isacharbar}{\kern0pt}\isanewline
\ \ {\isachardoublequoteopen}win{\isacharunderscore}{\kern0pt}count{\isacharunderscore}{\kern0pt}code\ {\isacharparenleft}{\kern0pt}p{\isacharhash}{\kern0pt}ps{\isacharparenright}{\kern0pt}\ a\ {\isacharequal}{\kern0pt}\isanewline
\ \ \ \ \ \ {\isacharparenleft}{\kern0pt}if\ {\isacharparenleft}{\kern0pt}above\ p\ a\ {\isacharequal}{\kern0pt}\ {\isacharbraceleft}{\kern0pt}a{\isacharbraceright}{\kern0pt}{\isacharparenright}{\kern0pt}\ then\ {\isadigit{1}}\ else\ {\isadigit{0}}{\isacharparenright}{\kern0pt}\ {\isacharplus}{\kern0pt}\ win{\isacharunderscore}{\kern0pt}count{\isacharunderscore}{\kern0pt}code\ ps\ a{\isachardoublequoteclose}\isanewline
\isanewline
\isacommand{fun}\isamarkupfalse%
\ prefer{\isacharunderscore}{\kern0pt}count\ {\isacharcolon}{\kern0pt}{\isacharcolon}{\kern0pt}\ {\isachardoublequoteopen}{\isacharprime}{\kern0pt}a\ Profile\ {\isasymRightarrow}\ {\isacharprime}{\kern0pt}a\ {\isasymRightarrow}\ {\isacharprime}{\kern0pt}a\ {\isasymRightarrow}\ nat{\isachardoublequoteclose}\ \isakeyword{where}\isanewline
\ \ {\isachardoublequoteopen}prefer{\isacharunderscore}{\kern0pt}count\ p\ x\ y\ {\isacharequal}{\kern0pt}\isanewline
\ \ \ \ \ \ card\ {\isacharbraceleft}{\kern0pt}i{\isacharcolon}{\kern0pt}{\isacharcolon}{\kern0pt}nat{\isachardot}{\kern0pt}\ i\ {\isacharless}{\kern0pt}\ size\ p\ {\isasymand}\ {\isacharparenleft}{\kern0pt}let\ r\ {\isacharequal}{\kern0pt}\ {\isacharparenleft}{\kern0pt}p{\isacharbang}{\kern0pt}i{\isacharparenright}{\kern0pt}\ in\ {\isacharparenleft}{\kern0pt}y\ {\isasympreceq}\isactrlsub r\ x{\isacharparenright}{\kern0pt}{\isacharparenright}{\kern0pt}{\isacharbraceright}{\kern0pt}{\isachardoublequoteclose}\isanewline
\isanewline
\isacommand{fun}\isamarkupfalse%
\ prefer{\isacharunderscore}{\kern0pt}count{\isacharunderscore}{\kern0pt}code\ {\isacharcolon}{\kern0pt}{\isacharcolon}{\kern0pt}\ {\isachardoublequoteopen}{\isacharprime}{\kern0pt}a\ Profile\ {\isasymRightarrow}\ {\isacharprime}{\kern0pt}a\ {\isasymRightarrow}\ {\isacharprime}{\kern0pt}a\ {\isasymRightarrow}\ nat{\isachardoublequoteclose}\ \isakeyword{where}\isanewline
\ \ {\isachardoublequoteopen}prefer{\isacharunderscore}{\kern0pt}count{\isacharunderscore}{\kern0pt}code\ Nil\ x\ y\ {\isacharequal}{\kern0pt}\ {\isadigit{0}}{\isachardoublequoteclose}\ {\isacharbar}{\kern0pt}\isanewline
\ \ {\isachardoublequoteopen}prefer{\isacharunderscore}{\kern0pt}count{\isacharunderscore}{\kern0pt}code\ {\isacharparenleft}{\kern0pt}p{\isacharhash}{\kern0pt}ps{\isacharparenright}{\kern0pt}\ x\ y\ {\isacharequal}{\kern0pt}\isanewline
\ \ \ \ \ \ {\isacharparenleft}{\kern0pt}if\ y\ {\isasympreceq}\isactrlsub p\ x\ then\ {\isadigit{1}}\ else\ {\isadigit{0}}{\isacharparenright}{\kern0pt}\ {\isacharplus}{\kern0pt}\ prefer{\isacharunderscore}{\kern0pt}count{\isacharunderscore}{\kern0pt}code\ ps\ x\ y{\isachardoublequoteclose}\isanewline
\isanewline
\isacommand{lemma}\isamarkupfalse%
\ set{\isacharunderscore}{\kern0pt}compr{\isacharcolon}{\kern0pt}\ {\isachardoublequoteopen}{\isacharbraceleft}{\kern0pt}\ f\ x\ {\isacharbar}{\kern0pt}\ x\ {\isachardot}{\kern0pt}\ x\ {\isasymin}\ S\ {\isacharbraceright}{\kern0pt}\ {\isacharequal}{\kern0pt}\ f\ {\isacharbackquote}{\kern0pt}\ S{\isachardoublequoteclose}\isanewline
%
\isadelimproof
\ \ %
\endisadelimproof
%
\isatagproof
\isacommand{by}\isamarkupfalse%
\ auto%
\endisatagproof
{\isafoldproof}%
%
\isadelimproof
\isanewline
%
\endisadelimproof
\isanewline
\isacommand{lemma}\isamarkupfalse%
\ pref{\isacharunderscore}{\kern0pt}count{\isacharunderscore}{\kern0pt}set{\isacharunderscore}{\kern0pt}compr{\isacharcolon}{\kern0pt}\ {\isachardoublequoteopen}{\isacharbraceleft}{\kern0pt}prefer{\isacharunderscore}{\kern0pt}count\ p\ x\ y\ {\isacharbar}{\kern0pt}\ y\ {\isachardot}{\kern0pt}\ y\ {\isasymin}\ A{\isacharminus}{\kern0pt}{\isacharbraceleft}{\kern0pt}x{\isacharbraceright}{\kern0pt}{\isacharbraceright}{\kern0pt}\ {\isacharequal}{\kern0pt}\isanewline
\ \ \ \ \ \ \ \ \ \ {\isacharparenleft}{\kern0pt}prefer{\isacharunderscore}{\kern0pt}count\ p\ x{\isacharparenright}{\kern0pt}\ {\isacharbackquote}{\kern0pt}\ {\isacharparenleft}{\kern0pt}A{\isacharminus}{\kern0pt}{\isacharbraceleft}{\kern0pt}x{\isacharbraceright}{\kern0pt}{\isacharparenright}{\kern0pt}{\isachardoublequoteclose}\isanewline
%
\isadelimproof
\ \ %
\endisadelimproof
%
\isatagproof
\isacommand{by}\isamarkupfalse%
\ auto%
\endisatagproof
{\isafoldproof}%
%
\isadelimproof
\isanewline
%
\endisadelimproof
\isanewline
\isacommand{lemma}\isamarkupfalse%
\ pref{\isacharunderscore}{\kern0pt}count{\isacharcolon}{\kern0pt}\isanewline
\ \ \isakeyword{assumes}\ prof{\isacharcolon}{\kern0pt}\ {\isachardoublequoteopen}profile\ A\ p{\isachardoublequoteclose}\isanewline
\ \ \isakeyword{assumes}\ x{\isacharunderscore}{\kern0pt}in{\isacharunderscore}{\kern0pt}A{\isacharcolon}{\kern0pt}\ {\isachardoublequoteopen}x\ {\isasymin}\ A{\isachardoublequoteclose}\isanewline
\ \ \isakeyword{assumes}\ y{\isacharunderscore}{\kern0pt}in{\isacharunderscore}{\kern0pt}A{\isacharcolon}{\kern0pt}\ {\isachardoublequoteopen}y\ {\isasymin}\ A{\isachardoublequoteclose}\isanewline
\ \ \isakeyword{assumes}\ neq{\isacharcolon}{\kern0pt}\ {\isachardoublequoteopen}x\ {\isasymnoteq}\ y{\isachardoublequoteclose}\isanewline
\ \ \isakeyword{shows}\ {\isachardoublequoteopen}prefer{\isacharunderscore}{\kern0pt}count\ p\ x\ y\ {\isacharequal}{\kern0pt}\ {\isacharparenleft}{\kern0pt}size\ p{\isacharparenright}{\kern0pt}\ {\isacharminus}{\kern0pt}\ {\isacharparenleft}{\kern0pt}prefer{\isacharunderscore}{\kern0pt}count\ p\ y\ x{\isacharparenright}{\kern0pt}{\isachardoublequoteclose}\isanewline
%
\isadelimproof
%
\endisadelimproof
%
\isatagproof
\isacommand{proof}\isamarkupfalse%
\ {\isacharminus}{\kern0pt}\isanewline
\ \ \isacommand{have}\isamarkupfalse%
\ {\isadigit{0}}{\isadigit{0}}{\isacharcolon}{\kern0pt}\ {\isachardoublequoteopen}card\ {\isacharbraceleft}{\kern0pt}i{\isacharcolon}{\kern0pt}{\isacharcolon}{\kern0pt}nat{\isachardot}{\kern0pt}\ i\ {\isacharless}{\kern0pt}\ size\ p{\isacharbraceright}{\kern0pt}\ {\isacharequal}{\kern0pt}\ size\ p{\isachardoublequoteclose}\isanewline
\ \ \ \ \isacommand{by}\isamarkupfalse%
\ simp\isanewline
\ \ \isacommand{have}\isamarkupfalse%
\ {\isadigit{1}}{\isadigit{0}}{\isacharcolon}{\kern0pt}\isanewline
\ \ \ \ {\isachardoublequoteopen}{\isacharbraceleft}{\kern0pt}i{\isacharcolon}{\kern0pt}{\isacharcolon}{\kern0pt}nat{\isachardot}{\kern0pt}\ i\ {\isacharless}{\kern0pt}\ size\ p\ {\isasymand}\ {\isacharparenleft}{\kern0pt}let\ r\ {\isacharequal}{\kern0pt}\ {\isacharparenleft}{\kern0pt}p{\isacharbang}{\kern0pt}i{\isacharparenright}{\kern0pt}\ in\ {\isacharparenleft}{\kern0pt}y\ {\isasympreceq}\isactrlsub r\ x{\isacharparenright}{\kern0pt}{\isacharparenright}{\kern0pt}{\isacharbraceright}{\kern0pt}\ {\isacharequal}{\kern0pt}\isanewline
\ \ \ \ \ \ \ \ {\isacharbraceleft}{\kern0pt}i{\isacharcolon}{\kern0pt}{\isacharcolon}{\kern0pt}nat{\isachardot}{\kern0pt}\ i\ {\isacharless}{\kern0pt}\ size\ p{\isacharbraceright}{\kern0pt}\ {\isacharminus}{\kern0pt}\isanewline
\ \ \ \ \ \ \ \ \ \ {\isacharbraceleft}{\kern0pt}i{\isacharcolon}{\kern0pt}{\isacharcolon}{\kern0pt}nat{\isachardot}{\kern0pt}\ i\ {\isacharless}{\kern0pt}\ size\ p\ {\isasymand}\ {\isasymnot}\ {\isacharparenleft}{\kern0pt}let\ r\ {\isacharequal}{\kern0pt}\ {\isacharparenleft}{\kern0pt}p{\isacharbang}{\kern0pt}i{\isacharparenright}{\kern0pt}\ in\ {\isacharparenleft}{\kern0pt}y\ {\isasympreceq}\isactrlsub r\ x{\isacharparenright}{\kern0pt}{\isacharparenright}{\kern0pt}{\isacharbraceright}{\kern0pt}{\isachardoublequoteclose}\isanewline
\ \ \ \ \isacommand{by}\isamarkupfalse%
\ auto\isanewline
\ \ \isacommand{have}\isamarkupfalse%
\ {\isadigit{0}}{\isacharcolon}{\kern0pt}\ {\isachardoublequoteopen}{\isasymforall}\ i{\isacharcolon}{\kern0pt}{\isacharcolon}{\kern0pt}nat\ {\isachardot}{\kern0pt}\ i\ {\isacharless}{\kern0pt}\ size\ p\ {\isasymlongrightarrow}\ linear{\isacharunderscore}{\kern0pt}order{\isacharunderscore}{\kern0pt}on\ A\ {\isacharparenleft}{\kern0pt}p{\isacharbang}{\kern0pt}i{\isacharparenright}{\kern0pt}{\isachardoublequoteclose}\isanewline
\ \ \ \ \isacommand{using}\isamarkupfalse%
\ prof\ profile{\isacharunderscore}{\kern0pt}def\isanewline
\ \ \ \ \isacommand{by}\isamarkupfalse%
\ metis\isanewline
\ \ \isacommand{hence}\isamarkupfalse%
\ {\isachardoublequoteopen}{\isasymforall}\ i{\isacharcolon}{\kern0pt}{\isacharcolon}{\kern0pt}nat\ {\isachardot}{\kern0pt}\ i\ {\isacharless}{\kern0pt}\ size\ p\ {\isasymlongrightarrow}\ connex\ A\ {\isacharparenleft}{\kern0pt}p{\isacharbang}{\kern0pt}i{\isacharparenright}{\kern0pt}{\isachardoublequoteclose}\isanewline
\ \ \ \ \isacommand{by}\isamarkupfalse%
\ {\isacharparenleft}{\kern0pt}simp\ add{\isacharcolon}{\kern0pt}\ lin{\isacharunderscore}{\kern0pt}ord{\isacharunderscore}{\kern0pt}imp{\isacharunderscore}{\kern0pt}connex{\isacharparenright}{\kern0pt}\isanewline
\ \ \isacommand{hence}\isamarkupfalse%
\ {\isadigit{1}}{\isacharcolon}{\kern0pt}\ {\isachardoublequoteopen}{\isasymforall}\ i{\isacharcolon}{\kern0pt}{\isacharcolon}{\kern0pt}nat\ {\isachardot}{\kern0pt}\ i\ {\isacharless}{\kern0pt}\ size\ p\ {\isasymlongrightarrow}\isanewline
\ \ \ \ \ \ \ \ \ \ \ \ \ \ {\isasymnot}\ {\isacharparenleft}{\kern0pt}let\ r\ {\isacharequal}{\kern0pt}\ {\isacharparenleft}{\kern0pt}p{\isacharbang}{\kern0pt}i{\isacharparenright}{\kern0pt}\ in\ {\isacharparenleft}{\kern0pt}y\ {\isasympreceq}\isactrlsub r\ x{\isacharparenright}{\kern0pt}{\isacharparenright}{\kern0pt}\ {\isasymlongrightarrow}\ {\isacharparenleft}{\kern0pt}let\ r\ {\isacharequal}{\kern0pt}\ {\isacharparenleft}{\kern0pt}p{\isacharbang}{\kern0pt}i{\isacharparenright}{\kern0pt}\ in\ {\isacharparenleft}{\kern0pt}x\ {\isasympreceq}\isactrlsub r\ y{\isacharparenright}{\kern0pt}{\isacharparenright}{\kern0pt}{\isachardoublequoteclose}\isanewline
\ \ \ \ \isacommand{using}\isamarkupfalse%
\ connex{\isacharunderscore}{\kern0pt}def\ x{\isacharunderscore}{\kern0pt}in{\isacharunderscore}{\kern0pt}A\ y{\isacharunderscore}{\kern0pt}in{\isacharunderscore}{\kern0pt}A\isanewline
\ \ \ \ \isacommand{by}\isamarkupfalse%
\ metis\isanewline
\ \ \isacommand{from}\isamarkupfalse%
\ {\isadigit{0}}\ \isacommand{have}\isamarkupfalse%
\isanewline
\ \ \ \ {\isachardoublequoteopen}{\isasymforall}\ i{\isacharcolon}{\kern0pt}{\isacharcolon}{\kern0pt}nat\ {\isachardot}{\kern0pt}\ i\ {\isacharless}{\kern0pt}\ size\ p\ {\isasymlongrightarrow}\ antisym\ \ {\isacharparenleft}{\kern0pt}p{\isacharbang}{\kern0pt}i{\isacharparenright}{\kern0pt}{\isachardoublequoteclose}\isanewline
\ \ \ \ \isacommand{using}\isamarkupfalse%
\ lin{\isacharunderscore}{\kern0pt}imp{\isacharunderscore}{\kern0pt}antisym\isanewline
\ \ \ \ \isacommand{by}\isamarkupfalse%
\ metis\isanewline
\ \ \isacommand{hence}\isamarkupfalse%
\ {\isachardoublequoteopen}{\isasymforall}\ i{\isacharcolon}{\kern0pt}{\isacharcolon}{\kern0pt}nat\ {\isachardot}{\kern0pt}\ i\ {\isacharless}{\kern0pt}\ size\ p\ {\isasymlongrightarrow}\ {\isacharparenleft}{\kern0pt}{\isacharparenleft}{\kern0pt}y{\isacharcomma}{\kern0pt}\ x{\isacharparenright}{\kern0pt}\ {\isasymin}\ {\isacharparenleft}{\kern0pt}p{\isacharbang}{\kern0pt}i{\isacharparenright}{\kern0pt}\ {\isasymlongrightarrow}\ {\isacharparenleft}{\kern0pt}x{\isacharcomma}{\kern0pt}\ y{\isacharparenright}{\kern0pt}\ {\isasymnotin}\ {\isacharparenleft}{\kern0pt}p{\isacharbang}{\kern0pt}i{\isacharparenright}{\kern0pt}{\isacharparenright}{\kern0pt}{\isachardoublequoteclose}\isanewline
\ \ \ \ \isacommand{using}\isamarkupfalse%
\ antisymD\ neq\isanewline
\ \ \ \ \isacommand{by}\isamarkupfalse%
\ metis\isanewline
\ \ \isacommand{hence}\isamarkupfalse%
\ {\isachardoublequoteopen}{\isasymforall}\ i{\isacharcolon}{\kern0pt}{\isacharcolon}{\kern0pt}nat\ {\isachardot}{\kern0pt}\ i\ {\isacharless}{\kern0pt}\ size\ p\ {\isasymlongrightarrow}\isanewline
\ \ \ \ \ \ \ \ \ \ {\isacharparenleft}{\kern0pt}{\isacharparenleft}{\kern0pt}let\ r\ {\isacharequal}{\kern0pt}\ {\isacharparenleft}{\kern0pt}p{\isacharbang}{\kern0pt}i{\isacharparenright}{\kern0pt}\ in\ {\isacharparenleft}{\kern0pt}y\ {\isasympreceq}\isactrlsub r\ x{\isacharparenright}{\kern0pt}{\isacharparenright}{\kern0pt}\ {\isasymlongrightarrow}\ {\isasymnot}\ {\isacharparenleft}{\kern0pt}let\ r\ {\isacharequal}{\kern0pt}\ {\isacharparenleft}{\kern0pt}p{\isacharbang}{\kern0pt}i{\isacharparenright}{\kern0pt}\ in\ {\isacharparenleft}{\kern0pt}x\ {\isasympreceq}\isactrlsub r\ y{\isacharparenright}{\kern0pt}{\isacharparenright}{\kern0pt}{\isacharparenright}{\kern0pt}{\isachardoublequoteclose}\isanewline
\ \ \ \ \isacommand{by}\isamarkupfalse%
\ simp\isanewline
\ \ \isacommand{with}\isamarkupfalse%
\ {\isadigit{1}}\ \isacommand{have}\isamarkupfalse%
\isanewline
\ \ \ \ {\isachardoublequoteopen}{\isasymforall}\ i{\isacharcolon}{\kern0pt}{\isacharcolon}{\kern0pt}nat\ {\isachardot}{\kern0pt}\ i\ {\isacharless}{\kern0pt}\ size\ p\ {\isasymlongrightarrow}\isanewline
\ \ \ \ \ \ {\isasymnot}\ {\isacharparenleft}{\kern0pt}let\ r\ {\isacharequal}{\kern0pt}\ {\isacharparenleft}{\kern0pt}p{\isacharbang}{\kern0pt}i{\isacharparenright}{\kern0pt}\ in\ {\isacharparenleft}{\kern0pt}y\ {\isasympreceq}\isactrlsub r\ x{\isacharparenright}{\kern0pt}{\isacharparenright}{\kern0pt}\ {\isacharequal}{\kern0pt}\ {\isacharparenleft}{\kern0pt}let\ r\ {\isacharequal}{\kern0pt}\ {\isacharparenleft}{\kern0pt}p{\isacharbang}{\kern0pt}i{\isacharparenright}{\kern0pt}\ in\ {\isacharparenleft}{\kern0pt}x\ {\isasympreceq}\isactrlsub r\ y{\isacharparenright}{\kern0pt}{\isacharparenright}{\kern0pt}{\isachardoublequoteclose}\isanewline
\ \ \ \ \isacommand{by}\isamarkupfalse%
\ metis\isanewline
\ \ \isacommand{hence}\isamarkupfalse%
\ {\isadigit{2}}{\isacharcolon}{\kern0pt}\isanewline
\ \ \ \ {\isachardoublequoteopen}{\isacharbraceleft}{\kern0pt}i{\isacharcolon}{\kern0pt}{\isacharcolon}{\kern0pt}nat{\isachardot}{\kern0pt}\ i\ {\isacharless}{\kern0pt}\ size\ p\ {\isasymand}\ {\isasymnot}\ {\isacharparenleft}{\kern0pt}let\ r\ {\isacharequal}{\kern0pt}\ {\isacharparenleft}{\kern0pt}p{\isacharbang}{\kern0pt}i{\isacharparenright}{\kern0pt}\ in\ {\isacharparenleft}{\kern0pt}y\ {\isasympreceq}\isactrlsub r\ x{\isacharparenright}{\kern0pt}{\isacharparenright}{\kern0pt}{\isacharbraceright}{\kern0pt}\ {\isacharequal}{\kern0pt}\isanewline
\ \ \ \ \ \ \ \ {\isacharbraceleft}{\kern0pt}i{\isacharcolon}{\kern0pt}{\isacharcolon}{\kern0pt}nat{\isachardot}{\kern0pt}\ i\ {\isacharless}{\kern0pt}\ size\ p\ {\isasymand}\ {\isacharparenleft}{\kern0pt}let\ r\ {\isacharequal}{\kern0pt}\ {\isacharparenleft}{\kern0pt}p{\isacharbang}{\kern0pt}i{\isacharparenright}{\kern0pt}\ in\ {\isacharparenleft}{\kern0pt}x\ {\isasympreceq}\isactrlsub r\ y{\isacharparenright}{\kern0pt}{\isacharparenright}{\kern0pt}{\isacharbraceright}{\kern0pt}{\isachardoublequoteclose}\isanewline
\ \ \ \ \isacommand{by}\isamarkupfalse%
\ metis\isanewline
\ \ \isacommand{hence}\isamarkupfalse%
\ {\isadigit{2}}{\isadigit{0}}{\isacharcolon}{\kern0pt}\isanewline
\ \ \ \ {\isachardoublequoteopen}{\isacharbraceleft}{\kern0pt}i{\isacharcolon}{\kern0pt}{\isacharcolon}{\kern0pt}nat{\isachardot}{\kern0pt}\ i\ {\isacharless}{\kern0pt}\ size\ p\ {\isasymand}\ {\isacharparenleft}{\kern0pt}let\ r\ {\isacharequal}{\kern0pt}\ {\isacharparenleft}{\kern0pt}p{\isacharbang}{\kern0pt}i{\isacharparenright}{\kern0pt}\ in\ {\isacharparenleft}{\kern0pt}y\ {\isasympreceq}\isactrlsub r\ x{\isacharparenright}{\kern0pt}{\isacharparenright}{\kern0pt}{\isacharbraceright}{\kern0pt}\ {\isacharequal}{\kern0pt}\isanewline
\ \ \ \ \ \ \ \ {\isacharbraceleft}{\kern0pt}i{\isacharcolon}{\kern0pt}{\isacharcolon}{\kern0pt}nat{\isachardot}{\kern0pt}\ i\ {\isacharless}{\kern0pt}\ size\ p{\isacharbraceright}{\kern0pt}\ {\isacharminus}{\kern0pt}\ {\isacharbraceleft}{\kern0pt}i{\isacharcolon}{\kern0pt}{\isacharcolon}{\kern0pt}nat{\isachardot}{\kern0pt}\ i\ {\isacharless}{\kern0pt}\ size\ p\ {\isasymand}\ {\isacharparenleft}{\kern0pt}let\ r\ {\isacharequal}{\kern0pt}\ {\isacharparenleft}{\kern0pt}p{\isacharbang}{\kern0pt}i{\isacharparenright}{\kern0pt}\ in\ {\isacharparenleft}{\kern0pt}x\ {\isasympreceq}\isactrlsub r\ y{\isacharparenright}{\kern0pt}{\isacharparenright}{\kern0pt}{\isacharbraceright}{\kern0pt}{\isachardoublequoteclose}\isanewline
\ \ \ \ \isacommand{using}\isamarkupfalse%
\ {\isachardoublequoteopen}{\isadigit{1}}{\isadigit{0}}{\isachardoublequoteclose}\ {\isachardoublequoteopen}{\isadigit{2}}{\isachardoublequoteclose}\isanewline
\ \ \ \ \isacommand{by}\isamarkupfalse%
\ simp\isanewline
\ \ \isacommand{have}\isamarkupfalse%
\ {\isachardoublequoteopen}{\isacharbraceleft}{\kern0pt}i{\isacharcolon}{\kern0pt}{\isacharcolon}{\kern0pt}nat{\isachardot}{\kern0pt}\ i\ {\isacharless}{\kern0pt}\ size\ p\ {\isasymand}\ {\isacharparenleft}{\kern0pt}let\ r\ {\isacharequal}{\kern0pt}\ {\isacharparenleft}{\kern0pt}p{\isacharbang}{\kern0pt}i{\isacharparenright}{\kern0pt}\ in\ {\isacharparenleft}{\kern0pt}x\ {\isasympreceq}\isactrlsub r\ y{\isacharparenright}{\kern0pt}{\isacharparenright}{\kern0pt}{\isacharbraceright}{\kern0pt}\ {\isasymsubseteq}\ {\isacharbraceleft}{\kern0pt}i{\isacharcolon}{\kern0pt}{\isacharcolon}{\kern0pt}nat{\isachardot}{\kern0pt}\ i\ {\isacharless}{\kern0pt}\ size\ p{\isacharbraceright}{\kern0pt}{\isachardoublequoteclose}\isanewline
\ \ \ \ \isacommand{by}\isamarkupfalse%
\ {\isacharparenleft}{\kern0pt}simp\ add{\isacharcolon}{\kern0pt}\ Collect{\isacharunderscore}{\kern0pt}mono{\isacharparenright}{\kern0pt}\isanewline
\ \ \isacommand{hence}\isamarkupfalse%
\ {\isadigit{3}}{\isadigit{0}}{\isacharcolon}{\kern0pt}\isanewline
\ \ \ \ {\isachardoublequoteopen}card\ {\isacharparenleft}{\kern0pt}{\isacharbraceleft}{\kern0pt}i{\isacharcolon}{\kern0pt}{\isacharcolon}{\kern0pt}nat{\isachardot}{\kern0pt}\ i\ {\isacharless}{\kern0pt}\ size\ p{\isacharbraceright}{\kern0pt}\ {\isacharminus}{\kern0pt}\isanewline
\ \ \ \ \ \ \ \ {\isacharbraceleft}{\kern0pt}i{\isacharcolon}{\kern0pt}{\isacharcolon}{\kern0pt}nat{\isachardot}{\kern0pt}\ i\ {\isacharless}{\kern0pt}\ size\ p\ {\isasymand}\ {\isacharparenleft}{\kern0pt}let\ r\ {\isacharequal}{\kern0pt}\ {\isacharparenleft}{\kern0pt}p{\isacharbang}{\kern0pt}i{\isacharparenright}{\kern0pt}\ in\ {\isacharparenleft}{\kern0pt}x\ {\isasympreceq}\isactrlsub r\ y{\isacharparenright}{\kern0pt}{\isacharparenright}{\kern0pt}{\isacharbraceright}{\kern0pt}{\isacharparenright}{\kern0pt}\ {\isacharequal}{\kern0pt}\isanewline
\ \ \ \ \ \ {\isacharparenleft}{\kern0pt}card\ {\isacharbraceleft}{\kern0pt}i{\isacharcolon}{\kern0pt}{\isacharcolon}{\kern0pt}nat{\isachardot}{\kern0pt}\ i\ {\isacharless}{\kern0pt}\ size\ p{\isacharbraceright}{\kern0pt}{\isacharparenright}{\kern0pt}\ {\isacharminus}{\kern0pt}\isanewline
\ \ \ \ \ \ \ \ card{\isacharparenleft}{\kern0pt}{\isacharbraceleft}{\kern0pt}i{\isacharcolon}{\kern0pt}{\isacharcolon}{\kern0pt}nat{\isachardot}{\kern0pt}\ i\ {\isacharless}{\kern0pt}\ size\ p\ {\isasymand}\ {\isacharparenleft}{\kern0pt}let\ r\ {\isacharequal}{\kern0pt}\ {\isacharparenleft}{\kern0pt}p{\isacharbang}{\kern0pt}i{\isacharparenright}{\kern0pt}\ in\ {\isacharparenleft}{\kern0pt}x\ {\isasympreceq}\isactrlsub r\ y{\isacharparenright}{\kern0pt}{\isacharparenright}{\kern0pt}{\isacharbraceright}{\kern0pt}{\isacharparenright}{\kern0pt}{\isachardoublequoteclose}\isanewline
\ \ \ \ \isacommand{by}\isamarkupfalse%
\ {\isacharparenleft}{\kern0pt}simp\ add{\isacharcolon}{\kern0pt}\ card{\isacharunderscore}{\kern0pt}Diff{\isacharunderscore}{\kern0pt}subset{\isacharparenright}{\kern0pt}\isanewline
\ \ \isacommand{have}\isamarkupfalse%
\ {\isachardoublequoteopen}prefer{\isacharunderscore}{\kern0pt}count\ p\ x\ y\ {\isacharequal}{\kern0pt}\isanewline
\ \ \ \ \ \ \ \ \ \ card\ {\isacharbraceleft}{\kern0pt}i{\isacharcolon}{\kern0pt}{\isacharcolon}{\kern0pt}nat{\isachardot}{\kern0pt}\ i\ {\isacharless}{\kern0pt}\ size\ p\ {\isasymand}\ {\isacharparenleft}{\kern0pt}let\ r\ {\isacharequal}{\kern0pt}\ {\isacharparenleft}{\kern0pt}p{\isacharbang}{\kern0pt}i{\isacharparenright}{\kern0pt}\ in\ {\isacharparenleft}{\kern0pt}y\ {\isasympreceq}\isactrlsub r\ x{\isacharparenright}{\kern0pt}{\isacharparenright}{\kern0pt}{\isacharbraceright}{\kern0pt}{\isachardoublequoteclose}\isanewline
\ \ \ \ \isacommand{by}\isamarkupfalse%
\ simp\isanewline
\ \ \isacommand{also}\isamarkupfalse%
\ \isacommand{have}\isamarkupfalse%
\isanewline
\ \ \ \ {\isachardoublequoteopen}{\isachardot}{\kern0pt}{\isachardot}{\kern0pt}{\isachardot}{\kern0pt}\ {\isacharequal}{\kern0pt}\ card{\isacharparenleft}{\kern0pt}{\isacharbraceleft}{\kern0pt}i{\isacharcolon}{\kern0pt}{\isacharcolon}{\kern0pt}nat{\isachardot}{\kern0pt}\ i\ {\isacharless}{\kern0pt}\ size\ p{\isacharbraceright}{\kern0pt}\ {\isacharminus}{\kern0pt}\isanewline
\ \ \ \ \ \ \ \ \ \ \ \ {\isacharbraceleft}{\kern0pt}i{\isacharcolon}{\kern0pt}{\isacharcolon}{\kern0pt}nat{\isachardot}{\kern0pt}\ i\ {\isacharless}{\kern0pt}\ size\ p\ {\isasymand}\ {\isacharparenleft}{\kern0pt}let\ r\ {\isacharequal}{\kern0pt}\ {\isacharparenleft}{\kern0pt}p{\isacharbang}{\kern0pt}i{\isacharparenright}{\kern0pt}\ in\ {\isacharparenleft}{\kern0pt}x\ {\isasympreceq}\isactrlsub r\ y{\isacharparenright}{\kern0pt}{\isacharparenright}{\kern0pt}{\isacharbraceright}{\kern0pt}{\isacharparenright}{\kern0pt}{\isachardoublequoteclose}\isanewline
\ \ \ \ \isacommand{using}\isamarkupfalse%
\ {\isachardoublequoteopen}{\isadigit{2}}{\isadigit{0}}{\isachardoublequoteclose}\isanewline
\ \ \ \ \isacommand{by}\isamarkupfalse%
\ simp\isanewline
\ \ \isacommand{also}\isamarkupfalse%
\ \isacommand{have}\isamarkupfalse%
\isanewline
\ \ \ \ {\isachardoublequoteopen}{\isachardot}{\kern0pt}{\isachardot}{\kern0pt}{\isachardot}{\kern0pt}\ {\isacharequal}{\kern0pt}\ {\isacharparenleft}{\kern0pt}card\ {\isacharbraceleft}{\kern0pt}i{\isacharcolon}{\kern0pt}{\isacharcolon}{\kern0pt}nat{\isachardot}{\kern0pt}\ i\ {\isacharless}{\kern0pt}\ size\ p{\isacharbraceright}{\kern0pt}{\isacharparenright}{\kern0pt}\ {\isacharminus}{\kern0pt}\isanewline
\ \ \ \ \ \ \ \ \ \ \ \ \ \ card{\isacharparenleft}{\kern0pt}{\isacharbraceleft}{\kern0pt}i{\isacharcolon}{\kern0pt}{\isacharcolon}{\kern0pt}nat{\isachardot}{\kern0pt}\ i\ {\isacharless}{\kern0pt}\ size\ p\ {\isasymand}\ {\isacharparenleft}{\kern0pt}let\ r\ {\isacharequal}{\kern0pt}\ {\isacharparenleft}{\kern0pt}p{\isacharbang}{\kern0pt}i{\isacharparenright}{\kern0pt}\ in\ {\isacharparenleft}{\kern0pt}x\ {\isasympreceq}\isactrlsub r\ y{\isacharparenright}{\kern0pt}{\isacharparenright}{\kern0pt}{\isacharbraceright}{\kern0pt}{\isacharparenright}{\kern0pt}{\isachardoublequoteclose}\isanewline
\ \ \ \ \isacommand{using}\isamarkupfalse%
\ {\isachardoublequoteopen}{\isadigit{3}}{\isadigit{0}}{\isachardoublequoteclose}\isanewline
\ \ \ \ \isacommand{by}\isamarkupfalse%
\ metis\isanewline
\ \ \isacommand{also}\isamarkupfalse%
\ \isacommand{have}\isamarkupfalse%
\isanewline
\ \ \ \ {\isachardoublequoteopen}{\isachardot}{\kern0pt}{\isachardot}{\kern0pt}{\isachardot}{\kern0pt}\ {\isacharequal}{\kern0pt}\ size\ p\ {\isacharminus}{\kern0pt}\ {\isacharparenleft}{\kern0pt}prefer{\isacharunderscore}{\kern0pt}count\ p\ y\ x{\isacharparenright}{\kern0pt}{\isachardoublequoteclose}\isanewline
\ \ \ \ \isacommand{by}\isamarkupfalse%
\ simp\isanewline
\ \ \isacommand{finally}\isamarkupfalse%
\ \isacommand{show}\isamarkupfalse%
\ {\isacharquery}{\kern0pt}thesis\isanewline
\ \ \ \ \isacommand{by}\isamarkupfalse%
\ {\isacharparenleft}{\kern0pt}simp\ add{\isacharcolon}{\kern0pt}\ {\isachardoublequoteopen}{\isadigit{2}}{\isadigit{0}}{\isachardoublequoteclose}\ {\isachardoublequoteopen}{\isadigit{3}}{\isadigit{0}}{\isachardoublequoteclose}{\isacharparenright}{\kern0pt}\isanewline
\isacommand{qed}\isamarkupfalse%
%
\endisatagproof
{\isafoldproof}%
%
\isadelimproof
\isanewline
%
\endisadelimproof
\isanewline
\isacommand{lemma}\isamarkupfalse%
\ pref{\isacharunderscore}{\kern0pt}count{\isacharunderscore}{\kern0pt}sym{\isacharcolon}{\kern0pt}\isanewline
\ \ \ \ \isakeyword{assumes}\ p{\isadigit{1}}{\isacharcolon}{\kern0pt}\ {\isachardoublequoteopen}prefer{\isacharunderscore}{\kern0pt}count\ p\ a\ x\ {\isasymge}\ prefer{\isacharunderscore}{\kern0pt}count\ p\ x\ b{\isachardoublequoteclose}\isanewline
\ \ \ \ \isakeyword{assumes}\ prof{\isacharcolon}{\kern0pt}\ {\isachardoublequoteopen}profile\ A\ p{\isachardoublequoteclose}\isanewline
\ \ \ \ \isakeyword{assumes}\ a{\isacharunderscore}{\kern0pt}in{\isacharunderscore}{\kern0pt}A{\isacharcolon}{\kern0pt}\ {\isachardoublequoteopen}a\ {\isasymin}\ A{\isachardoublequoteclose}\isanewline
\ \ \ \ \isakeyword{assumes}\ b{\isacharunderscore}{\kern0pt}in{\isacharunderscore}{\kern0pt}A{\isacharcolon}{\kern0pt}\ {\isachardoublequoteopen}b\ {\isasymin}\ A{\isachardoublequoteclose}\isanewline
\ \ \ \ \isakeyword{assumes}\ x{\isacharunderscore}{\kern0pt}in{\isacharunderscore}{\kern0pt}A{\isacharcolon}{\kern0pt}\ {\isachardoublequoteopen}x\ {\isasymin}\ A{\isachardoublequoteclose}\isanewline
\ \ \ \ \isakeyword{assumes}\ neq{\isadigit{1}}{\isacharcolon}{\kern0pt}\ {\isachardoublequoteopen}a\ {\isasymnoteq}\ x{\isachardoublequoteclose}\isanewline
\ \ \ \ \isakeyword{assumes}\ neq{\isadigit{2}}{\isacharcolon}{\kern0pt}\ {\isachardoublequoteopen}x\ {\isasymnoteq}\ b{\isachardoublequoteclose}\isanewline
\ \ \ \ \isakeyword{shows}\ {\isachardoublequoteopen}prefer{\isacharunderscore}{\kern0pt}count\ p\ b\ x\ {\isasymge}\ prefer{\isacharunderscore}{\kern0pt}count\ p\ x\ a{\isachardoublequoteclose}\isanewline
%
\isadelimproof
%
\endisadelimproof
%
\isatagproof
\isacommand{proof}\isamarkupfalse%
\ {\isacharminus}{\kern0pt}\isanewline
\ \ \isacommand{from}\isamarkupfalse%
\ prof\ a{\isacharunderscore}{\kern0pt}in{\isacharunderscore}{\kern0pt}A\ x{\isacharunderscore}{\kern0pt}in{\isacharunderscore}{\kern0pt}A\ neq{\isadigit{1}}\ \isacommand{have}\isamarkupfalse%
\ {\isadigit{0}}{\isacharcolon}{\kern0pt}\isanewline
\ \ \ \ {\isachardoublequoteopen}prefer{\isacharunderscore}{\kern0pt}count\ p\ a\ x\ {\isacharequal}{\kern0pt}\ {\isacharparenleft}{\kern0pt}size\ p{\isacharparenright}{\kern0pt}\ {\isacharminus}{\kern0pt}\ {\isacharparenleft}{\kern0pt}prefer{\isacharunderscore}{\kern0pt}count\ p\ x\ a{\isacharparenright}{\kern0pt}{\isachardoublequoteclose}\isanewline
\ \ \ \ \isacommand{using}\isamarkupfalse%
\ pref{\isacharunderscore}{\kern0pt}count\isanewline
\ \ \ \ \isacommand{by}\isamarkupfalse%
\ metis\isanewline
\ \ \isacommand{moreover}\isamarkupfalse%
\ \isacommand{from}\isamarkupfalse%
\ prof\ x{\isacharunderscore}{\kern0pt}in{\isacharunderscore}{\kern0pt}A\ b{\isacharunderscore}{\kern0pt}in{\isacharunderscore}{\kern0pt}A\ neq{\isadigit{2}}\ \isacommand{have}\isamarkupfalse%
\ {\isadigit{1}}{\isacharcolon}{\kern0pt}\isanewline
\ \ \ \ {\isachardoublequoteopen}prefer{\isacharunderscore}{\kern0pt}count\ p\ x\ b\ {\isacharequal}{\kern0pt}\ {\isacharparenleft}{\kern0pt}size\ p{\isacharparenright}{\kern0pt}\ {\isacharminus}{\kern0pt}\ {\isacharparenleft}{\kern0pt}prefer{\isacharunderscore}{\kern0pt}count\ p\ b\ x{\isacharparenright}{\kern0pt}{\isachardoublequoteclose}\isanewline
\ \ \ \ \isacommand{using}\isamarkupfalse%
\ pref{\isacharunderscore}{\kern0pt}count\isanewline
\ \ \ \ \isacommand{by}\isamarkupfalse%
\ {\isacharparenleft}{\kern0pt}metis\ {\isacharparenleft}{\kern0pt}mono{\isacharunderscore}{\kern0pt}tags{\isacharcomma}{\kern0pt}\ lifting{\isacharparenright}{\kern0pt}{\isacharparenright}{\kern0pt}\isanewline
\ \ \isacommand{hence}\isamarkupfalse%
\ {\isadigit{2}}{\isacharcolon}{\kern0pt}\ {\isachardoublequoteopen}{\isacharparenleft}{\kern0pt}size\ p{\isacharparenright}{\kern0pt}\ {\isacharminus}{\kern0pt}\ {\isacharparenleft}{\kern0pt}prefer{\isacharunderscore}{\kern0pt}count\ p\ x\ a{\isacharparenright}{\kern0pt}\ {\isasymge}\isanewline
\ \ \ \ \ \ \ \ \ \ \ \ \ \ {\isacharparenleft}{\kern0pt}size\ p{\isacharparenright}{\kern0pt}\ {\isacharminus}{\kern0pt}\ {\isacharparenleft}{\kern0pt}prefer{\isacharunderscore}{\kern0pt}count\ p\ b\ x{\isacharparenright}{\kern0pt}{\isachardoublequoteclose}\isanewline
\ \ \ \ \isacommand{using}\isamarkupfalse%
\ calculation\ p{\isadigit{1}}\isanewline
\ \ \ \ \isacommand{by}\isamarkupfalse%
\ auto\isanewline
\ \ \isacommand{hence}\isamarkupfalse%
\ {\isadigit{3}}{\isacharcolon}{\kern0pt}\ {\isachardoublequoteopen}{\isacharparenleft}{\kern0pt}prefer{\isacharunderscore}{\kern0pt}count\ p\ x\ a{\isacharparenright}{\kern0pt}\ {\isacharminus}{\kern0pt}\ {\isacharparenleft}{\kern0pt}size\ p{\isacharparenright}{\kern0pt}\ {\isasymle}\isanewline
\ \ \ \ \ \ \ \ \ \ \ \ \ \ {\isacharparenleft}{\kern0pt}prefer{\isacharunderscore}{\kern0pt}count\ p\ b\ x{\isacharparenright}{\kern0pt}\ {\isacharminus}{\kern0pt}\ {\isacharparenleft}{\kern0pt}size\ p{\isacharparenright}{\kern0pt}{\isachardoublequoteclose}\isanewline
\ \ \ \ \isacommand{using}\isamarkupfalse%
\ a{\isacharunderscore}{\kern0pt}in{\isacharunderscore}{\kern0pt}A\ diff{\isacharunderscore}{\kern0pt}is{\isacharunderscore}{\kern0pt}{\isadigit{0}}{\isacharunderscore}{\kern0pt}eq\ diff{\isacharunderscore}{\kern0pt}le{\isacharunderscore}{\kern0pt}self\ neq{\isadigit{1}}\isanewline
\ \ \ \ \ \ \ \ \ \ pref{\isacharunderscore}{\kern0pt}count\ prof\ x{\isacharunderscore}{\kern0pt}in{\isacharunderscore}{\kern0pt}A\isanewline
\ \ \ \ \isacommand{by}\isamarkupfalse%
\ {\isacharparenleft}{\kern0pt}metis\ {\isacharparenleft}{\kern0pt}no{\isacharunderscore}{\kern0pt}types{\isacharparenright}{\kern0pt}{\isacharparenright}{\kern0pt}\isanewline
\ \ \isacommand{hence}\isamarkupfalse%
\ {\isachardoublequoteopen}{\isacharparenleft}{\kern0pt}prefer{\isacharunderscore}{\kern0pt}count\ p\ x\ a{\isacharparenright}{\kern0pt}\ {\isasymle}\ {\isacharparenleft}{\kern0pt}prefer{\isacharunderscore}{\kern0pt}count\ p\ b\ x{\isacharparenright}{\kern0pt}{\isachardoublequoteclose}\isanewline
\ \ \ \ \isacommand{using}\isamarkupfalse%
\ {\isachardoublequoteopen}{\isadigit{1}}{\isachardoublequoteclose}\ {\isachardoublequoteopen}{\isadigit{3}}{\isachardoublequoteclose}\ calculation\ p{\isadigit{1}}\isanewline
\ \ \ \ \isacommand{by}\isamarkupfalse%
\ linarith\isanewline
\ \ \isacommand{thus}\isamarkupfalse%
\ {\isacharquery}{\kern0pt}thesis\isanewline
\ \ \ \ \isacommand{by}\isamarkupfalse%
\ linarith\isanewline
\isacommand{qed}\isamarkupfalse%
%
\endisatagproof
{\isafoldproof}%
%
\isadelimproof
\isanewline
%
\endisadelimproof
\isanewline
\isacommand{lemma}\isamarkupfalse%
\ empty{\isacharunderscore}{\kern0pt}prof{\isacharunderscore}{\kern0pt}imp{\isacharunderscore}{\kern0pt}zero{\isacharunderscore}{\kern0pt}pref{\isacharunderscore}{\kern0pt}count{\isacharcolon}{\kern0pt}\isanewline
\ \ \isakeyword{assumes}\ {\isachardoublequoteopen}p\ {\isacharequal}{\kern0pt}\ {\isacharbrackleft}{\kern0pt}{\isacharbrackright}{\kern0pt}{\isachardoublequoteclose}\isanewline
\ \ \isakeyword{shows}\ {\isachardoublequoteopen}{\isasymforall}\ x\ y{\isachardot}{\kern0pt}\ prefer{\isacharunderscore}{\kern0pt}count\ p\ x\ y\ {\isacharequal}{\kern0pt}\ {\isadigit{0}}{\isachardoublequoteclose}\isanewline
%
\isadelimproof
\ \ %
\endisadelimproof
%
\isatagproof
\isacommand{using}\isamarkupfalse%
\ assms\isanewline
\ \ \isacommand{by}\isamarkupfalse%
\ simp%
\endisatagproof
{\isafoldproof}%
%
\isadelimproof
\isanewline
%
\endisadelimproof
\isanewline
\isacommand{lemma}\isamarkupfalse%
\ pref{\isacharunderscore}{\kern0pt}count{\isacharunderscore}{\kern0pt}code{\isacharunderscore}{\kern0pt}incr{\isacharcolon}{\kern0pt}\isanewline
\ \ \isakeyword{assumes}\isanewline
\ \ \ \ {\isachardoublequoteopen}prefer{\isacharunderscore}{\kern0pt}count{\isacharunderscore}{\kern0pt}code\ ps\ x\ y\ {\isacharequal}{\kern0pt}\ n{\isachardoublequoteclose}\ \isakeyword{and}\isanewline
\ \ \ \ {\isachardoublequoteopen}y\ {\isasympreceq}\isactrlsub p\ x{\isachardoublequoteclose}\isanewline
\ \ \isakeyword{shows}\ {\isachardoublequoteopen}prefer{\isacharunderscore}{\kern0pt}count{\isacharunderscore}{\kern0pt}code\ {\isacharparenleft}{\kern0pt}p{\isacharhash}{\kern0pt}ps{\isacharparenright}{\kern0pt}\ x\ y\ {\isacharequal}{\kern0pt}\ n{\isacharplus}{\kern0pt}{\isadigit{1}}{\isachardoublequoteclose}\isanewline
%
\isadelimproof
\ \ %
\endisadelimproof
%
\isatagproof
\isacommand{using}\isamarkupfalse%
\ assms\isanewline
\ \ \isacommand{by}\isamarkupfalse%
\ simp%
\endisatagproof
{\isafoldproof}%
%
\isadelimproof
\isanewline
%
\endisadelimproof
\isanewline
\isacommand{lemma}\isamarkupfalse%
\ pref{\isacharunderscore}{\kern0pt}count{\isacharunderscore}{\kern0pt}code{\isacharunderscore}{\kern0pt}not{\isacharunderscore}{\kern0pt}smaller{\isacharunderscore}{\kern0pt}imp{\isacharunderscore}{\kern0pt}constant{\isacharcolon}{\kern0pt}\isanewline
\ \ \isakeyword{assumes}\isanewline
\ \ \ \ {\isachardoublequoteopen}prefer{\isacharunderscore}{\kern0pt}count{\isacharunderscore}{\kern0pt}code\ ps\ x\ y\ {\isacharequal}{\kern0pt}\ n{\isachardoublequoteclose}\ \isakeyword{and}\isanewline
\ \ \ \ {\isachardoublequoteopen}{\isasymnot}{\isacharparenleft}{\kern0pt}y\ {\isasympreceq}\isactrlsub p\ x{\isacharparenright}{\kern0pt}{\isachardoublequoteclose}\isanewline
\ \ \isakeyword{shows}\ {\isachardoublequoteopen}prefer{\isacharunderscore}{\kern0pt}count{\isacharunderscore}{\kern0pt}code\ {\isacharparenleft}{\kern0pt}p{\isacharhash}{\kern0pt}ps{\isacharparenright}{\kern0pt}\ x\ y\ {\isacharequal}{\kern0pt}\ n{\isachardoublequoteclose}\isanewline
%
\isadelimproof
\ \ %
\endisadelimproof
%
\isatagproof
\isacommand{using}\isamarkupfalse%
\ assms\isanewline
\ \ \isacommand{by}\isamarkupfalse%
\ simp%
\endisatagproof
{\isafoldproof}%
%
\isadelimproof
\isanewline
%
\endisadelimproof
\isanewline
\isacommand{fun}\isamarkupfalse%
\ wins\ {\isacharcolon}{\kern0pt}{\isacharcolon}{\kern0pt}\ {\isachardoublequoteopen}{\isacharprime}{\kern0pt}a\ {\isasymRightarrow}\ {\isacharprime}{\kern0pt}a\ Profile\ {\isasymRightarrow}\ {\isacharprime}{\kern0pt}a\ {\isasymRightarrow}\ bool{\isachardoublequoteclose}\ \isakeyword{where}\isanewline
\ \ {\isachardoublequoteopen}wins\ x\ p\ y\ {\isacharequal}{\kern0pt}\isanewline
\ \ \ \ {\isacharparenleft}{\kern0pt}prefer{\isacharunderscore}{\kern0pt}count\ p\ x\ y\ {\isachargreater}{\kern0pt}\ prefer{\isacharunderscore}{\kern0pt}count\ p\ y\ x{\isacharparenright}{\kern0pt}{\isachardoublequoteclose}\isanewline
\isanewline
\isacommand{fun}\isamarkupfalse%
\ wins{\isacharunderscore}{\kern0pt}code\ {\isacharcolon}{\kern0pt}{\isacharcolon}{\kern0pt}\ {\isachardoublequoteopen}{\isacharprime}{\kern0pt}a\ {\isasymRightarrow}\ {\isacharprime}{\kern0pt}a\ Profile\ {\isasymRightarrow}\ {\isacharprime}{\kern0pt}a\ {\isasymRightarrow}\ bool{\isachardoublequoteclose}\ \isakeyword{where}\isanewline
\ \ {\isachardoublequoteopen}wins{\isacharunderscore}{\kern0pt}code\ x\ p\ y\ {\isacharequal}{\kern0pt}\isanewline
\ \ \ \ {\isacharparenleft}{\kern0pt}prefer{\isacharunderscore}{\kern0pt}count{\isacharunderscore}{\kern0pt}code\ p\ x\ y\ {\isachargreater}{\kern0pt}\ prefer{\isacharunderscore}{\kern0pt}count{\isacharunderscore}{\kern0pt}code\ p\ y\ x{\isacharparenright}{\kern0pt}{\isachardoublequoteclose}\isanewline
\isanewline
\isanewline
\isacommand{lemma}\isamarkupfalse%
\ wins{\isacharunderscore}{\kern0pt}antisym{\isacharcolon}{\kern0pt}\isanewline
\ \ \isakeyword{assumes}\ {\isachardoublequoteopen}wins\ a\ p\ b{\isachardoublequoteclose}\isanewline
\ \ \isakeyword{shows}\ {\isachardoublequoteopen}{\isasymnot}\ wins\ b\ p\ a{\isachardoublequoteclose}\isanewline
%
\isadelimproof
\ \ %
\endisadelimproof
%
\isatagproof
\isacommand{using}\isamarkupfalse%
\ assms\isanewline
\ \ \isacommand{by}\isamarkupfalse%
\ simp%
\endisatagproof
{\isafoldproof}%
%
\isadelimproof
\isanewline
%
\endisadelimproof
\isanewline
\isacommand{lemma}\isamarkupfalse%
\ wins{\isacharunderscore}{\kern0pt}irreflex{\isacharcolon}{\kern0pt}\ {\isachardoublequoteopen}{\isasymnot}\ wins\ w\ p\ w{\isachardoublequoteclose}\isanewline
%
\isadelimproof
\ \ %
\endisadelimproof
%
\isatagproof
\isacommand{using}\isamarkupfalse%
\ wins{\isacharunderscore}{\kern0pt}antisym\isanewline
\ \ \isacommand{by}\isamarkupfalse%
\ metis%
\endisatagproof
{\isafoldproof}%
%
\isadelimproof
%
\endisadelimproof
%
\isadelimdocument
%
\endisadelimdocument
%
\isatagdocument
%
\isamarkupsubsection{Condorcet Winner%
}
\isamarkuptrue%
%
\endisatagdocument
{\isafolddocument}%
%
\isadelimdocument
%
\endisadelimdocument
\isacommand{fun}\isamarkupfalse%
\ condorcet{\isacharunderscore}{\kern0pt}winner\ {\isacharcolon}{\kern0pt}{\isacharcolon}{\kern0pt}\ {\isachardoublequoteopen}{\isacharprime}{\kern0pt}a\ set\ {\isasymRightarrow}\ {\isacharprime}{\kern0pt}a\ Profile\ {\isasymRightarrow}\ {\isacharprime}{\kern0pt}a\ {\isasymRightarrow}\ bool{\isachardoublequoteclose}\ \isakeyword{where}\isanewline
\ \ {\isachardoublequoteopen}condorcet{\isacharunderscore}{\kern0pt}winner\ A\ p\ w\ {\isacharequal}{\kern0pt}\isanewline
\ \ \ \ \ \ {\isacharparenleft}{\kern0pt}finite{\isacharunderscore}{\kern0pt}profile\ A\ p\ {\isasymand}\ \ w\ {\isasymin}\ A\ {\isasymand}\ {\isacharparenleft}{\kern0pt}{\isasymforall}x\ {\isasymin}\ A\ {\isacharminus}{\kern0pt}\ {\isacharbraceleft}{\kern0pt}w{\isacharbraceright}{\kern0pt}\ {\isachardot}{\kern0pt}\ wins\ w\ p\ x{\isacharparenright}{\kern0pt}{\isacharparenright}{\kern0pt}{\isachardoublequoteclose}\isanewline
\isanewline
\isacommand{fun}\isamarkupfalse%
\ condorcet{\isacharunderscore}{\kern0pt}winner{\isacharunderscore}{\kern0pt}code\ {\isacharcolon}{\kern0pt}{\isacharcolon}{\kern0pt}\ {\isachardoublequoteopen}{\isacharprime}{\kern0pt}a\ set\ {\isasymRightarrow}\ {\isacharprime}{\kern0pt}a\ Profile\ {\isasymRightarrow}\ {\isacharprime}{\kern0pt}a\ {\isasymRightarrow}\ bool{\isachardoublequoteclose}\ \isakeyword{where}\isanewline
\ \ {\isachardoublequoteopen}condorcet{\isacharunderscore}{\kern0pt}winner{\isacharunderscore}{\kern0pt}code\ A\ p\ w\ {\isacharequal}{\kern0pt}\isanewline
\ \ \ \ {\isacharparenleft}{\kern0pt}finite{\isacharunderscore}{\kern0pt}profile\ A\ p\ {\isasymand}\ \ w\ {\isasymin}\ A\ {\isasymand}\isanewline
\ \ \ \ \ \ {\isacharparenleft}{\kern0pt}{\isasymforall}x\ {\isasymin}\ A\ {\isacharminus}{\kern0pt}\ {\isacharbraceleft}{\kern0pt}w{\isacharbraceright}{\kern0pt}\ {\isachardot}{\kern0pt}\ wins{\isacharunderscore}{\kern0pt}code\ w\ p\ x{\isacharparenright}{\kern0pt}{\isacharparenright}{\kern0pt}{\isachardoublequoteclose}\isanewline
\isanewline
\isacommand{lemma}\isamarkupfalse%
\ cond{\isacharunderscore}{\kern0pt}winner{\isacharunderscore}{\kern0pt}unique{\isacharcolon}{\kern0pt}\isanewline
\ \ \isakeyword{assumes}\ winner{\isacharunderscore}{\kern0pt}c{\isacharcolon}{\kern0pt}\ {\isachardoublequoteopen}condorcet{\isacharunderscore}{\kern0pt}winner\ A\ p\ c{\isachardoublequoteclose}\ \isakeyword{and}\isanewline
\ \ \ \ \ \ \ \ \ \ winner{\isacharunderscore}{\kern0pt}w{\isacharcolon}{\kern0pt}\ {\isachardoublequoteopen}condorcet{\isacharunderscore}{\kern0pt}winner\ A\ p\ w{\isachardoublequoteclose}\isanewline
\ \ \isakeyword{shows}\ {\isachardoublequoteopen}w\ {\isacharequal}{\kern0pt}\ c{\isachardoublequoteclose}\isanewline
%
\isadelimproof
%
\endisadelimproof
%
\isatagproof
\isacommand{proof}\isamarkupfalse%
\ {\isacharparenleft}{\kern0pt}rule\ ccontr{\isacharparenright}{\kern0pt}\isanewline
\ \ \isacommand{assume}\isamarkupfalse%
\isanewline
\ \ \ \ assumption{\isacharcolon}{\kern0pt}\ {\isachardoublequoteopen}w\ {\isasymnoteq}\ c{\isachardoublequoteclose}\isanewline
\ \ \isacommand{from}\isamarkupfalse%
\ winner{\isacharunderscore}{\kern0pt}w\isanewline
\ \ \isacommand{have}\isamarkupfalse%
\ {\isachardoublequoteopen}wins\ w\ p\ c{\isachardoublequoteclose}\isanewline
\ \ \ \ \isacommand{using}\isamarkupfalse%
\ assumption\ insert{\isacharunderscore}{\kern0pt}Diff\ insert{\isacharunderscore}{\kern0pt}iff\ winner{\isacharunderscore}{\kern0pt}c\isanewline
\ \ \ \ \isacommand{by}\isamarkupfalse%
\ simp\isanewline
\ \ \isacommand{hence}\isamarkupfalse%
\ {\isachardoublequoteopen}{\isasymnot}\ wins\ c\ p\ w{\isachardoublequoteclose}\isanewline
\ \ \ \ \isacommand{by}\isamarkupfalse%
\ {\isacharparenleft}{\kern0pt}simp\ add{\isacharcolon}{\kern0pt}\ wins{\isacharunderscore}{\kern0pt}antisym{\isacharparenright}{\kern0pt}\isanewline
\ \ \isacommand{moreover}\isamarkupfalse%
\ \isacommand{from}\isamarkupfalse%
\ winner{\isacharunderscore}{\kern0pt}c\isanewline
\ \ \isacommand{have}\isamarkupfalse%
\isanewline
\ \ \ \ c{\isacharunderscore}{\kern0pt}wins{\isacharunderscore}{\kern0pt}against{\isacharunderscore}{\kern0pt}w{\isacharcolon}{\kern0pt}\ {\isachardoublequoteopen}wins\ c\ p\ w{\isachardoublequoteclose}\isanewline
\ \ \ \ \isacommand{using}\isamarkupfalse%
\ Diff{\isacharunderscore}{\kern0pt}iff\ assumption\isanewline
\ \ \ \ \ \ \ \ \ \ singletonD\ winner{\isacharunderscore}{\kern0pt}w\isanewline
\ \ \ \ \isacommand{by}\isamarkupfalse%
\ simp\isanewline
\ \ \isacommand{ultimately}\isamarkupfalse%
\ \isacommand{show}\isamarkupfalse%
\ False\isanewline
\ \ \ \ \isacommand{by}\isamarkupfalse%
\ simp\isanewline
\isacommand{qed}\isamarkupfalse%
%
\endisatagproof
{\isafoldproof}%
%
\isadelimproof
\isanewline
%
\endisadelimproof
\isanewline
\isacommand{lemma}\isamarkupfalse%
\ cond{\isacharunderscore}{\kern0pt}winner{\isacharunderscore}{\kern0pt}unique{\isadigit{2}}{\isacharcolon}{\kern0pt}\isanewline
\ \ \isakeyword{assumes}\ winner{\isacharcolon}{\kern0pt}\ {\isachardoublequoteopen}condorcet{\isacharunderscore}{\kern0pt}winner\ A\ p\ w{\isachardoublequoteclose}\ \isakeyword{and}\isanewline
\ \ \ \ \ \ \ \ \ \ not{\isacharunderscore}{\kern0pt}w{\isacharcolon}{\kern0pt}\ \ {\isachardoublequoteopen}x\ {\isasymnoteq}\ w{\isachardoublequoteclose}\ \isakeyword{and}\isanewline
\ \ \ \ \ \ \ \ \ \ in{\isacharunderscore}{\kern0pt}A{\isacharcolon}{\kern0pt}\ \ \ {\isachardoublequoteopen}x\ {\isasymin}\ A{\isachardoublequoteclose}\isanewline
\ \ \ \ \ \ \ \ \isakeyword{shows}\ {\isachardoublequoteopen}{\isasymnot}\ condorcet{\isacharunderscore}{\kern0pt}winner\ A\ p\ x{\isachardoublequoteclose}\isanewline
%
\isadelimproof
\ \ %
\endisadelimproof
%
\isatagproof
\isacommand{using}\isamarkupfalse%
\ not{\isacharunderscore}{\kern0pt}w\ cond{\isacharunderscore}{\kern0pt}winner{\isacharunderscore}{\kern0pt}unique\ winner\isanewline
\ \ \isacommand{by}\isamarkupfalse%
\ metis%
\endisatagproof
{\isafoldproof}%
%
\isadelimproof
\isanewline
%
\endisadelimproof
\isanewline
\isacommand{lemma}\isamarkupfalse%
\ cond{\isacharunderscore}{\kern0pt}winner{\isacharunderscore}{\kern0pt}unique{\isadigit{3}}{\isacharcolon}{\kern0pt}\isanewline
\ \ \isakeyword{assumes}\ {\isachardoublequoteopen}condorcet{\isacharunderscore}{\kern0pt}winner\ A\ p\ w{\isachardoublequoteclose}\isanewline
\ \ \isakeyword{shows}\ {\isachardoublequoteopen}{\isacharbraceleft}{\kern0pt}a\ {\isasymin}\ A{\isachardot}{\kern0pt}\ condorcet{\isacharunderscore}{\kern0pt}winner\ A\ p\ a{\isacharbraceright}{\kern0pt}\ {\isacharequal}{\kern0pt}\ {\isacharbraceleft}{\kern0pt}w{\isacharbraceright}{\kern0pt}{\isachardoublequoteclose}\isanewline
%
\isadelimproof
%
\endisadelimproof
%
\isatagproof
\isacommand{proof}\isamarkupfalse%
\ {\isacharparenleft}{\kern0pt}safe{\isacharcomma}{\kern0pt}\ simp{\isacharunderscore}{\kern0pt}all{\isacharcomma}{\kern0pt}\ safe{\isacharparenright}{\kern0pt}\isanewline
\ \ \isacommand{fix}\isamarkupfalse%
\isanewline
\ \ \ \ x\ {\isacharcolon}{\kern0pt}{\isacharcolon}{\kern0pt}\ {\isachardoublequoteopen}{\isacharprime}{\kern0pt}a{\isachardoublequoteclose}\isanewline
\ \ \isacommand{assume}\isamarkupfalse%
\isanewline
\ \ \ \ fin{\isacharunderscore}{\kern0pt}A{\isacharcolon}{\kern0pt}\ {\isachardoublequoteopen}finite\ A{\isachardoublequoteclose}\ \isakeyword{and}\isanewline
\ \ \ \ prof{\isacharunderscore}{\kern0pt}A{\isacharcolon}{\kern0pt}\ {\isachardoublequoteopen}profile\ A\ p{\isachardoublequoteclose}\ \isakeyword{and}\isanewline
\ \ \ \ x{\isacharunderscore}{\kern0pt}in{\isacharunderscore}{\kern0pt}A{\isacharcolon}{\kern0pt}\ {\isachardoublequoteopen}x\ {\isasymin}\ A{\isachardoublequoteclose}\ \isakeyword{and}\isanewline
\ \ \ \ x{\isacharunderscore}{\kern0pt}wins{\isacharcolon}{\kern0pt}\isanewline
\ \ \ \ \ \ {\isachardoublequoteopen}{\isasymforall}xa\ {\isasymin}\ A\ {\isacharminus}{\kern0pt}\ {\isacharbraceleft}{\kern0pt}x{\isacharbraceright}{\kern0pt}{\isachardot}{\kern0pt}\isanewline
\ \ \ \ \ \ \ \ card\ {\isacharbraceleft}{\kern0pt}i{\isachardot}{\kern0pt}\ i\ {\isacharless}{\kern0pt}\ size\ p\ {\isasymand}\ {\isacharparenleft}{\kern0pt}x{\isacharcomma}{\kern0pt}\ xa{\isacharparenright}{\kern0pt}\ {\isasymin}\ p{\isacharbang}{\kern0pt}i{\isacharbraceright}{\kern0pt}\ {\isacharless}{\kern0pt}\isanewline
\ \ \ \ \ \ \ \ \ \ card\ {\isacharbraceleft}{\kern0pt}i{\isachardot}{\kern0pt}\ i\ {\isacharless}{\kern0pt}\ size\ p\ {\isasymand}\ {\isacharparenleft}{\kern0pt}xa{\isacharcomma}{\kern0pt}\ x{\isacharparenright}{\kern0pt}\ {\isasymin}\ p{\isacharbang}{\kern0pt}i{\isacharbraceright}{\kern0pt}{\isachardoublequoteclose}\isanewline
\ \ \isacommand{from}\isamarkupfalse%
\ assms\ \isacommand{have}\isamarkupfalse%
\ assm{\isacharcolon}{\kern0pt}\isanewline
\ \ \ \ {\isachardoublequoteopen}finite{\isacharunderscore}{\kern0pt}profile\ A\ p\ {\isasymand}\ \ w\ {\isasymin}\ A\ {\isasymand}\isanewline
\ \ \ \ \ \ {\isacharparenleft}{\kern0pt}{\isasymforall}x\ {\isasymin}\ A\ {\isacharminus}{\kern0pt}\ {\isacharbraceleft}{\kern0pt}w{\isacharbraceright}{\kern0pt}{\isachardot}{\kern0pt}\isanewline
\ \ \ \ \ \ \ \ card\ {\isacharbraceleft}{\kern0pt}i{\isacharcolon}{\kern0pt}{\isacharcolon}{\kern0pt}nat{\isachardot}{\kern0pt}\ i\ {\isacharless}{\kern0pt}\ size\ p\ {\isasymand}\ {\isacharparenleft}{\kern0pt}w{\isacharcomma}{\kern0pt}\ x{\isacharparenright}{\kern0pt}\ {\isasymin}\ p{\isacharbang}{\kern0pt}i{\isacharbraceright}{\kern0pt}\ {\isacharless}{\kern0pt}\isanewline
\ \ \ \ \ \ \ \ \ \ card\ {\isacharbraceleft}{\kern0pt}i{\isacharcolon}{\kern0pt}{\isacharcolon}{\kern0pt}nat{\isachardot}{\kern0pt}\ i\ {\isacharless}{\kern0pt}\ size\ p\ {\isasymand}\ {\isacharparenleft}{\kern0pt}x{\isacharcomma}{\kern0pt}\ w{\isacharparenright}{\kern0pt}\ {\isasymin}\ p{\isacharbang}{\kern0pt}i{\isacharbraceright}{\kern0pt}{\isacharparenright}{\kern0pt}{\isachardoublequoteclose}\isanewline
\ \ \ \ \isacommand{by}\isamarkupfalse%
\ simp\isanewline
\ \ \isacommand{hence}\isamarkupfalse%
\isanewline
\ \ \ \ {\isachardoublequoteopen}{\isacharparenleft}{\kern0pt}{\isasymforall}x\ {\isasymin}\ A\ {\isacharminus}{\kern0pt}\ {\isacharbraceleft}{\kern0pt}w{\isacharbraceright}{\kern0pt}{\isachardot}{\kern0pt}\isanewline
\ \ \ \ \ \ card\ {\isacharbraceleft}{\kern0pt}i{\isacharcolon}{\kern0pt}{\isacharcolon}{\kern0pt}nat{\isachardot}{\kern0pt}\ i\ {\isacharless}{\kern0pt}\ size\ p\ {\isasymand}\ {\isacharparenleft}{\kern0pt}w{\isacharcomma}{\kern0pt}\ x{\isacharparenright}{\kern0pt}\ {\isasymin}\ p{\isacharbang}{\kern0pt}i{\isacharbraceright}{\kern0pt}\ {\isacharless}{\kern0pt}\isanewline
\ \ \ \ \ \ \ \ card\ {\isacharbraceleft}{\kern0pt}i{\isacharcolon}{\kern0pt}{\isacharcolon}{\kern0pt}nat{\isachardot}{\kern0pt}\ i\ {\isacharless}{\kern0pt}\ size\ p\ {\isasymand}\ {\isacharparenleft}{\kern0pt}x{\isacharcomma}{\kern0pt}\ w{\isacharparenright}{\kern0pt}\ {\isasymin}\ p{\isacharbang}{\kern0pt}i{\isacharbraceright}{\kern0pt}{\isacharparenright}{\kern0pt}{\isachardoublequoteclose}\isanewline
\ \ \ \ \isacommand{by}\isamarkupfalse%
\ simp\isanewline
\ \ \isacommand{hence}\isamarkupfalse%
\ w{\isacharunderscore}{\kern0pt}beats{\isacharunderscore}{\kern0pt}x{\isacharcolon}{\kern0pt}\isanewline
\ \ \ \ {\isachardoublequoteopen}x\ {\isasymnoteq}\ w\ {\isasymLongrightarrow}\isanewline
\ \ \ \ \ \ card\ {\isacharbraceleft}{\kern0pt}i{\isacharcolon}{\kern0pt}{\isacharcolon}{\kern0pt}nat{\isachardot}{\kern0pt}\ i\ {\isacharless}{\kern0pt}\ size\ p\ {\isasymand}\ {\isacharparenleft}{\kern0pt}w{\isacharcomma}{\kern0pt}\ x{\isacharparenright}{\kern0pt}\ {\isasymin}\ p{\isacharbang}{\kern0pt}i{\isacharbraceright}{\kern0pt}\ {\isacharless}{\kern0pt}\isanewline
\ \ \ \ \ \ \ \ card\ {\isacharbraceleft}{\kern0pt}i{\isacharcolon}{\kern0pt}{\isacharcolon}{\kern0pt}nat{\isachardot}{\kern0pt}\ i\ {\isacharless}{\kern0pt}\ size\ p\ {\isasymand}\ {\isacharparenleft}{\kern0pt}x{\isacharcomma}{\kern0pt}\ w{\isacharparenright}{\kern0pt}\ {\isasymin}\ p{\isacharbang}{\kern0pt}i{\isacharbraceright}{\kern0pt}{\isachardoublequoteclose}\isanewline
\ \ \ \ \isacommand{using}\isamarkupfalse%
\ x{\isacharunderscore}{\kern0pt}in{\isacharunderscore}{\kern0pt}A\isanewline
\ \ \ \ \isacommand{by}\isamarkupfalse%
\ simp\isanewline
\ \ \isacommand{also}\isamarkupfalse%
\ \isacommand{from}\isamarkupfalse%
\ assm\ \isacommand{have}\isamarkupfalse%
\isanewline
\ \ \ \ {\isachardoublequoteopen}finite{\isacharunderscore}{\kern0pt}profile\ A\ p{\isachardoublequoteclose}\isanewline
\ \ \ \ \isacommand{by}\isamarkupfalse%
\ simp\isanewline
\ \ \isacommand{moreover}\isamarkupfalse%
\ \isacommand{from}\isamarkupfalse%
\ assm\ \isacommand{have}\isamarkupfalse%
\isanewline
\ \ \ \ {\isachardoublequoteopen}w\ {\isasymin}\ A{\isachardoublequoteclose}\isanewline
\ \ \ \ \isacommand{by}\isamarkupfalse%
\ simp\isanewline
\ \ \isacommand{hence}\isamarkupfalse%
\ x{\isacharunderscore}{\kern0pt}beats{\isacharunderscore}{\kern0pt}w{\isacharcolon}{\kern0pt}\isanewline
\ \ \ \ {\isachardoublequoteopen}x\ {\isasymnoteq}\ w\ {\isasymLongrightarrow}\isanewline
\ \ \ \ \ \ card\ {\isacharbraceleft}{\kern0pt}i{\isachardot}{\kern0pt}\ i\ {\isacharless}{\kern0pt}\ size\ p\ {\isasymand}\ {\isacharparenleft}{\kern0pt}x{\isacharcomma}{\kern0pt}\ w{\isacharparenright}{\kern0pt}\ {\isasymin}\ p{\isacharbang}{\kern0pt}i{\isacharbraceright}{\kern0pt}\ {\isacharless}{\kern0pt}\isanewline
\ \ \ \ \ \ \ \ card\ {\isacharbraceleft}{\kern0pt}i{\isachardot}{\kern0pt}\ i\ {\isacharless}{\kern0pt}\ size\ p\ {\isasymand}\ {\isacharparenleft}{\kern0pt}w{\isacharcomma}{\kern0pt}\ x{\isacharparenright}{\kern0pt}\ {\isasymin}\ p{\isacharbang}{\kern0pt}i{\isacharbraceright}{\kern0pt}{\isachardoublequoteclose}\isanewline
\ \ \ \ \isacommand{using}\isamarkupfalse%
\ x{\isacharunderscore}{\kern0pt}wins\isanewline
\ \ \ \ \isacommand{by}\isamarkupfalse%
\ simp\isanewline
\ \ \isacommand{from}\isamarkupfalse%
\ w{\isacharunderscore}{\kern0pt}beats{\isacharunderscore}{\kern0pt}x\ x{\isacharunderscore}{\kern0pt}beats{\isacharunderscore}{\kern0pt}w\ \isacommand{show}\isamarkupfalse%
\isanewline
\ \ \ \ {\isachardoublequoteopen}x\ {\isacharequal}{\kern0pt}\ w{\isachardoublequoteclose}\isanewline
\ \ \ \ \isacommand{by}\isamarkupfalse%
\ linarith\isanewline
\isacommand{next}\isamarkupfalse%
\isanewline
\ \ \isacommand{fix}\isamarkupfalse%
\isanewline
\ \ \ \ x\ {\isacharcolon}{\kern0pt}{\isacharcolon}{\kern0pt}\ {\isachardoublequoteopen}{\isacharprime}{\kern0pt}a{\isachardoublequoteclose}\isanewline
\ \ \isacommand{from}\isamarkupfalse%
\ assms\ \isacommand{show}\isamarkupfalse%
\ {\isachardoublequoteopen}w\ {\isasymin}\ A{\isachardoublequoteclose}\isanewline
\ \ \ \ \isacommand{by}\isamarkupfalse%
\ simp\isanewline
\isacommand{next}\isamarkupfalse%
\isanewline
\ \ \isacommand{fix}\isamarkupfalse%
\isanewline
\ \ \ \ x\ {\isacharcolon}{\kern0pt}{\isacharcolon}{\kern0pt}\ {\isachardoublequoteopen}{\isacharprime}{\kern0pt}a{\isachardoublequoteclose}\isanewline
\ \ \isacommand{from}\isamarkupfalse%
\ assms\ \isacommand{show}\isamarkupfalse%
\ {\isachardoublequoteopen}finite\ A{\isachardoublequoteclose}\isanewline
\ \ \ \ \isacommand{by}\isamarkupfalse%
\ simp\isanewline
\isacommand{next}\isamarkupfalse%
\isanewline
\ \ \isacommand{fix}\isamarkupfalse%
\isanewline
\ \ \ \ x\ {\isacharcolon}{\kern0pt}{\isacharcolon}{\kern0pt}\ {\isachardoublequoteopen}{\isacharprime}{\kern0pt}a{\isachardoublequoteclose}\isanewline
\ \ \isacommand{from}\isamarkupfalse%
\ assms\ \isacommand{show}\isamarkupfalse%
\ {\isachardoublequoteopen}profile\ A\ p{\isachardoublequoteclose}\isanewline
\ \ \ \ \isacommand{by}\isamarkupfalse%
\ simp\isanewline
\isacommand{next}\isamarkupfalse%
\isanewline
\ \ \isacommand{fix}\isamarkupfalse%
\isanewline
\ \ \ \ x\ {\isacharcolon}{\kern0pt}{\isacharcolon}{\kern0pt}\ {\isachardoublequoteopen}{\isacharprime}{\kern0pt}a{\isachardoublequoteclose}\isanewline
\ \ \isacommand{from}\isamarkupfalse%
\ assms\ \isacommand{show}\isamarkupfalse%
\ {\isachardoublequoteopen}w\ {\isasymin}\ A{\isachardoublequoteclose}\isanewline
\ \ \ \ \isacommand{by}\isamarkupfalse%
\ simp\isanewline
\isacommand{next}\isamarkupfalse%
\isanewline
\ \ \isacommand{fix}\isamarkupfalse%
\isanewline
\ \ \ \ x\ {\isacharcolon}{\kern0pt}{\isacharcolon}{\kern0pt}\ {\isachardoublequoteopen}{\isacharprime}{\kern0pt}a{\isachardoublequoteclose}\ \isakeyword{and}\isanewline
\ \ \ \ xa\ {\isacharcolon}{\kern0pt}{\isacharcolon}{\kern0pt}\ {\isachardoublequoteopen}{\isacharprime}{\kern0pt}a{\isachardoublequoteclose}\isanewline
\ \ \isacommand{assume}\isamarkupfalse%
\isanewline
\ \ \ \ xa{\isacharunderscore}{\kern0pt}in{\isacharunderscore}{\kern0pt}A{\isacharcolon}{\kern0pt}\ {\isachardoublequoteopen}xa\ {\isasymin}\ A{\isachardoublequoteclose}\ \isakeyword{and}\isanewline
\ \ \ \ w{\isacharunderscore}{\kern0pt}wins{\isacharcolon}{\kern0pt}\isanewline
\ \ \ \ \ \ {\isachardoublequoteopen}{\isasymnot}\ card\ {\isacharbraceleft}{\kern0pt}i{\isachardot}{\kern0pt}\ i\ {\isacharless}{\kern0pt}\ length\ p\ {\isasymand}\ {\isacharparenleft}{\kern0pt}w{\isacharcomma}{\kern0pt}\ xa{\isacharparenright}{\kern0pt}\ {\isasymin}\ p{\isacharbang}{\kern0pt}i{\isacharbraceright}{\kern0pt}\ {\isacharless}{\kern0pt}\isanewline
\ \ \ \ \ \ \ \ card\ {\isacharbraceleft}{\kern0pt}i{\isachardot}{\kern0pt}\ i\ {\isacharless}{\kern0pt}\ length\ p\ {\isasymand}\ {\isacharparenleft}{\kern0pt}xa{\isacharcomma}{\kern0pt}\ w{\isacharparenright}{\kern0pt}\ {\isasymin}\ p{\isacharbang}{\kern0pt}i{\isacharbraceright}{\kern0pt}{\isachardoublequoteclose}\isanewline
\ \ \isacommand{from}\isamarkupfalse%
\ assms\ \isacommand{have}\isamarkupfalse%
\isanewline
\ \ \ \ {\isachardoublequoteopen}finite{\isacharunderscore}{\kern0pt}profile\ A\ p\ {\isasymand}\ \ w\ {\isasymin}\ A\ {\isasymand}\isanewline
\ \ \ \ \ \ {\isacharparenleft}{\kern0pt}{\isasymforall}x\ {\isasymin}\ A\ {\isacharminus}{\kern0pt}\ {\isacharbraceleft}{\kern0pt}w{\isacharbraceright}{\kern0pt}\ {\isachardot}{\kern0pt}\isanewline
\ \ \ \ \ \ \ \ card\ {\isacharbraceleft}{\kern0pt}i{\isacharcolon}{\kern0pt}{\isacharcolon}{\kern0pt}nat{\isachardot}{\kern0pt}\ i\ {\isacharless}{\kern0pt}\ size\ p\ {\isasymand}\ {\isacharparenleft}{\kern0pt}w{\isacharcomma}{\kern0pt}\ x{\isacharparenright}{\kern0pt}\ {\isasymin}\ p{\isacharbang}{\kern0pt}i{\isacharbraceright}{\kern0pt}\ {\isacharless}{\kern0pt}\isanewline
\ \ \ \ \ \ \ \ \ \ card\ {\isacharbraceleft}{\kern0pt}i{\isacharcolon}{\kern0pt}{\isacharcolon}{\kern0pt}nat{\isachardot}{\kern0pt}\ i\ {\isacharless}{\kern0pt}\ size\ p\ {\isasymand}\ {\isacharparenleft}{\kern0pt}x{\isacharcomma}{\kern0pt}\ w{\isacharparenright}{\kern0pt}\ {\isasymin}\ p{\isacharbang}{\kern0pt}i{\isacharbraceright}{\kern0pt}{\isacharparenright}{\kern0pt}{\isachardoublequoteclose}\isanewline
\ \ \ \ \isacommand{by}\isamarkupfalse%
\ simp\isanewline
\ \ \isacommand{thus}\isamarkupfalse%
\ {\isachardoublequoteopen}xa\ {\isacharequal}{\kern0pt}\ w{\isachardoublequoteclose}\isanewline
\ \ \ \ \isacommand{using}\isamarkupfalse%
\ xa{\isacharunderscore}{\kern0pt}in{\isacharunderscore}{\kern0pt}A\ w{\isacharunderscore}{\kern0pt}wins\ insert{\isacharunderscore}{\kern0pt}Diff\ insert{\isacharunderscore}{\kern0pt}iff\isanewline
\ \ \ \ \isacommand{by}\isamarkupfalse%
\ {\isacharparenleft}{\kern0pt}metis\ {\isacharparenleft}{\kern0pt}no{\isacharunderscore}{\kern0pt}types{\isacharcomma}{\kern0pt}\ lifting{\isacharparenright}{\kern0pt}{\isacharparenright}{\kern0pt}\isanewline
\isacommand{qed}\isamarkupfalse%
%
\endisatagproof
{\isafoldproof}%
%
\isadelimproof
%
\endisadelimproof
%
\isadelimdocument
%
\endisadelimdocument
%
\isatagdocument
%
\isamarkupsubsection{Limited Profile%
}
\isamarkuptrue%
%
\endisatagdocument
{\isafolddocument}%
%
\isadelimdocument
%
\endisadelimdocument
\isacommand{fun}\isamarkupfalse%
\ limit{\isacharunderscore}{\kern0pt}profile\ {\isacharcolon}{\kern0pt}{\isacharcolon}{\kern0pt}\ {\isachardoublequoteopen}{\isacharprime}{\kern0pt}a\ set\ {\isasymRightarrow}\ {\isacharprime}{\kern0pt}a\ Profile\ {\isasymRightarrow}\ {\isacharprime}{\kern0pt}a\ Profile{\isachardoublequoteclose}\ \isakeyword{where}\isanewline
\ \ {\isachardoublequoteopen}limit{\isacharunderscore}{\kern0pt}profile\ A\ p\ {\isacharequal}{\kern0pt}\ map\ {\isacharparenleft}{\kern0pt}limit\ A{\isacharparenright}{\kern0pt}\ p{\isachardoublequoteclose}\isanewline
\isanewline
\isacommand{lemma}\isamarkupfalse%
\ limit{\isacharunderscore}{\kern0pt}prof{\isacharunderscore}{\kern0pt}trans{\isacharcolon}{\kern0pt}\isanewline
\ \ \isakeyword{assumes}\isanewline
\ \ \ \ {\isachardoublequoteopen}B\ {\isasymsubseteq}\ A{\isachardoublequoteclose}\ \isakeyword{and}\isanewline
\ \ \ \ {\isachardoublequoteopen}C\ {\isasymsubseteq}\ B{\isachardoublequoteclose}\ \isakeyword{and}\isanewline
\ \ \ \ {\isachardoublequoteopen}finite{\isacharunderscore}{\kern0pt}profile\ A\ p{\isachardoublequoteclose}\isanewline
\ \ \isakeyword{shows}\ {\isachardoublequoteopen}limit{\isacharunderscore}{\kern0pt}profile\ C\ p\ {\isacharequal}{\kern0pt}\ limit{\isacharunderscore}{\kern0pt}profile\ C\ {\isacharparenleft}{\kern0pt}limit{\isacharunderscore}{\kern0pt}profile\ B\ p{\isacharparenright}{\kern0pt}{\isachardoublequoteclose}\isanewline
%
\isadelimproof
\ \ %
\endisadelimproof
%
\isatagproof
\isacommand{using}\isamarkupfalse%
\ assms\isanewline
\ \ \isacommand{by}\isamarkupfalse%
\ auto%
\endisatagproof
{\isafoldproof}%
%
\isadelimproof
\isanewline
%
\endisadelimproof
\isanewline
\isacommand{lemma}\isamarkupfalse%
\ limit{\isacharunderscore}{\kern0pt}profile{\isacharunderscore}{\kern0pt}sound{\isacharcolon}{\kern0pt}\isanewline
\ \ \isakeyword{assumes}\isanewline
\ \ \ \ profile{\isacharcolon}{\kern0pt}\ {\isachardoublequoteopen}finite{\isacharunderscore}{\kern0pt}profile\ S\ p{\isachardoublequoteclose}\ \isakeyword{and}\isanewline
\ \ \ \ subset{\isacharcolon}{\kern0pt}\ {\isachardoublequoteopen}A\ {\isasymsubseteq}\ S{\isachardoublequoteclose}\isanewline
\ \ \isakeyword{shows}\ {\isachardoublequoteopen}finite{\isacharunderscore}{\kern0pt}profile\ A\ {\isacharparenleft}{\kern0pt}limit{\isacharunderscore}{\kern0pt}profile\ A\ p{\isacharparenright}{\kern0pt}{\isachardoublequoteclose}\isanewline
%
\isadelimproof
%
\endisadelimproof
%
\isatagproof
\isacommand{proof}\isamarkupfalse%
\ {\isacharparenleft}{\kern0pt}simp{\isacharparenright}{\kern0pt}\isanewline
\ \ \isacommand{from}\isamarkupfalse%
\ profile\isanewline
\ \ \isacommand{show}\isamarkupfalse%
\ {\isachardoublequoteopen}finite{\isacharunderscore}{\kern0pt}profile\ A\ {\isacharparenleft}{\kern0pt}map\ {\isacharparenleft}{\kern0pt}limit\ A{\isacharparenright}{\kern0pt}\ p{\isacharparenright}{\kern0pt}{\isachardoublequoteclose}\isanewline
\ \ \ \ \isacommand{using}\isamarkupfalse%
\ length{\isacharunderscore}{\kern0pt}map\ limit{\isacharunderscore}{\kern0pt}presv{\isacharunderscore}{\kern0pt}lin{\isacharunderscore}{\kern0pt}ord\ nth{\isacharunderscore}{\kern0pt}map\isanewline
\ \ \ \ \ \ \ \ \ \ profile{\isacharunderscore}{\kern0pt}def\ subset\ infinite{\isacharunderscore}{\kern0pt}super\isanewline
\ \ \ \ \isacommand{by}\isamarkupfalse%
\ metis\isanewline
\isacommand{qed}\isamarkupfalse%
%
\endisatagproof
{\isafoldproof}%
%
\isadelimproof
\isanewline
%
\endisadelimproof
\isanewline
\isacommand{lemma}\isamarkupfalse%
\ limit{\isacharunderscore}{\kern0pt}prof{\isacharunderscore}{\kern0pt}presv{\isacharunderscore}{\kern0pt}size{\isacharcolon}{\kern0pt}\isanewline
\ \ \isakeyword{assumes}\ f{\isacharunderscore}{\kern0pt}prof{\isacharcolon}{\kern0pt}\ {\isachardoublequoteopen}finite{\isacharunderscore}{\kern0pt}profile\ S\ p{\isachardoublequoteclose}\ \isakeyword{and}\isanewline
\ \ \ \ \ \ \ \ \ \ subset{\isacharcolon}{\kern0pt}\ \ {\isachardoublequoteopen}A\ {\isasymsubseteq}\ S{\isachardoublequoteclose}\isanewline
\ \ \isakeyword{shows}\ {\isachardoublequoteopen}size\ p\ {\isacharequal}{\kern0pt}\ size\ {\isacharparenleft}{\kern0pt}limit{\isacharunderscore}{\kern0pt}profile\ A\ p{\isacharparenright}{\kern0pt}{\isachardoublequoteclose}\isanewline
%
\isadelimproof
\ \ %
\endisadelimproof
%
\isatagproof
\isacommand{by}\isamarkupfalse%
\ simp%
\endisatagproof
{\isafoldproof}%
%
\isadelimproof
%
\endisadelimproof
%
\isadelimdocument
%
\endisadelimdocument
%
\isatagdocument
%
\isamarkupsubsection{Lifting Property%
}
\isamarkuptrue%
%
\endisatagdocument
{\isafolddocument}%
%
\isadelimdocument
%
\endisadelimdocument
\isacommand{definition}\isamarkupfalse%
\ equiv{\isacharunderscore}{\kern0pt}prof{\isacharunderscore}{\kern0pt}except{\isacharunderscore}{\kern0pt}a\ {\isacharcolon}{\kern0pt}{\isacharcolon}{\kern0pt}\ {\isachardoublequoteopen}{\isacharprime}{\kern0pt}a\ set\ {\isasymRightarrow}\ {\isacharprime}{\kern0pt}a\ Profile\ {\isasymRightarrow}\ {\isacharprime}{\kern0pt}a\ Profile\ {\isasymRightarrow}\isanewline
\ \ \ \ \ \ \ \ \ \ \ \ \ \ \ \ \ \ \ \ \ \ \ \ \ \ \ \ \ \ \ \ \ \ \ \ \ \ \ \ {\isacharprime}{\kern0pt}a\ {\isasymRightarrow}\ bool{\isachardoublequoteclose}\ \isakeyword{where}\isanewline
\ \ {\isachardoublequoteopen}equiv{\isacharunderscore}{\kern0pt}prof{\isacharunderscore}{\kern0pt}except{\isacharunderscore}{\kern0pt}a\ A\ p\ q\ a\ {\isasymequiv}\isanewline
\ \ \ \ finite{\isacharunderscore}{\kern0pt}profile\ A\ p\ {\isasymand}\ finite{\isacharunderscore}{\kern0pt}profile\ A\ q\ {\isasymand}\ a\ {\isasymin}\ A\ {\isasymand}\ size\ p\ {\isacharequal}{\kern0pt}\ size\ q\ {\isasymand}\isanewline
\ \ \ \ {\isacharparenleft}{\kern0pt}{\isasymforall}i{\isacharcolon}{\kern0pt}{\isacharcolon}{\kern0pt}nat{\isachardot}{\kern0pt}\isanewline
\ \ \ \ \ \ i\ {\isacharless}{\kern0pt}\ size\ p\ {\isasymlongrightarrow}\isanewline
\ \ \ \ \ \ \ \ equiv{\isacharunderscore}{\kern0pt}rel{\isacharunderscore}{\kern0pt}except{\isacharunderscore}{\kern0pt}a\ A\ {\isacharparenleft}{\kern0pt}p{\isacharbang}{\kern0pt}i{\isacharparenright}{\kern0pt}\ {\isacharparenleft}{\kern0pt}q{\isacharbang}{\kern0pt}i{\isacharparenright}{\kern0pt}\ a{\isacharparenright}{\kern0pt}{\isachardoublequoteclose}\isanewline
\isanewline
\isanewline
\isacommand{definition}\isamarkupfalse%
\ lifted\ {\isacharcolon}{\kern0pt}{\isacharcolon}{\kern0pt}\ {\isachardoublequoteopen}{\isacharprime}{\kern0pt}a\ set\ {\isasymRightarrow}\ {\isacharprime}{\kern0pt}a\ Profile\ {\isasymRightarrow}\ {\isacharprime}{\kern0pt}a\ Profile\ {\isasymRightarrow}\ {\isacharprime}{\kern0pt}a\ {\isasymRightarrow}\ bool{\isachardoublequoteclose}\ \isakeyword{where}\isanewline
\ \ {\isachardoublequoteopen}lifted\ A\ p\ q\ a\ {\isasymequiv}\isanewline
\ \ \ \ finite{\isacharunderscore}{\kern0pt}profile\ A\ p\ {\isasymand}\ finite{\isacharunderscore}{\kern0pt}profile\ A\ q\ {\isasymand}\ a\ {\isasymin}\ A\ {\isasymand}\ size\ p\ {\isacharequal}{\kern0pt}\ size\ q\ {\isasymand}\isanewline
\ \ \ \ {\isacharparenleft}{\kern0pt}{\isasymforall}i{\isacharcolon}{\kern0pt}{\isacharcolon}{\kern0pt}nat{\isachardot}{\kern0pt}\isanewline
\ \ \ \ \ \ {\isacharparenleft}{\kern0pt}i\ {\isacharless}{\kern0pt}\ size\ p\ {\isasymand}\ {\isasymnot}Preference{\isacharunderscore}{\kern0pt}Relation{\isachardot}{\kern0pt}lifted\ A\ {\isacharparenleft}{\kern0pt}p{\isacharbang}{\kern0pt}i{\isacharparenright}{\kern0pt}\ {\isacharparenleft}{\kern0pt}q{\isacharbang}{\kern0pt}i{\isacharparenright}{\kern0pt}\ a{\isacharparenright}{\kern0pt}\ {\isasymlongrightarrow}\isanewline
\ \ \ \ \ \ \ \ {\isacharparenleft}{\kern0pt}p{\isacharbang}{\kern0pt}i{\isacharparenright}{\kern0pt}\ {\isacharequal}{\kern0pt}\ {\isacharparenleft}{\kern0pt}q{\isacharbang}{\kern0pt}i{\isacharparenright}{\kern0pt}{\isacharparenright}{\kern0pt}\ {\isasymand}\isanewline
\ \ \ \ {\isacharparenleft}{\kern0pt}{\isasymexists}i{\isacharcolon}{\kern0pt}{\isacharcolon}{\kern0pt}nat{\isachardot}{\kern0pt}\ i\ {\isacharless}{\kern0pt}\ size\ p\ {\isasymand}\ Preference{\isacharunderscore}{\kern0pt}Relation{\isachardot}{\kern0pt}lifted\ A\ {\isacharparenleft}{\kern0pt}p{\isacharbang}{\kern0pt}i{\isacharparenright}{\kern0pt}\ {\isacharparenleft}{\kern0pt}q{\isacharbang}{\kern0pt}i{\isacharparenright}{\kern0pt}\ a{\isacharparenright}{\kern0pt}{\isachardoublequoteclose}\isanewline
\isanewline
\isacommand{lemma}\isamarkupfalse%
\ lifted{\isacharunderscore}{\kern0pt}imp{\isacharunderscore}{\kern0pt}equiv{\isacharunderscore}{\kern0pt}prof{\isacharunderscore}{\kern0pt}except{\isacharunderscore}{\kern0pt}a{\isacharcolon}{\kern0pt}\isanewline
\ \ \isakeyword{assumes}\ lifted{\isacharcolon}{\kern0pt}\ {\isachardoublequoteopen}lifted\ A\ p\ q\ a{\isachardoublequoteclose}\isanewline
\ \ \isakeyword{shows}\ {\isachardoublequoteopen}equiv{\isacharunderscore}{\kern0pt}prof{\isacharunderscore}{\kern0pt}except{\isacharunderscore}{\kern0pt}a\ A\ p\ q\ a{\isachardoublequoteclose}\isanewline
%
\isadelimproof
%
\endisadelimproof
%
\isatagproof
\isacommand{proof}\isamarkupfalse%
\ {\isacharminus}{\kern0pt}\isanewline
\ \ \isacommand{have}\isamarkupfalse%
\isanewline
\ \ \ \ {\isachardoublequoteopen}{\isasymforall}i{\isacharcolon}{\kern0pt}{\isacharcolon}{\kern0pt}nat{\isachardot}{\kern0pt}\ i\ {\isacharless}{\kern0pt}\ size\ p\ {\isasymlongrightarrow}\isanewline
\ \ \ \ \ \ equiv{\isacharunderscore}{\kern0pt}rel{\isacharunderscore}{\kern0pt}except{\isacharunderscore}{\kern0pt}a\ A\ {\isacharparenleft}{\kern0pt}p{\isacharbang}{\kern0pt}i{\isacharparenright}{\kern0pt}\ {\isacharparenleft}{\kern0pt}q{\isacharbang}{\kern0pt}i{\isacharparenright}{\kern0pt}\ a{\isachardoublequoteclose}\isanewline
\ \ \isacommand{proof}\isamarkupfalse%
\isanewline
\ \ \ \ \isacommand{fix}\isamarkupfalse%
\ i\ {\isacharcolon}{\kern0pt}{\isacharcolon}{\kern0pt}\ nat\isanewline
\ \ \ \ \isacommand{show}\isamarkupfalse%
\isanewline
\ \ \ \ \ \ {\isachardoublequoteopen}i\ {\isacharless}{\kern0pt}\ size\ p\ {\isasymlongrightarrow}\isanewline
\ \ \ \ \ \ \ \ equiv{\isacharunderscore}{\kern0pt}rel{\isacharunderscore}{\kern0pt}except{\isacharunderscore}{\kern0pt}a\ A\ {\isacharparenleft}{\kern0pt}p{\isacharbang}{\kern0pt}i{\isacharparenright}{\kern0pt}\ {\isacharparenleft}{\kern0pt}q{\isacharbang}{\kern0pt}i{\isacharparenright}{\kern0pt}\ a{\isachardoublequoteclose}\isanewline
\ \ \ \ \isacommand{proof}\isamarkupfalse%
\isanewline
\ \ \ \ \ \ \isacommand{assume}\isamarkupfalse%
\ i{\isacharunderscore}{\kern0pt}ok{\isacharcolon}{\kern0pt}\ {\isachardoublequoteopen}i\ {\isacharless}{\kern0pt}\ size\ p{\isachardoublequoteclose}\isanewline
\ \ \ \ \ \ \isacommand{show}\isamarkupfalse%
\ {\isachardoublequoteopen}equiv{\isacharunderscore}{\kern0pt}rel{\isacharunderscore}{\kern0pt}except{\isacharunderscore}{\kern0pt}a\ A\ {\isacharparenleft}{\kern0pt}p{\isacharbang}{\kern0pt}i{\isacharparenright}{\kern0pt}\ {\isacharparenleft}{\kern0pt}q{\isacharbang}{\kern0pt}i{\isacharparenright}{\kern0pt}\ a{\isachardoublequoteclose}\isanewline
\ \ \ \ \ \ \ \ \isacommand{using}\isamarkupfalse%
\ lifted{\isacharunderscore}{\kern0pt}def\ i{\isacharunderscore}{\kern0pt}ok\ lifted\ profile{\isacharunderscore}{\kern0pt}def\ trivial{\isacharunderscore}{\kern0pt}equiv{\isacharunderscore}{\kern0pt}rel\isanewline
\ \ \ \ \ \ \ \ \ \ \ \ \ \ lifted{\isacharunderscore}{\kern0pt}imp{\isacharunderscore}{\kern0pt}equiv{\isacharunderscore}{\kern0pt}rel{\isacharunderscore}{\kern0pt}except{\isacharunderscore}{\kern0pt}a\isanewline
\ \ \ \ \ \ \ \ \isacommand{by}\isamarkupfalse%
\ metis\isanewline
\ \ \ \ \isacommand{qed}\isamarkupfalse%
\isanewline
\ \ \isacommand{qed}\isamarkupfalse%
\isanewline
\ \ \isacommand{thus}\isamarkupfalse%
\ {\isacharquery}{\kern0pt}thesis\isanewline
\ \ \ \ \isacommand{using}\isamarkupfalse%
\ lifted{\isacharunderscore}{\kern0pt}def\ lifted\ equiv{\isacharunderscore}{\kern0pt}prof{\isacharunderscore}{\kern0pt}except{\isacharunderscore}{\kern0pt}a{\isacharunderscore}{\kern0pt}def\isanewline
\ \ \ \ \isacommand{by}\isamarkupfalse%
\ metis\isanewline
\isacommand{qed}\isamarkupfalse%
%
\endisatagproof
{\isafoldproof}%
%
\isadelimproof
\isanewline
%
\endisadelimproof
\isanewline
\isacommand{lemma}\isamarkupfalse%
\ negl{\isacharunderscore}{\kern0pt}diff{\isacharunderscore}{\kern0pt}imp{\isacharunderscore}{\kern0pt}eq{\isacharunderscore}{\kern0pt}limit{\isacharunderscore}{\kern0pt}prof{\isacharcolon}{\kern0pt}\isanewline
\ \ \isakeyword{assumes}\isanewline
\ \ \ \ change{\isacharcolon}{\kern0pt}\ {\isachardoublequoteopen}equiv{\isacharunderscore}{\kern0pt}prof{\isacharunderscore}{\kern0pt}except{\isacharunderscore}{\kern0pt}a\ S\ p\ q\ a{\isachardoublequoteclose}\ \isakeyword{and}\isanewline
\ \ \ \ subset{\isacharcolon}{\kern0pt}\ {\isachardoublequoteopen}A\ {\isasymsubseteq}\ S{\isachardoublequoteclose}\ \isakeyword{and}\isanewline
\ \ \ \ notInA{\isacharcolon}{\kern0pt}\ {\isachardoublequoteopen}a\ {\isasymnotin}\ A{\isachardoublequoteclose}\isanewline
\ \ \isakeyword{shows}\ {\isachardoublequoteopen}limit{\isacharunderscore}{\kern0pt}profile\ A\ p\ {\isacharequal}{\kern0pt}\ limit{\isacharunderscore}{\kern0pt}profile\ A\ q{\isachardoublequoteclose}\isanewline
%
\isadelimproof
%
\endisadelimproof
%
\isatagproof
\isacommand{proof}\isamarkupfalse%
\ {\isacharminus}{\kern0pt}\isanewline
\ \ \isacommand{have}\isamarkupfalse%
\isanewline
\ \ \ \ {\isachardoublequoteopen}{\isasymforall}i{\isacharcolon}{\kern0pt}{\isacharcolon}{\kern0pt}nat{\isachardot}{\kern0pt}\ i\ {\isacharless}{\kern0pt}\ size\ p\ {\isasymlongrightarrow}\isanewline
\ \ \ \ \ \ equiv{\isacharunderscore}{\kern0pt}rel{\isacharunderscore}{\kern0pt}except{\isacharunderscore}{\kern0pt}a\ S\ {\isacharparenleft}{\kern0pt}p{\isacharbang}{\kern0pt}i{\isacharparenright}{\kern0pt}\ {\isacharparenleft}{\kern0pt}q{\isacharbang}{\kern0pt}i{\isacharparenright}{\kern0pt}\ a{\isachardoublequoteclose}\isanewline
\ \ \ \ \isacommand{using}\isamarkupfalse%
\ change\ equiv{\isacharunderscore}{\kern0pt}prof{\isacharunderscore}{\kern0pt}except{\isacharunderscore}{\kern0pt}a{\isacharunderscore}{\kern0pt}def\isanewline
\ \ \ \ \isacommand{by}\isamarkupfalse%
\ metis\isanewline
\ \ \isacommand{hence}\isamarkupfalse%
\ {\isachardoublequoteopen}{\isasymforall}i{\isacharcolon}{\kern0pt}{\isacharcolon}{\kern0pt}nat{\isachardot}{\kern0pt}\ i\ {\isacharless}{\kern0pt}\ size\ p\ {\isasymlongrightarrow}\ limit\ A\ {\isacharparenleft}{\kern0pt}p{\isacharbang}{\kern0pt}i{\isacharparenright}{\kern0pt}\ {\isacharequal}{\kern0pt}\ limit\ A\ {\isacharparenleft}{\kern0pt}q{\isacharbang}{\kern0pt}i{\isacharparenright}{\kern0pt}{\isachardoublequoteclose}\isanewline
\ \ \ \ \isacommand{using}\isamarkupfalse%
\ notInA\ negl{\isacharunderscore}{\kern0pt}diff{\isacharunderscore}{\kern0pt}imp{\isacharunderscore}{\kern0pt}eq{\isacharunderscore}{\kern0pt}limit\ subset\isanewline
\ \ \ \ \isacommand{by}\isamarkupfalse%
\ metis\isanewline
\ \ \isacommand{hence}\isamarkupfalse%
\ {\isachardoublequoteopen}map\ {\isacharparenleft}{\kern0pt}limit\ A{\isacharparenright}{\kern0pt}\ p\ {\isacharequal}{\kern0pt}\ map\ {\isacharparenleft}{\kern0pt}limit\ A{\isacharparenright}{\kern0pt}\ q{\isachardoublequoteclose}\isanewline
\ \ \ \ \isacommand{using}\isamarkupfalse%
\ change\ equiv{\isacharunderscore}{\kern0pt}prof{\isacharunderscore}{\kern0pt}except{\isacharunderscore}{\kern0pt}a{\isacharunderscore}{\kern0pt}def\isanewline
\ \ \ \ \ \ \ \ \ \ length{\isacharunderscore}{\kern0pt}map\ nth{\isacharunderscore}{\kern0pt}equalityI\ nth{\isacharunderscore}{\kern0pt}map\isanewline
\ \ \ \ \isacommand{by}\isamarkupfalse%
\ {\isacharparenleft}{\kern0pt}metis\ {\isacharparenleft}{\kern0pt}mono{\isacharunderscore}{\kern0pt}tags{\isacharcomma}{\kern0pt}\ lifting{\isacharparenright}{\kern0pt}{\isacharparenright}{\kern0pt}\isanewline
\ \ \isacommand{thus}\isamarkupfalse%
\ {\isacharquery}{\kern0pt}thesis\isanewline
\ \ \ \ \isacommand{by}\isamarkupfalse%
\ simp\isanewline
\isacommand{qed}\isamarkupfalse%
%
\endisatagproof
{\isafoldproof}%
%
\isadelimproof
\isanewline
%
\endisadelimproof
\isanewline
\isacommand{lemma}\isamarkupfalse%
\ limit{\isacharunderscore}{\kern0pt}prof{\isacharunderscore}{\kern0pt}eq{\isacharunderscore}{\kern0pt}or{\isacharunderscore}{\kern0pt}lifted{\isacharcolon}{\kern0pt}\isanewline
\ \ \isakeyword{assumes}\isanewline
\ \ \ \ lifted{\isacharcolon}{\kern0pt}\ {\isachardoublequoteopen}lifted\ S\ p\ q\ a{\isachardoublequoteclose}\ \isakeyword{and}\isanewline
\ \ \ \ subset{\isacharcolon}{\kern0pt}\ {\isachardoublequoteopen}A\ {\isasymsubseteq}\ S{\isachardoublequoteclose}\isanewline
\ \ \isakeyword{shows}\isanewline
\ \ \ \ {\isachardoublequoteopen}limit{\isacharunderscore}{\kern0pt}profile\ A\ p\ {\isacharequal}{\kern0pt}\ limit{\isacharunderscore}{\kern0pt}profile\ A\ q\ {\isasymor}\isanewline
\ \ \ \ \ \ \ \ lifted\ A\ {\isacharparenleft}{\kern0pt}limit{\isacharunderscore}{\kern0pt}profile\ A\ p{\isacharparenright}{\kern0pt}\ {\isacharparenleft}{\kern0pt}limit{\isacharunderscore}{\kern0pt}profile\ A\ q{\isacharparenright}{\kern0pt}\ a{\isachardoublequoteclose}\isanewline
%
\isadelimproof
%
\endisadelimproof
%
\isatagproof
\isacommand{proof}\isamarkupfalse%
\ cases\isanewline
\ \ \isacommand{assume}\isamarkupfalse%
\ inA{\isacharcolon}{\kern0pt}\ {\isachardoublequoteopen}a\ {\isasymin}\ A{\isachardoublequoteclose}\isanewline
\ \ \isacommand{have}\isamarkupfalse%
\isanewline
\ \ \ \ {\isachardoublequoteopen}{\isasymforall}i{\isacharcolon}{\kern0pt}{\isacharcolon}{\kern0pt}nat{\isachardot}{\kern0pt}\ i\ {\isacharless}{\kern0pt}\ size\ p\ {\isasymlongrightarrow}\isanewline
\ \ \ \ \ \ \ \ {\isacharparenleft}{\kern0pt}Preference{\isacharunderscore}{\kern0pt}Relation{\isachardot}{\kern0pt}lifted\ S\ {\isacharparenleft}{\kern0pt}p{\isacharbang}{\kern0pt}i{\isacharparenright}{\kern0pt}\ {\isacharparenleft}{\kern0pt}q{\isacharbang}{\kern0pt}i{\isacharparenright}{\kern0pt}\ a\ {\isasymor}\ {\isacharparenleft}{\kern0pt}p{\isacharbang}{\kern0pt}i{\isacharparenright}{\kern0pt}\ {\isacharequal}{\kern0pt}\ {\isacharparenleft}{\kern0pt}q{\isacharbang}{\kern0pt}i{\isacharparenright}{\kern0pt}{\isacharparenright}{\kern0pt}{\isachardoublequoteclose}\isanewline
\ \ \ \ \isacommand{using}\isamarkupfalse%
\ lifted{\isacharunderscore}{\kern0pt}def\ lifted\isanewline
\ \ \ \ \isacommand{by}\isamarkupfalse%
\ metis\isanewline
\ \ \isacommand{hence}\isamarkupfalse%
\ one{\isacharcolon}{\kern0pt}\isanewline
\ \ \ \ {\isachardoublequoteopen}{\isasymforall}i{\isacharcolon}{\kern0pt}{\isacharcolon}{\kern0pt}nat{\isachardot}{\kern0pt}\ i\ {\isacharless}{\kern0pt}\ size\ p\ {\isasymlongrightarrow}\isanewline
\ \ \ \ \ \ \ \ \ {\isacharparenleft}{\kern0pt}Preference{\isacharunderscore}{\kern0pt}Relation{\isachardot}{\kern0pt}lifted\ A\ {\isacharparenleft}{\kern0pt}limit\ A\ {\isacharparenleft}{\kern0pt}p{\isacharbang}{\kern0pt}i{\isacharparenright}{\kern0pt}{\isacharparenright}{\kern0pt}\ {\isacharparenleft}{\kern0pt}limit\ A\ {\isacharparenleft}{\kern0pt}q{\isacharbang}{\kern0pt}i{\isacharparenright}{\kern0pt}{\isacharparenright}{\kern0pt}\ a\ {\isasymor}\isanewline
\ \ \ \ \ \ \ \ \ \ \ {\isacharparenleft}{\kern0pt}limit\ A\ {\isacharparenleft}{\kern0pt}p{\isacharbang}{\kern0pt}i{\isacharparenright}{\kern0pt}{\isacharparenright}{\kern0pt}\ {\isacharequal}{\kern0pt}\ {\isacharparenleft}{\kern0pt}limit\ A\ {\isacharparenleft}{\kern0pt}q{\isacharbang}{\kern0pt}i{\isacharparenright}{\kern0pt}{\isacharparenright}{\kern0pt}{\isacharparenright}{\kern0pt}{\isachardoublequoteclose}\isanewline
\ \ \ \ \isacommand{using}\isamarkupfalse%
\ limit{\isacharunderscore}{\kern0pt}lifted{\isacharunderscore}{\kern0pt}imp{\isacharunderscore}{\kern0pt}eq{\isacharunderscore}{\kern0pt}or{\isacharunderscore}{\kern0pt}lifted\ subset\isanewline
\ \ \ \ \isacommand{by}\isamarkupfalse%
\ metis\isanewline
\ \ \isacommand{thus}\isamarkupfalse%
\ {\isacharquery}{\kern0pt}thesis\isanewline
\ \ \isacommand{proof}\isamarkupfalse%
\ cases\isanewline
\ \ \ \ \isacommand{assume}\isamarkupfalse%
\ {\isachardoublequoteopen}{\isasymforall}i{\isacharcolon}{\kern0pt}{\isacharcolon}{\kern0pt}nat{\isachardot}{\kern0pt}\ i\ {\isacharless}{\kern0pt}\ size\ p\ {\isasymlongrightarrow}\ {\isacharparenleft}{\kern0pt}limit\ A\ {\isacharparenleft}{\kern0pt}p{\isacharbang}{\kern0pt}i{\isacharparenright}{\kern0pt}{\isacharparenright}{\kern0pt}\ {\isacharequal}{\kern0pt}\ {\isacharparenleft}{\kern0pt}limit\ A\ {\isacharparenleft}{\kern0pt}q{\isacharbang}{\kern0pt}i{\isacharparenright}{\kern0pt}{\isacharparenright}{\kern0pt}{\isachardoublequoteclose}\isanewline
\ \ \ \ \isacommand{thus}\isamarkupfalse%
\ {\isacharquery}{\kern0pt}thesis\isanewline
\ \ \ \ \ \ \isacommand{using}\isamarkupfalse%
\ lifted{\isacharunderscore}{\kern0pt}def\ length{\isacharunderscore}{\kern0pt}map\ lifted\isanewline
\ \ \ \ \ \ \ \ \ \ \ \ limit{\isacharunderscore}{\kern0pt}profile{\isachardot}{\kern0pt}simps\ nth{\isacharunderscore}{\kern0pt}equalityI\ nth{\isacharunderscore}{\kern0pt}map\isanewline
\ \ \ \ \ \ \isacommand{by}\isamarkupfalse%
\ {\isacharparenleft}{\kern0pt}metis\ {\isacharparenleft}{\kern0pt}mono{\isacharunderscore}{\kern0pt}tags{\isacharcomma}{\kern0pt}\ lifting{\isacharparenright}{\kern0pt}{\isacharparenright}{\kern0pt}\isanewline
\ \ \isacommand{next}\isamarkupfalse%
\isanewline
\ \ \ \ \isacommand{assume}\isamarkupfalse%
\ assm{\isacharcolon}{\kern0pt}\isanewline
\ \ \ \ \ \ {\isachardoublequoteopen}{\isasymnot}{\isacharparenleft}{\kern0pt}{\isasymforall}i{\isacharcolon}{\kern0pt}{\isacharcolon}{\kern0pt}nat{\isachardot}{\kern0pt}\ i\ {\isacharless}{\kern0pt}\ size\ p\ {\isasymlongrightarrow}\ {\isacharparenleft}{\kern0pt}limit\ A\ {\isacharparenleft}{\kern0pt}p{\isacharbang}{\kern0pt}i{\isacharparenright}{\kern0pt}{\isacharparenright}{\kern0pt}\ {\isacharequal}{\kern0pt}\ {\isacharparenleft}{\kern0pt}limit\ A\ {\isacharparenleft}{\kern0pt}q{\isacharbang}{\kern0pt}i{\isacharparenright}{\kern0pt}{\isacharparenright}{\kern0pt}{\isacharparenright}{\kern0pt}{\isachardoublequoteclose}\isanewline
\ \ \ \ \isacommand{let}\isamarkupfalse%
\ {\isacharquery}{\kern0pt}p\ {\isacharequal}{\kern0pt}\ {\isachardoublequoteopen}limit{\isacharunderscore}{\kern0pt}profile\ A\ p{\isachardoublequoteclose}\isanewline
\ \ \ \ \isacommand{let}\isamarkupfalse%
\ {\isacharquery}{\kern0pt}q\ {\isacharequal}{\kern0pt}\ {\isachardoublequoteopen}limit{\isacharunderscore}{\kern0pt}profile\ A\ q{\isachardoublequoteclose}\isanewline
\ \ \ \ \isacommand{have}\isamarkupfalse%
\ {\isachardoublequoteopen}profile\ A\ {\isacharquery}{\kern0pt}p\ {\isasymand}\ profile\ A\ {\isacharquery}{\kern0pt}q{\isachardoublequoteclose}\isanewline
\ \ \ \ \ \ \isacommand{using}\isamarkupfalse%
\ lifted{\isacharunderscore}{\kern0pt}def\ lifted\ limit{\isacharunderscore}{\kern0pt}profile{\isacharunderscore}{\kern0pt}sound\ subset\isanewline
\ \ \ \ \ \ \isacommand{by}\isamarkupfalse%
\ metis\isanewline
\ \ \ \ \isacommand{moreover}\isamarkupfalse%
\ \isacommand{have}\isamarkupfalse%
\ {\isachardoublequoteopen}size\ {\isacharquery}{\kern0pt}p\ {\isacharequal}{\kern0pt}\ size\ {\isacharquery}{\kern0pt}q{\isachardoublequoteclose}\isanewline
\ \ \ \ \ \ \isacommand{using}\isamarkupfalse%
\ lifted{\isacharunderscore}{\kern0pt}def\ lifted\isanewline
\ \ \ \ \ \ \isacommand{by}\isamarkupfalse%
\ fastforce\isanewline
\ \ \ \ \isacommand{moreover}\isamarkupfalse%
\ \isacommand{have}\isamarkupfalse%
\isanewline
\ \ \ \ \ \ {\isachardoublequoteopen}{\isasymexists}i{\isacharcolon}{\kern0pt}{\isacharcolon}{\kern0pt}nat{\isachardot}{\kern0pt}\ i\ {\isacharless}{\kern0pt}\ size\ {\isacharquery}{\kern0pt}p\ {\isasymand}\ Preference{\isacharunderscore}{\kern0pt}Relation{\isachardot}{\kern0pt}lifted\ A\ {\isacharparenleft}{\kern0pt}{\isacharquery}{\kern0pt}p{\isacharbang}{\kern0pt}i{\isacharparenright}{\kern0pt}\ {\isacharparenleft}{\kern0pt}{\isacharquery}{\kern0pt}q{\isacharbang}{\kern0pt}i{\isacharparenright}{\kern0pt}\ a{\isachardoublequoteclose}\isanewline
\ \ \ \ \ \ \isacommand{using}\isamarkupfalse%
\ assm\ lifted{\isacharunderscore}{\kern0pt}def\ length{\isacharunderscore}{\kern0pt}map\ lifted\isanewline
\ \ \ \ \ \ \ \ \ \ \ \ limit{\isacharunderscore}{\kern0pt}profile{\isachardot}{\kern0pt}simps\ nth{\isacharunderscore}{\kern0pt}map\ one\isanewline
\ \ \ \ \ \ \isacommand{by}\isamarkupfalse%
\ {\isacharparenleft}{\kern0pt}metis\ {\isacharparenleft}{\kern0pt}no{\isacharunderscore}{\kern0pt}types{\isacharcomma}{\kern0pt}\ lifting{\isacharparenright}{\kern0pt}{\isacharparenright}{\kern0pt}\isanewline
\ \ \ \ \isacommand{moreover}\isamarkupfalse%
\ \isacommand{have}\isamarkupfalse%
\isanewline
\ \ \ \ \ \ {\isachardoublequoteopen}{\isasymforall}i{\isacharcolon}{\kern0pt}{\isacharcolon}{\kern0pt}nat{\isachardot}{\kern0pt}\isanewline
\ \ \ \ \ \ \ \ {\isacharparenleft}{\kern0pt}i\ {\isacharless}{\kern0pt}\ size\ {\isacharquery}{\kern0pt}p\ {\isasymand}\ {\isasymnot}Preference{\isacharunderscore}{\kern0pt}Relation{\isachardot}{\kern0pt}lifted\ A\ {\isacharparenleft}{\kern0pt}{\isacharquery}{\kern0pt}p{\isacharbang}{\kern0pt}i{\isacharparenright}{\kern0pt}\ {\isacharparenleft}{\kern0pt}{\isacharquery}{\kern0pt}q{\isacharbang}{\kern0pt}i{\isacharparenright}{\kern0pt}\ a{\isacharparenright}{\kern0pt}\ {\isasymlongrightarrow}\isanewline
\ \ \ \ \ \ \ \ \ \ {\isacharparenleft}{\kern0pt}{\isacharquery}{\kern0pt}p{\isacharbang}{\kern0pt}i{\isacharparenright}{\kern0pt}\ {\isacharequal}{\kern0pt}\ {\isacharparenleft}{\kern0pt}{\isacharquery}{\kern0pt}q{\isacharbang}{\kern0pt}i{\isacharparenright}{\kern0pt}{\isachardoublequoteclose}\isanewline
\ \ \ \ \ \ \isacommand{using}\isamarkupfalse%
\ lifted{\isacharunderscore}{\kern0pt}def\ length{\isacharunderscore}{\kern0pt}map\ lifted\isanewline
\ \ \ \ \ \ \ \ \ \ \ \ limit{\isacharunderscore}{\kern0pt}profile{\isachardot}{\kern0pt}simps\ nth{\isacharunderscore}{\kern0pt}map\ one\isanewline
\ \ \ \ \ \ \isacommand{by}\isamarkupfalse%
\ metis\isanewline
\ \ \ \ \isacommand{ultimately}\isamarkupfalse%
\ \isacommand{have}\isamarkupfalse%
\ {\isachardoublequoteopen}lifted\ A\ {\isacharquery}{\kern0pt}p\ {\isacharquery}{\kern0pt}q\ a{\isachardoublequoteclose}\isanewline
\ \ \ \ \ \ \isacommand{using}\isamarkupfalse%
\ lifted{\isacharunderscore}{\kern0pt}def\ inA\ lifted\ rev{\isacharunderscore}{\kern0pt}finite{\isacharunderscore}{\kern0pt}subset\ subset\isanewline
\ \ \ \ \ \ \isacommand{by}\isamarkupfalse%
\ {\isacharparenleft}{\kern0pt}metis\ {\isacharparenleft}{\kern0pt}no{\isacharunderscore}{\kern0pt}types{\isacharcomma}{\kern0pt}\ lifting{\isacharparenright}{\kern0pt}{\isacharparenright}{\kern0pt}\isanewline
\ \ \ \ \isacommand{thus}\isamarkupfalse%
\ {\isacharquery}{\kern0pt}thesis\isanewline
\ \ \ \ \ \ \isacommand{by}\isamarkupfalse%
\ simp\isanewline
\ \ \isacommand{qed}\isamarkupfalse%
\isanewline
\isacommand{next}\isamarkupfalse%
\isanewline
\ \ \isacommand{assume}\isamarkupfalse%
\ {\isachardoublequoteopen}a\ {\isasymnotin}\ A{\isachardoublequoteclose}\isanewline
\ \ \isacommand{thus}\isamarkupfalse%
\ {\isacharquery}{\kern0pt}thesis\isanewline
\ \ \ \ \isacommand{using}\isamarkupfalse%
\ lifted\ negl{\isacharunderscore}{\kern0pt}diff{\isacharunderscore}{\kern0pt}imp{\isacharunderscore}{\kern0pt}eq{\isacharunderscore}{\kern0pt}limit{\isacharunderscore}{\kern0pt}prof\ subset\isanewline
\ \ \ \ \ \ \ \ \ \ lifted{\isacharunderscore}{\kern0pt}imp{\isacharunderscore}{\kern0pt}equiv{\isacharunderscore}{\kern0pt}prof{\isacharunderscore}{\kern0pt}except{\isacharunderscore}{\kern0pt}a\isanewline
\ \ \ \ \isacommand{by}\isamarkupfalse%
\ metis\isanewline
\isacommand{qed}\isamarkupfalse%
%
\endisatagproof
{\isafoldproof}%
%
\isadelimproof
\isanewline
%
\endisadelimproof
%
\isadelimtheory
\isanewline
%
\endisadelimtheory
%
\isatagtheory
\isacommand{end}\isamarkupfalse%
%
\endisatagtheory
{\isafoldtheory}%
%
\isadelimtheory
%
\endisadelimtheory
%
\end{isabellebody}%
\endinput
%:%file=~/Documents/Studies/VotingRuleGenerator/virage/src/test/resources/old_theories/Compositional_Structures/Basic_Modules/Component_Types/Social_Choice_Types/Profile.thy%:%
%:%6=3%:%
%:%11=4%:%
%:%12=5%:%
%:%13=6%:%
%:%15=9%:%
%:%31=11%:%
%:%32=11%:%
%:%33=12%:%
%:%34=13%:%
%:%43=16%:%
%:%44=17%:%
%:%45=18%:%
%:%46=19%:%
%:%47=20%:%
%:%48=21%:%
%:%49=22%:%
%:%58=24%:%
%:%68=27%:%
%:%69=27%:%
%:%70=28%:%
%:%71=32%:%
%:%72=33%:%
%:%73=33%:%
%:%74=34%:%
%:%75=35%:%
%:%76=36%:%
%:%77=36%:%
%:%80=37%:%
%:%84=37%:%
%:%85=37%:%
%:%90=37%:%
%:%93=38%:%
%:%94=39%:%
%:%95=39%:%
%:%96=40%:%
%:%103=42%:%
%:%113=48%:%
%:%114=48%:%
%:%115=49%:%
%:%116=50%:%
%:%117=51%:%
%:%118=52%:%
%:%119=52%:%
%:%120=53%:%
%:%121=54%:%
%:%122=55%:%
%:%123=56%:%
%:%124=57%:%
%:%125=57%:%
%:%126=58%:%
%:%127=59%:%
%:%128=60%:%
%:%129=61%:%
%:%130=61%:%
%:%131=62%:%
%:%132=63%:%
%:%133=64%:%
%:%134=65%:%
%:%135=66%:%
%:%136=66%:%
%:%139=67%:%
%:%143=67%:%
%:%144=67%:%
%:%149=67%:%
%:%152=68%:%
%:%153=69%:%
%:%154=69%:%
%:%155=70%:%
%:%158=71%:%
%:%162=71%:%
%:%163=71%:%
%:%168=71%:%
%:%171=72%:%
%:%172=73%:%
%:%173=73%:%
%:%174=74%:%
%:%175=75%:%
%:%176=76%:%
%:%177=77%:%
%:%178=78%:%
%:%185=79%:%
%:%186=79%:%
%:%187=80%:%
%:%188=80%:%
%:%189=81%:%
%:%190=81%:%
%:%191=82%:%
%:%192=82%:%
%:%193=83%:%
%:%195=85%:%
%:%196=86%:%
%:%197=86%:%
%:%198=87%:%
%:%199=87%:%
%:%200=88%:%
%:%201=88%:%
%:%202=89%:%
%:%203=89%:%
%:%204=90%:%
%:%205=90%:%
%:%206=91%:%
%:%207=91%:%
%:%208=92%:%
%:%209=92%:%
%:%210=93%:%
%:%211=94%:%
%:%212=94%:%
%:%213=95%:%
%:%214=95%:%
%:%215=96%:%
%:%216=96%:%
%:%217=96%:%
%:%218=97%:%
%:%219=98%:%
%:%220=98%:%
%:%221=99%:%
%:%222=99%:%
%:%223=100%:%
%:%224=100%:%
%:%225=101%:%
%:%226=101%:%
%:%227=102%:%
%:%228=102%:%
%:%229=103%:%
%:%230=103%:%
%:%231=104%:%
%:%232=105%:%
%:%233=105%:%
%:%234=106%:%
%:%235=106%:%
%:%236=106%:%
%:%237=107%:%
%:%238=108%:%
%:%239=109%:%
%:%240=109%:%
%:%241=110%:%
%:%242=110%:%
%:%243=111%:%
%:%244=112%:%
%:%245=113%:%
%:%246=113%:%
%:%247=114%:%
%:%248=114%:%
%:%249=115%:%
%:%250=116%:%
%:%251=117%:%
%:%252=117%:%
%:%253=118%:%
%:%254=118%:%
%:%255=119%:%
%:%256=119%:%
%:%257=120%:%
%:%258=120%:%
%:%259=121%:%
%:%260=121%:%
%:%261=122%:%
%:%264=125%:%
%:%265=126%:%
%:%266=126%:%
%:%267=127%:%
%:%268=127%:%
%:%269=128%:%
%:%270=129%:%
%:%271=129%:%
%:%272=130%:%
%:%273=130%:%
%:%274=130%:%
%:%275=131%:%
%:%276=132%:%
%:%277=133%:%
%:%278=133%:%
%:%279=134%:%
%:%280=134%:%
%:%281=135%:%
%:%282=135%:%
%:%283=135%:%
%:%284=136%:%
%:%285=137%:%
%:%286=138%:%
%:%287=138%:%
%:%288=139%:%
%:%289=139%:%
%:%290=140%:%
%:%291=140%:%
%:%292=140%:%
%:%293=141%:%
%:%294=142%:%
%:%295=142%:%
%:%296=143%:%
%:%297=143%:%
%:%298=143%:%
%:%299=144%:%
%:%300=144%:%
%:%301=145%:%
%:%307=145%:%
%:%310=146%:%
%:%311=147%:%
%:%312=147%:%
%:%313=148%:%
%:%314=149%:%
%:%315=150%:%
%:%316=151%:%
%:%317=152%:%
%:%318=153%:%
%:%319=154%:%
%:%320=155%:%
%:%327=156%:%
%:%328=156%:%
%:%329=157%:%
%:%330=157%:%
%:%331=157%:%
%:%332=158%:%
%:%333=159%:%
%:%334=159%:%
%:%335=160%:%
%:%336=160%:%
%:%337=161%:%
%:%338=161%:%
%:%339=161%:%
%:%340=161%:%
%:%341=162%:%
%:%342=163%:%
%:%343=163%:%
%:%344=164%:%
%:%345=164%:%
%:%346=165%:%
%:%347=165%:%
%:%348=166%:%
%:%349=167%:%
%:%350=167%:%
%:%351=168%:%
%:%352=168%:%
%:%353=169%:%
%:%354=169%:%
%:%355=170%:%
%:%356=171%:%
%:%357=171%:%
%:%358=172%:%
%:%359=173%:%
%:%360=173%:%
%:%361=174%:%
%:%362=174%:%
%:%363=175%:%
%:%364=175%:%
%:%365=176%:%
%:%366=176%:%
%:%367=177%:%
%:%368=177%:%
%:%369=178%:%
%:%370=178%:%
%:%371=179%:%
%:%377=179%:%
%:%380=180%:%
%:%381=181%:%
%:%382=181%:%
%:%383=182%:%
%:%384=183%:%
%:%387=184%:%
%:%391=184%:%
%:%392=184%:%
%:%393=185%:%
%:%394=185%:%
%:%399=185%:%
%:%402=186%:%
%:%403=187%:%
%:%404=187%:%
%:%405=188%:%
%:%406=189%:%
%:%407=190%:%
%:%408=191%:%
%:%411=192%:%
%:%415=192%:%
%:%416=192%:%
%:%417=193%:%
%:%418=193%:%
%:%423=193%:%
%:%426=194%:%
%:%427=195%:%
%:%428=195%:%
%:%429=196%:%
%:%430=197%:%
%:%431=198%:%
%:%432=199%:%
%:%435=200%:%
%:%439=200%:%
%:%440=200%:%
%:%441=201%:%
%:%442=201%:%
%:%447=201%:%
%:%450=202%:%
%:%451=203%:%
%:%452=203%:%
%:%453=204%:%
%:%454=205%:%
%:%455=206%:%
%:%456=207%:%
%:%457=207%:%
%:%458=208%:%
%:%459=209%:%
%:%460=210%:%
%:%461=211%:%
%:%462=212%:%
%:%463=212%:%
%:%464=213%:%
%:%465=214%:%
%:%468=215%:%
%:%472=215%:%
%:%473=215%:%
%:%474=216%:%
%:%475=216%:%
%:%480=216%:%
%:%483=217%:%
%:%484=218%:%
%:%485=218%:%
%:%488=219%:%
%:%492=219%:%
%:%493=219%:%
%:%494=220%:%
%:%495=220%:%
%:%509=222%:%
%:%519=224%:%
%:%520=224%:%
%:%521=225%:%
%:%522=226%:%
%:%523=227%:%
%:%524=228%:%
%:%525=228%:%
%:%526=229%:%
%:%528=231%:%
%:%529=232%:%
%:%530=233%:%
%:%531=233%:%
%:%532=234%:%
%:%533=235%:%
%:%534=236%:%
%:%541=237%:%
%:%542=237%:%
%:%543=238%:%
%:%544=238%:%
%:%545=239%:%
%:%546=240%:%
%:%547=240%:%
%:%548=241%:%
%:%549=241%:%
%:%550=242%:%
%:%551=242%:%
%:%552=243%:%
%:%553=243%:%
%:%554=244%:%
%:%555=244%:%
%:%556=245%:%
%:%557=245%:%
%:%558=246%:%
%:%559=246%:%
%:%560=246%:%
%:%561=247%:%
%:%562=247%:%
%:%563=248%:%
%:%564=249%:%
%:%565=249%:%
%:%566=250%:%
%:%567=251%:%
%:%568=251%:%
%:%569=252%:%
%:%570=252%:%
%:%571=252%:%
%:%572=253%:%
%:%573=253%:%
%:%574=254%:%
%:%580=254%:%
%:%583=255%:%
%:%584=256%:%
%:%585=256%:%
%:%586=257%:%
%:%587=258%:%
%:%588=259%:%
%:%589=260%:%
%:%592=261%:%
%:%596=261%:%
%:%597=261%:%
%:%598=262%:%
%:%599=262%:%
%:%604=262%:%
%:%607=263%:%
%:%608=264%:%
%:%609=264%:%
%:%610=265%:%
%:%611=266%:%
%:%618=267%:%
%:%619=267%:%
%:%620=268%:%
%:%621=268%:%
%:%622=269%:%
%:%623=270%:%
%:%624=270%:%
%:%625=271%:%
%:%626=272%:%
%:%627=273%:%
%:%628=274%:%
%:%629=275%:%
%:%631=277%:%
%:%632=278%:%
%:%633=278%:%
%:%634=278%:%
%:%635=279%:%
%:%638=282%:%
%:%639=283%:%
%:%640=283%:%
%:%641=284%:%
%:%642=284%:%
%:%643=285%:%
%:%645=287%:%
%:%646=288%:%
%:%647=288%:%
%:%648=289%:%
%:%649=289%:%
%:%650=290%:%
%:%652=292%:%
%:%653=293%:%
%:%654=293%:%
%:%655=294%:%
%:%656=294%:%
%:%657=295%:%
%:%658=295%:%
%:%659=295%:%
%:%660=295%:%
%:%661=296%:%
%:%662=297%:%
%:%663=297%:%
%:%664=298%:%
%:%665=298%:%
%:%666=298%:%
%:%667=298%:%
%:%668=299%:%
%:%669=300%:%
%:%670=300%:%
%:%671=301%:%
%:%672=301%:%
%:%673=302%:%
%:%675=304%:%
%:%676=305%:%
%:%677=305%:%
%:%678=306%:%
%:%679=306%:%
%:%680=307%:%
%:%681=307%:%
%:%682=307%:%
%:%683=308%:%
%:%684=309%:%
%:%685=309%:%
%:%686=310%:%
%:%687=310%:%
%:%688=311%:%
%:%689=311%:%
%:%690=312%:%
%:%691=313%:%
%:%692=313%:%
%:%693=313%:%
%:%694=314%:%
%:%695=314%:%
%:%696=315%:%
%:%697=315%:%
%:%698=316%:%
%:%699=316%:%
%:%700=317%:%
%:%701=318%:%
%:%702=318%:%
%:%703=318%:%
%:%704=319%:%
%:%705=319%:%
%:%706=320%:%
%:%707=320%:%
%:%708=321%:%
%:%709=321%:%
%:%710=322%:%
%:%711=323%:%
%:%712=323%:%
%:%713=323%:%
%:%714=324%:%
%:%715=324%:%
%:%716=325%:%
%:%717=325%:%
%:%718=326%:%
%:%719=326%:%
%:%720=327%:%
%:%721=328%:%
%:%722=328%:%
%:%723=328%:%
%:%724=329%:%
%:%725=329%:%
%:%726=330%:%
%:%727=330%:%
%:%728=331%:%
%:%729=331%:%
%:%730=332%:%
%:%731=333%:%
%:%732=334%:%
%:%733=334%:%
%:%734=335%:%
%:%735=336%:%
%:%736=337%:%
%:%737=338%:%
%:%738=339%:%
%:%739=339%:%
%:%740=339%:%
%:%741=340%:%
%:%744=343%:%
%:%745=344%:%
%:%746=344%:%
%:%747=345%:%
%:%748=345%:%
%:%749=346%:%
%:%750=346%:%
%:%751=347%:%
%:%752=347%:%
%:%753=348%:%
%:%768=350%:%
%:%778=356%:%
%:%779=356%:%
%:%780=357%:%
%:%781=358%:%
%:%782=359%:%
%:%783=359%:%
%:%784=360%:%
%:%785=361%:%
%:%786=362%:%
%:%787=363%:%
%:%788=364%:%
%:%791=365%:%
%:%795=365%:%
%:%796=365%:%
%:%797=366%:%
%:%798=366%:%
%:%803=366%:%
%:%806=367%:%
%:%807=368%:%
%:%808=368%:%
%:%809=369%:%
%:%810=370%:%
%:%811=371%:%
%:%812=372%:%
%:%819=373%:%
%:%820=373%:%
%:%821=374%:%
%:%822=374%:%
%:%823=375%:%
%:%824=375%:%
%:%825=376%:%
%:%826=376%:%
%:%827=377%:%
%:%828=378%:%
%:%829=378%:%
%:%830=379%:%
%:%836=379%:%
%:%839=380%:%
%:%840=381%:%
%:%841=381%:%
%:%842=382%:%
%:%843=383%:%
%:%844=384%:%
%:%847=385%:%
%:%851=385%:%
%:%852=385%:%
%:%866=387%:%
%:%876=389%:%
%:%877=389%:%
%:%878=390%:%
%:%879=391%:%
%:%883=395%:%
%:%884=396%:%
%:%885=400%:%
%:%886=401%:%
%:%887=401%:%
%:%888=402%:%
%:%893=407%:%
%:%894=408%:%
%:%895=409%:%
%:%896=409%:%
%:%897=410%:%
%:%898=411%:%
%:%905=412%:%
%:%906=412%:%
%:%907=413%:%
%:%908=413%:%
%:%909=414%:%
%:%910=415%:%
%:%911=416%:%
%:%912=416%:%
%:%913=417%:%
%:%914=417%:%
%:%915=418%:%
%:%916=418%:%
%:%917=419%:%
%:%918=420%:%
%:%919=421%:%
%:%920=421%:%
%:%921=422%:%
%:%922=422%:%
%:%923=423%:%
%:%924=423%:%
%:%925=424%:%
%:%926=424%:%
%:%927=425%:%
%:%928=426%:%
%:%929=426%:%
%:%930=427%:%
%:%931=427%:%
%:%932=428%:%
%:%933=428%:%
%:%934=429%:%
%:%935=429%:%
%:%936=430%:%
%:%937=430%:%
%:%938=431%:%
%:%939=431%:%
%:%940=432%:%
%:%946=432%:%
%:%949=433%:%
%:%950=434%:%
%:%951=434%:%
%:%952=435%:%
%:%953=436%:%
%:%954=437%:%
%:%955=438%:%
%:%956=439%:%
%:%963=440%:%
%:%964=440%:%
%:%965=441%:%
%:%966=441%:%
%:%967=442%:%
%:%968=443%:%
%:%969=444%:%
%:%970=444%:%
%:%971=445%:%
%:%972=445%:%
%:%973=446%:%
%:%974=446%:%
%:%975=447%:%
%:%976=447%:%
%:%977=448%:%
%:%978=448%:%
%:%979=449%:%
%:%980=449%:%
%:%981=450%:%
%:%982=450%:%
%:%983=451%:%
%:%984=452%:%
%:%985=452%:%
%:%986=453%:%
%:%987=453%:%
%:%988=454%:%
%:%989=454%:%
%:%990=455%:%
%:%996=455%:%
%:%999=456%:%
%:%1000=457%:%
%:%1001=457%:%
%:%1002=458%:%
%:%1003=459%:%
%:%1004=460%:%
%:%1005=461%:%
%:%1006=462%:%
%:%1007=463%:%
%:%1014=464%:%
%:%1015=464%:%
%:%1016=465%:%
%:%1017=465%:%
%:%1018=466%:%
%:%1019=466%:%
%:%1020=467%:%
%:%1021=468%:%
%:%1022=469%:%
%:%1023=469%:%
%:%1024=470%:%
%:%1025=470%:%
%:%1026=471%:%
%:%1027=471%:%
%:%1028=472%:%
%:%1030=474%:%
%:%1031=475%:%
%:%1032=475%:%
%:%1033=476%:%
%:%1034=476%:%
%:%1035=477%:%
%:%1036=477%:%
%:%1037=478%:%
%:%1038=478%:%
%:%1039=479%:%
%:%1040=479%:%
%:%1041=480%:%
%:%1042=480%:%
%:%1043=481%:%
%:%1044=481%:%
%:%1045=482%:%
%:%1046=483%:%
%:%1047=483%:%
%:%1048=484%:%
%:%1049=484%:%
%:%1050=485%:%
%:%1051=485%:%
%:%1052=486%:%
%:%1053=487%:%
%:%1054=487%:%
%:%1055=488%:%
%:%1056=488%:%
%:%1057=489%:%
%:%1058=489%:%
%:%1059=490%:%
%:%1060=490%:%
%:%1061=491%:%
%:%1062=491%:%
%:%1063=492%:%
%:%1064=492%:%
%:%1065=492%:%
%:%1066=493%:%
%:%1067=493%:%
%:%1068=494%:%
%:%1069=494%:%
%:%1070=495%:%
%:%1071=495%:%
%:%1072=495%:%
%:%1073=496%:%
%:%1074=497%:%
%:%1075=497%:%
%:%1076=498%:%
%:%1077=499%:%
%:%1078=499%:%
%:%1079=500%:%
%:%1080=500%:%
%:%1081=500%:%
%:%1082=501%:%
%:%1084=503%:%
%:%1085=504%:%
%:%1086=504%:%
%:%1087=505%:%
%:%1088=506%:%
%:%1089=506%:%
%:%1090=507%:%
%:%1091=507%:%
%:%1092=507%:%
%:%1093=508%:%
%:%1094=508%:%
%:%1095=509%:%
%:%1096=509%:%
%:%1097=510%:%
%:%1098=510%:%
%:%1099=511%:%
%:%1100=511%:%
%:%1101=512%:%
%:%1102=512%:%
%:%1103=513%:%
%:%1104=513%:%
%:%1105=514%:%
%:%1106=514%:%
%:%1107=515%:%
%:%1108=515%:%
%:%1109=516%:%
%:%1110=516%:%
%:%1111=517%:%
%:%1112=518%:%
%:%1113=518%:%
%:%1114=519%:%
%:%1120=519%:%
%:%1125=520%:%
%:%1130=521%:%
%
\begin{isabellebody}%
\setisabellecontext{Electoral{\isacharunderscore}{\kern0pt}Module}%
%
\isadelimdocument
\isanewline
%
\endisadelimdocument
%
\isatagdocument
\isanewline
\isanewline
\isanewline
%
\isamarkupchapter{Component Types%
}
\isamarkuptrue%
%
\isamarkupsection{Electoral Module%
}
\isamarkuptrue%
%
\endisatagdocument
{\isafolddocument}%
%
\isadelimdocument
%
\endisadelimdocument
%
\isadelimtheory
%
\endisadelimtheory
%
\isatagtheory
\isacommand{theory}\isamarkupfalse%
\ Electoral{\isacharunderscore}{\kern0pt}Module\isanewline
\ \ \isakeyword{imports}\ {\isachardoublequoteopen}{\isachardot}{\kern0pt}{\isachardot}{\kern0pt}{\isacharslash}{\kern0pt}{\isachardot}{\kern0pt}{\isachardot}{\kern0pt}{\isacharslash}{\kern0pt}Social{\isacharunderscore}{\kern0pt}Choice{\isacharunderscore}{\kern0pt}Types{\isacharslash}{\kern0pt}Preference{\isacharunderscore}{\kern0pt}Relation{\isachardoublequoteclose}\isanewline
\ \ \ \ \ \ \ \ \ \ {\isachardoublequoteopen}{\isachardot}{\kern0pt}{\isachardot}{\kern0pt}{\isacharslash}{\kern0pt}{\isachardot}{\kern0pt}{\isachardot}{\kern0pt}{\isacharslash}{\kern0pt}Social{\isacharunderscore}{\kern0pt}Choice{\isacharunderscore}{\kern0pt}Types{\isacharslash}{\kern0pt}Profile{\isachardoublequoteclose}\isanewline
\ \ \ \ \ \ \ \ \ \ {\isachardoublequoteopen}{\isachardot}{\kern0pt}{\isachardot}{\kern0pt}{\isacharslash}{\kern0pt}{\isachardot}{\kern0pt}{\isachardot}{\kern0pt}{\isacharslash}{\kern0pt}Social{\isacharunderscore}{\kern0pt}Choice{\isacharunderscore}{\kern0pt}Types{\isacharslash}{\kern0pt}Result{\isachardoublequoteclose}\isanewline
\isanewline
\isakeyword{begin}%
\endisatagtheory
{\isafoldtheory}%
%
\isadelimtheory
%
\endisadelimtheory
%
\begin{isamarkuptext}%
Electoral modules are the principal component type of the composable modules
voting framework, as they are a generalization of voting rules in the sense of
social choice functions.
These are only the types used for electoral modules. Further restrictions are
encompassed by the electoral-module predicate.

An electoral module does not need to make final decisions for all alternatives,
but can instead defer the decision for some or all of them to other modules.
Hence, electoral modules partition the received (possibly empty) set of
alternatives into elected, rejected and deferred alternatives. In particular,
any of those sets, e.g., the set of winning (elected) alternatives, may also
be left empty, as long as they collectively still hold all the received
alternatives. Just like a voting rule, an electoral module also receives a
profile which holds the voters’ preferences, which, unlike a voting rule,
consider only the (sub-)set of alternatives that the module receives.%
\end{isamarkuptext}\isamarkuptrue%
%
\isadelimdocument
%
\endisadelimdocument
%
\isatagdocument
%
\isamarkupsubsection{Definition%
}
\isamarkuptrue%
%
\endisatagdocument
{\isafolddocument}%
%
\isadelimdocument
%
\endisadelimdocument
\isacommand{type{\isacharunderscore}{\kern0pt}synonym}\isamarkupfalse%
\ {\isacharprime}{\kern0pt}a\ Electoral{\isacharunderscore}{\kern0pt}Module\ {\isacharequal}{\kern0pt}\ {\isachardoublequoteopen}{\isacharprime}{\kern0pt}a\ set\ {\isasymRightarrow}\ {\isacharprime}{\kern0pt}a\ Profile\ {\isasymRightarrow}\ {\isacharprime}{\kern0pt}a\ Result{\isachardoublequoteclose}%
\isadelimdocument
%
\endisadelimdocument
%
\isatagdocument
%
\isamarkupsubsection{Auxiliary Definitions%
}
\isamarkuptrue%
%
\endisatagdocument
{\isafolddocument}%
%
\isadelimdocument
%
\endisadelimdocument
\isacommand{definition}\isamarkupfalse%
\ electoral{\isacharunderscore}{\kern0pt}module\ {\isacharcolon}{\kern0pt}{\isacharcolon}{\kern0pt}\ {\isachardoublequoteopen}\ {\isacharprime}{\kern0pt}a\ Electoral{\isacharunderscore}{\kern0pt}Module\ {\isasymRightarrow}\ bool{\isachardoublequoteclose}\ \isakeyword{where}\isanewline
\ \ {\isachardoublequoteopen}electoral{\isacharunderscore}{\kern0pt}module\ m\ {\isasymequiv}\ {\isasymforall}A\ p{\isachardot}{\kern0pt}\ finite{\isacharunderscore}{\kern0pt}profile\ A\ p\ {\isasymlongrightarrow}\ well{\isacharunderscore}{\kern0pt}formed\ A\ {\isacharparenleft}{\kern0pt}m\ A\ p{\isacharparenright}{\kern0pt}{\isachardoublequoteclose}\isanewline
\isanewline
\isacommand{lemma}\isamarkupfalse%
\ electoral{\isacharunderscore}{\kern0pt}modI{\isacharcolon}{\kern0pt}\isanewline
\ \ {\isachardoublequoteopen}{\isacharparenleft}{\kern0pt}{\isacharparenleft}{\kern0pt}{\isasymAnd}A\ p{\isachardot}{\kern0pt}\ {\isasymlbrakk}\ finite{\isacharunderscore}{\kern0pt}profile\ A\ p\ {\isasymrbrakk}\ {\isasymLongrightarrow}\ well{\isacharunderscore}{\kern0pt}formed\ A\ {\isacharparenleft}{\kern0pt}m\ A\ p{\isacharparenright}{\kern0pt}{\isacharparenright}{\kern0pt}\ {\isasymLongrightarrow}\isanewline
\ \ \ \ \ \ \ electoral{\isacharunderscore}{\kern0pt}module\ m{\isacharparenright}{\kern0pt}{\isachardoublequoteclose}\isanewline
%
\isadelimproof
\ \ %
\endisadelimproof
%
\isatagproof
\isacommand{unfolding}\isamarkupfalse%
\ electoral{\isacharunderscore}{\kern0pt}module{\isacharunderscore}{\kern0pt}def\isanewline
\ \ \isacommand{by}\isamarkupfalse%
\ auto%
\endisatagproof
{\isafoldproof}%
%
\isadelimproof
\isanewline
%
\endisadelimproof
\isanewline
\isanewline
\isacommand{abbreviation}\isamarkupfalse%
\ elect\ {\isacharcolon}{\kern0pt}{\isacharcolon}{\kern0pt}\isanewline
\ \ {\isachardoublequoteopen}{\isacharprime}{\kern0pt}a\ Electoral{\isacharunderscore}{\kern0pt}Module\ {\isasymRightarrow}\ {\isacharprime}{\kern0pt}a\ set\ {\isasymRightarrow}\ {\isacharprime}{\kern0pt}a\ Profile\ {\isasymRightarrow}\ {\isacharprime}{\kern0pt}a\ set{\isachardoublequoteclose}\ \isakeyword{where}\isanewline
\ \ {\isachardoublequoteopen}elect\ m\ A\ p\ {\isasymequiv}\ elect{\isacharunderscore}{\kern0pt}r\ {\isacharparenleft}{\kern0pt}m\ A\ p{\isacharparenright}{\kern0pt}{\isachardoublequoteclose}\isanewline
\isanewline
\isacommand{abbreviation}\isamarkupfalse%
\ reject\ {\isacharcolon}{\kern0pt}{\isacharcolon}{\kern0pt}\isanewline
\ \ {\isachardoublequoteopen}{\isacharprime}{\kern0pt}a\ Electoral{\isacharunderscore}{\kern0pt}Module\ {\isasymRightarrow}\ {\isacharprime}{\kern0pt}a\ set\ {\isasymRightarrow}\ {\isacharprime}{\kern0pt}a\ Profile\ {\isasymRightarrow}\ {\isacharprime}{\kern0pt}a\ set{\isachardoublequoteclose}\ \isakeyword{where}\isanewline
\ \ {\isachardoublequoteopen}reject\ m\ A\ p\ {\isasymequiv}\ reject{\isacharunderscore}{\kern0pt}r\ {\isacharparenleft}{\kern0pt}m\ A\ p{\isacharparenright}{\kern0pt}{\isachardoublequoteclose}\isanewline
\isanewline
\isacommand{abbreviation}\isamarkupfalse%
\ {\isachardoublequoteopen}defer{\isachardoublequoteclose}\ {\isacharcolon}{\kern0pt}{\isacharcolon}{\kern0pt}\isanewline
\ \ {\isachardoublequoteopen}{\isacharprime}{\kern0pt}a\ Electoral{\isacharunderscore}{\kern0pt}Module\ {\isasymRightarrow}\ {\isacharprime}{\kern0pt}a\ set\ {\isasymRightarrow}\ {\isacharprime}{\kern0pt}a\ Profile\ {\isasymRightarrow}\ {\isacharprime}{\kern0pt}a\ set{\isachardoublequoteclose}\ \isakeyword{where}\isanewline
\ \ {\isachardoublequoteopen}defer\ m\ A\ p\ {\isasymequiv}\ defer{\isacharunderscore}{\kern0pt}r\ {\isacharparenleft}{\kern0pt}m\ A\ p{\isacharparenright}{\kern0pt}{\isachardoublequoteclose}%
\isadelimdocument
%
\endisadelimdocument
%
\isatagdocument
%
\isamarkupsubsection{Equivalence Definitions%
}
\isamarkuptrue%
%
\endisatagdocument
{\isafolddocument}%
%
\isadelimdocument
%
\endisadelimdocument
\isacommand{definition}\isamarkupfalse%
\ prof{\isacharunderscore}{\kern0pt}contains{\isacharunderscore}{\kern0pt}result\ {\isacharcolon}{\kern0pt}{\isacharcolon}{\kern0pt}\ {\isachardoublequoteopen}{\isacharprime}{\kern0pt}a\ Electoral{\isacharunderscore}{\kern0pt}Module\ {\isasymRightarrow}\ {\isacharprime}{\kern0pt}a\ set\ {\isasymRightarrow}\ {\isacharprime}{\kern0pt}a\ Profile\ {\isasymRightarrow}\isanewline
\ \ \ \ \ \ \ \ \ \ \ \ \ \ \ \ \ \ \ \ \ \ \ \ \ \ \ \ \ \ \ \ \ \ \ \ \ \ \ {\isacharprime}{\kern0pt}a\ Profile\ {\isasymRightarrow}\ {\isacharprime}{\kern0pt}a\ {\isasymRightarrow}\ bool{\isachardoublequoteclose}\ \isakeyword{where}\isanewline
\ \ {\isachardoublequoteopen}prof{\isacharunderscore}{\kern0pt}contains{\isacharunderscore}{\kern0pt}result\ m\ A\ p\ q\ a\ {\isasymequiv}\isanewline
\ \ \ \ electoral{\isacharunderscore}{\kern0pt}module\ m\ {\isasymand}\ finite{\isacharunderscore}{\kern0pt}profile\ A\ p\ {\isasymand}\ finite{\isacharunderscore}{\kern0pt}profile\ A\ q\ {\isasymand}\ a\ {\isasymin}\ A\ {\isasymand}\isanewline
\ \ \ \ {\isacharparenleft}{\kern0pt}a\ {\isasymin}\ elect\ m\ A\ p\ {\isasymlongrightarrow}\ a\ {\isasymin}\ elect\ m\ A\ q{\isacharparenright}{\kern0pt}\ {\isasymand}\isanewline
\ \ \ \ {\isacharparenleft}{\kern0pt}a\ {\isasymin}\ reject\ m\ A\ p\ {\isasymlongrightarrow}\ a\ {\isasymin}\ reject\ m\ A\ q{\isacharparenright}{\kern0pt}\ {\isasymand}\isanewline
\ \ \ \ {\isacharparenleft}{\kern0pt}a\ {\isasymin}\ defer\ m\ A\ p\ {\isasymlongrightarrow}\ a\ {\isasymin}\ defer\ m\ A\ q{\isacharparenright}{\kern0pt}{\isachardoublequoteclose}\isanewline
\isanewline
\isacommand{definition}\isamarkupfalse%
\ prof{\isacharunderscore}{\kern0pt}leq{\isacharunderscore}{\kern0pt}result\ {\isacharcolon}{\kern0pt}{\isacharcolon}{\kern0pt}\ {\isachardoublequoteopen}{\isacharprime}{\kern0pt}a\ Electoral{\isacharunderscore}{\kern0pt}Module\ {\isasymRightarrow}\ {\isacharprime}{\kern0pt}a\ set\ {\isasymRightarrow}\ {\isacharprime}{\kern0pt}a\ Profile\ {\isasymRightarrow}\isanewline
\ \ \ \ \ \ \ \ \ \ \ \ \ \ \ \ \ \ \ \ \ \ \ \ \ \ \ \ \ \ \ \ \ \ {\isacharprime}{\kern0pt}a\ Profile\ {\isasymRightarrow}\ {\isacharprime}{\kern0pt}a\ {\isasymRightarrow}\ bool{\isachardoublequoteclose}\ \isakeyword{where}\isanewline
\ \ {\isachardoublequoteopen}prof{\isacharunderscore}{\kern0pt}leq{\isacharunderscore}{\kern0pt}result\ m\ A\ p\ q\ a\ {\isasymequiv}\isanewline
\ \ \ \ electoral{\isacharunderscore}{\kern0pt}module\ m\ {\isasymand}\ finite{\isacharunderscore}{\kern0pt}profile\ A\ p\ {\isasymand}\ finite{\isacharunderscore}{\kern0pt}profile\ A\ q\ {\isasymand}\ a\ {\isasymin}\ A\ {\isasymand}\isanewline
\ \ \ \ {\isacharparenleft}{\kern0pt}a\ {\isasymin}\ reject\ m\ A\ p\ {\isasymlongrightarrow}\ a\ {\isasymin}\ reject\ m\ A\ q{\isacharparenright}{\kern0pt}\ {\isasymand}\isanewline
\ \ \ \ {\isacharparenleft}{\kern0pt}a\ {\isasymin}\ defer\ m\ A\ p\ {\isasymlongrightarrow}\ a\ {\isasymnotin}\ elect\ m\ A\ q{\isacharparenright}{\kern0pt}{\isachardoublequoteclose}\isanewline
\isanewline
\isacommand{definition}\isamarkupfalse%
\ prof{\isacharunderscore}{\kern0pt}geq{\isacharunderscore}{\kern0pt}result\ {\isacharcolon}{\kern0pt}{\isacharcolon}{\kern0pt}\ {\isachardoublequoteopen}{\isacharprime}{\kern0pt}a\ Electoral{\isacharunderscore}{\kern0pt}Module\ {\isasymRightarrow}\ {\isacharprime}{\kern0pt}a\ set\ {\isasymRightarrow}\ {\isacharprime}{\kern0pt}a\ Profile\ {\isasymRightarrow}\isanewline
\ \ \ \ \ \ \ \ \ \ \ \ \ \ \ \ \ \ \ \ \ \ \ \ \ \ \ \ \ \ \ \ \ \ {\isacharprime}{\kern0pt}a\ Profile\ {\isasymRightarrow}\ {\isacharprime}{\kern0pt}a\ {\isasymRightarrow}\ bool{\isachardoublequoteclose}\ \isakeyword{where}\isanewline
\ \ {\isachardoublequoteopen}prof{\isacharunderscore}{\kern0pt}geq{\isacharunderscore}{\kern0pt}result\ m\ A\ p\ q\ a\ {\isasymequiv}\isanewline
\ \ \ \ electoral{\isacharunderscore}{\kern0pt}module\ m\ {\isasymand}\ finite{\isacharunderscore}{\kern0pt}profile\ A\ p\ {\isasymand}\ finite{\isacharunderscore}{\kern0pt}profile\ A\ q\ {\isasymand}\ a\ {\isasymin}\ A\ {\isasymand}\isanewline
\ \ \ \ {\isacharparenleft}{\kern0pt}a\ {\isasymin}\ elect\ m\ A\ p\ {\isasymlongrightarrow}\ a\ {\isasymin}\ elect\ m\ A\ q{\isacharparenright}{\kern0pt}\ {\isasymand}\isanewline
\ \ \ \ {\isacharparenleft}{\kern0pt}a\ {\isasymin}\ defer\ m\ A\ p\ {\isasymlongrightarrow}\ a\ {\isasymnotin}\ reject\ m\ A\ q{\isacharparenright}{\kern0pt}{\isachardoublequoteclose}\isanewline
\isanewline
\isacommand{definition}\isamarkupfalse%
\ mod{\isacharunderscore}{\kern0pt}contains{\isacharunderscore}{\kern0pt}result\ {\isacharcolon}{\kern0pt}{\isacharcolon}{\kern0pt}\ {\isachardoublequoteopen}{\isacharprime}{\kern0pt}a\ Electoral{\isacharunderscore}{\kern0pt}Module\ {\isasymRightarrow}\ {\isacharprime}{\kern0pt}a\ Electoral{\isacharunderscore}{\kern0pt}Module\ {\isasymRightarrow}\isanewline
\ \ \ \ \ \ \ \ \ \ \ \ \ \ \ \ \ \ \ \ \ \ \ \ \ \ \ \ \ \ \ \ \ \ \ \ \ \ {\isacharprime}{\kern0pt}a\ set\ {\isasymRightarrow}\ {\isacharprime}{\kern0pt}a\ Profile\ {\isasymRightarrow}\ {\isacharprime}{\kern0pt}a\ {\isasymRightarrow}\ bool{\isachardoublequoteclose}\ \isakeyword{where}\isanewline
\ \ {\isachardoublequoteopen}mod{\isacharunderscore}{\kern0pt}contains{\isacharunderscore}{\kern0pt}result\ m\ n\ A\ p\ a\ {\isasymequiv}\isanewline
\ \ \ \ electoral{\isacharunderscore}{\kern0pt}module\ m\ {\isasymand}\ electoral{\isacharunderscore}{\kern0pt}module\ n\ {\isasymand}\ finite{\isacharunderscore}{\kern0pt}profile\ A\ p\ {\isasymand}\ a\ {\isasymin}\ A\ {\isasymand}\isanewline
\ \ \ \ {\isacharparenleft}{\kern0pt}a\ {\isasymin}\ elect\ m\ A\ p\ {\isasymlongrightarrow}\ a\ {\isasymin}\ elect\ n\ A\ p{\isacharparenright}{\kern0pt}\ {\isasymand}\isanewline
\ \ \ \ {\isacharparenleft}{\kern0pt}a\ {\isasymin}\ reject\ m\ A\ p\ {\isasymlongrightarrow}\ a\ {\isasymin}\ reject\ n\ A\ p{\isacharparenright}{\kern0pt}\ {\isasymand}\isanewline
\ \ \ \ {\isacharparenleft}{\kern0pt}a\ {\isasymin}\ defer\ m\ A\ p\ {\isasymlongrightarrow}\ a\ {\isasymin}\ defer\ n\ A\ p{\isacharparenright}{\kern0pt}{\isachardoublequoteclose}%
\isadelimdocument
%
\endisadelimdocument
%
\isatagdocument
%
\isamarkupsubsection{Auxiliary Lemmata%
}
\isamarkuptrue%
%
\endisatagdocument
{\isafolddocument}%
%
\isadelimdocument
%
\endisadelimdocument
\isacommand{lemma}\isamarkupfalse%
\ combine{\isacharunderscore}{\kern0pt}ele{\isacharunderscore}{\kern0pt}rej{\isacharunderscore}{\kern0pt}def{\isacharcolon}{\kern0pt}\isanewline
\ \ \isakeyword{assumes}\isanewline
\ \ \ \ ele{\isacharcolon}{\kern0pt}\ {\isachardoublequoteopen}elect\ m\ A\ p\ {\isacharequal}{\kern0pt}\ e{\isachardoublequoteclose}\ \isakeyword{and}\isanewline
\ \ \ \ rej{\isacharcolon}{\kern0pt}\ {\isachardoublequoteopen}reject\ m\ A\ p\ {\isacharequal}{\kern0pt}\ r{\isachardoublequoteclose}\ \isakeyword{and}\isanewline
\ \ \ \ def{\isacharcolon}{\kern0pt}\ {\isachardoublequoteopen}defer\ m\ A\ p\ {\isacharequal}{\kern0pt}\ d{\isachardoublequoteclose}\isanewline
\ \ \isakeyword{shows}\ {\isachardoublequoteopen}m\ A\ p\ {\isacharequal}{\kern0pt}\ {\isacharparenleft}{\kern0pt}e{\isacharcomma}{\kern0pt}\ r{\isacharcomma}{\kern0pt}\ d{\isacharparenright}{\kern0pt}{\isachardoublequoteclose}\isanewline
%
\isadelimproof
\ \ %
\endisadelimproof
%
\isatagproof
\isacommand{using}\isamarkupfalse%
\ def\ ele\ rej\isanewline
\ \ \isacommand{by}\isamarkupfalse%
\ auto%
\endisatagproof
{\isafoldproof}%
%
\isadelimproof
\isanewline
%
\endisadelimproof
\isanewline
\isacommand{lemma}\isamarkupfalse%
\ par{\isacharunderscore}{\kern0pt}comp{\isacharunderscore}{\kern0pt}result{\isacharunderscore}{\kern0pt}sound{\isacharcolon}{\kern0pt}\isanewline
\ \ \isakeyword{assumes}\isanewline
\ \ \ \ mod{\isacharunderscore}{\kern0pt}m{\isacharcolon}{\kern0pt}\ {\isachardoublequoteopen}electoral{\isacharunderscore}{\kern0pt}module\ m{\isachardoublequoteclose}\ \isakeyword{and}\isanewline
\ \ \ \ f{\isacharunderscore}{\kern0pt}prof{\isacharcolon}{\kern0pt}\ {\isachardoublequoteopen}finite{\isacharunderscore}{\kern0pt}profile\ A\ p{\isachardoublequoteclose}\isanewline
\ \ \isakeyword{shows}\ {\isachardoublequoteopen}well{\isacharunderscore}{\kern0pt}formed\ A\ {\isacharparenleft}{\kern0pt}m\ A\ p{\isacharparenright}{\kern0pt}{\isachardoublequoteclose}\isanewline
%
\isadelimproof
\ \ %
\endisadelimproof
%
\isatagproof
\isacommand{using}\isamarkupfalse%
\ electoral{\isacharunderscore}{\kern0pt}module{\isacharunderscore}{\kern0pt}def\ mod{\isacharunderscore}{\kern0pt}m\ f{\isacharunderscore}{\kern0pt}prof\isanewline
\ \ \isacommand{by}\isamarkupfalse%
\ auto%
\endisatagproof
{\isafoldproof}%
%
\isadelimproof
\isanewline
%
\endisadelimproof
\isanewline
\isacommand{lemma}\isamarkupfalse%
\ result{\isacharunderscore}{\kern0pt}presv{\isacharunderscore}{\kern0pt}alts{\isacharcolon}{\kern0pt}\isanewline
\ \ \isakeyword{assumes}\isanewline
\ \ \ \ e{\isacharunderscore}{\kern0pt}mod{\isacharcolon}{\kern0pt}\ {\isachardoublequoteopen}electoral{\isacharunderscore}{\kern0pt}module\ m{\isachardoublequoteclose}\ \isakeyword{and}\isanewline
\ \ \ \ f{\isacharunderscore}{\kern0pt}prof{\isacharcolon}{\kern0pt}\ {\isachardoublequoteopen}finite{\isacharunderscore}{\kern0pt}profile\ A\ p{\isachardoublequoteclose}\isanewline
\ \ \isakeyword{shows}\ {\isachardoublequoteopen}{\isacharparenleft}{\kern0pt}elect\ m\ A\ p{\isacharparenright}{\kern0pt}\ {\isasymunion}\ {\isacharparenleft}{\kern0pt}reject\ m\ A\ p{\isacharparenright}{\kern0pt}\ {\isasymunion}\ {\isacharparenleft}{\kern0pt}defer\ m\ A\ p{\isacharparenright}{\kern0pt}\ {\isacharequal}{\kern0pt}\ A{\isachardoublequoteclose}\isanewline
%
\isadelimproof
%
\endisadelimproof
%
\isatagproof
\isacommand{proof}\isamarkupfalse%
\ {\isacharparenleft}{\kern0pt}safe{\isacharparenright}{\kern0pt}\isanewline
\ \ \isacommand{fix}\isamarkupfalse%
\isanewline
\ \ \ \ x\ {\isacharcolon}{\kern0pt}{\isacharcolon}{\kern0pt}\ {\isachardoublequoteopen}{\isacharprime}{\kern0pt}a{\isachardoublequoteclose}\isanewline
\ \ \isacommand{assume}\isamarkupfalse%
\isanewline
\ \ \ \ asm{\isacharcolon}{\kern0pt}\ {\isachardoublequoteopen}x\ {\isasymin}\ elect\ m\ A\ p{\isachardoublequoteclose}\isanewline
\ \ \isacommand{have}\isamarkupfalse%
\ partit{\isacharcolon}{\kern0pt}\isanewline
\ \ \ \ {\isachardoublequoteopen}{\isasymforall}A\ p{\isachardot}{\kern0pt}\isanewline
\ \ \ \ \ \ {\isasymnot}\ set{\isacharunderscore}{\kern0pt}equals{\isacharunderscore}{\kern0pt}partition\ {\isacharparenleft}{\kern0pt}A{\isacharcolon}{\kern0pt}{\isacharcolon}{\kern0pt}{\isacharprime}{\kern0pt}a\ set{\isacharparenright}{\kern0pt}\ p\ {\isasymor}\isanewline
\ \ \ \ \ \ \ \ {\isacharparenleft}{\kern0pt}{\isasymexists}B\ C\ D\ E{\isachardot}{\kern0pt}\ A\ {\isacharequal}{\kern0pt}\ B\ {\isasymand}\ p\ {\isacharequal}{\kern0pt}\ {\isacharparenleft}{\kern0pt}C{\isacharcomma}{\kern0pt}\ D{\isacharcomma}{\kern0pt}\ E{\isacharparenright}{\kern0pt}\ {\isasymand}\ C\ {\isasymunion}\ D\ {\isasymunion}\ E\ {\isacharequal}{\kern0pt}\ B{\isacharparenright}{\kern0pt}{\isachardoublequoteclose}\isanewline
\ \ \ \ \isacommand{by}\isamarkupfalse%
\ simp\isanewline
\ \ \isacommand{from}\isamarkupfalse%
\ e{\isacharunderscore}{\kern0pt}mod\ f{\isacharunderscore}{\kern0pt}prof\ \isacommand{have}\isamarkupfalse%
\ set{\isacharunderscore}{\kern0pt}partit{\isacharcolon}{\kern0pt}\isanewline
\ \ \ \ {\isachardoublequoteopen}set{\isacharunderscore}{\kern0pt}equals{\isacharunderscore}{\kern0pt}partition\ A\ {\isacharparenleft}{\kern0pt}m\ A\ p{\isacharparenright}{\kern0pt}{\isachardoublequoteclose}\isanewline
\ \ \ \ \isacommand{using}\isamarkupfalse%
\ electoral{\isacharunderscore}{\kern0pt}module{\isacharunderscore}{\kern0pt}def\isanewline
\ \ \ \ \isacommand{by}\isamarkupfalse%
\ auto\isanewline
\ \ \isacommand{thus}\isamarkupfalse%
\ {\isachardoublequoteopen}x\ {\isasymin}\ A{\isachardoublequoteclose}\isanewline
\ \ \ \ \isacommand{using}\isamarkupfalse%
\ UnI{\isadigit{1}}\ asm\ fstI\ set{\isacharunderscore}{\kern0pt}partit\ partit\isanewline
\ \ \ \ \isacommand{by}\isamarkupfalse%
\ {\isacharparenleft}{\kern0pt}metis\ {\isacharparenleft}{\kern0pt}no{\isacharunderscore}{\kern0pt}types{\isacharparenright}{\kern0pt}{\isacharparenright}{\kern0pt}\isanewline
\isacommand{next}\isamarkupfalse%
\isanewline
\ \ \isacommand{fix}\isamarkupfalse%
\isanewline
\ \ \ \ x\ {\isacharcolon}{\kern0pt}{\isacharcolon}{\kern0pt}\ {\isachardoublequoteopen}{\isacharprime}{\kern0pt}a{\isachardoublequoteclose}\isanewline
\ \ \isacommand{assume}\isamarkupfalse%
\isanewline
\ \ \ \ asm{\isacharcolon}{\kern0pt}\ {\isachardoublequoteopen}x\ {\isasymin}\ reject\ m\ A\ p{\isachardoublequoteclose}\isanewline
\ \ \isacommand{have}\isamarkupfalse%
\ partit{\isacharcolon}{\kern0pt}\isanewline
\ \ \ \ {\isachardoublequoteopen}{\isasymforall}A\ p{\isachardot}{\kern0pt}\isanewline
\ \ \ \ \ \ {\isasymnot}\ set{\isacharunderscore}{\kern0pt}equals{\isacharunderscore}{\kern0pt}partition\ {\isacharparenleft}{\kern0pt}A{\isacharcolon}{\kern0pt}{\isacharcolon}{\kern0pt}{\isacharprime}{\kern0pt}a\ set{\isacharparenright}{\kern0pt}\ p\ {\isasymor}\isanewline
\ \ \ \ \ \ \ \ {\isacharparenleft}{\kern0pt}{\isasymexists}B\ C\ D\ E{\isachardot}{\kern0pt}\ A\ {\isacharequal}{\kern0pt}\ B\ {\isasymand}\ p\ {\isacharequal}{\kern0pt}\ {\isacharparenleft}{\kern0pt}C{\isacharcomma}{\kern0pt}\ D{\isacharcomma}{\kern0pt}\ E{\isacharparenright}{\kern0pt}\ {\isasymand}\ C\ {\isasymunion}\ D\ {\isasymunion}\ E\ {\isacharequal}{\kern0pt}\ B{\isacharparenright}{\kern0pt}{\isachardoublequoteclose}\isanewline
\ \ \ \ \isacommand{by}\isamarkupfalse%
\ simp\isanewline
\ \ \isacommand{from}\isamarkupfalse%
\ e{\isacharunderscore}{\kern0pt}mod\ f{\isacharunderscore}{\kern0pt}prof\ \isacommand{have}\isamarkupfalse%
\ set{\isacharunderscore}{\kern0pt}partit{\isacharcolon}{\kern0pt}\isanewline
\ \ \ \ {\isachardoublequoteopen}set{\isacharunderscore}{\kern0pt}equals{\isacharunderscore}{\kern0pt}partition\ A\ {\isacharparenleft}{\kern0pt}m\ A\ p{\isacharparenright}{\kern0pt}{\isachardoublequoteclose}\isanewline
\ \ \ \ \isacommand{using}\isamarkupfalse%
\ electoral{\isacharunderscore}{\kern0pt}module{\isacharunderscore}{\kern0pt}def\isanewline
\ \ \ \ \isacommand{by}\isamarkupfalse%
\ auto\isanewline
\ \ \isacommand{thus}\isamarkupfalse%
\ {\isachardoublequoteopen}x\ {\isasymin}\ A{\isachardoublequoteclose}\isanewline
\ \ \ \ \isacommand{using}\isamarkupfalse%
\ UnI{\isadigit{1}}\ asm\ fstI\ set{\isacharunderscore}{\kern0pt}partit\ partit\isanewline
\ \ \ \ \ \ \ \ \ \ sndI\ subsetD\ sup{\isacharunderscore}{\kern0pt}ge{\isadigit{2}}\isanewline
\ \ \ \ \isacommand{by}\isamarkupfalse%
\ metis\isanewline
\isacommand{next}\isamarkupfalse%
\isanewline
\ \ \isacommand{fix}\isamarkupfalse%
\isanewline
\ \ \ \ x\ {\isacharcolon}{\kern0pt}{\isacharcolon}{\kern0pt}\ {\isachardoublequoteopen}{\isacharprime}{\kern0pt}a{\isachardoublequoteclose}\isanewline
\ \ \isacommand{assume}\isamarkupfalse%
\isanewline
\ \ \ \ asm{\isacharcolon}{\kern0pt}\ {\isachardoublequoteopen}x\ {\isasymin}\ defer\ m\ A\ p{\isachardoublequoteclose}\isanewline
\ \ \isacommand{have}\isamarkupfalse%
\ partit{\isacharcolon}{\kern0pt}\isanewline
\ \ \ \ {\isachardoublequoteopen}{\isasymforall}A\ p{\isachardot}{\kern0pt}\isanewline
\ \ \ \ \ \ {\isasymnot}\ set{\isacharunderscore}{\kern0pt}equals{\isacharunderscore}{\kern0pt}partition\ {\isacharparenleft}{\kern0pt}A{\isacharcolon}{\kern0pt}{\isacharcolon}{\kern0pt}{\isacharprime}{\kern0pt}a\ set{\isacharparenright}{\kern0pt}\ p\ {\isasymor}\isanewline
\ \ \ \ \ \ \ \ {\isacharparenleft}{\kern0pt}{\isasymexists}B\ C\ D\ E{\isachardot}{\kern0pt}\ A\ {\isacharequal}{\kern0pt}\ B\ {\isasymand}\ p\ {\isacharequal}{\kern0pt}\ {\isacharparenleft}{\kern0pt}C{\isacharcomma}{\kern0pt}\ D{\isacharcomma}{\kern0pt}\ E{\isacharparenright}{\kern0pt}\ {\isasymand}\ C\ {\isasymunion}\ D\ {\isasymunion}\ E\ {\isacharequal}{\kern0pt}\ B{\isacharparenright}{\kern0pt}{\isachardoublequoteclose}\isanewline
\ \ \ \ \isacommand{by}\isamarkupfalse%
\ simp\isanewline
\ \ \isacommand{from}\isamarkupfalse%
\ e{\isacharunderscore}{\kern0pt}mod\ f{\isacharunderscore}{\kern0pt}prof\ \isacommand{have}\isamarkupfalse%
\ set{\isacharunderscore}{\kern0pt}partit{\isacharcolon}{\kern0pt}\isanewline
\ \ \ \ {\isachardoublequoteopen}set{\isacharunderscore}{\kern0pt}equals{\isacharunderscore}{\kern0pt}partition\ A\ {\isacharparenleft}{\kern0pt}m\ A\ p{\isacharparenright}{\kern0pt}{\isachardoublequoteclose}\isanewline
\ \ \ \ \isacommand{using}\isamarkupfalse%
\ electoral{\isacharunderscore}{\kern0pt}module{\isacharunderscore}{\kern0pt}def\isanewline
\ \ \ \ \isacommand{by}\isamarkupfalse%
\ auto\isanewline
\ \ \isacommand{thus}\isamarkupfalse%
\ {\isachardoublequoteopen}x\ {\isasymin}\ A{\isachardoublequoteclose}\isanewline
\ \ \ \ \isacommand{using}\isamarkupfalse%
\ asm\ set{\isacharunderscore}{\kern0pt}partit\ partit\ sndI\ subsetD\ sup{\isacharunderscore}{\kern0pt}ge{\isadigit{2}}\isanewline
\ \ \ \ \isacommand{by}\isamarkupfalse%
\ metis\isanewline
\isacommand{next}\isamarkupfalse%
\isanewline
\ \ \isacommand{fix}\isamarkupfalse%
\isanewline
\ \ \ \ x\ {\isacharcolon}{\kern0pt}{\isacharcolon}{\kern0pt}\ {\isachardoublequoteopen}{\isacharprime}{\kern0pt}a{\isachardoublequoteclose}\isanewline
\ \ \isacommand{assume}\isamarkupfalse%
\isanewline
\ \ \ \ asm{\isadigit{1}}{\isacharcolon}{\kern0pt}\ {\isachardoublequoteopen}x\ {\isasymin}\ A{\isachardoublequoteclose}\ \isakeyword{and}\isanewline
\ \ \ \ asm{\isadigit{2}}{\isacharcolon}{\kern0pt}\ {\isachardoublequoteopen}x\ {\isasymnotin}\ defer\ m\ A\ p{\isachardoublequoteclose}\ \isakeyword{and}\isanewline
\ \ \ \ asm{\isadigit{3}}{\isacharcolon}{\kern0pt}\ {\isachardoublequoteopen}x\ {\isasymnotin}\ reject\ m\ A\ p{\isachardoublequoteclose}\isanewline
\ \ \isacommand{have}\isamarkupfalse%
\ partit{\isacharcolon}{\kern0pt}\isanewline
\ \ \ \ {\isachardoublequoteopen}{\isasymforall}A\ p{\isachardot}{\kern0pt}\isanewline
\ \ \ \ \ \ {\isasymnot}\ set{\isacharunderscore}{\kern0pt}equals{\isacharunderscore}{\kern0pt}partition\ {\isacharparenleft}{\kern0pt}A{\isacharcolon}{\kern0pt}{\isacharcolon}{\kern0pt}{\isacharprime}{\kern0pt}a\ set{\isacharparenright}{\kern0pt}\ p\ {\isasymor}\isanewline
\ \ \ \ \ \ \ \ {\isacharparenleft}{\kern0pt}{\isasymexists}B\ C\ D\ E{\isachardot}{\kern0pt}\ A\ {\isacharequal}{\kern0pt}\ B\ {\isasymand}\ p\ {\isacharequal}{\kern0pt}\ {\isacharparenleft}{\kern0pt}C{\isacharcomma}{\kern0pt}\ D{\isacharcomma}{\kern0pt}\ E{\isacharparenright}{\kern0pt}\ {\isasymand}\ C\ {\isasymunion}\ D\ {\isasymunion}\ E\ {\isacharequal}{\kern0pt}\ B{\isacharparenright}{\kern0pt}{\isachardoublequoteclose}\isanewline
\ \ \ \ \isacommand{by}\isamarkupfalse%
\ simp\isanewline
\ \ \isacommand{from}\isamarkupfalse%
\ e{\isacharunderscore}{\kern0pt}mod\ f{\isacharunderscore}{\kern0pt}prof\ \isacommand{have}\isamarkupfalse%
\ set{\isacharunderscore}{\kern0pt}partit{\isacharcolon}{\kern0pt}\isanewline
\ \ \ \ {\isachardoublequoteopen}set{\isacharunderscore}{\kern0pt}equals{\isacharunderscore}{\kern0pt}partition\ A\ {\isacharparenleft}{\kern0pt}m\ A\ p{\isacharparenright}{\kern0pt}{\isachardoublequoteclose}\isanewline
\ \ \ \ \isacommand{using}\isamarkupfalse%
\ electoral{\isacharunderscore}{\kern0pt}module{\isacharunderscore}{\kern0pt}def\isanewline
\ \ \ \ \isacommand{by}\isamarkupfalse%
\ auto\isanewline
\ \ \isacommand{show}\isamarkupfalse%
\ {\isachardoublequoteopen}x\ {\isasymin}\ elect\ m\ A\ p{\isachardoublequoteclose}\isanewline
\ \ \ \ \isacommand{using}\isamarkupfalse%
\ asm{\isadigit{1}}\ asm{\isadigit{2}}\ asm{\isadigit{3}}\ fst{\isacharunderscore}{\kern0pt}conv\ partit\isanewline
\ \ \ \ \ \ \ \ \ \ set{\isacharunderscore}{\kern0pt}partit\ snd{\isacharunderscore}{\kern0pt}conv\ Un{\isacharunderscore}{\kern0pt}iff\isanewline
\ \ \ \ \isacommand{by}\isamarkupfalse%
\ metis\isanewline
\isacommand{qed}\isamarkupfalse%
%
\endisatagproof
{\isafoldproof}%
%
\isadelimproof
\isanewline
%
\endisadelimproof
\isanewline
\isacommand{lemma}\isamarkupfalse%
\ result{\isacharunderscore}{\kern0pt}disj{\isacharcolon}{\kern0pt}\isanewline
\ \ \isakeyword{assumes}\isanewline
\ \ \ \ module{\isacharcolon}{\kern0pt}\ {\isachardoublequoteopen}electoral{\isacharunderscore}{\kern0pt}module\ m{\isachardoublequoteclose}\ \isakeyword{and}\isanewline
\ \ \ \ profile{\isacharcolon}{\kern0pt}\ {\isachardoublequoteopen}finite{\isacharunderscore}{\kern0pt}profile\ A\ p{\isachardoublequoteclose}\isanewline
\ \ \isakeyword{shows}\isanewline
\ \ \ \ {\isachardoublequoteopen}{\isacharparenleft}{\kern0pt}elect\ m\ A\ p{\isacharparenright}{\kern0pt}\ {\isasyminter}\ {\isacharparenleft}{\kern0pt}reject\ m\ A\ p{\isacharparenright}{\kern0pt}\ {\isacharequal}{\kern0pt}\ {\isacharbraceleft}{\kern0pt}{\isacharbraceright}{\kern0pt}\ {\isasymand}\isanewline
\ \ \ \ \ \ \ \ {\isacharparenleft}{\kern0pt}elect\ m\ A\ p{\isacharparenright}{\kern0pt}\ {\isasyminter}\ {\isacharparenleft}{\kern0pt}defer\ m\ A\ p{\isacharparenright}{\kern0pt}\ {\isacharequal}{\kern0pt}\ {\isacharbraceleft}{\kern0pt}{\isacharbraceright}{\kern0pt}\ {\isasymand}\isanewline
\ \ \ \ \ \ \ \ {\isacharparenleft}{\kern0pt}reject\ m\ A\ p{\isacharparenright}{\kern0pt}\ {\isasyminter}\ {\isacharparenleft}{\kern0pt}defer\ m\ A\ p{\isacharparenright}{\kern0pt}\ {\isacharequal}{\kern0pt}\ {\isacharbraceleft}{\kern0pt}{\isacharbraceright}{\kern0pt}{\isachardoublequoteclose}\isanewline
%
\isadelimproof
%
\endisadelimproof
%
\isatagproof
\isacommand{proof}\isamarkupfalse%
\ {\isacharparenleft}{\kern0pt}safe{\isacharcomma}{\kern0pt}\ simp{\isacharunderscore}{\kern0pt}all{\isacharparenright}{\kern0pt}\isanewline
\ \ \isacommand{fix}\isamarkupfalse%
\isanewline
\ \ \ \ x\ {\isacharcolon}{\kern0pt}{\isacharcolon}{\kern0pt}\ {\isachardoublequoteopen}{\isacharprime}{\kern0pt}a{\isachardoublequoteclose}\isanewline
\ \ \isacommand{assume}\isamarkupfalse%
\isanewline
\ \ \ \ asm{\isadigit{1}}{\isacharcolon}{\kern0pt}\ {\isachardoublequoteopen}x\ {\isasymin}\ elect\ m\ A\ p{\isachardoublequoteclose}\ \isakeyword{and}\isanewline
\ \ \ \ asm{\isadigit{2}}{\isacharcolon}{\kern0pt}\ {\isachardoublequoteopen}x\ {\isasymin}\ reject\ m\ A\ p{\isachardoublequoteclose}\isanewline
\ \ \isacommand{have}\isamarkupfalse%
\ partit{\isacharcolon}{\kern0pt}\isanewline
\ \ \ \ {\isachardoublequoteopen}{\isasymforall}A\ p{\isachardot}{\kern0pt}\isanewline
\ \ \ \ \ \ {\isasymnot}\ set{\isacharunderscore}{\kern0pt}equals{\isacharunderscore}{\kern0pt}partition\ {\isacharparenleft}{\kern0pt}A{\isacharcolon}{\kern0pt}{\isacharcolon}{\kern0pt}{\isacharprime}{\kern0pt}a\ set{\isacharparenright}{\kern0pt}\ p\ {\isasymor}\isanewline
\ \ \ \ \ \ \ \ {\isacharparenleft}{\kern0pt}{\isasymexists}B\ C\ D\ E{\isachardot}{\kern0pt}\ A\ {\isacharequal}{\kern0pt}\ B\ {\isasymand}\ p\ {\isacharequal}{\kern0pt}\ {\isacharparenleft}{\kern0pt}C{\isacharcomma}{\kern0pt}\ D{\isacharcomma}{\kern0pt}\ E{\isacharparenright}{\kern0pt}\ {\isasymand}\ C\ {\isasymunion}\ D\ {\isasymunion}\ E\ {\isacharequal}{\kern0pt}\ B{\isacharparenright}{\kern0pt}{\isachardoublequoteclose}\isanewline
\ \ \ \ \isacommand{by}\isamarkupfalse%
\ simp\isanewline
\ \ \isacommand{from}\isamarkupfalse%
\ module\ profile\ \isacommand{have}\isamarkupfalse%
\ set{\isacharunderscore}{\kern0pt}partit{\isacharcolon}{\kern0pt}\isanewline
\ \ \ \ {\isachardoublequoteopen}set{\isacharunderscore}{\kern0pt}equals{\isacharunderscore}{\kern0pt}partition\ A\ {\isacharparenleft}{\kern0pt}m\ A\ p{\isacharparenright}{\kern0pt}{\isachardoublequoteclose}\isanewline
\ \ \ \ \isacommand{using}\isamarkupfalse%
\ electoral{\isacharunderscore}{\kern0pt}module{\isacharunderscore}{\kern0pt}def\isanewline
\ \ \ \ \isacommand{by}\isamarkupfalse%
\ auto\isanewline
\ \ \isacommand{from}\isamarkupfalse%
\ profile\ \isacommand{have}\isamarkupfalse%
\ prof{\isacharunderscore}{\kern0pt}p{\isacharcolon}{\kern0pt}\isanewline
\ \ \ \ {\isachardoublequoteopen}finite\ A\ {\isasymand}\ profile\ A\ p{\isachardoublequoteclose}\isanewline
\ \ \ \ \isacommand{by}\isamarkupfalse%
\ simp\isanewline
\ \ \isacommand{from}\isamarkupfalse%
\ module\ prof{\isacharunderscore}{\kern0pt}p\ \isacommand{have}\isamarkupfalse%
\ wf{\isacharunderscore}{\kern0pt}A{\isacharunderscore}{\kern0pt}m{\isacharcolon}{\kern0pt}\isanewline
\ \ \ \ {\isachardoublequoteopen}well{\isacharunderscore}{\kern0pt}formed\ A\ {\isacharparenleft}{\kern0pt}m\ A\ p{\isacharparenright}{\kern0pt}{\isachardoublequoteclose}\isanewline
\ \ \ \ \isacommand{using}\isamarkupfalse%
\ electoral{\isacharunderscore}{\kern0pt}module{\isacharunderscore}{\kern0pt}def\isanewline
\ \ \ \ \isacommand{by}\isamarkupfalse%
\ metis\isanewline
\ \ \isacommand{show}\isamarkupfalse%
\ {\isachardoublequoteopen}False{\isachardoublequoteclose}\isanewline
\ \ \ \ \isacommand{using}\isamarkupfalse%
\ prod{\isachardot}{\kern0pt}exhaust{\isacharunderscore}{\kern0pt}sel\ DiffE\ UnCI\ asm{\isadigit{1}}\ asm{\isadigit{2}}\isanewline
\ \ \ \ \ \ \ \ \ \ module\ profile\ result{\isacharunderscore}{\kern0pt}imp{\isacharunderscore}{\kern0pt}rej\ wf{\isacharunderscore}{\kern0pt}A{\isacharunderscore}{\kern0pt}m\isanewline
\ \ \ \ \ \ \ \ \ \ prof{\isacharunderscore}{\kern0pt}p\ set{\isacharunderscore}{\kern0pt}partit\ partit\isanewline
\ \ \ \ \isacommand{by}\isamarkupfalse%
\ {\isacharparenleft}{\kern0pt}metis\ {\isacharparenleft}{\kern0pt}no{\isacharunderscore}{\kern0pt}types{\isacharparenright}{\kern0pt}{\isacharparenright}{\kern0pt}\isanewline
\isacommand{next}\isamarkupfalse%
\isanewline
\ \ \isacommand{fix}\isamarkupfalse%
\isanewline
\ \ \ \ x\ {\isacharcolon}{\kern0pt}{\isacharcolon}{\kern0pt}\ {\isachardoublequoteopen}{\isacharprime}{\kern0pt}a{\isachardoublequoteclose}\isanewline
\ \ \isacommand{assume}\isamarkupfalse%
\isanewline
\ \ \ \ asm{\isadigit{1}}{\isacharcolon}{\kern0pt}\ {\isachardoublequoteopen}x\ {\isasymin}\ elect\ m\ A\ p{\isachardoublequoteclose}\ \isakeyword{and}\isanewline
\ \ \ \ asm{\isadigit{2}}{\isacharcolon}{\kern0pt}\ {\isachardoublequoteopen}x\ {\isasymin}\ defer\ m\ A\ p{\isachardoublequoteclose}\isanewline
\ \ \isacommand{have}\isamarkupfalse%
\ partit{\isacharcolon}{\kern0pt}\isanewline
\ \ \ \ {\isachardoublequoteopen}{\isasymforall}A\ p{\isachardot}{\kern0pt}\isanewline
\ \ \ \ \ \ {\isasymnot}\ set{\isacharunderscore}{\kern0pt}equals{\isacharunderscore}{\kern0pt}partition\ {\isacharparenleft}{\kern0pt}A{\isacharcolon}{\kern0pt}{\isacharcolon}{\kern0pt}{\isacharprime}{\kern0pt}a\ set{\isacharparenright}{\kern0pt}\ p\ {\isasymor}\isanewline
\ \ \ \ \ \ \ \ {\isacharparenleft}{\kern0pt}{\isasymexists}B\ C\ D\ E{\isachardot}{\kern0pt}\ A\ {\isacharequal}{\kern0pt}\ B\ {\isasymand}\ p\ {\isacharequal}{\kern0pt}\ {\isacharparenleft}{\kern0pt}C{\isacharcomma}{\kern0pt}\ D{\isacharcomma}{\kern0pt}\ E{\isacharparenright}{\kern0pt}\ {\isasymand}\ C\ {\isasymunion}\ D\ {\isasymunion}\ E\ {\isacharequal}{\kern0pt}\ B{\isacharparenright}{\kern0pt}{\isachardoublequoteclose}\isanewline
\ \ \ \ \isacommand{by}\isamarkupfalse%
\ simp\isanewline
\ \ \isacommand{have}\isamarkupfalse%
\ disj{\isacharcolon}{\kern0pt}\isanewline
\ \ \ \ {\isachardoublequoteopen}{\isasymforall}p{\isachardot}{\kern0pt}\ {\isasymnot}\ disjoint{\isadigit{3}}\ p\ {\isasymor}\isanewline
\ \ \ \ \ \ {\isacharparenleft}{\kern0pt}{\isasymexists}B\ C\ D{\isachardot}{\kern0pt}\ p\ {\isacharequal}{\kern0pt}\ {\isacharparenleft}{\kern0pt}B{\isacharcolon}{\kern0pt}{\isacharcolon}{\kern0pt}{\isacharprime}{\kern0pt}a\ set{\isacharcomma}{\kern0pt}\ C{\isacharcomma}{\kern0pt}\ D{\isacharparenright}{\kern0pt}\ {\isasymand}\isanewline
\ \ \ \ \ \ \ \ B\ {\isasyminter}\ C\ {\isacharequal}{\kern0pt}\ {\isacharbraceleft}{\kern0pt}{\isacharbraceright}{\kern0pt}\ {\isasymand}\ B\ {\isasyminter}\ D\ {\isacharequal}{\kern0pt}\ {\isacharbraceleft}{\kern0pt}{\isacharbraceright}{\kern0pt}\ {\isasymand}\ C\ {\isasyminter}\ D\ {\isacharequal}{\kern0pt}\ {\isacharbraceleft}{\kern0pt}{\isacharbraceright}{\kern0pt}{\isacharparenright}{\kern0pt}{\isachardoublequoteclose}\isanewline
\ \ \ \ \isacommand{by}\isamarkupfalse%
\ simp\isanewline
\ \ \isacommand{from}\isamarkupfalse%
\ profile\ \isacommand{have}\isamarkupfalse%
\ prof{\isacharunderscore}{\kern0pt}p{\isacharcolon}{\kern0pt}\isanewline
\ \ \ \ {\isachardoublequoteopen}finite\ A\ {\isasymand}\ profile\ A\ p{\isachardoublequoteclose}\isanewline
\ \ \ \ \isacommand{by}\isamarkupfalse%
\ simp\isanewline
\ \ \isacommand{from}\isamarkupfalse%
\ module\ prof{\isacharunderscore}{\kern0pt}p\ \isacommand{have}\isamarkupfalse%
\ wf{\isacharunderscore}{\kern0pt}A{\isacharunderscore}{\kern0pt}m{\isacharcolon}{\kern0pt}\isanewline
\ \ \ \ {\isachardoublequoteopen}well{\isacharunderscore}{\kern0pt}formed\ A\ {\isacharparenleft}{\kern0pt}m\ A\ p{\isacharparenright}{\kern0pt}{\isachardoublequoteclose}\isanewline
\ \ \ \ \isacommand{using}\isamarkupfalse%
\ electoral{\isacharunderscore}{\kern0pt}module{\isacharunderscore}{\kern0pt}def\isanewline
\ \ \ \ \isacommand{by}\isamarkupfalse%
\ metis\isanewline
\ \ \isacommand{hence}\isamarkupfalse%
\ wf{\isacharunderscore}{\kern0pt}A{\isacharunderscore}{\kern0pt}m{\isacharunderscore}{\kern0pt}{\isadigit{0}}{\isacharcolon}{\kern0pt}\isanewline
\ \ \ \ {\isachardoublequoteopen}disjoint{\isadigit{3}}\ {\isacharparenleft}{\kern0pt}m\ A\ p{\isacharparenright}{\kern0pt}\ {\isasymand}\ set{\isacharunderscore}{\kern0pt}equals{\isacharunderscore}{\kern0pt}partition\ A\ {\isacharparenleft}{\kern0pt}m\ A\ p{\isacharparenright}{\kern0pt}{\isachardoublequoteclose}\isanewline
\ \ \ \ \isacommand{by}\isamarkupfalse%
\ simp\isanewline
\ \ \isacommand{hence}\isamarkupfalse%
\ disj{\isadigit{3}}{\isacharcolon}{\kern0pt}\isanewline
\ \ \ \ {\isachardoublequoteopen}disjoint{\isadigit{3}}\ {\isacharparenleft}{\kern0pt}m\ A\ p{\isacharparenright}{\kern0pt}{\isachardoublequoteclose}\isanewline
\ \ \ \ \isacommand{by}\isamarkupfalse%
\ simp\isanewline
\ \ \isacommand{have}\isamarkupfalse%
\ set{\isacharunderscore}{\kern0pt}partit{\isacharcolon}{\kern0pt}\isanewline
\ \ \ \ {\isachardoublequoteopen}set{\isacharunderscore}{\kern0pt}equals{\isacharunderscore}{\kern0pt}partition\ A\ {\isacharparenleft}{\kern0pt}m\ A\ p{\isacharparenright}{\kern0pt}{\isachardoublequoteclose}\isanewline
\ \ \ \ \isacommand{using}\isamarkupfalse%
\ wf{\isacharunderscore}{\kern0pt}A{\isacharunderscore}{\kern0pt}m{\isacharunderscore}{\kern0pt}{\isadigit{0}}\isanewline
\ \ \ \ \isacommand{by}\isamarkupfalse%
\ simp\isanewline
\ \ \isacommand{from}\isamarkupfalse%
\ disj{\isadigit{3}}\ \isacommand{obtain}\isamarkupfalse%
\isanewline
\ \ \ \ AA\ {\isacharcolon}{\kern0pt}{\isacharcolon}{\kern0pt}\ {\isachardoublequoteopen}{\isacharprime}{\kern0pt}a\ Result\ {\isasymRightarrow}\ {\isacharprime}{\kern0pt}a\ set{\isachardoublequoteclose}\ \isakeyword{and}\isanewline
\ \ \ \ AAa\ {\isacharcolon}{\kern0pt}{\isacharcolon}{\kern0pt}\ {\isachardoublequoteopen}{\isacharprime}{\kern0pt}a\ Result\ {\isasymRightarrow}\ {\isacharprime}{\kern0pt}a\ set{\isachardoublequoteclose}\ \isakeyword{and}\isanewline
\ \ \ \ AAb\ {\isacharcolon}{\kern0pt}{\isacharcolon}{\kern0pt}\ {\isachardoublequoteopen}{\isacharprime}{\kern0pt}a\ Result\ {\isasymRightarrow}\ {\isacharprime}{\kern0pt}a\ set{\isachardoublequoteclose}\isanewline
\ \ \ \ \isakeyword{where}\isanewline
\ \ \ \ {\isachardoublequoteopen}m\ A\ p\ {\isacharequal}{\kern0pt}\isanewline
\ \ \ \ \ \ {\isacharparenleft}{\kern0pt}AA\ {\isacharparenleft}{\kern0pt}m\ A\ p{\isacharparenright}{\kern0pt}{\isacharcomma}{\kern0pt}\ AAa\ {\isacharparenleft}{\kern0pt}m\ A\ p{\isacharparenright}{\kern0pt}{\isacharcomma}{\kern0pt}\ AAb\ {\isacharparenleft}{\kern0pt}m\ A\ p{\isacharparenright}{\kern0pt}{\isacharparenright}{\kern0pt}\ {\isasymand}\isanewline
\ \ \ \ \ \ \ \ AA\ {\isacharparenleft}{\kern0pt}m\ A\ p{\isacharparenright}{\kern0pt}\ {\isasyminter}\ AAa\ {\isacharparenleft}{\kern0pt}m\ A\ p{\isacharparenright}{\kern0pt}\ {\isacharequal}{\kern0pt}\ {\isacharbraceleft}{\kern0pt}{\isacharbraceright}{\kern0pt}\ {\isasymand}\isanewline
\ \ \ \ \ \ \ \ AA\ {\isacharparenleft}{\kern0pt}m\ A\ p{\isacharparenright}{\kern0pt}\ {\isasyminter}\ AAb\ {\isacharparenleft}{\kern0pt}m\ A\ p{\isacharparenright}{\kern0pt}\ {\isacharequal}{\kern0pt}\ {\isacharbraceleft}{\kern0pt}{\isacharbraceright}{\kern0pt}\ {\isasymand}\isanewline
\ \ \ \ \ \ \ \ AAa\ {\isacharparenleft}{\kern0pt}m\ A\ p{\isacharparenright}{\kern0pt}\ {\isasyminter}\ AAb\ {\isacharparenleft}{\kern0pt}m\ A\ p{\isacharparenright}{\kern0pt}\ {\isacharequal}{\kern0pt}\ {\isacharbraceleft}{\kern0pt}{\isacharbraceright}{\kern0pt}{\isachardoublequoteclose}\isanewline
\ \ \ \ \isacommand{using}\isamarkupfalse%
\ asm{\isadigit{1}}\ asm{\isadigit{2}}\ disj\isanewline
\ \ \ \ \isacommand{by}\isamarkupfalse%
\ metis\isanewline
\ \ \isacommand{hence}\isamarkupfalse%
\ {\isachardoublequoteopen}{\isacharparenleft}{\kern0pt}{\isacharparenleft}{\kern0pt}elect\ m\ A\ p{\isacharparenright}{\kern0pt}\ {\isasyminter}\ {\isacharparenleft}{\kern0pt}reject\ m\ A\ p{\isacharparenright}{\kern0pt}\ {\isacharequal}{\kern0pt}\ {\isacharbraceleft}{\kern0pt}{\isacharbraceright}{\kern0pt}{\isacharparenright}{\kern0pt}\ {\isasymand}\isanewline
\ \ \ \ \ \ \ \ \ \ {\isacharparenleft}{\kern0pt}{\isacharparenleft}{\kern0pt}elect\ m\ A\ p{\isacharparenright}{\kern0pt}\ {\isasyminter}\ {\isacharparenleft}{\kern0pt}defer\ m\ A\ p{\isacharparenright}{\kern0pt}\ {\isacharequal}{\kern0pt}\ {\isacharbraceleft}{\kern0pt}{\isacharbraceright}{\kern0pt}{\isacharparenright}{\kern0pt}\ {\isasymand}\isanewline
\ \ \ \ \ \ \ \ \ \ {\isacharparenleft}{\kern0pt}{\isacharparenleft}{\kern0pt}reject\ m\ A\ p{\isacharparenright}{\kern0pt}\ {\isasyminter}\ {\isacharparenleft}{\kern0pt}defer\ m\ A\ p{\isacharparenright}{\kern0pt}\ {\isacharequal}{\kern0pt}\ {\isacharbraceleft}{\kern0pt}{\isacharbraceright}{\kern0pt}{\isacharparenright}{\kern0pt}{\isachardoublequoteclose}\isanewline
\ \ \ \ \isacommand{using}\isamarkupfalse%
\ disj{\isadigit{3}}\ eq{\isacharunderscore}{\kern0pt}snd{\isacharunderscore}{\kern0pt}iff\ fstI\isanewline
\ \ \ \ \isacommand{by}\isamarkupfalse%
\ metis\isanewline
\ \ \isacommand{thus}\isamarkupfalse%
\ {\isachardoublequoteopen}False{\isachardoublequoteclose}\isanewline
\ \ \ \ \isacommand{using}\isamarkupfalse%
\ asm{\isadigit{1}}\ asm{\isadigit{2}}\ module\ profile\ wf{\isacharunderscore}{\kern0pt}A{\isacharunderscore}{\kern0pt}m\ prof{\isacharunderscore}{\kern0pt}p\isanewline
\ \ \ \ \ \ \ \ \ \ set{\isacharunderscore}{\kern0pt}partit\ partit\ disjoint{\isacharunderscore}{\kern0pt}iff{\isacharunderscore}{\kern0pt}not{\isacharunderscore}{\kern0pt}equal\isanewline
\ \ \ \ \isacommand{by}\isamarkupfalse%
\ {\isacharparenleft}{\kern0pt}metis\ {\isacharparenleft}{\kern0pt}no{\isacharunderscore}{\kern0pt}types{\isacharparenright}{\kern0pt}{\isacharparenright}{\kern0pt}\isanewline
\isacommand{next}\isamarkupfalse%
\isanewline
\ \ \isacommand{fix}\isamarkupfalse%
\isanewline
\ \ \ \ x\ {\isacharcolon}{\kern0pt}{\isacharcolon}{\kern0pt}\ {\isachardoublequoteopen}{\isacharprime}{\kern0pt}a{\isachardoublequoteclose}\isanewline
\ \ \isacommand{assume}\isamarkupfalse%
\isanewline
\ \ \ \ asm{\isadigit{1}}{\isacharcolon}{\kern0pt}\ {\isachardoublequoteopen}x\ {\isasymin}\ reject\ m\ A\ p{\isachardoublequoteclose}\ \isakeyword{and}\isanewline
\ \ \ \ asm{\isadigit{2}}{\isacharcolon}{\kern0pt}\ {\isachardoublequoteopen}x\ {\isasymin}\ defer\ m\ A\ p{\isachardoublequoteclose}\isanewline
\ \ \isacommand{have}\isamarkupfalse%
\ partit{\isacharcolon}{\kern0pt}\isanewline
\ \ \ \ {\isachardoublequoteopen}{\isasymforall}A\ p{\isachardot}{\kern0pt}\isanewline
\ \ \ \ \ \ {\isasymnot}\ set{\isacharunderscore}{\kern0pt}equals{\isacharunderscore}{\kern0pt}partition\ {\isacharparenleft}{\kern0pt}A{\isacharcolon}{\kern0pt}{\isacharcolon}{\kern0pt}{\isacharprime}{\kern0pt}a\ set{\isacharparenright}{\kern0pt}\ p\ {\isasymor}\isanewline
\ \ \ \ \ \ \ \ {\isacharparenleft}{\kern0pt}{\isasymexists}B\ C\ D\ E{\isachardot}{\kern0pt}\ A\ {\isacharequal}{\kern0pt}\ B\ {\isasymand}\ p\ {\isacharequal}{\kern0pt}\ {\isacharparenleft}{\kern0pt}C{\isacharcomma}{\kern0pt}\ D{\isacharcomma}{\kern0pt}\ E{\isacharparenright}{\kern0pt}\ {\isasymand}\ C\ {\isasymunion}\ D\ {\isasymunion}\ E\ {\isacharequal}{\kern0pt}\ B{\isacharparenright}{\kern0pt}{\isachardoublequoteclose}\isanewline
\ \ \ \ \isacommand{by}\isamarkupfalse%
\ simp\isanewline
\ \ \isacommand{from}\isamarkupfalse%
\ module\ profile\ \isacommand{have}\isamarkupfalse%
\ set{\isacharunderscore}{\kern0pt}partit{\isacharcolon}{\kern0pt}\isanewline
\ \ \ \ {\isachardoublequoteopen}set{\isacharunderscore}{\kern0pt}equals{\isacharunderscore}{\kern0pt}partition\ A\ {\isacharparenleft}{\kern0pt}m\ A\ p{\isacharparenright}{\kern0pt}{\isachardoublequoteclose}\isanewline
\ \ \ \ \isacommand{using}\isamarkupfalse%
\ electoral{\isacharunderscore}{\kern0pt}module{\isacharunderscore}{\kern0pt}def\isanewline
\ \ \ \ \isacommand{by}\isamarkupfalse%
\ auto\isanewline
\ \ \isacommand{from}\isamarkupfalse%
\ profile\ \isacommand{have}\isamarkupfalse%
\ prof{\isacharunderscore}{\kern0pt}p{\isacharcolon}{\kern0pt}\isanewline
\ \ \ \ {\isachardoublequoteopen}finite\ A\ {\isasymand}\ profile\ A\ p{\isachardoublequoteclose}\isanewline
\ \ \ \ \isacommand{by}\isamarkupfalse%
\ simp\isanewline
\ \ \isacommand{from}\isamarkupfalse%
\ module\ prof{\isacharunderscore}{\kern0pt}p\ \isacommand{have}\isamarkupfalse%
\ wf{\isacharunderscore}{\kern0pt}A{\isacharunderscore}{\kern0pt}m{\isacharcolon}{\kern0pt}\isanewline
\ \ \ \ {\isachardoublequoteopen}well{\isacharunderscore}{\kern0pt}formed\ A\ {\isacharparenleft}{\kern0pt}m\ A\ p{\isacharparenright}{\kern0pt}{\isachardoublequoteclose}\isanewline
\ \ \ \ \isacommand{using}\isamarkupfalse%
\ electoral{\isacharunderscore}{\kern0pt}module{\isacharunderscore}{\kern0pt}def\isanewline
\ \ \ \ \isacommand{by}\isamarkupfalse%
\ metis\isanewline
\ \ \isacommand{show}\isamarkupfalse%
\ {\isachardoublequoteopen}False{\isachardoublequoteclose}\isanewline
\ \ \ \ \isacommand{using}\isamarkupfalse%
\ prod{\isachardot}{\kern0pt}exhaust{\isacharunderscore}{\kern0pt}sel\ DiffE\ UnCI\ asm{\isadigit{1}}\ asm{\isadigit{2}}\isanewline
\ \ \ \ \ \ \ \ \ \ module\ profile\ result{\isacharunderscore}{\kern0pt}imp{\isacharunderscore}{\kern0pt}rej\ wf{\isacharunderscore}{\kern0pt}A{\isacharunderscore}{\kern0pt}m\isanewline
\ \ \ \ \ \ \ \ \ \ prof{\isacharunderscore}{\kern0pt}p\ set{\isacharunderscore}{\kern0pt}partit\ partit\isanewline
\ \ \ \ \isacommand{by}\isamarkupfalse%
\ {\isacharparenleft}{\kern0pt}metis\ {\isacharparenleft}{\kern0pt}no{\isacharunderscore}{\kern0pt}types{\isacharparenright}{\kern0pt}{\isacharparenright}{\kern0pt}\isanewline
\isacommand{qed}\isamarkupfalse%
%
\endisatagproof
{\isafoldproof}%
%
\isadelimproof
\isanewline
%
\endisadelimproof
\isanewline
\isacommand{lemma}\isamarkupfalse%
\ elect{\isacharunderscore}{\kern0pt}in{\isacharunderscore}{\kern0pt}alts{\isacharcolon}{\kern0pt}\isanewline
\ \ \isakeyword{assumes}\isanewline
\ \ \ \ e{\isacharunderscore}{\kern0pt}mod{\isacharcolon}{\kern0pt}\ {\isachardoublequoteopen}electoral{\isacharunderscore}{\kern0pt}module\ m{\isachardoublequoteclose}\ \isakeyword{and}\isanewline
\ \ \ \ f{\isacharunderscore}{\kern0pt}prof{\isacharcolon}{\kern0pt}\ {\isachardoublequoteopen}finite{\isacharunderscore}{\kern0pt}profile\ A\ p{\isachardoublequoteclose}\isanewline
\ \ \isakeyword{shows}\ {\isachardoublequoteopen}elect\ m\ A\ p\ {\isasymsubseteq}\ A{\isachardoublequoteclose}\isanewline
%
\isadelimproof
\ \ %
\endisadelimproof
%
\isatagproof
\isacommand{using}\isamarkupfalse%
\ le{\isacharunderscore}{\kern0pt}supI{\isadigit{1}}\ e{\isacharunderscore}{\kern0pt}mod\ f{\isacharunderscore}{\kern0pt}prof\ result{\isacharunderscore}{\kern0pt}presv{\isacharunderscore}{\kern0pt}alts\ sup{\isacharunderscore}{\kern0pt}ge{\isadigit{1}}\isanewline
\ \ \isacommand{by}\isamarkupfalse%
\ metis%
\endisatagproof
{\isafoldproof}%
%
\isadelimproof
\isanewline
%
\endisadelimproof
\isanewline
\isacommand{lemma}\isamarkupfalse%
\ reject{\isacharunderscore}{\kern0pt}in{\isacharunderscore}{\kern0pt}alts{\isacharcolon}{\kern0pt}\isanewline
\ \ \isakeyword{assumes}\isanewline
\ \ \ \ e{\isacharunderscore}{\kern0pt}mod{\isacharcolon}{\kern0pt}\ {\isachardoublequoteopen}electoral{\isacharunderscore}{\kern0pt}module\ m{\isachardoublequoteclose}\ \isakeyword{and}\isanewline
\ \ \ \ f{\isacharunderscore}{\kern0pt}prof{\isacharcolon}{\kern0pt}\ {\isachardoublequoteopen}finite{\isacharunderscore}{\kern0pt}profile\ A\ p{\isachardoublequoteclose}\isanewline
\ \ \isakeyword{shows}\ {\isachardoublequoteopen}reject\ m\ A\ p\ {\isasymsubseteq}\ A{\isachardoublequoteclose}\isanewline
%
\isadelimproof
\ \ %
\endisadelimproof
%
\isatagproof
\isacommand{using}\isamarkupfalse%
\ le{\isacharunderscore}{\kern0pt}supI{\isadigit{1}}\ e{\isacharunderscore}{\kern0pt}mod\ f{\isacharunderscore}{\kern0pt}prof\ result{\isacharunderscore}{\kern0pt}presv{\isacharunderscore}{\kern0pt}alts\ sup{\isacharunderscore}{\kern0pt}ge{\isadigit{2}}\isanewline
\ \ \isacommand{by}\isamarkupfalse%
\ fastforce%
\endisatagproof
{\isafoldproof}%
%
\isadelimproof
\isanewline
%
\endisadelimproof
\isanewline
\isacommand{lemma}\isamarkupfalse%
\ defer{\isacharunderscore}{\kern0pt}in{\isacharunderscore}{\kern0pt}alts{\isacharcolon}{\kern0pt}\isanewline
\ \ \isakeyword{assumes}\isanewline
\ \ \ \ e{\isacharunderscore}{\kern0pt}mod{\isacharcolon}{\kern0pt}\ {\isachardoublequoteopen}electoral{\isacharunderscore}{\kern0pt}module\ m{\isachardoublequoteclose}\ \isakeyword{and}\isanewline
\ \ \ \ f{\isacharunderscore}{\kern0pt}prof{\isacharcolon}{\kern0pt}\ {\isachardoublequoteopen}finite{\isacharunderscore}{\kern0pt}profile\ A\ p{\isachardoublequoteclose}\isanewline
\ \ \isakeyword{shows}\ {\isachardoublequoteopen}defer\ m\ A\ p\ {\isasymsubseteq}\ A{\isachardoublequoteclose}\isanewline
%
\isadelimproof
\ \ %
\endisadelimproof
%
\isatagproof
\isacommand{using}\isamarkupfalse%
\ e{\isacharunderscore}{\kern0pt}mod\ f{\isacharunderscore}{\kern0pt}prof\ result{\isacharunderscore}{\kern0pt}presv{\isacharunderscore}{\kern0pt}alts\isanewline
\ \ \isacommand{by}\isamarkupfalse%
\ auto%
\endisatagproof
{\isafoldproof}%
%
\isadelimproof
\isanewline
%
\endisadelimproof
\isanewline
\isacommand{lemma}\isamarkupfalse%
\ def{\isacharunderscore}{\kern0pt}presv{\isacharunderscore}{\kern0pt}fin{\isacharunderscore}{\kern0pt}prof{\isacharcolon}{\kern0pt}\isanewline
\ \ \isakeyword{assumes}\ module{\isacharcolon}{\kern0pt}\ \ {\isachardoublequoteopen}electoral{\isacharunderscore}{\kern0pt}module\ m{\isachardoublequoteclose}\ \isakeyword{and}\isanewline
\ \ \ \ \ \ \ \ \ \ f{\isacharunderscore}{\kern0pt}prof{\isacharcolon}{\kern0pt}\ {\isachardoublequoteopen}finite{\isacharunderscore}{\kern0pt}profile\ A\ p{\isachardoublequoteclose}\isanewline
\ \ \isakeyword{shows}\isanewline
\ \ \ \ {\isachardoublequoteopen}let\ new{\isacharunderscore}{\kern0pt}A\ {\isacharequal}{\kern0pt}\ defer\ m\ A\ p\ in\isanewline
\ \ \ \ \ \ \ \ finite{\isacharunderscore}{\kern0pt}profile\ new{\isacharunderscore}{\kern0pt}A\ {\isacharparenleft}{\kern0pt}limit{\isacharunderscore}{\kern0pt}profile\ new{\isacharunderscore}{\kern0pt}A\ p{\isacharparenright}{\kern0pt}{\isachardoublequoteclose}\isanewline
%
\isadelimproof
\ \ %
\endisadelimproof
%
\isatagproof
\isacommand{using}\isamarkupfalse%
\ defer{\isacharunderscore}{\kern0pt}in{\isacharunderscore}{\kern0pt}alts\ infinite{\isacharunderscore}{\kern0pt}super\isanewline
\ \ \ \ \ \ \ \ limit{\isacharunderscore}{\kern0pt}profile{\isacharunderscore}{\kern0pt}sound\ module\ f{\isacharunderscore}{\kern0pt}prof\isanewline
\ \ \isacommand{by}\isamarkupfalse%
\ metis%
\endisatagproof
{\isafoldproof}%
%
\isadelimproof
\isanewline
%
\endisadelimproof
\isanewline
\isanewline
\isacommand{lemma}\isamarkupfalse%
\ upper{\isacharunderscore}{\kern0pt}card{\isacharunderscore}{\kern0pt}bounds{\isacharunderscore}{\kern0pt}for{\isacharunderscore}{\kern0pt}result{\isacharcolon}{\kern0pt}\isanewline
\ \ \isakeyword{assumes}\isanewline
\ \ \ \ e{\isacharunderscore}{\kern0pt}mod{\isacharcolon}{\kern0pt}\ {\isachardoublequoteopen}electoral{\isacharunderscore}{\kern0pt}module\ m{\isachardoublequoteclose}\ \isakeyword{and}\isanewline
\ \ \ \ f{\isacharunderscore}{\kern0pt}prof{\isacharcolon}{\kern0pt}\ {\isachardoublequoteopen}finite{\isacharunderscore}{\kern0pt}profile\ A\ p{\isachardoublequoteclose}\isanewline
\ \ \isakeyword{shows}\isanewline
\ \ \ \ {\isachardoublequoteopen}card\ {\isacharparenleft}{\kern0pt}elect\ m\ A\ p{\isacharparenright}{\kern0pt}\ {\isasymle}\ card\ A\ {\isasymand}\isanewline
\ \ \ \ \ \ card\ {\isacharparenleft}{\kern0pt}reject\ m\ A\ p{\isacharparenright}{\kern0pt}\ {\isasymle}\ card\ A\ {\isasymand}\isanewline
\ \ \ \ \ \ card\ {\isacharparenleft}{\kern0pt}defer\ m\ A\ p{\isacharparenright}{\kern0pt}\ {\isasymle}\ card\ A\ {\isachardoublequoteclose}\isanewline
%
\isadelimproof
\ \ %
\endisadelimproof
%
\isatagproof
\isacommand{by}\isamarkupfalse%
\ {\isacharparenleft}{\kern0pt}simp\ add{\isacharcolon}{\kern0pt}\ card{\isacharunderscore}{\kern0pt}mono\ defer{\isacharunderscore}{\kern0pt}in{\isacharunderscore}{\kern0pt}alts\ elect{\isacharunderscore}{\kern0pt}in{\isacharunderscore}{\kern0pt}alts\isanewline
\ \ \ \ \ \ \ \ \ \ \ \ \ \ \ \ e{\isacharunderscore}{\kern0pt}mod\ f{\isacharunderscore}{\kern0pt}prof\ reject{\isacharunderscore}{\kern0pt}in{\isacharunderscore}{\kern0pt}alts{\isacharparenright}{\kern0pt}%
\endisatagproof
{\isafoldproof}%
%
\isadelimproof
\isanewline
%
\endisadelimproof
\isanewline
\isacommand{lemma}\isamarkupfalse%
\ reject{\isacharunderscore}{\kern0pt}not{\isacharunderscore}{\kern0pt}elec{\isacharunderscore}{\kern0pt}or{\isacharunderscore}{\kern0pt}def{\isacharcolon}{\kern0pt}\isanewline
\ \ \isakeyword{assumes}\isanewline
\ \ \ \ e{\isacharunderscore}{\kern0pt}mod{\isacharcolon}{\kern0pt}\ {\isachardoublequoteopen}electoral{\isacharunderscore}{\kern0pt}module\ m{\isachardoublequoteclose}\ \isakeyword{and}\isanewline
\ \ \ \ f{\isacharunderscore}{\kern0pt}prof{\isacharcolon}{\kern0pt}\ {\isachardoublequoteopen}finite{\isacharunderscore}{\kern0pt}profile\ A\ p{\isachardoublequoteclose}\isanewline
\ \ \isakeyword{shows}\ {\isachardoublequoteopen}reject\ m\ A\ p\ {\isacharequal}{\kern0pt}\ A\ {\isacharminus}{\kern0pt}\ {\isacharparenleft}{\kern0pt}elect\ m\ A\ p{\isacharparenright}{\kern0pt}\ {\isacharminus}{\kern0pt}\ {\isacharparenleft}{\kern0pt}defer\ m\ A\ p{\isacharparenright}{\kern0pt}{\isachardoublequoteclose}\isanewline
%
\isadelimproof
%
\endisadelimproof
%
\isatagproof
\isacommand{proof}\isamarkupfalse%
\ {\isacharminus}{\kern0pt}\isanewline
\ \ \isacommand{from}\isamarkupfalse%
\ e{\isacharunderscore}{\kern0pt}mod\ f{\isacharunderscore}{\kern0pt}prof\ \isacommand{have}\isamarkupfalse%
\ {\isadigit{0}}{\isacharcolon}{\kern0pt}\ {\isachardoublequoteopen}well{\isacharunderscore}{\kern0pt}formed\ A\ {\isacharparenleft}{\kern0pt}m\ A\ p{\isacharparenright}{\kern0pt}{\isachardoublequoteclose}\isanewline
\ \ \ \ \isacommand{by}\isamarkupfalse%
\ {\isacharparenleft}{\kern0pt}simp\ add{\isacharcolon}{\kern0pt}\ electoral{\isacharunderscore}{\kern0pt}module{\isacharunderscore}{\kern0pt}def{\isacharparenright}{\kern0pt}\isanewline
\ \ \isacommand{with}\isamarkupfalse%
\ e{\isacharunderscore}{\kern0pt}mod\ f{\isacharunderscore}{\kern0pt}prof\isanewline
\ \ \ \ \isacommand{have}\isamarkupfalse%
\ {\isachardoublequoteopen}{\isacharparenleft}{\kern0pt}elect\ m\ A\ p{\isacharparenright}{\kern0pt}\ {\isasymunion}\ {\isacharparenleft}{\kern0pt}reject\ m\ A\ p{\isacharparenright}{\kern0pt}\ {\isasymunion}\ {\isacharparenleft}{\kern0pt}defer\ m\ A\ p{\isacharparenright}{\kern0pt}\ {\isacharequal}{\kern0pt}\ A{\isachardoublequoteclose}\isanewline
\ \ \ \ \ \ \isacommand{using}\isamarkupfalse%
\ result{\isacharunderscore}{\kern0pt}presv{\isacharunderscore}{\kern0pt}alts\isanewline
\ \ \ \ \ \ \isacommand{by}\isamarkupfalse%
\ simp\isanewline
\ \ \ \ \isacommand{moreover}\isamarkupfalse%
\ \isacommand{from}\isamarkupfalse%
\ {\isadigit{0}}\ \isacommand{have}\isamarkupfalse%
\isanewline
\ \ \ \ \ \ {\isachardoublequoteopen}{\isacharparenleft}{\kern0pt}elect\ m\ A\ p{\isacharparenright}{\kern0pt}\ {\isasyminter}\ {\isacharparenleft}{\kern0pt}reject\ m\ A\ p{\isacharparenright}{\kern0pt}\ {\isacharequal}{\kern0pt}\ {\isacharbraceleft}{\kern0pt}{\isacharbraceright}{\kern0pt}\ {\isasymand}\isanewline
\ \ \ \ \ \ \ \ \ \ {\isacharparenleft}{\kern0pt}reject\ m\ A\ p{\isacharparenright}{\kern0pt}\ {\isasyminter}\ {\isacharparenleft}{\kern0pt}defer\ m\ A\ p{\isacharparenright}{\kern0pt}\ {\isacharequal}{\kern0pt}\ {\isacharbraceleft}{\kern0pt}{\isacharbraceright}{\kern0pt}{\isachardoublequoteclose}\isanewline
\ \ \ \ \isacommand{using}\isamarkupfalse%
\ e{\isacharunderscore}{\kern0pt}mod\ f{\isacharunderscore}{\kern0pt}prof\ result{\isacharunderscore}{\kern0pt}disj\isanewline
\ \ \ \ \isacommand{by}\isamarkupfalse%
\ blast\isanewline
\ \ \isacommand{ultimately}\isamarkupfalse%
\ \isacommand{show}\isamarkupfalse%
\ {\isacharquery}{\kern0pt}thesis\isanewline
\ \ \ \ \isacommand{by}\isamarkupfalse%
\ blast\isanewline
\isacommand{qed}\isamarkupfalse%
%
\endisatagproof
{\isafoldproof}%
%
\isadelimproof
\isanewline
%
\endisadelimproof
\isanewline
\isacommand{lemma}\isamarkupfalse%
\ elec{\isacharunderscore}{\kern0pt}and{\isacharunderscore}{\kern0pt}def{\isacharunderscore}{\kern0pt}not{\isacharunderscore}{\kern0pt}rej{\isacharcolon}{\kern0pt}\isanewline
\ \ \isakeyword{assumes}\isanewline
\ \ \ \ e{\isacharunderscore}{\kern0pt}mod{\isacharcolon}{\kern0pt}\ {\isachardoublequoteopen}electoral{\isacharunderscore}{\kern0pt}module\ m{\isachardoublequoteclose}\ \isakeyword{and}\isanewline
\ \ \ \ f{\isacharunderscore}{\kern0pt}prof{\isacharcolon}{\kern0pt}\ {\isachardoublequoteopen}finite{\isacharunderscore}{\kern0pt}profile\ A\ p{\isachardoublequoteclose}\isanewline
\ \ \isakeyword{shows}\ {\isachardoublequoteopen}elect\ m\ A\ p\ {\isasymunion}\ defer\ m\ A\ p\ {\isacharequal}{\kern0pt}\ A\ {\isacharminus}{\kern0pt}\ {\isacharparenleft}{\kern0pt}reject\ m\ A\ p{\isacharparenright}{\kern0pt}{\isachardoublequoteclose}\isanewline
%
\isadelimproof
%
\endisadelimproof
%
\isatagproof
\isacommand{proof}\isamarkupfalse%
\ {\isacharminus}{\kern0pt}\isanewline
\ \ \isacommand{from}\isamarkupfalse%
\ e{\isacharunderscore}{\kern0pt}mod\ f{\isacharunderscore}{\kern0pt}prof\ \isacommand{have}\isamarkupfalse%
\ {\isadigit{0}}{\isacharcolon}{\kern0pt}\ {\isachardoublequoteopen}well{\isacharunderscore}{\kern0pt}formed\ A\ {\isacharparenleft}{\kern0pt}m\ A\ p{\isacharparenright}{\kern0pt}{\isachardoublequoteclose}\isanewline
\ \ \ \ \isacommand{by}\isamarkupfalse%
\ {\isacharparenleft}{\kern0pt}simp\ add{\isacharcolon}{\kern0pt}\ electoral{\isacharunderscore}{\kern0pt}module{\isacharunderscore}{\kern0pt}def{\isacharparenright}{\kern0pt}\isanewline
\ \ \isacommand{hence}\isamarkupfalse%
\isanewline
\ \ \ \ {\isachardoublequoteopen}disjoint{\isadigit{3}}\ {\isacharparenleft}{\kern0pt}m\ A\ p{\isacharparenright}{\kern0pt}\ {\isasymand}\ set{\isacharunderscore}{\kern0pt}equals{\isacharunderscore}{\kern0pt}partition\ A\ {\isacharparenleft}{\kern0pt}m\ A\ p{\isacharparenright}{\kern0pt}{\isachardoublequoteclose}\isanewline
\ \ \ \ \isacommand{by}\isamarkupfalse%
\ simp\isanewline
\ \ \isacommand{with}\isamarkupfalse%
\ e{\isacharunderscore}{\kern0pt}mod\ f{\isacharunderscore}{\kern0pt}prof\isanewline
\ \ \isacommand{have}\isamarkupfalse%
\ {\isachardoublequoteopen}{\isacharparenleft}{\kern0pt}elect\ m\ A\ p{\isacharparenright}{\kern0pt}\ {\isasymunion}\ {\isacharparenleft}{\kern0pt}reject\ m\ A\ p{\isacharparenright}{\kern0pt}\ {\isasymunion}\ {\isacharparenleft}{\kern0pt}defer\ m\ A\ p{\isacharparenright}{\kern0pt}\ {\isacharequal}{\kern0pt}\ A{\isachardoublequoteclose}\isanewline
\ \ \ \ \isacommand{using}\isamarkupfalse%
\ e{\isacharunderscore}{\kern0pt}mod\ f{\isacharunderscore}{\kern0pt}prof\ result{\isacharunderscore}{\kern0pt}presv{\isacharunderscore}{\kern0pt}alts\isanewline
\ \ \ \ \isacommand{by}\isamarkupfalse%
\ blast\isanewline
\ \ \isacommand{moreover}\isamarkupfalse%
\ \isacommand{from}\isamarkupfalse%
\ {\isadigit{0}}\ \isacommand{have}\isamarkupfalse%
\isanewline
\ \ \ \ {\isachardoublequoteopen}{\isacharparenleft}{\kern0pt}elect\ m\ A\ p{\isacharparenright}{\kern0pt}\ {\isasyminter}\ {\isacharparenleft}{\kern0pt}reject\ m\ A\ p{\isacharparenright}{\kern0pt}\ {\isacharequal}{\kern0pt}\ {\isacharbraceleft}{\kern0pt}{\isacharbraceright}{\kern0pt}\ {\isasymand}\isanewline
\ \ \ \ \ \ \ \ {\isacharparenleft}{\kern0pt}reject\ m\ A\ p{\isacharparenright}{\kern0pt}\ {\isasyminter}\ {\isacharparenleft}{\kern0pt}defer\ m\ A\ p{\isacharparenright}{\kern0pt}\ {\isacharequal}{\kern0pt}\ {\isacharbraceleft}{\kern0pt}{\isacharbraceright}{\kern0pt}{\isachardoublequoteclose}\isanewline
\ \ \ \ \isacommand{using}\isamarkupfalse%
\ e{\isacharunderscore}{\kern0pt}mod\ f{\isacharunderscore}{\kern0pt}prof\ result{\isacharunderscore}{\kern0pt}disj\isanewline
\ \ \ \ \isacommand{by}\isamarkupfalse%
\ blast\isanewline
\ \ \isacommand{ultimately}\isamarkupfalse%
\ \isacommand{show}\isamarkupfalse%
\ {\isacharquery}{\kern0pt}thesis\isanewline
\ \ \ \ \isacommand{by}\isamarkupfalse%
\ blast\isanewline
\isacommand{qed}\isamarkupfalse%
%
\endisatagproof
{\isafoldproof}%
%
\isadelimproof
\isanewline
%
\endisadelimproof
\isanewline
\isacommand{lemma}\isamarkupfalse%
\ defer{\isacharunderscore}{\kern0pt}not{\isacharunderscore}{\kern0pt}elec{\isacharunderscore}{\kern0pt}or{\isacharunderscore}{\kern0pt}rej{\isacharcolon}{\kern0pt}\isanewline
\ \ \isakeyword{assumes}\isanewline
\ \ \ \ e{\isacharunderscore}{\kern0pt}mod{\isacharcolon}{\kern0pt}\ {\isachardoublequoteopen}electoral{\isacharunderscore}{\kern0pt}module\ m{\isachardoublequoteclose}\ \isakeyword{and}\isanewline
\ \ \ \ f{\isacharunderscore}{\kern0pt}prof{\isacharcolon}{\kern0pt}\ {\isachardoublequoteopen}finite{\isacharunderscore}{\kern0pt}profile\ A\ p{\isachardoublequoteclose}\isanewline
\ \ \isakeyword{shows}\ {\isachardoublequoteopen}defer\ m\ A\ p\ {\isacharequal}{\kern0pt}\ A\ {\isacharminus}{\kern0pt}\ {\isacharparenleft}{\kern0pt}elect\ m\ A\ p{\isacharparenright}{\kern0pt}\ {\isacharminus}{\kern0pt}\ {\isacharparenleft}{\kern0pt}reject\ m\ A\ p{\isacharparenright}{\kern0pt}{\isachardoublequoteclose}\isanewline
%
\isadelimproof
%
\endisadelimproof
%
\isatagproof
\isacommand{proof}\isamarkupfalse%
\ {\isacharminus}{\kern0pt}\isanewline
\ \ \isacommand{from}\isamarkupfalse%
\ e{\isacharunderscore}{\kern0pt}mod\ f{\isacharunderscore}{\kern0pt}prof\ \isacommand{have}\isamarkupfalse%
\ {\isadigit{0}}{\isacharcolon}{\kern0pt}\ {\isachardoublequoteopen}well{\isacharunderscore}{\kern0pt}formed\ A\ {\isacharparenleft}{\kern0pt}m\ A\ p{\isacharparenright}{\kern0pt}{\isachardoublequoteclose}\isanewline
\ \ \ \ \isacommand{by}\isamarkupfalse%
\ {\isacharparenleft}{\kern0pt}simp\ add{\isacharcolon}{\kern0pt}\ electoral{\isacharunderscore}{\kern0pt}module{\isacharunderscore}{\kern0pt}def{\isacharparenright}{\kern0pt}\isanewline
\ \ \isacommand{hence}\isamarkupfalse%
\ {\isachardoublequoteopen}{\isacharparenleft}{\kern0pt}elect\ m\ A\ p{\isacharparenright}{\kern0pt}\ {\isasymunion}\ {\isacharparenleft}{\kern0pt}reject\ m\ A\ p{\isacharparenright}{\kern0pt}\ {\isasymunion}\ {\isacharparenleft}{\kern0pt}defer\ m\ A\ p{\isacharparenright}{\kern0pt}\ {\isacharequal}{\kern0pt}\ A{\isachardoublequoteclose}\isanewline
\ \ \ \ \isacommand{using}\isamarkupfalse%
\ e{\isacharunderscore}{\kern0pt}mod\ f{\isacharunderscore}{\kern0pt}prof\ result{\isacharunderscore}{\kern0pt}presv{\isacharunderscore}{\kern0pt}alts\isanewline
\ \ \ \ \isacommand{by}\isamarkupfalse%
\ auto\isanewline
\ \ \isacommand{moreover}\isamarkupfalse%
\ \isacommand{from}\isamarkupfalse%
\ {\isadigit{0}}\ \isacommand{have}\isamarkupfalse%
\isanewline
\ \ \ \ {\isachardoublequoteopen}{\isacharparenleft}{\kern0pt}elect\ m\ A\ p{\isacharparenright}{\kern0pt}\ {\isasyminter}\ {\isacharparenleft}{\kern0pt}defer\ m\ A\ p{\isacharparenright}{\kern0pt}\ {\isacharequal}{\kern0pt}\ {\isacharbraceleft}{\kern0pt}{\isacharbraceright}{\kern0pt}\ {\isasymand}\isanewline
\ \ \ \ \ \ \ \ {\isacharparenleft}{\kern0pt}reject\ m\ A\ p{\isacharparenright}{\kern0pt}\ {\isasyminter}\ {\isacharparenleft}{\kern0pt}defer\ m\ A\ p{\isacharparenright}{\kern0pt}\ {\isacharequal}{\kern0pt}\ {\isacharbraceleft}{\kern0pt}{\isacharbraceright}{\kern0pt}{\isachardoublequoteclose}\isanewline
\ \ \ \ \ \ \isacommand{using}\isamarkupfalse%
\ e{\isacharunderscore}{\kern0pt}mod\ f{\isacharunderscore}{\kern0pt}prof\ result{\isacharunderscore}{\kern0pt}disj\isanewline
\ \ \ \ \ \ \isacommand{by}\isamarkupfalse%
\ blast\isanewline
\ \ \isacommand{ultimately}\isamarkupfalse%
\ \isacommand{show}\isamarkupfalse%
\ {\isacharquery}{\kern0pt}thesis\isanewline
\ \ \ \ \isacommand{by}\isamarkupfalse%
\ blast\isanewline
\isacommand{qed}\isamarkupfalse%
%
\endisatagproof
{\isafoldproof}%
%
\isadelimproof
\isanewline
%
\endisadelimproof
\isanewline
\isacommand{lemma}\isamarkupfalse%
\ electoral{\isacharunderscore}{\kern0pt}mod{\isacharunderscore}{\kern0pt}defer{\isacharunderscore}{\kern0pt}elem{\isacharcolon}{\kern0pt}\isanewline
\ \ \isakeyword{assumes}\isanewline
\ \ \ \ e{\isacharunderscore}{\kern0pt}mod{\isacharcolon}{\kern0pt}\ {\isachardoublequoteopen}electoral{\isacharunderscore}{\kern0pt}module\ m{\isachardoublequoteclose}\ \isakeyword{and}\isanewline
\ \ \ \ f{\isacharunderscore}{\kern0pt}prof{\isacharcolon}{\kern0pt}\ {\isachardoublequoteopen}finite{\isacharunderscore}{\kern0pt}profile\ A\ p{\isachardoublequoteclose}\ \isakeyword{and}\isanewline
\ \ \ \ alternative{\isacharcolon}{\kern0pt}\ {\isachardoublequoteopen}x\ {\isasymin}\ A{\isachardoublequoteclose}\ \isakeyword{and}\isanewline
\ \ \ \ not{\isacharunderscore}{\kern0pt}elected{\isacharcolon}{\kern0pt}\ {\isachardoublequoteopen}x\ {\isasymnotin}\ elect\ m\ A\ p{\isachardoublequoteclose}\ \isakeyword{and}\isanewline
\ \ \ \ not{\isacharunderscore}{\kern0pt}rejected{\isacharcolon}{\kern0pt}\ {\isachardoublequoteopen}x\ {\isasymnotin}\ reject\ m\ A\ p{\isachardoublequoteclose}\isanewline
\ \ \isakeyword{shows}\ {\isachardoublequoteopen}x\ {\isasymin}\ defer\ m\ A\ p{\isachardoublequoteclose}\isanewline
%
\isadelimproof
\ \ %
\endisadelimproof
%
\isatagproof
\isacommand{using}\isamarkupfalse%
\ DiffI\ e{\isacharunderscore}{\kern0pt}mod\ f{\isacharunderscore}{\kern0pt}prof\ alternative\isanewline
\ \ \ \ \ \ \ \ not{\isacharunderscore}{\kern0pt}elected\ not{\isacharunderscore}{\kern0pt}rejected\isanewline
\ \ \ \ \ \ \ \ reject{\isacharunderscore}{\kern0pt}not{\isacharunderscore}{\kern0pt}elec{\isacharunderscore}{\kern0pt}or{\isacharunderscore}{\kern0pt}def\isanewline
\ \ \isacommand{by}\isamarkupfalse%
\ metis%
\endisatagproof
{\isafoldproof}%
%
\isadelimproof
\isanewline
%
\endisadelimproof
\isanewline
\isacommand{lemma}\isamarkupfalse%
\ mod{\isacharunderscore}{\kern0pt}contains{\isacharunderscore}{\kern0pt}result{\isacharunderscore}{\kern0pt}comm{\isacharcolon}{\kern0pt}\isanewline
\ \ \isakeyword{assumes}\ {\isachardoublequoteopen}mod{\isacharunderscore}{\kern0pt}contains{\isacharunderscore}{\kern0pt}result\ m\ n\ A\ p\ a{\isachardoublequoteclose}\isanewline
\ \ \isakeyword{shows}\ {\isachardoublequoteopen}mod{\isacharunderscore}{\kern0pt}contains{\isacharunderscore}{\kern0pt}result\ n\ m\ A\ p\ a{\isachardoublequoteclose}\isanewline
%
\isadelimproof
\ \ %
\endisadelimproof
%
\isatagproof
\isacommand{using}\isamarkupfalse%
\ IntI\ assms\ electoral{\isacharunderscore}{\kern0pt}mod{\isacharunderscore}{\kern0pt}defer{\isacharunderscore}{\kern0pt}elem\ empty{\isacharunderscore}{\kern0pt}iff\isanewline
\ \ \ \ \ \ \ \ mod{\isacharunderscore}{\kern0pt}contains{\isacharunderscore}{\kern0pt}result{\isacharunderscore}{\kern0pt}def\ result{\isacharunderscore}{\kern0pt}disj\isanewline
\ \ \isacommand{by}\isamarkupfalse%
\ {\isacharparenleft}{\kern0pt}smt\ {\isacharparenleft}{\kern0pt}verit{\isacharcomma}{\kern0pt}\ ccfv{\isacharunderscore}{\kern0pt}threshold{\isacharparenright}{\kern0pt}{\isacharparenright}{\kern0pt}%
\endisatagproof
{\isafoldproof}%
%
\isadelimproof
\isanewline
%
\endisadelimproof
\isanewline
\isacommand{lemma}\isamarkupfalse%
\ not{\isacharunderscore}{\kern0pt}rej{\isacharunderscore}{\kern0pt}imp{\isacharunderscore}{\kern0pt}elec{\isacharunderscore}{\kern0pt}or{\isacharunderscore}{\kern0pt}def{\isacharcolon}{\kern0pt}\isanewline
\ \ \isakeyword{assumes}\isanewline
\ \ \ \ e{\isacharunderscore}{\kern0pt}mod{\isacharcolon}{\kern0pt}\ {\isachardoublequoteopen}electoral{\isacharunderscore}{\kern0pt}module\ m{\isachardoublequoteclose}\ \isakeyword{and}\isanewline
\ \ \ \ f{\isacharunderscore}{\kern0pt}prof{\isacharcolon}{\kern0pt}\ {\isachardoublequoteopen}finite{\isacharunderscore}{\kern0pt}profile\ A\ p{\isachardoublequoteclose}\ \isakeyword{and}\isanewline
\ \ \ \ alternative{\isacharcolon}{\kern0pt}\ {\isachardoublequoteopen}x\ {\isasymin}\ A{\isachardoublequoteclose}\ \isakeyword{and}\isanewline
\ \ \ \ not{\isacharunderscore}{\kern0pt}rejected{\isacharcolon}{\kern0pt}\ {\isachardoublequoteopen}x\ {\isasymnotin}\ reject\ m\ A\ p{\isachardoublequoteclose}\isanewline
\ \ \isakeyword{shows}\ {\isachardoublequoteopen}x\ {\isasymin}\ elect\ m\ A\ p\ {\isasymor}\ x\ {\isasymin}\ defer\ m\ A\ p{\isachardoublequoteclose}\isanewline
%
\isadelimproof
\ \ %
\endisadelimproof
%
\isatagproof
\isacommand{using}\isamarkupfalse%
\ alternative\ electoral{\isacharunderscore}{\kern0pt}mod{\isacharunderscore}{\kern0pt}defer{\isacharunderscore}{\kern0pt}elem\isanewline
\ \ \ \ \ \ \ \ e{\isacharunderscore}{\kern0pt}mod\ not{\isacharunderscore}{\kern0pt}rejected\ f{\isacharunderscore}{\kern0pt}prof\isanewline
\ \ \isacommand{by}\isamarkupfalse%
\ metis%
\endisatagproof
{\isafoldproof}%
%
\isadelimproof
\isanewline
%
\endisadelimproof
\isanewline
\isacommand{lemma}\isamarkupfalse%
\ eq{\isacharunderscore}{\kern0pt}alts{\isacharunderscore}{\kern0pt}in{\isacharunderscore}{\kern0pt}profs{\isacharunderscore}{\kern0pt}imp{\isacharunderscore}{\kern0pt}eq{\isacharunderscore}{\kern0pt}results{\isacharcolon}{\kern0pt}\isanewline
\ \ \isakeyword{assumes}\isanewline
\ \ \ \ eq{\isacharcolon}{\kern0pt}\ {\isachardoublequoteopen}{\isasymforall}a\ {\isasymin}\ A{\isachardot}{\kern0pt}\ prof{\isacharunderscore}{\kern0pt}contains{\isacharunderscore}{\kern0pt}result\ m\ A\ p\ q\ a{\isachardoublequoteclose}\ \isakeyword{and}\isanewline
\ \ \ \ \isanewline
\ \ \ \ input{\isacharcolon}{\kern0pt}\ {\isachardoublequoteopen}electoral{\isacharunderscore}{\kern0pt}module\ m\ {\isasymand}\ finite{\isacharunderscore}{\kern0pt}profile\ A\ p\ {\isasymand}\ finite{\isacharunderscore}{\kern0pt}profile\ A\ q{\isachardoublequoteclose}\isanewline
\ \ \isakeyword{shows}\ {\isachardoublequoteopen}m\ A\ p\ {\isacharequal}{\kern0pt}\ m\ A\ q{\isachardoublequoteclose}\isanewline
%
\isadelimproof
%
\endisadelimproof
%
\isatagproof
\isacommand{proof}\isamarkupfalse%
\ {\isacharminus}{\kern0pt}\isanewline
\ \ \isacommand{have}\isamarkupfalse%
\ {\isachardoublequoteopen}{\isasymforall}a\ {\isasymin}\ elect\ m\ A\ p{\isachardot}{\kern0pt}\ a\ {\isasymin}\ elect\ m\ A\ q{\isachardoublequoteclose}\isanewline
\ \ \ \ \isacommand{using}\isamarkupfalse%
\ elect{\isacharunderscore}{\kern0pt}in{\isacharunderscore}{\kern0pt}alts\ eq\ prof{\isacharunderscore}{\kern0pt}contains{\isacharunderscore}{\kern0pt}result{\isacharunderscore}{\kern0pt}def\ input\ in{\isacharunderscore}{\kern0pt}mono\isanewline
\ \ \ \ \isacommand{by}\isamarkupfalse%
\ metis\isanewline
\ \ \isacommand{moreover}\isamarkupfalse%
\ \isacommand{have}\isamarkupfalse%
\ {\isachardoublequoteopen}{\isasymforall}a\ {\isasymin}\ elect\ m\ A\ q{\isachardot}{\kern0pt}\ a\ {\isasymin}\ elect\ m\ A\ p{\isachardoublequoteclose}\isanewline
\ \ \ \ \isacommand{using}\isamarkupfalse%
\ contra{\isacharunderscore}{\kern0pt}subsetD\ disjoint{\isacharunderscore}{\kern0pt}iff{\isacharunderscore}{\kern0pt}not{\isacharunderscore}{\kern0pt}equal\ elect{\isacharunderscore}{\kern0pt}in{\isacharunderscore}{\kern0pt}alts\isanewline
\ \ \ \ \ \ \ \ \ \ electoral{\isacharunderscore}{\kern0pt}mod{\isacharunderscore}{\kern0pt}defer{\isacharunderscore}{\kern0pt}elem\ eq\ prof{\isacharunderscore}{\kern0pt}contains{\isacharunderscore}{\kern0pt}result{\isacharunderscore}{\kern0pt}def\ input\isanewline
\ \ \ \ \ \ \ \ \ \ result{\isacharunderscore}{\kern0pt}disj\isanewline
\ \ \ \ \isacommand{by}\isamarkupfalse%
\ {\isacharparenleft}{\kern0pt}smt\ {\isacharparenleft}{\kern0pt}verit{\isacharcomma}{\kern0pt}\ best{\isacharparenright}{\kern0pt}{\isacharparenright}{\kern0pt}\isanewline
\ \ \isacommand{moreover}\isamarkupfalse%
\ \isacommand{have}\isamarkupfalse%
\ {\isachardoublequoteopen}{\isasymforall}a\ {\isasymin}\ reject\ m\ A\ p{\isachardot}{\kern0pt}\ a\ {\isasymin}\ reject\ m\ A\ q{\isachardoublequoteclose}\isanewline
\ \ \ \ \isacommand{using}\isamarkupfalse%
\ reject{\isacharunderscore}{\kern0pt}in{\isacharunderscore}{\kern0pt}alts\ eq\ prof{\isacharunderscore}{\kern0pt}contains{\isacharunderscore}{\kern0pt}result{\isacharunderscore}{\kern0pt}def\ input\ in{\isacharunderscore}{\kern0pt}mono\isanewline
\ \ \ \ \isacommand{by}\isamarkupfalse%
\ fastforce\isanewline
\ \ \isacommand{moreover}\isamarkupfalse%
\ \isacommand{have}\isamarkupfalse%
\ {\isachardoublequoteopen}{\isasymforall}a\ {\isasymin}\ reject\ m\ A\ q{\isachardot}{\kern0pt}\ a\ {\isasymin}\ reject\ m\ A\ p{\isachardoublequoteclose}\isanewline
\ \ \ \ \isacommand{using}\isamarkupfalse%
\ contra{\isacharunderscore}{\kern0pt}subsetD\ disjoint{\isacharunderscore}{\kern0pt}iff{\isacharunderscore}{\kern0pt}not{\isacharunderscore}{\kern0pt}equal\ reject{\isacharunderscore}{\kern0pt}in{\isacharunderscore}{\kern0pt}alts\isanewline
\ \ \ \ \ \ \ \ \ \ electoral{\isacharunderscore}{\kern0pt}mod{\isacharunderscore}{\kern0pt}defer{\isacharunderscore}{\kern0pt}elem\ eq\ prof{\isacharunderscore}{\kern0pt}contains{\isacharunderscore}{\kern0pt}result{\isacharunderscore}{\kern0pt}def\isanewline
\ \ \ \ \ \ \ \ \ \ input\ result{\isacharunderscore}{\kern0pt}disj\isanewline
\ \ \ \ \isacommand{by}\isamarkupfalse%
\ {\isacharparenleft}{\kern0pt}smt\ {\isacharparenleft}{\kern0pt}verit{\isacharcomma}{\kern0pt}\ ccfv{\isacharunderscore}{\kern0pt}SIG{\isacharparenright}{\kern0pt}{\isacharparenright}{\kern0pt}\isanewline
\ \ \isacommand{moreover}\isamarkupfalse%
\ \isacommand{have}\isamarkupfalse%
\ {\isachardoublequoteopen}{\isasymforall}a\ {\isasymin}\ defer\ m\ A\ p{\isachardot}{\kern0pt}\ a\ {\isasymin}\ defer\ m\ A\ q{\isachardoublequoteclose}\isanewline
\ \ \ \ \isacommand{using}\isamarkupfalse%
\ defer{\isacharunderscore}{\kern0pt}in{\isacharunderscore}{\kern0pt}alts\ eq\ prof{\isacharunderscore}{\kern0pt}contains{\isacharunderscore}{\kern0pt}result{\isacharunderscore}{\kern0pt}def\ input\ in{\isacharunderscore}{\kern0pt}mono\isanewline
\ \ \ \ \isacommand{by}\isamarkupfalse%
\ fastforce\isanewline
\ \ \isacommand{moreover}\isamarkupfalse%
\ \isacommand{have}\isamarkupfalse%
\ {\isachardoublequoteopen}{\isasymforall}a\ {\isasymin}\ defer\ m\ A\ q{\isachardot}{\kern0pt}\ a\ {\isasymin}\ defer\ m\ A\ p{\isachardoublequoteclose}\isanewline
\ \ \ \ \isacommand{using}\isamarkupfalse%
\ contra{\isacharunderscore}{\kern0pt}subsetD\ disjoint{\isacharunderscore}{\kern0pt}iff{\isacharunderscore}{\kern0pt}not{\isacharunderscore}{\kern0pt}equal\ defer{\isacharunderscore}{\kern0pt}in{\isacharunderscore}{\kern0pt}alts\isanewline
\ \ \ \ \ \ \ \ \ \ electoral{\isacharunderscore}{\kern0pt}mod{\isacharunderscore}{\kern0pt}defer{\isacharunderscore}{\kern0pt}elem\ eq\ prof{\isacharunderscore}{\kern0pt}contains{\isacharunderscore}{\kern0pt}result{\isacharunderscore}{\kern0pt}def\isanewline
\ \ \ \ \ \ \ \ \ \ input\ result{\isacharunderscore}{\kern0pt}disj\isanewline
\ \ \ \ \isacommand{by}\isamarkupfalse%
\ {\isacharparenleft}{\kern0pt}smt\ {\isacharparenleft}{\kern0pt}verit{\isacharcomma}{\kern0pt}\ best{\isacharparenright}{\kern0pt}{\isacharparenright}{\kern0pt}\isanewline
\ \ \isacommand{ultimately}\isamarkupfalse%
\ \isacommand{show}\isamarkupfalse%
\ {\isacharquery}{\kern0pt}thesis\isanewline
\ \ \ \ \isacommand{using}\isamarkupfalse%
\ prod{\isachardot}{\kern0pt}collapse\ subsetI\ subset{\isacharunderscore}{\kern0pt}antisym\isanewline
\ \ \ \ \isacommand{by}\isamarkupfalse%
\ metis\isanewline
\isacommand{qed}\isamarkupfalse%
%
\endisatagproof
{\isafoldproof}%
%
\isadelimproof
\isanewline
%
\endisadelimproof
\isanewline
\isacommand{lemma}\isamarkupfalse%
\ eq{\isacharunderscore}{\kern0pt}def{\isacharunderscore}{\kern0pt}and{\isacharunderscore}{\kern0pt}elect{\isacharunderscore}{\kern0pt}imp{\isacharunderscore}{\kern0pt}eq{\isacharcolon}{\kern0pt}\isanewline
\ \ \isakeyword{assumes}\isanewline
\ \ \ \ {\isachardoublequoteopen}electoral{\isacharunderscore}{\kern0pt}module\ m{\isachardoublequoteclose}\ \isakeyword{and}\isanewline
\ \ \ \ {\isachardoublequoteopen}electoral{\isacharunderscore}{\kern0pt}module\ n{\isachardoublequoteclose}\ \isakeyword{and}\isanewline
\ \ \ \ {\isachardoublequoteopen}finite{\isacharunderscore}{\kern0pt}profile\ A\ p{\isachardoublequoteclose}\ \isakeyword{and}\isanewline
\ \ \ \ {\isachardoublequoteopen}finite{\isacharunderscore}{\kern0pt}profile\ A\ q{\isachardoublequoteclose}\ \isakeyword{and}\isanewline
\ \ \ \ {\isachardoublequoteopen}elect\ m\ A\ p\ {\isacharequal}{\kern0pt}\ elect\ n\ A\ q{\isachardoublequoteclose}\ \isakeyword{and}\isanewline
\ \ \ \ {\isachardoublequoteopen}defer\ m\ A\ p\ {\isacharequal}{\kern0pt}\ defer\ n\ A\ q{\isachardoublequoteclose}\isanewline
\ \ \isakeyword{shows}\ {\isachardoublequoteopen}m\ A\ p\ {\isacharequal}{\kern0pt}\ n\ A\ q{\isachardoublequoteclose}\isanewline
%
\isadelimproof
%
\endisadelimproof
%
\isatagproof
\isacommand{proof}\isamarkupfalse%
\ {\isacharminus}{\kern0pt}\isanewline
\ \ \isacommand{have}\isamarkupfalse%
\ disj{\isacharunderscore}{\kern0pt}m{\isacharcolon}{\kern0pt}\isanewline
\ \ \ \ {\isachardoublequoteopen}disjoint{\isadigit{3}}\ {\isacharparenleft}{\kern0pt}m\ A\ p{\isacharparenright}{\kern0pt}{\isachardoublequoteclose}\isanewline
\ \ \ \ \isacommand{using}\isamarkupfalse%
\ assms{\isacharparenleft}{\kern0pt}{\isadigit{1}}{\isacharparenright}{\kern0pt}\ assms{\isacharparenleft}{\kern0pt}{\isadigit{3}}{\isacharparenright}{\kern0pt}\ electoral{\isacharunderscore}{\kern0pt}module{\isacharunderscore}{\kern0pt}def\isanewline
\ \ \ \ \isacommand{by}\isamarkupfalse%
\ auto\isanewline
\ \ \isacommand{have}\isamarkupfalse%
\ disj{\isacharunderscore}{\kern0pt}n{\isacharcolon}{\kern0pt}\isanewline
\ \ \ \ {\isachardoublequoteopen}disjoint{\isadigit{3}}\ {\isacharparenleft}{\kern0pt}n\ A\ q{\isacharparenright}{\kern0pt}{\isachardoublequoteclose}\isanewline
\ \ \ \ \isacommand{using}\isamarkupfalse%
\ assms{\isacharparenleft}{\kern0pt}{\isadigit{2}}{\isacharparenright}{\kern0pt}\ assms{\isacharparenleft}{\kern0pt}{\isadigit{4}}{\isacharparenright}{\kern0pt}\ electoral{\isacharunderscore}{\kern0pt}module{\isacharunderscore}{\kern0pt}def\isanewline
\ \ \ \ \isacommand{by}\isamarkupfalse%
\ auto\isanewline
\ \ \isacommand{have}\isamarkupfalse%
\ set{\isacharunderscore}{\kern0pt}partit{\isacharunderscore}{\kern0pt}m{\isacharcolon}{\kern0pt}\isanewline
\ \ \ \ {\isachardoublequoteopen}set{\isacharunderscore}{\kern0pt}equals{\isacharunderscore}{\kern0pt}partition\ A\ {\isacharparenleft}{\kern0pt}{\isacharparenleft}{\kern0pt}elect\ m\ A\ p{\isacharparenright}{\kern0pt}{\isacharcomma}{\kern0pt}\ {\isacharparenleft}{\kern0pt}reject\ m\ A\ p{\isacharparenright}{\kern0pt}{\isacharcomma}{\kern0pt}\ {\isacharparenleft}{\kern0pt}defer\ m\ A\ p{\isacharparenright}{\kern0pt}{\isacharparenright}{\kern0pt}{\isachardoublequoteclose}\isanewline
\ \ \ \ \isacommand{using}\isamarkupfalse%
\ assms{\isacharparenleft}{\kern0pt}{\isadigit{1}}{\isacharparenright}{\kern0pt}\ assms{\isacharparenleft}{\kern0pt}{\isadigit{3}}{\isacharparenright}{\kern0pt}\ electoral{\isacharunderscore}{\kern0pt}module{\isacharunderscore}{\kern0pt}def\isanewline
\ \ \ \ \isacommand{by}\isamarkupfalse%
\ auto\isanewline
\ \ \isacommand{moreover}\isamarkupfalse%
\ \isacommand{have}\isamarkupfalse%
\isanewline
\ \ \ \ {\isachardoublequoteopen}disjoint{\isadigit{3}}\ {\isacharparenleft}{\kern0pt}{\isacharparenleft}{\kern0pt}elect\ m\ A\ p{\isacharparenright}{\kern0pt}{\isacharcomma}{\kern0pt}{\isacharparenleft}{\kern0pt}reject\ m\ A\ p{\isacharparenright}{\kern0pt}{\isacharcomma}{\kern0pt}{\isacharparenleft}{\kern0pt}defer\ m\ A\ p{\isacharparenright}{\kern0pt}{\isacharparenright}{\kern0pt}{\isachardoublequoteclose}\isanewline
\ \ \ \ \isacommand{using}\isamarkupfalse%
\ disj{\isacharunderscore}{\kern0pt}m\ prod{\isachardot}{\kern0pt}collapse\isanewline
\ \ \ \ \isacommand{by}\isamarkupfalse%
\ metis\isanewline
\ \ \isacommand{have}\isamarkupfalse%
\ set{\isacharunderscore}{\kern0pt}partit{\isacharunderscore}{\kern0pt}n{\isacharcolon}{\kern0pt}\isanewline
\ \ \ \ {\isachardoublequoteopen}set{\isacharunderscore}{\kern0pt}equals{\isacharunderscore}{\kern0pt}partition\ A\ {\isacharparenleft}{\kern0pt}{\isacharparenleft}{\kern0pt}elect\ n\ A\ q{\isacharparenright}{\kern0pt}{\isacharcomma}{\kern0pt}\ {\isacharparenleft}{\kern0pt}reject\ n\ A\ q{\isacharparenright}{\kern0pt}{\isacharcomma}{\kern0pt}\ {\isacharparenleft}{\kern0pt}defer\ n\ A\ q{\isacharparenright}{\kern0pt}{\isacharparenright}{\kern0pt}{\isachardoublequoteclose}\isanewline
\ \ \ \ \isacommand{using}\isamarkupfalse%
\ assms{\isacharparenleft}{\kern0pt}{\isadigit{2}}{\isacharparenright}{\kern0pt}\ assms{\isacharparenleft}{\kern0pt}{\isadigit{4}}{\isacharparenright}{\kern0pt}\ electoral{\isacharunderscore}{\kern0pt}module{\isacharunderscore}{\kern0pt}def\isanewline
\ \ \ \ \isacommand{by}\isamarkupfalse%
\ auto\isanewline
\ \ \isacommand{moreover}\isamarkupfalse%
\ \isacommand{have}\isamarkupfalse%
\isanewline
\ \ \ \ {\isachardoublequoteopen}disjoint{\isadigit{3}}\ {\isacharparenleft}{\kern0pt}{\isacharparenleft}{\kern0pt}elect\ n\ A\ q{\isacharparenright}{\kern0pt}{\isacharcomma}{\kern0pt}{\isacharparenleft}{\kern0pt}reject\ n\ A\ q{\isacharparenright}{\kern0pt}{\isacharcomma}{\kern0pt}{\isacharparenleft}{\kern0pt}defer\ n\ A\ q{\isacharparenright}{\kern0pt}{\isacharparenright}{\kern0pt}{\isachardoublequoteclose}\isanewline
\ \ \ \ \isacommand{using}\isamarkupfalse%
\ disj{\isacharunderscore}{\kern0pt}n\ prod{\isachardot}{\kern0pt}collapse\isanewline
\ \ \ \ \isacommand{by}\isamarkupfalse%
\ metis\isanewline
\ \ \isacommand{have}\isamarkupfalse%
\ reject{\isacharunderscore}{\kern0pt}p{\isacharcolon}{\kern0pt}\isanewline
\ \ \ \ {\isachardoublequoteopen}reject\ m\ A\ p\ {\isacharequal}{\kern0pt}\ A\ {\isacharminus}{\kern0pt}\ {\isacharparenleft}{\kern0pt}{\isacharparenleft}{\kern0pt}elect\ m\ A\ p{\isacharparenright}{\kern0pt}\ {\isasymunion}\ {\isacharparenleft}{\kern0pt}defer\ m\ A\ p{\isacharparenright}{\kern0pt}{\isacharparenright}{\kern0pt}{\isachardoublequoteclose}\isanewline
\ \ \ \ \isacommand{using}\isamarkupfalse%
\ assms{\isacharparenleft}{\kern0pt}{\isadigit{1}}{\isacharparenright}{\kern0pt}\ assms{\isacharparenleft}{\kern0pt}{\isadigit{3}}{\isacharparenright}{\kern0pt}\ combine{\isacharunderscore}{\kern0pt}ele{\isacharunderscore}{\kern0pt}rej{\isacharunderscore}{\kern0pt}def\isanewline
\ \ \ \ \ \ \ \ \ \ electoral{\isacharunderscore}{\kern0pt}module{\isacharunderscore}{\kern0pt}def\ result{\isacharunderscore}{\kern0pt}imp{\isacharunderscore}{\kern0pt}rej\isanewline
\ \ \ \ \isacommand{by}\isamarkupfalse%
\ metis\isanewline
\ \ \isacommand{have}\isamarkupfalse%
\ reject{\isacharunderscore}{\kern0pt}q{\isacharcolon}{\kern0pt}\isanewline
\ \ \ \ {\isachardoublequoteopen}reject\ n\ A\ q\ {\isacharequal}{\kern0pt}\ A\ {\isacharminus}{\kern0pt}\ {\isacharparenleft}{\kern0pt}{\isacharparenleft}{\kern0pt}elect\ n\ A\ q{\isacharparenright}{\kern0pt}\ {\isasymunion}\ {\isacharparenleft}{\kern0pt}defer\ n\ A\ q{\isacharparenright}{\kern0pt}{\isacharparenright}{\kern0pt}{\isachardoublequoteclose}\isanewline
\ \ \ \ \isacommand{using}\isamarkupfalse%
\ assms{\isacharparenleft}{\kern0pt}{\isadigit{2}}{\isacharparenright}{\kern0pt}\ assms{\isacharparenleft}{\kern0pt}{\isadigit{4}}{\isacharparenright}{\kern0pt}\ combine{\isacharunderscore}{\kern0pt}ele{\isacharunderscore}{\kern0pt}rej{\isacharunderscore}{\kern0pt}def\isanewline
\ \ \ \ \ \ \ \ \ \ electoral{\isacharunderscore}{\kern0pt}module{\isacharunderscore}{\kern0pt}def\ result{\isacharunderscore}{\kern0pt}imp{\isacharunderscore}{\kern0pt}rej\isanewline
\ \ \ \ \isacommand{by}\isamarkupfalse%
\ metis\isanewline
\ \ \isacommand{from}\isamarkupfalse%
\ reject{\isacharunderscore}{\kern0pt}p\ reject{\isacharunderscore}{\kern0pt}q\ \isacommand{show}\isamarkupfalse%
\ {\isacharquery}{\kern0pt}thesis\isanewline
\ \ \ \ \isacommand{by}\isamarkupfalse%
\ {\isacharparenleft}{\kern0pt}simp\ add{\isacharcolon}{\kern0pt}\ assms{\isacharparenleft}{\kern0pt}{\isadigit{5}}{\isacharparenright}{\kern0pt}\ assms{\isacharparenleft}{\kern0pt}{\isadigit{6}}{\isacharparenright}{\kern0pt}\ prod{\isacharunderscore}{\kern0pt}eqI{\isacharparenright}{\kern0pt}\isanewline
\isacommand{qed}\isamarkupfalse%
%
\endisatagproof
{\isafoldproof}%
%
\isadelimproof
\isanewline
%
\endisadelimproof
%
\isadelimtheory
\isanewline
%
\endisadelimtheory
%
\isatagtheory
\isacommand{end}\isamarkupfalse%
%
\endisatagtheory
{\isafoldtheory}%
%
\isadelimtheory
%
\endisadelimtheory
%
\end{isabellebody}%
\endinput
%:%file=~/Documents/Studies/VotingRuleGenerator/virage/src/test/resources/verifiedVotingRuleConstruction/theories/Compositional_Framework/Components/Electoral_Module.thy%:%
%:%6=3%:%
%:%11=4%:%
%:%12=5%:%
%:%13=6%:%
%:%15=9%:%
%:%19=11%:%
%:%35=13%:%
%:%36=13%:%
%:%37=14%:%
%:%38=15%:%
%:%39=16%:%
%:%40=17%:%
%:%41=18%:%
%:%50=21%:%
%:%51=22%:%
%:%52=23%:%
%:%53=24%:%
%:%54=25%:%
%:%55=26%:%
%:%56=27%:%
%:%57=28%:%
%:%58=29%:%
%:%59=30%:%
%:%60=31%:%
%:%61=32%:%
%:%62=33%:%
%:%63=34%:%
%:%64=35%:%
%:%73=37%:%
%:%83=40%:%
%:%84=40%:%
%:%91=42%:%
%:%101=52%:%
%:%102=52%:%
%:%103=53%:%
%:%104=54%:%
%:%105=55%:%
%:%106=55%:%
%:%107=56%:%
%:%108=57%:%
%:%111=58%:%
%:%115=58%:%
%:%116=58%:%
%:%117=59%:%
%:%118=59%:%
%:%123=59%:%
%:%126=60%:%
%:%127=64%:%
%:%128=65%:%
%:%129=65%:%
%:%130=66%:%
%:%131=67%:%
%:%132=68%:%
%:%133=69%:%
%:%134=69%:%
%:%135=70%:%
%:%136=71%:%
%:%137=72%:%
%:%138=73%:%
%:%139=73%:%
%:%140=74%:%
%:%141=75%:%
%:%148=97%:%
%:%158=99%:%
%:%159=99%:%
%:%160=100%:%
%:%161=101%:%
%:%165=105%:%
%:%166=106%:%
%:%167=107%:%
%:%168=107%:%
%:%169=108%:%
%:%170=109%:%
%:%173=112%:%
%:%174=113%:%
%:%175=114%:%
%:%176=114%:%
%:%177=115%:%
%:%178=116%:%
%:%181=119%:%
%:%182=120%:%
%:%183=121%:%
%:%184=121%:%
%:%185=122%:%
%:%186=123%:%
%:%197=129%:%
%:%207=131%:%
%:%208=131%:%
%:%209=132%:%
%:%210=133%:%
%:%211=134%:%
%:%212=135%:%
%:%213=136%:%
%:%216=137%:%
%:%220=137%:%
%:%221=137%:%
%:%222=138%:%
%:%223=138%:%
%:%228=138%:%
%:%231=139%:%
%:%232=140%:%
%:%233=140%:%
%:%234=141%:%
%:%235=142%:%
%:%236=143%:%
%:%237=144%:%
%:%240=145%:%
%:%244=145%:%
%:%245=145%:%
%:%246=146%:%
%:%247=146%:%
%:%252=146%:%
%:%255=147%:%
%:%256=148%:%
%:%257=148%:%
%:%258=149%:%
%:%259=150%:%
%:%260=151%:%
%:%261=152%:%
%:%268=153%:%
%:%269=153%:%
%:%270=154%:%
%:%271=154%:%
%:%272=155%:%
%:%273=156%:%
%:%274=156%:%
%:%275=157%:%
%:%276=158%:%
%:%277=158%:%
%:%278=159%:%
%:%280=161%:%
%:%281=162%:%
%:%282=162%:%
%:%283=163%:%
%:%284=163%:%
%:%285=163%:%
%:%286=164%:%
%:%287=165%:%
%:%288=165%:%
%:%289=166%:%
%:%290=166%:%
%:%291=167%:%
%:%292=167%:%
%:%293=168%:%
%:%294=168%:%
%:%295=169%:%
%:%296=169%:%
%:%297=170%:%
%:%298=170%:%
%:%299=171%:%
%:%300=171%:%
%:%301=172%:%
%:%302=173%:%
%:%303=173%:%
%:%304=174%:%
%:%305=175%:%
%:%306=175%:%
%:%307=176%:%
%:%309=178%:%
%:%310=179%:%
%:%311=179%:%
%:%312=180%:%
%:%313=180%:%
%:%314=180%:%
%:%315=181%:%
%:%316=182%:%
%:%317=182%:%
%:%318=183%:%
%:%319=183%:%
%:%320=184%:%
%:%321=184%:%
%:%322=185%:%
%:%323=185%:%
%:%324=186%:%
%:%325=187%:%
%:%326=187%:%
%:%327=188%:%
%:%328=188%:%
%:%329=189%:%
%:%330=189%:%
%:%331=190%:%
%:%332=191%:%
%:%333=191%:%
%:%334=192%:%
%:%335=193%:%
%:%336=193%:%
%:%337=194%:%
%:%339=196%:%
%:%340=197%:%
%:%341=197%:%
%:%342=198%:%
%:%343=198%:%
%:%344=198%:%
%:%345=199%:%
%:%346=200%:%
%:%347=200%:%
%:%348=201%:%
%:%349=201%:%
%:%350=202%:%
%:%351=202%:%
%:%352=203%:%
%:%353=203%:%
%:%354=204%:%
%:%355=204%:%
%:%356=205%:%
%:%357=205%:%
%:%358=206%:%
%:%359=206%:%
%:%360=207%:%
%:%361=208%:%
%:%362=208%:%
%:%363=209%:%
%:%364=210%:%
%:%365=211%:%
%:%366=212%:%
%:%367=212%:%
%:%368=213%:%
%:%370=215%:%
%:%371=216%:%
%:%372=216%:%
%:%373=217%:%
%:%374=217%:%
%:%375=217%:%
%:%376=218%:%
%:%377=219%:%
%:%378=219%:%
%:%379=220%:%
%:%380=220%:%
%:%381=221%:%
%:%382=221%:%
%:%383=222%:%
%:%384=222%:%
%:%385=223%:%
%:%386=224%:%
%:%387=224%:%
%:%388=225%:%
%:%394=225%:%
%:%397=226%:%
%:%398=227%:%
%:%399=227%:%
%:%400=228%:%
%:%401=229%:%
%:%402=230%:%
%:%403=231%:%
%:%404=232%:%
%:%406=234%:%
%:%413=235%:%
%:%414=235%:%
%:%415=236%:%
%:%416=236%:%
%:%417=237%:%
%:%418=238%:%
%:%419=238%:%
%:%420=239%:%
%:%421=240%:%
%:%422=241%:%
%:%423=241%:%
%:%424=242%:%
%:%426=244%:%
%:%427=245%:%
%:%428=245%:%
%:%429=246%:%
%:%430=246%:%
%:%431=246%:%
%:%432=247%:%
%:%433=248%:%
%:%434=248%:%
%:%435=249%:%
%:%436=249%:%
%:%437=250%:%
%:%438=250%:%
%:%439=250%:%
%:%440=251%:%
%:%441=252%:%
%:%442=252%:%
%:%443=253%:%
%:%444=253%:%
%:%445=253%:%
%:%446=254%:%
%:%447=255%:%
%:%448=255%:%
%:%449=256%:%
%:%450=256%:%
%:%451=257%:%
%:%452=257%:%
%:%453=258%:%
%:%454=258%:%
%:%455=259%:%
%:%456=260%:%
%:%457=261%:%
%:%458=261%:%
%:%459=262%:%
%:%460=262%:%
%:%461=263%:%
%:%462=263%:%
%:%463=264%:%
%:%464=265%:%
%:%465=265%:%
%:%466=266%:%
%:%467=267%:%
%:%468=268%:%
%:%469=268%:%
%:%470=269%:%
%:%472=271%:%
%:%473=272%:%
%:%474=272%:%
%:%475=273%:%
%:%476=273%:%
%:%477=274%:%
%:%479=276%:%
%:%480=277%:%
%:%481=277%:%
%:%482=278%:%
%:%483=278%:%
%:%484=278%:%
%:%485=279%:%
%:%486=280%:%
%:%487=280%:%
%:%488=281%:%
%:%489=281%:%
%:%490=281%:%
%:%491=282%:%
%:%492=283%:%
%:%493=283%:%
%:%494=284%:%
%:%495=284%:%
%:%496=285%:%
%:%497=285%:%
%:%498=286%:%
%:%499=287%:%
%:%500=287%:%
%:%501=288%:%
%:%502=288%:%
%:%503=289%:%
%:%504=290%:%
%:%505=290%:%
%:%506=291%:%
%:%507=291%:%
%:%508=292%:%
%:%509=293%:%
%:%510=293%:%
%:%511=294%:%
%:%512=294%:%
%:%513=295%:%
%:%514=295%:%
%:%515=295%:%
%:%516=296%:%
%:%517=297%:%
%:%518=298%:%
%:%519=299%:%
%:%520=300%:%
%:%524=304%:%
%:%525=305%:%
%:%526=305%:%
%:%527=306%:%
%:%528=306%:%
%:%529=307%:%
%:%530=307%:%
%:%532=309%:%
%:%533=310%:%
%:%534=310%:%
%:%535=311%:%
%:%536=311%:%
%:%537=312%:%
%:%538=312%:%
%:%539=313%:%
%:%540=313%:%
%:%541=314%:%
%:%542=315%:%
%:%543=315%:%
%:%544=316%:%
%:%545=316%:%
%:%546=317%:%
%:%547=317%:%
%:%548=318%:%
%:%549=319%:%
%:%550=319%:%
%:%551=320%:%
%:%552=321%:%
%:%553=322%:%
%:%554=322%:%
%:%555=323%:%
%:%557=325%:%
%:%558=326%:%
%:%559=326%:%
%:%560=327%:%
%:%561=327%:%
%:%562=327%:%
%:%563=328%:%
%:%564=329%:%
%:%565=329%:%
%:%566=330%:%
%:%567=330%:%
%:%568=331%:%
%:%569=331%:%
%:%570=331%:%
%:%571=332%:%
%:%572=333%:%
%:%573=333%:%
%:%574=334%:%
%:%575=334%:%
%:%576=334%:%
%:%577=335%:%
%:%578=336%:%
%:%579=336%:%
%:%580=337%:%
%:%581=337%:%
%:%582=338%:%
%:%583=338%:%
%:%584=339%:%
%:%585=339%:%
%:%586=340%:%
%:%587=341%:%
%:%588=342%:%
%:%589=342%:%
%:%590=343%:%
%:%596=343%:%
%:%599=344%:%
%:%600=345%:%
%:%601=345%:%
%:%602=346%:%
%:%603=347%:%
%:%604=348%:%
%:%605=349%:%
%:%608=350%:%
%:%612=350%:%
%:%613=350%:%
%:%614=351%:%
%:%615=351%:%
%:%620=351%:%
%:%623=352%:%
%:%624=353%:%
%:%625=353%:%
%:%626=354%:%
%:%627=355%:%
%:%628=356%:%
%:%629=357%:%
%:%632=358%:%
%:%636=358%:%
%:%637=358%:%
%:%638=359%:%
%:%639=359%:%
%:%644=359%:%
%:%647=360%:%
%:%648=361%:%
%:%649=361%:%
%:%650=362%:%
%:%651=363%:%
%:%652=364%:%
%:%653=365%:%
%:%656=366%:%
%:%660=366%:%
%:%661=366%:%
%:%662=367%:%
%:%663=367%:%
%:%668=367%:%
%:%671=368%:%
%:%672=369%:%
%:%673=369%:%
%:%674=370%:%
%:%675=371%:%
%:%676=372%:%
%:%677=373%:%
%:%678=374%:%
%:%681=375%:%
%:%685=375%:%
%:%686=375%:%
%:%687=376%:%
%:%688=377%:%
%:%689=377%:%
%:%694=377%:%
%:%697=378%:%
%:%698=382%:%
%:%699=383%:%
%:%700=383%:%
%:%701=384%:%
%:%702=385%:%
%:%703=386%:%
%:%704=387%:%
%:%705=388%:%
%:%707=390%:%
%:%710=391%:%
%:%714=391%:%
%:%715=391%:%
%:%716=392%:%
%:%721=392%:%
%:%724=393%:%
%:%725=394%:%
%:%726=394%:%
%:%727=395%:%
%:%728=396%:%
%:%729=397%:%
%:%730=398%:%
%:%737=399%:%
%:%738=399%:%
%:%739=400%:%
%:%740=400%:%
%:%741=400%:%
%:%742=401%:%
%:%743=401%:%
%:%744=402%:%
%:%745=402%:%
%:%746=403%:%
%:%747=403%:%
%:%748=404%:%
%:%749=404%:%
%:%750=405%:%
%:%751=405%:%
%:%752=406%:%
%:%753=406%:%
%:%754=406%:%
%:%755=406%:%
%:%756=407%:%
%:%757=408%:%
%:%758=409%:%
%:%759=409%:%
%:%760=410%:%
%:%761=410%:%
%:%762=411%:%
%:%763=411%:%
%:%764=411%:%
%:%765=412%:%
%:%766=412%:%
%:%767=413%:%
%:%773=413%:%
%:%776=414%:%
%:%777=415%:%
%:%778=415%:%
%:%779=416%:%
%:%780=417%:%
%:%781=418%:%
%:%782=419%:%
%:%789=420%:%
%:%790=420%:%
%:%791=421%:%
%:%792=421%:%
%:%793=421%:%
%:%794=422%:%
%:%795=422%:%
%:%796=423%:%
%:%797=423%:%
%:%798=424%:%
%:%799=425%:%
%:%800=425%:%
%:%801=426%:%
%:%802=426%:%
%:%803=427%:%
%:%804=427%:%
%:%805=428%:%
%:%806=428%:%
%:%807=429%:%
%:%808=429%:%
%:%809=430%:%
%:%810=430%:%
%:%811=430%:%
%:%812=430%:%
%:%813=431%:%
%:%814=432%:%
%:%815=433%:%
%:%816=433%:%
%:%817=434%:%
%:%818=434%:%
%:%819=435%:%
%:%820=435%:%
%:%821=435%:%
%:%822=436%:%
%:%823=436%:%
%:%824=437%:%
%:%830=437%:%
%:%833=438%:%
%:%834=439%:%
%:%835=439%:%
%:%836=440%:%
%:%837=441%:%
%:%838=442%:%
%:%839=443%:%
%:%846=444%:%
%:%847=444%:%
%:%848=445%:%
%:%849=445%:%
%:%850=445%:%
%:%851=446%:%
%:%852=446%:%
%:%853=447%:%
%:%854=447%:%
%:%855=448%:%
%:%856=448%:%
%:%857=449%:%
%:%858=449%:%
%:%859=450%:%
%:%860=450%:%
%:%861=450%:%
%:%862=450%:%
%:%863=451%:%
%:%864=452%:%
%:%865=453%:%
%:%866=453%:%
%:%867=454%:%
%:%868=454%:%
%:%869=455%:%
%:%870=455%:%
%:%871=455%:%
%:%872=456%:%
%:%873=456%:%
%:%874=457%:%
%:%880=457%:%
%:%883=458%:%
%:%884=459%:%
%:%885=459%:%
%:%886=460%:%
%:%887=461%:%
%:%888=462%:%
%:%889=463%:%
%:%890=464%:%
%:%891=465%:%
%:%892=466%:%
%:%895=467%:%
%:%899=467%:%
%:%900=467%:%
%:%901=468%:%
%:%902=469%:%
%:%903=470%:%
%:%904=470%:%
%:%909=470%:%
%:%912=471%:%
%:%913=472%:%
%:%914=472%:%
%:%915=473%:%
%:%916=474%:%
%:%919=475%:%
%:%923=475%:%
%:%924=475%:%
%:%925=476%:%
%:%926=477%:%
%:%927=477%:%
%:%932=477%:%
%:%935=478%:%
%:%936=479%:%
%:%937=479%:%
%:%938=480%:%
%:%939=481%:%
%:%940=482%:%
%:%941=483%:%
%:%942=484%:%
%:%943=485%:%
%:%946=486%:%
%:%950=486%:%
%:%951=486%:%
%:%952=487%:%
%:%953=488%:%
%:%954=488%:%
%:%959=488%:%
%:%962=489%:%
%:%963=490%:%
%:%964=490%:%
%:%965=491%:%
%:%966=492%:%
%:%967=493%:%
%:%968=494%:%
%:%969=495%:%
%:%976=496%:%
%:%977=496%:%
%:%978=497%:%
%:%979=497%:%
%:%980=498%:%
%:%981=498%:%
%:%982=499%:%
%:%983=499%:%
%:%984=500%:%
%:%985=500%:%
%:%986=500%:%
%:%987=501%:%
%:%988=501%:%
%:%989=502%:%
%:%990=503%:%
%:%991=504%:%
%:%992=504%:%
%:%993=505%:%
%:%994=505%:%
%:%995=505%:%
%:%996=506%:%
%:%997=506%:%
%:%998=507%:%
%:%999=507%:%
%:%1000=508%:%
%:%1001=508%:%
%:%1002=508%:%
%:%1003=509%:%
%:%1004=509%:%
%:%1005=510%:%
%:%1006=511%:%
%:%1007=512%:%
%:%1008=512%:%
%:%1009=513%:%
%:%1010=513%:%
%:%1011=513%:%
%:%1012=514%:%
%:%1013=514%:%
%:%1014=515%:%
%:%1015=515%:%
%:%1016=516%:%
%:%1017=516%:%
%:%1018=516%:%
%:%1019=517%:%
%:%1020=517%:%
%:%1021=518%:%
%:%1022=519%:%
%:%1023=520%:%
%:%1024=520%:%
%:%1025=521%:%
%:%1026=521%:%
%:%1027=521%:%
%:%1028=522%:%
%:%1029=522%:%
%:%1030=523%:%
%:%1031=523%:%
%:%1032=524%:%
%:%1038=524%:%
%:%1041=525%:%
%:%1042=526%:%
%:%1043=526%:%
%:%1044=527%:%
%:%1045=528%:%
%:%1046=529%:%
%:%1047=530%:%
%:%1048=531%:%
%:%1049=532%:%
%:%1050=533%:%
%:%1051=534%:%
%:%1058=535%:%
%:%1059=535%:%
%:%1060=536%:%
%:%1061=536%:%
%:%1062=537%:%
%:%1063=538%:%
%:%1064=538%:%
%:%1065=539%:%
%:%1066=539%:%
%:%1067=540%:%
%:%1068=540%:%
%:%1069=541%:%
%:%1070=542%:%
%:%1071=542%:%
%:%1072=543%:%
%:%1073=543%:%
%:%1074=544%:%
%:%1075=544%:%
%:%1076=545%:%
%:%1077=546%:%
%:%1078=546%:%
%:%1079=547%:%
%:%1080=547%:%
%:%1081=548%:%
%:%1082=548%:%
%:%1083=548%:%
%:%1084=549%:%
%:%1085=550%:%
%:%1086=550%:%
%:%1087=551%:%
%:%1088=551%:%
%:%1089=552%:%
%:%1090=552%:%
%:%1091=553%:%
%:%1092=554%:%
%:%1093=554%:%
%:%1094=555%:%
%:%1095=555%:%
%:%1096=556%:%
%:%1097=556%:%
%:%1098=556%:%
%:%1099=557%:%
%:%1100=558%:%
%:%1101=558%:%
%:%1102=559%:%
%:%1103=559%:%
%:%1104=560%:%
%:%1105=560%:%
%:%1106=561%:%
%:%1107=562%:%
%:%1108=562%:%
%:%1109=563%:%
%:%1110=564%:%
%:%1111=564%:%
%:%1112=565%:%
%:%1113=565%:%
%:%1114=566%:%
%:%1115=567%:%
%:%1116=567%:%
%:%1117=568%:%
%:%1118=569%:%
%:%1119=569%:%
%:%1120=570%:%
%:%1121=570%:%
%:%1122=570%:%
%:%1123=571%:%
%:%1124=571%:%
%:%1125=572%:%
%:%1131=572%:%
%:%1136=573%:%
%:%1141=574%:%
%
\begin{isabellebody}%
\setisabellecontext{Evaluation{\isacharunderscore}{\kern0pt}Function}%
%
\isadelimdocument
\isanewline
%
\endisadelimdocument
%
\isatagdocument
\isanewline
%
\isamarkupsection{Evaluation Function%
}
\isamarkuptrue%
%
\endisatagdocument
{\isafolddocument}%
%
\isadelimdocument
%
\endisadelimdocument
%
\isadelimtheory
%
\endisadelimtheory
%
\isatagtheory
\isacommand{theory}\isamarkupfalse%
\ Evaluation{\isacharunderscore}{\kern0pt}Function\isanewline
\ \ \isakeyword{imports}\ {\isachardoublequoteopen}{\isachardot}{\kern0pt}{\isachardot}{\kern0pt}{\isacharslash}{\kern0pt}{\isachardot}{\kern0pt}{\isachardot}{\kern0pt}{\isacharslash}{\kern0pt}Social{\isacharunderscore}{\kern0pt}Choice{\isacharunderscore}{\kern0pt}Types{\isacharslash}{\kern0pt}Profile{\isachardoublequoteclose}\isanewline
\isakeyword{begin}%
\endisatagtheory
{\isafoldtheory}%
%
\isadelimtheory
%
\endisadelimtheory
%
\begin{isamarkuptext}%
This is the evaluation function. From a set of currently eligible alternatives,
the evaluation function computes a numerical value that is then to be used for
further (s)election, e.g., by the elimination module.%
\end{isamarkuptext}\isamarkuptrue%
%
\isadelimdocument
%
\endisadelimdocument
%
\isatagdocument
%
\isamarkupsubsection{Definition%
}
\isamarkuptrue%
%
\endisatagdocument
{\isafolddocument}%
%
\isadelimdocument
%
\endisadelimdocument
\isacommand{type{\isacharunderscore}{\kern0pt}synonym}\isamarkupfalse%
\ {\isacharprime}{\kern0pt}a\ Evaluation{\isacharunderscore}{\kern0pt}Function\ {\isacharequal}{\kern0pt}\ {\isachardoublequoteopen}{\isacharprime}{\kern0pt}a\ \ {\isasymRightarrow}\ {\isacharprime}{\kern0pt}a\ set\ {\isasymRightarrow}\ {\isacharprime}{\kern0pt}a\ Profile\ {\isasymRightarrow}\ nat{\isachardoublequoteclose}\isanewline
%
\isadelimtheory
\isanewline
%
\endisadelimtheory
%
\isatagtheory
\isacommand{end}\isamarkupfalse%
%
\endisatagtheory
{\isafoldtheory}%
%
\isadelimtheory
%
\endisadelimtheory
%
\end{isabellebody}%
\endinput
%:%file=~/Documents/Studies/VotingRuleGenerator/virage/src/test/resources/verifiedVotingRuleConstruction/theories/Compositional_Framework/Components/Evaluation_Function.thy%:%
%:%6=3%:%
%:%11=4%:%
%:%13=7%:%
%:%29=9%:%
%:%30=9%:%
%:%31=10%:%
%:%32=11%:%
%:%41=14%:%
%:%42=15%:%
%:%43=16%:%
%:%52=18%:%
%:%62=20%:%
%:%63=20%:%
%:%66=21%:%
%:%71=22%:%
%
\begin{isabellebody}%
\setisabellecontext{Aggregator}%
%
\isadelimdocument
\isanewline
%
\endisadelimdocument
%
\isatagdocument
\isanewline
\isanewline
\isanewline
%
\isamarkupsection{Aggregator%
}
\isamarkuptrue%
%
\endisatagdocument
{\isafolddocument}%
%
\isadelimdocument
%
\endisadelimdocument
%
\isadelimtheory
%
\endisadelimtheory
%
\isatagtheory
\isacommand{theory}\isamarkupfalse%
\ Aggregator\isanewline
\ \ \isakeyword{imports}\ {\isachardoublequoteopen}Social{\isacharunderscore}{\kern0pt}Choice{\isacharunderscore}{\kern0pt}Types{\isacharslash}{\kern0pt}Result{\isachardoublequoteclose}\isanewline
\isakeyword{begin}%
\endisatagtheory
{\isafoldtheory}%
%
\isadelimtheory
%
\endisadelimtheory
%
\begin{isamarkuptext}%
An aggregator gets two partitions (results of electoral modules) as input and
output another partition. They are used to aggregate results of parallel
composed electoral modules.
They are commutative, i.e., the order of the aggregated modules does not affect
the resulting aggregation. Moreover, they are conservative in the sense that
the resulting decisions are subsets of the two given partitions' decisions.%
\end{isamarkuptext}\isamarkuptrue%
%
\isadelimdocument
%
\endisadelimdocument
%
\isatagdocument
%
\isamarkupsubsection{Definition%
}
\isamarkuptrue%
%
\endisatagdocument
{\isafolddocument}%
%
\isadelimdocument
%
\endisadelimdocument
\isacommand{type{\isacharunderscore}{\kern0pt}synonym}\isamarkupfalse%
\ {\isacharprime}{\kern0pt}a\ Aggregator\ {\isacharequal}{\kern0pt}\ {\isachardoublequoteopen}{\isacharprime}{\kern0pt}a\ set\ {\isasymRightarrow}\ {\isacharprime}{\kern0pt}a\ Result\ {\isasymRightarrow}\ {\isacharprime}{\kern0pt}a\ Result\ {\isasymRightarrow}\ {\isacharprime}{\kern0pt}a\ Result{\isachardoublequoteclose}\isanewline
\isanewline
\isacommand{definition}\isamarkupfalse%
\ aggregator\ {\isacharcolon}{\kern0pt}{\isacharcolon}{\kern0pt}\ {\isachardoublequoteopen}{\isacharprime}{\kern0pt}a\ Aggregator\ {\isasymRightarrow}\ bool{\isachardoublequoteclose}\ \isakeyword{where}\isanewline
\ \ {\isachardoublequoteopen}aggregator\ agg\ {\isasymequiv}\isanewline
\ \ \ \ {\isasymforall}A\ e{\isadigit{1}}\ e{\isadigit{2}}\ d{\isadigit{1}}\ d{\isadigit{2}}\ r{\isadigit{1}}\ r{\isadigit{2}}{\isachardot}{\kern0pt}\isanewline
\ \ \ \ \ \ {\isacharparenleft}{\kern0pt}well{\isacharunderscore}{\kern0pt}formed\ A\ {\isacharparenleft}{\kern0pt}e{\isadigit{1}}{\isacharcomma}{\kern0pt}\ r{\isadigit{1}}{\isacharcomma}{\kern0pt}\ d{\isadigit{1}}{\isacharparenright}{\kern0pt}\ {\isasymand}\ well{\isacharunderscore}{\kern0pt}formed\ A\ {\isacharparenleft}{\kern0pt}e{\isadigit{2}}{\isacharcomma}{\kern0pt}\ r{\isadigit{2}}{\isacharcomma}{\kern0pt}\ d{\isadigit{2}}{\isacharparenright}{\kern0pt}{\isacharparenright}{\kern0pt}\ {\isasymlongrightarrow}\isanewline
\ \ \ \ \ \ well{\isacharunderscore}{\kern0pt}formed\ A\ {\isacharparenleft}{\kern0pt}agg\ A\ {\isacharparenleft}{\kern0pt}e{\isadigit{1}}{\isacharcomma}{\kern0pt}\ r{\isadigit{1}}{\isacharcomma}{\kern0pt}\ d{\isadigit{1}}{\isacharparenright}{\kern0pt}\ {\isacharparenleft}{\kern0pt}e{\isadigit{2}}{\isacharcomma}{\kern0pt}\ r{\isadigit{2}}{\isacharcomma}{\kern0pt}\ d{\isadigit{2}}{\isacharparenright}{\kern0pt}{\isacharparenright}{\kern0pt}{\isachardoublequoteclose}%
\isadelimdocument
%
\endisadelimdocument
%
\isatagdocument
%
\isamarkupsubsection{Properties%
}
\isamarkuptrue%
%
\endisatagdocument
{\isafolddocument}%
%
\isadelimdocument
%
\endisadelimdocument
\isacommand{definition}\isamarkupfalse%
\ agg{\isacharunderscore}{\kern0pt}commutative\ {\isacharcolon}{\kern0pt}{\isacharcolon}{\kern0pt}\ {\isachardoublequoteopen}{\isacharprime}{\kern0pt}a\ Aggregator\ {\isasymRightarrow}\ bool{\isachardoublequoteclose}\ \isakeyword{where}\isanewline
\ \ {\isachardoublequoteopen}agg{\isacharunderscore}{\kern0pt}commutative\ agg\ {\isasymequiv}\isanewline
\ \ \ \ aggregator\ agg\ {\isasymand}\ {\isacharparenleft}{\kern0pt}{\isasymforall}A\ e{\isadigit{1}}\ e{\isadigit{2}}\ d{\isadigit{1}}\ d{\isadigit{2}}\ r{\isadigit{1}}\ r{\isadigit{2}}{\isachardot}{\kern0pt}\isanewline
\ \ \ \ \ \ agg\ A\ {\isacharparenleft}{\kern0pt}e{\isadigit{1}}{\isacharcomma}{\kern0pt}\ r{\isadigit{1}}{\isacharcomma}{\kern0pt}\ d{\isadigit{1}}{\isacharparenright}{\kern0pt}\ {\isacharparenleft}{\kern0pt}e{\isadigit{2}}{\isacharcomma}{\kern0pt}\ r{\isadigit{2}}{\isacharcomma}{\kern0pt}\ d{\isadigit{2}}{\isacharparenright}{\kern0pt}\ {\isacharequal}{\kern0pt}\ agg\ A\ {\isacharparenleft}{\kern0pt}e{\isadigit{2}}{\isacharcomma}{\kern0pt}\ r{\isadigit{2}}{\isacharcomma}{\kern0pt}\ d{\isadigit{2}}{\isacharparenright}{\kern0pt}\ {\isacharparenleft}{\kern0pt}e{\isadigit{1}}{\isacharcomma}{\kern0pt}\ r{\isadigit{1}}{\isacharcomma}{\kern0pt}\ d{\isadigit{1}}{\isacharparenright}{\kern0pt}{\isacharparenright}{\kern0pt}{\isachardoublequoteclose}\isanewline
\isanewline
\isacommand{definition}\isamarkupfalse%
\ agg{\isacharunderscore}{\kern0pt}conservative\ {\isacharcolon}{\kern0pt}{\isacharcolon}{\kern0pt}\ {\isachardoublequoteopen}{\isacharprime}{\kern0pt}a\ Aggregator\ {\isasymRightarrow}\ bool{\isachardoublequoteclose}\ \isakeyword{where}\isanewline
\ \ {\isachardoublequoteopen}agg{\isacharunderscore}{\kern0pt}conservative\ agg\ {\isasymequiv}\isanewline
\ \ \ \ aggregator\ agg\ {\isasymand}\isanewline
\ \ \ \ {\isacharparenleft}{\kern0pt}{\isasymforall}A\ e{\isadigit{1}}\ e{\isadigit{2}}\ d{\isadigit{1}}\ d{\isadigit{2}}\ r{\isadigit{1}}\ r{\isadigit{2}}{\isachardot}{\kern0pt}\isanewline
\ \ \ \ \ \ {\isacharparenleft}{\kern0pt}{\isacharparenleft}{\kern0pt}well{\isacharunderscore}{\kern0pt}formed\ A\ {\isacharparenleft}{\kern0pt}e{\isadigit{1}}{\isacharcomma}{\kern0pt}\ r{\isadigit{1}}{\isacharcomma}{\kern0pt}\ d{\isadigit{1}}{\isacharparenright}{\kern0pt}\ {\isasymand}\ well{\isacharunderscore}{\kern0pt}formed\ A\ {\isacharparenleft}{\kern0pt}e{\isadigit{2}}{\isacharcomma}{\kern0pt}\ r{\isadigit{2}}{\isacharcomma}{\kern0pt}\ d{\isadigit{2}}{\isacharparenright}{\kern0pt}{\isacharparenright}{\kern0pt}\ {\isasymlongrightarrow}\isanewline
\ \ \ \ \ \ \ \ elect{\isacharunderscore}{\kern0pt}r\ {\isacharparenleft}{\kern0pt}agg\ A\ {\isacharparenleft}{\kern0pt}e{\isadigit{1}}{\isacharcomma}{\kern0pt}\ r{\isadigit{1}}{\isacharcomma}{\kern0pt}\ d{\isadigit{1}}{\isacharparenright}{\kern0pt}\ {\isacharparenleft}{\kern0pt}e{\isadigit{2}}{\isacharcomma}{\kern0pt}\ r{\isadigit{2}}{\isacharcomma}{\kern0pt}\ d{\isadigit{2}}{\isacharparenright}{\kern0pt}{\isacharparenright}{\kern0pt}\ {\isasymsubseteq}\ {\isacharparenleft}{\kern0pt}e{\isadigit{1}}\ {\isasymunion}\ e{\isadigit{2}}{\isacharparenright}{\kern0pt}\ {\isasymand}\isanewline
\ \ \ \ \ \ \ \ reject{\isacharunderscore}{\kern0pt}r\ {\isacharparenleft}{\kern0pt}agg\ A\ {\isacharparenleft}{\kern0pt}e{\isadigit{1}}{\isacharcomma}{\kern0pt}\ r{\isadigit{1}}{\isacharcomma}{\kern0pt}\ d{\isadigit{1}}{\isacharparenright}{\kern0pt}\ {\isacharparenleft}{\kern0pt}e{\isadigit{2}}{\isacharcomma}{\kern0pt}\ r{\isadigit{2}}{\isacharcomma}{\kern0pt}\ d{\isadigit{2}}{\isacharparenright}{\kern0pt}{\isacharparenright}{\kern0pt}\ {\isasymsubseteq}\ {\isacharparenleft}{\kern0pt}r{\isadigit{1}}\ {\isasymunion}\ r{\isadigit{2}}{\isacharparenright}{\kern0pt}\ {\isasymand}\isanewline
\ \ \ \ \ \ \ \ defer{\isacharunderscore}{\kern0pt}r\ {\isacharparenleft}{\kern0pt}agg\ A\ {\isacharparenleft}{\kern0pt}e{\isadigit{1}}{\isacharcomma}{\kern0pt}\ r{\isadigit{1}}{\isacharcomma}{\kern0pt}\ d{\isadigit{1}}{\isacharparenright}{\kern0pt}\ {\isacharparenleft}{\kern0pt}e{\isadigit{2}}{\isacharcomma}{\kern0pt}\ r{\isadigit{2}}{\isacharcomma}{\kern0pt}\ d{\isadigit{2}}{\isacharparenright}{\kern0pt}{\isacharparenright}{\kern0pt}\ {\isasymsubseteq}\ {\isacharparenleft}{\kern0pt}d{\isadigit{1}}\ {\isasymunion}\ d{\isadigit{2}}{\isacharparenright}{\kern0pt}{\isacharparenright}{\kern0pt}{\isacharparenright}{\kern0pt}{\isachardoublequoteclose}\isanewline
%
\isadelimtheory
\isanewline
%
\endisadelimtheory
%
\isatagtheory
\isacommand{end}\isamarkupfalse%
%
\endisatagtheory
{\isafoldtheory}%
%
\isadelimtheory
%
\endisadelimtheory
%
\end{isabellebody}%
\endinput
%:%file=~/Documents/Studies/VotingRuleGenerator/virage/src/test/resources/old_theories/Compositional_Structures/Basic_Modules/Component_Types/Aggregator.thy%:%
%:%6=3%:%
%:%11=4%:%
%:%12=5%:%
%:%13=6%:%
%:%15=9%:%
%:%31=11%:%
%:%32=11%:%
%:%33=12%:%
%:%34=13%:%
%:%43=16%:%
%:%44=17%:%
%:%45=18%:%
%:%46=19%:%
%:%47=20%:%
%:%48=21%:%
%:%57=23%:%
%:%67=25%:%
%:%68=25%:%
%:%69=26%:%
%:%70=27%:%
%:%71=27%:%
%:%72=28%:%
%:%82=33%:%
%:%92=35%:%
%:%93=35%:%
%:%94=36%:%
%:%96=38%:%
%:%97=39%:%
%:%98=40%:%
%:%99=40%:%
%:%100=41%:%
%:%106=47%:%
%:%109=48%:%
%:%114=49%:%
%
\begin{isabellebody}%
\setisabellecontext{Termination{\isacharunderscore}{\kern0pt}Condition}%
%
\isadelimdocument
\isanewline
%
\endisadelimdocument
%
\isatagdocument
\isanewline
\isanewline
%
\isamarkupsection{Termination Condition%
}
\isamarkuptrue%
%
\endisatagdocument
{\isafolddocument}%
%
\isadelimdocument
%
\endisadelimdocument
%
\isadelimtheory
%
\endisadelimtheory
%
\isatagtheory
\isacommand{theory}\isamarkupfalse%
\ Termination{\isacharunderscore}{\kern0pt}Condition\isanewline
\ \ \isakeyword{imports}\ {\isachardoublequoteopen}{\isachardot}{\kern0pt}{\isachardot}{\kern0pt}{\isacharslash}{\kern0pt}{\isachardot}{\kern0pt}{\isachardot}{\kern0pt}{\isacharslash}{\kern0pt}Social{\isacharunderscore}{\kern0pt}Choice{\isacharunderscore}{\kern0pt}Types{\isacharslash}{\kern0pt}Result{\isachardoublequoteclose}\isanewline
\isakeyword{begin}%
\endisatagtheory
{\isafoldtheory}%
%
\isadelimtheory
%
\endisadelimtheory
%
\begin{isamarkuptext}%
The termination condition is used in loops. It decides whether or not to
terminate the loop after each iteration, depending on the current state
of the loop.%
\end{isamarkuptext}\isamarkuptrue%
%
\isadelimdocument
%
\endisadelimdocument
%
\isatagdocument
%
\isamarkupsubsection{Definition%
}
\isamarkuptrue%
%
\endisatagdocument
{\isafolddocument}%
%
\isadelimdocument
%
\endisadelimdocument
\isacommand{type{\isacharunderscore}{\kern0pt}synonym}\isamarkupfalse%
\ {\isacharprime}{\kern0pt}a\ Termination{\isacharunderscore}{\kern0pt}Condition\ {\isacharequal}{\kern0pt}\ {\isachardoublequoteopen}{\isacharprime}{\kern0pt}a\ Result\ {\isasymRightarrow}\ bool{\isachardoublequoteclose}\isanewline
%
\isadelimtheory
\isanewline
%
\endisadelimtheory
%
\isatagtheory
\isacommand{end}\isamarkupfalse%
%
\endisatagtheory
{\isafoldtheory}%
%
\isadelimtheory
%
\endisadelimtheory
%
\end{isabellebody}%
\endinput
%:%file=~/Documents/Studies/VotingRuleGenerator/virage/src/test/resources/verifiedVotingRuleConstruction/theories/Compositional_Framework/Components/Termination_Condition.thy%:%
%:%6=3%:%
%:%11=4%:%
%:%12=5%:%
%:%14=8%:%
%:%30=10%:%
%:%31=10%:%
%:%32=11%:%
%:%33=12%:%
%:%42=15%:%
%:%43=16%:%
%:%44=17%:%
%:%53=19%:%
%:%63=21%:%
%:%64=21%:%
%:%67=22%:%
%:%72=23%:%
%
\begin{isabellebody}%
\setisabellecontext{Defer{\isacharunderscore}{\kern0pt}Equal{\isacharunderscore}{\kern0pt}Condition}%
%
\isadelimdocument
\isanewline
%
\endisadelimdocument
%
\isatagdocument
\isanewline
\isanewline
%
\isamarkupsection{Defer Equal Condition%
}
\isamarkuptrue%
%
\endisatagdocument
{\isafolddocument}%
%
\isadelimdocument
%
\endisadelimdocument
%
\isadelimtheory
%
\endisadelimtheory
%
\isatagtheory
\isacommand{theory}\isamarkupfalse%
\ Defer{\isacharunderscore}{\kern0pt}Equal{\isacharunderscore}{\kern0pt}Condition\isanewline
\ \ \isakeyword{imports}\ {\isachardoublequoteopen}{\isachardot}{\kern0pt}{\isachardot}{\kern0pt}{\isacharslash}{\kern0pt}Termination{\isacharunderscore}{\kern0pt}Condition{\isachardoublequoteclose}\isanewline
\isakeyword{begin}%
\endisatagtheory
{\isafoldtheory}%
%
\isadelimtheory
%
\endisadelimtheory
%
\begin{isamarkuptext}%
This is a family of termination conditions. For a natural number n,
the according defer-equal condition is true if and only if the given
result's defer-set contains exactly n elements.%
\end{isamarkuptext}\isamarkuptrue%
%
\isadelimdocument
%
\endisadelimdocument
%
\isatagdocument
%
\isamarkupsubsection{Definition%
}
\isamarkuptrue%
%
\endisatagdocument
{\isafolddocument}%
%
\isadelimdocument
%
\endisadelimdocument
\isacommand{fun}\isamarkupfalse%
\ defer{\isacharunderscore}{\kern0pt}equal{\isacharunderscore}{\kern0pt}condition\ {\isacharcolon}{\kern0pt}{\isacharcolon}{\kern0pt}\ {\isachardoublequoteopen}nat\ {\isasymRightarrow}\ {\isacharprime}{\kern0pt}a\ Termination{\isacharunderscore}{\kern0pt}Condition{\isachardoublequoteclose}\ \isakeyword{where}\isanewline
\ \ {\isachardoublequoteopen}defer{\isacharunderscore}{\kern0pt}equal{\isacharunderscore}{\kern0pt}condition\ n\ result\ {\isacharequal}{\kern0pt}\ {\isacharparenleft}{\kern0pt}let\ {\isacharparenleft}{\kern0pt}e{\isacharcomma}{\kern0pt}\ r{\isacharcomma}{\kern0pt}\ d{\isacharparenright}{\kern0pt}\ {\isacharequal}{\kern0pt}\ result\ in\ card\ d\ {\isacharequal}{\kern0pt}\ n{\isacharparenright}{\kern0pt}{\isachardoublequoteclose}\isanewline
%
\isadelimtheory
\isanewline
%
\endisadelimtheory
%
\isatagtheory
\isacommand{end}\isamarkupfalse%
%
\endisatagtheory
{\isafoldtheory}%
%
\isadelimtheory
%
\endisadelimtheory
%
\end{isabellebody}%
\endinput
%:%file=~/Documents/Studies/VotingRuleGenerator/virage/src/test/resources/verifiedVotingRuleConstruction/theories/Compositional_Framework/Components/Basic_Modules/Defer_Equal_Condition.thy%:%
%:%6=3%:%
%:%11=4%:%
%:%12=5%:%
%:%14=8%:%
%:%30=10%:%
%:%31=10%:%
%:%32=11%:%
%:%33=12%:%
%:%42=15%:%
%:%43=16%:%
%:%44=17%:%
%:%53=19%:%
%:%63=21%:%
%:%64=21%:%
%:%65=22%:%
%:%68=23%:%
%:%73=24%:%
%
\begin{isabellebody}%
\setisabellecontext{Defer{\isacharunderscore}{\kern0pt}Module}%
%
\isadelimdocument
\isanewline
%
\endisadelimdocument
%
\isatagdocument
\isanewline
\isanewline
%
\isamarkupchapter{Basic Modules%
}
\isamarkuptrue%
%
\isamarkupsection{Defer Module%
}
\isamarkuptrue%
%
\endisatagdocument
{\isafolddocument}%
%
\isadelimdocument
%
\endisadelimdocument
%
\isadelimtheory
%
\endisadelimtheory
%
\isatagtheory
\isacommand{theory}\isamarkupfalse%
\ Defer{\isacharunderscore}{\kern0pt}Module\isanewline
\ \ \isakeyword{imports}\ {\isachardoublequoteopen}{\isachardot}{\kern0pt}{\isachardot}{\kern0pt}{\isacharslash}{\kern0pt}Electoral{\isacharunderscore}{\kern0pt}Module{\isachardoublequoteclose}\isanewline
\isakeyword{begin}%
\endisatagtheory
{\isafoldtheory}%
%
\isadelimtheory
%
\endisadelimtheory
%
\begin{isamarkuptext}%
The defer module is not concerned about the voter's ballots, and
simply defers all alternatives.
It is primarily used for defining an empty loop.%
\end{isamarkuptext}\isamarkuptrue%
%
\isadelimdocument
%
\endisadelimdocument
%
\isatagdocument
%
\isamarkupsubsection{Definition%
}
\isamarkuptrue%
%
\endisatagdocument
{\isafolddocument}%
%
\isadelimdocument
%
\endisadelimdocument
\isacommand{fun}\isamarkupfalse%
\ defer{\isacharunderscore}{\kern0pt}module\ {\isacharcolon}{\kern0pt}{\isacharcolon}{\kern0pt}\ {\isachardoublequoteopen}{\isacharprime}{\kern0pt}a\ Electoral{\isacharunderscore}{\kern0pt}Module{\isachardoublequoteclose}\ \isakeyword{where}\isanewline
\ \ {\isachardoublequoteopen}defer{\isacharunderscore}{\kern0pt}module\ A\ p\ {\isacharequal}{\kern0pt}\ {\isacharparenleft}{\kern0pt}{\isacharbraceleft}{\kern0pt}{\isacharbraceright}{\kern0pt}{\isacharcomma}{\kern0pt}\ {\isacharbraceleft}{\kern0pt}{\isacharbraceright}{\kern0pt}{\isacharcomma}{\kern0pt}\ A{\isacharparenright}{\kern0pt}{\isachardoublequoteclose}%
\isadelimdocument
%
\endisadelimdocument
%
\isatagdocument
%
\isamarkupsubsection{Soundness%
}
\isamarkuptrue%
%
\endisatagdocument
{\isafolddocument}%
%
\isadelimdocument
%
\endisadelimdocument
\isacommand{theorem}\isamarkupfalse%
\ def{\isacharunderscore}{\kern0pt}mod{\isacharunderscore}{\kern0pt}sound{\isacharbrackleft}{\kern0pt}simp{\isacharbrackright}{\kern0pt}{\isacharcolon}{\kern0pt}\ {\isachardoublequoteopen}electoral{\isacharunderscore}{\kern0pt}module\ defer{\isacharunderscore}{\kern0pt}module{\isachardoublequoteclose}\isanewline
%
\isadelimproof
\ \ %
\endisadelimproof
%
\isatagproof
\isacommand{unfolding}\isamarkupfalse%
\ electoral{\isacharunderscore}{\kern0pt}module{\isacharunderscore}{\kern0pt}def\isanewline
\ \ \isacommand{by}\isamarkupfalse%
\ simp%
\endisatagproof
{\isafoldproof}%
%
\isadelimproof
\isanewline
%
\endisadelimproof
%
\isadelimtheory
\isanewline
%
\endisadelimtheory
%
\isatagtheory
\isacommand{end}\isamarkupfalse%
%
\endisatagtheory
{\isafoldtheory}%
%
\isadelimtheory
%
\endisadelimtheory
%
\end{isabellebody}%
\endinput
%:%file=~/Documents/Studies/VotingRuleGenerator/virage/src/test/resources/verifiedVotingRuleConstruction/theories/Compositional_Framework/Components/Basic_Modules/Defer_Module.thy%:%
%:%6=3%:%
%:%11=4%:%
%:%12=5%:%
%:%14=8%:%
%:%18=10%:%
%:%34=12%:%
%:%35=12%:%
%:%36=13%:%
%:%37=14%:%
%:%46=17%:%
%:%47=18%:%
%:%48=19%:%
%:%57=21%:%
%:%67=23%:%
%:%68=23%:%
%:%69=24%:%
%:%76=26%:%
%:%86=28%:%
%:%87=28%:%
%:%90=29%:%
%:%94=29%:%
%:%95=29%:%
%:%96=30%:%
%:%97=30%:%
%:%102=30%:%
%:%107=31%:%
%:%112=32%:%
%
\begin{isabellebody}%
\setisabellecontext{Drop{\isacharunderscore}{\kern0pt}Module}%
%
\isadelimdocument
\isanewline
%
\endisadelimdocument
%
\isatagdocument
\isanewline
\isanewline
%
\isamarkupsection{Drop Module%
}
\isamarkuptrue%
%
\endisatagdocument
{\isafolddocument}%
%
\isadelimdocument
%
\endisadelimdocument
%
\isadelimtheory
%
\endisadelimtheory
%
\isatagtheory
\isacommand{theory}\isamarkupfalse%
\ Drop{\isacharunderscore}{\kern0pt}Module\isanewline
\ \ \isakeyword{imports}\ {\isachardoublequoteopen}Component{\isacharunderscore}{\kern0pt}Types{\isacharslash}{\kern0pt}Electoral{\isacharunderscore}{\kern0pt}Module{\isachardoublequoteclose}\isanewline
\isakeyword{begin}%
\endisatagtheory
{\isafoldtheory}%
%
\isadelimtheory
%
\endisadelimtheory
%
\begin{isamarkuptext}%
This is a family of electoral modules. For a natural number n and a
lexicon (linear order) r of all alternatives, the according drop module
rejects the lexicographically first n alternatives (from A) and
defers the rest.
It is primarily used as counterpart to the pass module in a
parallel composition, in order to segment the alternatives into
two groups.%
\end{isamarkuptext}\isamarkuptrue%
%
\isadelimdocument
%
\endisadelimdocument
%
\isatagdocument
%
\isamarkupsubsection{Definition%
}
\isamarkuptrue%
%
\endisatagdocument
{\isafolddocument}%
%
\isadelimdocument
%
\endisadelimdocument
\isacommand{fun}\isamarkupfalse%
\ drop{\isacharunderscore}{\kern0pt}module\ {\isacharcolon}{\kern0pt}{\isacharcolon}{\kern0pt}\ {\isachardoublequoteopen}nat\ {\isasymRightarrow}\ {\isacharprime}{\kern0pt}a\ Preference{\isacharunderscore}{\kern0pt}Relation\ {\isasymRightarrow}\ {\isacharprime}{\kern0pt}a\ Electoral{\isacharunderscore}{\kern0pt}Module{\isachardoublequoteclose}\ \isakeyword{where}\isanewline
\ \ {\isachardoublequoteopen}drop{\isacharunderscore}{\kern0pt}module\ n\ r\ A\ p\ {\isacharequal}{\kern0pt}\isanewline
\ \ \ \ {\isacharparenleft}{\kern0pt}{\isacharbraceleft}{\kern0pt}{\isacharbraceright}{\kern0pt}{\isacharcomma}{\kern0pt}\isanewline
\ \ \ \ {\isacharbraceleft}{\kern0pt}a\ {\isasymin}\ A{\isachardot}{\kern0pt}\ card{\isacharparenleft}{\kern0pt}above\ {\isacharparenleft}{\kern0pt}limit\ A\ r{\isacharparenright}{\kern0pt}\ a{\isacharparenright}{\kern0pt}\ {\isasymle}\ n{\isacharbraceright}{\kern0pt}{\isacharcomma}{\kern0pt}\isanewline
\ \ \ \ {\isacharbraceleft}{\kern0pt}a\ {\isasymin}\ A{\isachardot}{\kern0pt}\ card{\isacharparenleft}{\kern0pt}above\ {\isacharparenleft}{\kern0pt}limit\ A\ r{\isacharparenright}{\kern0pt}\ a{\isacharparenright}{\kern0pt}\ {\isachargreater}{\kern0pt}\ n{\isacharbraceright}{\kern0pt}{\isacharparenright}{\kern0pt}{\isachardoublequoteclose}%
\isadelimdocument
%
\endisadelimdocument
%
\isatagdocument
%
\isamarkupsubsection{Soundness%
}
\isamarkuptrue%
%
\endisatagdocument
{\isafolddocument}%
%
\isadelimdocument
%
\endisadelimdocument
\isacommand{theorem}\isamarkupfalse%
\ drop{\isacharunderscore}{\kern0pt}mod{\isacharunderscore}{\kern0pt}sound{\isacharbrackleft}{\kern0pt}simp{\isacharbrackright}{\kern0pt}{\isacharcolon}{\kern0pt}\isanewline
\ \ \isakeyword{assumes}\ order{\isacharcolon}{\kern0pt}\ {\isachardoublequoteopen}linear{\isacharunderscore}{\kern0pt}order\ r{\isachardoublequoteclose}\isanewline
\ \ \isakeyword{shows}\ {\isachardoublequoteopen}electoral{\isacharunderscore}{\kern0pt}module\ {\isacharparenleft}{\kern0pt}drop{\isacharunderscore}{\kern0pt}module\ n\ r{\isacharparenright}{\kern0pt}{\isachardoublequoteclose}\isanewline
%
\isadelimproof
%
\endisadelimproof
%
\isatagproof
\isacommand{proof}\isamarkupfalse%
\ {\isacharminus}{\kern0pt}\isanewline
\ \ \isacommand{let}\isamarkupfalse%
\ {\isacharquery}{\kern0pt}mod\ {\isacharequal}{\kern0pt}\ {\isachardoublequoteopen}drop{\isacharunderscore}{\kern0pt}module\ n\ r{\isachardoublequoteclose}\isanewline
\ \ \isacommand{have}\isamarkupfalse%
\isanewline
\ \ \ \ {\isachardoublequoteopen}{\isasymforall}A\ p{\isachardot}{\kern0pt}\ finite{\isacharunderscore}{\kern0pt}profile\ A\ p\ {\isasymlongrightarrow}\isanewline
\ \ \ \ \ \ \ \ {\isacharparenleft}{\kern0pt}{\isasymforall}a\ {\isasymin}\ A{\isachardot}{\kern0pt}\ a\ {\isasymin}\ {\isacharbraceleft}{\kern0pt}x\ {\isasymin}\ A{\isachardot}{\kern0pt}\ card{\isacharparenleft}{\kern0pt}above\ {\isacharparenleft}{\kern0pt}limit\ A\ r{\isacharparenright}{\kern0pt}\ x{\isacharparenright}{\kern0pt}\ {\isasymle}\ n{\isacharbraceright}{\kern0pt}\ {\isasymor}\isanewline
\ \ \ \ \ \ \ \ \ \ \ \ a\ {\isasymin}\ {\isacharbraceleft}{\kern0pt}x\ {\isasymin}\ A{\isachardot}{\kern0pt}\ card{\isacharparenleft}{\kern0pt}above\ {\isacharparenleft}{\kern0pt}limit\ A\ r{\isacharparenright}{\kern0pt}\ x{\isacharparenright}{\kern0pt}\ {\isachargreater}{\kern0pt}\ n{\isacharbraceright}{\kern0pt}{\isacharparenright}{\kern0pt}{\isachardoublequoteclose}\isanewline
\ \ \ \ \isacommand{by}\isamarkupfalse%
\ auto\isanewline
\ \ \isacommand{hence}\isamarkupfalse%
\isanewline
\ \ \ \ {\isachardoublequoteopen}{\isasymforall}A\ p{\isachardot}{\kern0pt}\ finite{\isacharunderscore}{\kern0pt}profile\ A\ p\ {\isasymlongrightarrow}\isanewline
\ \ \ \ \ \ \ \ {\isacharbraceleft}{\kern0pt}a\ {\isasymin}\ A{\isachardot}{\kern0pt}\ card{\isacharparenleft}{\kern0pt}above\ {\isacharparenleft}{\kern0pt}limit\ A\ r{\isacharparenright}{\kern0pt}\ a{\isacharparenright}{\kern0pt}\ {\isasymle}\ n{\isacharbraceright}{\kern0pt}\ {\isasymunion}\isanewline
\ \ \ \ \ \ \ \ {\isacharbraceleft}{\kern0pt}a\ {\isasymin}\ A{\isachardot}{\kern0pt}\ card{\isacharparenleft}{\kern0pt}above\ {\isacharparenleft}{\kern0pt}limit\ A\ r{\isacharparenright}{\kern0pt}\ a{\isacharparenright}{\kern0pt}\ {\isachargreater}{\kern0pt}\ n{\isacharbraceright}{\kern0pt}\ {\isacharequal}{\kern0pt}\ A{\isachardoublequoteclose}\isanewline
\ \ \ \ \isacommand{by}\isamarkupfalse%
\ blast\isanewline
\ \ \isacommand{hence}\isamarkupfalse%
\ {\isadigit{0}}{\isacharcolon}{\kern0pt}\isanewline
\ \ \ \ {\isachardoublequoteopen}{\isasymforall}A\ p{\isachardot}{\kern0pt}\ finite{\isacharunderscore}{\kern0pt}profile\ A\ p\ {\isasymlongrightarrow}\isanewline
\ \ \ \ \ \ \ \ set{\isacharunderscore}{\kern0pt}equals{\isacharunderscore}{\kern0pt}partition\ A\ {\isacharparenleft}{\kern0pt}drop{\isacharunderscore}{\kern0pt}module\ n\ r\ A\ p{\isacharparenright}{\kern0pt}{\isachardoublequoteclose}\isanewline
\ \ \ \ \isacommand{by}\isamarkupfalse%
\ simp\isanewline
\ \ \isacommand{have}\isamarkupfalse%
\isanewline
\ \ \ \ {\isachardoublequoteopen}{\isasymforall}A\ p{\isachardot}{\kern0pt}\ finite{\isacharunderscore}{\kern0pt}profile\ A\ p\ {\isasymlongrightarrow}\isanewline
\ \ \ \ \ \ \ \ {\isacharparenleft}{\kern0pt}{\isasymforall}a\ {\isasymin}\ A{\isachardot}{\kern0pt}\ {\isasymnot}{\isacharparenleft}{\kern0pt}a\ {\isasymin}\ {\isacharbraceleft}{\kern0pt}x\ {\isasymin}\ A{\isachardot}{\kern0pt}\ card{\isacharparenleft}{\kern0pt}above\ {\isacharparenleft}{\kern0pt}limit\ A\ r{\isacharparenright}{\kern0pt}\ x{\isacharparenright}{\kern0pt}\ {\isasymle}\ n{\isacharbraceright}{\kern0pt}\ {\isasymand}\isanewline
\ \ \ \ \ \ \ \ \ \ \ \ a\ {\isasymin}\ {\isacharbraceleft}{\kern0pt}x\ {\isasymin}\ A{\isachardot}{\kern0pt}\ card{\isacharparenleft}{\kern0pt}above\ {\isacharparenleft}{\kern0pt}limit\ A\ r{\isacharparenright}{\kern0pt}\ x{\isacharparenright}{\kern0pt}\ {\isachargreater}{\kern0pt}\ n{\isacharbraceright}{\kern0pt}{\isacharparenright}{\kern0pt}{\isacharparenright}{\kern0pt}{\isachardoublequoteclose}\isanewline
\ \ \ \ \isacommand{by}\isamarkupfalse%
\ auto\isanewline
\ \ \isacommand{hence}\isamarkupfalse%
\isanewline
\ \ \ \ {\isachardoublequoteopen}{\isasymforall}A\ p{\isachardot}{\kern0pt}\ finite{\isacharunderscore}{\kern0pt}profile\ A\ p\ {\isasymlongrightarrow}\isanewline
\ \ \ \ \ \ \ \ {\isacharbraceleft}{\kern0pt}a\ {\isasymin}\ A{\isachardot}{\kern0pt}\ card{\isacharparenleft}{\kern0pt}above\ {\isacharparenleft}{\kern0pt}limit\ A\ r{\isacharparenright}{\kern0pt}\ a{\isacharparenright}{\kern0pt}\ {\isasymle}\ n{\isacharbraceright}{\kern0pt}\ {\isasyminter}\isanewline
\ \ \ \ \ \ \ \ {\isacharbraceleft}{\kern0pt}a\ {\isasymin}\ A{\isachardot}{\kern0pt}\ card{\isacharparenleft}{\kern0pt}above\ {\isacharparenleft}{\kern0pt}limit\ A\ r{\isacharparenright}{\kern0pt}\ a{\isacharparenright}{\kern0pt}\ {\isachargreater}{\kern0pt}\ n{\isacharbraceright}{\kern0pt}\ {\isacharequal}{\kern0pt}\ {\isacharbraceleft}{\kern0pt}{\isacharbraceright}{\kern0pt}{\isachardoublequoteclose}\isanewline
\ \ \ \ \isacommand{by}\isamarkupfalse%
\ blast\isanewline
\ \ \isacommand{hence}\isamarkupfalse%
\ {\isadigit{1}}{\isacharcolon}{\kern0pt}\ {\isachardoublequoteopen}{\isasymforall}A\ p{\isachardot}{\kern0pt}\ finite{\isacharunderscore}{\kern0pt}profile\ A\ p\ {\isasymlongrightarrow}\ disjoint{\isadigit{3}}\ {\isacharparenleft}{\kern0pt}{\isacharquery}{\kern0pt}mod\ A\ p{\isacharparenright}{\kern0pt}{\isachardoublequoteclose}\isanewline
\ \ \ \ \isacommand{by}\isamarkupfalse%
\ simp\isanewline
\ \ \isacommand{from}\isamarkupfalse%
\ {\isadigit{0}}\ {\isadigit{1}}\ \isacommand{have}\isamarkupfalse%
\isanewline
\ \ \ \ {\isachardoublequoteopen}{\isasymforall}A\ p{\isachardot}{\kern0pt}\ finite{\isacharunderscore}{\kern0pt}profile\ A\ p\ {\isasymlongrightarrow}\isanewline
\ \ \ \ \ \ \ \ well{\isacharunderscore}{\kern0pt}formed\ A\ {\isacharparenleft}{\kern0pt}{\isacharquery}{\kern0pt}mod\ A\ p{\isacharparenright}{\kern0pt}{\isachardoublequoteclose}\isanewline
\ \ \ \ \isacommand{by}\isamarkupfalse%
\ simp\isanewline
\ \ \isacommand{hence}\isamarkupfalse%
\isanewline
\ \ \ \ {\isachardoublequoteopen}{\isasymforall}A\ p{\isachardot}{\kern0pt}\ finite{\isacharunderscore}{\kern0pt}profile\ A\ p\ {\isasymlongrightarrow}\isanewline
\ \ \ \ \ \ \ \ well{\isacharunderscore}{\kern0pt}formed\ A\ {\isacharparenleft}{\kern0pt}{\isacharquery}{\kern0pt}mod\ A\ p{\isacharparenright}{\kern0pt}{\isachardoublequoteclose}\isanewline
\ \ \ \ \isacommand{by}\isamarkupfalse%
\ simp\isanewline
\ \ \isacommand{thus}\isamarkupfalse%
\ {\isacharquery}{\kern0pt}thesis\isanewline
\ \ \ \ \isacommand{using}\isamarkupfalse%
\ electoral{\isacharunderscore}{\kern0pt}modI\isanewline
\ \ \ \ \isacommand{by}\isamarkupfalse%
\ metis\isanewline
\isacommand{qed}\isamarkupfalse%
%
\endisatagproof
{\isafoldproof}%
%
\isadelimproof
%
\endisadelimproof
%
\isadelimdocument
%
\endisadelimdocument
%
\isatagdocument
%
\isamarkupsubsection{Non-Electing%
}
\isamarkuptrue%
%
\endisatagdocument
{\isafolddocument}%
%
\isadelimdocument
%
\endisadelimdocument
\isacommand{theorem}\isamarkupfalse%
\ drop{\isacharunderscore}{\kern0pt}mod{\isacharunderscore}{\kern0pt}non{\isacharunderscore}{\kern0pt}electing{\isacharbrackleft}{\kern0pt}simp{\isacharbrackright}{\kern0pt}{\isacharcolon}{\kern0pt}\isanewline
\ \ \isakeyword{assumes}\ order{\isacharcolon}{\kern0pt}\ {\isachardoublequoteopen}linear{\isacharunderscore}{\kern0pt}order\ r{\isachardoublequoteclose}\isanewline
\ \ \isakeyword{shows}\ {\isachardoublequoteopen}non{\isacharunderscore}{\kern0pt}electing\ {\isacharparenleft}{\kern0pt}drop{\isacharunderscore}{\kern0pt}module\ n\ r{\isacharparenright}{\kern0pt}{\isachardoublequoteclose}\isanewline
%
\isadelimproof
\ \ %
\endisadelimproof
%
\isatagproof
\isacommand{by}\isamarkupfalse%
\ {\isacharparenleft}{\kern0pt}simp\ add{\isacharcolon}{\kern0pt}\ non{\isacharunderscore}{\kern0pt}electing{\isacharunderscore}{\kern0pt}def\ order{\isacharparenright}{\kern0pt}%
\endisatagproof
{\isafoldproof}%
%
\isadelimproof
%
\endisadelimproof
%
\isadelimdocument
%
\endisadelimdocument
%
\isatagdocument
%
\isamarkupsubsection{Properties%
}
\isamarkuptrue%
%
\endisatagdocument
{\isafolddocument}%
%
\isadelimdocument
%
\endisadelimdocument
\isacommand{theorem}\isamarkupfalse%
\ drop{\isacharunderscore}{\kern0pt}mod{\isacharunderscore}{\kern0pt}def{\isacharunderscore}{\kern0pt}lift{\isacharunderscore}{\kern0pt}inv{\isacharbrackleft}{\kern0pt}simp{\isacharbrackright}{\kern0pt}{\isacharcolon}{\kern0pt}\isanewline
\ \ \isakeyword{assumes}\ order{\isacharcolon}{\kern0pt}\ {\isachardoublequoteopen}linear{\isacharunderscore}{\kern0pt}order\ r{\isachardoublequoteclose}\isanewline
\ \ \isakeyword{shows}\ {\isachardoublequoteopen}defer{\isacharunderscore}{\kern0pt}lift{\isacharunderscore}{\kern0pt}invariance\ {\isacharparenleft}{\kern0pt}drop{\isacharunderscore}{\kern0pt}module\ n\ r{\isacharparenright}{\kern0pt}{\isachardoublequoteclose}\isanewline
%
\isadelimproof
\ \ %
\endisadelimproof
%
\isatagproof
\isacommand{by}\isamarkupfalse%
\ {\isacharparenleft}{\kern0pt}simp\ add{\isacharcolon}{\kern0pt}\ order\ defer{\isacharunderscore}{\kern0pt}lift{\isacharunderscore}{\kern0pt}invariance{\isacharunderscore}{\kern0pt}def{\isacharparenright}{\kern0pt}%
\endisatagproof
{\isafoldproof}%
%
\isadelimproof
\isanewline
%
\endisadelimproof
%
\isadelimtheory
\isanewline
%
\endisadelimtheory
%
\isatagtheory
\isacommand{end}\isamarkupfalse%
%
\endisatagtheory
{\isafoldtheory}%
%
\isadelimtheory
%
\endisadelimtheory
%
\end{isabellebody}%
\endinput
%:%file=~/Documents/Studies/VotingRuleGenerator/virage/src/test/resources/old_theories/Compositional_Structures/Basic_Modules/Drop_Module.thy%:%
%:%6=3%:%
%:%11=4%:%
%:%12=5%:%
%:%14=8%:%
%:%30=10%:%
%:%31=10%:%
%:%32=11%:%
%:%33=12%:%
%:%42=15%:%
%:%43=16%:%
%:%44=17%:%
%:%45=18%:%
%:%46=19%:%
%:%47=20%:%
%:%48=21%:%
%:%57=23%:%
%:%67=25%:%
%:%68=25%:%
%:%69=26%:%
%:%79=31%:%
%:%89=33%:%
%:%90=33%:%
%:%91=34%:%
%:%92=35%:%
%:%99=36%:%
%:%100=36%:%
%:%101=37%:%
%:%102=37%:%
%:%103=38%:%
%:%104=38%:%
%:%105=39%:%
%:%107=41%:%
%:%108=42%:%
%:%109=42%:%
%:%110=43%:%
%:%111=43%:%
%:%112=44%:%
%:%114=46%:%
%:%115=47%:%
%:%116=47%:%
%:%117=48%:%
%:%118=48%:%
%:%119=49%:%
%:%120=50%:%
%:%121=51%:%
%:%122=51%:%
%:%123=52%:%
%:%124=52%:%
%:%125=53%:%
%:%127=55%:%
%:%128=56%:%
%:%129=56%:%
%:%130=57%:%
%:%131=57%:%
%:%132=58%:%
%:%134=60%:%
%:%135=61%:%
%:%136=61%:%
%:%137=62%:%
%:%138=62%:%
%:%139=63%:%
%:%140=63%:%
%:%141=64%:%
%:%142=64%:%
%:%143=64%:%
%:%144=65%:%
%:%145=66%:%
%:%146=67%:%
%:%147=67%:%
%:%148=68%:%
%:%149=68%:%
%:%150=69%:%
%:%151=70%:%
%:%152=71%:%
%:%153=71%:%
%:%154=72%:%
%:%155=72%:%
%:%156=73%:%
%:%157=73%:%
%:%158=74%:%
%:%159=74%:%
%:%160=75%:%
%:%175=77%:%
%:%185=80%:%
%:%186=80%:%
%:%187=81%:%
%:%188=82%:%
%:%191=83%:%
%:%195=83%:%
%:%196=83%:%
%:%210=85%:%
%:%220=88%:%
%:%221=88%:%
%:%222=89%:%
%:%223=90%:%
%:%226=91%:%
%:%230=91%:%
%:%231=91%:%
%:%236=91%:%
%:%241=92%:%
%:%246=93%:%
%
\begin{isabellebody}%
\setisabellecontext{Pass{\isacharunderscore}{\kern0pt}Module}%
%
\isadelimdocument
\isanewline
%
\endisadelimdocument
%
\isatagdocument
\isanewline
\isanewline
%
\isamarkupsection{Pass Module%
}
\isamarkuptrue%
%
\endisatagdocument
{\isafolddocument}%
%
\isadelimdocument
%
\endisadelimdocument
%
\isadelimtheory
%
\endisadelimtheory
%
\isatagtheory
\isacommand{theory}\isamarkupfalse%
\ Pass{\isacharunderscore}{\kern0pt}Module\isanewline
\ \ \isakeyword{imports}\ {\isachardoublequoteopen}{\isachardot}{\kern0pt}{\isachardot}{\kern0pt}{\isacharslash}{\kern0pt}Electoral{\isacharunderscore}{\kern0pt}Module{\isachardoublequoteclose}\isanewline
\isakeyword{begin}%
\endisatagtheory
{\isafoldtheory}%
%
\isadelimtheory
%
\endisadelimtheory
%
\begin{isamarkuptext}%
This is a family of electoral modules. For a natural number n and a
lexicon (linear order) r of all alternatives, the according pass module
defers the lexicographically first n alternatives (from A) and rejects
the rest. It is primarily used as counterpart to the drop module in a
parallel composition in order to segment the alternatives into two groups.%
\end{isamarkuptext}\isamarkuptrue%
%
\isadelimdocument
%
\endisadelimdocument
%
\isatagdocument
%
\isamarkupsubsection{Definition%
}
\isamarkuptrue%
%
\endisatagdocument
{\isafolddocument}%
%
\isadelimdocument
%
\endisadelimdocument
\isacommand{fun}\isamarkupfalse%
\ pass{\isacharunderscore}{\kern0pt}module\ {\isacharcolon}{\kern0pt}{\isacharcolon}{\kern0pt}\ {\isachardoublequoteopen}nat\ {\isasymRightarrow}\ {\isacharprime}{\kern0pt}a\ Preference{\isacharunderscore}{\kern0pt}Relation\ {\isasymRightarrow}\ {\isacharprime}{\kern0pt}a\ Electoral{\isacharunderscore}{\kern0pt}Module{\isachardoublequoteclose}\ \isakeyword{where}\isanewline
\ \ {\isachardoublequoteopen}pass{\isacharunderscore}{\kern0pt}module\ n\ r\ A\ p\ {\isacharequal}{\kern0pt}\isanewline
\ \ \ \ {\isacharparenleft}{\kern0pt}{\isacharbraceleft}{\kern0pt}{\isacharbraceright}{\kern0pt}{\isacharcomma}{\kern0pt}\isanewline
\ \ \ \ {\isacharbraceleft}{\kern0pt}a\ {\isasymin}\ A{\isachardot}{\kern0pt}\ card{\isacharparenleft}{\kern0pt}above\ {\isacharparenleft}{\kern0pt}limit\ A\ r{\isacharparenright}{\kern0pt}\ a{\isacharparenright}{\kern0pt}\ {\isachargreater}{\kern0pt}\ n{\isacharbraceright}{\kern0pt}{\isacharcomma}{\kern0pt}\isanewline
\ \ \ \ {\isacharbraceleft}{\kern0pt}a\ {\isasymin}\ A{\isachardot}{\kern0pt}\ card{\isacharparenleft}{\kern0pt}above\ {\isacharparenleft}{\kern0pt}limit\ A\ r{\isacharparenright}{\kern0pt}\ a{\isacharparenright}{\kern0pt}\ {\isasymle}\ n{\isacharbraceright}{\kern0pt}{\isacharparenright}{\kern0pt}{\isachardoublequoteclose}%
\isadelimdocument
%
\endisadelimdocument
%
\isatagdocument
%
\isamarkupsubsection{Soundness%
}
\isamarkuptrue%
%
\endisatagdocument
{\isafolddocument}%
%
\isadelimdocument
%
\endisadelimdocument
\isacommand{theorem}\isamarkupfalse%
\ pass{\isacharunderscore}{\kern0pt}mod{\isacharunderscore}{\kern0pt}sound{\isacharbrackleft}{\kern0pt}simp{\isacharbrackright}{\kern0pt}{\isacharcolon}{\kern0pt}\isanewline
\ \ \isakeyword{assumes}\ order{\isacharcolon}{\kern0pt}\ {\isachardoublequoteopen}linear{\isacharunderscore}{\kern0pt}order\ r{\isachardoublequoteclose}\isanewline
\ \ \isakeyword{shows}\ {\isachardoublequoteopen}electoral{\isacharunderscore}{\kern0pt}module\ {\isacharparenleft}{\kern0pt}pass{\isacharunderscore}{\kern0pt}module\ n\ r{\isacharparenright}{\kern0pt}{\isachardoublequoteclose}\isanewline
%
\isadelimproof
%
\endisadelimproof
%
\isatagproof
\isacommand{proof}\isamarkupfalse%
\ {\isacharminus}{\kern0pt}\isanewline
\ \ \isacommand{let}\isamarkupfalse%
\ {\isacharquery}{\kern0pt}mod\ {\isacharequal}{\kern0pt}\ {\isachardoublequoteopen}pass{\isacharunderscore}{\kern0pt}module\ n\ r{\isachardoublequoteclose}\isanewline
\ \ \isacommand{have}\isamarkupfalse%
\isanewline
\ \ \ \ {\isachardoublequoteopen}{\isasymforall}\ A\ p{\isachardot}{\kern0pt}\ finite{\isacharunderscore}{\kern0pt}profile\ A\ p\ {\isasymlongrightarrow}\isanewline
\ \ \ \ \ \ \ \ \ \ {\isacharparenleft}{\kern0pt}{\isasymforall}a\ {\isasymin}\ A{\isachardot}{\kern0pt}\ a\ {\isasymin}\ {\isacharbraceleft}{\kern0pt}x\ {\isasymin}\ A{\isachardot}{\kern0pt}\ card{\isacharparenleft}{\kern0pt}above\ {\isacharparenleft}{\kern0pt}limit\ A\ r{\isacharparenright}{\kern0pt}\ x{\isacharparenright}{\kern0pt}\ {\isachargreater}{\kern0pt}\ n{\isacharbraceright}{\kern0pt}\ {\isasymor}\isanewline
\ \ \ \ \ \ \ \ \ \ \ \ \ \ \ \ \ \ \ a\ {\isasymin}\ {\isacharbraceleft}{\kern0pt}x\ {\isasymin}\ A{\isachardot}{\kern0pt}\ card{\isacharparenleft}{\kern0pt}above\ {\isacharparenleft}{\kern0pt}limit\ A\ r{\isacharparenright}{\kern0pt}\ x{\isacharparenright}{\kern0pt}\ {\isasymle}\ n{\isacharbraceright}{\kern0pt}{\isacharparenright}{\kern0pt}{\isachardoublequoteclose}\isanewline
\ \ \ \ \isacommand{using}\isamarkupfalse%
\ CollectI\ not{\isacharunderscore}{\kern0pt}less\isanewline
\ \ \ \ \isacommand{by}\isamarkupfalse%
\ metis\isanewline
\ \ \isacommand{hence}\isamarkupfalse%
\isanewline
\ \ \ \ {\isachardoublequoteopen}{\isasymforall}\ A\ p{\isachardot}{\kern0pt}\ finite{\isacharunderscore}{\kern0pt}profile\ A\ p\ {\isasymlongrightarrow}\isanewline
\ \ \ \ \ \ \ \ \ \ {\isacharbraceleft}{\kern0pt}a\ {\isasymin}\ A{\isachardot}{\kern0pt}\ card{\isacharparenleft}{\kern0pt}above\ {\isacharparenleft}{\kern0pt}limit\ A\ r{\isacharparenright}{\kern0pt}\ a{\isacharparenright}{\kern0pt}\ {\isachargreater}{\kern0pt}\ n{\isacharbraceright}{\kern0pt}\ {\isasymunion}\isanewline
\ \ \ \ \ \ \ \ \ \ {\isacharbraceleft}{\kern0pt}a\ {\isasymin}\ A{\isachardot}{\kern0pt}\ card{\isacharparenleft}{\kern0pt}above\ {\isacharparenleft}{\kern0pt}limit\ A\ r{\isacharparenright}{\kern0pt}\ a{\isacharparenright}{\kern0pt}\ {\isasymle}\ n{\isacharbraceright}{\kern0pt}\ {\isacharequal}{\kern0pt}\ A{\isachardoublequoteclose}\isanewline
\ \ \ \ \isacommand{by}\isamarkupfalse%
\ blast\isanewline
\ \ \isacommand{hence}\isamarkupfalse%
\ {\isadigit{0}}{\isacharcolon}{\kern0pt}\isanewline
\ \ \ \ {\isachardoublequoteopen}{\isasymforall}\ A\ p{\isachardot}{\kern0pt}\ finite{\isacharunderscore}{\kern0pt}profile\ A\ p\ {\isasymlongrightarrow}\ set{\isacharunderscore}{\kern0pt}equals{\isacharunderscore}{\kern0pt}partition\ A\ {\isacharparenleft}{\kern0pt}pass{\isacharunderscore}{\kern0pt}module\ n\ r\ A\ p{\isacharparenright}{\kern0pt}{\isachardoublequoteclose}\isanewline
\ \ \ \ \isacommand{by}\isamarkupfalse%
\ simp\isanewline
\ \ \isacommand{have}\isamarkupfalse%
\isanewline
\ \ \ \ {\isachardoublequoteopen}{\isasymforall}\ A\ p{\isachardot}{\kern0pt}\ finite{\isacharunderscore}{\kern0pt}profile\ A\ p\ {\isasymlongrightarrow}\isanewline
\ \ \ \ \ \ {\isacharparenleft}{\kern0pt}{\isasymforall}a\ {\isasymin}\ A{\isachardot}{\kern0pt}\ {\isasymnot}{\isacharparenleft}{\kern0pt}a\ {\isasymin}\ {\isacharbraceleft}{\kern0pt}x\ {\isasymin}\ A{\isachardot}{\kern0pt}\ card{\isacharparenleft}{\kern0pt}above\ {\isacharparenleft}{\kern0pt}limit\ A\ r{\isacharparenright}{\kern0pt}\ x{\isacharparenright}{\kern0pt}\ {\isachargreater}{\kern0pt}\ n{\isacharbraceright}{\kern0pt}\ {\isasymand}\isanewline
\ \ \ \ \ \ \ \ \ \ \ \ \ \ \ \ \ a\ {\isasymin}\ {\isacharbraceleft}{\kern0pt}x\ {\isasymin}\ A{\isachardot}{\kern0pt}\ card{\isacharparenleft}{\kern0pt}above\ {\isacharparenleft}{\kern0pt}limit\ A\ r{\isacharparenright}{\kern0pt}\ x{\isacharparenright}{\kern0pt}\ {\isasymle}\ n{\isacharbraceright}{\kern0pt}{\isacharparenright}{\kern0pt}{\isacharparenright}{\kern0pt}{\isachardoublequoteclose}\isanewline
\ \ \ \ \isacommand{by}\isamarkupfalse%
\ auto\isanewline
\ \ \isacommand{hence}\isamarkupfalse%
\isanewline
\ \ \ \ {\isachardoublequoteopen}{\isasymforall}\ A\ p{\isachardot}{\kern0pt}\ finite{\isacharunderscore}{\kern0pt}profile\ A\ p\ {\isasymlongrightarrow}\isanewline
\ \ \ \ \ \ {\isacharbraceleft}{\kern0pt}a\ {\isasymin}\ A{\isachardot}{\kern0pt}\ card{\isacharparenleft}{\kern0pt}above\ {\isacharparenleft}{\kern0pt}limit\ A\ r{\isacharparenright}{\kern0pt}\ a{\isacharparenright}{\kern0pt}\ {\isachargreater}{\kern0pt}\ n{\isacharbraceright}{\kern0pt}\ {\isasyminter}\isanewline
\ \ \ \ \ \ {\isacharbraceleft}{\kern0pt}a\ {\isasymin}\ A{\isachardot}{\kern0pt}\ card{\isacharparenleft}{\kern0pt}above\ {\isacharparenleft}{\kern0pt}limit\ A\ r{\isacharparenright}{\kern0pt}\ a{\isacharparenright}{\kern0pt}\ {\isasymle}\ n{\isacharbraceright}{\kern0pt}\ {\isacharequal}{\kern0pt}\ {\isacharbraceleft}{\kern0pt}{\isacharbraceright}{\kern0pt}{\isachardoublequoteclose}\isanewline
\ \ \ \ \isacommand{by}\isamarkupfalse%
\ blast\isanewline
\ \ \isacommand{hence}\isamarkupfalse%
\ {\isadigit{1}}{\isacharcolon}{\kern0pt}\isanewline
\ \ \ \ {\isachardoublequoteopen}{\isasymforall}\ A\ p{\isachardot}{\kern0pt}\ finite{\isacharunderscore}{\kern0pt}profile\ A\ p\ {\isasymlongrightarrow}\ disjoint{\isadigit{3}}\ {\isacharparenleft}{\kern0pt}{\isacharquery}{\kern0pt}mod\ A\ p{\isacharparenright}{\kern0pt}{\isachardoublequoteclose}\isanewline
\ \ \ \ \isacommand{by}\isamarkupfalse%
\ simp\isanewline
\ \ \isacommand{from}\isamarkupfalse%
\ {\isadigit{0}}\ {\isadigit{1}}\isanewline
\ \ \isacommand{have}\isamarkupfalse%
\isanewline
\ \ \ \ {\isachardoublequoteopen}{\isasymforall}\ A\ p{\isachardot}{\kern0pt}\ finite{\isacharunderscore}{\kern0pt}profile\ A\ p\ {\isasymlongrightarrow}\ well{\isacharunderscore}{\kern0pt}formed\ A\ {\isacharparenleft}{\kern0pt}{\isacharquery}{\kern0pt}mod\ A\ p{\isacharparenright}{\kern0pt}{\isachardoublequoteclose}\isanewline
\ \ \ \ \isacommand{by}\isamarkupfalse%
\ simp\isanewline
\ \ \isacommand{hence}\isamarkupfalse%
\isanewline
\ \ \ \ {\isachardoublequoteopen}{\isasymforall}\ A\ p{\isachardot}{\kern0pt}\ finite{\isacharunderscore}{\kern0pt}profile\ A\ p\ {\isasymlongrightarrow}\ well{\isacharunderscore}{\kern0pt}formed\ A\ {\isacharparenleft}{\kern0pt}{\isacharquery}{\kern0pt}mod\ A\ p{\isacharparenright}{\kern0pt}{\isachardoublequoteclose}\isanewline
\ \ \ \ \isacommand{by}\isamarkupfalse%
\ simp\isanewline
\ \ \isacommand{thus}\isamarkupfalse%
\ {\isacharquery}{\kern0pt}thesis\isanewline
\ \ \ \ \isacommand{using}\isamarkupfalse%
\ electoral{\isacharunderscore}{\kern0pt}modI\isanewline
\ \ \ \ \isacommand{by}\isamarkupfalse%
\ metis\isanewline
\isacommand{qed}\isamarkupfalse%
%
\endisatagproof
{\isafoldproof}%
%
\isadelimproof
\isanewline
%
\endisadelimproof
%
\isadelimtheory
\isanewline
%
\endisadelimtheory
%
\isatagtheory
\isacommand{end}\isamarkupfalse%
%
\endisatagtheory
{\isafoldtheory}%
%
\isadelimtheory
%
\endisadelimtheory
%
\end{isabellebody}%
\endinput
%:%file=~/Documents/Studies/VotingRuleGenerator/virage/src/test/resources/verifiedVotingRuleConstruction/theories/Compositional_Framework/Components/Basic_Modules/Pass_Module.thy%:%
%:%6=3%:%
%:%11=4%:%
%:%12=5%:%
%:%14=8%:%
%:%30=10%:%
%:%31=10%:%
%:%32=11%:%
%:%33=12%:%
%:%42=15%:%
%:%43=16%:%
%:%44=17%:%
%:%45=18%:%
%:%46=19%:%
%:%55=21%:%
%:%65=23%:%
%:%66=23%:%
%:%67=24%:%
%:%77=29%:%
%:%87=31%:%
%:%88=31%:%
%:%89=32%:%
%:%90=33%:%
%:%97=34%:%
%:%98=34%:%
%:%99=35%:%
%:%100=35%:%
%:%101=36%:%
%:%102=36%:%
%:%103=37%:%
%:%105=39%:%
%:%106=40%:%
%:%107=40%:%
%:%108=41%:%
%:%109=41%:%
%:%110=42%:%
%:%111=42%:%
%:%112=43%:%
%:%114=45%:%
%:%115=46%:%
%:%116=46%:%
%:%117=47%:%
%:%118=47%:%
%:%119=48%:%
%:%120=49%:%
%:%121=49%:%
%:%122=50%:%
%:%123=50%:%
%:%124=51%:%
%:%126=53%:%
%:%127=54%:%
%:%128=54%:%
%:%129=55%:%
%:%130=55%:%
%:%131=56%:%
%:%133=58%:%
%:%134=59%:%
%:%135=59%:%
%:%136=60%:%
%:%137=60%:%
%:%138=61%:%
%:%139=62%:%
%:%140=62%:%
%:%141=63%:%
%:%142=63%:%
%:%143=64%:%
%:%144=64%:%
%:%145=65%:%
%:%146=66%:%
%:%147=66%:%
%:%148=67%:%
%:%149=67%:%
%:%150=68%:%
%:%151=69%:%
%:%152=69%:%
%:%153=70%:%
%:%154=70%:%
%:%155=71%:%
%:%156=71%:%
%:%157=72%:%
%:%158=72%:%
%:%159=73%:%
%:%165=73%:%
%:%170=74%:%
%:%175=75%:%
%
\begin{isabellebody}%
\setisabellecontext{Elect{\isacharunderscore}{\kern0pt}Module}%
%
\isadelimdocument
\isanewline
%
\endisadelimdocument
%
\isatagdocument
\isanewline
\isanewline
%
\isamarkupsection{Elect Module%
}
\isamarkuptrue%
%
\endisatagdocument
{\isafolddocument}%
%
\isadelimdocument
%
\endisadelimdocument
%
\isadelimtheory
%
\endisadelimtheory
%
\isatagtheory
\isacommand{theory}\isamarkupfalse%
\ Elect{\isacharunderscore}{\kern0pt}Module\isanewline
\ \ \isakeyword{imports}\ {\isachardoublequoteopen}{\isachardot}{\kern0pt}{\isachardot}{\kern0pt}{\isacharslash}{\kern0pt}Electoral{\isacharunderscore}{\kern0pt}Module{\isachardoublequoteclose}\isanewline
\isakeyword{begin}%
\endisatagtheory
{\isafoldtheory}%
%
\isadelimtheory
%
\endisadelimtheory
%
\begin{isamarkuptext}%
The elect module is not concerned about the voter's ballots, and
just elects all alternatives. It is primarily used in sequence after
an electoral module that only defers alternatives to finalize the decision,
thereby inducing a proper voting rule in the social choice sense.%
\end{isamarkuptext}\isamarkuptrue%
%
\isadelimdocument
%
\endisadelimdocument
%
\isatagdocument
%
\isamarkupsubsection{Definition%
}
\isamarkuptrue%
%
\endisatagdocument
{\isafolddocument}%
%
\isadelimdocument
%
\endisadelimdocument
\isacommand{fun}\isamarkupfalse%
\ elect{\isacharunderscore}{\kern0pt}module\ {\isacharcolon}{\kern0pt}{\isacharcolon}{\kern0pt}\ {\isachardoublequoteopen}{\isacharprime}{\kern0pt}a\ Electoral{\isacharunderscore}{\kern0pt}Module{\isachardoublequoteclose}\ \isakeyword{where}\isanewline
\ \ {\isachardoublequoteopen}elect{\isacharunderscore}{\kern0pt}module\ A\ p\ {\isacharequal}{\kern0pt}\ {\isacharparenleft}{\kern0pt}A{\isacharcomma}{\kern0pt}\ {\isacharbraceleft}{\kern0pt}{\isacharbraceright}{\kern0pt}{\isacharcomma}{\kern0pt}\ {\isacharbraceleft}{\kern0pt}{\isacharbraceright}{\kern0pt}{\isacharparenright}{\kern0pt}{\isachardoublequoteclose}%
\isadelimdocument
%
\endisadelimdocument
%
\isatagdocument
%
\isamarkupsubsection{Soundness%
}
\isamarkuptrue%
%
\endisatagdocument
{\isafolddocument}%
%
\isadelimdocument
%
\endisadelimdocument
\isacommand{theorem}\isamarkupfalse%
\ elect{\isacharunderscore}{\kern0pt}mod{\isacharunderscore}{\kern0pt}sound{\isacharbrackleft}{\kern0pt}simp{\isacharbrackright}{\kern0pt}{\isacharcolon}{\kern0pt}\ {\isachardoublequoteopen}electoral{\isacharunderscore}{\kern0pt}module\ elect{\isacharunderscore}{\kern0pt}module{\isachardoublequoteclose}\isanewline
%
\isadelimproof
\ \ %
\endisadelimproof
%
\isatagproof
\isacommand{unfolding}\isamarkupfalse%
\ electoral{\isacharunderscore}{\kern0pt}module{\isacharunderscore}{\kern0pt}def\isanewline
\ \ \isacommand{by}\isamarkupfalse%
\ simp%
\endisatagproof
{\isafoldproof}%
%
\isadelimproof
\isanewline
%
\endisadelimproof
%
\isadelimtheory
\isanewline
%
\endisadelimtheory
%
\isatagtheory
\isacommand{end}\isamarkupfalse%
%
\endisatagtheory
{\isafoldtheory}%
%
\isadelimtheory
%
\endisadelimtheory
%
\end{isabellebody}%
\endinput
%:%file=~/Documents/Studies/VotingRuleGenerator/virage/src/test/resources/verifiedVotingRuleConstruction/theories/Compositional_Framework/Components/Basic_Modules/Elect_Module.thy%:%
%:%6=3%:%
%:%11=4%:%
%:%12=5%:%
%:%14=8%:%
%:%30=10%:%
%:%31=10%:%
%:%32=11%:%
%:%33=12%:%
%:%42=15%:%
%:%43=16%:%
%:%44=17%:%
%:%45=18%:%
%:%54=20%:%
%:%64=22%:%
%:%65=22%:%
%:%66=23%:%
%:%73=25%:%
%:%83=27%:%
%:%84=27%:%
%:%87=28%:%
%:%91=28%:%
%:%92=28%:%
%:%93=29%:%
%:%94=29%:%
%:%99=29%:%
%:%104=30%:%
%:%109=31%:%
%
\begin{isabellebody}%
\setisabellecontext{Elimination{\isacharunderscore}{\kern0pt}Module}%
%
\isadelimdocument
\isanewline
%
\endisadelimdocument
%
\isatagdocument
\isanewline
%
\isamarkupsection{Elimination Module%
}
\isamarkuptrue%
%
\endisatagdocument
{\isafolddocument}%
%
\isadelimdocument
%
\endisadelimdocument
%
\isadelimtheory
%
\endisadelimtheory
%
\isatagtheory
\isacommand{theory}\isamarkupfalse%
\ Elimination{\isacharunderscore}{\kern0pt}Module\isanewline
\ \ \isakeyword{imports}\ Evaluation{\isacharunderscore}{\kern0pt}Function\isanewline
\ \ \ \ \ \ \ \ \ \ Electoral{\isacharunderscore}{\kern0pt}Module\isanewline
\isakeyword{begin}%
\endisatagtheory
{\isafoldtheory}%
%
\isadelimtheory
%
\endisadelimtheory
%
\begin{isamarkuptext}%
This is the elimination module. It rejects a set of alternatives only if these
are not all alternatives. The alternatives potentially to be rejected are put
in a so-called elimination set. These are all alternatives that score below
a preset threshold value that depends on the specific voting rule.%
\end{isamarkuptext}\isamarkuptrue%
%
\isadelimdocument
%
\endisadelimdocument
%
\isatagdocument
%
\isamarkupsubsection{Definition%
}
\isamarkuptrue%
%
\endisatagdocument
{\isafolddocument}%
%
\isadelimdocument
%
\endisadelimdocument
\isacommand{type{\isacharunderscore}{\kern0pt}synonym}\isamarkupfalse%
\ Threshold{\isacharunderscore}{\kern0pt}Value\ {\isacharequal}{\kern0pt}\ nat\isanewline
\isanewline
\isacommand{type{\isacharunderscore}{\kern0pt}synonym}\isamarkupfalse%
\ {\isacharprime}{\kern0pt}a\ Electoral{\isacharunderscore}{\kern0pt}Set\ {\isacharequal}{\kern0pt}\ {\isachardoublequoteopen}{\isacharprime}{\kern0pt}a\ set\ {\isasymRightarrow}\ {\isacharprime}{\kern0pt}a\ Profile\ {\isasymRightarrow}\ {\isacharprime}{\kern0pt}a\ set{\isachardoublequoteclose}\isanewline
\isanewline
\isacommand{fun}\isamarkupfalse%
\ elimination{\isacharunderscore}{\kern0pt}set\ {\isacharcolon}{\kern0pt}{\isacharcolon}{\kern0pt}\ {\isachardoublequoteopen}{\isacharprime}{\kern0pt}a\ Evaluation{\isacharunderscore}{\kern0pt}Function\ {\isasymRightarrow}\ Threshold{\isacharunderscore}{\kern0pt}Value\ {\isasymRightarrow}\isanewline
\ \ \ \ \ \ \ \ \ \ \ \ \ \ \ \ \ \ \ \ \ \ \ \ \ \ \ \ {\isacharparenleft}{\kern0pt}nat\ {\isasymRightarrow}\ Threshold{\isacharunderscore}{\kern0pt}Value\ {\isasymRightarrow}\ bool{\isacharparenright}{\kern0pt}\ {\isasymRightarrow}\isanewline
\ \ \ \ \ \ \ \ \ \ \ \ \ \ \ \ \ \ \ \ \ \ \ \ \ \ \ \ \ \ {\isacharprime}{\kern0pt}a\ Electoral{\isacharunderscore}{\kern0pt}Set{\isachardoublequoteclose}\ \isakeyword{where}\isanewline
\ {\isachardoublequoteopen}elimination{\isacharunderscore}{\kern0pt}set\ e\ t\ r\ A\ p\ {\isacharequal}{\kern0pt}\ {\isacharbraceleft}{\kern0pt}a\ {\isasymin}\ A\ {\isachardot}{\kern0pt}\ r\ {\isacharparenleft}{\kern0pt}e\ a\ A\ p{\isacharparenright}{\kern0pt}\ t\ {\isacharbraceright}{\kern0pt}{\isachardoublequoteclose}\isanewline
\isanewline
\isacommand{fun}\isamarkupfalse%
\ elimination{\isacharunderscore}{\kern0pt}module\ {\isacharcolon}{\kern0pt}{\isacharcolon}{\kern0pt}\ {\isachardoublequoteopen}{\isacharprime}{\kern0pt}a\ Evaluation{\isacharunderscore}{\kern0pt}Function\ {\isasymRightarrow}\ Threshold{\isacharunderscore}{\kern0pt}Value\ {\isasymRightarrow}\isanewline
\ \ \ \ \ \ \ \ {\isacharparenleft}{\kern0pt}nat\ {\isasymRightarrow}\ nat\ {\isasymRightarrow}\ bool{\isacharparenright}{\kern0pt}\ {\isasymRightarrow}\ {\isacharprime}{\kern0pt}a\ Electoral{\isacharunderscore}{\kern0pt}Module{\isachardoublequoteclose}\ \isakeyword{where}\isanewline
\ \ {\isachardoublequoteopen}elimination{\isacharunderscore}{\kern0pt}module\ e\ t\ r\ A\ p\ {\isacharequal}{\kern0pt}\isanewline
\ \ \ \ \ \ {\isacharparenleft}{\kern0pt}if\ {\isacharparenleft}{\kern0pt}elimination{\isacharunderscore}{\kern0pt}set\ e\ t\ r\ A\ p{\isacharparenright}{\kern0pt}\ {\isasymnoteq}\ A\isanewline
\ \ \ \ \ \ \ \ then\ {\isacharparenleft}{\kern0pt}{\isacharbraceleft}{\kern0pt}{\isacharbraceright}{\kern0pt}{\isacharcomma}{\kern0pt}\ {\isacharparenleft}{\kern0pt}elimination{\isacharunderscore}{\kern0pt}set\ e\ t\ r\ A\ p{\isacharparenright}{\kern0pt}{\isacharcomma}{\kern0pt}\ A\ {\isacharminus}{\kern0pt}\ {\isacharparenleft}{\kern0pt}elimination{\isacharunderscore}{\kern0pt}set\ e\ t\ r\ A\ p{\isacharparenright}{\kern0pt}{\isacharparenright}{\kern0pt}\isanewline
\ \ \ \ \ \ \ \ else\ {\isacharparenleft}{\kern0pt}{\isacharbraceleft}{\kern0pt}{\isacharbraceright}{\kern0pt}{\isacharcomma}{\kern0pt}{\isacharbraceleft}{\kern0pt}{\isacharbraceright}{\kern0pt}{\isacharcomma}{\kern0pt}A{\isacharparenright}{\kern0pt}{\isacharparenright}{\kern0pt}{\isachardoublequoteclose}%
\isadelimdocument
%
\endisadelimdocument
%
\isatagdocument
%
\isamarkupsubsection{Common Eliminators%
}
\isamarkuptrue%
%
\endisatagdocument
{\isafolddocument}%
%
\isadelimdocument
%
\endisadelimdocument
\isacommand{fun}\isamarkupfalse%
\ less{\isacharunderscore}{\kern0pt}eliminator\ {\isacharcolon}{\kern0pt}{\isacharcolon}{\kern0pt}\ {\isachardoublequoteopen}{\isacharprime}{\kern0pt}a\ Evaluation{\isacharunderscore}{\kern0pt}Function\ {\isasymRightarrow}\ Threshold{\isacharunderscore}{\kern0pt}Value\ {\isasymRightarrow}\isanewline
\ \ \ \ \ \ \ \ \ \ \ \ \ \ \ \ \ \ \ \ \ \ \ \ \ \ \ \ {\isacharprime}{\kern0pt}a\ Electoral{\isacharunderscore}{\kern0pt}Module{\isachardoublequoteclose}\ \isakeyword{where}\isanewline
\ \ {\isachardoublequoteopen}less{\isacharunderscore}{\kern0pt}eliminator\ e\ t\ A\ p\ {\isacharequal}{\kern0pt}\ elimination{\isacharunderscore}{\kern0pt}module\ e\ t\ {\isacharparenleft}{\kern0pt}{\isacharless}{\kern0pt}{\isacharparenright}{\kern0pt}\ A\ p{\isachardoublequoteclose}\isanewline
\isanewline
\isacommand{fun}\isamarkupfalse%
\ max{\isacharunderscore}{\kern0pt}eliminator\ {\isacharcolon}{\kern0pt}{\isacharcolon}{\kern0pt}\ {\isachardoublequoteopen}{\isacharprime}{\kern0pt}a\ Evaluation{\isacharunderscore}{\kern0pt}Function\ {\isasymRightarrow}\ {\isacharprime}{\kern0pt}a\ Electoral{\isacharunderscore}{\kern0pt}Module{\isachardoublequoteclose}\ \isakeyword{where}\isanewline
\ \ {\isachardoublequoteopen}max{\isacharunderscore}{\kern0pt}eliminator\ e\ A\ p\ {\isacharequal}{\kern0pt}\isanewline
\ \ \ \ less{\isacharunderscore}{\kern0pt}eliminator\ e\ {\isacharparenleft}{\kern0pt}Max\ {\isacharbraceleft}{\kern0pt}e\ x\ A\ p\ {\isacharbar}{\kern0pt}\ x{\isachardot}{\kern0pt}\ x\ {\isasymin}\ A{\isacharbraceright}{\kern0pt}{\isacharparenright}{\kern0pt}\ A\ p{\isachardoublequoteclose}\isanewline
\isanewline
\isacommand{fun}\isamarkupfalse%
\ leq{\isacharunderscore}{\kern0pt}eliminator\ {\isacharcolon}{\kern0pt}{\isacharcolon}{\kern0pt}\ {\isachardoublequoteopen}{\isacharprime}{\kern0pt}a\ Evaluation{\isacharunderscore}{\kern0pt}Function\ {\isasymRightarrow}\ Threshold{\isacharunderscore}{\kern0pt}Value\ {\isasymRightarrow}\isanewline
\ \ \ \ \ \ \ \ \ \ \ \ \ \ \ \ \ \ \ \ \ \ \ \ \ \ \ \ {\isacharprime}{\kern0pt}a\ Electoral{\isacharunderscore}{\kern0pt}Module{\isachardoublequoteclose}\ \isakeyword{where}\isanewline
\ \ {\isachardoublequoteopen}leq{\isacharunderscore}{\kern0pt}eliminator\ e\ t\ A\ p\ {\isacharequal}{\kern0pt}\ elimination{\isacharunderscore}{\kern0pt}module\ e\ t\ {\isacharparenleft}{\kern0pt}{\isasymle}{\isacharparenright}{\kern0pt}\ A\ p{\isachardoublequoteclose}\isanewline
\isanewline
\isacommand{fun}\isamarkupfalse%
\ min{\isacharunderscore}{\kern0pt}eliminator\ {\isacharcolon}{\kern0pt}{\isacharcolon}{\kern0pt}\ {\isachardoublequoteopen}{\isacharprime}{\kern0pt}a\ Evaluation{\isacharunderscore}{\kern0pt}Function\ {\isasymRightarrow}\ {\isacharprime}{\kern0pt}a\ Electoral{\isacharunderscore}{\kern0pt}Module{\isachardoublequoteclose}\ \isakeyword{where}\isanewline
\ \ {\isachardoublequoteopen}min{\isacharunderscore}{\kern0pt}eliminator\ e\ A\ p\ {\isacharequal}{\kern0pt}\isanewline
\ \ \ \ leq{\isacharunderscore}{\kern0pt}eliminator\ e\ {\isacharparenleft}{\kern0pt}Min\ {\isacharbraceleft}{\kern0pt}e\ x\ A\ p\ {\isacharbar}{\kern0pt}\ x{\isachardot}{\kern0pt}\ x\ {\isasymin}\ A{\isacharbraceright}{\kern0pt}{\isacharparenright}{\kern0pt}\ A\ p{\isachardoublequoteclose}\isanewline
\isanewline
\isacommand{fun}\isamarkupfalse%
\ average\ {\isacharcolon}{\kern0pt}{\isacharcolon}{\kern0pt}\ {\isachardoublequoteopen}{\isacharprime}{\kern0pt}a\ Evaluation{\isacharunderscore}{\kern0pt}Function\ {\isasymRightarrow}\ {\isacharprime}{\kern0pt}a\ set\ {\isasymRightarrow}\ {\isacharprime}{\kern0pt}a\ Profile\ {\isasymRightarrow}\isanewline
\ \ \ \ \ \ \ \ \ \ \ \ \ \ \ \ \ \ \ \ Threshold{\isacharunderscore}{\kern0pt}Value{\isachardoublequoteclose}\ \isakeyword{where}\isanewline
\ \ {\isachardoublequoteopen}average\ e\ A\ p\ {\isacharequal}{\kern0pt}\ {\isacharparenleft}{\kern0pt}{\isasymSum}x\ {\isasymin}\ A{\isachardot}{\kern0pt}\ e\ x\ A\ p{\isacharparenright}{\kern0pt}\ div\ {\isacharparenleft}{\kern0pt}card\ A{\isacharparenright}{\kern0pt}{\isachardoublequoteclose}\isanewline
\isanewline
\isacommand{fun}\isamarkupfalse%
\ less{\isacharunderscore}{\kern0pt}average{\isacharunderscore}{\kern0pt}eliminator\ {\isacharcolon}{\kern0pt}{\isacharcolon}{\kern0pt}\ {\isachardoublequoteopen}{\isacharprime}{\kern0pt}a\ Evaluation{\isacharunderscore}{\kern0pt}Function\ {\isasymRightarrow}\isanewline
\ \ \ \ \ \ \ \ \ \ \ \ \ \ \ \ \ \ \ \ \ \ \ \ \ \ \ \ \ \ \ \ {\isacharprime}{\kern0pt}a\ Electoral{\isacharunderscore}{\kern0pt}Module{\isachardoublequoteclose}\ \isakeyword{where}\isanewline
\ \ {\isachardoublequoteopen}less{\isacharunderscore}{\kern0pt}average{\isacharunderscore}{\kern0pt}eliminator\ e\ A\ p\ {\isacharequal}{\kern0pt}\ less{\isacharunderscore}{\kern0pt}eliminator\ e\ {\isacharparenleft}{\kern0pt}average\ e\ A\ p{\isacharparenright}{\kern0pt}\ A\ p{\isachardoublequoteclose}\isanewline
\isanewline
\isacommand{fun}\isamarkupfalse%
\ leq{\isacharunderscore}{\kern0pt}average{\isacharunderscore}{\kern0pt}eliminator\ {\isacharcolon}{\kern0pt}{\isacharcolon}{\kern0pt}\ {\isachardoublequoteopen}{\isacharprime}{\kern0pt}a\ Evaluation{\isacharunderscore}{\kern0pt}Function\ {\isasymRightarrow}\isanewline
\ \ \ \ \ \ \ \ \ \ \ \ \ \ \ \ \ \ \ \ \ \ \ \ \ \ \ \ \ \ \ \ {\isacharprime}{\kern0pt}a\ Electoral{\isacharunderscore}{\kern0pt}Module{\isachardoublequoteclose}\ \isakeyword{where}\isanewline
\ \ {\isachardoublequoteopen}leq{\isacharunderscore}{\kern0pt}average{\isacharunderscore}{\kern0pt}eliminator\ e\ A\ p\ {\isacharequal}{\kern0pt}\ leq{\isacharunderscore}{\kern0pt}eliminator\ e\ {\isacharparenleft}{\kern0pt}average\ e\ A\ p{\isacharparenright}{\kern0pt}\ A\ p{\isachardoublequoteclose}%
\isadelimdocument
%
\endisadelimdocument
%
\isatagdocument
%
\isamarkupsubsection{Soundness%
}
\isamarkuptrue%
%
\endisatagdocument
{\isafolddocument}%
%
\isadelimdocument
%
\endisadelimdocument
\isacommand{lemma}\isamarkupfalse%
\ elim{\isacharunderscore}{\kern0pt}mod{\isacharunderscore}{\kern0pt}sound{\isacharbrackleft}{\kern0pt}simp{\isacharbrackright}{\kern0pt}{\isacharcolon}{\kern0pt}\ {\isachardoublequoteopen}electoral{\isacharunderscore}{\kern0pt}module\ {\isacharparenleft}{\kern0pt}elimination{\isacharunderscore}{\kern0pt}module\ e\ t\ r{\isacharparenright}{\kern0pt}{\isachardoublequoteclose}\isanewline
%
\isadelimproof
%
\endisadelimproof
%
\isatagproof
\isacommand{proof}\isamarkupfalse%
\ {\isacharparenleft}{\kern0pt}unfold\ electoral{\isacharunderscore}{\kern0pt}module{\isacharunderscore}{\kern0pt}def{\isacharcomma}{\kern0pt}\ safe{\isacharparenright}{\kern0pt}\isanewline
\ \ \isacommand{fix}\isamarkupfalse%
\isanewline
\ \ \ \ A\ {\isacharcolon}{\kern0pt}{\isacharcolon}{\kern0pt}\ {\isachardoublequoteopen}{\isacharprime}{\kern0pt}a\ set{\isachardoublequoteclose}\ \isakeyword{and}\isanewline
\ \ \ \ p\ {\isacharcolon}{\kern0pt}{\isacharcolon}{\kern0pt}\ {\isachardoublequoteopen}{\isacharprime}{\kern0pt}a\ Profile{\isachardoublequoteclose}\isanewline
\ \ \isacommand{have}\isamarkupfalse%
\ {\isachardoublequoteopen}set{\isacharunderscore}{\kern0pt}equals{\isacharunderscore}{\kern0pt}partition\ A\ {\isacharparenleft}{\kern0pt}elimination{\isacharunderscore}{\kern0pt}module\ e\ t\ r\ A\ p{\isacharparenright}{\kern0pt}{\isachardoublequoteclose}\isanewline
\ \ \ \ \isacommand{by}\isamarkupfalse%
\ auto\isanewline
\ \ \isacommand{thus}\isamarkupfalse%
\ {\isachardoublequoteopen}well{\isacharunderscore}{\kern0pt}formed\ A\ {\isacharparenleft}{\kern0pt}elimination{\isacharunderscore}{\kern0pt}module\ e\ t\ r\ A\ p{\isacharparenright}{\kern0pt}{\isachardoublequoteclose}\isanewline
\ \ \ \ \isacommand{by}\isamarkupfalse%
\ simp\isanewline
\isacommand{qed}\isamarkupfalse%
%
\endisatagproof
{\isafoldproof}%
%
\isadelimproof
\isanewline
%
\endisadelimproof
\isanewline
\isacommand{lemma}\isamarkupfalse%
\ less{\isacharunderscore}{\kern0pt}elim{\isacharunderscore}{\kern0pt}sound{\isacharbrackleft}{\kern0pt}simp{\isacharbrackright}{\kern0pt}{\isacharcolon}{\kern0pt}\ {\isachardoublequoteopen}electoral{\isacharunderscore}{\kern0pt}module\ {\isacharparenleft}{\kern0pt}less{\isacharunderscore}{\kern0pt}eliminator\ e\ t{\isacharparenright}{\kern0pt}{\isachardoublequoteclose}\isanewline
%
\isadelimproof
\ \ %
\endisadelimproof
%
\isatagproof
\isacommand{unfolding}\isamarkupfalse%
\ electoral{\isacharunderscore}{\kern0pt}module{\isacharunderscore}{\kern0pt}def\isanewline
\isacommand{proof}\isamarkupfalse%
\ {\isacharparenleft}{\kern0pt}safe{\isacharcomma}{\kern0pt}\ simp{\isacharparenright}{\kern0pt}\isanewline
\ \ \isacommand{fix}\isamarkupfalse%
\isanewline
\ \ \ \ A\ {\isacharcolon}{\kern0pt}{\isacharcolon}{\kern0pt}\ {\isachardoublequoteopen}{\isacharprime}{\kern0pt}a\ set{\isachardoublequoteclose}\ \isakeyword{and}\isanewline
\ \ \ \ p\ {\isacharcolon}{\kern0pt}{\isacharcolon}{\kern0pt}\ {\isachardoublequoteopen}{\isacharprime}{\kern0pt}a\ Profile{\isachardoublequoteclose}\isanewline
\ \ \isacommand{show}\isamarkupfalse%
\isanewline
\ \ \ \ {\isachardoublequoteopen}{\isacharbraceleft}{\kern0pt}a\ {\isasymin}\ A{\isachardot}{\kern0pt}\ e\ a\ A\ p\ {\isacharless}{\kern0pt}\ t{\isacharbraceright}{\kern0pt}\ {\isasymnoteq}\ A\ {\isasymlongrightarrow}\isanewline
\ \ \ \ \ \ {\isacharbraceleft}{\kern0pt}a\ {\isasymin}\ A{\isachardot}{\kern0pt}\ e\ a\ A\ p\ {\isacharless}{\kern0pt}\ t{\isacharbraceright}{\kern0pt}\ {\isasymunion}\ A\ {\isacharequal}{\kern0pt}\ A{\isachardoublequoteclose}\isanewline
\ \ \ \ \isacommand{by}\isamarkupfalse%
\ safe\isanewline
\isacommand{qed}\isamarkupfalse%
%
\endisatagproof
{\isafoldproof}%
%
\isadelimproof
\isanewline
%
\endisadelimproof
\isanewline
\isacommand{lemma}\isamarkupfalse%
\ leq{\isacharunderscore}{\kern0pt}elim{\isacharunderscore}{\kern0pt}sound{\isacharbrackleft}{\kern0pt}simp{\isacharbrackright}{\kern0pt}{\isacharcolon}{\kern0pt}\ {\isachardoublequoteopen}electoral{\isacharunderscore}{\kern0pt}module\ {\isacharparenleft}{\kern0pt}leq{\isacharunderscore}{\kern0pt}eliminator\ e\ t{\isacharparenright}{\kern0pt}{\isachardoublequoteclose}\isanewline
%
\isadelimproof
\ \ %
\endisadelimproof
%
\isatagproof
\isacommand{unfolding}\isamarkupfalse%
\ electoral{\isacharunderscore}{\kern0pt}module{\isacharunderscore}{\kern0pt}def\isanewline
\isacommand{proof}\isamarkupfalse%
\ {\isacharparenleft}{\kern0pt}safe{\isacharcomma}{\kern0pt}\ simp{\isacharparenright}{\kern0pt}\isanewline
\ \ \isacommand{fix}\isamarkupfalse%
\isanewline
\ \ \ \ A\ {\isacharcolon}{\kern0pt}{\isacharcolon}{\kern0pt}\ {\isachardoublequoteopen}{\isacharprime}{\kern0pt}a\ set{\isachardoublequoteclose}\ \isakeyword{and}\isanewline
\ \ \ \ p\ {\isacharcolon}{\kern0pt}{\isacharcolon}{\kern0pt}\ {\isachardoublequoteopen}{\isacharprime}{\kern0pt}a\ Profile{\isachardoublequoteclose}\isanewline
\ \ \isacommand{show}\isamarkupfalse%
\isanewline
\ \ \ \ {\isachardoublequoteopen}{\isacharbraceleft}{\kern0pt}a\ {\isasymin}\ A{\isachardot}{\kern0pt}\ e\ a\ A\ p\ {\isasymle}\ t{\isacharbraceright}{\kern0pt}\ {\isasymnoteq}\ A\ {\isasymlongrightarrow}\isanewline
\ \ \ \ \ \ {\isacharbraceleft}{\kern0pt}a\ {\isasymin}\ A{\isachardot}{\kern0pt}\ e\ a\ A\ p\ {\isasymle}\ t{\isacharbraceright}{\kern0pt}\ {\isasymunion}\ A\ {\isacharequal}{\kern0pt}\ A{\isachardoublequoteclose}\isanewline
\ \ \ \ \isacommand{by}\isamarkupfalse%
\ safe\isanewline
\isacommand{qed}\isamarkupfalse%
%
\endisatagproof
{\isafoldproof}%
%
\isadelimproof
\isanewline
%
\endisadelimproof
\isanewline
\isacommand{lemma}\isamarkupfalse%
\ max{\isacharunderscore}{\kern0pt}elim{\isacharunderscore}{\kern0pt}sound{\isacharbrackleft}{\kern0pt}simp{\isacharbrackright}{\kern0pt}{\isacharcolon}{\kern0pt}\ {\isachardoublequoteopen}electoral{\isacharunderscore}{\kern0pt}module\ {\isacharparenleft}{\kern0pt}max{\isacharunderscore}{\kern0pt}eliminator\ e{\isacharparenright}{\kern0pt}{\isachardoublequoteclose}\isanewline
%
\isadelimproof
\ \ %
\endisadelimproof
%
\isatagproof
\isacommand{unfolding}\isamarkupfalse%
\ electoral{\isacharunderscore}{\kern0pt}module{\isacharunderscore}{\kern0pt}def\isanewline
\isacommand{proof}\isamarkupfalse%
\ {\isacharparenleft}{\kern0pt}safe{\isacharcomma}{\kern0pt}\ simp{\isacharparenright}{\kern0pt}\isanewline
\ \ \isacommand{fix}\isamarkupfalse%
\isanewline
\ \ \ \ A\ {\isacharcolon}{\kern0pt}{\isacharcolon}{\kern0pt}\ {\isachardoublequoteopen}{\isacharprime}{\kern0pt}a\ set{\isachardoublequoteclose}\ \isakeyword{and}\isanewline
\ \ \ \ p\ {\isacharcolon}{\kern0pt}{\isacharcolon}{\kern0pt}\ {\isachardoublequoteopen}{\isacharprime}{\kern0pt}a\ Profile{\isachardoublequoteclose}\isanewline
\ \ \isacommand{show}\isamarkupfalse%
\isanewline
\ \ \ \ {\isachardoublequoteopen}{\isacharbraceleft}{\kern0pt}a\ {\isasymin}\ A{\isachardot}{\kern0pt}\ e\ a\ A\ p\ {\isacharless}{\kern0pt}\ Max\ {\isacharbraceleft}{\kern0pt}e\ x\ A\ p\ {\isacharbar}{\kern0pt}x{\isachardot}{\kern0pt}\ x\ {\isasymin}\ A{\isacharbraceright}{\kern0pt}{\isacharbraceright}{\kern0pt}\ {\isasymnoteq}\ A\ {\isasymlongrightarrow}\isanewline
\ \ \ \ \ \ {\isacharbraceleft}{\kern0pt}a\ {\isasymin}\ A{\isachardot}{\kern0pt}\ e\ a\ A\ p\ {\isacharless}{\kern0pt}\ Max\ {\isacharbraceleft}{\kern0pt}e\ x\ A\ p\ {\isacharbar}{\kern0pt}x{\isachardot}{\kern0pt}\ x\ {\isasymin}\ A{\isacharbraceright}{\kern0pt}{\isacharbraceright}{\kern0pt}\ {\isasymunion}\ A\ {\isacharequal}{\kern0pt}\ A{\isachardoublequoteclose}\isanewline
\ \ \ \ \isacommand{by}\isamarkupfalse%
\ safe\isanewline
\isacommand{qed}\isamarkupfalse%
%
\endisatagproof
{\isafoldproof}%
%
\isadelimproof
\isanewline
%
\endisadelimproof
\isanewline
\isacommand{lemma}\isamarkupfalse%
\ min{\isacharunderscore}{\kern0pt}elim{\isacharunderscore}{\kern0pt}sound{\isacharbrackleft}{\kern0pt}simp{\isacharbrackright}{\kern0pt}{\isacharcolon}{\kern0pt}\ {\isachardoublequoteopen}electoral{\isacharunderscore}{\kern0pt}module\ {\isacharparenleft}{\kern0pt}min{\isacharunderscore}{\kern0pt}eliminator\ e{\isacharparenright}{\kern0pt}{\isachardoublequoteclose}\isanewline
%
\isadelimproof
\ \ %
\endisadelimproof
%
\isatagproof
\isacommand{unfolding}\isamarkupfalse%
\ electoral{\isacharunderscore}{\kern0pt}module{\isacharunderscore}{\kern0pt}def\isanewline
\isacommand{proof}\isamarkupfalse%
\ {\isacharparenleft}{\kern0pt}safe{\isacharcomma}{\kern0pt}\ simp{\isacharparenright}{\kern0pt}\isanewline
\ \ \isacommand{fix}\isamarkupfalse%
\isanewline
\ \ \ \ A\ {\isacharcolon}{\kern0pt}{\isacharcolon}{\kern0pt}\ {\isachardoublequoteopen}{\isacharprime}{\kern0pt}a\ set{\isachardoublequoteclose}\ \isakeyword{and}\isanewline
\ \ \ \ p\ {\isacharcolon}{\kern0pt}{\isacharcolon}{\kern0pt}\ {\isachardoublequoteopen}{\isacharprime}{\kern0pt}a\ Profile{\isachardoublequoteclose}\isanewline
\ \ \isacommand{show}\isamarkupfalse%
\isanewline
\ \ \ \ {\isachardoublequoteopen}{\isacharbraceleft}{\kern0pt}a\ {\isasymin}\ A{\isachardot}{\kern0pt}\ e\ a\ A\ p\ {\isasymle}\ Min\ {\isacharbraceleft}{\kern0pt}e\ x\ A\ p\ {\isacharbar}{\kern0pt}x{\isachardot}{\kern0pt}\ x\ {\isasymin}\ A{\isacharbraceright}{\kern0pt}{\isacharbraceright}{\kern0pt}\ {\isasymnoteq}\ A\ {\isasymlongrightarrow}\isanewline
\ \ \ \ \ \ {\isacharbraceleft}{\kern0pt}a\ {\isasymin}\ A{\isachardot}{\kern0pt}\ e\ a\ A\ p\ {\isasymle}\ Min\ {\isacharbraceleft}{\kern0pt}e\ x\ A\ p\ {\isacharbar}{\kern0pt}x{\isachardot}{\kern0pt}\ x\ {\isasymin}\ A{\isacharbraceright}{\kern0pt}{\isacharbraceright}{\kern0pt}\ {\isasymunion}\ A\ {\isacharequal}{\kern0pt}\ A{\isachardoublequoteclose}\isanewline
\ \ \ \ \isacommand{by}\isamarkupfalse%
\ safe\isanewline
\isacommand{qed}\isamarkupfalse%
%
\endisatagproof
{\isafoldproof}%
%
\isadelimproof
\isanewline
%
\endisadelimproof
\isanewline
\isacommand{lemma}\isamarkupfalse%
\ less{\isacharunderscore}{\kern0pt}avg{\isacharunderscore}{\kern0pt}elim{\isacharunderscore}{\kern0pt}sound{\isacharbrackleft}{\kern0pt}simp{\isacharbrackright}{\kern0pt}{\isacharcolon}{\kern0pt}\ {\isachardoublequoteopen}electoral{\isacharunderscore}{\kern0pt}module\ {\isacharparenleft}{\kern0pt}less{\isacharunderscore}{\kern0pt}average{\isacharunderscore}{\kern0pt}eliminator\ e{\isacharparenright}{\kern0pt}{\isachardoublequoteclose}\isanewline
%
\isadelimproof
\ \ %
\endisadelimproof
%
\isatagproof
\isacommand{unfolding}\isamarkupfalse%
\ electoral{\isacharunderscore}{\kern0pt}module{\isacharunderscore}{\kern0pt}def\isanewline
\isacommand{proof}\isamarkupfalse%
\ {\isacharparenleft}{\kern0pt}safe{\isacharcomma}{\kern0pt}\ simp{\isacharparenright}{\kern0pt}\isanewline
\ \ \isacommand{fix}\isamarkupfalse%
\isanewline
\ \ \ \ A\ {\isacharcolon}{\kern0pt}{\isacharcolon}{\kern0pt}\ {\isachardoublequoteopen}{\isacharprime}{\kern0pt}a\ set{\isachardoublequoteclose}\ \isakeyword{and}\isanewline
\ \ \ \ p\ {\isacharcolon}{\kern0pt}{\isacharcolon}{\kern0pt}\ {\isachardoublequoteopen}{\isacharprime}{\kern0pt}a\ Profile{\isachardoublequoteclose}\isanewline
\ \ \isacommand{show}\isamarkupfalse%
\isanewline
\ \ \ \ {\isachardoublequoteopen}{\isacharbraceleft}{\kern0pt}a\ {\isasymin}\ A{\isachardot}{\kern0pt}\ e\ a\ A\ p\ {\isacharless}{\kern0pt}\ {\isacharparenleft}{\kern0pt}{\isasymSum}x{\isasymin}A{\isachardot}{\kern0pt}\ e\ x\ A\ p{\isacharparenright}{\kern0pt}\ div\ card\ A{\isacharbraceright}{\kern0pt}\ {\isasymnoteq}\ A\ {\isasymlongrightarrow}\isanewline
\ \ \ \ \ \ {\isacharbraceleft}{\kern0pt}a\ {\isasymin}\ A{\isachardot}{\kern0pt}\ e\ a\ A\ p\ {\isacharless}{\kern0pt}\ {\isacharparenleft}{\kern0pt}{\isasymSum}x{\isasymin}A{\isachardot}{\kern0pt}\ e\ x\ A\ p{\isacharparenright}{\kern0pt}\ div\ card\ A{\isacharbraceright}{\kern0pt}\ {\isasymunion}\ A\ {\isacharequal}{\kern0pt}\ A{\isachardoublequoteclose}\isanewline
\ \ \ \ \isacommand{by}\isamarkupfalse%
\ safe\isanewline
\isacommand{qed}\isamarkupfalse%
%
\endisatagproof
{\isafoldproof}%
%
\isadelimproof
\isanewline
%
\endisadelimproof
\isanewline
\isacommand{lemma}\isamarkupfalse%
\ leq{\isacharunderscore}{\kern0pt}avg{\isacharunderscore}{\kern0pt}elim{\isacharunderscore}{\kern0pt}sound{\isacharbrackleft}{\kern0pt}simp{\isacharbrackright}{\kern0pt}{\isacharcolon}{\kern0pt}\ {\isachardoublequoteopen}electoral{\isacharunderscore}{\kern0pt}module\ {\isacharparenleft}{\kern0pt}leq{\isacharunderscore}{\kern0pt}average{\isacharunderscore}{\kern0pt}eliminator\ e{\isacharparenright}{\kern0pt}{\isachardoublequoteclose}\isanewline
%
\isadelimproof
\ \ %
\endisadelimproof
%
\isatagproof
\isacommand{unfolding}\isamarkupfalse%
\ electoral{\isacharunderscore}{\kern0pt}module{\isacharunderscore}{\kern0pt}def\isanewline
\isacommand{proof}\isamarkupfalse%
\ {\isacharparenleft}{\kern0pt}safe{\isacharcomma}{\kern0pt}\ simp{\isacharparenright}{\kern0pt}\isanewline
\ \ \isacommand{fix}\isamarkupfalse%
\isanewline
\ \ \ \ A\ {\isacharcolon}{\kern0pt}{\isacharcolon}{\kern0pt}\ {\isachardoublequoteopen}{\isacharprime}{\kern0pt}a\ set{\isachardoublequoteclose}\ \isakeyword{and}\isanewline
\ \ \ \ p\ {\isacharcolon}{\kern0pt}{\isacharcolon}{\kern0pt}\ {\isachardoublequoteopen}{\isacharprime}{\kern0pt}a\ Profile{\isachardoublequoteclose}\isanewline
\ \ \isacommand{show}\isamarkupfalse%
\isanewline
\ \ \ \ {\isachardoublequoteopen}{\isacharbraceleft}{\kern0pt}a\ {\isasymin}\ A{\isachardot}{\kern0pt}\ e\ a\ A\ p\ {\isasymle}\ {\isacharparenleft}{\kern0pt}{\isasymSum}x{\isasymin}A{\isachardot}{\kern0pt}\ e\ x\ A\ p{\isacharparenright}{\kern0pt}\ div\ card\ A{\isacharbraceright}{\kern0pt}\ {\isasymnoteq}\ A\ {\isasymlongrightarrow}\isanewline
\ \ \ \ \ \ {\isacharbraceleft}{\kern0pt}a\ {\isasymin}\ A{\isachardot}{\kern0pt}\ e\ a\ A\ p\ {\isasymle}\ {\isacharparenleft}{\kern0pt}{\isasymSum}x{\isasymin}A{\isachardot}{\kern0pt}\ e\ x\ A\ p{\isacharparenright}{\kern0pt}\ div\ card\ A{\isacharbraceright}{\kern0pt}\ {\isasymunion}\ A\ {\isacharequal}{\kern0pt}\ A{\isachardoublequoteclose}\isanewline
\ \ \ \ \isacommand{by}\isamarkupfalse%
\ safe\isanewline
\isacommand{qed}\isamarkupfalse%
%
\endisatagproof
{\isafoldproof}%
%
\isadelimproof
%
\endisadelimproof
%
\isadelimdocument
%
\endisadelimdocument
%
\isatagdocument
%
\isamarkupsubsection{Non-Electing%
}
\isamarkuptrue%
%
\endisatagdocument
{\isafolddocument}%
%
\isadelimdocument
%
\endisadelimdocument
\isacommand{lemma}\isamarkupfalse%
\ elim{\isacharunderscore}{\kern0pt}mod{\isacharunderscore}{\kern0pt}non{\isacharunderscore}{\kern0pt}electing{\isacharcolon}{\kern0pt}\isanewline
\ \ \isakeyword{assumes}\ profile{\isacharcolon}{\kern0pt}\ {\isachardoublequoteopen}finite{\isacharunderscore}{\kern0pt}profile\ A\ p{\isachardoublequoteclose}\isanewline
\ \ \isakeyword{shows}\ {\isachardoublequoteopen}non{\isacharunderscore}{\kern0pt}electing\ {\isacharparenleft}{\kern0pt}elimination{\isacharunderscore}{\kern0pt}module\ e\ t\ r\ {\isacharparenright}{\kern0pt}{\isachardoublequoteclose}\isanewline
%
\isadelimproof
\ \ %
\endisadelimproof
%
\isatagproof
\isacommand{by}\isamarkupfalse%
\ {\isacharparenleft}{\kern0pt}simp\ add{\isacharcolon}{\kern0pt}\ non{\isacharunderscore}{\kern0pt}electing{\isacharunderscore}{\kern0pt}def{\isacharparenright}{\kern0pt}%
\endisatagproof
{\isafoldproof}%
%
\isadelimproof
\isanewline
%
\endisadelimproof
\isanewline
\isacommand{lemma}\isamarkupfalse%
\ less{\isacharunderscore}{\kern0pt}elim{\isacharunderscore}{\kern0pt}non{\isacharunderscore}{\kern0pt}electing{\isacharcolon}{\kern0pt}\isanewline
\ \ \isakeyword{assumes}\ profile{\isacharcolon}{\kern0pt}\ {\isachardoublequoteopen}finite{\isacharunderscore}{\kern0pt}profile\ A\ p{\isachardoublequoteclose}\isanewline
\ \ \isakeyword{shows}\ {\isachardoublequoteopen}non{\isacharunderscore}{\kern0pt}electing\ {\isacharparenleft}{\kern0pt}less{\isacharunderscore}{\kern0pt}eliminator\ e\ t{\isacharparenright}{\kern0pt}{\isachardoublequoteclose}\isanewline
%
\isadelimproof
\ \ %
\endisadelimproof
%
\isatagproof
\isacommand{using}\isamarkupfalse%
\ elim{\isacharunderscore}{\kern0pt}mod{\isacharunderscore}{\kern0pt}non{\isacharunderscore}{\kern0pt}electing\ profile\ less{\isacharunderscore}{\kern0pt}elim{\isacharunderscore}{\kern0pt}sound\isanewline
\ \ \isacommand{by}\isamarkupfalse%
\ {\isacharparenleft}{\kern0pt}simp\ add{\isacharcolon}{\kern0pt}\ non{\isacharunderscore}{\kern0pt}electing{\isacharunderscore}{\kern0pt}def{\isacharparenright}{\kern0pt}%
\endisatagproof
{\isafoldproof}%
%
\isadelimproof
\isanewline
%
\endisadelimproof
\isanewline
\isacommand{lemma}\isamarkupfalse%
\ leq{\isacharunderscore}{\kern0pt}elim{\isacharunderscore}{\kern0pt}non{\isacharunderscore}{\kern0pt}electing{\isacharcolon}{\kern0pt}\isanewline
\ \ \isakeyword{assumes}\ profile{\isacharcolon}{\kern0pt}\ {\isachardoublequoteopen}finite{\isacharunderscore}{\kern0pt}profile\ A\ p{\isachardoublequoteclose}\isanewline
\ \ \isakeyword{shows}\ {\isachardoublequoteopen}non{\isacharunderscore}{\kern0pt}electing\ {\isacharparenleft}{\kern0pt}leq{\isacharunderscore}{\kern0pt}eliminator\ e\ t{\isacharparenright}{\kern0pt}{\isachardoublequoteclose}\isanewline
%
\isadelimproof
%
\endisadelimproof
%
\isatagproof
\isacommand{proof}\isamarkupfalse%
\ {\isacharminus}{\kern0pt}\isanewline
\ \ \isacommand{have}\isamarkupfalse%
\ {\isachardoublequoteopen}non{\isacharunderscore}{\kern0pt}electing\ {\isacharparenleft}{\kern0pt}elimination{\isacharunderscore}{\kern0pt}module\ e\ t\ {\isacharparenleft}{\kern0pt}{\isasymle}{\isacharparenright}{\kern0pt}{\isacharparenright}{\kern0pt}{\isachardoublequoteclose}\isanewline
\ \ \ \ \isacommand{by}\isamarkupfalse%
\ {\isacharparenleft}{\kern0pt}simp\ add{\isacharcolon}{\kern0pt}\ non{\isacharunderscore}{\kern0pt}electing{\isacharunderscore}{\kern0pt}def{\isacharparenright}{\kern0pt}\isanewline
\ \ \isacommand{thus}\isamarkupfalse%
\ {\isacharquery}{\kern0pt}thesis\isanewline
\ \ \ \ \isacommand{by}\isamarkupfalse%
\ {\isacharparenleft}{\kern0pt}simp\ add{\isacharcolon}{\kern0pt}\ non{\isacharunderscore}{\kern0pt}electing{\isacharunderscore}{\kern0pt}def{\isacharparenright}{\kern0pt}\isanewline
\isacommand{qed}\isamarkupfalse%
%
\endisatagproof
{\isafoldproof}%
%
\isadelimproof
\isanewline
%
\endisadelimproof
\isanewline
\isacommand{lemma}\isamarkupfalse%
\ max{\isacharunderscore}{\kern0pt}elim{\isacharunderscore}{\kern0pt}non{\isacharunderscore}{\kern0pt}electing{\isacharcolon}{\kern0pt}\isanewline
\ \ \isakeyword{assumes}\ profile{\isacharcolon}{\kern0pt}\ {\isachardoublequoteopen}finite{\isacharunderscore}{\kern0pt}profile\ A\ p{\isachardoublequoteclose}\isanewline
\ \ \isakeyword{shows}\ {\isachardoublequoteopen}non{\isacharunderscore}{\kern0pt}electing\ {\isacharparenleft}{\kern0pt}max{\isacharunderscore}{\kern0pt}eliminator\ e{\isacharparenright}{\kern0pt}{\isachardoublequoteclose}\isanewline
%
\isadelimproof
%
\endisadelimproof
%
\isatagproof
\isacommand{proof}\isamarkupfalse%
\ {\isacharminus}{\kern0pt}\isanewline
\ \ \isacommand{have}\isamarkupfalse%
\ {\isachardoublequoteopen}non{\isacharunderscore}{\kern0pt}electing\ {\isacharparenleft}{\kern0pt}elimination{\isacharunderscore}{\kern0pt}module\ e\ t\ {\isacharparenleft}{\kern0pt}{\isacharless}{\kern0pt}{\isacharparenright}{\kern0pt}{\isacharparenright}{\kern0pt}{\isachardoublequoteclose}\isanewline
\ \ \ \ \isacommand{by}\isamarkupfalse%
\ {\isacharparenleft}{\kern0pt}simp\ add{\isacharcolon}{\kern0pt}\ non{\isacharunderscore}{\kern0pt}electing{\isacharunderscore}{\kern0pt}def{\isacharparenright}{\kern0pt}\isanewline
\ \ \isacommand{thus}\isamarkupfalse%
\ {\isacharquery}{\kern0pt}thesis\isanewline
\ \ \ \ \isacommand{by}\isamarkupfalse%
\ {\isacharparenleft}{\kern0pt}simp\ add{\isacharcolon}{\kern0pt}\ non{\isacharunderscore}{\kern0pt}electing{\isacharunderscore}{\kern0pt}def{\isacharparenright}{\kern0pt}\isanewline
\isacommand{qed}\isamarkupfalse%
%
\endisatagproof
{\isafoldproof}%
%
\isadelimproof
\isanewline
%
\endisadelimproof
\isanewline
\isacommand{lemma}\isamarkupfalse%
\ min{\isacharunderscore}{\kern0pt}elim{\isacharunderscore}{\kern0pt}non{\isacharunderscore}{\kern0pt}electing{\isacharcolon}{\kern0pt}\isanewline
\ \ \isakeyword{assumes}\ profile{\isacharcolon}{\kern0pt}\ {\isachardoublequoteopen}finite{\isacharunderscore}{\kern0pt}profile\ A\ p{\isachardoublequoteclose}\isanewline
\ \ \isakeyword{shows}\ {\isachardoublequoteopen}non{\isacharunderscore}{\kern0pt}electing\ {\isacharparenleft}{\kern0pt}min{\isacharunderscore}{\kern0pt}eliminator\ e{\isacharparenright}{\kern0pt}{\isachardoublequoteclose}\isanewline
%
\isadelimproof
%
\endisadelimproof
%
\isatagproof
\isacommand{proof}\isamarkupfalse%
\ {\isacharminus}{\kern0pt}\isanewline
\ \ \isacommand{have}\isamarkupfalse%
\ {\isachardoublequoteopen}non{\isacharunderscore}{\kern0pt}electing\ {\isacharparenleft}{\kern0pt}elimination{\isacharunderscore}{\kern0pt}module\ e\ t\ {\isacharparenleft}{\kern0pt}{\isacharless}{\kern0pt}{\isacharparenright}{\kern0pt}{\isacharparenright}{\kern0pt}{\isachardoublequoteclose}\isanewline
\ \ \ \ \isacommand{by}\isamarkupfalse%
\ {\isacharparenleft}{\kern0pt}simp\ add{\isacharcolon}{\kern0pt}\ non{\isacharunderscore}{\kern0pt}electing{\isacharunderscore}{\kern0pt}def{\isacharparenright}{\kern0pt}\isanewline
\ \ \isacommand{thus}\isamarkupfalse%
\ {\isacharquery}{\kern0pt}thesis\isanewline
\ \ \ \ \isacommand{by}\isamarkupfalse%
\ {\isacharparenleft}{\kern0pt}simp\ add{\isacharcolon}{\kern0pt}\ non{\isacharunderscore}{\kern0pt}electing{\isacharunderscore}{\kern0pt}def{\isacharparenright}{\kern0pt}\isanewline
\isacommand{qed}\isamarkupfalse%
%
\endisatagproof
{\isafoldproof}%
%
\isadelimproof
\isanewline
%
\endisadelimproof
\isanewline
\isacommand{lemma}\isamarkupfalse%
\ less{\isacharunderscore}{\kern0pt}avg{\isacharunderscore}{\kern0pt}elim{\isacharunderscore}{\kern0pt}non{\isacharunderscore}{\kern0pt}electing{\isacharcolon}{\kern0pt}\isanewline
\ \ \isakeyword{assumes}\ profile{\isacharcolon}{\kern0pt}\ {\isachardoublequoteopen}finite{\isacharunderscore}{\kern0pt}profile\ A\ p{\isachardoublequoteclose}\isanewline
\ \ \isakeyword{shows}\ {\isachardoublequoteopen}non{\isacharunderscore}{\kern0pt}electing\ {\isacharparenleft}{\kern0pt}less{\isacharunderscore}{\kern0pt}average{\isacharunderscore}{\kern0pt}eliminator\ e{\isacharparenright}{\kern0pt}{\isachardoublequoteclose}\isanewline
%
\isadelimproof
%
\endisadelimproof
%
\isatagproof
\isacommand{proof}\isamarkupfalse%
\ {\isacharminus}{\kern0pt}\isanewline
\ \ \isacommand{have}\isamarkupfalse%
\ {\isachardoublequoteopen}non{\isacharunderscore}{\kern0pt}electing\ {\isacharparenleft}{\kern0pt}elimination{\isacharunderscore}{\kern0pt}module\ e\ t\ {\isacharparenleft}{\kern0pt}{\isacharless}{\kern0pt}{\isacharparenright}{\kern0pt}{\isacharparenright}{\kern0pt}{\isachardoublequoteclose}\isanewline
\ \ \ \ \isacommand{by}\isamarkupfalse%
\ {\isacharparenleft}{\kern0pt}simp\ add{\isacharcolon}{\kern0pt}\ non{\isacharunderscore}{\kern0pt}electing{\isacharunderscore}{\kern0pt}def{\isacharparenright}{\kern0pt}\isanewline
\ \ \isacommand{thus}\isamarkupfalse%
\ {\isacharquery}{\kern0pt}thesis\isanewline
\ \ \ \ \isacommand{by}\isamarkupfalse%
\ {\isacharparenleft}{\kern0pt}simp\ add{\isacharcolon}{\kern0pt}\ non{\isacharunderscore}{\kern0pt}electing{\isacharunderscore}{\kern0pt}def{\isacharparenright}{\kern0pt}\isanewline
\isacommand{qed}\isamarkupfalse%
%
\endisatagproof
{\isafoldproof}%
%
\isadelimproof
\isanewline
%
\endisadelimproof
\isanewline
\isacommand{lemma}\isamarkupfalse%
\ leq{\isacharunderscore}{\kern0pt}avg{\isacharunderscore}{\kern0pt}elim{\isacharunderscore}{\kern0pt}non{\isacharunderscore}{\kern0pt}electing{\isacharcolon}{\kern0pt}\isanewline
\ \ \isakeyword{assumes}\ profile{\isacharcolon}{\kern0pt}\ {\isachardoublequoteopen}finite{\isacharunderscore}{\kern0pt}profile\ A\ p{\isachardoublequoteclose}\isanewline
\ \ \isakeyword{shows}\ {\isachardoublequoteopen}non{\isacharunderscore}{\kern0pt}electing\ {\isacharparenleft}{\kern0pt}leq{\isacharunderscore}{\kern0pt}average{\isacharunderscore}{\kern0pt}eliminator\ e{\isacharparenright}{\kern0pt}{\isachardoublequoteclose}\isanewline
%
\isadelimproof
%
\endisadelimproof
%
\isatagproof
\isacommand{proof}\isamarkupfalse%
\ {\isacharminus}{\kern0pt}\isanewline
\ \ \isacommand{have}\isamarkupfalse%
\ {\isachardoublequoteopen}non{\isacharunderscore}{\kern0pt}electing\ {\isacharparenleft}{\kern0pt}elimination{\isacharunderscore}{\kern0pt}module\ e\ t\ {\isacharparenleft}{\kern0pt}{\isasymle}{\isacharparenright}{\kern0pt}{\isacharparenright}{\kern0pt}{\isachardoublequoteclose}\isanewline
\ \ \ \ \isacommand{by}\isamarkupfalse%
\ {\isacharparenleft}{\kern0pt}simp\ add{\isacharcolon}{\kern0pt}\ non{\isacharunderscore}{\kern0pt}electing{\isacharunderscore}{\kern0pt}def{\isacharparenright}{\kern0pt}\isanewline
\ \ \isacommand{thus}\isamarkupfalse%
\ {\isacharquery}{\kern0pt}thesis\isanewline
\ \ \ \ \isacommand{by}\isamarkupfalse%
\ {\isacharparenleft}{\kern0pt}simp\ add{\isacharcolon}{\kern0pt}\ non{\isacharunderscore}{\kern0pt}electing{\isacharunderscore}{\kern0pt}def{\isacharparenright}{\kern0pt}\isanewline
\isacommand{qed}\isamarkupfalse%
%
\endisatagproof
{\isafoldproof}%
%
\isadelimproof
%
\endisadelimproof
%
\isadelimdocument
%
\endisadelimdocument
%
\isatagdocument
%
\isamarkupsubsection{Inference Rules%
}
\isamarkuptrue%
%
\endisatagdocument
{\isafolddocument}%
%
\isadelimdocument
%
\endisadelimdocument
\isacommand{theorem}\isamarkupfalse%
\ cr{\isacharunderscore}{\kern0pt}eval{\isacharunderscore}{\kern0pt}imp{\isacharunderscore}{\kern0pt}ccomp{\isacharunderscore}{\kern0pt}max{\isacharunderscore}{\kern0pt}elim{\isacharbrackleft}{\kern0pt}simp{\isacharbrackright}{\kern0pt}{\isacharcolon}{\kern0pt}\isanewline
\ \ \isakeyword{assumes}\isanewline
\ \ \ \ profile{\isacharcolon}{\kern0pt}\ {\isachardoublequoteopen}finite{\isacharunderscore}{\kern0pt}profile\ A\ p{\isachardoublequoteclose}\ \isakeyword{and}\isanewline
\ \ \ \ rating{\isacharcolon}{\kern0pt}\ {\isachardoublequoteopen}condorcet{\isacharunderscore}{\kern0pt}rating\ e{\isachardoublequoteclose}\isanewline
\ \ \isakeyword{shows}\isanewline
\ \ \ \ {\isachardoublequoteopen}condorcet{\isacharunderscore}{\kern0pt}compatibility\ {\isacharparenleft}{\kern0pt}max{\isacharunderscore}{\kern0pt}eliminator\ e{\isacharparenright}{\kern0pt}{\isachardoublequoteclose}\isanewline
%
\isadelimproof
\ \ %
\endisadelimproof
%
\isatagproof
\isacommand{unfolding}\isamarkupfalse%
\ condorcet{\isacharunderscore}{\kern0pt}compatibility{\isacharunderscore}{\kern0pt}def\isanewline
\isacommand{proof}\isamarkupfalse%
\ {\isacharparenleft}{\kern0pt}auto{\isacharparenright}{\kern0pt}\isanewline
\ \ \isacommand{have}\isamarkupfalse%
\ f{\isadigit{1}}{\isacharcolon}{\kern0pt}\isanewline
\ \ \ \ {\isachardoublequoteopen}{\isasymAnd}A\ p\ w\ x{\isachardot}{\kern0pt}\ condorcet{\isacharunderscore}{\kern0pt}winner\ A\ p\ w\ {\isasymLongrightarrow}\isanewline
\ \ \ \ \ \ finite\ A\ {\isasymLongrightarrow}\ w\ {\isasymin}\ A\ {\isasymLongrightarrow}\ e\ w\ A\ p\ {\isacharless}{\kern0pt}\ Max\ {\isacharbraceleft}{\kern0pt}e\ x\ A\ p\ {\isacharbar}{\kern0pt}x{\isachardot}{\kern0pt}\ x\ {\isasymin}\ A{\isacharbraceright}{\kern0pt}\ {\isasymLongrightarrow}\isanewline
\ \ \ \ \ \ \ \ x\ {\isasymin}\ A\ {\isasymLongrightarrow}\ e\ x\ A\ p\ {\isacharless}{\kern0pt}\ Max\ {\isacharbraceleft}{\kern0pt}e\ x\ A\ p\ {\isacharbar}{\kern0pt}x{\isachardot}{\kern0pt}\ x\ {\isasymin}\ A{\isacharbraceright}{\kern0pt}{\isachardoublequoteclose}\isanewline
\ \ \ \ \isacommand{using}\isamarkupfalse%
\ rating\isanewline
\ \ \ \ \isacommand{by}\isamarkupfalse%
\ {\isacharparenleft}{\kern0pt}simp\ add{\isacharcolon}{\kern0pt}\ cond{\isacharunderscore}{\kern0pt}winner{\isacharunderscore}{\kern0pt}imp{\isacharunderscore}{\kern0pt}max{\isacharunderscore}{\kern0pt}eval{\isacharunderscore}{\kern0pt}val{\isacharparenright}{\kern0pt}\isanewline
\ \ \isacommand{thus}\isamarkupfalse%
\isanewline
\ \ \ \ {\isachardoublequoteopen}{\isasymAnd}A\ p\ w\ x{\isachardot}{\kern0pt}\isanewline
\ \ \ \ \ \ profile\ A\ p\ {\isasymLongrightarrow}\ w\ {\isasymin}\ A\ {\isasymLongrightarrow}\isanewline
\ \ \ \ \ \ \ \ {\isasymforall}x{\isasymin}A\ {\isacharminus}{\kern0pt}\ {\isacharbraceleft}{\kern0pt}w{\isacharbraceright}{\kern0pt}{\isachardot}{\kern0pt}\isanewline
\ \ \ \ \ \ \ \ \ \ card\ {\isacharbraceleft}{\kern0pt}i{\isachardot}{\kern0pt}\ i\ {\isacharless}{\kern0pt}\ length\ p\ {\isasymand}\ {\isacharparenleft}{\kern0pt}w{\isacharcomma}{\kern0pt}\ x{\isacharparenright}{\kern0pt}\ {\isasymin}\ {\isacharparenleft}{\kern0pt}p{\isacharbang}{\kern0pt}i{\isacharparenright}{\kern0pt}{\isacharbraceright}{\kern0pt}\ {\isacharless}{\kern0pt}\isanewline
\ \ \ \ \ \ \ \ \ \ \ \ card\ {\isacharbraceleft}{\kern0pt}i{\isachardot}{\kern0pt}\ i\ {\isacharless}{\kern0pt}\ length\ p\ {\isasymand}\ {\isacharparenleft}{\kern0pt}x{\isacharcomma}{\kern0pt}\ w{\isacharparenright}{\kern0pt}\ {\isasymin}\ {\isacharparenleft}{\kern0pt}p{\isacharbang}{\kern0pt}i{\isacharparenright}{\kern0pt}{\isacharbraceright}{\kern0pt}\ {\isasymLongrightarrow}\isanewline
\ \ \ \ \ \ \ \ \ \ \ \ \ \ finite\ A\ {\isasymLongrightarrow}\ e\ w\ A\ p\ {\isacharless}{\kern0pt}\ Max\ {\isacharbraceleft}{\kern0pt}e\ x\ A\ p\ {\isacharbar}{\kern0pt}\ x{\isachardot}{\kern0pt}\ x\ {\isasymin}\ A{\isacharbraceright}{\kern0pt}\ {\isasymLongrightarrow}\isanewline
\ \ \ \ \ \ \ \ \ \ \ \ \ \ \ \ x\ {\isasymin}\ A\ {\isasymLongrightarrow}\ e\ x\ A\ p\ {\isacharless}{\kern0pt}\ Max\ {\isacharbraceleft}{\kern0pt}e\ x\ A\ p\ {\isacharbar}{\kern0pt}\ x{\isachardot}{\kern0pt}\ x\ {\isasymin}\ A{\isacharbraceright}{\kern0pt}{\isachardoublequoteclose}\isanewline
\ \ \ \ \isacommand{by}\isamarkupfalse%
\ simp\isanewline
\isacommand{qed}\isamarkupfalse%
%
\endisatagproof
{\isafoldproof}%
%
\isadelimproof
\isanewline
%
\endisadelimproof
\isanewline
\isacommand{lemma}\isamarkupfalse%
\ cr{\isacharunderscore}{\kern0pt}eval{\isacharunderscore}{\kern0pt}imp{\isacharunderscore}{\kern0pt}dcc{\isacharunderscore}{\kern0pt}max{\isacharunderscore}{\kern0pt}elim{\isacharunderscore}{\kern0pt}helper{\isadigit{1}}{\isacharcolon}{\kern0pt}\isanewline
\ \ \isakeyword{assumes}\isanewline
\ \ \ \ f{\isacharunderscore}{\kern0pt}prof{\isacharcolon}{\kern0pt}\ {\isachardoublequoteopen}finite{\isacharunderscore}{\kern0pt}profile\ A\ p{\isachardoublequoteclose}\ \isakeyword{and}\isanewline
\ \ \ \ rating{\isacharcolon}{\kern0pt}\ {\isachardoublequoteopen}condorcet{\isacharunderscore}{\kern0pt}rating\ e{\isachardoublequoteclose}\ \isakeyword{and}\isanewline
\ \ \ \ winner{\isacharcolon}{\kern0pt}\ {\isachardoublequoteopen}condorcet{\isacharunderscore}{\kern0pt}winner\ A\ p\ w{\isachardoublequoteclose}\isanewline
\ \ \isakeyword{shows}\ {\isachardoublequoteopen}elimination{\isacharunderscore}{\kern0pt}set\ e\ {\isacharparenleft}{\kern0pt}Max\ {\isacharbraceleft}{\kern0pt}e\ x\ A\ p\ {\isacharbar}{\kern0pt}\ x{\isachardot}{\kern0pt}\ x\ {\isasymin}\ A{\isacharbraceright}{\kern0pt}{\isacharparenright}{\kern0pt}\ {\isacharparenleft}{\kern0pt}{\isacharless}{\kern0pt}{\isacharparenright}{\kern0pt}\ A\ p\ {\isacharequal}{\kern0pt}\ A\ {\isacharminus}{\kern0pt}\ {\isacharbraceleft}{\kern0pt}w{\isacharbraceright}{\kern0pt}{\isachardoublequoteclose}\isanewline
%
\isadelimproof
%
\endisadelimproof
%
\isatagproof
\isacommand{proof}\isamarkupfalse%
\ {\isacharparenleft}{\kern0pt}safe{\isacharcomma}{\kern0pt}\ simp{\isacharunderscore}{\kern0pt}all{\isacharcomma}{\kern0pt}\ safe{\isacharparenright}{\kern0pt}\isanewline
\ \ \isacommand{assume}\isamarkupfalse%
\isanewline
\ \ \ \ w{\isacharunderscore}{\kern0pt}in{\isacharunderscore}{\kern0pt}A{\isacharcolon}{\kern0pt}\ {\isachardoublequoteopen}w\ {\isasymin}\ A{\isachardoublequoteclose}\ \isakeyword{and}\isanewline
\ \ \ \ max{\isacharcolon}{\kern0pt}\ {\isachardoublequoteopen}e\ w\ A\ p\ {\isacharless}{\kern0pt}\ Max\ {\isacharbraceleft}{\kern0pt}e\ x\ A\ p\ {\isacharbar}{\kern0pt}x{\isachardot}{\kern0pt}\ x\ {\isasymin}\ A{\isacharbraceright}{\kern0pt}{\isachardoublequoteclose}\isanewline
\ \ \isacommand{show}\isamarkupfalse%
\ {\isachardoublequoteopen}False{\isachardoublequoteclose}\isanewline
\ \ \ \ \isacommand{using}\isamarkupfalse%
\ cond{\isacharunderscore}{\kern0pt}winner{\isacharunderscore}{\kern0pt}imp{\isacharunderscore}{\kern0pt}max{\isacharunderscore}{\kern0pt}eval{\isacharunderscore}{\kern0pt}val\isanewline
\ \ \ \ \ \ \ \ \ \ rating\ winner\ f{\isacharunderscore}{\kern0pt}prof\ max\isanewline
\ \ \ \ \isacommand{by}\isamarkupfalse%
\ fastforce\isanewline
\isacommand{next}\isamarkupfalse%
\isanewline
\ \ \isacommand{fix}\isamarkupfalse%
\isanewline
\ \ \ \ x\ {\isacharcolon}{\kern0pt}{\isacharcolon}{\kern0pt}\ {\isachardoublequoteopen}{\isacharprime}{\kern0pt}a{\isachardoublequoteclose}\isanewline
\ \ \isacommand{assume}\isamarkupfalse%
\isanewline
\ \ \ \ x{\isacharunderscore}{\kern0pt}in{\isacharunderscore}{\kern0pt}A{\isacharcolon}{\kern0pt}\ {\isachardoublequoteopen}x\ {\isasymin}\ A{\isachardoublequoteclose}\ \isakeyword{and}\isanewline
\ \ \ \ not{\isacharunderscore}{\kern0pt}max{\isacharcolon}{\kern0pt}\ {\isachardoublequoteopen}{\isasymnot}\ e\ x\ A\ p\ {\isacharless}{\kern0pt}\ Max\ {\isacharbraceleft}{\kern0pt}e\ y\ A\ p\ {\isacharbar}{\kern0pt}y{\isachardot}{\kern0pt}\ y\ {\isasymin}\ A{\isacharbraceright}{\kern0pt}{\isachardoublequoteclose}\isanewline
\ \ \isacommand{show}\isamarkupfalse%
\ {\isachardoublequoteopen}x\ {\isacharequal}{\kern0pt}\ w{\isachardoublequoteclose}\isanewline
\ \ \ \ \isacommand{using}\isamarkupfalse%
\ non{\isacharunderscore}{\kern0pt}cond{\isacharunderscore}{\kern0pt}winner{\isacharunderscore}{\kern0pt}not{\isacharunderscore}{\kern0pt}max{\isacharunderscore}{\kern0pt}eval\ x{\isacharunderscore}{\kern0pt}in{\isacharunderscore}{\kern0pt}A\isanewline
\ \ \ \ \ \ \ \ \ \ rating\ winner\ f{\isacharunderscore}{\kern0pt}prof\ not{\isacharunderscore}{\kern0pt}max\isanewline
\ \ \ \ \isacommand{by}\isamarkupfalse%
\ {\isacharparenleft}{\kern0pt}metis\ {\isacharparenleft}{\kern0pt}mono{\isacharunderscore}{\kern0pt}tags{\isacharcomma}{\kern0pt}\ lifting{\isacharparenright}{\kern0pt}{\isacharparenright}{\kern0pt}\isanewline
\isacommand{qed}\isamarkupfalse%
%
\endisatagproof
{\isafoldproof}%
%
\isadelimproof
\isanewline
%
\endisadelimproof
\isanewline
\isanewline
\isacommand{theorem}\isamarkupfalse%
\ cr{\isacharunderscore}{\kern0pt}eval{\isacharunderscore}{\kern0pt}imp{\isacharunderscore}{\kern0pt}dcc{\isacharunderscore}{\kern0pt}max{\isacharunderscore}{\kern0pt}elim{\isacharbrackleft}{\kern0pt}simp{\isacharbrackright}{\kern0pt}{\isacharcolon}{\kern0pt}\isanewline
\ \ \isakeyword{assumes}\ rating{\isacharcolon}{\kern0pt}\ {\isachardoublequoteopen}condorcet{\isacharunderscore}{\kern0pt}rating\ e{\isachardoublequoteclose}\isanewline
\ \ \isakeyword{shows}\ {\isachardoublequoteopen}defer{\isacharunderscore}{\kern0pt}condorcet{\isacharunderscore}{\kern0pt}consistency\ {\isacharparenleft}{\kern0pt}max{\isacharunderscore}{\kern0pt}eliminator\ e{\isacharparenright}{\kern0pt}{\isachardoublequoteclose}\isanewline
%
\isadelimproof
\ \ %
\endisadelimproof
%
\isatagproof
\isacommand{unfolding}\isamarkupfalse%
\ defer{\isacharunderscore}{\kern0pt}condorcet{\isacharunderscore}{\kern0pt}consistency{\isacharunderscore}{\kern0pt}def\isanewline
\isacommand{proof}\isamarkupfalse%
\ {\isacharparenleft}{\kern0pt}safe{\isacharcomma}{\kern0pt}\ simp{\isacharparenright}{\kern0pt}\isanewline
\ \ \isacommand{fix}\isamarkupfalse%
\isanewline
\ \ \ \ A\ {\isacharcolon}{\kern0pt}{\isacharcolon}{\kern0pt}\ {\isachardoublequoteopen}{\isacharprime}{\kern0pt}a\ set{\isachardoublequoteclose}\ \isakeyword{and}\isanewline
\ \ \ \ p\ {\isacharcolon}{\kern0pt}{\isacharcolon}{\kern0pt}\ {\isachardoublequoteopen}{\isacharprime}{\kern0pt}a\ Profile{\isachardoublequoteclose}\ \isakeyword{and}\isanewline
\ \ \ \ w\ {\isacharcolon}{\kern0pt}{\isacharcolon}{\kern0pt}\ {\isachardoublequoteopen}{\isacharprime}{\kern0pt}a{\isachardoublequoteclose}\isanewline
\ \ \isacommand{assume}\isamarkupfalse%
\isanewline
\ \ \ \ winner{\isacharcolon}{\kern0pt}\ {\isachardoublequoteopen}condorcet{\isacharunderscore}{\kern0pt}winner\ A\ p\ w{\isachardoublequoteclose}\ \isakeyword{and}\isanewline
\ \ \ \ finite{\isacharcolon}{\kern0pt}\ {\isachardoublequoteopen}finite\ A{\isachardoublequoteclose}\isanewline
\ \ \isacommand{let}\isamarkupfalse%
\ {\isacharquery}{\kern0pt}trsh\ {\isacharequal}{\kern0pt}\ {\isachardoublequoteopen}{\isacharparenleft}{\kern0pt}Max\ {\isacharbraceleft}{\kern0pt}e\ y\ A\ p\ {\isacharbar}{\kern0pt}\ y{\isachardot}{\kern0pt}\ y\ {\isasymin}\ A{\isacharbraceright}{\kern0pt}{\isacharparenright}{\kern0pt}{\isachardoublequoteclose}\isanewline
\ \ \isacommand{show}\isamarkupfalse%
\isanewline
\ \ \ \ {\isachardoublequoteopen}max{\isacharunderscore}{\kern0pt}eliminator\ e\ A\ p\ {\isacharequal}{\kern0pt}\isanewline
\ \ \ \ \ \ {\isacharparenleft}{\kern0pt}{\isacharbraceleft}{\kern0pt}{\isacharbraceright}{\kern0pt}{\isacharcomma}{\kern0pt}\isanewline
\ \ \ \ \ \ \ \ A\ {\isacharminus}{\kern0pt}\ defer\ {\isacharparenleft}{\kern0pt}max{\isacharunderscore}{\kern0pt}eliminator\ e{\isacharparenright}{\kern0pt}\ A\ p{\isacharcomma}{\kern0pt}\isanewline
\ \ \ \ \ \ \ \ {\isacharbraceleft}{\kern0pt}a\ {\isasymin}\ A{\isachardot}{\kern0pt}\ condorcet{\isacharunderscore}{\kern0pt}winner\ A\ p\ a{\isacharbraceright}{\kern0pt}{\isacharparenright}{\kern0pt}{\isachardoublequoteclose}\isanewline
\ \ \isacommand{proof}\isamarkupfalse%
\ {\isacharparenleft}{\kern0pt}cases\ {\isachardoublequoteopen}elimination{\isacharunderscore}{\kern0pt}set\ e\ {\isacharparenleft}{\kern0pt}{\isacharquery}{\kern0pt}trsh{\isacharparenright}{\kern0pt}\ {\isacharparenleft}{\kern0pt}{\isacharless}{\kern0pt}{\isacharparenright}{\kern0pt}\ A\ p\ {\isasymnoteq}\ A{\isachardoublequoteclose}{\isacharparenright}{\kern0pt}\isanewline
\ \ \ \ \isacommand{case}\isamarkupfalse%
\ True\isanewline
\ \ \ \ \isacommand{have}\isamarkupfalse%
\ profile{\isacharcolon}{\kern0pt}\ {\isachardoublequoteopen}finite{\isacharunderscore}{\kern0pt}profile\ A\ p{\isachardoublequoteclose}\isanewline
\ \ \ \ \ \ \isacommand{using}\isamarkupfalse%
\ winner\isanewline
\ \ \ \ \ \ \isacommand{by}\isamarkupfalse%
\ simp\isanewline
\ \ \ \ \isacommand{with}\isamarkupfalse%
\ rating\ winner\ \isacommand{have}\isamarkupfalse%
\ {\isadigit{0}}{\isacharcolon}{\kern0pt}\isanewline
\ \ \ \ \ \ {\isachardoublequoteopen}{\isacharparenleft}{\kern0pt}elimination{\isacharunderscore}{\kern0pt}set\ e\ {\isacharquery}{\kern0pt}trsh\ {\isacharparenleft}{\kern0pt}{\isacharless}{\kern0pt}{\isacharparenright}{\kern0pt}\ A\ p{\isacharparenright}{\kern0pt}\ {\isacharequal}{\kern0pt}\ A\ {\isacharminus}{\kern0pt}\ {\isacharbraceleft}{\kern0pt}w{\isacharbraceright}{\kern0pt}{\isachardoublequoteclose}\isanewline
\ \ \ \ \ \ \isacommand{using}\isamarkupfalse%
\ cr{\isacharunderscore}{\kern0pt}eval{\isacharunderscore}{\kern0pt}imp{\isacharunderscore}{\kern0pt}dcc{\isacharunderscore}{\kern0pt}max{\isacharunderscore}{\kern0pt}elim{\isacharunderscore}{\kern0pt}helper{\isadigit{1}}\isanewline
\ \ \ \ \ \ \isacommand{by}\isamarkupfalse%
\ {\isacharparenleft}{\kern0pt}metis\ {\isacharparenleft}{\kern0pt}mono{\isacharunderscore}{\kern0pt}tags{\isacharcomma}{\kern0pt}\ lifting{\isacharparenright}{\kern0pt}{\isacharparenright}{\kern0pt}\isanewline
\ \ \ \ \isacommand{have}\isamarkupfalse%
\isanewline
\ \ \ \ \ \ {\isachardoublequoteopen}max{\isacharunderscore}{\kern0pt}eliminator\ e\ A\ p\ {\isacharequal}{\kern0pt}\isanewline
\ \ \ \ \ \ \ \ {\isacharparenleft}{\kern0pt}{\isacharbraceleft}{\kern0pt}{\isacharbraceright}{\kern0pt}{\isacharcomma}{\kern0pt}\isanewline
\ \ \ \ \ \ \ \ \ \ {\isacharparenleft}{\kern0pt}elimination{\isacharunderscore}{\kern0pt}set\ e\ {\isacharquery}{\kern0pt}trsh\ {\isacharparenleft}{\kern0pt}{\isacharless}{\kern0pt}{\isacharparenright}{\kern0pt}\ A\ p{\isacharparenright}{\kern0pt}{\isacharcomma}{\kern0pt}\isanewline
\ \ \ \ \ \ \ \ \ \ A\ {\isacharminus}{\kern0pt}\ {\isacharparenleft}{\kern0pt}elimination{\isacharunderscore}{\kern0pt}set\ e\ {\isacharquery}{\kern0pt}trsh\ {\isacharparenleft}{\kern0pt}{\isacharless}{\kern0pt}{\isacharparenright}{\kern0pt}\ A\ p{\isacharparenright}{\kern0pt}{\isacharparenright}{\kern0pt}{\isachardoublequoteclose}\isanewline
\ \ \ \ \ \ \isacommand{using}\isamarkupfalse%
\ True\isanewline
\ \ \ \ \ \ \isacommand{by}\isamarkupfalse%
\ simp\isanewline
\ \ \ \ \isacommand{also}\isamarkupfalse%
\ \isacommand{have}\isamarkupfalse%
\ {\isachardoublequoteopen}{\isachardot}{\kern0pt}{\isachardot}{\kern0pt}{\isachardot}{\kern0pt}\ {\isacharequal}{\kern0pt}\ {\isacharparenleft}{\kern0pt}{\isacharbraceleft}{\kern0pt}{\isacharbraceright}{\kern0pt}{\isacharcomma}{\kern0pt}\ A\ {\isacharminus}{\kern0pt}\ {\isacharbraceleft}{\kern0pt}w{\isacharbraceright}{\kern0pt}{\isacharcomma}{\kern0pt}\ A\ {\isacharminus}{\kern0pt}\ {\isacharparenleft}{\kern0pt}A\ {\isacharminus}{\kern0pt}\ {\isacharbraceleft}{\kern0pt}w{\isacharbraceright}{\kern0pt}{\isacharparenright}{\kern0pt}{\isacharparenright}{\kern0pt}{\isachardoublequoteclose}\isanewline
\ \ \ \ \ \ \isacommand{using}\isamarkupfalse%
\ {\isachardoublequoteopen}{\isadigit{0}}{\isachardoublequoteclose}\isanewline
\ \ \ \ \ \ \isacommand{by}\isamarkupfalse%
\ presburger\isanewline
\ \ \ \ \isacommand{also}\isamarkupfalse%
\ \isacommand{have}\isamarkupfalse%
\ {\isachardoublequoteopen}{\isachardot}{\kern0pt}{\isachardot}{\kern0pt}{\isachardot}{\kern0pt}\ {\isacharequal}{\kern0pt}\ {\isacharparenleft}{\kern0pt}{\isacharbraceleft}{\kern0pt}{\isacharbraceright}{\kern0pt}{\isacharcomma}{\kern0pt}\ A\ {\isacharminus}{\kern0pt}\ {\isacharbraceleft}{\kern0pt}w{\isacharbraceright}{\kern0pt}{\isacharcomma}{\kern0pt}\ {\isacharbraceleft}{\kern0pt}w{\isacharbraceright}{\kern0pt}{\isacharparenright}{\kern0pt}{\isachardoublequoteclose}\isanewline
\ \ \ \ \ \ \isacommand{using}\isamarkupfalse%
\ winner\isanewline
\ \ \ \ \ \ \isacommand{by}\isamarkupfalse%
\ auto\isanewline
\ \ \ \ \isacommand{also}\isamarkupfalse%
\ \isacommand{have}\isamarkupfalse%
\ {\isachardoublequoteopen}{\isachardot}{\kern0pt}{\isachardot}{\kern0pt}{\isachardot}{\kern0pt}\ {\isacharequal}{\kern0pt}\ {\isacharparenleft}{\kern0pt}{\isacharbraceleft}{\kern0pt}{\isacharbraceright}{\kern0pt}{\isacharcomma}{\kern0pt}A\ {\isacharminus}{\kern0pt}\ defer\ {\isacharparenleft}{\kern0pt}max{\isacharunderscore}{\kern0pt}eliminator\ e{\isacharparenright}{\kern0pt}\ A\ p{\isacharcomma}{\kern0pt}\ {\isacharbraceleft}{\kern0pt}w{\isacharbraceright}{\kern0pt}{\isacharparenright}{\kern0pt}{\isachardoublequoteclose}\isanewline
\ \ \ \ \ \ \isacommand{using}\isamarkupfalse%
\ calculation\isanewline
\ \ \ \ \ \ \isacommand{by}\isamarkupfalse%
\ auto\isanewline
\ \ \ \ \isacommand{also}\isamarkupfalse%
\ \isacommand{have}\isamarkupfalse%
\isanewline
\ \ \ \ \ \ {\isachardoublequoteopen}{\isachardot}{\kern0pt}{\isachardot}{\kern0pt}{\isachardot}{\kern0pt}\ {\isacharequal}{\kern0pt}\isanewline
\ \ \ \ \ \ \ \ {\isacharparenleft}{\kern0pt}{\isacharbraceleft}{\kern0pt}{\isacharbraceright}{\kern0pt}{\isacharcomma}{\kern0pt}\isanewline
\ \ \ \ \ \ \ \ \ \ A\ {\isacharminus}{\kern0pt}\ defer\ {\isacharparenleft}{\kern0pt}max{\isacharunderscore}{\kern0pt}eliminator\ e{\isacharparenright}{\kern0pt}\ A\ p{\isacharcomma}{\kern0pt}\isanewline
\ \ \ \ \ \ \ \ \ \ {\isacharbraceleft}{\kern0pt}d\ {\isasymin}\ A{\isachardot}{\kern0pt}\ condorcet{\isacharunderscore}{\kern0pt}winner\ A\ p\ d{\isacharbraceright}{\kern0pt}{\isacharparenright}{\kern0pt}{\isachardoublequoteclose}\isanewline
\ \ \ \ \ \ \isacommand{using}\isamarkupfalse%
\ cond{\isacharunderscore}{\kern0pt}winner{\isacharunderscore}{\kern0pt}unique{\isadigit{3}}\ winner\ Collect{\isacharunderscore}{\kern0pt}cong\isanewline
\ \ \ \ \ \ \isacommand{by}\isamarkupfalse%
\ {\isacharparenleft}{\kern0pt}metis\ {\isacharparenleft}{\kern0pt}no{\isacharunderscore}{\kern0pt}types{\isacharcomma}{\kern0pt}\ lifting{\isacharparenright}{\kern0pt}{\isacharparenright}{\kern0pt}\isanewline
\ \ \ \ \isacommand{finally}\isamarkupfalse%
\ \isacommand{show}\isamarkupfalse%
\ {\isacharquery}{\kern0pt}thesis\isanewline
\ \ \ \ \ \ \isacommand{using}\isamarkupfalse%
\ finite\ winner\isanewline
\ \ \ \ \ \ \isacommand{by}\isamarkupfalse%
\ metis\isanewline
\ \ \isacommand{next}\isamarkupfalse%
\isanewline
\ \ \ \ \isacommand{case}\isamarkupfalse%
\ False\isanewline
\ \ \ \ \isacommand{thus}\isamarkupfalse%
\ {\isacharquery}{\kern0pt}thesis\isanewline
\ \ \ \ \isacommand{proof}\isamarkupfalse%
\ {\isacharminus}{\kern0pt}\isanewline
\ \ \ \ \ \ \isacommand{have}\isamarkupfalse%
\ f{\isadigit{1}}{\isacharcolon}{\kern0pt}\isanewline
\ \ \ \ \ \ \ \ {\isachardoublequoteopen}finite\ A\ {\isasymand}\ profile\ A\ p\ {\isasymand}\ w\ {\isasymin}\ A\ {\isasymand}\ {\isacharparenleft}{\kern0pt}{\isasymforall}a{\isachardot}{\kern0pt}\ a\ {\isasymnotin}\ A\ {\isacharminus}{\kern0pt}\ {\isacharbraceleft}{\kern0pt}w{\isacharbraceright}{\kern0pt}\ {\isasymor}\ wins\ w\ p\ a{\isacharparenright}{\kern0pt}{\isachardoublequoteclose}\isanewline
\ \ \ \ \ \ \ \ \isacommand{using}\isamarkupfalse%
\ winner\isanewline
\ \ \ \ \ \ \ \ \isacommand{by}\isamarkupfalse%
\ auto\isanewline
\ \ \ \ \ \ \isacommand{hence}\isamarkupfalse%
\isanewline
\ \ \ \ \ \ \ \ {\isachardoublequoteopen}{\isacharquery}{\kern0pt}trsh\ {\isacharequal}{\kern0pt}\ e\ w\ A\ p{\isachardoublequoteclose}\isanewline
\ \ \ \ \ \ \ \ \isacommand{using}\isamarkupfalse%
\ rating\ winner\isanewline
\ \ \ \ \ \ \ \ \isacommand{by}\isamarkupfalse%
\ {\isacharparenleft}{\kern0pt}simp\ add{\isacharcolon}{\kern0pt}\ cond{\isacharunderscore}{\kern0pt}winner{\isacharunderscore}{\kern0pt}imp{\isacharunderscore}{\kern0pt}max{\isacharunderscore}{\kern0pt}eval{\isacharunderscore}{\kern0pt}val{\isacharparenright}{\kern0pt}\isanewline
\ \ \ \ \ \ \isacommand{hence}\isamarkupfalse%
\ False\isanewline
\ \ \ \ \ \ \ \ \isacommand{using}\isamarkupfalse%
\ f{\isadigit{1}}\ False\isanewline
\ \ \ \ \ \ \ \ \isacommand{by}\isamarkupfalse%
\ auto\isanewline
\ \ \ \ \ \ \isacommand{thus}\isamarkupfalse%
\ {\isacharquery}{\kern0pt}thesis\isanewline
\ \ \ \ \ \ \ \ \isacommand{by}\isamarkupfalse%
\ simp\isanewline
\ \ \ \ \isacommand{qed}\isamarkupfalse%
\isanewline
\ \ \isacommand{qed}\isamarkupfalse%
\isanewline
\isacommand{qed}\isamarkupfalse%
%
\endisatagproof
{\isafoldproof}%
%
\isadelimproof
\isanewline
%
\endisadelimproof
%
\isadelimtheory
\isanewline
%
\endisadelimtheory
%
\isatagtheory
\isacommand{end}\isamarkupfalse%
%
\endisatagtheory
{\isafoldtheory}%
%
\isadelimtheory
%
\endisadelimtheory
%
\end{isabellebody}%
\endinput
%:%file=~/Documents/Studies/VotingRuleGenerator/virage/src/test/resources/old_theories/Compositional_Structures/Basic_Modules/Component_Types/Elimination_Module.thy%:%
%:%6=3%:%
%:%11=4%:%
%:%13=7%:%
%:%29=9%:%
%:%30=9%:%
%:%31=10%:%
%:%32=11%:%
%:%33=12%:%
%:%42=15%:%
%:%43=16%:%
%:%44=17%:%
%:%45=18%:%
%:%54=20%:%
%:%64=22%:%
%:%65=22%:%
%:%66=23%:%
%:%67=24%:%
%:%68=24%:%
%:%69=25%:%
%:%70=26%:%
%:%71=26%:%
%:%73=28%:%
%:%74=29%:%
%:%75=30%:%
%:%76=31%:%
%:%77=31%:%
%:%78=32%:%
%:%79=33%:%
%:%89=38%:%
%:%99=40%:%
%:%100=40%:%
%:%101=41%:%
%:%102=42%:%
%:%103=43%:%
%:%104=44%:%
%:%105=44%:%
%:%106=45%:%
%:%107=46%:%
%:%108=47%:%
%:%109=48%:%
%:%110=48%:%
%:%111=49%:%
%:%112=50%:%
%:%113=51%:%
%:%114=52%:%
%:%115=52%:%
%:%116=53%:%
%:%117=54%:%
%:%118=55%:%
%:%119=56%:%
%:%120=56%:%
%:%121=57%:%
%:%122=58%:%
%:%123=59%:%
%:%124=60%:%
%:%125=60%:%
%:%126=61%:%
%:%127=62%:%
%:%128=63%:%
%:%129=64%:%
%:%130=64%:%
%:%131=65%:%
%:%132=66%:%
%:%139=68%:%
%:%149=70%:%
%:%150=70%:%
%:%157=71%:%
%:%158=71%:%
%:%159=72%:%
%:%160=72%:%
%:%161=73%:%
%:%162=74%:%
%:%163=75%:%
%:%164=75%:%
%:%165=76%:%
%:%166=76%:%
%:%167=77%:%
%:%168=77%:%
%:%169=78%:%
%:%170=78%:%
%:%171=79%:%
%:%177=79%:%
%:%180=80%:%
%:%181=81%:%
%:%182=81%:%
%:%185=82%:%
%:%189=82%:%
%:%190=82%:%
%:%191=83%:%
%:%192=83%:%
%:%193=84%:%
%:%194=84%:%
%:%195=85%:%
%:%196=86%:%
%:%197=87%:%
%:%198=87%:%
%:%199=88%:%
%:%200=89%:%
%:%201=90%:%
%:%202=90%:%
%:%203=91%:%
%:%209=91%:%
%:%212=92%:%
%:%213=93%:%
%:%214=93%:%
%:%217=94%:%
%:%221=94%:%
%:%222=94%:%
%:%223=95%:%
%:%224=95%:%
%:%225=96%:%
%:%226=96%:%
%:%227=97%:%
%:%228=98%:%
%:%229=99%:%
%:%230=99%:%
%:%231=100%:%
%:%232=101%:%
%:%233=102%:%
%:%234=102%:%
%:%235=103%:%
%:%241=103%:%
%:%244=104%:%
%:%245=105%:%
%:%246=105%:%
%:%249=106%:%
%:%253=106%:%
%:%254=106%:%
%:%255=107%:%
%:%256=107%:%
%:%257=108%:%
%:%258=108%:%
%:%259=109%:%
%:%260=110%:%
%:%261=111%:%
%:%262=111%:%
%:%263=112%:%
%:%264=113%:%
%:%265=114%:%
%:%266=114%:%
%:%267=115%:%
%:%273=115%:%
%:%276=116%:%
%:%277=117%:%
%:%278=117%:%
%:%281=118%:%
%:%285=118%:%
%:%286=118%:%
%:%287=119%:%
%:%288=119%:%
%:%289=120%:%
%:%290=120%:%
%:%291=121%:%
%:%292=122%:%
%:%293=123%:%
%:%294=123%:%
%:%295=124%:%
%:%296=125%:%
%:%297=126%:%
%:%298=126%:%
%:%299=127%:%
%:%305=127%:%
%:%308=128%:%
%:%309=129%:%
%:%310=129%:%
%:%313=130%:%
%:%317=130%:%
%:%318=130%:%
%:%319=131%:%
%:%320=131%:%
%:%321=132%:%
%:%322=132%:%
%:%323=133%:%
%:%324=134%:%
%:%325=135%:%
%:%326=135%:%
%:%327=136%:%
%:%328=137%:%
%:%329=138%:%
%:%330=138%:%
%:%331=139%:%
%:%337=139%:%
%:%340=140%:%
%:%341=141%:%
%:%342=141%:%
%:%345=142%:%
%:%349=142%:%
%:%350=142%:%
%:%351=143%:%
%:%352=143%:%
%:%353=144%:%
%:%354=144%:%
%:%355=145%:%
%:%356=146%:%
%:%357=147%:%
%:%358=147%:%
%:%359=148%:%
%:%360=149%:%
%:%361=150%:%
%:%362=150%:%
%:%363=151%:%
%:%378=153%:%
%:%388=155%:%
%:%389=155%:%
%:%390=156%:%
%:%391=157%:%
%:%394=158%:%
%:%398=158%:%
%:%399=158%:%
%:%404=158%:%
%:%407=159%:%
%:%408=160%:%
%:%409=160%:%
%:%410=161%:%
%:%411=162%:%
%:%414=163%:%
%:%418=163%:%
%:%419=163%:%
%:%420=164%:%
%:%421=164%:%
%:%426=164%:%
%:%429=165%:%
%:%430=166%:%
%:%431=166%:%
%:%432=167%:%
%:%433=168%:%
%:%440=169%:%
%:%441=169%:%
%:%442=170%:%
%:%443=170%:%
%:%444=171%:%
%:%445=171%:%
%:%446=172%:%
%:%447=172%:%
%:%448=173%:%
%:%449=173%:%
%:%450=174%:%
%:%456=174%:%
%:%459=175%:%
%:%460=176%:%
%:%461=176%:%
%:%462=177%:%
%:%463=178%:%
%:%470=179%:%
%:%471=179%:%
%:%472=180%:%
%:%473=180%:%
%:%474=181%:%
%:%475=181%:%
%:%476=182%:%
%:%477=182%:%
%:%478=183%:%
%:%479=183%:%
%:%480=184%:%
%:%486=184%:%
%:%489=185%:%
%:%490=186%:%
%:%491=186%:%
%:%492=187%:%
%:%493=188%:%
%:%500=189%:%
%:%501=189%:%
%:%502=190%:%
%:%503=190%:%
%:%504=191%:%
%:%505=191%:%
%:%506=192%:%
%:%507=192%:%
%:%508=193%:%
%:%509=193%:%
%:%510=194%:%
%:%516=194%:%
%:%519=195%:%
%:%520=196%:%
%:%521=196%:%
%:%522=197%:%
%:%523=198%:%
%:%530=199%:%
%:%531=199%:%
%:%532=200%:%
%:%533=200%:%
%:%534=201%:%
%:%535=201%:%
%:%536=202%:%
%:%537=202%:%
%:%538=203%:%
%:%539=203%:%
%:%540=204%:%
%:%546=204%:%
%:%549=205%:%
%:%550=206%:%
%:%551=206%:%
%:%552=207%:%
%:%553=208%:%
%:%560=209%:%
%:%561=209%:%
%:%562=210%:%
%:%563=210%:%
%:%564=211%:%
%:%565=211%:%
%:%566=212%:%
%:%567=212%:%
%:%568=213%:%
%:%569=213%:%
%:%570=214%:%
%:%585=216%:%
%:%595=220%:%
%:%596=220%:%
%:%597=221%:%
%:%598=222%:%
%:%599=223%:%
%:%600=224%:%
%:%601=225%:%
%:%604=226%:%
%:%608=226%:%
%:%609=226%:%
%:%610=227%:%
%:%611=227%:%
%:%612=228%:%
%:%613=228%:%
%:%614=229%:%
%:%616=231%:%
%:%617=232%:%
%:%618=232%:%
%:%619=233%:%
%:%620=233%:%
%:%621=234%:%
%:%622=234%:%
%:%623=235%:%
%:%629=241%:%
%:%630=242%:%
%:%631=242%:%
%:%632=243%:%
%:%638=243%:%
%:%641=244%:%
%:%642=245%:%
%:%643=245%:%
%:%644=246%:%
%:%645=247%:%
%:%646=248%:%
%:%647=249%:%
%:%648=250%:%
%:%655=251%:%
%:%656=251%:%
%:%657=252%:%
%:%658=252%:%
%:%659=253%:%
%:%660=254%:%
%:%661=255%:%
%:%662=255%:%
%:%663=256%:%
%:%664=256%:%
%:%665=257%:%
%:%666=258%:%
%:%667=258%:%
%:%668=259%:%
%:%669=259%:%
%:%670=260%:%
%:%671=260%:%
%:%672=261%:%
%:%673=262%:%
%:%674=262%:%
%:%675=263%:%
%:%676=264%:%
%:%677=265%:%
%:%678=265%:%
%:%679=266%:%
%:%680=266%:%
%:%681=267%:%
%:%682=268%:%
%:%683=268%:%
%:%684=269%:%
%:%690=269%:%
%:%693=270%:%
%:%694=274%:%
%:%695=275%:%
%:%696=275%:%
%:%697=276%:%
%:%698=277%:%
%:%701=278%:%
%:%705=278%:%
%:%706=278%:%
%:%707=279%:%
%:%708=279%:%
%:%709=280%:%
%:%710=280%:%
%:%711=281%:%
%:%712=282%:%
%:%713=283%:%
%:%714=284%:%
%:%715=284%:%
%:%716=285%:%
%:%717=286%:%
%:%718=287%:%
%:%719=287%:%
%:%720=288%:%
%:%721=288%:%
%:%722=289%:%
%:%725=292%:%
%:%726=293%:%
%:%727=293%:%
%:%728=294%:%
%:%729=294%:%
%:%730=295%:%
%:%731=295%:%
%:%732=296%:%
%:%733=296%:%
%:%734=297%:%
%:%735=297%:%
%:%736=298%:%
%:%737=298%:%
%:%738=298%:%
%:%739=299%:%
%:%740=300%:%
%:%741=300%:%
%:%742=301%:%
%:%743=301%:%
%:%744=302%:%
%:%745=302%:%
%:%746=303%:%
%:%749=306%:%
%:%750=307%:%
%:%751=307%:%
%:%752=308%:%
%:%753=308%:%
%:%754=309%:%
%:%755=309%:%
%:%756=309%:%
%:%757=310%:%
%:%758=310%:%
%:%759=311%:%
%:%760=311%:%
%:%761=312%:%
%:%762=312%:%
%:%763=312%:%
%:%764=313%:%
%:%765=313%:%
%:%766=314%:%
%:%767=314%:%
%:%768=315%:%
%:%769=315%:%
%:%770=315%:%
%:%771=316%:%
%:%772=316%:%
%:%773=317%:%
%:%774=317%:%
%:%775=318%:%
%:%776=318%:%
%:%777=318%:%
%:%778=319%:%
%:%781=322%:%
%:%782=323%:%
%:%783=323%:%
%:%784=324%:%
%:%785=324%:%
%:%786=325%:%
%:%787=325%:%
%:%788=325%:%
%:%789=326%:%
%:%790=326%:%
%:%791=327%:%
%:%792=327%:%
%:%793=328%:%
%:%794=328%:%
%:%795=329%:%
%:%796=329%:%
%:%797=330%:%
%:%798=330%:%
%:%799=331%:%
%:%800=331%:%
%:%801=332%:%
%:%802=332%:%
%:%803=333%:%
%:%804=334%:%
%:%805=334%:%
%:%806=335%:%
%:%807=335%:%
%:%808=336%:%
%:%809=336%:%
%:%810=337%:%
%:%811=338%:%
%:%812=338%:%
%:%813=339%:%
%:%814=339%:%
%:%815=340%:%
%:%816=340%:%
%:%817=341%:%
%:%818=341%:%
%:%819=342%:%
%:%820=342%:%
%:%821=343%:%
%:%822=343%:%
%:%823=344%:%
%:%824=344%:%
%:%825=345%:%
%:%826=345%:%
%:%827=346%:%
%:%828=346%:%
%:%829=347%:%
%:%835=347%:%
%:%840=348%:%
%:%845=349%:%
%
\begin{isabellebody}%
\setisabellecontext{Maximum{\isacharunderscore}{\kern0pt}Aggregator}%
%
\isadelimdocument
\isanewline
%
\endisadelimdocument
%
\isatagdocument
\isanewline
\isanewline
%
\isamarkupsection{Maximum Aggregator%
}
\isamarkuptrue%
%
\endisatagdocument
{\isafolddocument}%
%
\isadelimdocument
%
\endisadelimdocument
%
\isadelimtheory
%
\endisadelimtheory
%
\isatagtheory
\isacommand{theory}\isamarkupfalse%
\ Maximum{\isacharunderscore}{\kern0pt}Aggregator\isanewline
\ \ \isakeyword{imports}\ Aggregator\isanewline
\isakeyword{begin}%
\endisatagtheory
{\isafoldtheory}%
%
\isadelimtheory
%
\endisadelimtheory
%
\begin{isamarkuptext}%
The max(imum) aggregator takes two partitions of an alternative set A as
input. It returns a partition where every alternative receives the maximum
result of the two input partitions.%
\end{isamarkuptext}\isamarkuptrue%
%
\isadelimdocument
%
\endisadelimdocument
%
\isatagdocument
%
\isamarkupsubsection{Definition%
}
\isamarkuptrue%
%
\endisatagdocument
{\isafolddocument}%
%
\isadelimdocument
%
\endisadelimdocument
\isacommand{fun}\isamarkupfalse%
\ max{\isacharunderscore}{\kern0pt}aggregator\ {\isacharcolon}{\kern0pt}{\isacharcolon}{\kern0pt}\ {\isachardoublequoteopen}{\isacharprime}{\kern0pt}a\ Aggregator{\isachardoublequoteclose}\ \isakeyword{where}\isanewline
\ \ {\isachardoublequoteopen}max{\isacharunderscore}{\kern0pt}aggregator\ A\ {\isacharparenleft}{\kern0pt}e{\isadigit{1}}{\isacharcomma}{\kern0pt}\ r{\isadigit{1}}{\isacharcomma}{\kern0pt}\ d{\isadigit{1}}{\isacharparenright}{\kern0pt}\ {\isacharparenleft}{\kern0pt}e{\isadigit{2}}{\isacharcomma}{\kern0pt}\ r{\isadigit{2}}{\isacharcomma}{\kern0pt}\ d{\isadigit{2}}{\isacharparenright}{\kern0pt}\ {\isacharequal}{\kern0pt}\isanewline
\ \ \ \ {\isacharparenleft}{\kern0pt}e{\isadigit{1}}\ {\isasymunion}\ e{\isadigit{2}}{\isacharcomma}{\kern0pt}\isanewline
\ \ \ \ \ A\ {\isacharminus}{\kern0pt}\ {\isacharparenleft}{\kern0pt}e{\isadigit{1}}\ {\isasymunion}\ e{\isadigit{2}}\ {\isasymunion}\ d{\isadigit{1}}\ {\isasymunion}\ d{\isadigit{2}}{\isacharparenright}{\kern0pt}{\isacharcomma}{\kern0pt}\isanewline
\ \ \ \ \ {\isacharparenleft}{\kern0pt}d{\isadigit{1}}\ {\isasymunion}\ d{\isadigit{2}}{\isacharparenright}{\kern0pt}\ {\isacharminus}{\kern0pt}\ {\isacharparenleft}{\kern0pt}e{\isadigit{1}}\ {\isasymunion}\ e{\isadigit{2}}{\isacharparenright}{\kern0pt}{\isacharparenright}{\kern0pt}{\isachardoublequoteclose}%
\isadelimdocument
%
\endisadelimdocument
%
\isatagdocument
%
\isamarkupsubsection{Auxiliary Lemma%
}
\isamarkuptrue%
%
\endisatagdocument
{\isafolddocument}%
%
\isadelimdocument
%
\endisadelimdocument
\isacommand{lemma}\isamarkupfalse%
\ max{\isacharunderscore}{\kern0pt}agg{\isacharunderscore}{\kern0pt}rej{\isacharunderscore}{\kern0pt}set{\isacharcolon}{\kern0pt}\ {\isachardoublequoteopen}{\isacharparenleft}{\kern0pt}well{\isacharunderscore}{\kern0pt}formed\ A\ {\isacharparenleft}{\kern0pt}e{\isadigit{1}}{\isacharcomma}{\kern0pt}\ r{\isadigit{1}}{\isacharcomma}{\kern0pt}\ d{\isadigit{1}}{\isacharparenright}{\kern0pt}\ {\isasymand}\isanewline
\ \ \ \ \ \ \ \ \ \ \ \ \ \ \ \ \ \ \ \ \ \ \ \ \ \ well{\isacharunderscore}{\kern0pt}formed\ A\ {\isacharparenleft}{\kern0pt}e{\isadigit{2}}{\isacharcomma}{\kern0pt}\ r{\isadigit{2}}{\isacharcomma}{\kern0pt}\ d{\isadigit{2}}{\isacharparenright}{\kern0pt}{\isacharparenright}{\kern0pt}\ {\isasymlongrightarrow}\isanewline
\ \ \ \ \ \ \ \ \ \ \ reject{\isacharunderscore}{\kern0pt}r\ {\isacharparenleft}{\kern0pt}max{\isacharunderscore}{\kern0pt}aggregator\ A\ {\isacharparenleft}{\kern0pt}e{\isadigit{1}}{\isacharcomma}{\kern0pt}\ r{\isadigit{1}}{\isacharcomma}{\kern0pt}\ d{\isadigit{1}}{\isacharparenright}{\kern0pt}\ {\isacharparenleft}{\kern0pt}e{\isadigit{2}}{\isacharcomma}{\kern0pt}\ r{\isadigit{2}}{\isacharcomma}{\kern0pt}\ d{\isadigit{2}}{\isacharparenright}{\kern0pt}{\isacharparenright}{\kern0pt}\ {\isacharequal}{\kern0pt}\ r{\isadigit{1}}\ {\isasyminter}\ r{\isadigit{2}}{\isachardoublequoteclose}\isanewline
%
\isadelimproof
%
\endisadelimproof
%
\isatagproof
\isacommand{proof}\isamarkupfalse%
\ {\isacharminus}{\kern0pt}\isanewline
\ \ \isacommand{have}\isamarkupfalse%
\ {\isachardoublequoteopen}well{\isacharunderscore}{\kern0pt}formed\ A\ {\isacharparenleft}{\kern0pt}e{\isadigit{1}}{\isacharcomma}{\kern0pt}\ r{\isadigit{1}}{\isacharcomma}{\kern0pt}\ d{\isadigit{1}}{\isacharparenright}{\kern0pt}\ {\isasymlongrightarrow}\ \ A\ {\isacharminus}{\kern0pt}\ {\isacharparenleft}{\kern0pt}e{\isadigit{1}}\ {\isasymunion}\ d{\isadigit{1}}{\isacharparenright}{\kern0pt}\ {\isacharequal}{\kern0pt}\ r{\isadigit{1}}{\isachardoublequoteclose}\isanewline
\ \ \ \ \isacommand{by}\isamarkupfalse%
\ {\isacharparenleft}{\kern0pt}simp\ add{\isacharcolon}{\kern0pt}\ result{\isacharunderscore}{\kern0pt}imp{\isacharunderscore}{\kern0pt}rej{\isacharparenright}{\kern0pt}\isanewline
\ \ \isacommand{moreover}\isamarkupfalse%
\ \isacommand{have}\isamarkupfalse%
\isanewline
\ \ \ \ {\isachardoublequoteopen}well{\isacharunderscore}{\kern0pt}formed\ A\ {\isacharparenleft}{\kern0pt}e{\isadigit{2}}{\isacharcomma}{\kern0pt}\ r{\isadigit{2}}{\isacharcomma}{\kern0pt}\ d{\isadigit{2}}{\isacharparenright}{\kern0pt}\ {\isasymlongrightarrow}\ \ A\ {\isacharminus}{\kern0pt}\ {\isacharparenleft}{\kern0pt}e{\isadigit{2}}\ {\isasymunion}\ d{\isadigit{2}}{\isacharparenright}{\kern0pt}\ {\isacharequal}{\kern0pt}\ r{\isadigit{2}}{\isachardoublequoteclose}\isanewline
\ \ \ \ \isacommand{by}\isamarkupfalse%
\ {\isacharparenleft}{\kern0pt}simp\ add{\isacharcolon}{\kern0pt}\ result{\isacharunderscore}{\kern0pt}imp{\isacharunderscore}{\kern0pt}rej{\isacharparenright}{\kern0pt}\isanewline
\ \ \isacommand{ultimately}\isamarkupfalse%
\ \isacommand{have}\isamarkupfalse%
\isanewline
\ \ \ \ {\isachardoublequoteopen}{\isacharparenleft}{\kern0pt}well{\isacharunderscore}{\kern0pt}formed\ A\ {\isacharparenleft}{\kern0pt}e{\isadigit{1}}{\isacharcomma}{\kern0pt}\ r{\isadigit{1}}{\isacharcomma}{\kern0pt}\ d{\isadigit{1}}{\isacharparenright}{\kern0pt}\ {\isasymand}\ well{\isacharunderscore}{\kern0pt}formed\ A\ {\isacharparenleft}{\kern0pt}e{\isadigit{2}}{\isacharcomma}{\kern0pt}\ r{\isadigit{2}}{\isacharcomma}{\kern0pt}\ d{\isadigit{2}}{\isacharparenright}{\kern0pt}{\isacharparenright}{\kern0pt}\ {\isasymlongrightarrow}\isanewline
\ \ \ \ \ \ \ \ A\ {\isacharminus}{\kern0pt}\ {\isacharparenleft}{\kern0pt}e{\isadigit{1}}\ {\isasymunion}\ e{\isadigit{2}}\ {\isasymunion}\ d{\isadigit{1}}\ {\isasymunion}\ d{\isadigit{2}}{\isacharparenright}{\kern0pt}\ {\isacharequal}{\kern0pt}\ r{\isadigit{1}}\ {\isasyminter}\ r{\isadigit{2}}{\isachardoublequoteclose}\isanewline
\ \ \ \ \isacommand{by}\isamarkupfalse%
\ blast\isanewline
\ \ \isacommand{moreover}\isamarkupfalse%
\ \isacommand{have}\isamarkupfalse%
\isanewline
\ \ \ \ {\isachardoublequoteopen}{\isacharbraceleft}{\kern0pt}l\ {\isasymin}\ A{\isachardot}{\kern0pt}\ l\ {\isasymnotin}\ e{\isadigit{1}}\ {\isasymunion}\ e{\isadigit{2}}\ {\isasymunion}\ d{\isadigit{1}}\ {\isasymunion}\ d{\isadigit{2}}{\isacharbraceright}{\kern0pt}\ {\isacharequal}{\kern0pt}\ A\ {\isacharminus}{\kern0pt}\ {\isacharparenleft}{\kern0pt}e{\isadigit{1}}\ {\isasymunion}\ e{\isadigit{2}}\ {\isasymunion}\ d{\isadigit{1}}\ {\isasymunion}\ d{\isadigit{2}}{\isacharparenright}{\kern0pt}{\isachardoublequoteclose}\isanewline
\ \ \ \ \isacommand{by}\isamarkupfalse%
\ {\isacharparenleft}{\kern0pt}simp\ add{\isacharcolon}{\kern0pt}\ set{\isacharunderscore}{\kern0pt}diff{\isacharunderscore}{\kern0pt}eq{\isacharparenright}{\kern0pt}\isanewline
\ \ \isacommand{ultimately}\isamarkupfalse%
\ \isacommand{show}\isamarkupfalse%
\ {\isacharquery}{\kern0pt}thesis\isanewline
\ \ \ \ \isacommand{by}\isamarkupfalse%
\ simp\isanewline
\isacommand{qed}\isamarkupfalse%
%
\endisatagproof
{\isafoldproof}%
%
\isadelimproof
%
\endisadelimproof
%
\isadelimdocument
%
\endisadelimdocument
%
\isatagdocument
%
\isamarkupsubsection{Soundness%
}
\isamarkuptrue%
%
\endisatagdocument
{\isafolddocument}%
%
\isadelimdocument
%
\endisadelimdocument
\isacommand{theorem}\isamarkupfalse%
\ max{\isacharunderscore}{\kern0pt}agg{\isacharunderscore}{\kern0pt}sound{\isacharbrackleft}{\kern0pt}simp{\isacharbrackright}{\kern0pt}{\isacharcolon}{\kern0pt}\ {\isachardoublequoteopen}aggregator\ max{\isacharunderscore}{\kern0pt}aggregator{\isachardoublequoteclose}\isanewline
%
\isadelimproof
\ \ %
\endisadelimproof
%
\isatagproof
\isacommand{unfolding}\isamarkupfalse%
\ aggregator{\isacharunderscore}{\kern0pt}def\isanewline
\isacommand{proof}\isamarkupfalse%
\ {\isacharparenleft}{\kern0pt}simp{\isacharcomma}{\kern0pt}\ safe{\isacharparenright}{\kern0pt}\isanewline
\ \ \isacommand{fix}\isamarkupfalse%
\isanewline
\ \ \ \ A\ {\isacharcolon}{\kern0pt}{\isacharcolon}{\kern0pt}\ {\isachardoublequoteopen}{\isacharprime}{\kern0pt}a\ set{\isachardoublequoteclose}\ \isakeyword{and}\isanewline
\ \ \ \ e{\isadigit{1}}\ {\isacharcolon}{\kern0pt}{\isacharcolon}{\kern0pt}\ {\isachardoublequoteopen}{\isacharprime}{\kern0pt}a\ set{\isachardoublequoteclose}\ \isakeyword{and}\isanewline
\ \ \ \ e{\isadigit{2}}\ {\isacharcolon}{\kern0pt}{\isacharcolon}{\kern0pt}\ {\isachardoublequoteopen}{\isacharprime}{\kern0pt}a\ set{\isachardoublequoteclose}\ \isakeyword{and}\isanewline
\ \ \ \ d{\isadigit{1}}\ {\isacharcolon}{\kern0pt}{\isacharcolon}{\kern0pt}\ {\isachardoublequoteopen}{\isacharprime}{\kern0pt}a\ set{\isachardoublequoteclose}\ \isakeyword{and}\isanewline
\ \ \ \ d{\isadigit{2}}\ {\isacharcolon}{\kern0pt}{\isacharcolon}{\kern0pt}\ {\isachardoublequoteopen}{\isacharprime}{\kern0pt}a\ set{\isachardoublequoteclose}\ \isakeyword{and}\isanewline
\ \ \ \ r{\isadigit{1}}\ {\isacharcolon}{\kern0pt}{\isacharcolon}{\kern0pt}\ {\isachardoublequoteopen}{\isacharprime}{\kern0pt}a\ set{\isachardoublequoteclose}\ \isakeyword{and}\isanewline
\ \ \ \ r{\isadigit{2}}\ {\isacharcolon}{\kern0pt}{\isacharcolon}{\kern0pt}\ {\isachardoublequoteopen}{\isacharprime}{\kern0pt}a\ set{\isachardoublequoteclose}\ \isakeyword{and}\isanewline
\ \ \ \ x\ {\isacharcolon}{\kern0pt}{\isacharcolon}{\kern0pt}\ {\isachardoublequoteopen}{\isacharprime}{\kern0pt}a{\isachardoublequoteclose}\isanewline
\ \ \isacommand{assume}\isamarkupfalse%
\isanewline
\ \ \ \ asm{\isadigit{1}}{\isacharcolon}{\kern0pt}\ {\isachardoublequoteopen}e{\isadigit{2}}\ {\isasymunion}\ r{\isadigit{2}}\ {\isasymunion}\ d{\isadigit{2}}\ {\isacharequal}{\kern0pt}\ e{\isadigit{1}}\ {\isasymunion}\ r{\isadigit{1}}\ {\isasymunion}\ d{\isadigit{1}}{\isachardoublequoteclose}\ \isakeyword{and}\isanewline
\ \ \ \ asm{\isadigit{2}}{\isacharcolon}{\kern0pt}\ {\isachardoublequoteopen}x\ {\isasymnotin}\ d{\isadigit{1}}{\isachardoublequoteclose}\ \isakeyword{and}\isanewline
\ \ \ \ asm{\isadigit{3}}{\isacharcolon}{\kern0pt}\ {\isachardoublequoteopen}x\ {\isasymnotin}\ r{\isadigit{1}}{\isachardoublequoteclose}\ \isakeyword{and}\isanewline
\ \ \ \ asm{\isadigit{4}}{\isacharcolon}{\kern0pt}\ {\isachardoublequoteopen}x\ {\isasymin}\ e{\isadigit{2}}{\isachardoublequoteclose}\isanewline
\ \ \isacommand{show}\isamarkupfalse%
\ {\isachardoublequoteopen}x\ {\isasymin}\ e{\isadigit{1}}{\isachardoublequoteclose}\isanewline
\ \ \ \ \isacommand{using}\isamarkupfalse%
\ asm{\isadigit{1}}\ asm{\isadigit{2}}\ asm{\isadigit{3}}\ asm{\isadigit{4}}\isanewline
\ \ \ \ \isacommand{by}\isamarkupfalse%
\ auto\isanewline
\isacommand{next}\isamarkupfalse%
\isanewline
\ \ \isacommand{fix}\isamarkupfalse%
\isanewline
\ \ \ \ A\ {\isacharcolon}{\kern0pt}{\isacharcolon}{\kern0pt}\ {\isachardoublequoteopen}{\isacharprime}{\kern0pt}a\ set{\isachardoublequoteclose}\ \isakeyword{and}\isanewline
\ \ \ \ e{\isadigit{1}}\ {\isacharcolon}{\kern0pt}{\isacharcolon}{\kern0pt}\ {\isachardoublequoteopen}{\isacharprime}{\kern0pt}a\ set{\isachardoublequoteclose}\ \isakeyword{and}\isanewline
\ \ \ \ e{\isadigit{2}}\ {\isacharcolon}{\kern0pt}{\isacharcolon}{\kern0pt}\ {\isachardoublequoteopen}{\isacharprime}{\kern0pt}a\ set{\isachardoublequoteclose}\ \isakeyword{and}\isanewline
\ \ \ \ d{\isadigit{1}}\ {\isacharcolon}{\kern0pt}{\isacharcolon}{\kern0pt}\ {\isachardoublequoteopen}{\isacharprime}{\kern0pt}a\ set{\isachardoublequoteclose}\ \isakeyword{and}\isanewline
\ \ \ \ d{\isadigit{2}}\ {\isacharcolon}{\kern0pt}{\isacharcolon}{\kern0pt}\ {\isachardoublequoteopen}{\isacharprime}{\kern0pt}a\ set{\isachardoublequoteclose}\ \isakeyword{and}\isanewline
\ \ \ \ r{\isadigit{1}}\ {\isacharcolon}{\kern0pt}{\isacharcolon}{\kern0pt}\ {\isachardoublequoteopen}{\isacharprime}{\kern0pt}a\ set{\isachardoublequoteclose}\ \isakeyword{and}\isanewline
\ \ \ \ r{\isadigit{2}}\ {\isacharcolon}{\kern0pt}{\isacharcolon}{\kern0pt}\ {\isachardoublequoteopen}{\isacharprime}{\kern0pt}a\ set{\isachardoublequoteclose}\ \isakeyword{and}\isanewline
\ \ \ \ x\ {\isacharcolon}{\kern0pt}{\isacharcolon}{\kern0pt}\ {\isachardoublequoteopen}{\isacharprime}{\kern0pt}a{\isachardoublequoteclose}\isanewline
\ \ \isacommand{assume}\isamarkupfalse%
\isanewline
\ \ \ \ asm{\isadigit{1}}{\isacharcolon}{\kern0pt}\ {\isachardoublequoteopen}e{\isadigit{2}}\ {\isasymunion}\ r{\isadigit{2}}\ {\isasymunion}\ d{\isadigit{2}}\ {\isacharequal}{\kern0pt}\ e{\isadigit{1}}\ {\isasymunion}\ r{\isadigit{1}}\ {\isasymunion}\ d{\isadigit{1}}{\isachardoublequoteclose}\ \isakeyword{and}\isanewline
\ \ \ \ asm{\isadigit{2}}{\isacharcolon}{\kern0pt}\ {\isachardoublequoteopen}x\ {\isasymnotin}\ d{\isadigit{1}}{\isachardoublequoteclose}\ \isakeyword{and}\isanewline
\ \ \ \ asm{\isadigit{3}}{\isacharcolon}{\kern0pt}\ {\isachardoublequoteopen}x\ {\isasymnotin}\ r{\isadigit{1}}{\isachardoublequoteclose}\ \isakeyword{and}\isanewline
\ \ \ \ asm{\isadigit{4}}{\isacharcolon}{\kern0pt}\ {\isachardoublequoteopen}x\ {\isasymin}\ d{\isadigit{2}}{\isachardoublequoteclose}\isanewline
\ \ \isacommand{show}\isamarkupfalse%
\ {\isachardoublequoteopen}x\ {\isasymin}\ e{\isadigit{1}}{\isachardoublequoteclose}\isanewline
\ \ \ \ \isacommand{using}\isamarkupfalse%
\ asm{\isadigit{1}}\ asm{\isadigit{2}}\ asm{\isadigit{3}}\ asm{\isadigit{4}}\isanewline
\ \ \ \ \isacommand{by}\isamarkupfalse%
\ auto\isanewline
\isacommand{qed}\isamarkupfalse%
%
\endisatagproof
{\isafoldproof}%
%
\isadelimproof
%
\endisadelimproof
%
\isadelimdocument
%
\endisadelimdocument
%
\isatagdocument
%
\isamarkupsubsection{Properties%
}
\isamarkuptrue%
%
\endisatagdocument
{\isafolddocument}%
%
\isadelimdocument
%
\endisadelimdocument
\isacommand{theorem}\isamarkupfalse%
\ max{\isacharunderscore}{\kern0pt}agg{\isacharunderscore}{\kern0pt}consv{\isacharbrackleft}{\kern0pt}simp{\isacharbrackright}{\kern0pt}{\isacharcolon}{\kern0pt}\ {\isachardoublequoteopen}agg{\isacharunderscore}{\kern0pt}conservative\ max{\isacharunderscore}{\kern0pt}aggregator{\isachardoublequoteclose}\isanewline
%
\isadelimproof
%
\endisadelimproof
%
\isatagproof
\isacommand{proof}\isamarkupfalse%
\ {\isacharminus}{\kern0pt}\isanewline
\ \ \isacommand{have}\isamarkupfalse%
\isanewline
\ \ \ \ {\isachardoublequoteopen}{\isasymforall}A\ e{\isadigit{1}}\ e{\isadigit{2}}\ d{\isadigit{1}}\ d{\isadigit{2}}\ r{\isadigit{1}}\ r{\isadigit{2}}{\isachardot}{\kern0pt}\isanewline
\ \ \ \ \ \ \ \ \ \ {\isacharparenleft}{\kern0pt}well{\isacharunderscore}{\kern0pt}formed\ A\ {\isacharparenleft}{\kern0pt}e{\isadigit{1}}{\isacharcomma}{\kern0pt}\ r{\isadigit{1}}{\isacharcomma}{\kern0pt}\ d{\isadigit{1}}{\isacharparenright}{\kern0pt}\ {\isasymand}\ well{\isacharunderscore}{\kern0pt}formed\ A\ {\isacharparenleft}{\kern0pt}e{\isadigit{2}}{\isacharcomma}{\kern0pt}\ r{\isadigit{2}}{\isacharcomma}{\kern0pt}\ d{\isadigit{2}}{\isacharparenright}{\kern0pt}{\isacharparenright}{\kern0pt}\ {\isasymlongrightarrow}\isanewline
\ \ \ \ \ \ reject{\isacharunderscore}{\kern0pt}r\ {\isacharparenleft}{\kern0pt}max{\isacharunderscore}{\kern0pt}aggregator\ A\ {\isacharparenleft}{\kern0pt}e{\isadigit{1}}{\isacharcomma}{\kern0pt}\ r{\isadigit{1}}{\isacharcomma}{\kern0pt}\ d{\isadigit{1}}{\isacharparenright}{\kern0pt}\ {\isacharparenleft}{\kern0pt}e{\isadigit{2}}{\isacharcomma}{\kern0pt}\ r{\isadigit{2}}{\isacharcomma}{\kern0pt}\ d{\isadigit{2}}{\isacharparenright}{\kern0pt}{\isacharparenright}{\kern0pt}\ {\isacharequal}{\kern0pt}\ r{\isadigit{1}}\ {\isasyminter}\ r{\isadigit{2}}{\isachardoublequoteclose}\isanewline
\ \ \ \ \isacommand{using}\isamarkupfalse%
\ max{\isacharunderscore}{\kern0pt}agg{\isacharunderscore}{\kern0pt}rej{\isacharunderscore}{\kern0pt}set\isanewline
\ \ \ \ \isacommand{by}\isamarkupfalse%
\ blast\isanewline
\ \ \isacommand{hence}\isamarkupfalse%
\isanewline
\ \ \ \ {\isachardoublequoteopen}{\isasymforall}A\ e{\isadigit{1}}\ e{\isadigit{2}}\ d{\isadigit{1}}\ d{\isadigit{2}}\ r{\isadigit{1}}\ r{\isadigit{2}}{\isachardot}{\kern0pt}\isanewline
\ \ \ \ \ \ \ \ \ \ \ \ {\isacharparenleft}{\kern0pt}well{\isacharunderscore}{\kern0pt}formed\ A\ {\isacharparenleft}{\kern0pt}e{\isadigit{1}}{\isacharcomma}{\kern0pt}\ r{\isadigit{1}}{\isacharcomma}{\kern0pt}\ d{\isadigit{1}}{\isacharparenright}{\kern0pt}\ {\isasymand}\ well{\isacharunderscore}{\kern0pt}formed\ A\ {\isacharparenleft}{\kern0pt}e{\isadigit{2}}{\isacharcomma}{\kern0pt}\ r{\isadigit{2}}{\isacharcomma}{\kern0pt}\ d{\isadigit{2}}{\isacharparenright}{\kern0pt}{\isacharparenright}{\kern0pt}\ {\isasymlongrightarrow}\isanewline
\ \ \ \ \ \ \ \ reject{\isacharunderscore}{\kern0pt}r\ {\isacharparenleft}{\kern0pt}max{\isacharunderscore}{\kern0pt}aggregator\ A\ {\isacharparenleft}{\kern0pt}e{\isadigit{1}}{\isacharcomma}{\kern0pt}\ r{\isadigit{1}}{\isacharcomma}{\kern0pt}\ d{\isadigit{1}}{\isacharparenright}{\kern0pt}\ {\isacharparenleft}{\kern0pt}e{\isadigit{2}}{\isacharcomma}{\kern0pt}\ r{\isadigit{2}}{\isacharcomma}{\kern0pt}\ d{\isadigit{2}}{\isacharparenright}{\kern0pt}{\isacharparenright}{\kern0pt}\ {\isasymsubseteq}\ r{\isadigit{1}}\ {\isasyminter}\ r{\isadigit{2}}{\isachardoublequoteclose}\isanewline
\ \ \ \ \isacommand{by}\isamarkupfalse%
\ blast\isanewline
\ \ \isacommand{moreover}\isamarkupfalse%
\ \isacommand{have}\isamarkupfalse%
\isanewline
\ \ \ \ {\isachardoublequoteopen}{\isasymforall}A\ e{\isadigit{1}}\ e{\isadigit{2}}\ d{\isadigit{1}}\ d{\isadigit{2}}\ r{\isadigit{1}}\ r{\isadigit{2}}{\isachardot}{\kern0pt}\isanewline
\ \ \ \ \ \ \ \ {\isacharparenleft}{\kern0pt}well{\isacharunderscore}{\kern0pt}formed\ A\ {\isacharparenleft}{\kern0pt}e{\isadigit{1}}{\isacharcomma}{\kern0pt}\ r{\isadigit{1}}{\isacharcomma}{\kern0pt}\ d{\isadigit{1}}{\isacharparenright}{\kern0pt}\ {\isasymand}\ well{\isacharunderscore}{\kern0pt}formed\ A\ {\isacharparenleft}{\kern0pt}e{\isadigit{2}}{\isacharcomma}{\kern0pt}\ r{\isadigit{2}}{\isacharcomma}{\kern0pt}\ d{\isadigit{2}}{\isacharparenright}{\kern0pt}{\isacharparenright}{\kern0pt}\ {\isasymlongrightarrow}\isanewline
\ \ \ \ \ \ \ \ \ \ \ \ elect{\isacharunderscore}{\kern0pt}r\ {\isacharparenleft}{\kern0pt}max{\isacharunderscore}{\kern0pt}aggregator\ A\ {\isacharparenleft}{\kern0pt}e{\isadigit{1}}{\isacharcomma}{\kern0pt}\ r{\isadigit{1}}{\isacharcomma}{\kern0pt}\ d{\isadigit{1}}{\isacharparenright}{\kern0pt}\ {\isacharparenleft}{\kern0pt}e{\isadigit{2}}{\isacharcomma}{\kern0pt}\ r{\isadigit{2}}{\isacharcomma}{\kern0pt}\ d{\isadigit{2}}{\isacharparenright}{\kern0pt}{\isacharparenright}{\kern0pt}\ {\isasymsubseteq}\ {\isacharparenleft}{\kern0pt}e{\isadigit{1}}\ {\isasymunion}\ e{\isadigit{2}}{\isacharparenright}{\kern0pt}{\isachardoublequoteclose}\isanewline
\ \ \ \ \isacommand{by}\isamarkupfalse%
\ {\isacharparenleft}{\kern0pt}simp\ add{\isacharcolon}{\kern0pt}\ subset{\isacharunderscore}{\kern0pt}eq{\isacharparenright}{\kern0pt}\isanewline
\ \ \isacommand{ultimately}\isamarkupfalse%
\ \isacommand{have}\isamarkupfalse%
\isanewline
\ \ \ \ {\isachardoublequoteopen}{\isasymforall}A\ e{\isadigit{1}}\ e{\isadigit{2}}\ d{\isadigit{1}}\ d{\isadigit{2}}\ r{\isadigit{1}}\ r{\isadigit{2}}{\isachardot}{\kern0pt}\isanewline
\ \ \ \ \ \ \ \ {\isacharparenleft}{\kern0pt}well{\isacharunderscore}{\kern0pt}formed\ A\ {\isacharparenleft}{\kern0pt}e{\isadigit{1}}{\isacharcomma}{\kern0pt}\ r{\isadigit{1}}{\isacharcomma}{\kern0pt}\ d{\isadigit{1}}{\isacharparenright}{\kern0pt}\ {\isasymand}\ well{\isacharunderscore}{\kern0pt}formed\ A\ {\isacharparenleft}{\kern0pt}e{\isadigit{2}}{\isacharcomma}{\kern0pt}\ r{\isadigit{2}}{\isacharcomma}{\kern0pt}\ d{\isadigit{2}}{\isacharparenright}{\kern0pt}{\isacharparenright}{\kern0pt}\ {\isasymlongrightarrow}\isanewline
\ \ \ \ \ \ \ \ \ \ \ \ {\isacharparenleft}{\kern0pt}elect{\isacharunderscore}{\kern0pt}r\ {\isacharparenleft}{\kern0pt}max{\isacharunderscore}{\kern0pt}aggregator\ A\ {\isacharparenleft}{\kern0pt}e{\isadigit{1}}{\isacharcomma}{\kern0pt}\ r{\isadigit{1}}{\isacharcomma}{\kern0pt}\ d{\isadigit{1}}{\isacharparenright}{\kern0pt}\ {\isacharparenleft}{\kern0pt}e{\isadigit{2}}{\isacharcomma}{\kern0pt}\ r{\isadigit{2}}{\isacharcomma}{\kern0pt}\ d{\isadigit{2}}{\isacharparenright}{\kern0pt}{\isacharparenright}{\kern0pt}\ {\isasymsubseteq}\ {\isacharparenleft}{\kern0pt}e{\isadigit{1}}\ {\isasymunion}\ e{\isadigit{2}}{\isacharparenright}{\kern0pt}\ {\isasymand}\isanewline
\ \ \ \ \ \ \ \ \ \ \ \ \ reject{\isacharunderscore}{\kern0pt}r\ {\isacharparenleft}{\kern0pt}max{\isacharunderscore}{\kern0pt}aggregator\ A\ {\isacharparenleft}{\kern0pt}e{\isadigit{1}}{\isacharcomma}{\kern0pt}\ r{\isadigit{1}}{\isacharcomma}{\kern0pt}\ d{\isadigit{1}}{\isacharparenright}{\kern0pt}\ {\isacharparenleft}{\kern0pt}e{\isadigit{2}}{\isacharcomma}{\kern0pt}\ r{\isadigit{2}}{\isacharcomma}{\kern0pt}\ d{\isadigit{2}}{\isacharparenright}{\kern0pt}{\isacharparenright}{\kern0pt}\ {\isasymsubseteq}\ {\isacharparenleft}{\kern0pt}r{\isadigit{1}}\ {\isasymunion}\ r{\isadigit{2}}{\isacharparenright}{\kern0pt}{\isacharparenright}{\kern0pt}{\isachardoublequoteclose}\isanewline
\ \ \ \ \isacommand{by}\isamarkupfalse%
\ blast\isanewline
\ \ \isacommand{moreover}\isamarkupfalse%
\ \isacommand{have}\isamarkupfalse%
\isanewline
\ \ \ \ {\isachardoublequoteopen}{\isasymforall}A\ e{\isadigit{1}}\ e{\isadigit{2}}\ d{\isadigit{1}}\ d{\isadigit{2}}\ r{\isadigit{1}}\ r{\isadigit{2}}{\isachardot}{\kern0pt}\isanewline
\ \ \ \ \ \ \ \ {\isacharparenleft}{\kern0pt}well{\isacharunderscore}{\kern0pt}formed\ A\ {\isacharparenleft}{\kern0pt}e{\isadigit{1}}{\isacharcomma}{\kern0pt}\ r{\isadigit{1}}{\isacharcomma}{\kern0pt}\ d{\isadigit{1}}{\isacharparenright}{\kern0pt}\ {\isasymand}\ well{\isacharunderscore}{\kern0pt}formed\ A\ {\isacharparenleft}{\kern0pt}e{\isadigit{2}}{\isacharcomma}{\kern0pt}\ r{\isadigit{2}}{\isacharcomma}{\kern0pt}\ d{\isadigit{2}}{\isacharparenright}{\kern0pt}{\isacharparenright}{\kern0pt}\ {\isasymlongrightarrow}\isanewline
\ \ \ \ \ \ \ \ \ \ \ \ defer{\isacharunderscore}{\kern0pt}r\ {\isacharparenleft}{\kern0pt}max{\isacharunderscore}{\kern0pt}aggregator\ A\ {\isacharparenleft}{\kern0pt}e{\isadigit{1}}{\isacharcomma}{\kern0pt}\ r{\isadigit{1}}{\isacharcomma}{\kern0pt}\ d{\isadigit{1}}{\isacharparenright}{\kern0pt}\ {\isacharparenleft}{\kern0pt}e{\isadigit{2}}{\isacharcomma}{\kern0pt}\ r{\isadigit{2}}{\isacharcomma}{\kern0pt}\ d{\isadigit{2}}{\isacharparenright}{\kern0pt}{\isacharparenright}{\kern0pt}\ {\isasymsubseteq}\ {\isacharparenleft}{\kern0pt}d{\isadigit{1}}\ {\isasymunion}\ d{\isadigit{2}}{\isacharparenright}{\kern0pt}{\isachardoublequoteclose}\isanewline
\ \ \ \ \isacommand{by}\isamarkupfalse%
\ auto\isanewline
\ \ \isacommand{ultimately}\isamarkupfalse%
\ \isacommand{have}\isamarkupfalse%
\isanewline
\ \ \ \ {\isachardoublequoteopen}{\isasymforall}A\ e{\isadigit{1}}\ e{\isadigit{2}}\ d{\isadigit{1}}\ d{\isadigit{2}}\ r{\isadigit{1}}\ r{\isadigit{2}}{\isachardot}{\kern0pt}\isanewline
\ \ \ \ \ \ \ \ {\isacharparenleft}{\kern0pt}well{\isacharunderscore}{\kern0pt}formed\ A\ {\isacharparenleft}{\kern0pt}e{\isadigit{1}}{\isacharcomma}{\kern0pt}\ r{\isadigit{1}}{\isacharcomma}{\kern0pt}\ d{\isadigit{1}}{\isacharparenright}{\kern0pt}\ {\isasymand}\ well{\isacharunderscore}{\kern0pt}formed\ A\ {\isacharparenleft}{\kern0pt}e{\isadigit{2}}{\isacharcomma}{\kern0pt}\ r{\isadigit{2}}{\isacharcomma}{\kern0pt}\ d{\isadigit{2}}{\isacharparenright}{\kern0pt}{\isacharparenright}{\kern0pt}\ {\isasymlongrightarrow}\isanewline
\ \ \ \ \ \ \ \ \ \ \ \ {\isacharparenleft}{\kern0pt}elect{\isacharunderscore}{\kern0pt}r\ {\isacharparenleft}{\kern0pt}max{\isacharunderscore}{\kern0pt}aggregator\ A\ {\isacharparenleft}{\kern0pt}e{\isadigit{1}}{\isacharcomma}{\kern0pt}\ r{\isadigit{1}}{\isacharcomma}{\kern0pt}\ d{\isadigit{1}}{\isacharparenright}{\kern0pt}\ {\isacharparenleft}{\kern0pt}e{\isadigit{2}}{\isacharcomma}{\kern0pt}\ r{\isadigit{2}}{\isacharcomma}{\kern0pt}\ d{\isadigit{2}}{\isacharparenright}{\kern0pt}{\isacharparenright}{\kern0pt}\ {\isasymsubseteq}\ {\isacharparenleft}{\kern0pt}e{\isadigit{1}}\ {\isasymunion}\ e{\isadigit{2}}{\isacharparenright}{\kern0pt}\ {\isasymand}\isanewline
\ \ \ \ \ \ \ \ \ \ \ \ reject{\isacharunderscore}{\kern0pt}r\ {\isacharparenleft}{\kern0pt}max{\isacharunderscore}{\kern0pt}aggregator\ A\ {\isacharparenleft}{\kern0pt}e{\isadigit{1}}{\isacharcomma}{\kern0pt}\ r{\isadigit{1}}{\isacharcomma}{\kern0pt}\ d{\isadigit{1}}{\isacharparenright}{\kern0pt}\ {\isacharparenleft}{\kern0pt}e{\isadigit{2}}{\isacharcomma}{\kern0pt}\ r{\isadigit{2}}{\isacharcomma}{\kern0pt}\ d{\isadigit{2}}{\isacharparenright}{\kern0pt}{\isacharparenright}{\kern0pt}\ {\isasymsubseteq}\ {\isacharparenleft}{\kern0pt}r{\isadigit{1}}\ {\isasymunion}\ r{\isadigit{2}}{\isacharparenright}{\kern0pt}\ {\isasymand}\isanewline
\ \ \ \ \ \ \ \ \ \ \ \ defer{\isacharunderscore}{\kern0pt}r\ {\isacharparenleft}{\kern0pt}max{\isacharunderscore}{\kern0pt}aggregator\ A\ {\isacharparenleft}{\kern0pt}e{\isadigit{1}}{\isacharcomma}{\kern0pt}\ r{\isadigit{1}}{\isacharcomma}{\kern0pt}\ d{\isadigit{1}}{\isacharparenright}{\kern0pt}\ {\isacharparenleft}{\kern0pt}e{\isadigit{2}}{\isacharcomma}{\kern0pt}\ r{\isadigit{2}}{\isacharcomma}{\kern0pt}\ d{\isadigit{2}}{\isacharparenright}{\kern0pt}{\isacharparenright}{\kern0pt}\ {\isasymsubseteq}\ {\isacharparenleft}{\kern0pt}d{\isadigit{1}}\ {\isasymunion}\ d{\isadigit{2}}{\isacharparenright}{\kern0pt}{\isacharparenright}{\kern0pt}{\isachardoublequoteclose}\isanewline
\ \ \ \ \isacommand{by}\isamarkupfalse%
\ blast\isanewline
\ \ \isacommand{thus}\isamarkupfalse%
\ {\isacharquery}{\kern0pt}thesis\isanewline
\ \ \ \ \isacommand{by}\isamarkupfalse%
\ {\isacharparenleft}{\kern0pt}simp\ add{\isacharcolon}{\kern0pt}\ agg{\isacharunderscore}{\kern0pt}conservative{\isacharunderscore}{\kern0pt}def{\isacharparenright}{\kern0pt}\isanewline
\isacommand{qed}\isamarkupfalse%
%
\endisatagproof
{\isafoldproof}%
%
\isadelimproof
\isanewline
%
\endisadelimproof
\isanewline
\isanewline
\isacommand{theorem}\isamarkupfalse%
\ max{\isacharunderscore}{\kern0pt}agg{\isacharunderscore}{\kern0pt}comm{\isacharbrackleft}{\kern0pt}simp{\isacharbrackright}{\kern0pt}{\isacharcolon}{\kern0pt}\ {\isachardoublequoteopen}agg{\isacharunderscore}{\kern0pt}commutative\ max{\isacharunderscore}{\kern0pt}aggregator{\isachardoublequoteclose}\isanewline
%
\isadelimproof
\ \ %
\endisadelimproof
%
\isatagproof
\isacommand{unfolding}\isamarkupfalse%
\ agg{\isacharunderscore}{\kern0pt}commutative{\isacharunderscore}{\kern0pt}def\isanewline
\isacommand{proof}\isamarkupfalse%
\ {\isacharparenleft}{\kern0pt}safe{\isacharparenright}{\kern0pt}\isanewline
\ \ \isacommand{show}\isamarkupfalse%
\ {\isachardoublequoteopen}aggregator\ max{\isacharunderscore}{\kern0pt}aggregator{\isachardoublequoteclose}\isanewline
\ \ \ \ \isacommand{by}\isamarkupfalse%
\ simp\isanewline
\isacommand{next}\isamarkupfalse%
\isanewline
\ \ \isacommand{fix}\isamarkupfalse%
\isanewline
\ \ \ \ A\ {\isacharcolon}{\kern0pt}{\isacharcolon}{\kern0pt}\ {\isachardoublequoteopen}{\isacharprime}{\kern0pt}a\ set{\isachardoublequoteclose}\ \isakeyword{and}\isanewline
\ \ \ \ e{\isadigit{1}}\ {\isacharcolon}{\kern0pt}{\isacharcolon}{\kern0pt}\ {\isachardoublequoteopen}{\isacharprime}{\kern0pt}a\ set{\isachardoublequoteclose}\ \isakeyword{and}\isanewline
\ \ \ \ e{\isadigit{2}}\ {\isacharcolon}{\kern0pt}{\isacharcolon}{\kern0pt}\ {\isachardoublequoteopen}{\isacharprime}{\kern0pt}a\ set{\isachardoublequoteclose}\ \isakeyword{and}\isanewline
\ \ \ \ d{\isadigit{1}}\ {\isacharcolon}{\kern0pt}{\isacharcolon}{\kern0pt}\ {\isachardoublequoteopen}{\isacharprime}{\kern0pt}a\ set{\isachardoublequoteclose}\ \isakeyword{and}\isanewline
\ \ \ \ d{\isadigit{2}}\ {\isacharcolon}{\kern0pt}{\isacharcolon}{\kern0pt}\ {\isachardoublequoteopen}{\isacharprime}{\kern0pt}a\ set{\isachardoublequoteclose}\ \isakeyword{and}\isanewline
\ \ \ \ r{\isadigit{1}}\ {\isacharcolon}{\kern0pt}{\isacharcolon}{\kern0pt}\ {\isachardoublequoteopen}{\isacharprime}{\kern0pt}a\ set{\isachardoublequoteclose}\ \isakeyword{and}\isanewline
\ \ \ \ r{\isadigit{2}}\ {\isacharcolon}{\kern0pt}{\isacharcolon}{\kern0pt}\ {\isachardoublequoteopen}{\isacharprime}{\kern0pt}a\ set{\isachardoublequoteclose}\isanewline
\ \ \isacommand{show}\isamarkupfalse%
\isanewline
\ \ \ \ {\isachardoublequoteopen}max{\isacharunderscore}{\kern0pt}aggregator\ A\ {\isacharparenleft}{\kern0pt}e{\isadigit{1}}{\isacharcomma}{\kern0pt}\ r{\isadigit{1}}{\isacharcomma}{\kern0pt}\ d{\isadigit{1}}{\isacharparenright}{\kern0pt}\ {\isacharparenleft}{\kern0pt}e{\isadigit{2}}{\isacharcomma}{\kern0pt}\ r{\isadigit{2}}{\isacharcomma}{\kern0pt}\ d{\isadigit{2}}{\isacharparenright}{\kern0pt}\ {\isacharequal}{\kern0pt}\isanewline
\ \ \ \ \ \ max{\isacharunderscore}{\kern0pt}aggregator\ A\ {\isacharparenleft}{\kern0pt}e{\isadigit{2}}{\isacharcomma}{\kern0pt}\ r{\isadigit{2}}{\isacharcomma}{\kern0pt}\ d{\isadigit{2}}{\isacharparenright}{\kern0pt}\ {\isacharparenleft}{\kern0pt}e{\isadigit{1}}{\isacharcomma}{\kern0pt}\ r{\isadigit{1}}{\isacharcomma}{\kern0pt}\ d{\isadigit{1}}{\isacharparenright}{\kern0pt}{\isachardoublequoteclose}\isanewline
\ \ \isacommand{by}\isamarkupfalse%
\ auto\isanewline
\isacommand{qed}\isamarkupfalse%
%
\endisatagproof
{\isafoldproof}%
%
\isadelimproof
\isanewline
%
\endisadelimproof
%
\isadelimtheory
\isanewline
%
\endisadelimtheory
%
\isatagtheory
\isacommand{end}\isamarkupfalse%
%
\endisatagtheory
{\isafoldtheory}%
%
\isadelimtheory
%
\endisadelimtheory
%
\end{isabellebody}%
\endinput
%:%file=~/Documents/Studies/VotingRuleGenerator/virage/src/test/resources/old_theories/Compositional_Structures/Basic_Modules/Component_Types/Maximum_Aggregator.thy%:%
%:%6=3%:%
%:%11=4%:%
%:%12=5%:%
%:%14=8%:%
%:%30=10%:%
%:%31=10%:%
%:%32=11%:%
%:%33=12%:%
%:%42=15%:%
%:%43=16%:%
%:%44=17%:%
%:%53=19%:%
%:%63=21%:%
%:%64=21%:%
%:%65=22%:%
%:%75=27%:%
%:%85=29%:%
%:%86=29%:%
%:%88=31%:%
%:%95=32%:%
%:%96=32%:%
%:%97=33%:%
%:%98=33%:%
%:%99=34%:%
%:%100=34%:%
%:%101=35%:%
%:%102=35%:%
%:%103=35%:%
%:%104=36%:%
%:%105=37%:%
%:%106=37%:%
%:%107=38%:%
%:%108=38%:%
%:%109=38%:%
%:%110=39%:%
%:%111=40%:%
%:%112=41%:%
%:%113=41%:%
%:%114=42%:%
%:%115=42%:%
%:%116=42%:%
%:%117=43%:%
%:%118=44%:%
%:%119=44%:%
%:%120=45%:%
%:%121=45%:%
%:%122=45%:%
%:%123=46%:%
%:%124=46%:%
%:%125=47%:%
%:%140=49%:%
%:%150=51%:%
%:%151=51%:%
%:%154=52%:%
%:%158=52%:%
%:%159=52%:%
%:%160=53%:%
%:%161=53%:%
%:%162=54%:%
%:%163=54%:%
%:%164=55%:%
%:%165=56%:%
%:%166=57%:%
%:%167=58%:%
%:%168=59%:%
%:%169=60%:%
%:%170=61%:%
%:%171=62%:%
%:%172=63%:%
%:%173=63%:%
%:%174=64%:%
%:%175=65%:%
%:%176=66%:%
%:%177=67%:%
%:%178=68%:%
%:%179=68%:%
%:%180=69%:%
%:%181=69%:%
%:%182=70%:%
%:%183=70%:%
%:%184=71%:%
%:%185=71%:%
%:%186=72%:%
%:%187=72%:%
%:%188=73%:%
%:%189=74%:%
%:%190=75%:%
%:%191=76%:%
%:%192=77%:%
%:%193=78%:%
%:%194=79%:%
%:%195=80%:%
%:%196=81%:%
%:%197=81%:%
%:%198=82%:%
%:%199=83%:%
%:%200=84%:%
%:%201=85%:%
%:%202=86%:%
%:%203=86%:%
%:%204=87%:%
%:%205=87%:%
%:%206=88%:%
%:%207=88%:%
%:%208=89%:%
%:%223=91%:%
%:%233=94%:%
%:%234=94%:%
%:%241=95%:%
%:%242=95%:%
%:%243=96%:%
%:%244=96%:%
%:%245=97%:%
%:%247=99%:%
%:%248=100%:%
%:%249=100%:%
%:%250=101%:%
%:%251=101%:%
%:%252=102%:%
%:%253=102%:%
%:%254=103%:%
%:%256=105%:%
%:%257=106%:%
%:%258=106%:%
%:%259=107%:%
%:%260=107%:%
%:%261=107%:%
%:%262=108%:%
%:%264=110%:%
%:%265=111%:%
%:%266=111%:%
%:%267=112%:%
%:%268=112%:%
%:%269=112%:%
%:%270=113%:%
%:%273=116%:%
%:%274=117%:%
%:%275=117%:%
%:%276=118%:%
%:%277=118%:%
%:%278=118%:%
%:%279=119%:%
%:%281=121%:%
%:%282=122%:%
%:%283=122%:%
%:%284=123%:%
%:%285=123%:%
%:%286=123%:%
%:%287=124%:%
%:%291=128%:%
%:%292=129%:%
%:%293=129%:%
%:%294=130%:%
%:%295=130%:%
%:%296=131%:%
%:%297=131%:%
%:%298=132%:%
%:%304=132%:%
%:%307=133%:%
%:%308=134%:%
%:%309=135%:%
%:%310=135%:%
%:%313=136%:%
%:%317=136%:%
%:%318=136%:%
%:%319=137%:%
%:%320=137%:%
%:%321=138%:%
%:%322=138%:%
%:%323=139%:%
%:%324=139%:%
%:%325=140%:%
%:%326=140%:%
%:%327=141%:%
%:%328=141%:%
%:%329=142%:%
%:%330=143%:%
%:%331=144%:%
%:%332=145%:%
%:%333=146%:%
%:%334=147%:%
%:%335=148%:%
%:%336=149%:%
%:%337=149%:%
%:%338=150%:%
%:%339=151%:%
%:%340=152%:%
%:%341=152%:%
%:%342=153%:%
%:%348=153%:%
%:%353=154%:%
%:%358=155%:%
%
\begin{isabellebody}%
\setisabellecontext{Plurality{\isacharunderscore}{\kern0pt}Module}%
%
\isadelimdocument
\isanewline
%
\endisadelimdocument
%
\isatagdocument
\isanewline
\isanewline
%
\isamarkupsection{Plurality Module%
}
\isamarkuptrue%
%
\endisatagdocument
{\isafolddocument}%
%
\isadelimdocument
%
\endisadelimdocument
%
\isadelimtheory
%
\endisadelimtheory
%
\isatagtheory
\isacommand{theory}\isamarkupfalse%
\ Plurality{\isacharunderscore}{\kern0pt}Module\isanewline
\ \ \isakeyword{imports}\ {\isachardoublequoteopen}{\isachardot}{\kern0pt}{\isachardot}{\kern0pt}{\isacharslash}{\kern0pt}Electoral{\isacharunderscore}{\kern0pt}Module{\isachardoublequoteclose}\isanewline
\isakeyword{begin}%
\endisatagtheory
{\isafoldtheory}%
%
\isadelimtheory
%
\endisadelimtheory
%
\begin{isamarkuptext}%
The plurality module implements the plurality voting rule.
The plurality rule elects all modules with the maximum amount of top
preferences among all alternatives, and rejects all the other alternatives.
It is electing and induces the classical plurality (voting) rule
from social-choice theory.%
\end{isamarkuptext}\isamarkuptrue%
%
\isadelimdocument
%
\endisadelimdocument
%
\isatagdocument
%
\isamarkupsubsection{Definition%
}
\isamarkuptrue%
%
\endisatagdocument
{\isafolddocument}%
%
\isadelimdocument
%
\endisadelimdocument
\isacommand{fun}\isamarkupfalse%
\ plurality\ {\isacharcolon}{\kern0pt}{\isacharcolon}{\kern0pt}\ {\isachardoublequoteopen}{\isacharprime}{\kern0pt}a\ Electoral{\isacharunderscore}{\kern0pt}Module{\isachardoublequoteclose}\ \isakeyword{where}\isanewline
\ \ {\isachardoublequoteopen}plurality\ A\ p\ {\isacharequal}{\kern0pt}\isanewline
\ \ \ \ {\isacharparenleft}{\kern0pt}{\isacharbraceleft}{\kern0pt}a\ {\isasymin}\ A{\isachardot}{\kern0pt}\ {\isasymforall}x\ {\isasymin}\ A{\isachardot}{\kern0pt}\ win{\isacharunderscore}{\kern0pt}count\ p\ x\ {\isasymle}\ win{\isacharunderscore}{\kern0pt}count\ p\ a{\isacharbraceright}{\kern0pt}{\isacharcomma}{\kern0pt}\isanewline
\ \ \ \ \ {\isacharbraceleft}{\kern0pt}a\ {\isasymin}\ A{\isachardot}{\kern0pt}\ {\isasymexists}x\ {\isasymin}\ A{\isachardot}{\kern0pt}\ win{\isacharunderscore}{\kern0pt}count\ p\ x\ {\isachargreater}{\kern0pt}\ win{\isacharunderscore}{\kern0pt}count\ p\ a{\isacharbraceright}{\kern0pt}{\isacharcomma}{\kern0pt}\isanewline
\ \ \ \ \ {\isacharbraceleft}{\kern0pt}{\isacharbraceright}{\kern0pt}{\isacharparenright}{\kern0pt}{\isachardoublequoteclose}%
\isadelimdocument
%
\endisadelimdocument
%
\isatagdocument
%
\isamarkupsubsection{Soundness%
}
\isamarkuptrue%
%
\endisatagdocument
{\isafolddocument}%
%
\isadelimdocument
%
\endisadelimdocument
\isacommand{theorem}\isamarkupfalse%
\ plurality{\isacharunderscore}{\kern0pt}sound{\isacharbrackleft}{\kern0pt}simp{\isacharbrackright}{\kern0pt}{\isacharcolon}{\kern0pt}\ {\isachardoublequoteopen}electoral{\isacharunderscore}{\kern0pt}module\ plurality{\isachardoublequoteclose}\isanewline
%
\isadelimproof
%
\endisadelimproof
%
\isatagproof
\isacommand{proof}\isamarkupfalse%
\ {\isacharminus}{\kern0pt}\isanewline
\ \ \isacommand{have}\isamarkupfalse%
\isanewline
\ \ \ \ {\isachardoublequoteopen}{\isasymforall}A\ p{\isachardot}{\kern0pt}\isanewline
\ \ \ \ \ \ let\ elect\ {\isacharequal}{\kern0pt}\ {\isacharbraceleft}{\kern0pt}a\ {\isasymin}\ {\isacharparenleft}{\kern0pt}A{\isacharcolon}{\kern0pt}{\isacharcolon}{\kern0pt}{\isacharprime}{\kern0pt}a\ set{\isacharparenright}{\kern0pt}{\isachardot}{\kern0pt}\ {\isasymforall}x\ {\isasymin}\ A{\isachardot}{\kern0pt}\ win{\isacharunderscore}{\kern0pt}count\ p\ x\ {\isasymle}\ win{\isacharunderscore}{\kern0pt}count\ p\ a{\isacharbraceright}{\kern0pt}{\isacharsemicolon}{\kern0pt}\isanewline
\ \ \ \ \ \ reject\ {\isacharequal}{\kern0pt}\ {\isacharbraceleft}{\kern0pt}a\ {\isasymin}\ A{\isachardot}{\kern0pt}\ {\isasymexists}x\ {\isasymin}\ A{\isachardot}{\kern0pt}\ win{\isacharunderscore}{\kern0pt}count\ p\ x\ {\isachargreater}{\kern0pt}\ win{\isacharunderscore}{\kern0pt}count\ p\ a{\isacharbraceright}{\kern0pt}\ in\isanewline
\ \ \ \ elect\ {\isasyminter}\ reject\ {\isacharequal}{\kern0pt}\ {\isacharbraceleft}{\kern0pt}{\isacharbraceright}{\kern0pt}{\isachardoublequoteclose}\isanewline
\ \ \ \ \isacommand{by}\isamarkupfalse%
\ auto\isanewline
\ \ \isacommand{hence}\isamarkupfalse%
\ disjoint{\isacharcolon}{\kern0pt}\isanewline
\ \ \ \ {\isachardoublequoteopen}{\isasymforall}A\ p{\isachardot}{\kern0pt}\isanewline
\ \ \ \ \ \ let\ elect\ {\isacharequal}{\kern0pt}\ {\isacharbraceleft}{\kern0pt}a\ {\isasymin}\ {\isacharparenleft}{\kern0pt}A{\isacharcolon}{\kern0pt}{\isacharcolon}{\kern0pt}{\isacharprime}{\kern0pt}a\ set{\isacharparenright}{\kern0pt}{\isachardot}{\kern0pt}\ {\isasymforall}x\ {\isasymin}\ A{\isachardot}{\kern0pt}\ win{\isacharunderscore}{\kern0pt}count\ p\ x\ {\isasymle}\ win{\isacharunderscore}{\kern0pt}count\ p\ a{\isacharbraceright}{\kern0pt}{\isacharsemicolon}{\kern0pt}\isanewline
\ \ \ \ \ \ reject\ {\isacharequal}{\kern0pt}\ {\isacharbraceleft}{\kern0pt}a\ {\isasymin}\ A{\isachardot}{\kern0pt}\ {\isasymexists}x\ {\isasymin}\ A{\isachardot}{\kern0pt}\ win{\isacharunderscore}{\kern0pt}count\ p\ x\ {\isachargreater}{\kern0pt}\ win{\isacharunderscore}{\kern0pt}count\ p\ a{\isacharbraceright}{\kern0pt}\ in\isanewline
\ \ \ \ disjoint{\isadigit{3}}\ {\isacharparenleft}{\kern0pt}elect{\isacharcomma}{\kern0pt}\ reject{\isacharcomma}{\kern0pt}\ {\isacharbraceleft}{\kern0pt}{\isacharbraceright}{\kern0pt}{\isacharparenright}{\kern0pt}{\isachardoublequoteclose}\isanewline
\ \ \ \ \isacommand{by}\isamarkupfalse%
\ simp\isanewline
\ \ \isacommand{have}\isamarkupfalse%
\isanewline
\ \ \ \ {\isachardoublequoteopen}{\isasymforall}A\ p{\isachardot}{\kern0pt}\isanewline
\ \ \ \ \ \ let\ elect\ {\isacharequal}{\kern0pt}\ {\isacharbraceleft}{\kern0pt}a\ {\isasymin}\ {\isacharparenleft}{\kern0pt}A{\isacharcolon}{\kern0pt}{\isacharcolon}{\kern0pt}{\isacharprime}{\kern0pt}a\ set{\isacharparenright}{\kern0pt}{\isachardot}{\kern0pt}\ {\isasymforall}x\ {\isasymin}\ A{\isachardot}{\kern0pt}\ win{\isacharunderscore}{\kern0pt}count\ p\ x\ {\isasymle}\ win{\isacharunderscore}{\kern0pt}count\ p\ a{\isacharbraceright}{\kern0pt}{\isacharsemicolon}{\kern0pt}\isanewline
\ \ \ \ \ \ reject\ {\isacharequal}{\kern0pt}\ {\isacharbraceleft}{\kern0pt}a\ {\isasymin}\ A{\isachardot}{\kern0pt}\ {\isasymexists}x\ {\isasymin}\ A{\isachardot}{\kern0pt}\ win{\isacharunderscore}{\kern0pt}count\ p\ x\ {\isachargreater}{\kern0pt}\ win{\isacharunderscore}{\kern0pt}count\ p\ a{\isacharbraceright}{\kern0pt}\ in\isanewline
\ \ \ \ elect\ {\isasymunion}\ reject\ {\isacharequal}{\kern0pt}\ A{\isachardoublequoteclose}\isanewline
\ \ \ \ \isacommand{using}\isamarkupfalse%
\ not{\isacharunderscore}{\kern0pt}le{\isacharunderscore}{\kern0pt}imp{\isacharunderscore}{\kern0pt}less\isanewline
\ \ \ \ \isacommand{by}\isamarkupfalse%
\ auto\isanewline
\ \ \isacommand{hence}\isamarkupfalse%
\ unity{\isacharcolon}{\kern0pt}\isanewline
\ \ \ \ {\isachardoublequoteopen}{\isasymforall}A\ p{\isachardot}{\kern0pt}\isanewline
\ \ \ \ \ \ let\ elect\ {\isacharequal}{\kern0pt}\ {\isacharbraceleft}{\kern0pt}a\ {\isasymin}\ {\isacharparenleft}{\kern0pt}A{\isacharcolon}{\kern0pt}{\isacharcolon}{\kern0pt}{\isacharprime}{\kern0pt}a\ set{\isacharparenright}{\kern0pt}{\isachardot}{\kern0pt}\ {\isasymforall}x\ {\isasymin}\ A{\isachardot}{\kern0pt}\ win{\isacharunderscore}{\kern0pt}count\ p\ x\ {\isasymle}\ win{\isacharunderscore}{\kern0pt}count\ p\ a{\isacharbraceright}{\kern0pt}{\isacharsemicolon}{\kern0pt}\isanewline
\ \ \ \ \ \ reject\ {\isacharequal}{\kern0pt}\ {\isacharbraceleft}{\kern0pt}a\ {\isasymin}\ A{\isachardot}{\kern0pt}\ {\isasymexists}x\ {\isasymin}\ A{\isachardot}{\kern0pt}\ win{\isacharunderscore}{\kern0pt}count\ p\ x\ {\isachargreater}{\kern0pt}\ win{\isacharunderscore}{\kern0pt}count\ p\ a{\isacharbraceright}{\kern0pt}\ in\isanewline
\ \ \ \ set{\isacharunderscore}{\kern0pt}equals{\isacharunderscore}{\kern0pt}partition\ A\ {\isacharparenleft}{\kern0pt}elect{\isacharcomma}{\kern0pt}\ reject{\isacharcomma}{\kern0pt}\ {\isacharbraceleft}{\kern0pt}{\isacharbraceright}{\kern0pt}{\isacharparenright}{\kern0pt}{\isachardoublequoteclose}\isanewline
\ \ \ \ \isacommand{by}\isamarkupfalse%
\ simp\isanewline
\ \ \isacommand{from}\isamarkupfalse%
\ disjoint\ unity\ \isacommand{show}\isamarkupfalse%
\ {\isacharquery}{\kern0pt}thesis\isanewline
\ \ \ \ \isacommand{by}\isamarkupfalse%
\ {\isacharparenleft}{\kern0pt}simp\ add{\isacharcolon}{\kern0pt}\ electoral{\isacharunderscore}{\kern0pt}modI{\isacharparenright}{\kern0pt}\isanewline
\isacommand{qed}\isamarkupfalse%
%
\endisatagproof
{\isafoldproof}%
%
\isadelimproof
\isanewline
%
\endisadelimproof
%
\isadelimML
\isanewline
%
\endisadelimML
%
\isatagML
\isacommand{ML}\isamarkupfalse%
\ {\isacartoucheopen}\isanewline
\ \ \ \ {\isacharparenleft}{\kern0pt}{\isacharasterisk}{\kern0pt}\ val\ w\ {\isacharequal}{\kern0pt}\ {\isacharparenleft}{\kern0pt}Defs{\isachardot}{\kern0pt}all{\isacharunderscore}{\kern0pt}specifications{\isacharunderscore}{\kern0pt}of\ {\isacharparenleft}{\kern0pt}Theory{\isachardot}{\kern0pt}defs{\isacharunderscore}{\kern0pt}of\ {\isacharat}{\kern0pt}{\isacharbraceleft}{\kern0pt}theory{\isacharbraceright}{\kern0pt}{\isacharparenright}{\kern0pt}{\isacharparenright}{\kern0pt}\isanewline
\ \ \ \ val\ x\ {\isacharequal}{\kern0pt}\ filter\ {\isacharparenleft}{\kern0pt}fn\ x\ {\isacharequal}{\kern0pt}{\isachargreater}{\kern0pt}\ String{\isachardot}{\kern0pt}isPrefix\ {\isachardoublequote}{\kern0pt}Seq{\isachardoublequote}{\kern0pt}\ x{\isacharparenright}{\kern0pt}\ {\isacharparenleft}{\kern0pt}map\ {\isacharparenleft}{\kern0pt}fn\ x\ {\isacharequal}{\kern0pt}{\isachargreater}{\kern0pt}\ snd\ {\isacharparenleft}{\kern0pt}fst\ x{\isacharparenright}{\kern0pt}{\isacharparenright}{\kern0pt}\isanewline
\ \ \ \ \ \ \ \ {\isacharparenleft}{\kern0pt}Defs{\isachardot}{\kern0pt}all{\isacharunderscore}{\kern0pt}specifications{\isacharunderscore}{\kern0pt}of\ {\isacharparenleft}{\kern0pt}Theory{\isachardot}{\kern0pt}defs{\isacharunderscore}{\kern0pt}of\ {\isacharat}{\kern0pt}{\isacharbraceleft}{\kern0pt}theory{\isacharbraceright}{\kern0pt}{\isacharparenright}{\kern0pt}{\isacharparenright}{\kern0pt}{\isacharparenright}{\kern0pt}\ {\isacharasterisk}{\kern0pt}{\isacharparenright}{\kern0pt}\isanewline
\ \ \ \ val\ y\ {\isacharequal}{\kern0pt}\ filter\ {\isacharparenleft}{\kern0pt}fn\ x\ {\isacharequal}{\kern0pt}{\isachargreater}{\kern0pt}\ {\isacharparenleft}{\kern0pt}String{\isachardot}{\kern0pt}isPrefix\ {\isacharparenleft}{\kern0pt}{\isachardoublequote}{\kern0pt}Plurality{\isacharunderscore}{\kern0pt}Module{\isachardot}{\kern0pt}plurality{\isachardoublequote}{\kern0pt}{\isacharparenright}{\kern0pt}\ {\isacharparenleft}{\kern0pt}snd\ {\isacharparenleft}{\kern0pt}fst\ {\isacharparenleft}{\kern0pt}x{\isacharparenright}{\kern0pt}{\isacharparenright}{\kern0pt}{\isacharparenright}{\kern0pt}{\isacharparenright}{\kern0pt}{\isacharparenright}{\kern0pt}\ \isanewline
\ \ \ \ \ \ \ \ {\isacharparenleft}{\kern0pt}Defs{\isachardot}{\kern0pt}all{\isacharunderscore}{\kern0pt}specifications{\isacharunderscore}{\kern0pt}of\ {\isacharparenleft}{\kern0pt}Theory{\isachardot}{\kern0pt}defs{\isacharunderscore}{\kern0pt}of\ {\isacharat}{\kern0pt}{\isacharbraceleft}{\kern0pt}theory{\isacharbraceright}{\kern0pt}{\isacharparenright}{\kern0pt}{\isacharparenright}{\kern0pt}\isanewline
\ \ \ \ val\ z\ {\isacharequal}{\kern0pt}\ filter\ {\isacharparenleft}{\kern0pt}fn\ x\ {\isacharequal}{\kern0pt}{\isachargreater}{\kern0pt}\ String{\isachardot}{\kern0pt}isPrefix\ {\isacharparenleft}{\kern0pt}{\isachardoublequote}{\kern0pt}Plurality{\isacharunderscore}{\kern0pt}Module{\isachardot}{\kern0pt}plurality{\isacharunderscore}{\kern0pt}def{\isachardoublequote}{\kern0pt}{\isacharparenright}{\kern0pt}\ {\isacharparenleft}{\kern0pt}{\isacharhash}{\kern0pt}description\ x{\isacharparenright}{\kern0pt}{\isacharparenright}{\kern0pt}\ {\isacharparenleft}{\kern0pt}map\ {\isacharparenleft}{\kern0pt}fn\ x\ {\isacharequal}{\kern0pt}{\isachargreater}{\kern0pt}\ {\isacharparenleft}{\kern0pt}hd\ {\isacharparenleft}{\kern0pt}snd\ x{\isacharparenright}{\kern0pt}{\isacharparenright}{\kern0pt}{\isacharparenright}{\kern0pt}\ y{\isacharparenright}{\kern0pt}\isanewline
\isanewline
\ \ \ \ val\ def\ {\isacharequal}{\kern0pt}\ hd\ z\isanewline
\ \ \ \ val\ def{\isacharunderscore}{\kern0pt}rhs\ {\isacharequal}{\kern0pt}\ {\isacharhash}{\kern0pt}rhs\ def\isanewline
\isanewline
\ \ \ \ val\ def{\isacharunderscore}{\kern0pt}funs\ {\isacharequal}{\kern0pt}\ filter\ {\isacharparenleft}{\kern0pt}fn\ x\ {\isacharequal}{\kern0pt}{\isachargreater}{\kern0pt}\ String{\isachardot}{\kern0pt}isPrefix\ {\isacharparenleft}{\kern0pt}{\isachardoublequote}{\kern0pt}fun{\isachardoublequote}{\kern0pt}{\isacharparenright}{\kern0pt}\ {\isacharparenleft}{\kern0pt}snd\ {\isacharparenleft}{\kern0pt}fst\ x{\isacharparenright}{\kern0pt}{\isacharparenright}{\kern0pt}{\isacharparenright}{\kern0pt}\ {\isacharparenleft}{\kern0pt}def{\isacharunderscore}{\kern0pt}rhs{\isacharparenright}{\kern0pt}\isanewline
\ \ \ \ val\ def{\isacharunderscore}{\kern0pt}consts\ {\isacharequal}{\kern0pt}\ filter\ {\isacharparenleft}{\kern0pt}fn\ x\ {\isacharequal}{\kern0pt}{\isachargreater}{\kern0pt}\ not\ {\isacharparenleft}{\kern0pt}String{\isachardot}{\kern0pt}isPrefix\ {\isacharparenleft}{\kern0pt}{\isachardoublequote}{\kern0pt}fun{\isachardoublequote}{\kern0pt}{\isacharparenright}{\kern0pt}\ {\isacharparenleft}{\kern0pt}snd\ {\isacharparenleft}{\kern0pt}fst\ x{\isacharparenright}{\kern0pt}{\isacharparenright}{\kern0pt}{\isacharparenright}{\kern0pt}{\isacharparenright}{\kern0pt}\ {\isacharparenleft}{\kern0pt}def{\isacharunderscore}{\kern0pt}rhs{\isacharparenright}{\kern0pt}\isanewline
\isanewline
\ \ \ \ val\ fun{\isacharunderscore}{\kern0pt}type\ {\isacharequal}{\kern0pt}\ Defs{\isachardot}{\kern0pt}pretty{\isacharunderscore}{\kern0pt}entry\ {\isacharparenleft}{\kern0pt}Defs{\isachardot}{\kern0pt}global{\isacharunderscore}{\kern0pt}context\ {\isacharat}{\kern0pt}{\isacharbraceleft}{\kern0pt}theory{\isacharbraceright}{\kern0pt}{\isacharparenright}{\kern0pt}\ {\isacharparenleft}{\kern0pt}List{\isachardot}{\kern0pt}last\ def{\isacharunderscore}{\kern0pt}funs{\isacharparenright}{\kern0pt}\isanewline
\isanewline
\ \ \ {\isacharparenleft}{\kern0pt}{\isacharasterisk}{\kern0pt}\ val\ a\ {\isacharequal}{\kern0pt}\ Defs{\isachardot}{\kern0pt}pretty{\isacharunderscore}{\kern0pt}entry\ {\isacharparenleft}{\kern0pt}Defs{\isachardot}{\kern0pt}global{\isacharunderscore}{\kern0pt}context\ {\isacharat}{\kern0pt}{\isacharbraceleft}{\kern0pt}theory{\isacharbraceright}{\kern0pt}{\isacharparenright}{\kern0pt}\ {\isacharparenleft}{\kern0pt}nth\ {\isacharparenleft}{\kern0pt}snd\ {\isacharparenleft}{\kern0pt}{\isacharhash}{\kern0pt}lhs\ {\isacharparenleft}{\kern0pt}hd\ z{\isacharparenright}{\kern0pt}{\isacharcomma}{\kern0pt}\ {\isacharhash}{\kern0pt}rhs\ {\isacharparenleft}{\kern0pt}hd\ z{\isacharparenright}{\kern0pt}{\isacharparenright}{\kern0pt}\ {\isacharparenright}{\kern0pt}{\isadigit{2}}{\isacharparenright}{\kern0pt}\ {\isacharasterisk}{\kern0pt}{\isacharparenright}{\kern0pt}\isanewline
\ {\isacartoucheclose}%
\endisatagML
{\isafoldML}%
%
\isadelimML
\isanewline
%
\endisadelimML
%
\isadelimtheory
\isanewline
%
\endisadelimtheory
%
\isatagtheory
\isacommand{end}\isamarkupfalse%
%
\endisatagtheory
{\isafoldtheory}%
%
\isadelimtheory
%
\endisadelimtheory
%
\end{isabellebody}%
\endinput
%:%file=~/Documents/Studies/VotingRuleGenerator/virage/src/test/resources/verifiedVotingRuleConstruction/theories/Compositional_Framework/Components/Basic_Modules/Plurality_Module.thy%:%
%:%6=3%:%
%:%11=4%:%
%:%12=5%:%
%:%14=8%:%
%:%30=10%:%
%:%31=10%:%
%:%32=11%:%
%:%33=12%:%
%:%42=15%:%
%:%43=16%:%
%:%44=17%:%
%:%45=18%:%
%:%46=19%:%
%:%55=21%:%
%:%65=23%:%
%:%66=23%:%
%:%67=24%:%
%:%77=29%:%
%:%87=31%:%
%:%88=31%:%
%:%95=32%:%
%:%96=32%:%
%:%97=33%:%
%:%98=33%:%
%:%99=34%:%
%:%102=37%:%
%:%103=38%:%
%:%104=38%:%
%:%105=39%:%
%:%106=39%:%
%:%107=40%:%
%:%110=43%:%
%:%111=44%:%
%:%112=44%:%
%:%113=45%:%
%:%114=45%:%
%:%115=46%:%
%:%118=49%:%
%:%119=50%:%
%:%120=50%:%
%:%121=51%:%
%:%122=51%:%
%:%123=52%:%
%:%124=52%:%
%:%125=53%:%
%:%128=56%:%
%:%129=57%:%
%:%130=57%:%
%:%131=58%:%
%:%132=58%:%
%:%133=58%:%
%:%134=59%:%
%:%135=59%:%
%:%136=60%:%
%:%142=60%:%
%:%147=61%:%
%:%152=62%:%
%:%153=62%:%
%:%175=79%:%
%:%180=80%:%
%:%185=81%:%
%
\begin{isabellebody}%
\setisabellecontext{Composite{\isacharunderscore}{\kern0pt}Elimination{\isacharunderscore}{\kern0pt}Modules}%
%
\isadelimtheory
%
\endisadelimtheory
%
\isatagtheory
\isacommand{theory}\isamarkupfalse%
\ Composite{\isacharunderscore}{\kern0pt}Elimination{\isacharunderscore}{\kern0pt}Modules\isanewline
\ \ \isakeyword{imports}\ {\isachardoublequoteopen}{\isachardot}{\kern0pt}{\isachardot}{\kern0pt}{\isacharslash}{\kern0pt}Electoral{\isacharunderscore}{\kern0pt}Module{\isachardoublequoteclose}\isanewline
\ \ \ \ \ \ \ \ \ \ {\isachardoublequoteopen}{\isachardot}{\kern0pt}{\isachardot}{\kern0pt}{\isacharslash}{\kern0pt}Evaluation{\isacharunderscore}{\kern0pt}Function{\isachardoublequoteclose}\isanewline
\ \ \ \ \ \ \ \ \ \ {\isachardoublequoteopen}{\isachardot}{\kern0pt}{\isachardot}{\kern0pt}{\isacharslash}{\kern0pt}Basic{\isacharunderscore}{\kern0pt}Modules{\isacharslash}{\kern0pt}Elimination{\isacharunderscore}{\kern0pt}Module{\isachardoublequoteclose}\isanewline
\isanewline
\isakeyword{begin}%
\endisatagtheory
{\isafoldtheory}%
%
\isadelimtheory
%
\endisadelimtheory
%
\isadelimdocument
%
\endisadelimdocument
%
\isatagdocument
%
\isamarkupsection{Borda Module%
}
\isamarkuptrue%
%
\endisatagdocument
{\isafolddocument}%
%
\isadelimdocument
%
\endisadelimdocument
%
\begin{isamarkuptext}%
This is the Borda module used by the Borda rule. The Borda rule is a voting
rule, where on each ballot, each alternative is assigned a score that depends
on how many alternatives are ranked below. The sum of all such scores for an
alternative is hence called their Borda score. The alternative with the highest
Borda score is elected. The module implemented herein only rejects the
alternatives not elected by the voting rule, and defers the alternatives that
would be elected by the full voting rule.%
\end{isamarkuptext}\isamarkuptrue%
%
\isadelimdocument
%
\endisadelimdocument
%
\isatagdocument
%
\isamarkupsubsection{Definition%
}
\isamarkuptrue%
%
\endisatagdocument
{\isafolddocument}%
%
\isadelimdocument
%
\endisadelimdocument
\isacommand{fun}\isamarkupfalse%
\ borda{\isacharunderscore}{\kern0pt}score\ {\isacharcolon}{\kern0pt}{\isacharcolon}{\kern0pt}\ {\isachardoublequoteopen}{\isacharprime}{\kern0pt}a\ Evaluation{\isacharunderscore}{\kern0pt}Function{\isachardoublequoteclose}\ \isakeyword{where}\isanewline
\ \ {\isachardoublequoteopen}borda{\isacharunderscore}{\kern0pt}score\ x\ A\ p\ {\isacharequal}{\kern0pt}\ {\isacharparenleft}{\kern0pt}{\isasymSum}y\ {\isasymin}\ A{\isachardot}{\kern0pt}\ {\isacharparenleft}{\kern0pt}prefer{\isacharunderscore}{\kern0pt}count\ p\ x\ y{\isacharparenright}{\kern0pt}{\isacharparenright}{\kern0pt}{\isachardoublequoteclose}\isanewline
\isanewline
\isacommand{fun}\isamarkupfalse%
\ borda\ {\isacharcolon}{\kern0pt}{\isacharcolon}{\kern0pt}\ {\isachardoublequoteopen}{\isacharprime}{\kern0pt}a\ Electoral{\isacharunderscore}{\kern0pt}Module{\isachardoublequoteclose}\ \isakeyword{where}\isanewline
\ \ {\isachardoublequoteopen}borda\ A\ p\ {\isacharequal}{\kern0pt}\ max{\isacharunderscore}{\kern0pt}eliminator\ borda{\isacharunderscore}{\kern0pt}score\ A\ p{\isachardoublequoteclose}%
\isadelimdocument
%
\endisadelimdocument
%
\isatagdocument
%
\isamarkupsection{Condorcet Module%
}
\isamarkuptrue%
%
\endisatagdocument
{\isafolddocument}%
%
\isadelimdocument
%
\endisadelimdocument
%
\begin{isamarkuptext}%
This is the Condorcet module used by the Condorcet (voting) rule. The Condorcet
rule is a voting rule that implements the Condorcet criterion, i.e., it elects
the Condorcet winner if it exists, otherwise a tie remains between all
alternatives. The module implemented herein only rejects the alternatives not
elected by the voting rule, and defers the alternatives that would be elected
by the full voting rule.%
\end{isamarkuptext}\isamarkuptrue%
%
\isadelimdocument
%
\endisadelimdocument
%
\isatagdocument
%
\isamarkupsubsection{Definition%
}
\isamarkuptrue%
%
\endisatagdocument
{\isafolddocument}%
%
\isadelimdocument
%
\endisadelimdocument
\isacommand{fun}\isamarkupfalse%
\ condorcet{\isacharunderscore}{\kern0pt}score\ {\isacharcolon}{\kern0pt}{\isacharcolon}{\kern0pt}\ {\isachardoublequoteopen}{\isacharprime}{\kern0pt}a\ Evaluation{\isacharunderscore}{\kern0pt}Function{\isachardoublequoteclose}\ \isakeyword{where}\isanewline
\ \ {\isachardoublequoteopen}condorcet{\isacharunderscore}{\kern0pt}score\ x\ A\ p\ {\isacharequal}{\kern0pt}\isanewline
\ \ \ \ {\isacharparenleft}{\kern0pt}if\ {\isacharparenleft}{\kern0pt}condorcet{\isacharunderscore}{\kern0pt}winner\ A\ p\ x{\isacharparenright}{\kern0pt}\ then\ {\isadigit{1}}\ else\ {\isadigit{0}}{\isacharparenright}{\kern0pt}{\isachardoublequoteclose}\isanewline
\isanewline
\isacommand{fun}\isamarkupfalse%
\ condorcet\ {\isacharcolon}{\kern0pt}{\isacharcolon}{\kern0pt}\ {\isachardoublequoteopen}{\isacharprime}{\kern0pt}a\ Electoral{\isacharunderscore}{\kern0pt}Module{\isachardoublequoteclose}\ \isakeyword{where}\isanewline
\ \ {\isachardoublequoteopen}condorcet\ A\ p\ {\isacharequal}{\kern0pt}\ {\isacharparenleft}{\kern0pt}max{\isacharunderscore}{\kern0pt}eliminator\ condorcet{\isacharunderscore}{\kern0pt}score{\isacharparenright}{\kern0pt}\ A\ p{\isachardoublequoteclose}%
\isadelimdocument
%
\endisadelimdocument
%
\isatagdocument
%
\isamarkupsection{Copeland Module%
}
\isamarkuptrue%
%
\endisatagdocument
{\isafolddocument}%
%
\isadelimdocument
%
\endisadelimdocument
%
\begin{isamarkuptext}%
This is the Copeland module used by the Copeland voting rule. The Copeland
rule elects the alternatives with the highest difference between the amount
of simple-majority wins and the amount of simple-majority losses. The module
implemented herein only rejects the alternatives not elected by the voting
rule, and defers the alternatives that would be elected by the full voting
rule.%
\end{isamarkuptext}\isamarkuptrue%
%
\isadelimdocument
%
\endisadelimdocument
%
\isatagdocument
%
\isamarkupsubsection{Definition%
}
\isamarkuptrue%
%
\endisatagdocument
{\isafolddocument}%
%
\isadelimdocument
%
\endisadelimdocument
\isacommand{fun}\isamarkupfalse%
\ copeland{\isacharunderscore}{\kern0pt}score\ {\isacharcolon}{\kern0pt}{\isacharcolon}{\kern0pt}\ {\isachardoublequoteopen}{\isacharprime}{\kern0pt}a\ Evaluation{\isacharunderscore}{\kern0pt}Function{\isachardoublequoteclose}\ \isakeyword{where}\isanewline
\ \ {\isachardoublequoteopen}copeland{\isacharunderscore}{\kern0pt}score\ x\ A\ p\ {\isacharequal}{\kern0pt}\isanewline
\ \ \ \ card{\isacharbraceleft}{\kern0pt}y\ {\isasymin}\ A\ {\isachardot}{\kern0pt}\ wins\ x\ p\ y{\isacharbraceright}{\kern0pt}\ {\isacharminus}{\kern0pt}\ card{\isacharbraceleft}{\kern0pt}y\ {\isasymin}\ A\ {\isachardot}{\kern0pt}\ wins\ y\ p\ x{\isacharbraceright}{\kern0pt}{\isachardoublequoteclose}\isanewline
\isanewline
\isacommand{fun}\isamarkupfalse%
\ copeland\ {\isacharcolon}{\kern0pt}{\isacharcolon}{\kern0pt}\ {\isachardoublequoteopen}{\isacharprime}{\kern0pt}a\ Electoral{\isacharunderscore}{\kern0pt}Module{\isachardoublequoteclose}\ \isakeyword{where}\isanewline
\ \ {\isachardoublequoteopen}copeland\ A\ p\ {\isacharequal}{\kern0pt}\ max{\isacharunderscore}{\kern0pt}eliminator\ copeland{\isacharunderscore}{\kern0pt}score\ A\ p{\isachardoublequoteclose}%
\isadelimdocument
%
\endisadelimdocument
%
\isatagdocument
%
\isamarkupsubsection{Lemmata%
}
\isamarkuptrue%
%
\endisatagdocument
{\isafolddocument}%
%
\isadelimdocument
%
\endisadelimdocument
\isacommand{lemma}\isamarkupfalse%
\ cond{\isacharunderscore}{\kern0pt}winner{\isacharunderscore}{\kern0pt}imp{\isacharunderscore}{\kern0pt}win{\isacharunderscore}{\kern0pt}count{\isacharcolon}{\kern0pt}\isanewline
\ \ \isakeyword{assumes}\ winner{\isacharcolon}{\kern0pt}\ {\isachardoublequoteopen}condorcet{\isacharunderscore}{\kern0pt}winner\ A\ p\ w{\isachardoublequoteclose}\isanewline
\ \ \isakeyword{shows}\ {\isachardoublequoteopen}card\ {\isacharbraceleft}{\kern0pt}y\ {\isasymin}\ A\ {\isachardot}{\kern0pt}\ wins\ w\ p\ y{\isacharbraceright}{\kern0pt}\ {\isacharequal}{\kern0pt}\ card\ A\ {\isacharminus}{\kern0pt}{\isadigit{1}}{\isachardoublequoteclose}\isanewline
%
\isadelimproof
%
\endisadelimproof
%
\isatagproof
\isacommand{proof}\isamarkupfalse%
\ {\isacharminus}{\kern0pt}\isanewline
\ \ \isacommand{from}\isamarkupfalse%
\ winner\isanewline
\ \ \isacommand{have}\isamarkupfalse%
\ {\isadigit{0}}{\isacharcolon}{\kern0pt}\ {\isachardoublequoteopen}{\isasymforall}x\ {\isasymin}\ A\ {\isacharminus}{\kern0pt}\ {\isacharbraceleft}{\kern0pt}w{\isacharbraceright}{\kern0pt}\ {\isachardot}{\kern0pt}\ wins\ w\ p\ x{\isachardoublequoteclose}\isanewline
\ \ \ \ \isacommand{by}\isamarkupfalse%
\ simp\isanewline
\ \ \isacommand{have}\isamarkupfalse%
\ {\isadigit{1}}{\isacharcolon}{\kern0pt}\ {\isachardoublequoteopen}{\isasymforall}M\ {\isachardot}{\kern0pt}\ {\isacharbraceleft}{\kern0pt}x\ {\isasymin}\ M\ {\isachardot}{\kern0pt}\ True{\isacharbraceright}{\kern0pt}\ {\isacharequal}{\kern0pt}\ M{\isachardoublequoteclose}\isanewline
\ \ \ \ \isacommand{by}\isamarkupfalse%
\ blast\isanewline
\ \ \isacommand{from}\isamarkupfalse%
\ {\isadigit{0}}\ {\isadigit{1}}\isanewline
\ \ \isacommand{have}\isamarkupfalse%
\ {\isachardoublequoteopen}{\isacharbraceleft}{\kern0pt}x\ {\isasymin}\ A\ {\isacharminus}{\kern0pt}\ {\isacharbraceleft}{\kern0pt}w{\isacharbraceright}{\kern0pt}\ {\isachardot}{\kern0pt}\ wins\ w\ p\ x{\isacharbraceright}{\kern0pt}\ {\isacharequal}{\kern0pt}\ A\ {\isacharminus}{\kern0pt}\ {\isacharbraceleft}{\kern0pt}w{\isacharbraceright}{\kern0pt}{\isachardoublequoteclose}\isanewline
\ \ \ \ \isacommand{by}\isamarkupfalse%
\ blast\isanewline
\ \ \isacommand{hence}\isamarkupfalse%
\ {\isadigit{1}}{\isadigit{0}}{\isacharcolon}{\kern0pt}\ {\isachardoublequoteopen}card\ {\isacharbraceleft}{\kern0pt}x\ {\isasymin}\ A\ {\isacharminus}{\kern0pt}\ {\isacharbraceleft}{\kern0pt}w{\isacharbraceright}{\kern0pt}\ {\isachardot}{\kern0pt}\ wins\ w\ p\ x{\isacharbraceright}{\kern0pt}\ {\isacharequal}{\kern0pt}\ card\ {\isacharparenleft}{\kern0pt}A\ {\isacharminus}{\kern0pt}\ {\isacharbraceleft}{\kern0pt}w{\isacharbraceright}{\kern0pt}{\isacharparenright}{\kern0pt}{\isachardoublequoteclose}\isanewline
\ \ \ \ \isacommand{by}\isamarkupfalse%
\ simp\isanewline
\ \ \isacommand{from}\isamarkupfalse%
\ winner\isanewline
\ \ \isacommand{have}\isamarkupfalse%
\ {\isadigit{1}}{\isadigit{1}}{\isacharcolon}{\kern0pt}\ {\isachardoublequoteopen}w\ {\isasymin}\ A{\isachardoublequoteclose}\isanewline
\ \ \ \ \isacommand{by}\isamarkupfalse%
\ simp\isanewline
\ \ \isacommand{hence}\isamarkupfalse%
\ {\isachardoublequoteopen}card\ {\isacharparenleft}{\kern0pt}A\ {\isacharminus}{\kern0pt}\ {\isacharbraceleft}{\kern0pt}w{\isacharbraceright}{\kern0pt}{\isacharparenright}{\kern0pt}\ {\isacharequal}{\kern0pt}\ card\ A\ {\isacharminus}{\kern0pt}\ {\isadigit{1}}{\isachardoublequoteclose}\isanewline
\ \ \ \ \isacommand{using}\isamarkupfalse%
\ card{\isacharunderscore}{\kern0pt}Diff{\isacharunderscore}{\kern0pt}singleton\ condorcet{\isacharunderscore}{\kern0pt}winner{\isachardot}{\kern0pt}simps\ winner\isanewline
\ \ \ \ \isacommand{by}\isamarkupfalse%
\ metis\isanewline
\ \ \isacommand{hence}\isamarkupfalse%
\ amount{\isadigit{1}}{\isacharcolon}{\kern0pt}\isanewline
\ \ \ \ {\isachardoublequoteopen}card\ {\isacharbraceleft}{\kern0pt}x\ {\isasymin}\ A\ {\isacharminus}{\kern0pt}\ {\isacharbraceleft}{\kern0pt}w{\isacharbraceright}{\kern0pt}\ {\isachardot}{\kern0pt}\ wins\ w\ p\ x{\isacharbraceright}{\kern0pt}\ {\isacharequal}{\kern0pt}\ card\ {\isacharparenleft}{\kern0pt}A{\isacharparenright}{\kern0pt}\ {\isacharminus}{\kern0pt}\ {\isadigit{1}}{\isachardoublequoteclose}\isanewline
\ \ \ \ \isacommand{using}\isamarkupfalse%
\ {\isachardoublequoteopen}{\isadigit{1}}{\isadigit{0}}{\isachardoublequoteclose}\isanewline
\ \ \ \ \isacommand{by}\isamarkupfalse%
\ linarith\isanewline
\ \ \isacommand{have}\isamarkupfalse%
\ {\isadigit{2}}{\isacharcolon}{\kern0pt}\ {\isachardoublequoteopen}{\isasymforall}x\ {\isasymin}\ {\isacharbraceleft}{\kern0pt}w{\isacharbraceright}{\kern0pt}\ {\isachardot}{\kern0pt}\ {\isasymnot}\ wins\ x\ p\ x{\isachardoublequoteclose}\isanewline
\ \ \ \ \isacommand{by}\isamarkupfalse%
\ {\isacharparenleft}{\kern0pt}simp\ add{\isacharcolon}{\kern0pt}\ wins{\isacharunderscore}{\kern0pt}irreflex{\isacharparenright}{\kern0pt}\isanewline
\ \ \isacommand{have}\isamarkupfalse%
\ {\isadigit{3}}{\isacharcolon}{\kern0pt}\ {\isachardoublequoteopen}{\isasymforall}M\ {\isachardot}{\kern0pt}\ {\isacharbraceleft}{\kern0pt}x\ {\isasymin}\ M\ {\isachardot}{\kern0pt}\ False{\isacharbraceright}{\kern0pt}\ {\isacharequal}{\kern0pt}\ {\isacharbraceleft}{\kern0pt}{\isacharbraceright}{\kern0pt}{\isachardoublequoteclose}\isanewline
\ \ \ \ \isacommand{by}\isamarkupfalse%
\ blast\isanewline
\ \ \isacommand{from}\isamarkupfalse%
\ {\isadigit{2}}\ {\isadigit{3}}\isanewline
\ \ \isacommand{have}\isamarkupfalse%
\ {\isachardoublequoteopen}{\isacharbraceleft}{\kern0pt}x\ {\isasymin}\ {\isacharbraceleft}{\kern0pt}w{\isacharbraceright}{\kern0pt}\ {\isachardot}{\kern0pt}\ wins\ w\ p\ x{\isacharbraceright}{\kern0pt}\ {\isacharequal}{\kern0pt}\ {\isacharbraceleft}{\kern0pt}{\isacharbraceright}{\kern0pt}{\isachardoublequoteclose}\isanewline
\ \ \ \ \isacommand{by}\isamarkupfalse%
\ blast\isanewline
\ \ \isacommand{hence}\isamarkupfalse%
\ amount{\isadigit{2}}{\isacharcolon}{\kern0pt}\ {\isachardoublequoteopen}card\ {\isacharbraceleft}{\kern0pt}x\ {\isasymin}\ {\isacharbraceleft}{\kern0pt}w{\isacharbraceright}{\kern0pt}\ {\isachardot}{\kern0pt}\ wins\ w\ p\ x{\isacharbraceright}{\kern0pt}\ {\isacharequal}{\kern0pt}\ {\isadigit{0}}{\isachardoublequoteclose}\isanewline
\ \ \ \ \isacommand{by}\isamarkupfalse%
\ simp\isanewline
\ \ \isacommand{have}\isamarkupfalse%
\ disjunct{\isacharcolon}{\kern0pt}\isanewline
\ \ \ \ {\isachardoublequoteopen}{\isacharbraceleft}{\kern0pt}x\ {\isasymin}\ A\ {\isacharminus}{\kern0pt}\ {\isacharbraceleft}{\kern0pt}w{\isacharbraceright}{\kern0pt}\ {\isachardot}{\kern0pt}\ wins\ w\ p\ x{\isacharbraceright}{\kern0pt}\ {\isasyminter}\ {\isacharbraceleft}{\kern0pt}x\ {\isasymin}\ {\isacharbraceleft}{\kern0pt}w{\isacharbraceright}{\kern0pt}\ {\isachardot}{\kern0pt}\ wins\ w\ p\ x{\isacharbraceright}{\kern0pt}\ {\isacharequal}{\kern0pt}\ {\isacharbraceleft}{\kern0pt}{\isacharbraceright}{\kern0pt}{\isachardoublequoteclose}\isanewline
\ \ \ \ \isacommand{by}\isamarkupfalse%
\ blast\isanewline
\ \ \isacommand{have}\isamarkupfalse%
\ union{\isacharcolon}{\kern0pt}\isanewline
\ \ \ \ {\isachardoublequoteopen}{\isacharbraceleft}{\kern0pt}x\ {\isasymin}\ A\ {\isacharminus}{\kern0pt}\ {\isacharbraceleft}{\kern0pt}w{\isacharbraceright}{\kern0pt}\ {\isachardot}{\kern0pt}\ wins\ w\ p\ x{\isacharbraceright}{\kern0pt}\ {\isasymunion}\ {\isacharbraceleft}{\kern0pt}x\ {\isasymin}\ {\isacharbraceleft}{\kern0pt}w{\isacharbraceright}{\kern0pt}\ {\isachardot}{\kern0pt}\ wins\ w\ p\ x{\isacharbraceright}{\kern0pt}\ {\isacharequal}{\kern0pt}\isanewline
\ \ \ \ \ \ \ \ {\isacharbraceleft}{\kern0pt}x\ {\isasymin}\ A\ {\isachardot}{\kern0pt}\ wins\ w\ p\ x{\isacharbraceright}{\kern0pt}{\isachardoublequoteclose}\isanewline
\ \ \ \ \isacommand{using}\isamarkupfalse%
\ {\isachardoublequoteopen}{\isadigit{2}}{\isachardoublequoteclose}\isanewline
\ \ \ \ \isacommand{by}\isamarkupfalse%
\ blast\isanewline
\ \ \isacommand{have}\isamarkupfalse%
\ finiteness{\isadigit{1}}{\isacharcolon}{\kern0pt}\ {\isachardoublequoteopen}finite\ {\isacharbraceleft}{\kern0pt}x\ {\isasymin}\ A\ {\isacharminus}{\kern0pt}\ {\isacharbraceleft}{\kern0pt}w{\isacharbraceright}{\kern0pt}\ {\isachardot}{\kern0pt}\ wins\ w\ p\ x{\isacharbraceright}{\kern0pt}{\isachardoublequoteclose}\isanewline
\ \ \ \ \isacommand{using}\isamarkupfalse%
\ condorcet{\isacharunderscore}{\kern0pt}winner{\isachardot}{\kern0pt}simps\ winner\isanewline
\ \ \ \ \isacommand{by}\isamarkupfalse%
\ fastforce\isanewline
\ \ \isacommand{have}\isamarkupfalse%
\ finiteness{\isadigit{2}}{\isacharcolon}{\kern0pt}\ {\isachardoublequoteopen}finite\ {\isacharbraceleft}{\kern0pt}x\ {\isasymin}\ {\isacharbraceleft}{\kern0pt}w{\isacharbraceright}{\kern0pt}\ {\isachardot}{\kern0pt}\ wins\ w\ p\ x{\isacharbraceright}{\kern0pt}{\isachardoublequoteclose}\isanewline
\ \ \ \ \isacommand{by}\isamarkupfalse%
\ simp\isanewline
\ \ \isacommand{from}\isamarkupfalse%
\ finiteness{\isadigit{1}}\ finiteness{\isadigit{2}}\ disjunct\ card{\isacharunderscore}{\kern0pt}Un{\isacharunderscore}{\kern0pt}disjoint\isanewline
\ \ \isacommand{have}\isamarkupfalse%
\isanewline
\ \ \ \ {\isachardoublequoteopen}card\ {\isacharparenleft}{\kern0pt}{\isacharbraceleft}{\kern0pt}x\ {\isasymin}\ A\ {\isacharminus}{\kern0pt}\ {\isacharbraceleft}{\kern0pt}w{\isacharbraceright}{\kern0pt}\ {\isachardot}{\kern0pt}\ wins\ w\ p\ x{\isacharbraceright}{\kern0pt}\ {\isasymunion}\ {\isacharbraceleft}{\kern0pt}x\ {\isasymin}\ {\isacharbraceleft}{\kern0pt}w{\isacharbraceright}{\kern0pt}\ {\isachardot}{\kern0pt}\ wins\ w\ p\ x{\isacharbraceright}{\kern0pt}{\isacharparenright}{\kern0pt}\ {\isacharequal}{\kern0pt}\isanewline
\ \ \ \ \ \ \ \ card\ {\isacharbraceleft}{\kern0pt}x\ {\isasymin}\ A\ {\isacharminus}{\kern0pt}\ {\isacharbraceleft}{\kern0pt}w{\isacharbraceright}{\kern0pt}\ {\isachardot}{\kern0pt}\ wins\ w\ p\ x{\isacharbraceright}{\kern0pt}\ {\isacharplus}{\kern0pt}\ card\ {\isacharbraceleft}{\kern0pt}x\ {\isasymin}\ {\isacharbraceleft}{\kern0pt}w{\isacharbraceright}{\kern0pt}\ {\isachardot}{\kern0pt}\ wins\ w\ p\ x{\isacharbraceright}{\kern0pt}{\isachardoublequoteclose}\isanewline
\ \ \ \ \isacommand{by}\isamarkupfalse%
\ blast\isanewline
\ \ \isacommand{with}\isamarkupfalse%
\ union\isanewline
\ \ \isacommand{have}\isamarkupfalse%
\ {\isachardoublequoteopen}card\ {\isacharbraceleft}{\kern0pt}x\ {\isasymin}\ A\ {\isachardot}{\kern0pt}\ wins\ w\ p\ x{\isacharbraceright}{\kern0pt}\ {\isacharequal}{\kern0pt}\isanewline
\ \ \ \ \ \ \ \ \ \ card\ {\isacharbraceleft}{\kern0pt}x\ {\isasymin}\ A\ {\isacharminus}{\kern0pt}\ {\isacharbraceleft}{\kern0pt}w{\isacharbraceright}{\kern0pt}\ {\isachardot}{\kern0pt}\ wins\ w\ p\ x{\isacharbraceright}{\kern0pt}\ {\isacharplus}{\kern0pt}\ card\ {\isacharbraceleft}{\kern0pt}x\ {\isasymin}\ {\isacharbraceleft}{\kern0pt}w{\isacharbraceright}{\kern0pt}\ {\isachardot}{\kern0pt}\ wins\ w\ p\ x{\isacharbraceright}{\kern0pt}{\isachardoublequoteclose}\isanewline
\ \ \ \ \isacommand{by}\isamarkupfalse%
\ simp\isanewline
\ \ \isacommand{with}\isamarkupfalse%
\ amount{\isadigit{1}}\ amount{\isadigit{2}}\isanewline
\ \ \isacommand{show}\isamarkupfalse%
\ {\isacharquery}{\kern0pt}thesis\isanewline
\ \ \ \ \isacommand{by}\isamarkupfalse%
\ linarith\isanewline
\isacommand{qed}\isamarkupfalse%
%
\endisatagproof
{\isafoldproof}%
%
\isadelimproof
\isanewline
%
\endisadelimproof
\isanewline
\isanewline
\isacommand{lemma}\isamarkupfalse%
\ cond{\isacharunderscore}{\kern0pt}winner{\isacharunderscore}{\kern0pt}imp{\isacharunderscore}{\kern0pt}loss{\isacharunderscore}{\kern0pt}count{\isacharcolon}{\kern0pt}\isanewline
\ \ \isakeyword{assumes}\ winner{\isacharcolon}{\kern0pt}\ {\isachardoublequoteopen}condorcet{\isacharunderscore}{\kern0pt}winner\ A\ p\ w{\isachardoublequoteclose}\isanewline
\ \ \isakeyword{shows}\ {\isachardoublequoteopen}card\ {\isacharbraceleft}{\kern0pt}y\ {\isasymin}\ A\ {\isachardot}{\kern0pt}\ wins\ y\ p\ w{\isacharbraceright}{\kern0pt}\ {\isacharequal}{\kern0pt}\ {\isadigit{0}}{\isachardoublequoteclose}\isanewline
%
\isadelimproof
\ \ %
\endisadelimproof
%
\isatagproof
\isacommand{using}\isamarkupfalse%
\ Collect{\isacharunderscore}{\kern0pt}empty{\isacharunderscore}{\kern0pt}eq\ card{\isacharunderscore}{\kern0pt}eq{\isacharunderscore}{\kern0pt}{\isadigit{0}}{\isacharunderscore}{\kern0pt}iff\ condorcet{\isacharunderscore}{\kern0pt}winner{\isachardot}{\kern0pt}simps\isanewline
\ \ \ \ \ \ \ \ insert{\isacharunderscore}{\kern0pt}Diff\ insert{\isacharunderscore}{\kern0pt}iff\ wins{\isacharunderscore}{\kern0pt}antisym\ winner\isanewline
\ \ \isacommand{by}\isamarkupfalse%
\ {\isacharparenleft}{\kern0pt}metis\ {\isacharparenleft}{\kern0pt}no{\isacharunderscore}{\kern0pt}types{\isacharcomma}{\kern0pt}\ lifting{\isacharparenright}{\kern0pt}{\isacharparenright}{\kern0pt}%
\endisatagproof
{\isafoldproof}%
%
\isadelimproof
\isanewline
%
\endisadelimproof
\isanewline
\isanewline
\isacommand{lemma}\isamarkupfalse%
\ cond{\isacharunderscore}{\kern0pt}winner{\isacharunderscore}{\kern0pt}imp{\isacharunderscore}{\kern0pt}copeland{\isacharunderscore}{\kern0pt}score{\isacharcolon}{\kern0pt}\isanewline
\ \ \isakeyword{assumes}\ winner{\isacharcolon}{\kern0pt}\ {\isachardoublequoteopen}condorcet{\isacharunderscore}{\kern0pt}winner\ A\ p\ w{\isachardoublequoteclose}\isanewline
\ \ \isakeyword{shows}\ {\isachardoublequoteopen}copeland{\isacharunderscore}{\kern0pt}score\ w\ A\ p\ {\isacharequal}{\kern0pt}\ card\ A\ {\isacharminus}{\kern0pt}{\isadigit{1}}{\isachardoublequoteclose}\isanewline
%
\isadelimproof
\ \ %
\endisadelimproof
%
\isatagproof
\isacommand{unfolding}\isamarkupfalse%
\ copeland{\isacharunderscore}{\kern0pt}score{\isachardot}{\kern0pt}simps\isanewline
\isacommand{proof}\isamarkupfalse%
\ {\isacharminus}{\kern0pt}\isanewline
\ \ \isacommand{show}\isamarkupfalse%
\isanewline
\ \ \ \ {\isachardoublequoteopen}card\ {\isacharbraceleft}{\kern0pt}y\ {\isasymin}\ A{\isachardot}{\kern0pt}\ wins\ w\ p\ y{\isacharbraceright}{\kern0pt}\ {\isacharminus}{\kern0pt}\ card\ {\isacharbraceleft}{\kern0pt}y\ {\isasymin}\ A{\isachardot}{\kern0pt}\ wins\ y\ p\ w{\isacharbraceright}{\kern0pt}\ {\isacharequal}{\kern0pt}\isanewline
\ \ \ \ \ \ card\ A\ {\isacharminus}{\kern0pt}\ {\isadigit{1}}{\isachardoublequoteclose}\isanewline
\ \ \ \ \isacommand{using}\isamarkupfalse%
\ cond{\isacharunderscore}{\kern0pt}winner{\isacharunderscore}{\kern0pt}imp{\isacharunderscore}{\kern0pt}loss{\isacharunderscore}{\kern0pt}count\isanewline
\ \ \ \ \ \ \ \ cond{\isacharunderscore}{\kern0pt}winner{\isacharunderscore}{\kern0pt}imp{\isacharunderscore}{\kern0pt}win{\isacharunderscore}{\kern0pt}count\ winner\isanewline
\ \ \isacommand{proof}\isamarkupfalse%
\ {\isacharminus}{\kern0pt}\isanewline
\ \ \ \ \isacommand{have}\isamarkupfalse%
\ f{\isadigit{1}}{\isacharcolon}{\kern0pt}\ {\isachardoublequoteopen}card\ {\isacharbraceleft}{\kern0pt}a\ {\isasymin}\ A{\isachardot}{\kern0pt}\ wins\ w\ p\ a{\isacharbraceright}{\kern0pt}\ {\isacharequal}{\kern0pt}\ card\ A\ {\isacharminus}{\kern0pt}\ {\isadigit{1}}{\isachardoublequoteclose}\isanewline
\ \ \ \ \ \ \isacommand{using}\isamarkupfalse%
\ cond{\isacharunderscore}{\kern0pt}winner{\isacharunderscore}{\kern0pt}imp{\isacharunderscore}{\kern0pt}win{\isacharunderscore}{\kern0pt}count\ winner\isanewline
\ \ \ \ \ \ \isacommand{by}\isamarkupfalse%
\ simp\isanewline
\ \ \ \ \isacommand{have}\isamarkupfalse%
\ f{\isadigit{2}}{\isacharcolon}{\kern0pt}\ {\isachardoublequoteopen}card\ {\isacharbraceleft}{\kern0pt}a\ {\isasymin}\ A{\isachardot}{\kern0pt}\ wins\ a\ p\ w{\isacharbraceright}{\kern0pt}\ {\isacharequal}{\kern0pt}\ {\isadigit{0}}{\isachardoublequoteclose}\isanewline
\ \ \ \ \ \ \isacommand{using}\isamarkupfalse%
\ cond{\isacharunderscore}{\kern0pt}winner{\isacharunderscore}{\kern0pt}imp{\isacharunderscore}{\kern0pt}loss{\isacharunderscore}{\kern0pt}count\ winner\isanewline
\ \ \ \ \ \ \isacommand{by}\isamarkupfalse%
\ {\isacharparenleft}{\kern0pt}metis\ {\isacharparenleft}{\kern0pt}no{\isacharunderscore}{\kern0pt}types{\isacharparenright}{\kern0pt}{\isacharparenright}{\kern0pt}\isanewline
\ \ \ \ \isacommand{have}\isamarkupfalse%
\ {\isachardoublequoteopen}card\ A\ {\isacharminus}{\kern0pt}\ {\isadigit{1}}\ {\isacharminus}{\kern0pt}\ {\isadigit{0}}\ {\isacharequal}{\kern0pt}\ card\ A\ {\isacharminus}{\kern0pt}\ {\isadigit{1}}{\isachardoublequoteclose}\isanewline
\ \ \ \ \ \ \isacommand{by}\isamarkupfalse%
\ simp\isanewline
\ \ \ \ \isacommand{thus}\isamarkupfalse%
\ {\isacharquery}{\kern0pt}thesis\isanewline
\ \ \ \ \ \ \isacommand{using}\isamarkupfalse%
\ f{\isadigit{2}}\ f{\isadigit{1}}\isanewline
\ \ \ \ \ \ \isacommand{by}\isamarkupfalse%
\ simp\isanewline
\ \ \isacommand{qed}\isamarkupfalse%
\isanewline
\isacommand{qed}\isamarkupfalse%
%
\endisatagproof
{\isafoldproof}%
%
\isadelimproof
\isanewline
%
\endisadelimproof
\isanewline
\isanewline
\isacommand{lemma}\isamarkupfalse%
\ non{\isacharunderscore}{\kern0pt}cond{\isacharunderscore}{\kern0pt}winner{\isacharunderscore}{\kern0pt}imp{\isacharunderscore}{\kern0pt}win{\isacharunderscore}{\kern0pt}count{\isacharcolon}{\kern0pt}\isanewline
\ \ \isakeyword{assumes}\isanewline
\ \ \ \ winner{\isacharcolon}{\kern0pt}\ {\isachardoublequoteopen}condorcet{\isacharunderscore}{\kern0pt}winner\ A\ p\ w{\isachardoublequoteclose}\ \isakeyword{and}\isanewline
\ \ \ \ loser{\isacharcolon}{\kern0pt}\ {\isachardoublequoteopen}l\ {\isasymnoteq}\ w{\isachardoublequoteclose}\ \isakeyword{and}\isanewline
\ \ \ \ l{\isacharunderscore}{\kern0pt}in{\isacharunderscore}{\kern0pt}A{\isacharcolon}{\kern0pt}\ {\isachardoublequoteopen}l\ {\isasymin}\ A{\isachardoublequoteclose}\isanewline
\ \ \isakeyword{shows}\ {\isachardoublequoteopen}card\ {\isacharbraceleft}{\kern0pt}y\ {\isasymin}\ A\ {\isachardot}{\kern0pt}\ wins\ l\ p\ y{\isacharbraceright}{\kern0pt}\ {\isacharless}{\kern0pt}{\isacharequal}{\kern0pt}\ card\ A\ {\isacharminus}{\kern0pt}\ {\isadigit{2}}{\isachardoublequoteclose}\isanewline
%
\isadelimproof
%
\endisadelimproof
%
\isatagproof
\isacommand{proof}\isamarkupfalse%
\ {\isacharminus}{\kern0pt}\isanewline
\ \ \isacommand{from}\isamarkupfalse%
\ winner\ loser\ l{\isacharunderscore}{\kern0pt}in{\isacharunderscore}{\kern0pt}A\isanewline
\ \ \isacommand{have}\isamarkupfalse%
\ {\isachardoublequoteopen}wins\ w\ p\ l{\isachardoublequoteclose}\isanewline
\ \ \ \ \isacommand{by}\isamarkupfalse%
\ simp\isanewline
\ \ \isacommand{hence}\isamarkupfalse%
\ {\isadigit{0}}{\isacharcolon}{\kern0pt}\ {\isachardoublequoteopen}{\isasymnot}\ wins\ l\ p\ w{\isachardoublequoteclose}\isanewline
\ \ \ \ \isacommand{by}\isamarkupfalse%
\ {\isacharparenleft}{\kern0pt}simp\ add{\isacharcolon}{\kern0pt}\ wins{\isacharunderscore}{\kern0pt}antisym{\isacharparenright}{\kern0pt}\isanewline
\ \ \isacommand{have}\isamarkupfalse%
\ {\isadigit{1}}{\isacharcolon}{\kern0pt}\ {\isachardoublequoteopen}{\isasymnot}\ wins\ l\ p\ l{\isachardoublequoteclose}\isanewline
\ \ \ \ \isacommand{by}\isamarkupfalse%
\ {\isacharparenleft}{\kern0pt}simp\ add{\isacharcolon}{\kern0pt}\ wins{\isacharunderscore}{\kern0pt}irreflex{\isacharparenright}{\kern0pt}\isanewline
\ \ \isacommand{from}\isamarkupfalse%
\ {\isadigit{0}}\ {\isadigit{1}}\ \isacommand{have}\isamarkupfalse%
\ {\isadigit{2}}{\isacharcolon}{\kern0pt}\isanewline
\ \ \ \ {\isachardoublequoteopen}{\isacharbraceleft}{\kern0pt}y\ {\isasymin}\ A\ {\isachardot}{\kern0pt}\ wins\ l\ p\ y{\isacharbraceright}{\kern0pt}\ {\isacharequal}{\kern0pt}\isanewline
\ \ \ \ \ \ \ \ {\isacharbraceleft}{\kern0pt}y\ {\isasymin}\ A{\isacharminus}{\kern0pt}{\isacharbraceleft}{\kern0pt}l{\isacharcomma}{\kern0pt}w{\isacharbraceright}{\kern0pt}\ {\isachardot}{\kern0pt}\ wins\ l\ p\ y{\isacharbraceright}{\kern0pt}{\isachardoublequoteclose}\isanewline
\ \ \ \ \isacommand{by}\isamarkupfalse%
\ blast\isanewline
\ \ \isacommand{have}\isamarkupfalse%
\ {\isadigit{3}}{\isacharcolon}{\kern0pt}\ {\isachardoublequoteopen}{\isasymforall}\ M\ f\ {\isachardot}{\kern0pt}\ finite\ M\ {\isasymlongrightarrow}\ card\ {\isacharbraceleft}{\kern0pt}x\ {\isasymin}\ M\ {\isachardot}{\kern0pt}\ f\ x{\isacharbraceright}{\kern0pt}\ {\isasymle}\ card\ M{\isachardoublequoteclose}\isanewline
\ \ \ \ \isacommand{by}\isamarkupfalse%
\ {\isacharparenleft}{\kern0pt}simp\ add{\isacharcolon}{\kern0pt}\ card{\isacharunderscore}{\kern0pt}mono{\isacharparenright}{\kern0pt}\isanewline
\ \ \isacommand{have}\isamarkupfalse%
\ {\isadigit{4}}{\isacharcolon}{\kern0pt}\ {\isachardoublequoteopen}finite\ {\isacharparenleft}{\kern0pt}A{\isacharminus}{\kern0pt}{\isacharbraceleft}{\kern0pt}l{\isacharcomma}{\kern0pt}w{\isacharbraceright}{\kern0pt}{\isacharparenright}{\kern0pt}{\isachardoublequoteclose}\isanewline
\ \ \ \ \isacommand{using}\isamarkupfalse%
\ condorcet{\isacharunderscore}{\kern0pt}winner{\isachardot}{\kern0pt}simps\ finite{\isacharunderscore}{\kern0pt}Diff\ winner\isanewline
\ \ \ \ \isacommand{by}\isamarkupfalse%
\ metis\isanewline
\ \ \isacommand{from}\isamarkupfalse%
\ {\isadigit{3}}\ {\isadigit{4}}\ \isacommand{have}\isamarkupfalse%
\ {\isadigit{5}}{\isacharcolon}{\kern0pt}\isanewline
\ \ \ \ {\isachardoublequoteopen}card\ {\isacharbraceleft}{\kern0pt}y\ {\isasymin}\ A{\isacharminus}{\kern0pt}{\isacharbraceleft}{\kern0pt}l{\isacharcomma}{\kern0pt}w{\isacharbraceright}{\kern0pt}\ {\isachardot}{\kern0pt}\ wins\ l\ p\ y{\isacharbraceright}{\kern0pt}\ {\isasymle}\isanewline
\ \ \ \ \ \ card\ {\isacharparenleft}{\kern0pt}A{\isacharminus}{\kern0pt}{\isacharbraceleft}{\kern0pt}l{\isacharcomma}{\kern0pt}w{\isacharbraceright}{\kern0pt}{\isacharparenright}{\kern0pt}{\isachardoublequoteclose}\isanewline
\ \ \ \ \isacommand{by}\isamarkupfalse%
\ {\isacharparenleft}{\kern0pt}metis\ {\isacharparenleft}{\kern0pt}full{\isacharunderscore}{\kern0pt}types{\isacharparenright}{\kern0pt}{\isacharparenright}{\kern0pt}\isanewline
\ \ \isacommand{have}\isamarkupfalse%
\ {\isachardoublequoteopen}w\ {\isasymin}\ A{\isachardoublequoteclose}\isanewline
\ \ \ \ \isacommand{using}\isamarkupfalse%
\ condorcet{\isacharunderscore}{\kern0pt}winner{\isachardot}{\kern0pt}simps\ winner\isanewline
\ \ \ \ \isacommand{by}\isamarkupfalse%
\ metis\isanewline
\ \ \isacommand{with}\isamarkupfalse%
\ l{\isacharunderscore}{\kern0pt}in{\isacharunderscore}{\kern0pt}A\isanewline
\ \ \isacommand{have}\isamarkupfalse%
\ {\isachardoublequoteopen}card{\isacharparenleft}{\kern0pt}A{\isacharminus}{\kern0pt}{\isacharbraceleft}{\kern0pt}l{\isacharcomma}{\kern0pt}w{\isacharbraceright}{\kern0pt}{\isacharparenright}{\kern0pt}\ {\isacharequal}{\kern0pt}\ card\ A\ {\isacharminus}{\kern0pt}\ card\ {\isacharbraceleft}{\kern0pt}l{\isacharcomma}{\kern0pt}w{\isacharbraceright}{\kern0pt}{\isachardoublequoteclose}\isanewline
\ \ \ \ \isacommand{by}\isamarkupfalse%
\ {\isacharparenleft}{\kern0pt}simp\ add{\isacharcolon}{\kern0pt}\ card{\isacharunderscore}{\kern0pt}Diff{\isacharunderscore}{\kern0pt}subset{\isacharparenright}{\kern0pt}\isanewline
\ \ \isacommand{hence}\isamarkupfalse%
\ {\isachardoublequoteopen}card{\isacharparenleft}{\kern0pt}A{\isacharminus}{\kern0pt}{\isacharbraceleft}{\kern0pt}l{\isacharcomma}{\kern0pt}w{\isacharbraceright}{\kern0pt}{\isacharparenright}{\kern0pt}\ {\isacharequal}{\kern0pt}\ card\ A\ {\isacharminus}{\kern0pt}\ {\isadigit{2}}{\isachardoublequoteclose}\isanewline
\ \ \ \ \isacommand{by}\isamarkupfalse%
\ {\isacharparenleft}{\kern0pt}simp\ add{\isacharcolon}{\kern0pt}\ loser{\isacharparenright}{\kern0pt}\isanewline
\ \ \isacommand{with}\isamarkupfalse%
\ {\isadigit{5}}\ {\isadigit{2}}\isanewline
\ \ \isacommand{show}\isamarkupfalse%
\ {\isacharquery}{\kern0pt}thesis\isanewline
\ \ \ \ \isacommand{by}\isamarkupfalse%
\ simp\isanewline
\isacommand{qed}\isamarkupfalse%
%
\endisatagproof
{\isafoldproof}%
%
\isadelimproof
%
\endisadelimproof
%
\isadelimdocument
%
\endisadelimdocument
%
\isatagdocument
%
\isamarkupsection{Minimax Module%
}
\isamarkuptrue%
%
\endisatagdocument
{\isafolddocument}%
%
\isadelimdocument
%
\endisadelimdocument
%
\begin{isamarkuptext}%
This is the Minimax module used by the Minimax voting rule. The Minimax rule
elects the alternatives with the highest Minimax score. The module implemented
herein only rejects the alternatives not elected by the voting rule, and defers
the alternatives that would be elected by the full voting rule.%
\end{isamarkuptext}\isamarkuptrue%
%
\isadelimdocument
%
\endisadelimdocument
%
\isatagdocument
%
\isamarkupsubsection{Definition%
}
\isamarkuptrue%
%
\endisatagdocument
{\isafolddocument}%
%
\isadelimdocument
%
\endisadelimdocument
\isacommand{fun}\isamarkupfalse%
\ minimax{\isacharunderscore}{\kern0pt}score\ {\isacharcolon}{\kern0pt}{\isacharcolon}{\kern0pt}\ {\isachardoublequoteopen}{\isacharprime}{\kern0pt}a\ Evaluation{\isacharunderscore}{\kern0pt}Function{\isachardoublequoteclose}\ \isakeyword{where}\isanewline
\ \ {\isachardoublequoteopen}minimax{\isacharunderscore}{\kern0pt}score\ x\ A\ p\ {\isacharequal}{\kern0pt}\isanewline
\ \ \ \ Min\ {\isacharbraceleft}{\kern0pt}prefer{\isacharunderscore}{\kern0pt}count\ p\ x\ y\ {\isacharbar}{\kern0pt}y\ {\isachardot}{\kern0pt}\ y\ {\isasymin}\ A{\isacharminus}{\kern0pt}{\isacharbraceleft}{\kern0pt}x{\isacharbraceright}{\kern0pt}{\isacharbraceright}{\kern0pt}{\isachardoublequoteclose}\isanewline
\isanewline
\isacommand{fun}\isamarkupfalse%
\ minimax\ {\isacharcolon}{\kern0pt}{\isacharcolon}{\kern0pt}\ {\isachardoublequoteopen}{\isacharprime}{\kern0pt}a\ Electoral{\isacharunderscore}{\kern0pt}Module{\isachardoublequoteclose}\ \isakeyword{where}\isanewline
\ \ {\isachardoublequoteopen}minimax\ A\ p\ {\isacharequal}{\kern0pt}\ max{\isacharunderscore}{\kern0pt}eliminator\ minimax{\isacharunderscore}{\kern0pt}score\ A\ p{\isachardoublequoteclose}%
\isadelimdocument
%
\endisadelimdocument
%
\isatagdocument
%
\isamarkupsubsection{Lemma%
}
\isamarkuptrue%
%
\endisatagdocument
{\isafolddocument}%
%
\isadelimdocument
%
\endisadelimdocument
\isacommand{lemma}\isamarkupfalse%
\ non{\isacharunderscore}{\kern0pt}cond{\isacharunderscore}{\kern0pt}winner{\isacharunderscore}{\kern0pt}minimax{\isacharunderscore}{\kern0pt}score{\isacharcolon}{\kern0pt}\isanewline
\ \ \isakeyword{assumes}\isanewline
\ \ \ \ prof{\isacharcolon}{\kern0pt}\ {\isachardoublequoteopen}profile\ A\ p{\isachardoublequoteclose}\ \isakeyword{and}\isanewline
\ \ \ \ winner{\isacharcolon}{\kern0pt}\ {\isachardoublequoteopen}condorcet{\isacharunderscore}{\kern0pt}winner\ A\ p\ w{\isachardoublequoteclose}\ \isakeyword{and}\isanewline
\ \ \ \ l{\isacharunderscore}{\kern0pt}in{\isacharunderscore}{\kern0pt}A{\isacharcolon}{\kern0pt}\ {\isachardoublequoteopen}l\ {\isasymin}\ A{\isachardoublequoteclose}\ \isakeyword{and}\isanewline
\ \ \ \ l{\isacharunderscore}{\kern0pt}neq{\isacharunderscore}{\kern0pt}w{\isacharcolon}{\kern0pt}\ {\isachardoublequoteopen}l\ {\isasymnoteq}\ w{\isachardoublequoteclose}\isanewline
\ \ \isakeyword{shows}\ {\isachardoublequoteopen}minimax{\isacharunderscore}{\kern0pt}score\ l\ A\ p\ {\isasymle}\ prefer{\isacharunderscore}{\kern0pt}count\ p\ l\ w{\isachardoublequoteclose}\isanewline
%
\isadelimproof
%
\endisadelimproof
%
\isatagproof
\isacommand{proof}\isamarkupfalse%
\ {\isacharminus}{\kern0pt}\isanewline
\ \ \isacommand{let}\isamarkupfalse%
\isanewline
\ \ \ \ {\isacharquery}{\kern0pt}set\ {\isacharequal}{\kern0pt}\ {\isachardoublequoteopen}{\isacharbraceleft}{\kern0pt}prefer{\isacharunderscore}{\kern0pt}count\ p\ l\ y\ {\isacharbar}{\kern0pt}y\ {\isachardot}{\kern0pt}\ y\ {\isasymin}\ A{\isacharminus}{\kern0pt}{\isacharbraceleft}{\kern0pt}l{\isacharbraceright}{\kern0pt}{\isacharbraceright}{\kern0pt}{\isachardoublequoteclose}\ \isakeyword{and}\isanewline
\ \ \ \ \ \ {\isacharquery}{\kern0pt}lscore\ {\isacharequal}{\kern0pt}\ {\isachardoublequoteopen}minimax{\isacharunderscore}{\kern0pt}score\ l\ A\ p{\isachardoublequoteclose}\isanewline
\ \ \isacommand{have}\isamarkupfalse%
\ {\isachardoublequoteopen}finite\ A{\isachardoublequoteclose}\isanewline
\ \ \ \ \isacommand{using}\isamarkupfalse%
\ prof\ condorcet{\isacharunderscore}{\kern0pt}winner{\isachardot}{\kern0pt}simps\ winner\isanewline
\ \ \ \ \isacommand{by}\isamarkupfalse%
\ metis\isanewline
\ \ \isacommand{hence}\isamarkupfalse%
\ {\isachardoublequoteopen}finite\ {\isacharparenleft}{\kern0pt}A{\isacharminus}{\kern0pt}{\isacharbraceleft}{\kern0pt}l{\isacharbraceright}{\kern0pt}{\isacharparenright}{\kern0pt}{\isachardoublequoteclose}\isanewline
\ \ \ \ \isacommand{using}\isamarkupfalse%
\ finite{\isacharunderscore}{\kern0pt}Diff\isanewline
\ \ \ \ \isacommand{by}\isamarkupfalse%
\ simp\isanewline
\ \ \isacommand{hence}\isamarkupfalse%
\ finite{\isacharcolon}{\kern0pt}\ {\isachardoublequoteopen}finite\ {\isacharquery}{\kern0pt}set{\isachardoublequoteclose}\isanewline
\ \ \ \ \isacommand{by}\isamarkupfalse%
\ simp\isanewline
\ \ \isacommand{have}\isamarkupfalse%
\ {\isachardoublequoteopen}w\ {\isasymin}\ A{\isachardoublequoteclose}\isanewline
\ \ \ \ \isacommand{using}\isamarkupfalse%
\ condorcet{\isacharunderscore}{\kern0pt}winner{\isachardot}{\kern0pt}simps\ winner\isanewline
\ \ \ \ \isacommand{by}\isamarkupfalse%
\ metis\isanewline
\ \ \isacommand{hence}\isamarkupfalse%
\ {\isadigit{0}}{\isacharcolon}{\kern0pt}\ {\isachardoublequoteopen}w\ {\isasymin}\ A{\isacharminus}{\kern0pt}{\isacharbraceleft}{\kern0pt}l{\isacharbraceright}{\kern0pt}{\isachardoublequoteclose}\isanewline
\ \ \ \ \isacommand{using}\isamarkupfalse%
\ l{\isacharunderscore}{\kern0pt}neq{\isacharunderscore}{\kern0pt}w\isanewline
\ \ \ \ \isacommand{by}\isamarkupfalse%
\ force\isanewline
\ \ \isacommand{hence}\isamarkupfalse%
\ not{\isacharunderscore}{\kern0pt}empty{\isacharcolon}{\kern0pt}\ {\isachardoublequoteopen}{\isacharquery}{\kern0pt}set\ {\isasymnoteq}\ {\isacharbraceleft}{\kern0pt}{\isacharbraceright}{\kern0pt}{\isachardoublequoteclose}\isanewline
\ \ \ \ \isacommand{by}\isamarkupfalse%
\ blast\isanewline
\ \ \isanewline
\ \ \isacommand{have}\isamarkupfalse%
\ {\isachardoublequoteopen}{\isacharquery}{\kern0pt}lscore\ {\isacharequal}{\kern0pt}\ Min\ {\isacharquery}{\kern0pt}set{\isachardoublequoteclose}\isanewline
\ \ \ \ \isacommand{by}\isamarkupfalse%
\ simp\isanewline
\ \ \isacommand{hence}\isamarkupfalse%
\ {\isadigit{1}}{\isacharcolon}{\kern0pt}\ {\isachardoublequoteopen}{\isacharquery}{\kern0pt}lscore\ {\isasymin}\ {\isacharquery}{\kern0pt}set\ {\isasymand}\ {\isacharparenleft}{\kern0pt}{\isasymforall}p\ {\isasymin}\ {\isacharquery}{\kern0pt}set{\isachardot}{\kern0pt}\ {\isacharquery}{\kern0pt}lscore\ {\isasymle}\ p{\isacharparenright}{\kern0pt}{\isachardoublequoteclose}\isanewline
\ \ \ \ \isacommand{using}\isamarkupfalse%
\ local{\isachardot}{\kern0pt}finite\ not{\isacharunderscore}{\kern0pt}empty\ Min{\isacharunderscore}{\kern0pt}le\ Min{\isacharunderscore}{\kern0pt}eq{\isacharunderscore}{\kern0pt}iff\isanewline
\ \ \ \ \isacommand{by}\isamarkupfalse%
\ {\isacharparenleft}{\kern0pt}metis\ {\isacharparenleft}{\kern0pt}no{\isacharunderscore}{\kern0pt}types{\isacharcomma}{\kern0pt}\ lifting{\isacharparenright}{\kern0pt}{\isacharparenright}{\kern0pt}\isanewline
\ \ \isacommand{thus}\isamarkupfalse%
\ {\isacharquery}{\kern0pt}thesis\isanewline
\ \ \ \ \isacommand{using}\isamarkupfalse%
\ {\isachardoublequoteopen}{\isadigit{0}}{\isachardoublequoteclose}\isanewline
\ \ \ \ \isacommand{by}\isamarkupfalse%
\ auto\isanewline
\isacommand{qed}\isamarkupfalse%
%
\endisatagproof
{\isafoldproof}%
%
\isadelimproof
\isanewline
%
\endisadelimproof
%
\isadelimtheory
\isanewline
%
\endisadelimtheory
%
\isatagtheory
\isacommand{end}\isamarkupfalse%
%
\endisatagtheory
{\isafoldtheory}%
%
\isadelimtheory
%
\endisadelimtheory
%
\end{isabellebody}%
\endinput
%:%file=~/Documents/Studies/VotingRuleGenerator/virage/src/test/resources/verifiedVotingRuleConstruction/theories/Compositional_Framework/Components/Composites/Composite_Elimination_Modules.thy%:%
%:%10=1%:%
%:%11=1%:%
%:%12=2%:%
%:%13=3%:%
%:%14=4%:%
%:%15=5%:%
%:%16=6%:%
%:%30=8%:%
%:%42=11%:%
%:%43=12%:%
%:%44=13%:%
%:%45=14%:%
%:%46=15%:%
%:%47=16%:%
%:%48=17%:%
%:%57=19%:%
%:%67=21%:%
%:%68=21%:%
%:%69=22%:%
%:%70=23%:%
%:%71=24%:%
%:%72=24%:%
%:%73=25%:%
%:%80=27%:%
%:%92=30%:%
%:%93=31%:%
%:%94=32%:%
%:%95=33%:%
%:%96=34%:%
%:%97=35%:%
%:%106=37%:%
%:%116=39%:%
%:%117=39%:%
%:%118=40%:%
%:%119=41%:%
%:%120=42%:%
%:%121=43%:%
%:%122=43%:%
%:%123=44%:%
%:%130=46%:%
%:%142=49%:%
%:%143=50%:%
%:%144=51%:%
%:%145=52%:%
%:%146=53%:%
%:%147=54%:%
%:%156=56%:%
%:%166=58%:%
%:%167=58%:%
%:%168=59%:%
%:%169=60%:%
%:%170=61%:%
%:%171=62%:%
%:%172=62%:%
%:%173=63%:%
%:%180=65%:%
%:%190=68%:%
%:%191=68%:%
%:%192=69%:%
%:%193=70%:%
%:%200=71%:%
%:%201=71%:%
%:%202=72%:%
%:%203=72%:%
%:%204=73%:%
%:%205=73%:%
%:%206=74%:%
%:%207=74%:%
%:%208=75%:%
%:%209=75%:%
%:%210=76%:%
%:%211=76%:%
%:%212=77%:%
%:%213=77%:%
%:%214=78%:%
%:%215=78%:%
%:%216=79%:%
%:%217=79%:%
%:%218=80%:%
%:%219=80%:%
%:%220=81%:%
%:%221=81%:%
%:%222=82%:%
%:%223=82%:%
%:%224=83%:%
%:%225=83%:%
%:%226=84%:%
%:%227=84%:%
%:%228=85%:%
%:%229=85%:%
%:%230=86%:%
%:%231=86%:%
%:%232=87%:%
%:%233=87%:%
%:%234=88%:%
%:%235=88%:%
%:%236=89%:%
%:%237=90%:%
%:%238=90%:%
%:%239=91%:%
%:%240=91%:%
%:%241=92%:%
%:%242=92%:%
%:%243=93%:%
%:%244=93%:%
%:%245=94%:%
%:%246=94%:%
%:%247=95%:%
%:%248=95%:%
%:%249=96%:%
%:%250=96%:%
%:%251=97%:%
%:%252=97%:%
%:%253=98%:%
%:%254=98%:%
%:%255=99%:%
%:%256=99%:%
%:%257=100%:%
%:%258=100%:%
%:%259=101%:%
%:%260=101%:%
%:%261=102%:%
%:%262=103%:%
%:%263=103%:%
%:%264=104%:%
%:%265=104%:%
%:%266=105%:%
%:%267=106%:%
%:%268=107%:%
%:%269=107%:%
%:%270=108%:%
%:%271=108%:%
%:%272=109%:%
%:%273=109%:%
%:%274=110%:%
%:%275=110%:%
%:%276=111%:%
%:%277=111%:%
%:%278=112%:%
%:%279=112%:%
%:%280=113%:%
%:%281=113%:%
%:%282=114%:%
%:%283=114%:%
%:%284=115%:%
%:%285=115%:%
%:%286=116%:%
%:%287=117%:%
%:%288=118%:%
%:%289=118%:%
%:%290=119%:%
%:%291=119%:%
%:%292=120%:%
%:%293=120%:%
%:%294=121%:%
%:%295=122%:%
%:%296=122%:%
%:%297=123%:%
%:%298=123%:%
%:%299=124%:%
%:%300=124%:%
%:%301=125%:%
%:%302=125%:%
%:%303=126%:%
%:%309=126%:%
%:%312=127%:%
%:%313=128%:%
%:%314=129%:%
%:%315=129%:%
%:%316=130%:%
%:%317=131%:%
%:%320=132%:%
%:%324=132%:%
%:%325=132%:%
%:%326=133%:%
%:%327=134%:%
%:%328=134%:%
%:%333=134%:%
%:%336=135%:%
%:%337=136%:%
%:%338=137%:%
%:%339=137%:%
%:%340=138%:%
%:%341=139%:%
%:%344=140%:%
%:%348=140%:%
%:%349=140%:%
%:%350=141%:%
%:%351=141%:%
%:%352=142%:%
%:%353=142%:%
%:%354=143%:%
%:%355=144%:%
%:%356=145%:%
%:%357=145%:%
%:%358=146%:%
%:%359=147%:%
%:%360=147%:%
%:%361=148%:%
%:%362=148%:%
%:%363=149%:%
%:%364=149%:%
%:%365=150%:%
%:%366=150%:%
%:%367=151%:%
%:%368=151%:%
%:%369=152%:%
%:%370=152%:%
%:%371=153%:%
%:%372=153%:%
%:%373=154%:%
%:%374=154%:%
%:%375=155%:%
%:%376=155%:%
%:%377=156%:%
%:%378=156%:%
%:%379=157%:%
%:%380=157%:%
%:%381=158%:%
%:%382=158%:%
%:%383=159%:%
%:%384=159%:%
%:%385=160%:%
%:%391=160%:%
%:%394=161%:%
%:%395=165%:%
%:%396=166%:%
%:%397=166%:%
%:%398=167%:%
%:%399=168%:%
%:%400=169%:%
%:%401=170%:%
%:%402=171%:%
%:%409=172%:%
%:%410=172%:%
%:%411=173%:%
%:%412=173%:%
%:%413=174%:%
%:%414=174%:%
%:%415=175%:%
%:%416=175%:%
%:%417=176%:%
%:%418=176%:%
%:%419=177%:%
%:%420=177%:%
%:%421=178%:%
%:%422=178%:%
%:%423=179%:%
%:%424=179%:%
%:%425=180%:%
%:%426=180%:%
%:%427=180%:%
%:%428=181%:%
%:%429=182%:%
%:%430=183%:%
%:%431=183%:%
%:%432=184%:%
%:%433=184%:%
%:%434=185%:%
%:%435=185%:%
%:%436=186%:%
%:%437=186%:%
%:%438=187%:%
%:%439=187%:%
%:%440=188%:%
%:%441=188%:%
%:%442=189%:%
%:%443=189%:%
%:%444=189%:%
%:%445=190%:%
%:%446=191%:%
%:%447=192%:%
%:%448=192%:%
%:%449=193%:%
%:%450=193%:%
%:%451=194%:%
%:%452=194%:%
%:%453=195%:%
%:%454=195%:%
%:%455=196%:%
%:%456=196%:%
%:%457=197%:%
%:%458=197%:%
%:%459=198%:%
%:%460=198%:%
%:%461=199%:%
%:%462=199%:%
%:%463=200%:%
%:%464=200%:%
%:%465=201%:%
%:%466=201%:%
%:%467=202%:%
%:%468=202%:%
%:%469=203%:%
%:%470=203%:%
%:%471=204%:%
%:%486=206%:%
%:%498=209%:%
%:%499=210%:%
%:%500=211%:%
%:%501=212%:%
%:%510=214%:%
%:%520=216%:%
%:%521=216%:%
%:%522=217%:%
%:%523=218%:%
%:%524=219%:%
%:%525=220%:%
%:%526=220%:%
%:%527=221%:%
%:%534=223%:%
%:%544=225%:%
%:%545=225%:%
%:%546=226%:%
%:%547=227%:%
%:%548=228%:%
%:%549=229%:%
%:%550=230%:%
%:%551=231%:%
%:%558=232%:%
%:%559=232%:%
%:%560=233%:%
%:%561=233%:%
%:%562=234%:%
%:%563=235%:%
%:%564=236%:%
%:%565=236%:%
%:%566=237%:%
%:%567=237%:%
%:%568=238%:%
%:%569=238%:%
%:%570=239%:%
%:%571=239%:%
%:%572=240%:%
%:%573=240%:%
%:%574=241%:%
%:%575=241%:%
%:%576=242%:%
%:%577=242%:%
%:%578=243%:%
%:%579=243%:%
%:%580=244%:%
%:%581=244%:%
%:%582=245%:%
%:%583=245%:%
%:%584=246%:%
%:%585=246%:%
%:%586=247%:%
%:%587=247%:%
%:%588=248%:%
%:%589=248%:%
%:%590=249%:%
%:%591=249%:%
%:%592=250%:%
%:%593=250%:%
%:%594=251%:%
%:%595=251%:%
%:%596=252%:%
%:%596=253%:%
%:%597=254%:%
%:%598=254%:%
%:%599=255%:%
%:%600=255%:%
%:%601=256%:%
%:%602=256%:%
%:%603=257%:%
%:%604=257%:%
%:%605=258%:%
%:%606=258%:%
%:%607=259%:%
%:%608=259%:%
%:%609=260%:%
%:%610=260%:%
%:%611=261%:%
%:%612=261%:%
%:%613=262%:%
%:%619=262%:%
%:%624=263%:%
%:%629=264%:%
%
\begin{isabellebody}%
\setisabellecontext{Result{\isacharunderscore}{\kern0pt}Properties}%
%
\isadelimtheory
%
\endisadelimtheory
%
\isatagtheory
\isacommand{theory}\isamarkupfalse%
\ Result{\isacharunderscore}{\kern0pt}Properties\isanewline
\ \ \isakeyword{imports}\ {\isachardoublequoteopen}{\isachardot}{\kern0pt}{\isachardot}{\kern0pt}{\isacharslash}{\kern0pt}Components{\isacharslash}{\kern0pt}Electoral{\isacharunderscore}{\kern0pt}Module{\isachardoublequoteclose}\isanewline
\isanewline
\isakeyword{begin}%
\endisatagtheory
{\isafoldtheory}%
%
\isadelimtheory
\isanewline
%
\endisadelimtheory
\isanewline
\isanewline
\isacommand{definition}\isamarkupfalse%
\ electing\ {\isacharcolon}{\kern0pt}{\isacharcolon}{\kern0pt}\ {\isachardoublequoteopen}{\isacharprime}{\kern0pt}a\ Electoral{\isacharunderscore}{\kern0pt}Module\ {\isasymRightarrow}\ bool{\isachardoublequoteclose}\ \isakeyword{where}\isanewline
\ \ {\isachardoublequoteopen}electing\ m\ {\isasymequiv}\isanewline
\ \ \ \ electoral{\isacharunderscore}{\kern0pt}module\ m\ {\isasymand}\isanewline
\ \ \ \ \ \ {\isacharparenleft}{\kern0pt}{\isasymforall}A\ p{\isachardot}{\kern0pt}\ {\isacharparenleft}{\kern0pt}A\ {\isasymnoteq}\ {\isacharbraceleft}{\kern0pt}{\isacharbraceright}{\kern0pt}\ {\isasymand}\ finite{\isacharunderscore}{\kern0pt}profile\ A\ p{\isacharparenright}{\kern0pt}\ {\isasymlongrightarrow}\ elect\ m\ A\ p\ {\isasymnoteq}\ {\isacharbraceleft}{\kern0pt}{\isacharbraceright}{\kern0pt}{\isacharparenright}{\kern0pt}{\isachardoublequoteclose}\isanewline
\isanewline
\isacommand{lemma}\isamarkupfalse%
\ electing{\isacharunderscore}{\kern0pt}for{\isacharunderscore}{\kern0pt}only{\isacharunderscore}{\kern0pt}alt{\isacharcolon}{\kern0pt}\isanewline
\ \ \isakeyword{assumes}\isanewline
\ \ \ \ one{\isacharunderscore}{\kern0pt}alt{\isacharcolon}{\kern0pt}\ {\isachardoublequoteopen}card\ A\ {\isacharequal}{\kern0pt}\ {\isadigit{1}}{\isachardoublequoteclose}\ \isakeyword{and}\isanewline
\ \ \ \ electing{\isacharcolon}{\kern0pt}\ {\isachardoublequoteopen}electing\ m{\isachardoublequoteclose}\ \isakeyword{and}\isanewline
\ \ \ \ f{\isacharunderscore}{\kern0pt}prof{\isacharcolon}{\kern0pt}\ {\isachardoublequoteopen}finite{\isacharunderscore}{\kern0pt}profile\ A\ p{\isachardoublequoteclose}\isanewline
\ \ \isakeyword{shows}\ {\isachardoublequoteopen}elect\ m\ A\ p\ {\isacharequal}{\kern0pt}\ A{\isachardoublequoteclose}\isanewline
%
\isadelimproof
\ \ %
\endisadelimproof
%
\isatagproof
\isacommand{using}\isamarkupfalse%
\ Int{\isacharunderscore}{\kern0pt}empty{\isacharunderscore}{\kern0pt}right\ Int{\isacharunderscore}{\kern0pt}insert{\isacharunderscore}{\kern0pt}right\ card{\isacharunderscore}{\kern0pt}{\isadigit{1}}{\isacharunderscore}{\kern0pt}singletonE\isanewline
\ \ \ \ \ \ \ \ elect{\isacharunderscore}{\kern0pt}in{\isacharunderscore}{\kern0pt}alts\ electing\ electing{\isacharunderscore}{\kern0pt}def\ inf{\isachardot}{\kern0pt}orderE\isanewline
\ \ \ \ \ \ \ \ one{\isacharunderscore}{\kern0pt}alt\ f{\isacharunderscore}{\kern0pt}prof\isanewline
\ \ \isacommand{by}\isamarkupfalse%
\ {\isacharparenleft}{\kern0pt}smt\ {\isacharparenleft}{\kern0pt}verit{\isacharcomma}{\kern0pt}\ del{\isacharunderscore}{\kern0pt}insts{\isacharparenright}{\kern0pt}{\isacharparenright}{\kern0pt}%
\endisatagproof
{\isafoldproof}%
%
\isadelimproof
\isanewline
%
\endisadelimproof
\isanewline
\isanewline
\isanewline
\isacommand{definition}\isamarkupfalse%
\ non{\isacharunderscore}{\kern0pt}electing\ {\isacharcolon}{\kern0pt}{\isacharcolon}{\kern0pt}\ {\isachardoublequoteopen}{\isacharprime}{\kern0pt}a\ Electoral{\isacharunderscore}{\kern0pt}Module\ {\isasymRightarrow}\ bool{\isachardoublequoteclose}\ \isakeyword{where}\isanewline
\ \ {\isachardoublequoteopen}non{\isacharunderscore}{\kern0pt}electing\ m\ {\isasymequiv}\isanewline
\ \ \ \ electoral{\isacharunderscore}{\kern0pt}module\ m\ {\isasymand}\ {\isacharparenleft}{\kern0pt}{\isasymforall}A\ p{\isachardot}{\kern0pt}\ finite{\isacharunderscore}{\kern0pt}profile\ A\ p\ {\isasymlongrightarrow}\ elect\ m\ A\ p\ {\isacharequal}{\kern0pt}\ {\isacharbraceleft}{\kern0pt}{\isacharbraceright}{\kern0pt}{\isacharparenright}{\kern0pt}{\isachardoublequoteclose}\isanewline
\isanewline
\isanewline
\isacommand{definition}\isamarkupfalse%
\ decrementing\ {\isacharcolon}{\kern0pt}{\isacharcolon}{\kern0pt}\ {\isachardoublequoteopen}{\isacharprime}{\kern0pt}a\ Electoral{\isacharunderscore}{\kern0pt}Module\ {\isasymRightarrow}\ bool{\isachardoublequoteclose}\ \isakeyword{where}\isanewline
\ \ {\isachardoublequoteopen}decrementing\ m\ {\isasymequiv}\isanewline
\ \ \ \ electoral{\isacharunderscore}{\kern0pt}module\ m\ {\isasymand}\ {\isacharparenleft}{\kern0pt}\isanewline
\ \ \ \ \ \ {\isasymforall}\ A\ p\ {\isachardot}{\kern0pt}\ finite{\isacharunderscore}{\kern0pt}profile\ A\ p\ {\isasymlongrightarrow}\isanewline
\ \ \ \ \ \ \ \ \ \ {\isacharparenleft}{\kern0pt}card\ A\ {\isachargreater}{\kern0pt}\ {\isadigit{1}}\ {\isasymlongrightarrow}\ card\ {\isacharparenleft}{\kern0pt}reject\ m\ A\ p{\isacharparenright}{\kern0pt}\ {\isasymge}\ {\isadigit{1}}{\isacharparenright}{\kern0pt}{\isacharparenright}{\kern0pt}{\isachardoublequoteclose}\isanewline
\isanewline
\isanewline
\isacommand{definition}\isamarkupfalse%
\ non{\isacharunderscore}{\kern0pt}blocking\ {\isacharcolon}{\kern0pt}{\isacharcolon}{\kern0pt}\ {\isachardoublequoteopen}{\isacharprime}{\kern0pt}a\ Electoral{\isacharunderscore}{\kern0pt}Module\ {\isasymRightarrow}\ bool{\isachardoublequoteclose}\ \isakeyword{where}\isanewline
\ \ {\isachardoublequoteopen}non{\isacharunderscore}{\kern0pt}blocking\ m\ {\isasymequiv}\isanewline
\ \ \ \ electoral{\isacharunderscore}{\kern0pt}module\ m\ {\isasymand}\isanewline
\ \ \ \ \ \ {\isacharparenleft}{\kern0pt}{\isasymforall}A\ p{\isachardot}{\kern0pt}\isanewline
\ \ \ \ \ \ \ \ \ \ {\isacharparenleft}{\kern0pt}{\isacharparenleft}{\kern0pt}A\ {\isasymnoteq}\ {\isacharbraceleft}{\kern0pt}{\isacharbraceright}{\kern0pt}\ {\isasymand}\ finite{\isacharunderscore}{\kern0pt}profile\ A\ p{\isacharparenright}{\kern0pt}\ {\isasymlongrightarrow}\ reject\ m\ A\ p\ {\isasymnoteq}\ A{\isacharparenright}{\kern0pt}{\isacharparenright}{\kern0pt}{\isachardoublequoteclose}\isanewline
\isanewline
\isanewline
\isanewline
\isacommand{definition}\isamarkupfalse%
\ elects\ {\isacharcolon}{\kern0pt}{\isacharcolon}{\kern0pt}\ {\isachardoublequoteopen}nat\ {\isasymRightarrow}\ {\isacharprime}{\kern0pt}a\ Electoral{\isacharunderscore}{\kern0pt}Module\ {\isasymRightarrow}\ bool{\isachardoublequoteclose}\ \isakeyword{where}\isanewline
\ \ {\isachardoublequoteopen}elects\ n\ m\ {\isasymequiv}\isanewline
\ \ \ \ electoral{\isacharunderscore}{\kern0pt}module\ m\ {\isasymand}\isanewline
\ \ \ \ \ \ {\isacharparenleft}{\kern0pt}{\isasymforall}A\ p{\isachardot}{\kern0pt}\ {\isacharparenleft}{\kern0pt}card\ A\ {\isasymge}\ n\ {\isasymand}\ finite{\isacharunderscore}{\kern0pt}profile\ A\ p{\isacharparenright}{\kern0pt}\ {\isasymlongrightarrow}\ card\ {\isacharparenleft}{\kern0pt}elect\ m\ A\ p{\isacharparenright}{\kern0pt}\ {\isacharequal}{\kern0pt}\ n{\isacharparenright}{\kern0pt}{\isachardoublequoteclose}\isanewline
\isanewline
\isanewline
\isacommand{definition}\isamarkupfalse%
\ defers\ {\isacharcolon}{\kern0pt}{\isacharcolon}{\kern0pt}\ {\isachardoublequoteopen}nat\ {\isasymRightarrow}\ {\isacharprime}{\kern0pt}a\ Electoral{\isacharunderscore}{\kern0pt}Module\ {\isasymRightarrow}\ bool{\isachardoublequoteclose}\ \isakeyword{where}\isanewline
\ \ {\isachardoublequoteopen}defers\ n\ m\ {\isasymequiv}\isanewline
\ \ \ \ electoral{\isacharunderscore}{\kern0pt}module\ m\ {\isasymand}\isanewline
\ \ \ \ \ \ {\isacharparenleft}{\kern0pt}{\isasymforall}A\ p{\isachardot}{\kern0pt}\ {\isacharparenleft}{\kern0pt}card\ A\ {\isasymge}\ n\ {\isasymand}\ finite{\isacharunderscore}{\kern0pt}profile\ A\ p{\isacharparenright}{\kern0pt}\ {\isasymlongrightarrow}\isanewline
\ \ \ \ \ \ \ \ \ \ card\ {\isacharparenleft}{\kern0pt}defer\ m\ A\ p{\isacharparenright}{\kern0pt}\ {\isacharequal}{\kern0pt}\ n{\isacharparenright}{\kern0pt}{\isachardoublequoteclose}\isanewline
\isanewline
\isanewline
\isacommand{definition}\isamarkupfalse%
\ rejects\ {\isacharcolon}{\kern0pt}{\isacharcolon}{\kern0pt}\ {\isachardoublequoteopen}nat\ {\isasymRightarrow}\ {\isacharprime}{\kern0pt}a\ Electoral{\isacharunderscore}{\kern0pt}Module\ {\isasymRightarrow}\ bool{\isachardoublequoteclose}\ \isakeyword{where}\isanewline
\ \ {\isachardoublequoteopen}rejects\ n\ m\ {\isasymequiv}\isanewline
\ \ \ \ electoral{\isacharunderscore}{\kern0pt}module\ m\ {\isasymand}\isanewline
\ \ \ \ \ \ {\isacharparenleft}{\kern0pt}{\isasymforall}A\ p{\isachardot}{\kern0pt}\ {\isacharparenleft}{\kern0pt}card\ A\ {\isasymge}\ n\ {\isasymand}\ finite{\isacharunderscore}{\kern0pt}profile\ A\ p{\isacharparenright}{\kern0pt}\ {\isasymlongrightarrow}\ card\ {\isacharparenleft}{\kern0pt}reject\ m\ A\ p{\isacharparenright}{\kern0pt}\ {\isacharequal}{\kern0pt}\ n{\isacharparenright}{\kern0pt}{\isachardoublequoteclose}\isanewline
\isanewline
\isanewline
\isacommand{definition}\isamarkupfalse%
\ eliminates\ {\isacharcolon}{\kern0pt}{\isacharcolon}{\kern0pt}\ {\isachardoublequoteopen}nat\ {\isasymRightarrow}\ {\isacharprime}{\kern0pt}a\ Electoral{\isacharunderscore}{\kern0pt}Module\ {\isasymRightarrow}\ bool{\isachardoublequoteclose}\ \isakeyword{where}\isanewline
\ \ {\isachardoublequoteopen}eliminates\ n\ m\ {\isasymequiv}\isanewline
\ \ \ \ electoral{\isacharunderscore}{\kern0pt}module\ m\ {\isasymand}\isanewline
\ \ \ \ \ \ {\isacharparenleft}{\kern0pt}{\isasymforall}A\ p{\isachardot}{\kern0pt}\ {\isacharparenleft}{\kern0pt}card\ A\ {\isachargreater}{\kern0pt}\ n\ {\isasymand}\ finite{\isacharunderscore}{\kern0pt}profile\ A\ p{\isacharparenright}{\kern0pt}\ {\isasymlongrightarrow}\ card\ {\isacharparenleft}{\kern0pt}reject\ m\ A\ p{\isacharparenright}{\kern0pt}\ {\isacharequal}{\kern0pt}\ n{\isacharparenright}{\kern0pt}{\isachardoublequoteclose}\isanewline
\isanewline
\isacommand{lemma}\isamarkupfalse%
\ single{\isacharunderscore}{\kern0pt}elim{\isacharunderscore}{\kern0pt}imp{\isacharunderscore}{\kern0pt}red{\isacharunderscore}{\kern0pt}def{\isacharunderscore}{\kern0pt}set{\isacharcolon}{\kern0pt}\isanewline
\ \ \isakeyword{assumes}\isanewline
\ \ \ \ eliminating{\isacharcolon}{\kern0pt}\ {\isachardoublequoteopen}eliminates\ {\isadigit{1}}\ m{\isachardoublequoteclose}\ \isakeyword{and}\isanewline
\ \ \ \ leftover{\isacharunderscore}{\kern0pt}alternatives{\isacharcolon}{\kern0pt}\ {\isachardoublequoteopen}card\ A\ {\isachargreater}{\kern0pt}\ {\isadigit{1}}{\isachardoublequoteclose}\ \isakeyword{and}\isanewline
\ \ \ \ f{\isacharunderscore}{\kern0pt}prof{\isacharcolon}{\kern0pt}\ {\isachardoublequoteopen}finite{\isacharunderscore}{\kern0pt}profile\ A\ p{\isachardoublequoteclose}\isanewline
\ \ \isakeyword{shows}\ {\isachardoublequoteopen}defer\ m\ A\ p\ {\isasymsubset}\ A{\isachardoublequoteclose}\isanewline
%
\isadelimproof
\ \ %
\endisadelimproof
%
\isatagproof
\isacommand{using}\isamarkupfalse%
\ Diff{\isacharunderscore}{\kern0pt}eq{\isacharunderscore}{\kern0pt}empty{\isacharunderscore}{\kern0pt}iff\ Diff{\isacharunderscore}{\kern0pt}subset\ card{\isacharunderscore}{\kern0pt}eq{\isacharunderscore}{\kern0pt}{\isadigit{0}}{\isacharunderscore}{\kern0pt}iff\ defer{\isacharunderscore}{\kern0pt}in{\isacharunderscore}{\kern0pt}alts\isanewline
\ \ \ \ \ \ \ \ eliminates{\isacharunderscore}{\kern0pt}def\ eliminating\ eq{\isacharunderscore}{\kern0pt}iff\ leftover{\isacharunderscore}{\kern0pt}alternatives\isanewline
\ \ \ \ \ \ \ \ not{\isacharunderscore}{\kern0pt}one{\isacharunderscore}{\kern0pt}le{\isacharunderscore}{\kern0pt}zero\ f{\isacharunderscore}{\kern0pt}prof\ psubsetI\ reject{\isacharunderscore}{\kern0pt}not{\isacharunderscore}{\kern0pt}elec{\isacharunderscore}{\kern0pt}or{\isacharunderscore}{\kern0pt}def\isanewline
\ \ \isacommand{by}\isamarkupfalse%
\ metis%
\endisatagproof
{\isafoldproof}%
%
\isadelimproof
\isanewline
%
\endisadelimproof
\isanewline
\isacommand{lemma}\isamarkupfalse%
\ single{\isacharunderscore}{\kern0pt}elim{\isacharunderscore}{\kern0pt}decr{\isacharunderscore}{\kern0pt}def{\isacharunderscore}{\kern0pt}card{\isacharcolon}{\kern0pt}\isanewline
\ \ \isakeyword{assumes}\isanewline
\ \ \ \ rejecting{\isacharcolon}{\kern0pt}\ {\isachardoublequoteopen}rejects\ {\isadigit{1}}\ m{\isachardoublequoteclose}\ \isakeyword{and}\isanewline
\ \ \ \ not{\isacharunderscore}{\kern0pt}empty{\isacharcolon}{\kern0pt}\ {\isachardoublequoteopen}A\ {\isasymnoteq}\ {\isacharbraceleft}{\kern0pt}{\isacharbraceright}{\kern0pt}{\isachardoublequoteclose}\ \isakeyword{and}\isanewline
\ \ \ \ non{\isacharunderscore}{\kern0pt}electing{\isacharcolon}{\kern0pt}\ {\isachardoublequoteopen}non{\isacharunderscore}{\kern0pt}electing\ m{\isachardoublequoteclose}\ \isakeyword{and}\isanewline
\ \ \ \ f{\isacharunderscore}{\kern0pt}prof{\isacharcolon}{\kern0pt}\ {\isachardoublequoteopen}finite{\isacharunderscore}{\kern0pt}profile\ A\ p{\isachardoublequoteclose}\isanewline
\ \ \isakeyword{shows}\ {\isachardoublequoteopen}card\ {\isacharparenleft}{\kern0pt}defer\ m\ A\ p{\isacharparenright}{\kern0pt}\ {\isacharequal}{\kern0pt}\ card\ A\ {\isacharminus}{\kern0pt}\ {\isadigit{1}}{\isachardoublequoteclose}\isanewline
%
\isadelimproof
\ \ %
\endisadelimproof
%
\isatagproof
\isacommand{using}\isamarkupfalse%
\ Diff{\isacharunderscore}{\kern0pt}empty\ One{\isacharunderscore}{\kern0pt}nat{\isacharunderscore}{\kern0pt}def\ Suc{\isacharunderscore}{\kern0pt}leI\ card{\isacharunderscore}{\kern0pt}Diff{\isacharunderscore}{\kern0pt}subset\ card{\isacharunderscore}{\kern0pt}gt{\isacharunderscore}{\kern0pt}{\isadigit{0}}{\isacharunderscore}{\kern0pt}iff\isanewline
\ \ \ \ \ \ \ \ defer{\isacharunderscore}{\kern0pt}not{\isacharunderscore}{\kern0pt}elec{\isacharunderscore}{\kern0pt}or{\isacharunderscore}{\kern0pt}rej\ finite{\isacharunderscore}{\kern0pt}subset\ non{\isacharunderscore}{\kern0pt}electing\isanewline
\ \ \ \ \ \ \ \ non{\isacharunderscore}{\kern0pt}electing{\isacharunderscore}{\kern0pt}def\ not{\isacharunderscore}{\kern0pt}empty\ f{\isacharunderscore}{\kern0pt}prof\ reject{\isacharunderscore}{\kern0pt}in{\isacharunderscore}{\kern0pt}alts\ rejecting\isanewline
\ \ \ \ \ \ \ \ rejects{\isacharunderscore}{\kern0pt}def\isanewline
\ \ \isacommand{by}\isamarkupfalse%
\ {\isacharparenleft}{\kern0pt}smt\ {\isacharparenleft}{\kern0pt}verit{\isacharcomma}{\kern0pt}\ ccfv{\isacharunderscore}{\kern0pt}threshold{\isacharparenright}{\kern0pt}{\isacharparenright}{\kern0pt}%
\endisatagproof
{\isafoldproof}%
%
\isadelimproof
\isanewline
%
\endisadelimproof
\isanewline
\isacommand{lemma}\isamarkupfalse%
\ single{\isacharunderscore}{\kern0pt}elim{\isacharunderscore}{\kern0pt}decr{\isacharunderscore}{\kern0pt}def{\isacharunderscore}{\kern0pt}card{\isadigit{2}}{\isacharcolon}{\kern0pt}\isanewline
\ \ \isakeyword{assumes}\isanewline
\ \ \ \ eliminating{\isacharcolon}{\kern0pt}\ {\isachardoublequoteopen}eliminates\ {\isadigit{1}}\ m{\isachardoublequoteclose}\ \isakeyword{and}\isanewline
\ \ \ \ not{\isacharunderscore}{\kern0pt}empty{\isacharcolon}{\kern0pt}\ {\isachardoublequoteopen}card\ A\ {\isachargreater}{\kern0pt}\ {\isadigit{1}}{\isachardoublequoteclose}\ \isakeyword{and}\isanewline
\ \ \ \ non{\isacharunderscore}{\kern0pt}electing{\isacharcolon}{\kern0pt}\ {\isachardoublequoteopen}non{\isacharunderscore}{\kern0pt}electing\ m{\isachardoublequoteclose}\ \isakeyword{and}\isanewline
\ \ \ \ f{\isacharunderscore}{\kern0pt}prof{\isacharcolon}{\kern0pt}\ {\isachardoublequoteopen}finite{\isacharunderscore}{\kern0pt}profile\ A\ p{\isachardoublequoteclose}\isanewline
\ \ \isakeyword{shows}\ {\isachardoublequoteopen}card\ {\isacharparenleft}{\kern0pt}defer\ m\ A\ p{\isacharparenright}{\kern0pt}\ {\isacharequal}{\kern0pt}\ card\ A\ {\isacharminus}{\kern0pt}\ {\isadigit{1}}{\isachardoublequoteclose}\isanewline
%
\isadelimproof
\ \ %
\endisadelimproof
%
\isatagproof
\isacommand{using}\isamarkupfalse%
\ Diff{\isacharunderscore}{\kern0pt}empty\ One{\isacharunderscore}{\kern0pt}nat{\isacharunderscore}{\kern0pt}def\ Suc{\isacharunderscore}{\kern0pt}leI\ card{\isacharunderscore}{\kern0pt}Diff{\isacharunderscore}{\kern0pt}subset\ card{\isacharunderscore}{\kern0pt}gt{\isacharunderscore}{\kern0pt}{\isadigit{0}}{\isacharunderscore}{\kern0pt}iff\isanewline
\ \ \ \ \ \ \ \ defer{\isacharunderscore}{\kern0pt}not{\isacharunderscore}{\kern0pt}elec{\isacharunderscore}{\kern0pt}or{\isacharunderscore}{\kern0pt}rej\ finite{\isacharunderscore}{\kern0pt}subset\ non{\isacharunderscore}{\kern0pt}electing\isanewline
\ \ \ \ \ \ \ \ non{\isacharunderscore}{\kern0pt}electing{\isacharunderscore}{\kern0pt}def\ not{\isacharunderscore}{\kern0pt}empty\ f{\isacharunderscore}{\kern0pt}prof\ reject{\isacharunderscore}{\kern0pt}in{\isacharunderscore}{\kern0pt}alts\isanewline
\ \ \ \ \ \ \ \ eliminating\ eliminates{\isacharunderscore}{\kern0pt}def\isanewline
\ \ \isacommand{by}\isamarkupfalse%
\ {\isacharparenleft}{\kern0pt}smt\ {\isacharparenleft}{\kern0pt}verit{\isacharparenright}{\kern0pt}{\isacharparenright}{\kern0pt}%
\endisatagproof
{\isafoldproof}%
%
\isadelimproof
\isanewline
%
\endisadelimproof
\isanewline
\isanewline
\isacommand{definition}\isamarkupfalse%
\ defer{\isacharunderscore}{\kern0pt}deciding\ {\isacharcolon}{\kern0pt}{\isacharcolon}{\kern0pt}\ {\isachardoublequoteopen}{\isacharprime}{\kern0pt}a\ Electoral{\isacharunderscore}{\kern0pt}Module\ {\isasymRightarrow}\ bool{\isachardoublequoteclose}\ \isakeyword{where}\isanewline
\ \ {\isachardoublequoteopen}defer{\isacharunderscore}{\kern0pt}deciding\ m\ {\isasymequiv}\isanewline
\ \ \ \ electoral{\isacharunderscore}{\kern0pt}module\ m\ {\isasymand}\ non{\isacharunderscore}{\kern0pt}electing\ m\ {\isasymand}\ defers\ {\isadigit{1}}\ m{\isachardoublequoteclose}\isanewline
%
\isadelimtheory
\isanewline
%
\endisadelimtheory
%
\isatagtheory
\isacommand{end}\isamarkupfalse%
%
\endisatagtheory
{\isafoldtheory}%
%
\isadelimtheory
%
\endisadelimtheory
%
\end{isabellebody}%
\endinput
%:%file=~/Documents/Studies/VotingRuleGenerator/virage/src/test/resources/verifiedVotingRuleConstruction/theories/Compositional_Framework/Properties/Result_Properties.thy%:%
%:%10=1%:%
%:%11=1%:%
%:%12=2%:%
%:%13=3%:%
%:%14=4%:%
%:%19=4%:%
%:%22=5%:%
%:%23=9%:%
%:%24=10%:%
%:%25=10%:%
%:%26=11%:%
%:%28=13%:%
%:%29=14%:%
%:%30=15%:%
%:%31=15%:%
%:%32=16%:%
%:%33=17%:%
%:%34=18%:%
%:%35=19%:%
%:%36=20%:%
%:%39=21%:%
%:%43=21%:%
%:%44=21%:%
%:%45=22%:%
%:%46=23%:%
%:%47=24%:%
%:%48=24%:%
%:%53=24%:%
%:%56=25%:%
%:%57=26%:%
%:%58=30%:%
%:%59=31%:%
%:%60=31%:%
%:%61=32%:%
%:%62=33%:%
%:%63=34%:%
%:%64=38%:%
%:%65=39%:%
%:%66=39%:%
%:%67=40%:%
%:%70=43%:%
%:%71=44%:%
%:%72=48%:%
%:%73=49%:%
%:%74=49%:%
%:%75=50%:%
%:%78=53%:%
%:%79=54%:%
%:%80=55%:%
%:%81=59%:%
%:%82=60%:%
%:%83=60%:%
%:%84=61%:%
%:%86=63%:%
%:%87=64%:%
%:%88=68%:%
%:%89=69%:%
%:%90=69%:%
%:%91=70%:%
%:%94=73%:%
%:%95=74%:%
%:%96=78%:%
%:%97=79%:%
%:%98=79%:%
%:%99=80%:%
%:%101=82%:%
%:%102=83%:%
%:%103=87%:%
%:%104=88%:%
%:%105=88%:%
%:%106=89%:%
%:%108=91%:%
%:%109=92%:%
%:%110=93%:%
%:%111=93%:%
%:%112=94%:%
%:%113=95%:%
%:%114=96%:%
%:%115=97%:%
%:%116=98%:%
%:%119=99%:%
%:%123=99%:%
%:%124=99%:%
%:%125=100%:%
%:%126=101%:%
%:%127=102%:%
%:%128=102%:%
%:%133=102%:%
%:%136=103%:%
%:%137=104%:%
%:%138=104%:%
%:%139=105%:%
%:%140=106%:%
%:%141=107%:%
%:%142=108%:%
%:%143=109%:%
%:%144=110%:%
%:%147=111%:%
%:%151=111%:%
%:%152=111%:%
%:%153=112%:%
%:%154=113%:%
%:%155=114%:%
%:%156=115%:%
%:%157=115%:%
%:%162=115%:%
%:%165=116%:%
%:%166=117%:%
%:%167=117%:%
%:%168=118%:%
%:%169=119%:%
%:%170=120%:%
%:%171=121%:%
%:%172=122%:%
%:%173=123%:%
%:%176=124%:%
%:%180=124%:%
%:%181=124%:%
%:%182=125%:%
%:%183=126%:%
%:%184=127%:%
%:%185=128%:%
%:%186=128%:%
%:%191=128%:%
%:%194=129%:%
%:%195=136%:%
%:%196=137%:%
%:%197=137%:%
%:%198=138%:%
%:%199=139%:%
%:%202=140%:%
%:%207=141%:%
%
\begin{isabellebody}%
\setisabellecontext{Sequential{\isacharunderscore}{\kern0pt}Composition}%
%
\isadelimdocument
\isanewline
%
\endisadelimdocument
%
\isatagdocument
\isanewline
\isanewline
%
\isamarkupsection{Sequential Composition%
}
\isamarkuptrue%
%
\endisatagdocument
{\isafolddocument}%
%
\isadelimdocument
%
\endisadelimdocument
%
\isadelimtheory
%
\endisadelimtheory
%
\isatagtheory
\isacommand{theory}\isamarkupfalse%
\ Sequential{\isacharunderscore}{\kern0pt}Composition\isanewline
\ \ \isakeyword{imports}\ {\isachardoublequoteopen}{\isachardot}{\kern0pt}{\isachardot}{\kern0pt}{\isacharslash}{\kern0pt}Electoral{\isacharunderscore}{\kern0pt}Module{\isachardoublequoteclose}\isanewline
\ \ \ \ \ \ \ \ \ \ {\isachardoublequoteopen}{\isachardot}{\kern0pt}{\isachardot}{\kern0pt}{\isacharslash}{\kern0pt}{\isachardot}{\kern0pt}{\isachardot}{\kern0pt}{\isacharslash}{\kern0pt}Properties{\isacharslash}{\kern0pt}Result{\isacharunderscore}{\kern0pt}Properties{\isachardoublequoteclose}\isanewline
\isanewline
\isakeyword{begin}%
\endisatagtheory
{\isafoldtheory}%
%
\isadelimtheory
%
\endisadelimtheory
%
\begin{isamarkuptext}%
The sequential composition creates a new electoral module from
two electoral modules. In a sequential composition, the second
electoral module makes decisions over alternatives deferred by
the first electoral module.%
\end{isamarkuptext}\isamarkuptrue%
%
\isadelimdocument
%
\endisadelimdocument
%
\isatagdocument
%
\isamarkupsubsection{Definition%
}
\isamarkuptrue%
%
\endisatagdocument
{\isafolddocument}%
%
\isadelimdocument
%
\endisadelimdocument
\isacommand{fun}\isamarkupfalse%
\ sequential{\isacharunderscore}{\kern0pt}composition\ {\isacharcolon}{\kern0pt}{\isacharcolon}{\kern0pt}\ {\isachardoublequoteopen}{\isacharprime}{\kern0pt}a\ Electoral{\isacharunderscore}{\kern0pt}Module\ {\isasymRightarrow}\ {\isacharprime}{\kern0pt}a\ Electoral{\isacharunderscore}{\kern0pt}Module\ {\isasymRightarrow}\isanewline
\ \ \ \ \ \ \ \ {\isacharprime}{\kern0pt}a\ Electoral{\isacharunderscore}{\kern0pt}Module{\isachardoublequoteclose}\ \isakeyword{where}\isanewline
\ \ {\isachardoublequoteopen}sequential{\isacharunderscore}{\kern0pt}composition\ m\ n\ A\ p\ {\isacharequal}{\kern0pt}\isanewline
\ \ \ \ {\isacharparenleft}{\kern0pt}let\ new{\isacharunderscore}{\kern0pt}A\ {\isacharequal}{\kern0pt}\ defer\ m\ A\ p{\isacharsemicolon}{\kern0pt}\isanewline
\ \ \ \ \ \ \ \ new{\isacharunderscore}{\kern0pt}p\ {\isacharequal}{\kern0pt}\ limit{\isacharunderscore}{\kern0pt}profile\ new{\isacharunderscore}{\kern0pt}A\ p\ in\ {\isacharparenleft}{\kern0pt}\isanewline
\ \ \ \ \ \ \ \ \ \ \ \ \ \ \ \ \ \ {\isacharparenleft}{\kern0pt}elect\ m\ A\ p{\isacharparenright}{\kern0pt}\ {\isasymunion}\ {\isacharparenleft}{\kern0pt}elect\ n\ new{\isacharunderscore}{\kern0pt}A\ new{\isacharunderscore}{\kern0pt}p{\isacharparenright}{\kern0pt}{\isacharcomma}{\kern0pt}\isanewline
\ \ \ \ \ \ \ \ \ \ \ \ \ \ \ \ \ \ {\isacharparenleft}{\kern0pt}reject\ m\ A\ p{\isacharparenright}{\kern0pt}\ {\isasymunion}\ {\isacharparenleft}{\kern0pt}reject\ n\ new{\isacharunderscore}{\kern0pt}A\ new{\isacharunderscore}{\kern0pt}p{\isacharparenright}{\kern0pt}{\isacharcomma}{\kern0pt}\isanewline
\ \ \ \ \ \ \ \ \ \ \ \ \ \ \ \ \ \ defer\ n\ new{\isacharunderscore}{\kern0pt}A\ new{\isacharunderscore}{\kern0pt}p{\isacharparenright}{\kern0pt}{\isacharparenright}{\kern0pt}{\isachardoublequoteclose}\isanewline
\isanewline
\isacommand{abbreviation}\isamarkupfalse%
\ sequence\ {\isacharcolon}{\kern0pt}{\isacharcolon}{\kern0pt}\isanewline
\ \ {\isachardoublequoteopen}{\isacharprime}{\kern0pt}a\ Electoral{\isacharunderscore}{\kern0pt}Module\ {\isasymRightarrow}\ {\isacharprime}{\kern0pt}a\ Electoral{\isacharunderscore}{\kern0pt}Module\ {\isasymRightarrow}\ {\isacharprime}{\kern0pt}a\ Electoral{\isacharunderscore}{\kern0pt}Module{\isachardoublequoteclose}\isanewline
\ \ \ \ \ {\isacharparenleft}{\kern0pt}\isakeyword{infix}\ {\isachardoublequoteopen}{\isasymtriangleright}{\isachardoublequoteclose}\ {\isadigit{5}}{\isadigit{0}}{\isacharparenright}{\kern0pt}\ \isakeyword{where}\isanewline
\ \ {\isachardoublequoteopen}m\ {\isasymtriangleright}\ n\ {\isacharequal}{\kern0pt}{\isacharequal}{\kern0pt}\ sequential{\isacharunderscore}{\kern0pt}composition\ m\ n{\isachardoublequoteclose}\isanewline
\isanewline
\isacommand{lemma}\isamarkupfalse%
\ seq{\isacharunderscore}{\kern0pt}comp{\isacharunderscore}{\kern0pt}presv{\isacharunderscore}{\kern0pt}disj{\isacharcolon}{\kern0pt}\isanewline
\ \ \isakeyword{assumes}\ module{\isacharunderscore}{\kern0pt}m{\isacharcolon}{\kern0pt}\ {\isachardoublequoteopen}electoral{\isacharunderscore}{\kern0pt}module\ m{\isachardoublequoteclose}\ \isakeyword{and}\isanewline
\ \ \ \ \ \ \ \ \ \ module{\isacharunderscore}{\kern0pt}n{\isacharcolon}{\kern0pt}\ {\isachardoublequoteopen}electoral{\isacharunderscore}{\kern0pt}module\ n{\isachardoublequoteclose}\ \isakeyword{and}\isanewline
\ \ \ \ \ \ \ \ \ \ f{\isacharunderscore}{\kern0pt}prof{\isacharcolon}{\kern0pt}\ \ {\isachardoublequoteopen}finite{\isacharunderscore}{\kern0pt}profile\ A\ p{\isachardoublequoteclose}\isanewline
\ \ \isakeyword{shows}\ {\isachardoublequoteopen}disjoint{\isadigit{3}}\ {\isacharparenleft}{\kern0pt}{\isacharparenleft}{\kern0pt}m\ {\isasymtriangleright}\ n{\isacharparenright}{\kern0pt}\ A\ p{\isacharparenright}{\kern0pt}{\isachardoublequoteclose}\isanewline
%
\isadelimproof
%
\endisadelimproof
%
\isatagproof
\isacommand{proof}\isamarkupfalse%
\ {\isacharminus}{\kern0pt}\isanewline
\ \ \isacommand{let}\isamarkupfalse%
\ {\isacharquery}{\kern0pt}new{\isacharunderscore}{\kern0pt}A\ {\isacharequal}{\kern0pt}\ {\isachardoublequoteopen}defer\ m\ A\ p{\isachardoublequoteclose}\isanewline
\ \ \isacommand{let}\isamarkupfalse%
\ {\isacharquery}{\kern0pt}new{\isacharunderscore}{\kern0pt}p\ {\isacharequal}{\kern0pt}\ {\isachardoublequoteopen}limit{\isacharunderscore}{\kern0pt}profile\ {\isacharquery}{\kern0pt}new{\isacharunderscore}{\kern0pt}A\ p{\isachardoublequoteclose}\isanewline
\ \ \isacommand{from}\isamarkupfalse%
\ module{\isacharunderscore}{\kern0pt}m\ f{\isacharunderscore}{\kern0pt}prof\ \isacommand{have}\isamarkupfalse%
\ disjoint{\isacharunderscore}{\kern0pt}m{\isacharcolon}{\kern0pt}\ {\isachardoublequoteopen}disjoint{\isadigit{3}}\ {\isacharparenleft}{\kern0pt}m\ A\ p{\isacharparenright}{\kern0pt}{\isachardoublequoteclose}\isanewline
\ \ \ \ \isacommand{using}\isamarkupfalse%
\ electoral{\isacharunderscore}{\kern0pt}module{\isacharunderscore}{\kern0pt}def\ well{\isacharunderscore}{\kern0pt}formed{\isachardot}{\kern0pt}simps\isanewline
\ \ \ \ \isacommand{by}\isamarkupfalse%
\ blast\isanewline
\ \ \isacommand{from}\isamarkupfalse%
\ module{\isacharunderscore}{\kern0pt}m\ module{\isacharunderscore}{\kern0pt}n\ def{\isacharunderscore}{\kern0pt}presv{\isacharunderscore}{\kern0pt}fin{\isacharunderscore}{\kern0pt}prof\ f{\isacharunderscore}{\kern0pt}prof\ \isacommand{have}\isamarkupfalse%
\ disjoint{\isacharunderscore}{\kern0pt}n{\isacharcolon}{\kern0pt}\isanewline
\ \ \ \ {\isachardoublequoteopen}{\isacharparenleft}{\kern0pt}disjoint{\isadigit{3}}\ {\isacharparenleft}{\kern0pt}n\ {\isacharquery}{\kern0pt}new{\isacharunderscore}{\kern0pt}A\ {\isacharquery}{\kern0pt}new{\isacharunderscore}{\kern0pt}p{\isacharparenright}{\kern0pt}{\isacharparenright}{\kern0pt}{\isachardoublequoteclose}\isanewline
\ \ \ \ \isacommand{using}\isamarkupfalse%
\ electoral{\isacharunderscore}{\kern0pt}module{\isacharunderscore}{\kern0pt}def\ well{\isacharunderscore}{\kern0pt}formed{\isachardot}{\kern0pt}simps\isanewline
\ \ \ \ \isacommand{by}\isamarkupfalse%
\ metis\isanewline
\ \ \isacommand{with}\isamarkupfalse%
\ disjoint{\isacharunderscore}{\kern0pt}m\ module{\isacharunderscore}{\kern0pt}m\ module{\isacharunderscore}{\kern0pt}n\ f{\isacharunderscore}{\kern0pt}prof\ \isacommand{have}\isamarkupfalse%
\ {\isadigit{0}}{\isacharcolon}{\kern0pt}\isanewline
\ \ \ \ {\isachardoublequoteopen}{\isacharparenleft}{\kern0pt}elect\ m\ A\ p\ {\isasyminter}\ reject\ n\ {\isacharquery}{\kern0pt}new{\isacharunderscore}{\kern0pt}A\ {\isacharquery}{\kern0pt}new{\isacharunderscore}{\kern0pt}p{\isacharparenright}{\kern0pt}\ {\isacharequal}{\kern0pt}\ {\isacharbraceleft}{\kern0pt}{\isacharbraceright}{\kern0pt}{\isachardoublequoteclose}\isanewline
\ \ \ \ \isacommand{using}\isamarkupfalse%
\ disjoint{\isacharunderscore}{\kern0pt}iff{\isacharunderscore}{\kern0pt}not{\isacharunderscore}{\kern0pt}equal\ reject{\isacharunderscore}{\kern0pt}in{\isacharunderscore}{\kern0pt}alts\isanewline
\ \ \ \ \ \ \ \ \ \ def{\isacharunderscore}{\kern0pt}presv{\isacharunderscore}{\kern0pt}fin{\isacharunderscore}{\kern0pt}prof\ result{\isacharunderscore}{\kern0pt}disj\ subset{\isacharunderscore}{\kern0pt}eq\isanewline
\ \ \ \ \isacommand{by}\isamarkupfalse%
\ {\isacharparenleft}{\kern0pt}smt\ {\isacharparenleft}{\kern0pt}verit{\isacharcomma}{\kern0pt}\ best{\isacharparenright}{\kern0pt}{\isacharparenright}{\kern0pt}\isanewline
\ \ \isacommand{from}\isamarkupfalse%
\ disjoint{\isacharunderscore}{\kern0pt}m\ disjoint{\isacharunderscore}{\kern0pt}n\ def{\isacharunderscore}{\kern0pt}presv{\isacharunderscore}{\kern0pt}fin{\isacharunderscore}{\kern0pt}prof\ f{\isacharunderscore}{\kern0pt}prof\isanewline
\ \ \ \ \ \ \ module{\isacharunderscore}{\kern0pt}m\ module{\isacharunderscore}{\kern0pt}n\ \isacommand{have}\isamarkupfalse%
\ {\isadigit{1}}{\isacharcolon}{\kern0pt}\isanewline
\ \ \ \ {\isachardoublequoteopen}{\isacharparenleft}{\kern0pt}elect\ m\ A\ p\ {\isasyminter}\ defer\ n\ {\isacharquery}{\kern0pt}new{\isacharunderscore}{\kern0pt}A\ {\isacharquery}{\kern0pt}new{\isacharunderscore}{\kern0pt}p{\isacharparenright}{\kern0pt}\ {\isacharequal}{\kern0pt}\ {\isacharbraceleft}{\kern0pt}{\isacharbraceright}{\kern0pt}{\isachardoublequoteclose}\isanewline
\ \ \ \ \isacommand{using}\isamarkupfalse%
\ defer{\isacharunderscore}{\kern0pt}in{\isacharunderscore}{\kern0pt}alts\ disjoint{\isacharunderscore}{\kern0pt}iff{\isacharunderscore}{\kern0pt}not{\isacharunderscore}{\kern0pt}equal\isanewline
\ \ \ \ \ \ \ \ \ \ rev{\isacharunderscore}{\kern0pt}subsetD\ result{\isacharunderscore}{\kern0pt}disj\ distrib{\isacharunderscore}{\kern0pt}imp{\isadigit{2}}\isanewline
\ \ \ \ \ \ \ \ \ \ Int{\isacharunderscore}{\kern0pt}Un{\isacharunderscore}{\kern0pt}distrib\ inf{\isacharunderscore}{\kern0pt}sup{\isacharunderscore}{\kern0pt}distrib{\isadigit{1}}\isanewline
\ \ \ \ \ \ \ \ \ \ result{\isacharunderscore}{\kern0pt}presv{\isacharunderscore}{\kern0pt}alts\ sup{\isacharunderscore}{\kern0pt}bot{\isachardot}{\kern0pt}left{\isacharunderscore}{\kern0pt}neutral\isanewline
\ \ \ \ \ \ \ \ \ \ sup{\isacharunderscore}{\kern0pt}bot{\isachardot}{\kern0pt}neutr{\isacharunderscore}{\kern0pt}eq{\isacharunderscore}{\kern0pt}iff\ sup{\isacharunderscore}{\kern0pt}bot{\isacharunderscore}{\kern0pt}right\ {\isachardoublequoteopen}{\isadigit{0}}{\isachardoublequoteclose}\isanewline
\ \ \ \ \isacommand{by}\isamarkupfalse%
\ {\isacharparenleft}{\kern0pt}smt\ {\isacharparenleft}{\kern0pt}verit{\isacharcomma}{\kern0pt}\ del{\isacharunderscore}{\kern0pt}insts{\isacharparenright}{\kern0pt}{\isacharparenright}{\kern0pt}\isanewline
\ \ \isacommand{from}\isamarkupfalse%
\ disjoint{\isacharunderscore}{\kern0pt}m\ disjoint{\isacharunderscore}{\kern0pt}n\ def{\isacharunderscore}{\kern0pt}presv{\isacharunderscore}{\kern0pt}fin{\isacharunderscore}{\kern0pt}prof\ f{\isacharunderscore}{\kern0pt}prof\isanewline
\ \ \ \ \ \ \ module{\isacharunderscore}{\kern0pt}m\ module{\isacharunderscore}{\kern0pt}n\ \isacommand{have}\isamarkupfalse%
\ {\isadigit{2}}{\isacharcolon}{\kern0pt}\isanewline
\ \ \ \ {\isachardoublequoteopen}{\isacharparenleft}{\kern0pt}reject\ m\ A\ p\ {\isasyminter}\ reject\ n\ {\isacharquery}{\kern0pt}new{\isacharunderscore}{\kern0pt}A\ {\isacharquery}{\kern0pt}new{\isacharunderscore}{\kern0pt}p{\isacharparenright}{\kern0pt}\ {\isacharequal}{\kern0pt}\ {\isacharbraceleft}{\kern0pt}{\isacharbraceright}{\kern0pt}{\isachardoublequoteclose}\isanewline
\ \ \ \ \isacommand{using}\isamarkupfalse%
\ disjoint{\isacharunderscore}{\kern0pt}iff{\isacharunderscore}{\kern0pt}not{\isacharunderscore}{\kern0pt}equal\ reject{\isacharunderscore}{\kern0pt}in{\isacharunderscore}{\kern0pt}alts\isanewline
\ \ \ \ \ \ \ \ \ \ set{\isacharunderscore}{\kern0pt}rev{\isacharunderscore}{\kern0pt}mp\ result{\isacharunderscore}{\kern0pt}disj\ Int{\isacharunderscore}{\kern0pt}Un{\isacharunderscore}{\kern0pt}distrib{\isadigit{2}}\isanewline
\ \ \ \ \ \ \ \ \ \ Un{\isacharunderscore}{\kern0pt}Diff{\isacharunderscore}{\kern0pt}Int\ boolean{\isacharunderscore}{\kern0pt}algebra{\isacharunderscore}{\kern0pt}cancel{\isachardot}{\kern0pt}inf{\isadigit{2}}\isanewline
\ \ \ \ \ \ \ \ \ \ inf{\isachardot}{\kern0pt}order{\isacharunderscore}{\kern0pt}iff\ inf{\isacharunderscore}{\kern0pt}sup{\isacharunderscore}{\kern0pt}aci{\isacharparenleft}{\kern0pt}{\isadigit{1}}{\isacharparenright}{\kern0pt}\ subsetD\isanewline
\ \ \ \ \isacommand{by}\isamarkupfalse%
\ {\isacharparenleft}{\kern0pt}smt\ {\isacharparenleft}{\kern0pt}verit{\isacharcomma}{\kern0pt}\ ccfv{\isacharunderscore}{\kern0pt}threshold{\isacharparenright}{\kern0pt}{\isacharparenright}{\kern0pt}\isanewline
\ \ \isacommand{from}\isamarkupfalse%
\ disjoint{\isacharunderscore}{\kern0pt}m\ disjoint{\isacharunderscore}{\kern0pt}n\ def{\isacharunderscore}{\kern0pt}presv{\isacharunderscore}{\kern0pt}fin{\isacharunderscore}{\kern0pt}prof\ f{\isacharunderscore}{\kern0pt}prof\isanewline
\ \ \ \ \ \ \ module{\isacharunderscore}{\kern0pt}m\ module{\isacharunderscore}{\kern0pt}n\ \isacommand{have}\isamarkupfalse%
\ {\isadigit{3}}{\isacharcolon}{\kern0pt}\isanewline
\ \ \ \ {\isachardoublequoteopen}{\isacharparenleft}{\kern0pt}reject\ m\ A\ p\ {\isasyminter}\ elect\ n\ {\isacharquery}{\kern0pt}new{\isacharunderscore}{\kern0pt}A\ {\isacharquery}{\kern0pt}new{\isacharunderscore}{\kern0pt}p{\isacharparenright}{\kern0pt}\ {\isacharequal}{\kern0pt}\ {\isacharbraceleft}{\kern0pt}{\isacharbraceright}{\kern0pt}{\isachardoublequoteclose}\isanewline
\ \ \ \ \isacommand{using}\isamarkupfalse%
\ disjoint{\isacharunderscore}{\kern0pt}iff{\isacharunderscore}{\kern0pt}not{\isacharunderscore}{\kern0pt}equal\ elect{\isacharunderscore}{\kern0pt}in{\isacharunderscore}{\kern0pt}alts\ set{\isacharunderscore}{\kern0pt}rev{\isacharunderscore}{\kern0pt}mp\isanewline
\ \ \ \ \ \ \ \ \ \ result{\isacharunderscore}{\kern0pt}disj\ Int{\isacharunderscore}{\kern0pt}commute\ boolean{\isacharunderscore}{\kern0pt}algebra{\isacharunderscore}{\kern0pt}cancel{\isachardot}{\kern0pt}inf{\isadigit{2}}\isanewline
\ \ \ \ \ \ \ \ \ \ defer{\isacharunderscore}{\kern0pt}not{\isacharunderscore}{\kern0pt}elec{\isacharunderscore}{\kern0pt}or{\isacharunderscore}{\kern0pt}rej\ inf{\isachardot}{\kern0pt}commute\ inf{\isachardot}{\kern0pt}orderE\ inf{\isacharunderscore}{\kern0pt}commute\isanewline
\ \ \ \ \isacommand{by}\isamarkupfalse%
\ {\isacharparenleft}{\kern0pt}smt\ {\isacharparenleft}{\kern0pt}verit{\isacharcomma}{\kern0pt}\ ccfv{\isacharunderscore}{\kern0pt}threshold{\isacharparenright}{\kern0pt}{\isacharparenright}{\kern0pt}\isanewline
\ \ \isacommand{from}\isamarkupfalse%
\ {\isadigit{0}}\ {\isadigit{1}}\ {\isadigit{2}}\ {\isadigit{3}}\ disjoint{\isacharunderscore}{\kern0pt}m\ disjoint{\isacharunderscore}{\kern0pt}n\ module{\isacharunderscore}{\kern0pt}m\ module{\isacharunderscore}{\kern0pt}n\ f{\isacharunderscore}{\kern0pt}prof\ \isacommand{have}\isamarkupfalse%
\isanewline
\ \ \ \ {\isachardoublequoteopen}{\isacharparenleft}{\kern0pt}elect\ m\ A\ p\ {\isasymunion}\ elect\ n\ {\isacharquery}{\kern0pt}new{\isacharunderscore}{\kern0pt}A\ {\isacharquery}{\kern0pt}new{\isacharunderscore}{\kern0pt}p{\isacharparenright}{\kern0pt}\ {\isasyminter}\isanewline
\ \ \ \ \ \ \ \ \ \ {\isacharparenleft}{\kern0pt}reject\ m\ A\ p\ {\isasymunion}\ reject\ n\ {\isacharquery}{\kern0pt}new{\isacharunderscore}{\kern0pt}A\ {\isacharquery}{\kern0pt}new{\isacharunderscore}{\kern0pt}p{\isacharparenright}{\kern0pt}\ {\isacharequal}{\kern0pt}\ {\isacharbraceleft}{\kern0pt}{\isacharbraceright}{\kern0pt}{\isachardoublequoteclose}\isanewline
\ \ \ \ \isacommand{using}\isamarkupfalse%
\ inf{\isacharunderscore}{\kern0pt}sup{\isacharunderscore}{\kern0pt}aci{\isacharparenleft}{\kern0pt}{\isadigit{1}}{\isacharparenright}{\kern0pt}\ inf{\isacharunderscore}{\kern0pt}sup{\isacharunderscore}{\kern0pt}distrib{\isadigit{2}}\ def{\isacharunderscore}{\kern0pt}presv{\isacharunderscore}{\kern0pt}fin{\isacharunderscore}{\kern0pt}prof\isanewline
\ \ \ \ \ \ \ \ \ \ result{\isacharunderscore}{\kern0pt}disj\ sup{\isacharunderscore}{\kern0pt}inf{\isacharunderscore}{\kern0pt}absorb\ sup{\isacharunderscore}{\kern0pt}inf{\isacharunderscore}{\kern0pt}distrib{\isadigit{1}}\isanewline
\ \ \ \ \ \ \ \ \ \ distrib{\isacharparenleft}{\kern0pt}{\isadigit{3}}{\isacharparenright}{\kern0pt}\ sup{\isacharunderscore}{\kern0pt}eq{\isacharunderscore}{\kern0pt}bot{\isacharunderscore}{\kern0pt}iff\isanewline
\ \ \ \ \isacommand{by}\isamarkupfalse%
\ {\isacharparenleft}{\kern0pt}smt\ {\isacharparenleft}{\kern0pt}verit{\isacharcomma}{\kern0pt}\ ccfv{\isacharunderscore}{\kern0pt}threshold{\isacharparenright}{\kern0pt}{\isacharparenright}{\kern0pt}\isanewline
\ \ \isacommand{moreover}\isamarkupfalse%
\ \isacommand{from}\isamarkupfalse%
\ {\isadigit{0}}\ {\isadigit{1}}\ {\isadigit{2}}\ {\isadigit{3}}\ disjoint{\isacharunderscore}{\kern0pt}n\ module{\isacharunderscore}{\kern0pt}m\ module{\isacharunderscore}{\kern0pt}n\ f{\isacharunderscore}{\kern0pt}prof\ \isacommand{have}\isamarkupfalse%
\isanewline
\ \ \ \ {\isachardoublequoteopen}{\isacharparenleft}{\kern0pt}elect\ m\ A\ p\ {\isasymunion}\ elect\ n\ {\isacharquery}{\kern0pt}new{\isacharunderscore}{\kern0pt}A\ {\isacharquery}{\kern0pt}new{\isacharunderscore}{\kern0pt}p{\isacharparenright}{\kern0pt}\ {\isasyminter}\isanewline
\ \ \ \ \ \ \ \ \ \ {\isacharparenleft}{\kern0pt}defer\ n\ {\isacharquery}{\kern0pt}new{\isacharunderscore}{\kern0pt}A\ {\isacharquery}{\kern0pt}new{\isacharunderscore}{\kern0pt}p{\isacharparenright}{\kern0pt}\ {\isacharequal}{\kern0pt}\ {\isacharbraceleft}{\kern0pt}{\isacharbraceright}{\kern0pt}{\isachardoublequoteclose}\isanewline
\ \ \ \ \isacommand{using}\isamarkupfalse%
\ Int{\isacharunderscore}{\kern0pt}Un{\isacharunderscore}{\kern0pt}distrib{\isadigit{2}}\ Un{\isacharunderscore}{\kern0pt}empty\ def{\isacharunderscore}{\kern0pt}presv{\isacharunderscore}{\kern0pt}fin{\isacharunderscore}{\kern0pt}prof\ result{\isacharunderscore}{\kern0pt}disj\isanewline
\ \ \ \ \isacommand{by}\isamarkupfalse%
\ metis\isanewline
\ \ \isacommand{moreover}\isamarkupfalse%
\ \isacommand{from}\isamarkupfalse%
\ {\isadigit{0}}\ {\isadigit{1}}\ {\isadigit{2}}\ {\isadigit{3}}\ f{\isacharunderscore}{\kern0pt}prof\ disjoint{\isacharunderscore}{\kern0pt}m\ disjoint{\isacharunderscore}{\kern0pt}n\ module{\isacharunderscore}{\kern0pt}m\ module{\isacharunderscore}{\kern0pt}n\isanewline
\ \ \isacommand{have}\isamarkupfalse%
\isanewline
\ \ \ \ {\isachardoublequoteopen}{\isacharparenleft}{\kern0pt}reject\ m\ A\ p\ {\isasymunion}\ reject\ n\ {\isacharquery}{\kern0pt}new{\isacharunderscore}{\kern0pt}A\ {\isacharquery}{\kern0pt}new{\isacharunderscore}{\kern0pt}p{\isacharparenright}{\kern0pt}\ {\isasyminter}\isanewline
\ \ \ \ \ \ \ \ \ \ {\isacharparenleft}{\kern0pt}defer\ n\ {\isacharquery}{\kern0pt}new{\isacharunderscore}{\kern0pt}A\ {\isacharquery}{\kern0pt}new{\isacharunderscore}{\kern0pt}p{\isacharparenright}{\kern0pt}\ {\isacharequal}{\kern0pt}\ {\isacharbraceleft}{\kern0pt}{\isacharbraceright}{\kern0pt}{\isachardoublequoteclose}\isanewline
\ \ \ \ \isacommand{using}\isamarkupfalse%
\ Int{\isacharunderscore}{\kern0pt}Un{\isacharunderscore}{\kern0pt}distrib{\isadigit{2}}\ defer{\isacharunderscore}{\kern0pt}in{\isacharunderscore}{\kern0pt}alts\ distrib{\isacharunderscore}{\kern0pt}imp{\isadigit{2}}\isanewline
\ \ \ \ \ \ \ \ \ \ def{\isacharunderscore}{\kern0pt}presv{\isacharunderscore}{\kern0pt}fin{\isacharunderscore}{\kern0pt}prof\ result{\isacharunderscore}{\kern0pt}disj\ subset{\isacharunderscore}{\kern0pt}Un{\isacharunderscore}{\kern0pt}eq\isanewline
\ \ \ \ \ \ \ \ \ \ sup{\isacharunderscore}{\kern0pt}inf{\isacharunderscore}{\kern0pt}distrib{\isadigit{1}}\isanewline
\ \ \ \ \isacommand{by}\isamarkupfalse%
\ {\isacharparenleft}{\kern0pt}smt\ {\isacharparenleft}{\kern0pt}verit{\isacharparenright}{\kern0pt}{\isacharparenright}{\kern0pt}\isanewline
\ \ \isacommand{ultimately}\isamarkupfalse%
\ \isacommand{have}\isamarkupfalse%
\isanewline
\ \ \ \ {\isachardoublequoteopen}disjoint{\isadigit{3}}\ {\isacharparenleft}{\kern0pt}elect\ m\ A\ p\ {\isasymunion}\ elect\ n\ {\isacharquery}{\kern0pt}new{\isacharunderscore}{\kern0pt}A\ {\isacharquery}{\kern0pt}new{\isacharunderscore}{\kern0pt}p{\isacharcomma}{\kern0pt}\isanewline
\ \ \ \ \ \ \ \ \ \ \ \ \ \ \ \ reject\ m\ A\ p\ {\isasymunion}\ reject\ n\ {\isacharquery}{\kern0pt}new{\isacharunderscore}{\kern0pt}A\ {\isacharquery}{\kern0pt}new{\isacharunderscore}{\kern0pt}p{\isacharcomma}{\kern0pt}\isanewline
\ \ \ \ \ \ \ \ \ \ \ \ \ \ \ \ defer\ n\ {\isacharquery}{\kern0pt}new{\isacharunderscore}{\kern0pt}A\ {\isacharquery}{\kern0pt}new{\isacharunderscore}{\kern0pt}p{\isacharparenright}{\kern0pt}{\isachardoublequoteclose}\isanewline
\ \ \ \ \isacommand{by}\isamarkupfalse%
\ simp\isanewline
\ \ \isacommand{thus}\isamarkupfalse%
\ {\isacharquery}{\kern0pt}thesis\isanewline
\ \ \ \ \isacommand{using}\isamarkupfalse%
\ sequential{\isacharunderscore}{\kern0pt}composition{\isachardot}{\kern0pt}simps\isanewline
\ \ \ \ \isacommand{by}\isamarkupfalse%
\ metis\isanewline
\isacommand{qed}\isamarkupfalse%
%
\endisatagproof
{\isafoldproof}%
%
\isadelimproof
\isanewline
%
\endisadelimproof
\isanewline
\isacommand{lemma}\isamarkupfalse%
\ seq{\isacharunderscore}{\kern0pt}comp{\isacharunderscore}{\kern0pt}presv{\isacharunderscore}{\kern0pt}alts{\isacharcolon}{\kern0pt}\isanewline
\ \ \isakeyword{assumes}\ module{\isacharunderscore}{\kern0pt}m{\isacharcolon}{\kern0pt}\ {\isachardoublequoteopen}electoral{\isacharunderscore}{\kern0pt}module\ m{\isachardoublequoteclose}\ \isakeyword{and}\isanewline
\ \ \ \ \ \ \ \ \ \ module{\isacharunderscore}{\kern0pt}n{\isacharcolon}{\kern0pt}\ {\isachardoublequoteopen}electoral{\isacharunderscore}{\kern0pt}module\ n{\isachardoublequoteclose}\ \isakeyword{and}\isanewline
\ \ \ \ \ \ \ \ \ \ f{\isacharunderscore}{\kern0pt}prof{\isacharcolon}{\kern0pt}\ \ {\isachardoublequoteopen}finite{\isacharunderscore}{\kern0pt}profile\ A\ p{\isachardoublequoteclose}\isanewline
\ \ \isakeyword{shows}\ {\isachardoublequoteopen}set{\isacharunderscore}{\kern0pt}equals{\isacharunderscore}{\kern0pt}partition\ A\ {\isacharparenleft}{\kern0pt}{\isacharparenleft}{\kern0pt}m\ {\isasymtriangleright}\ n{\isacharparenright}{\kern0pt}\ A\ p{\isacharparenright}{\kern0pt}{\isachardoublequoteclose}\isanewline
%
\isadelimproof
%
\endisadelimproof
%
\isatagproof
\isacommand{proof}\isamarkupfalse%
\ {\isacharminus}{\kern0pt}\isanewline
\ \ \isacommand{let}\isamarkupfalse%
\ {\isacharquery}{\kern0pt}new{\isacharunderscore}{\kern0pt}A\ {\isacharequal}{\kern0pt}\ {\isachardoublequoteopen}defer\ m\ A\ p{\isachardoublequoteclose}\isanewline
\ \ \isacommand{let}\isamarkupfalse%
\ {\isacharquery}{\kern0pt}new{\isacharunderscore}{\kern0pt}p\ {\isacharequal}{\kern0pt}\ {\isachardoublequoteopen}limit{\isacharunderscore}{\kern0pt}profile\ {\isacharquery}{\kern0pt}new{\isacharunderscore}{\kern0pt}A\ p{\isachardoublequoteclose}\isanewline
\ \ \isacommand{from}\isamarkupfalse%
\ module{\isacharunderscore}{\kern0pt}m\ f{\isacharunderscore}{\kern0pt}prof\ \isacommand{have}\isamarkupfalse%
\ {\isachardoublequoteopen}set{\isacharunderscore}{\kern0pt}equals{\isacharunderscore}{\kern0pt}partition\ A\ {\isacharparenleft}{\kern0pt}m\ A\ p{\isacharparenright}{\kern0pt}{\isachardoublequoteclose}\isanewline
\ \ \ \ \isacommand{by}\isamarkupfalse%
\ {\isacharparenleft}{\kern0pt}simp\ add{\isacharcolon}{\kern0pt}\ electoral{\isacharunderscore}{\kern0pt}module{\isacharunderscore}{\kern0pt}def{\isacharparenright}{\kern0pt}\isanewline
\ \ \isacommand{with}\isamarkupfalse%
\ module{\isacharunderscore}{\kern0pt}m\ f{\isacharunderscore}{\kern0pt}prof\ \isacommand{have}\isamarkupfalse%
\ {\isadigit{0}}{\isacharcolon}{\kern0pt}\isanewline
\ \ \ \ {\isachardoublequoteopen}elect\ m\ A\ p\ {\isasymunion}\ reject\ m\ A\ p\ {\isasymunion}\ {\isacharquery}{\kern0pt}new{\isacharunderscore}{\kern0pt}A\ {\isacharequal}{\kern0pt}\ A{\isachardoublequoteclose}\isanewline
\ \ \ \ \isacommand{by}\isamarkupfalse%
\ {\isacharparenleft}{\kern0pt}simp\ add{\isacharcolon}{\kern0pt}\ result{\isacharunderscore}{\kern0pt}presv{\isacharunderscore}{\kern0pt}alts{\isacharparenright}{\kern0pt}\isanewline
\ \ \isacommand{from}\isamarkupfalse%
\ module{\isacharunderscore}{\kern0pt}n\ def{\isacharunderscore}{\kern0pt}presv{\isacharunderscore}{\kern0pt}fin{\isacharunderscore}{\kern0pt}prof\ f{\isacharunderscore}{\kern0pt}prof\ module{\isacharunderscore}{\kern0pt}m\ \isacommand{have}\isamarkupfalse%
\isanewline
\ \ \ \ {\isachardoublequoteopen}set{\isacharunderscore}{\kern0pt}equals{\isacharunderscore}{\kern0pt}partition\ {\isacharquery}{\kern0pt}new{\isacharunderscore}{\kern0pt}A\ {\isacharparenleft}{\kern0pt}n\ {\isacharquery}{\kern0pt}new{\isacharunderscore}{\kern0pt}A\ {\isacharquery}{\kern0pt}new{\isacharunderscore}{\kern0pt}p{\isacharparenright}{\kern0pt}{\isachardoublequoteclose}\isanewline
\ \ \ \ \isacommand{using}\isamarkupfalse%
\ electoral{\isacharunderscore}{\kern0pt}module{\isacharunderscore}{\kern0pt}def\ well{\isacharunderscore}{\kern0pt}formed{\isachardot}{\kern0pt}simps\isanewline
\ \ \ \ \isacommand{by}\isamarkupfalse%
\ metis\isanewline
\ \ \isacommand{with}\isamarkupfalse%
\ module{\isacharunderscore}{\kern0pt}m\ module{\isacharunderscore}{\kern0pt}n\ f{\isacharunderscore}{\kern0pt}prof\ \isacommand{have}\isamarkupfalse%
\ {\isadigit{1}}{\isacharcolon}{\kern0pt}\isanewline
\ \ \ \ {\isachardoublequoteopen}elect\ n\ {\isacharquery}{\kern0pt}new{\isacharunderscore}{\kern0pt}A\ {\isacharquery}{\kern0pt}new{\isacharunderscore}{\kern0pt}p\ {\isasymunion}\isanewline
\ \ \ \ \ \ \ \ reject\ n\ {\isacharquery}{\kern0pt}new{\isacharunderscore}{\kern0pt}A\ {\isacharquery}{\kern0pt}new{\isacharunderscore}{\kern0pt}p\ {\isasymunion}\isanewline
\ \ \ \ \ \ \ \ defer\ n\ {\isacharquery}{\kern0pt}new{\isacharunderscore}{\kern0pt}A\ {\isacharquery}{\kern0pt}new{\isacharunderscore}{\kern0pt}p\ {\isacharequal}{\kern0pt}\ {\isacharquery}{\kern0pt}new{\isacharunderscore}{\kern0pt}A{\isachardoublequoteclose}\isanewline
\ \ \ \ \isacommand{using}\isamarkupfalse%
\ def{\isacharunderscore}{\kern0pt}presv{\isacharunderscore}{\kern0pt}fin{\isacharunderscore}{\kern0pt}prof\ result{\isacharunderscore}{\kern0pt}presv{\isacharunderscore}{\kern0pt}alts\isanewline
\ \ \ \ \isacommand{by}\isamarkupfalse%
\ metis\isanewline
\ \ \isacommand{from}\isamarkupfalse%
\ {\isadigit{0}}\ {\isadigit{1}}\ \isacommand{have}\isamarkupfalse%
\isanewline
\ \ \ \ {\isachardoublequoteopen}{\isacharparenleft}{\kern0pt}elect\ m\ A\ p\ {\isasymunion}\ elect\ n\ {\isacharquery}{\kern0pt}new{\isacharunderscore}{\kern0pt}A\ {\isacharquery}{\kern0pt}new{\isacharunderscore}{\kern0pt}p{\isacharparenright}{\kern0pt}\ {\isasymunion}\isanewline
\ \ \ \ \ \ \ \ {\isacharparenleft}{\kern0pt}reject\ m\ A\ p\ {\isasymunion}\ reject\ n\ {\isacharquery}{\kern0pt}new{\isacharunderscore}{\kern0pt}A\ {\isacharquery}{\kern0pt}new{\isacharunderscore}{\kern0pt}p{\isacharparenright}{\kern0pt}\ {\isasymunion}\isanewline
\ \ \ \ \ \ \ \ \ defer\ n\ {\isacharquery}{\kern0pt}new{\isacharunderscore}{\kern0pt}A\ {\isacharquery}{\kern0pt}new{\isacharunderscore}{\kern0pt}p\ {\isacharequal}{\kern0pt}\ A{\isachardoublequoteclose}\isanewline
\ \ \ \ \isacommand{by}\isamarkupfalse%
\ blast\isanewline
\ \ \isacommand{hence}\isamarkupfalse%
\isanewline
\ \ \ \ {\isachardoublequoteopen}set{\isacharunderscore}{\kern0pt}equals{\isacharunderscore}{\kern0pt}partition\ A\isanewline
\ \ \ \ \ \ {\isacharparenleft}{\kern0pt}elect\ m\ A\ p\ {\isasymunion}\ elect\ n\ {\isacharquery}{\kern0pt}new{\isacharunderscore}{\kern0pt}A\ {\isacharquery}{\kern0pt}new{\isacharunderscore}{\kern0pt}p{\isacharcomma}{\kern0pt}\isanewline
\ \ \ \ \ \ reject\ m\ A\ p\ {\isasymunion}\ reject\ n\ {\isacharquery}{\kern0pt}new{\isacharunderscore}{\kern0pt}A\ {\isacharquery}{\kern0pt}new{\isacharunderscore}{\kern0pt}p{\isacharcomma}{\kern0pt}\isanewline
\ \ \ \ \ \ defer\ n\ {\isacharquery}{\kern0pt}new{\isacharunderscore}{\kern0pt}A\ {\isacharquery}{\kern0pt}new{\isacharunderscore}{\kern0pt}p{\isacharparenright}{\kern0pt}{\isachardoublequoteclose}\isanewline
\ \ \ \ \isacommand{by}\isamarkupfalse%
\ simp\isanewline
\ \ \isacommand{thus}\isamarkupfalse%
\ {\isacharquery}{\kern0pt}thesis\isanewline
\ \ \ \ \isacommand{using}\isamarkupfalse%
\ sequential{\isacharunderscore}{\kern0pt}composition{\isachardot}{\kern0pt}simps\isanewline
\ \ \ \ \isacommand{by}\isamarkupfalse%
\ metis\isanewline
\isacommand{qed}\isamarkupfalse%
%
\endisatagproof
{\isafoldproof}%
%
\isadelimproof
%
\endisadelimproof
%
\isadelimdocument
%
\endisadelimdocument
%
\isatagdocument
%
\isamarkupsubsection{Soundness%
}
\isamarkuptrue%
%
\endisatagdocument
{\isafolddocument}%
%
\isadelimdocument
%
\endisadelimdocument
\isacommand{theorem}\isamarkupfalse%
\ seq{\isacharunderscore}{\kern0pt}comp{\isacharunderscore}{\kern0pt}sound{\isacharbrackleft}{\kern0pt}simp{\isacharbrackright}{\kern0pt}{\isacharcolon}{\kern0pt}\isanewline
\ \ \isakeyword{assumes}\ module{\isacharunderscore}{\kern0pt}m{\isacharcolon}{\kern0pt}\ {\isachardoublequoteopen}electoral{\isacharunderscore}{\kern0pt}module\ m{\isachardoublequoteclose}\ \isakeyword{and}\isanewline
\ \ \ \ \ \ \ \ \ \ module{\isacharunderscore}{\kern0pt}n{\isacharcolon}{\kern0pt}\ {\isachardoublequoteopen}electoral{\isacharunderscore}{\kern0pt}module\ n{\isachardoublequoteclose}\isanewline
\ \ \ \ \ \ \ \ \isakeyword{shows}\ {\isachardoublequoteopen}electoral{\isacharunderscore}{\kern0pt}module\ {\isacharparenleft}{\kern0pt}m\ {\isasymtriangleright}\ n{\isacharparenright}{\kern0pt}{\isachardoublequoteclose}\isanewline
%
\isadelimproof
\ \ %
\endisadelimproof
%
\isatagproof
\isacommand{unfolding}\isamarkupfalse%
\ electoral{\isacharunderscore}{\kern0pt}module{\isacharunderscore}{\kern0pt}def\isanewline
\isacommand{proof}\isamarkupfalse%
\ {\isacharparenleft}{\kern0pt}safe{\isacharparenright}{\kern0pt}\isanewline
\ \ \isacommand{fix}\isamarkupfalse%
\isanewline
\ \ \ \ A\ {\isacharcolon}{\kern0pt}{\isacharcolon}{\kern0pt}\ {\isachardoublequoteopen}{\isacharprime}{\kern0pt}a\ set{\isachardoublequoteclose}\ \isakeyword{and}\isanewline
\ \ \ \ p\ {\isacharcolon}{\kern0pt}{\isacharcolon}{\kern0pt}\ {\isachardoublequoteopen}{\isacharprime}{\kern0pt}a\ Profile{\isachardoublequoteclose}\isanewline
\ \ \isacommand{assume}\isamarkupfalse%
\isanewline
\ \ \ \ fin{\isacharunderscore}{\kern0pt}A{\isacharcolon}{\kern0pt}\ {\isachardoublequoteopen}finite\ A{\isachardoublequoteclose}\ \isakeyword{and}\isanewline
\ \ \ \ prof{\isacharunderscore}{\kern0pt}A{\isacharcolon}{\kern0pt}\ {\isachardoublequoteopen}profile\ A\ p{\isachardoublequoteclose}\isanewline
\ \ \isacommand{have}\isamarkupfalse%
\ {\isachardoublequoteopen}{\isasymforall}r{\isachardot}{\kern0pt}\ well{\isacharunderscore}{\kern0pt}formed\ {\isacharparenleft}{\kern0pt}A{\isacharcolon}{\kern0pt}{\isacharcolon}{\kern0pt}{\isacharprime}{\kern0pt}a\ set{\isacharparenright}{\kern0pt}\ r\ {\isacharequal}{\kern0pt}\isanewline
\ \ \ \ \ \ \ \ \ \ {\isacharparenleft}{\kern0pt}disjoint{\isadigit{3}}\ r\ {\isasymand}\ set{\isacharunderscore}{\kern0pt}equals{\isacharunderscore}{\kern0pt}partition\ A\ r{\isacharparenright}{\kern0pt}{\isachardoublequoteclose}\isanewline
\ \ \ \ \isacommand{by}\isamarkupfalse%
\ simp\isanewline
\ \ \isacommand{thus}\isamarkupfalse%
\ {\isachardoublequoteopen}well{\isacharunderscore}{\kern0pt}formed\ A\ {\isacharparenleft}{\kern0pt}{\isacharparenleft}{\kern0pt}m\ {\isasymtriangleright}\ n{\isacharparenright}{\kern0pt}\ A\ p{\isacharparenright}{\kern0pt}{\isachardoublequoteclose}\isanewline
\ \ \ \ \isacommand{using}\isamarkupfalse%
\ module{\isacharunderscore}{\kern0pt}m\ module{\isacharunderscore}{\kern0pt}n\ seq{\isacharunderscore}{\kern0pt}comp{\isacharunderscore}{\kern0pt}presv{\isacharunderscore}{\kern0pt}disj\isanewline
\ \ \ \ \ \ \ \ \ \ seq{\isacharunderscore}{\kern0pt}comp{\isacharunderscore}{\kern0pt}presv{\isacharunderscore}{\kern0pt}alts\ fin{\isacharunderscore}{\kern0pt}A\ prof{\isacharunderscore}{\kern0pt}A\isanewline
\ \ \ \ \isacommand{by}\isamarkupfalse%
\ metis\isanewline
\isacommand{qed}\isamarkupfalse%
%
\endisatagproof
{\isafoldproof}%
%
\isadelimproof
%
\endisadelimproof
%
\isadelimdocument
%
\endisadelimdocument
%
\isatagdocument
%
\isamarkupsubsection{Lemmata%
}
\isamarkuptrue%
%
\endisatagdocument
{\isafolddocument}%
%
\isadelimdocument
%
\endisadelimdocument
\isacommand{lemma}\isamarkupfalse%
\ seq{\isacharunderscore}{\kern0pt}comp{\isacharunderscore}{\kern0pt}dec{\isacharunderscore}{\kern0pt}only{\isacharunderscore}{\kern0pt}def{\isacharcolon}{\kern0pt}\isanewline
\ \ \isakeyword{assumes}\isanewline
\ \ \ \ module{\isacharunderscore}{\kern0pt}m{\isacharcolon}{\kern0pt}\ {\isachardoublequoteopen}electoral{\isacharunderscore}{\kern0pt}module\ m{\isachardoublequoteclose}\ \isakeyword{and}\isanewline
\ \ \ \ module{\isacharunderscore}{\kern0pt}n{\isacharcolon}{\kern0pt}\ {\isachardoublequoteopen}electoral{\isacharunderscore}{\kern0pt}module\ n{\isachardoublequoteclose}\ \isakeyword{and}\isanewline
\ \ \ \ f{\isacharunderscore}{\kern0pt}prof{\isacharcolon}{\kern0pt}\ {\isachardoublequoteopen}finite{\isacharunderscore}{\kern0pt}profile\ A\ p{\isachardoublequoteclose}\ \isakeyword{and}\isanewline
\ \ \ \ empty{\isacharunderscore}{\kern0pt}defer{\isacharcolon}{\kern0pt}\ {\isachardoublequoteopen}defer\ m\ A\ p\ {\isacharequal}{\kern0pt}\ {\isacharbraceleft}{\kern0pt}{\isacharbraceright}{\kern0pt}{\isachardoublequoteclose}\isanewline
\ \ \isakeyword{shows}\ {\isachardoublequoteopen}{\isacharparenleft}{\kern0pt}m\ {\isasymtriangleright}\ n{\isacharparenright}{\kern0pt}\ A\ p\ {\isacharequal}{\kern0pt}\ \ m\ A\ p{\isachardoublequoteclose}\isanewline
%
\isadelimproof
\ \ %
\endisadelimproof
%
\isatagproof
\isacommand{using}\isamarkupfalse%
\ Int{\isacharunderscore}{\kern0pt}lower{\isadigit{1}}\ Un{\isacharunderscore}{\kern0pt}absorb{\isadigit{2}}\ bot{\isachardot}{\kern0pt}extremum{\isacharunderscore}{\kern0pt}uniqueI\ defer{\isacharunderscore}{\kern0pt}in{\isacharunderscore}{\kern0pt}alts\isanewline
\ \ \ \ \ \ \ \ elect{\isacharunderscore}{\kern0pt}in{\isacharunderscore}{\kern0pt}alts\ empty{\isacharunderscore}{\kern0pt}defer\ module{\isacharunderscore}{\kern0pt}m\ module{\isacharunderscore}{\kern0pt}n\ prod{\isachardot}{\kern0pt}collapse\isanewline
\ \ \ \ \ \ \ \ f{\isacharunderscore}{\kern0pt}prof\ reject{\isacharunderscore}{\kern0pt}in{\isacharunderscore}{\kern0pt}alts\ sequential{\isacharunderscore}{\kern0pt}composition{\isachardot}{\kern0pt}simps\isanewline
\ \ \ \ \ \ \ \ def{\isacharunderscore}{\kern0pt}presv{\isacharunderscore}{\kern0pt}fin{\isacharunderscore}{\kern0pt}prof\ result{\isacharunderscore}{\kern0pt}disj\isanewline
\ \ \isacommand{by}\isamarkupfalse%
\ {\isacharparenleft}{\kern0pt}smt\ {\isacharparenleft}{\kern0pt}verit{\isacharparenright}{\kern0pt}{\isacharparenright}{\kern0pt}%
\endisatagproof
{\isafoldproof}%
%
\isadelimproof
\isanewline
%
\endisadelimproof
\isanewline
\isacommand{lemma}\isamarkupfalse%
\ seq{\isacharunderscore}{\kern0pt}comp{\isacharunderscore}{\kern0pt}def{\isacharunderscore}{\kern0pt}then{\isacharunderscore}{\kern0pt}elect{\isacharcolon}{\kern0pt}\isanewline
\ \ \isakeyword{assumes}\isanewline
\ \ \ \ n{\isacharunderscore}{\kern0pt}electing{\isacharunderscore}{\kern0pt}m{\isacharcolon}{\kern0pt}\ {\isachardoublequoteopen}non{\isacharunderscore}{\kern0pt}electing\ m{\isachardoublequoteclose}\ \isakeyword{and}\isanewline
\ \ \ \ def{\isacharunderscore}{\kern0pt}one{\isacharunderscore}{\kern0pt}m{\isacharcolon}{\kern0pt}\ {\isachardoublequoteopen}defers\ {\isadigit{1}}\ m{\isachardoublequoteclose}\ \isakeyword{and}\isanewline
\ \ \ \ electing{\isacharunderscore}{\kern0pt}n{\isacharcolon}{\kern0pt}\ {\isachardoublequoteopen}electing\ n{\isachardoublequoteclose}\ \isakeyword{and}\isanewline
\ \ \ \ f{\isacharunderscore}{\kern0pt}prof{\isacharcolon}{\kern0pt}\ {\isachardoublequoteopen}finite{\isacharunderscore}{\kern0pt}profile\ A\ p{\isachardoublequoteclose}\isanewline
\ \ \isakeyword{shows}\ {\isachardoublequoteopen}elect\ {\isacharparenleft}{\kern0pt}m\ {\isasymtriangleright}\ n{\isacharparenright}{\kern0pt}\ A\ p\ {\isacharequal}{\kern0pt}\ defer\ m\ A\ p{\isachardoublequoteclose}\isanewline
%
\isadelimproof
%
\endisadelimproof
%
\isatagproof
\isacommand{proof}\isamarkupfalse%
\ cases\isanewline
\ \ \isacommand{assume}\isamarkupfalse%
\ {\isachardoublequoteopen}A\ {\isacharequal}{\kern0pt}\ {\isacharbraceleft}{\kern0pt}{\isacharbraceright}{\kern0pt}{\isachardoublequoteclose}\isanewline
\ \ \isacommand{with}\isamarkupfalse%
\ electing{\isacharunderscore}{\kern0pt}n\ n{\isacharunderscore}{\kern0pt}electing{\isacharunderscore}{\kern0pt}m\ f{\isacharunderscore}{\kern0pt}prof\ \isacommand{show}\isamarkupfalse%
\ {\isacharquery}{\kern0pt}thesis\isanewline
\ \ \ \ \isacommand{using}\isamarkupfalse%
\ bot{\isachardot}{\kern0pt}extremum{\isacharunderscore}{\kern0pt}uniqueI\ defer{\isacharunderscore}{\kern0pt}in{\isacharunderscore}{\kern0pt}alts\ elect{\isacharunderscore}{\kern0pt}in{\isacharunderscore}{\kern0pt}alts\isanewline
\ \ \ \ \ \ \ \ \ \ electing{\isacharunderscore}{\kern0pt}def\ non{\isacharunderscore}{\kern0pt}electing{\isacharunderscore}{\kern0pt}def\ seq{\isacharunderscore}{\kern0pt}comp{\isacharunderscore}{\kern0pt}sound\isanewline
\ \ \ \ \isacommand{by}\isamarkupfalse%
\ metis\isanewline
\isacommand{next}\isamarkupfalse%
\isanewline
\ \ \isacommand{assume}\isamarkupfalse%
\ assm{\isacharcolon}{\kern0pt}\ {\isachardoublequoteopen}A\ {\isasymnoteq}\ {\isacharbraceleft}{\kern0pt}{\isacharbraceright}{\kern0pt}{\isachardoublequoteclose}\isanewline
\ \ \isacommand{from}\isamarkupfalse%
\ n{\isacharunderscore}{\kern0pt}electing{\isacharunderscore}{\kern0pt}m\ f{\isacharunderscore}{\kern0pt}prof\ \isacommand{have}\isamarkupfalse%
\ ele{\isacharcolon}{\kern0pt}\ {\isachardoublequoteopen}elect\ m\ A\ p\ {\isacharequal}{\kern0pt}\ {\isacharbraceleft}{\kern0pt}{\isacharbraceright}{\kern0pt}{\isachardoublequoteclose}\isanewline
\ \ \ \ \isacommand{using}\isamarkupfalse%
\ non{\isacharunderscore}{\kern0pt}electing{\isacharunderscore}{\kern0pt}def\isanewline
\ \ \ \ \isacommand{by}\isamarkupfalse%
\ auto\isanewline
\ \ \isacommand{from}\isamarkupfalse%
\ assm\ def{\isacharunderscore}{\kern0pt}one{\isacharunderscore}{\kern0pt}m\ f{\isacharunderscore}{\kern0pt}prof\ finite\ \isacommand{have}\isamarkupfalse%
\ def{\isacharunderscore}{\kern0pt}card{\isacharcolon}{\kern0pt}\isanewline
\ \ \ \ {\isachardoublequoteopen}card\ {\isacharparenleft}{\kern0pt}defer\ m\ A\ p{\isacharparenright}{\kern0pt}\ {\isacharequal}{\kern0pt}\ {\isadigit{1}}{\isachardoublequoteclose}\isanewline
\ \ \ \ \isacommand{by}\isamarkupfalse%
\ {\isacharparenleft}{\kern0pt}simp\ add{\isacharcolon}{\kern0pt}\ Suc{\isacharunderscore}{\kern0pt}leI\ card{\isacharunderscore}{\kern0pt}gt{\isacharunderscore}{\kern0pt}{\isadigit{0}}{\isacharunderscore}{\kern0pt}iff\ defers{\isacharunderscore}{\kern0pt}def{\isacharparenright}{\kern0pt}\isanewline
\ \ \isacommand{with}\isamarkupfalse%
\ n{\isacharunderscore}{\kern0pt}electing{\isacharunderscore}{\kern0pt}m\ f{\isacharunderscore}{\kern0pt}prof\ \isacommand{have}\isamarkupfalse%
\ def{\isacharcolon}{\kern0pt}\isanewline
\ \ \ \ {\isachardoublequoteopen}{\isasymexists}a\ {\isasymin}\ A{\isachardot}{\kern0pt}\ defer\ m\ A\ p\ {\isacharequal}{\kern0pt}\ {\isacharbraceleft}{\kern0pt}a{\isacharbraceright}{\kern0pt}{\isachardoublequoteclose}\isanewline
\ \ \ \ \isacommand{using}\isamarkupfalse%
\ card{\isacharunderscore}{\kern0pt}{\isadigit{1}}{\isacharunderscore}{\kern0pt}singletonE\ defer{\isacharunderscore}{\kern0pt}in{\isacharunderscore}{\kern0pt}alts\isanewline
\ \ \ \ \ \ \ \ \ \ non{\isacharunderscore}{\kern0pt}electing{\isacharunderscore}{\kern0pt}def\ singletonI\ subsetCE\isanewline
\ \ \ \ \isacommand{by}\isamarkupfalse%
\ metis\isanewline
\ \ \isacommand{from}\isamarkupfalse%
\ ele\ def\ n{\isacharunderscore}{\kern0pt}electing{\isacharunderscore}{\kern0pt}m\ \isacommand{have}\isamarkupfalse%
\ rej{\isacharcolon}{\kern0pt}\isanewline
\ \ \ \ {\isachardoublequoteopen}{\isasymexists}a\ {\isasymin}\ A{\isachardot}{\kern0pt}\ reject\ m\ A\ p\ {\isacharequal}{\kern0pt}\ A{\isacharminus}{\kern0pt}{\isacharbraceleft}{\kern0pt}a{\isacharbraceright}{\kern0pt}{\isachardoublequoteclose}\isanewline
\ \ \ \ \isacommand{using}\isamarkupfalse%
\ Diff{\isacharunderscore}{\kern0pt}empty\ def{\isacharunderscore}{\kern0pt}one{\isacharunderscore}{\kern0pt}m\ defers{\isacharunderscore}{\kern0pt}def\ f{\isacharunderscore}{\kern0pt}prof\ reject{\isacharunderscore}{\kern0pt}not{\isacharunderscore}{\kern0pt}elec{\isacharunderscore}{\kern0pt}or{\isacharunderscore}{\kern0pt}def\isanewline
\ \ \ \ \isacommand{by}\isamarkupfalse%
\ metis\isanewline
\ \ \isacommand{from}\isamarkupfalse%
\ ele\ rej\ def\ n{\isacharunderscore}{\kern0pt}electing{\isacharunderscore}{\kern0pt}m\ f{\isacharunderscore}{\kern0pt}prof\ \isacommand{have}\isamarkupfalse%
\ res{\isacharunderscore}{\kern0pt}m{\isacharcolon}{\kern0pt}\isanewline
\ \ \ \ {\isachardoublequoteopen}{\isasymexists}a\ {\isasymin}\ A{\isachardot}{\kern0pt}\ m\ A\ p\ {\isacharequal}{\kern0pt}\ {\isacharparenleft}{\kern0pt}{\isacharbraceleft}{\kern0pt}{\isacharbraceright}{\kern0pt}{\isacharcomma}{\kern0pt}\ A{\isacharminus}{\kern0pt}{\isacharbraceleft}{\kern0pt}a{\isacharbraceright}{\kern0pt}{\isacharcomma}{\kern0pt}\ {\isacharbraceleft}{\kern0pt}a{\isacharbraceright}{\kern0pt}{\isacharparenright}{\kern0pt}{\isachardoublequoteclose}\isanewline
\ \ \ \ \isacommand{using}\isamarkupfalse%
\ Diff{\isacharunderscore}{\kern0pt}empty\ combine{\isacharunderscore}{\kern0pt}ele{\isacharunderscore}{\kern0pt}rej{\isacharunderscore}{\kern0pt}def\ non{\isacharunderscore}{\kern0pt}electing{\isacharunderscore}{\kern0pt}def\isanewline
\ \ \ \ \ \ \ \ \ \ reject{\isacharunderscore}{\kern0pt}not{\isacharunderscore}{\kern0pt}elec{\isacharunderscore}{\kern0pt}or{\isacharunderscore}{\kern0pt}def\isanewline
\ \ \ \ \isacommand{by}\isamarkupfalse%
\ metis\isanewline
\ \ \isacommand{hence}\isamarkupfalse%
\isanewline
\ \ \ \ {\isachardoublequoteopen}{\isasymexists}a\ {\isasymin}\ A{\isachardot}{\kern0pt}\ elect\ {\isacharparenleft}{\kern0pt}m\ {\isasymtriangleright}\ n{\isacharparenright}{\kern0pt}\ A\ p\ {\isacharequal}{\kern0pt}\isanewline
\ \ \ \ \ \ \ \ elect\ n\ {\isacharbraceleft}{\kern0pt}a{\isacharbraceright}{\kern0pt}\ {\isacharparenleft}{\kern0pt}limit{\isacharunderscore}{\kern0pt}profile\ {\isacharbraceleft}{\kern0pt}a{\isacharbraceright}{\kern0pt}\ p{\isacharparenright}{\kern0pt}{\isachardoublequoteclose}\isanewline
\ \ \ \ \isacommand{using}\isamarkupfalse%
\ prod{\isachardot}{\kern0pt}sel{\isacharparenleft}{\kern0pt}{\isadigit{1}}{\isacharparenright}{\kern0pt}\ prod{\isachardot}{\kern0pt}sel{\isacharparenleft}{\kern0pt}{\isadigit{2}}{\isacharparenright}{\kern0pt}\ sequential{\isacharunderscore}{\kern0pt}composition{\isachardot}{\kern0pt}simps\isanewline
\ \ \ \ \ \ \ \ \ \ sup{\isacharunderscore}{\kern0pt}bot{\isachardot}{\kern0pt}left{\isacharunderscore}{\kern0pt}neutral\isanewline
\ \ \ \ \isacommand{by}\isamarkupfalse%
\ metis\isanewline
\ \ \isacommand{with}\isamarkupfalse%
\ def{\isacharunderscore}{\kern0pt}card\ def\ electing{\isacharunderscore}{\kern0pt}n\ n{\isacharunderscore}{\kern0pt}electing{\isacharunderscore}{\kern0pt}m\ f{\isacharunderscore}{\kern0pt}prof\ \isacommand{have}\isamarkupfalse%
\isanewline
\ \ \ \ {\isachardoublequoteopen}{\isasymexists}a\ {\isasymin}\ A{\isachardot}{\kern0pt}\ elect\ {\isacharparenleft}{\kern0pt}m\ {\isasymtriangleright}\ n{\isacharparenright}{\kern0pt}\ A\ p\ {\isacharequal}{\kern0pt}\ {\isacharbraceleft}{\kern0pt}a{\isacharbraceright}{\kern0pt}{\isachardoublequoteclose}\isanewline
\ \ \ \ \isacommand{using}\isamarkupfalse%
\ electing{\isacharunderscore}{\kern0pt}for{\isacharunderscore}{\kern0pt}only{\isacharunderscore}{\kern0pt}alt\ non{\isacharunderscore}{\kern0pt}electing{\isacharunderscore}{\kern0pt}def\ prod{\isachardot}{\kern0pt}sel\isanewline
\ \ \ \ \ \ \ \ \ \ sequential{\isacharunderscore}{\kern0pt}composition{\isachardot}{\kern0pt}simps\ def{\isacharunderscore}{\kern0pt}presv{\isacharunderscore}{\kern0pt}fin{\isacharunderscore}{\kern0pt}prof\isanewline
\ \ \ \ \ \ \ \ \ \ sup{\isacharunderscore}{\kern0pt}bot{\isachardot}{\kern0pt}left{\isacharunderscore}{\kern0pt}neutral\isanewline
\ \ \ \ \isacommand{by}\isamarkupfalse%
\ metis\isanewline
\ \ \isacommand{with}\isamarkupfalse%
\ def\ def{\isacharunderscore}{\kern0pt}card\ electing{\isacharunderscore}{\kern0pt}n\ n{\isacharunderscore}{\kern0pt}electing{\isacharunderscore}{\kern0pt}m\ f{\isacharunderscore}{\kern0pt}prof\ res{\isacharunderscore}{\kern0pt}m\isanewline
\ \ \isacommand{show}\isamarkupfalse%
\ {\isacharquery}{\kern0pt}thesis\isanewline
\ \ \ \ \isacommand{using}\isamarkupfalse%
\ Diff{\isacharunderscore}{\kern0pt}disjoint\ Diff{\isacharunderscore}{\kern0pt}insert{\isacharunderscore}{\kern0pt}absorb\ Int{\isacharunderscore}{\kern0pt}insert{\isacharunderscore}{\kern0pt}right\isanewline
\ \ \ \ \ \ \ \ \ \ Un{\isacharunderscore}{\kern0pt}Diff{\isacharunderscore}{\kern0pt}Int\ electing{\isacharunderscore}{\kern0pt}for{\isacharunderscore}{\kern0pt}only{\isacharunderscore}{\kern0pt}alt\ empty{\isacharunderscore}{\kern0pt}iff\isanewline
\ \ \ \ \ \ \ \ \ \ non{\isacharunderscore}{\kern0pt}electing{\isacharunderscore}{\kern0pt}def\ prod{\isachardot}{\kern0pt}sel\ sequential{\isacharunderscore}{\kern0pt}composition{\isachardot}{\kern0pt}simps\isanewline
\ \ \ \ \ \ \ \ \ \ def{\isacharunderscore}{\kern0pt}presv{\isacharunderscore}{\kern0pt}fin{\isacharunderscore}{\kern0pt}prof\ singletonI\ f{\isacharunderscore}{\kern0pt}prof\isanewline
\ \ \ \ \isacommand{by}\isamarkupfalse%
\ {\isacharparenleft}{\kern0pt}smt\ {\isacharparenleft}{\kern0pt}verit{\isacharcomma}{\kern0pt}\ best{\isacharparenright}{\kern0pt}{\isacharparenright}{\kern0pt}\isanewline
\isacommand{qed}\isamarkupfalse%
%
\endisatagproof
{\isafoldproof}%
%
\isadelimproof
\isanewline
%
\endisadelimproof
\isanewline
\isacommand{lemma}\isamarkupfalse%
\ seq{\isacharunderscore}{\kern0pt}comp{\isacharunderscore}{\kern0pt}def{\isacharunderscore}{\kern0pt}card{\isacharunderscore}{\kern0pt}bounded{\isacharcolon}{\kern0pt}\isanewline
\ \ \isakeyword{assumes}\isanewline
\ \ \ \ module{\isacharunderscore}{\kern0pt}m{\isacharcolon}{\kern0pt}\ {\isachardoublequoteopen}electoral{\isacharunderscore}{\kern0pt}module\ m{\isachardoublequoteclose}\ \isakeyword{and}\isanewline
\ \ \ \ module{\isacharunderscore}{\kern0pt}n{\isacharcolon}{\kern0pt}\ {\isachardoublequoteopen}electoral{\isacharunderscore}{\kern0pt}module\ n{\isachardoublequoteclose}\ \isakeyword{and}\isanewline
\ \ \ \ f{\isacharunderscore}{\kern0pt}prof{\isacharcolon}{\kern0pt}\ {\isachardoublequoteopen}finite{\isacharunderscore}{\kern0pt}profile\ A\ p{\isachardoublequoteclose}\isanewline
\ \ \isakeyword{shows}\ {\isachardoublequoteopen}card\ {\isacharparenleft}{\kern0pt}defer\ {\isacharparenleft}{\kern0pt}m\ {\isasymtriangleright}\ n{\isacharparenright}{\kern0pt}\ A\ p{\isacharparenright}{\kern0pt}\ {\isasymle}\ card\ {\isacharparenleft}{\kern0pt}defer\ m\ A\ p{\isacharparenright}{\kern0pt}{\isachardoublequoteclose}\isanewline
%
\isadelimproof
\ \ %
\endisadelimproof
%
\isatagproof
\isacommand{using}\isamarkupfalse%
\ card{\isacharunderscore}{\kern0pt}mono\ defer{\isacharunderscore}{\kern0pt}in{\isacharunderscore}{\kern0pt}alts\ module{\isacharunderscore}{\kern0pt}m\ module{\isacharunderscore}{\kern0pt}n\ f{\isacharunderscore}{\kern0pt}prof\isanewline
\ \ \ \ \ \ \ \ sequential{\isacharunderscore}{\kern0pt}composition{\isachardot}{\kern0pt}simps\ def{\isacharunderscore}{\kern0pt}presv{\isacharunderscore}{\kern0pt}fin{\isacharunderscore}{\kern0pt}prof\ snd{\isacharunderscore}{\kern0pt}conv\isanewline
\ \ \isacommand{by}\isamarkupfalse%
\ metis%
\endisatagproof
{\isafoldproof}%
%
\isadelimproof
\isanewline
%
\endisadelimproof
\isanewline
\isacommand{lemma}\isamarkupfalse%
\ seq{\isacharunderscore}{\kern0pt}comp{\isacharunderscore}{\kern0pt}def{\isacharunderscore}{\kern0pt}set{\isacharunderscore}{\kern0pt}bounded{\isacharcolon}{\kern0pt}\isanewline
\ \ \isakeyword{assumes}\isanewline
\ \ \ \ module{\isacharunderscore}{\kern0pt}m{\isacharcolon}{\kern0pt}\ {\isachardoublequoteopen}electoral{\isacharunderscore}{\kern0pt}module\ m{\isachardoublequoteclose}\ \isakeyword{and}\isanewline
\ \ \ \ module{\isacharunderscore}{\kern0pt}n{\isacharcolon}{\kern0pt}\ {\isachardoublequoteopen}electoral{\isacharunderscore}{\kern0pt}module\ n{\isachardoublequoteclose}\ \isakeyword{and}\isanewline
\ \ \ \ f{\isacharunderscore}{\kern0pt}prof{\isacharcolon}{\kern0pt}\ {\isachardoublequoteopen}finite{\isacharunderscore}{\kern0pt}profile\ A\ p{\isachardoublequoteclose}\isanewline
\ \ \isakeyword{shows}\ {\isachardoublequoteopen}defer\ {\isacharparenleft}{\kern0pt}m\ {\isasymtriangleright}\ n{\isacharparenright}{\kern0pt}\ A\ p\ {\isasymsubseteq}\ defer\ m\ A\ p{\isachardoublequoteclose}\isanewline
%
\isadelimproof
\ \ %
\endisadelimproof
%
\isatagproof
\isacommand{using}\isamarkupfalse%
\ defer{\isacharunderscore}{\kern0pt}in{\isacharunderscore}{\kern0pt}alts\ module{\isacharunderscore}{\kern0pt}m\ module{\isacharunderscore}{\kern0pt}n\ prod{\isachardot}{\kern0pt}sel{\isacharparenleft}{\kern0pt}{\isadigit{2}}{\isacharparenright}{\kern0pt}\ f{\isacharunderscore}{\kern0pt}prof\isanewline
\ \ \ \ \ \ \ \ sequential{\isacharunderscore}{\kern0pt}composition{\isachardot}{\kern0pt}simps\ def{\isacharunderscore}{\kern0pt}presv{\isacharunderscore}{\kern0pt}fin{\isacharunderscore}{\kern0pt}prof\isanewline
\ \ \isacommand{by}\isamarkupfalse%
\ metis%
\endisatagproof
{\isafoldproof}%
%
\isadelimproof
\isanewline
%
\endisadelimproof
\isanewline
\isacommand{lemma}\isamarkupfalse%
\ seq{\isacharunderscore}{\kern0pt}comp{\isacharunderscore}{\kern0pt}defers{\isacharunderscore}{\kern0pt}def{\isacharunderscore}{\kern0pt}set{\isacharcolon}{\kern0pt}\isanewline
\ \ \isakeyword{assumes}\isanewline
\ \ \ \ module{\isacharunderscore}{\kern0pt}m{\isacharcolon}{\kern0pt}\ {\isachardoublequoteopen}electoral{\isacharunderscore}{\kern0pt}module\ m{\isachardoublequoteclose}\ \isakeyword{and}\isanewline
\ \ \ \ module{\isacharunderscore}{\kern0pt}n{\isacharcolon}{\kern0pt}\ {\isachardoublequoteopen}electoral{\isacharunderscore}{\kern0pt}module\ n{\isachardoublequoteclose}\ \isakeyword{and}\isanewline
\ \ \ \ f{\isacharunderscore}{\kern0pt}prof{\isacharcolon}{\kern0pt}\ {\isachardoublequoteopen}finite{\isacharunderscore}{\kern0pt}profile\ A\ p{\isachardoublequoteclose}\isanewline
\ \ \isakeyword{shows}\isanewline
\ \ \ \ {\isachardoublequoteopen}defer\ {\isacharparenleft}{\kern0pt}m\ {\isasymtriangleright}\ n{\isacharparenright}{\kern0pt}\ A\ p\ {\isacharequal}{\kern0pt}\isanewline
\ \ \ \ \ \ defer\ n\ {\isacharparenleft}{\kern0pt}defer\ m\ A\ p{\isacharparenright}{\kern0pt}\ {\isacharparenleft}{\kern0pt}limit{\isacharunderscore}{\kern0pt}profile\ {\isacharparenleft}{\kern0pt}defer\ m\ A\ p{\isacharparenright}{\kern0pt}\ p{\isacharparenright}{\kern0pt}{\isachardoublequoteclose}\isanewline
%
\isadelimproof
\ \ %
\endisadelimproof
%
\isatagproof
\isacommand{using}\isamarkupfalse%
\ sequential{\isacharunderscore}{\kern0pt}composition{\isachardot}{\kern0pt}simps\ snd{\isacharunderscore}{\kern0pt}conv\isanewline
\ \ \isacommand{by}\isamarkupfalse%
\ metis%
\endisatagproof
{\isafoldproof}%
%
\isadelimproof
\isanewline
%
\endisadelimproof
\isanewline
\isacommand{lemma}\isamarkupfalse%
\ seq{\isacharunderscore}{\kern0pt}comp{\isacharunderscore}{\kern0pt}def{\isacharunderscore}{\kern0pt}then{\isacharunderscore}{\kern0pt}elect{\isacharunderscore}{\kern0pt}elec{\isacharunderscore}{\kern0pt}set{\isacharcolon}{\kern0pt}\isanewline
\ \ \isakeyword{assumes}\isanewline
\ \ \ \ module{\isacharunderscore}{\kern0pt}m{\isacharcolon}{\kern0pt}\ {\isachardoublequoteopen}electoral{\isacharunderscore}{\kern0pt}module\ m{\isachardoublequoteclose}\ \isakeyword{and}\isanewline
\ \ \ \ module{\isacharunderscore}{\kern0pt}n{\isacharcolon}{\kern0pt}\ {\isachardoublequoteopen}electoral{\isacharunderscore}{\kern0pt}module\ n{\isachardoublequoteclose}\ \isakeyword{and}\isanewline
\ \ \ \ f{\isacharunderscore}{\kern0pt}prof{\isacharcolon}{\kern0pt}\ {\isachardoublequoteopen}finite{\isacharunderscore}{\kern0pt}profile\ A\ p{\isachardoublequoteclose}\isanewline
\ \ \isakeyword{shows}\isanewline
\ \ \ \ {\isachardoublequoteopen}elect\ {\isacharparenleft}{\kern0pt}m\ {\isasymtriangleright}\ n{\isacharparenright}{\kern0pt}\ A\ p\ {\isacharequal}{\kern0pt}\isanewline
\ \ \ \ \ \ elect\ n\ {\isacharparenleft}{\kern0pt}defer\ m\ A\ p{\isacharparenright}{\kern0pt}\ {\isacharparenleft}{\kern0pt}limit{\isacharunderscore}{\kern0pt}profile\ {\isacharparenleft}{\kern0pt}defer\ m\ A\ p{\isacharparenright}{\kern0pt}\ p{\isacharparenright}{\kern0pt}\ {\isasymunion}\isanewline
\ \ \ \ \ \ {\isacharparenleft}{\kern0pt}elect\ m\ A\ p{\isacharparenright}{\kern0pt}{\isachardoublequoteclose}\isanewline
%
\isadelimproof
\ \ %
\endisadelimproof
%
\isatagproof
\isacommand{using}\isamarkupfalse%
\ Un{\isacharunderscore}{\kern0pt}commute\ fst{\isacharunderscore}{\kern0pt}conv\ sequential{\isacharunderscore}{\kern0pt}composition{\isachardot}{\kern0pt}simps\isanewline
\ \ \isacommand{by}\isamarkupfalse%
\ metis%
\endisatagproof
{\isafoldproof}%
%
\isadelimproof
\isanewline
%
\endisadelimproof
\isanewline
\isacommand{lemma}\isamarkupfalse%
\ seq{\isacharunderscore}{\kern0pt}comp{\isacharunderscore}{\kern0pt}elim{\isacharunderscore}{\kern0pt}one{\isacharunderscore}{\kern0pt}red{\isacharunderscore}{\kern0pt}def{\isacharunderscore}{\kern0pt}set{\isacharcolon}{\kern0pt}\isanewline
\ \ \isakeyword{assumes}\isanewline
\ \ \ \ module{\isacharunderscore}{\kern0pt}m{\isacharcolon}{\kern0pt}\ {\isachardoublequoteopen}electoral{\isacharunderscore}{\kern0pt}module\ m{\isachardoublequoteclose}\ \isakeyword{and}\isanewline
\ \ \ \ module{\isacharunderscore}{\kern0pt}n{\isacharcolon}{\kern0pt}\ {\isachardoublequoteopen}eliminates\ {\isadigit{1}}\ n{\isachardoublequoteclose}\ \isakeyword{and}\isanewline
\ \ \ \ f{\isacharunderscore}{\kern0pt}prof{\isacharcolon}{\kern0pt}\ {\isachardoublequoteopen}finite{\isacharunderscore}{\kern0pt}profile\ A\ p{\isachardoublequoteclose}\ \isakeyword{and}\isanewline
\ \ \ \ enough{\isacharunderscore}{\kern0pt}leftover{\isacharcolon}{\kern0pt}\ {\isachardoublequoteopen}card\ {\isacharparenleft}{\kern0pt}defer\ m\ A\ p{\isacharparenright}{\kern0pt}\ {\isachargreater}{\kern0pt}\ {\isadigit{1}}{\isachardoublequoteclose}\isanewline
\ \ \isakeyword{shows}\ {\isachardoublequoteopen}defer\ {\isacharparenleft}{\kern0pt}m\ {\isasymtriangleright}\ n{\isacharparenright}{\kern0pt}\ A\ p\ {\isasymsubset}\ defer\ m\ A\ p{\isachardoublequoteclose}\isanewline
%
\isadelimproof
\ \ %
\endisadelimproof
%
\isatagproof
\isacommand{using}\isamarkupfalse%
\ enough{\isacharunderscore}{\kern0pt}leftover\ module{\isacharunderscore}{\kern0pt}m\ module{\isacharunderscore}{\kern0pt}n\ f{\isacharunderscore}{\kern0pt}prof\isanewline
\ \ \ \ \ \ \ \ sequential{\isacharunderscore}{\kern0pt}composition{\isachardot}{\kern0pt}simps\ def{\isacharunderscore}{\kern0pt}presv{\isacharunderscore}{\kern0pt}fin{\isacharunderscore}{\kern0pt}prof\isanewline
\ \ \ \ \ \ \ \ single{\isacharunderscore}{\kern0pt}elim{\isacharunderscore}{\kern0pt}imp{\isacharunderscore}{\kern0pt}red{\isacharunderscore}{\kern0pt}def{\isacharunderscore}{\kern0pt}set\ snd{\isacharunderscore}{\kern0pt}conv\isanewline
\ \ \isacommand{by}\isamarkupfalse%
\ metis%
\endisatagproof
{\isafoldproof}%
%
\isadelimproof
\isanewline
%
\endisadelimproof
\isanewline
\isacommand{lemma}\isamarkupfalse%
\ seq{\isacharunderscore}{\kern0pt}comp{\isacharunderscore}{\kern0pt}def{\isacharunderscore}{\kern0pt}set{\isacharunderscore}{\kern0pt}sound{\isacharcolon}{\kern0pt}\isanewline
\ \ \isakeyword{assumes}\isanewline
\ \ \ \ {\isachardoublequoteopen}electoral{\isacharunderscore}{\kern0pt}module\ m{\isachardoublequoteclose}\ \isakeyword{and}\isanewline
\ \ \ \ {\isachardoublequoteopen}electoral{\isacharunderscore}{\kern0pt}module\ n{\isachardoublequoteclose}\ \isakeyword{and}\isanewline
\ \ \ \ {\isachardoublequoteopen}finite{\isacharunderscore}{\kern0pt}profile\ A\ p{\isachardoublequoteclose}\isanewline
\ \ \isakeyword{shows}\ {\isachardoublequoteopen}defer\ {\isacharparenleft}{\kern0pt}m\ {\isasymtriangleright}\ n{\isacharparenright}{\kern0pt}\ A\ p\ {\isasymsubseteq}\ defer\ m\ A\ p{\isachardoublequoteclose}\isanewline
%
\isadelimproof
%
\endisadelimproof
%
\isatagproof
\isacommand{proof}\isamarkupfalse%
\ {\isacharminus}{\kern0pt}\isanewline
\ \ \isacommand{have}\isamarkupfalse%
\ {\isachardoublequoteopen}{\isasymforall}A\ p{\isachardot}{\kern0pt}\ finite{\isacharunderscore}{\kern0pt}profile\ A\ p\ {\isasymlongrightarrow}\ well{\isacharunderscore}{\kern0pt}formed\ A\ {\isacharparenleft}{\kern0pt}n\ A\ p{\isacharparenright}{\kern0pt}{\isachardoublequoteclose}\isanewline
\ \ \ \ \isacommand{using}\isamarkupfalse%
\ assms{\isacharparenleft}{\kern0pt}{\isadigit{2}}{\isacharparenright}{\kern0pt}\ electoral{\isacharunderscore}{\kern0pt}module{\isacharunderscore}{\kern0pt}def\isanewline
\ \ \ \ \isacommand{by}\isamarkupfalse%
\ auto\isanewline
\ \ \isacommand{hence}\isamarkupfalse%
\isanewline
\ \ \ \ {\isachardoublequoteopen}finite{\isacharunderscore}{\kern0pt}profile\ {\isacharparenleft}{\kern0pt}defer\ m\ A\ p{\isacharparenright}{\kern0pt}\ {\isacharparenleft}{\kern0pt}limit{\isacharunderscore}{\kern0pt}profile\ {\isacharparenleft}{\kern0pt}defer\ m\ A\ p{\isacharparenright}{\kern0pt}\ p{\isacharparenright}{\kern0pt}\ {\isasymlongrightarrow}\isanewline
\ \ \ \ \ \ \ \ well{\isacharunderscore}{\kern0pt}formed\ {\isacharparenleft}{\kern0pt}defer\ m\ A\ p{\isacharparenright}{\kern0pt}\isanewline
\ \ \ \ \ \ \ \ \ \ {\isacharparenleft}{\kern0pt}n\ {\isacharparenleft}{\kern0pt}defer\ m\ A\ p{\isacharparenright}{\kern0pt}\ {\isacharparenleft}{\kern0pt}limit{\isacharunderscore}{\kern0pt}profile\ {\isacharparenleft}{\kern0pt}defer\ m\ A\ p{\isacharparenright}{\kern0pt}\ p{\isacharparenright}{\kern0pt}{\isacharparenright}{\kern0pt}{\isachardoublequoteclose}\isanewline
\ \ \ \ \isacommand{by}\isamarkupfalse%
\ simp\isanewline
\ \ \isacommand{hence}\isamarkupfalse%
\isanewline
\ \ \ \ {\isachardoublequoteopen}well{\isacharunderscore}{\kern0pt}formed\ {\isacharparenleft}{\kern0pt}defer\ m\ A\ p{\isacharparenright}{\kern0pt}\ {\isacharparenleft}{\kern0pt}n\ {\isacharparenleft}{\kern0pt}defer\ m\ A\ p{\isacharparenright}{\kern0pt}\isanewline
\ \ \ \ \ \ {\isacharparenleft}{\kern0pt}limit{\isacharunderscore}{\kern0pt}profile\ {\isacharparenleft}{\kern0pt}defer\ m\ A\ p{\isacharparenright}{\kern0pt}\ p{\isacharparenright}{\kern0pt}{\isacharparenright}{\kern0pt}{\isachardoublequoteclose}\isanewline
\ \ \ \ \isacommand{using}\isamarkupfalse%
\ assms{\isacharparenleft}{\kern0pt}{\isadigit{1}}{\isacharparenright}{\kern0pt}\ assms{\isacharparenleft}{\kern0pt}{\isadigit{3}}{\isacharparenright}{\kern0pt}\ def{\isacharunderscore}{\kern0pt}presv{\isacharunderscore}{\kern0pt}fin{\isacharunderscore}{\kern0pt}prof\isanewline
\ \ \ \ \isacommand{by}\isamarkupfalse%
\ metis\isanewline
\ \ \isacommand{thus}\isamarkupfalse%
\ {\isacharquery}{\kern0pt}thesis\isanewline
\ \ \ \ \isacommand{using}\isamarkupfalse%
\ assms\ seq{\isacharunderscore}{\kern0pt}comp{\isacharunderscore}{\kern0pt}def{\isacharunderscore}{\kern0pt}set{\isacharunderscore}{\kern0pt}bounded\isanewline
\ \ \ \ \isacommand{by}\isamarkupfalse%
\ blast\isanewline
\isacommand{qed}\isamarkupfalse%
%
\endisatagproof
{\isafoldproof}%
%
\isadelimproof
\isanewline
%
\endisadelimproof
\isanewline
\isacommand{lemma}\isamarkupfalse%
\ seq{\isacharunderscore}{\kern0pt}comp{\isacharunderscore}{\kern0pt}def{\isacharunderscore}{\kern0pt}set{\isacharunderscore}{\kern0pt}trans{\isacharcolon}{\kern0pt}\isanewline
\ \ \isakeyword{assumes}\isanewline
\ \ \ \ {\isachardoublequoteopen}a\ {\isasymin}\ {\isacharparenleft}{\kern0pt}defer\ {\isacharparenleft}{\kern0pt}m\ {\isasymtriangleright}\ n{\isacharparenright}{\kern0pt}\ A\ p{\isacharparenright}{\kern0pt}{\isachardoublequoteclose}\ \isakeyword{and}\isanewline
\ \ \ \ {\isachardoublequoteopen}electoral{\isacharunderscore}{\kern0pt}module\ m\ {\isasymand}\ electoral{\isacharunderscore}{\kern0pt}module\ n{\isachardoublequoteclose}\ \isakeyword{and}\isanewline
\ \ \ \ {\isachardoublequoteopen}finite{\isacharunderscore}{\kern0pt}profile\ A\ p{\isachardoublequoteclose}\isanewline
\ \ \isakeyword{shows}\isanewline
\ \ \ \ {\isachardoublequoteopen}a\ {\isasymin}\ defer\ n\ {\isacharparenleft}{\kern0pt}defer\ m\ A\ p{\isacharparenright}{\kern0pt}\isanewline
\ \ \ \ \ \ {\isacharparenleft}{\kern0pt}limit{\isacharunderscore}{\kern0pt}profile\ {\isacharparenleft}{\kern0pt}defer\ m\ A\ p{\isacharparenright}{\kern0pt}\ p{\isacharparenright}{\kern0pt}\ {\isasymand}\isanewline
\ \ \ \ \ \ a\ {\isasymin}\ defer\ m\ A\ p{\isachardoublequoteclose}\isanewline
%
\isadelimproof
\ \ %
\endisadelimproof
%
\isatagproof
\isacommand{using}\isamarkupfalse%
\ seq{\isacharunderscore}{\kern0pt}comp{\isacharunderscore}{\kern0pt}def{\isacharunderscore}{\kern0pt}set{\isacharunderscore}{\kern0pt}bounded\ assms{\isacharparenleft}{\kern0pt}{\isadigit{1}}{\isacharparenright}{\kern0pt}\ assms{\isacharparenleft}{\kern0pt}{\isadigit{2}}{\isacharparenright}{\kern0pt}\isanewline
\ \ \ \ \ \ \ \ assms{\isacharparenleft}{\kern0pt}{\isadigit{3}}{\isacharparenright}{\kern0pt}\ in{\isacharunderscore}{\kern0pt}mono\ seq{\isacharunderscore}{\kern0pt}comp{\isacharunderscore}{\kern0pt}defers{\isacharunderscore}{\kern0pt}def{\isacharunderscore}{\kern0pt}set\isanewline
\ \ \isacommand{by}\isamarkupfalse%
\ {\isacharparenleft}{\kern0pt}metis\ {\isacharparenleft}{\kern0pt}no{\isacharunderscore}{\kern0pt}types{\isacharcomma}{\kern0pt}\ hide{\isacharunderscore}{\kern0pt}lams{\isacharparenright}{\kern0pt}{\isacharparenright}{\kern0pt}%
\endisatagproof
{\isafoldproof}%
%
\isadelimproof
\isanewline
%
\endisadelimproof
%
\isadelimML
\isanewline
%
\endisadelimML
%
\isatagML
\isacommand{ML}\isamarkupfalse%
\ {\isacartoucheopen}\isanewline
\ \ \ \ {\isacharparenleft}{\kern0pt}{\isacharasterisk}{\kern0pt}\ val\ w\ {\isacharequal}{\kern0pt}\ {\isacharparenleft}{\kern0pt}Defs{\isachardot}{\kern0pt}all{\isacharunderscore}{\kern0pt}specifications{\isacharunderscore}{\kern0pt}of\ {\isacharparenleft}{\kern0pt}Theory{\isachardot}{\kern0pt}defs{\isacharunderscore}{\kern0pt}of\ {\isacharat}{\kern0pt}{\isacharbraceleft}{\kern0pt}theory{\isacharbraceright}{\kern0pt}{\isacharparenright}{\kern0pt}{\isacharparenright}{\kern0pt}\isanewline
\ \ \ \ val\ x\ {\isacharequal}{\kern0pt}\ filter\ {\isacharparenleft}{\kern0pt}fn\ x\ {\isacharequal}{\kern0pt}{\isachargreater}{\kern0pt}\ String{\isachardot}{\kern0pt}isPrefix\ {\isachardoublequote}{\kern0pt}Seq{\isachardoublequote}{\kern0pt}\ x{\isacharparenright}{\kern0pt}\ {\isacharparenleft}{\kern0pt}map\ {\isacharparenleft}{\kern0pt}fn\ x\ {\isacharequal}{\kern0pt}{\isachargreater}{\kern0pt}\ snd\ {\isacharparenleft}{\kern0pt}fst\ x{\isacharparenright}{\kern0pt}{\isacharparenright}{\kern0pt}\isanewline
\ \ \ \ \ \ \ \ {\isacharparenleft}{\kern0pt}Defs{\isachardot}{\kern0pt}all{\isacharunderscore}{\kern0pt}specifications{\isacharunderscore}{\kern0pt}of\ {\isacharparenleft}{\kern0pt}Theory{\isachardot}{\kern0pt}defs{\isacharunderscore}{\kern0pt}of\ {\isacharat}{\kern0pt}{\isacharbraceleft}{\kern0pt}theory{\isacharbraceright}{\kern0pt}{\isacharparenright}{\kern0pt}{\isacharparenright}{\kern0pt}{\isacharparenright}{\kern0pt}\ {\isacharasterisk}{\kern0pt}{\isacharparenright}{\kern0pt}\isanewline
\ \ \ \ val\ y\ {\isacharequal}{\kern0pt}\ filter\ {\isacharparenleft}{\kern0pt}fn\ x\ {\isacharequal}{\kern0pt}{\isachargreater}{\kern0pt}\ {\isacharparenleft}{\kern0pt}String{\isachardot}{\kern0pt}isPrefix\ {\isacharparenleft}{\kern0pt}{\isachardoublequote}{\kern0pt}Sequential{\isacharunderscore}{\kern0pt}Composition{\isachardot}{\kern0pt}sequential{\isacharunderscore}{\kern0pt}composition{\isachardoublequote}{\kern0pt}{\isacharparenright}{\kern0pt}\ {\isacharparenleft}{\kern0pt}snd\ {\isacharparenleft}{\kern0pt}fst\ {\isacharparenleft}{\kern0pt}x{\isacharparenright}{\kern0pt}{\isacharparenright}{\kern0pt}{\isacharparenright}{\kern0pt}{\isacharparenright}{\kern0pt}{\isacharparenright}{\kern0pt}\ \isanewline
\ \ \ \ \ \ \ \ {\isacharparenleft}{\kern0pt}Defs{\isachardot}{\kern0pt}all{\isacharunderscore}{\kern0pt}specifications{\isacharunderscore}{\kern0pt}of\ {\isacharparenleft}{\kern0pt}Theory{\isachardot}{\kern0pt}defs{\isacharunderscore}{\kern0pt}of\ {\isacharat}{\kern0pt}{\isacharbraceleft}{\kern0pt}theory{\isacharbraceright}{\kern0pt}{\isacharparenright}{\kern0pt}{\isacharparenright}{\kern0pt}\isanewline
\ \ \ \ val\ z\ {\isacharequal}{\kern0pt}\ filter\ {\isacharparenleft}{\kern0pt}fn\ x\ {\isacharequal}{\kern0pt}{\isachargreater}{\kern0pt}\ String{\isachardot}{\kern0pt}isPrefix\ {\isacharparenleft}{\kern0pt}{\isachardoublequote}{\kern0pt}Sequential{\isacharunderscore}{\kern0pt}Composition{\isachardot}{\kern0pt}sequential{\isacharunderscore}{\kern0pt}composition{\isacharunderscore}{\kern0pt}def{\isachardoublequote}{\kern0pt}{\isacharparenright}{\kern0pt}\ {\isacharparenleft}{\kern0pt}{\isacharhash}{\kern0pt}description\ x{\isacharparenright}{\kern0pt}{\isacharparenright}{\kern0pt}\ {\isacharparenleft}{\kern0pt}map\ {\isacharparenleft}{\kern0pt}fn\ x\ {\isacharequal}{\kern0pt}{\isachargreater}{\kern0pt}\ {\isacharparenleft}{\kern0pt}hd\ {\isacharparenleft}{\kern0pt}snd\ x{\isacharparenright}{\kern0pt}{\isacharparenright}{\kern0pt}{\isacharparenright}{\kern0pt}\ y{\isacharparenright}{\kern0pt}\isanewline
\isanewline
\ \ \ \ val\ def\ {\isacharequal}{\kern0pt}\ hd\ z\isanewline
\ \ \ \ val\ def{\isacharunderscore}{\kern0pt}rhs\ {\isacharequal}{\kern0pt}\ {\isacharhash}{\kern0pt}rhs\ def\isanewline
\isanewline
\ \ \ \ val\ def{\isacharunderscore}{\kern0pt}funs\ {\isacharequal}{\kern0pt}\ filter\ {\isacharparenleft}{\kern0pt}fn\ x\ {\isacharequal}{\kern0pt}{\isachargreater}{\kern0pt}\ String{\isachardot}{\kern0pt}isPrefix\ {\isacharparenleft}{\kern0pt}{\isachardoublequote}{\kern0pt}fun{\isachardoublequote}{\kern0pt}{\isacharparenright}{\kern0pt}\ {\isacharparenleft}{\kern0pt}snd\ {\isacharparenleft}{\kern0pt}fst\ x{\isacharparenright}{\kern0pt}{\isacharparenright}{\kern0pt}{\isacharparenright}{\kern0pt}\ {\isacharparenleft}{\kern0pt}def{\isacharunderscore}{\kern0pt}rhs{\isacharparenright}{\kern0pt}\isanewline
\ \ \ \ val\ def{\isacharunderscore}{\kern0pt}consts\ {\isacharequal}{\kern0pt}\ filter\ {\isacharparenleft}{\kern0pt}fn\ x\ {\isacharequal}{\kern0pt}{\isachargreater}{\kern0pt}\ not\ {\isacharparenleft}{\kern0pt}String{\isachardot}{\kern0pt}isPrefix\ {\isacharparenleft}{\kern0pt}{\isachardoublequote}{\kern0pt}fun{\isachardoublequote}{\kern0pt}{\isacharparenright}{\kern0pt}\ {\isacharparenleft}{\kern0pt}snd\ {\isacharparenleft}{\kern0pt}fst\ x{\isacharparenright}{\kern0pt}{\isacharparenright}{\kern0pt}{\isacharparenright}{\kern0pt}{\isacharparenright}{\kern0pt}\ {\isacharparenleft}{\kern0pt}def{\isacharunderscore}{\kern0pt}rhs{\isacharparenright}{\kern0pt}\isanewline
\isanewline
\ \ \ \ val\ fun{\isacharunderscore}{\kern0pt}type\ {\isacharequal}{\kern0pt}\ Defs{\isachardot}{\kern0pt}pretty{\isacharunderscore}{\kern0pt}entry\ {\isacharparenleft}{\kern0pt}Defs{\isachardot}{\kern0pt}global{\isacharunderscore}{\kern0pt}context\ {\isacharat}{\kern0pt}{\isacharbraceleft}{\kern0pt}theory{\isacharbraceright}{\kern0pt}{\isacharparenright}{\kern0pt}\ {\isacharparenleft}{\kern0pt}List{\isachardot}{\kern0pt}last\ def{\isacharunderscore}{\kern0pt}funs{\isacharparenright}{\kern0pt}\isanewline
\isanewline
\ \ \ {\isacharparenleft}{\kern0pt}{\isacharasterisk}{\kern0pt}\ val\ a\ {\isacharequal}{\kern0pt}\ Defs{\isachardot}{\kern0pt}pretty{\isacharunderscore}{\kern0pt}entry\ {\isacharparenleft}{\kern0pt}Defs{\isachardot}{\kern0pt}global{\isacharunderscore}{\kern0pt}context\ {\isacharat}{\kern0pt}{\isacharbraceleft}{\kern0pt}theory{\isacharbraceright}{\kern0pt}{\isacharparenright}{\kern0pt}\ {\isacharparenleft}{\kern0pt}nth\ {\isacharparenleft}{\kern0pt}snd\ {\isacharparenleft}{\kern0pt}{\isacharhash}{\kern0pt}lhs\ {\isacharparenleft}{\kern0pt}hd\ z{\isacharparenright}{\kern0pt}{\isacharcomma}{\kern0pt}\ {\isacharhash}{\kern0pt}rhs\ {\isacharparenleft}{\kern0pt}hd\ z{\isacharparenright}{\kern0pt}{\isacharparenright}{\kern0pt}\ {\isacharparenright}{\kern0pt}{\isadigit{2}}{\isacharparenright}{\kern0pt}\ {\isacharasterisk}{\kern0pt}{\isacharparenright}{\kern0pt}\isanewline
\ {\isacartoucheclose}\isanewline
\isanewline
\isacommand{ML}\isamarkupfalse%
\ {\isacartoucheopen}\ val\ facts\ {\isacharequal}{\kern0pt}\ Global{\isacharunderscore}{\kern0pt}Theory{\isachardot}{\kern0pt}facts{\isacharunderscore}{\kern0pt}of\ {\isacharat}{\kern0pt}{\isacharbraceleft}{\kern0pt}theory{\isacharbraceright}{\kern0pt}\ \isanewline
\ \ \ \ \ val\ filtered\ {\isacharequal}{\kern0pt}\ filter\ {\isacharparenleft}{\kern0pt}fn\ x\ {\isacharequal}{\kern0pt}{\isachargreater}{\kern0pt}\ true{\isacharparenright}{\kern0pt}\ {\isacharbrackleft}{\kern0pt}{\isacharbrackright}{\kern0pt}{\isacartoucheclose}\isanewline
\isanewline
\isacommand{ML{\isacharunderscore}{\kern0pt}val}\isamarkupfalse%
\ {\isacartoucheopen}val\ seq{\isacharunderscore}{\kern0pt}consts\ {\isacharequal}{\kern0pt}\ filter\ {\isacharparenleft}{\kern0pt}fn\ x\ {\isacharequal}{\kern0pt}{\isachargreater}{\kern0pt}\ {\isacharparenleft}{\kern0pt}String{\isachardot}{\kern0pt}isPrefix\ {\isacharparenleft}{\kern0pt}Context{\isachardot}{\kern0pt}theory{\isacharunderscore}{\kern0pt}name\ {\isacharat}{\kern0pt}{\isacharbraceleft}{\kern0pt}theory{\isacharbraceright}{\kern0pt}{\isacharparenright}{\kern0pt}\ {\isacharparenleft}{\kern0pt}snd\ {\isacharparenleft}{\kern0pt}fst\ {\isacharparenleft}{\kern0pt}x{\isacharparenright}{\kern0pt}{\isacharparenright}{\kern0pt}{\isacharparenright}{\kern0pt}{\isacharparenright}{\kern0pt}{\isacharparenright}{\kern0pt}\ \isanewline
\ \ \ \ {\isacharparenleft}{\kern0pt}Defs{\isachardot}{\kern0pt}all{\isacharunderscore}{\kern0pt}specifications{\isacharunderscore}{\kern0pt}of\ {\isacharparenleft}{\kern0pt}Theory{\isachardot}{\kern0pt}defs{\isacharunderscore}{\kern0pt}of\ {\isacharat}{\kern0pt}{\isacharbraceleft}{\kern0pt}theory{\isacharbraceright}{\kern0pt}{\isacharparenright}{\kern0pt}{\isacharparenright}{\kern0pt}\isanewline
\ \ \ \ val\ names\ {\isacharequal}{\kern0pt}\ map\ {\isacharparenleft}{\kern0pt}fn\ x\ {\isacharequal}{\kern0pt}{\isachargreater}{\kern0pt}\ {\isacharparenleft}{\kern0pt}{\isacharhash}{\kern0pt}description\ {\isacharparenleft}{\kern0pt}hd\ {\isacharparenleft}{\kern0pt}snd{\isacharparenleft}{\kern0pt}x{\isacharparenright}{\kern0pt}{\isacharparenright}{\kern0pt}{\isacharparenright}{\kern0pt}{\isacharparenright}{\kern0pt}{\isacharparenright}{\kern0pt}\ seq{\isacharunderscore}{\kern0pt}consts\isanewline
\ \ \ \ val\ seq{\isacharunderscore}{\kern0pt}defs\ {\isacharequal}{\kern0pt}\ filter\ {\isacharparenleft}{\kern0pt}String{\isachardot}{\kern0pt}isSuffix\ {\isacharparenleft}{\kern0pt}{\isachardoublequote}{\kern0pt}{\isacharunderscore}{\kern0pt}def{\isachardoublequote}{\kern0pt}{\isacharparenright}{\kern0pt}{\isacharparenright}{\kern0pt}\ names\ \isanewline
\ \ \ \ val\ strip{\isacharunderscore}{\kern0pt}fun\ {\isacharequal}{\kern0pt}\ fn\ x\ {\isacharequal}{\kern0pt}{\isachargreater}{\kern0pt}\ String{\isachardot}{\kern0pt}substring\ {\isacharparenleft}{\kern0pt}{\isacharparenleft}{\kern0pt}x{\isacharparenright}{\kern0pt}{\isacharcomma}{\kern0pt}{\isadigit{0}}{\isacharcomma}{\kern0pt}{\isacharparenleft}{\kern0pt}{\isacharparenleft}{\kern0pt}String{\isachardot}{\kern0pt}size\ x{\isacharparenright}{\kern0pt}{\isacharminus}{\kern0pt}{\isadigit{4}}{\isacharparenright}{\kern0pt}{\isacharparenright}{\kern0pt}\isanewline
\ \ \ \ val\ seq{\isacharunderscore}{\kern0pt}defs{\isacharunderscore}{\kern0pt}stripped\ {\isacharequal}{\kern0pt}\ map\ strip{\isacharunderscore}{\kern0pt}fun\ seq{\isacharunderscore}{\kern0pt}defs\isanewline
\isanewline
\ \ \ \ fun\ member{\isacharunderscore}{\kern0pt}of\ {\isacharparenleft}{\kern0pt}item{\isacharcomma}{\kern0pt}\ list{\isacharparenright}{\kern0pt}\ {\isacharequal}{\kern0pt}\ List{\isachardot}{\kern0pt}exists\ {\isacharparenleft}{\kern0pt}fn\ x\ {\isacharequal}{\kern0pt}{\isachargreater}{\kern0pt}\ x\ {\isacharequal}{\kern0pt}\ item{\isacharparenright}{\kern0pt}\ list\isanewline
\ \ \ \ val\ filter{\isacharunderscore}{\kern0pt}fun\ {\isacharequal}{\kern0pt}\ fn\ x\ {\isacharequal}{\kern0pt}{\isachargreater}{\kern0pt}\ member{\isacharunderscore}{\kern0pt}of\ {\isacharparenleft}{\kern0pt}x{\isacharcomma}{\kern0pt}\ seq{\isacharunderscore}{\kern0pt}defs{\isacharunderscore}{\kern0pt}stripped{\isacharparenright}{\kern0pt}\isanewline
\isanewline
\ \ \ \ val\ constants\ {\isacharequal}{\kern0pt}\ {\isacharhash}{\kern0pt}constants\ {\isacharparenleft}{\kern0pt}Consts{\isachardot}{\kern0pt}dest\ {\isacharparenleft}{\kern0pt}Sign{\isachardot}{\kern0pt}consts{\isacharunderscore}{\kern0pt}of\ {\isacharat}{\kern0pt}{\isacharbraceleft}{\kern0pt}theory{\isacharbraceright}{\kern0pt}{\isacharparenright}{\kern0pt}{\isacharparenright}{\kern0pt}\isanewline
\ \ \ \ val\ filtered{\isacharunderscore}{\kern0pt}constants\ {\isacharequal}{\kern0pt}\ filter\ {\isacharparenleft}{\kern0pt}filter{\isacharunderscore}{\kern0pt}fun{\isacharparenright}{\kern0pt}\ {\isacharparenleft}{\kern0pt}map\ fst\ constants{\isacharparenright}{\kern0pt}\isanewline
\isanewline
\ \ \ \ val\ reduce{\isacharunderscore}{\kern0pt}noise\ {\isacharequal}{\kern0pt}\ fn\ x\ {\isacharequal}{\kern0pt}{\isachargreater}{\kern0pt}\ not\ {\isacharparenleft}{\kern0pt}{\isacharparenleft}{\kern0pt}String{\isachardot}{\kern0pt}isSuffix\ {\isachardoublequote}{\kern0pt}{\isacharunderscore}{\kern0pt}graph{\isachardoublequote}{\kern0pt}\ x{\isacharparenright}{\kern0pt}\ \isanewline
\ \ \ \ \ \ \ \ \ \ \ \ \ \ \ \ \ \ \ \ \ \ \ \ \ \ \ \ \ \ \ \ \ \ \ \ \ \ \ \ orelse\ {\isacharparenleft}{\kern0pt}String{\isachardot}{\kern0pt}isSuffix\ {\isachardoublequote}{\kern0pt}{\isacharunderscore}{\kern0pt}sumC{\isachardoublequote}{\kern0pt}\ x{\isacharparenright}{\kern0pt}\ \isanewline
\ \ \ \ \ \ \ \ \ \ \ \ \ \ \ \ \ \ \ \ \ \ \ \ \ \ \ \ \ \ \ \ \ \ \ \ \ \ \ \ orelse\ {\isacharparenleft}{\kern0pt}String{\isachardot}{\kern0pt}isSuffix\ {\isachardoublequote}{\kern0pt}{\isacharunderscore}{\kern0pt}rel{\isachardoublequote}{\kern0pt}\ x{\isacharparenright}{\kern0pt}{\isacharparenright}{\kern0pt}\ \isanewline
\isanewline
\ \ \ \ val\ actual{\isacharunderscore}{\kern0pt}constants\ {\isacharequal}{\kern0pt}\ filter\ reduce{\isacharunderscore}{\kern0pt}noise\ filtered{\isacharunderscore}{\kern0pt}constants\isanewline
\isanewline
\ \ \ \ val\ constants\ {\isacharequal}{\kern0pt}\ {\isacharhash}{\kern0pt}constants\ {\isacharparenleft}{\kern0pt}Consts{\isachardot}{\kern0pt}dest\ {\isacharparenleft}{\kern0pt}Sign{\isachardot}{\kern0pt}consts{\isacharunderscore}{\kern0pt}of\ {\isacharat}{\kern0pt}{\isacharbraceleft}{\kern0pt}theory{\isacharbraceright}{\kern0pt}{\isacharparenright}{\kern0pt}{\isacharparenright}{\kern0pt}\isanewline
\ \ \ \ val\ filter{\isacharunderscore}{\kern0pt}fun{\isadigit{2}}\ {\isacharequal}{\kern0pt}\ fn\ x\ {\isacharequal}{\kern0pt}{\isachargreater}{\kern0pt}\ member{\isacharunderscore}{\kern0pt}of\ {\isacharparenleft}{\kern0pt}fst\ x{\isacharcomma}{\kern0pt}\ actual{\isacharunderscore}{\kern0pt}constants{\isacharparenright}{\kern0pt}\isanewline
\ \ \ \ val\ sign\ {\isacharequal}{\kern0pt}\ map\ {\isacharparenleft}{\kern0pt}fn\ x\ {\isacharequal}{\kern0pt}{\isachargreater}{\kern0pt}\ {\isacharparenleft}{\kern0pt}fst\ {\isacharparenleft}{\kern0pt}snd\ x{\isacharparenright}{\kern0pt}{\isacharparenright}{\kern0pt}{\isacharparenright}{\kern0pt}\ {\isacharparenleft}{\kern0pt}filter\ filter{\isacharunderscore}{\kern0pt}fun{\isadigit{2}}\ constants{\isacharparenright}{\kern0pt}\isanewline
\isanewline
\isanewline
\isanewline
\ \ \ \ val\ x\ {\isacharequal}{\kern0pt}\ hd\ sign\isanewline
\isanewline
\ \ \ \ val\ x{\isacharunderscore}{\kern0pt}string\ {\isacharequal}{\kern0pt}\ Syntax{\isachardot}{\kern0pt}string{\isacharunderscore}{\kern0pt}of{\isacharunderscore}{\kern0pt}typ{\isacharunderscore}{\kern0pt}global\ {\isacharat}{\kern0pt}{\isacharbraceleft}{\kern0pt}theory{\isacharbraceright}{\kern0pt}\ x\isanewline
\isanewline
\ \ \ \ val\ x{\isacharunderscore}{\kern0pt}pretty\ {\isacharequal}{\kern0pt}\ Syntax{\isachardot}{\kern0pt}pretty{\isacharunderscore}{\kern0pt}typ{\isacharunderscore}{\kern0pt}global\ {\isacharat}{\kern0pt}{\isacharbraceleft}{\kern0pt}theory{\isacharbraceright}{\kern0pt}\ x\isanewline
\isanewline
\ \ \ \ val\ x{\isacharunderscore}{\kern0pt}pretty{\isacharunderscore}{\kern0pt}string\ {\isacharequal}{\kern0pt}\ Pretty{\isachardot}{\kern0pt}string{\isacharunderscore}{\kern0pt}of\ x{\isacharunderscore}{\kern0pt}pretty\isanewline
\isanewline
\ \ \ \ val\ y\ {\isacharequal}{\kern0pt}\ Logic{\isachardot}{\kern0pt}mk{\isacharunderscore}{\kern0pt}type\ x\isanewline
\isanewline
\ \ \ \ val\ n\ {\isacharequal}{\kern0pt}\ Term{\isachardot}{\kern0pt}size{\isacharunderscore}{\kern0pt}of{\isacharunderscore}{\kern0pt}typ\ x\isanewline
\isanewline
\ \ \ \ val\ pm\ {\isacharequal}{\kern0pt}\ fn\ Basic{\isacharunderscore}{\kern0pt}Term{\isachardot}{\kern0pt}Type\ x\ {\isacharequal}{\kern0pt}{\isachargreater}{\kern0pt}\ snd\ x\isanewline
\isanewline
\ \ \ \ fun\ collapse\ {\isacharparenleft}{\kern0pt}lst{\isacharcolon}{\kern0pt}\ {\isacharprime}{\kern0pt}a\ list\ list{\isacharparenright}{\kern0pt}\ {\isacharequal}{\kern0pt}\ case\ lst\ of\isanewline
\ \ \ \ \ \ \ \ {\isacharbrackleft}{\kern0pt}{\isacharbrackright}{\kern0pt}\ {\isacharequal}{\kern0pt}{\isachargreater}{\kern0pt}\ {\isacharbrackleft}{\kern0pt}{\isacharbrackright}{\kern0pt}\isanewline
\ \ \ \ \ \ {\isacharbar}{\kern0pt}\ hd{\isacharcolon}{\kern0pt}{\isacharcolon}{\kern0pt}tl\ {\isacharequal}{\kern0pt}{\isachargreater}{\kern0pt}\ hd\ {\isacharat}{\kern0pt}\ collapse\ tl\isanewline
\isanewline
\ \ \ \ val\ a\ {\isacharequal}{\kern0pt}\ fn\ x\ {\isacharequal}{\kern0pt}{\isachargreater}{\kern0pt}\isanewline
\ \ \ \ \ \ let\isanewline
\ \ \ \ \ \ \ \ fun\ typ{\isacharunderscore}{\kern0pt}to{\isacharunderscore}{\kern0pt}string\ {\isacharparenleft}{\kern0pt}x{\isacharcolon}{\kern0pt}\ Basic{\isacharunderscore}{\kern0pt}Term{\isachardot}{\kern0pt}typ{\isacharparenright}{\kern0pt}{\isacharcolon}{\kern0pt}\ string\ {\isacharequal}{\kern0pt}\ case\ x\ of\isanewline
\ \ \ \ \ \ \ \ \ \ \ \ Type\ x\ {\isacharequal}{\kern0pt}{\isachargreater}{\kern0pt}\ {\isachardoublequote}{\kern0pt}{\isacharparenleft}{\kern0pt}{\isachardoublequote}{\kern0pt}\ {\isacharcircum}{\kern0pt}\ {\isacharparenleft}{\kern0pt}fst\ x{\isacharparenright}{\kern0pt}\ {\isacharcircum}{\kern0pt}\ String{\isachardot}{\kern0pt}concat\ {\isacharparenleft}{\kern0pt}map\ {\isacharparenleft}{\kern0pt}typ{\isacharunderscore}{\kern0pt}to{\isacharunderscore}{\kern0pt}string{\isacharparenright}{\kern0pt}\ {\isacharparenleft}{\kern0pt}snd\ x{\isacharparenright}{\kern0pt}{\isacharparenright}{\kern0pt}\ {\isacharcircum}{\kern0pt}\ {\isachardoublequote}{\kern0pt}{\isacharparenright}{\kern0pt}{\isachardoublequote}{\kern0pt}\isanewline
\ \ \ \ \ \ \ \ \ \ {\isacharbar}{\kern0pt}\ \ {\isacharunderscore}{\kern0pt}\ {\isacharequal}{\kern0pt}{\isachargreater}{\kern0pt}\ {\isachardoublequote}{\kern0pt}{\isacharparenleft}{\kern0pt}{\isacharquery}{\kern0pt}{\isacharprime}{\kern0pt}a{\isacharparenright}{\kern0pt}{\isachardoublequote}{\kern0pt}\isanewline
\ \ \ \ \ \ \ in\isanewline
\ \ \ \ \ \ \ \ typ{\isacharunderscore}{\kern0pt}to{\isacharunderscore}{\kern0pt}string\ x\isanewline
\ \ \ \ \ \ end\isanewline
\ \ \ \ \ \ \ \isanewline
\ \ \ \ val\ x{\isacharunderscore}{\kern0pt}string{\isadigit{2}}\ {\isacharequal}{\kern0pt}\ a\ x\isanewline
\isanewline
\ \ \ \ val\ z\ {\isacharequal}{\kern0pt}\ Term{\isachardot}{\kern0pt}is{\isacharunderscore}{\kern0pt}Bound\ y\isanewline
\isanewline
\isanewline
\isanewline
val\ a\ {\isacharequal}{\kern0pt}\ fn\ thy\ {\isacharequal}{\kern0pt}{\isachargreater}{\kern0pt}\ {\isacharparenleft}{\kern0pt}{\isacharparenleft}{\kern0pt}map\ {\isacharparenleft}{\kern0pt}fn\ x\ {\isacharequal}{\kern0pt}{\isachargreater}{\kern0pt}\ {\isacharparenleft}{\kern0pt}fst\ x{\isacharparenright}{\kern0pt}{\isacharparenright}{\kern0pt}\ {\isacharparenleft}{\kern0pt}{\isacharhash}{\kern0pt}constants\ {\isacharparenleft}{\kern0pt}Consts{\isachardot}{\kern0pt}dest\ {\isacharparenleft}{\kern0pt}Sign{\isachardot}{\kern0pt}consts{\isacharunderscore}{\kern0pt}of\ thy{\isacharparenright}{\kern0pt}{\isacharparenright}{\kern0pt}{\isacharparenright}{\kern0pt}{\isacharparenright}{\kern0pt}{\isacharcomma}{\kern0pt}map\ {\isacharparenleft}{\kern0pt}fn\ x\ {\isacharequal}{\kern0pt}{\isachargreater}{\kern0pt}\ let\ fun\ typ{\isacharunderscore}{\kern0pt}to{\isacharunderscore}{\kern0pt}string\ {\isacharparenleft}{\kern0pt}x{\isacharcolon}{\kern0pt}\ Basic{\isacharunderscore}{\kern0pt}Term{\isachardot}{\kern0pt}typ{\isacharparenright}{\kern0pt}{\isacharcolon}{\kern0pt}\ string\ {\isacharequal}{\kern0pt}\ case\ x\ of\isanewline
Type\ x\ {\isacharequal}{\kern0pt}{\isachargreater}{\kern0pt}\ {\isachardoublequote}{\kern0pt}{\isacharparenleft}{\kern0pt}{\isachardoublequote}{\kern0pt}\ {\isacharcircum}{\kern0pt}\ {\isacharparenleft}{\kern0pt}fst\ x{\isacharparenright}{\kern0pt}\ {\isacharcircum}{\kern0pt}\ String{\isachardot}{\kern0pt}concat\ {\isacharparenleft}{\kern0pt}map\ {\isacharparenleft}{\kern0pt}typ{\isacharunderscore}{\kern0pt}to{\isacharunderscore}{\kern0pt}string{\isacharparenright}{\kern0pt}\ {\isacharparenleft}{\kern0pt}snd\ x{\isacharparenright}{\kern0pt}{\isacharparenright}{\kern0pt}\ {\isacharcircum}{\kern0pt}\ {\isachardoublequote}{\kern0pt}{\isacharparenright}{\kern0pt}{\isachardoublequote}{\kern0pt}\isanewline
{\isacharbar}{\kern0pt}\ {\isacharunderscore}{\kern0pt}\ {\isacharequal}{\kern0pt}{\isachargreater}{\kern0pt}\ {\isachardoublequote}{\kern0pt}{\isacharparenleft}{\kern0pt}{\isacharquery}{\kern0pt}{\isacharprime}{\kern0pt}a{\isacharparenright}{\kern0pt}{\isachardoublequote}{\kern0pt}\ in\ typ{\isacharunderscore}{\kern0pt}to{\isacharunderscore}{\kern0pt}string\ x\ end{\isacharparenright}{\kern0pt}{\isacharparenleft}{\kern0pt}map\ {\isacharparenleft}{\kern0pt}fn\ x\ {\isacharequal}{\kern0pt}{\isachargreater}{\kern0pt}\ {\isacharparenleft}{\kern0pt}fst\ {\isacharparenleft}{\kern0pt}snd\ x{\isacharparenright}{\kern0pt}{\isacharparenright}{\kern0pt}{\isacharparenright}{\kern0pt}\ {\isacharparenleft}{\kern0pt}{\isacharhash}{\kern0pt}constants\ {\isacharparenleft}{\kern0pt}Consts{\isachardot}{\kern0pt}dest\ {\isacharparenleft}{\kern0pt}Sign{\isachardot}{\kern0pt}consts{\isacharunderscore}{\kern0pt}of\ thy{\isacharparenright}{\kern0pt}{\isacharparenright}{\kern0pt}{\isacharparenright}{\kern0pt}{\isacharparenright}{\kern0pt}{\isacharparenright}{\kern0pt}\isanewline
\isanewline
\ \ val\ b\ {\isacharequal}{\kern0pt}\ rev\ {\isacharparenleft}{\kern0pt}ListPair{\isachardot}{\kern0pt}zip\ {\isacharparenleft}{\kern0pt}a\ {\isacharat}{\kern0pt}{\isacharbraceleft}{\kern0pt}theory{\isacharbraceright}{\kern0pt}{\isacharparenright}{\kern0pt}{\isacharparenright}{\kern0pt}\isanewline
\isanewline
\ \ \ \ {\isacartoucheclose}\isanewline
\isanewline
\isacommand{ML}\isamarkupfalse%
\ {\isacartoucheopen}\isanewline
\ \ \ \ val\ constants\ {\isacharequal}{\kern0pt}\ {\isacharhash}{\kern0pt}constants\ {\isacharparenleft}{\kern0pt}Consts{\isachardot}{\kern0pt}dest\ {\isacharparenleft}{\kern0pt}Sign{\isachardot}{\kern0pt}consts{\isacharunderscore}{\kern0pt}of\ {\isacharat}{\kern0pt}{\isacharbraceleft}{\kern0pt}theory{\isacharbraceright}{\kern0pt}{\isacharparenright}{\kern0pt}{\isacharparenright}{\kern0pt}\isanewline
\isanewline
\ \ \ \ val\ sign{\isadigit{2}}\ {\isacharequal}{\kern0pt}rev\ \ {\isacharparenleft}{\kern0pt}\ map\ {\isacharparenleft}{\kern0pt}fn\ x\ {\isacharequal}{\kern0pt}{\isachargreater}{\kern0pt}\ {\isacharparenleft}{\kern0pt}fst\ \ x{\isacharparenright}{\kern0pt}{\isacharparenright}{\kern0pt}\ constants\ {\isacharparenright}{\kern0pt}\ {\isacartoucheclose}\isanewline
\isanewline
\isanewline
\isacommand{ML{\isacharunderscore}{\kern0pt}val}\isamarkupfalse%
\ {\isacartoucheopen}\isanewline
val\ thms\ {\isacharequal}{\kern0pt}\ map\ {\isacharparenleft}{\kern0pt}fn\ x\ {\isacharequal}{\kern0pt}{\isachargreater}{\kern0pt}\ fst\ {\isacharparenleft}{\kern0pt}snd\ x{\isacharparenright}{\kern0pt}{\isacharparenright}{\kern0pt}\ {\isacharparenleft}{\kern0pt}Global{\isacharunderscore}{\kern0pt}Theory{\isachardot}{\kern0pt}dest{\isacharunderscore}{\kern0pt}thm{\isacharunderscore}{\kern0pt}names\ {\isacharat}{\kern0pt}{\isacharbraceleft}{\kern0pt}theory{\isacharbraceright}{\kern0pt}{\isacharparenright}{\kern0pt}{\isacharsemicolon}{\kern0pt}\isanewline
val\ thm\ {\isacharequal}{\kern0pt}\ Global{\isacharunderscore}{\kern0pt}Theory{\isachardot}{\kern0pt}get{\isacharunderscore}{\kern0pt}thm\ {\isacharat}{\kern0pt}{\isacharbraceleft}{\kern0pt}theory{\isacharbraceright}{\kern0pt}\ {\isacharparenleft}{\kern0pt}hd\ {\isacharparenleft}{\kern0pt}rev\ thms{\isacharparenright}{\kern0pt}{\isacharparenright}{\kern0pt}{\isacharsemicolon}{\kern0pt}\isanewline
{\isacharparenleft}{\kern0pt}Syntax{\isachardot}{\kern0pt}string{\isacharunderscore}{\kern0pt}of{\isacharunderscore}{\kern0pt}term{\isacharunderscore}{\kern0pt}global\ \isactrltheory {\isasymopen}Main{\isasymclose}\ {\isacharparenleft}{\kern0pt}Thm{\isachardot}{\kern0pt}prop{\isacharunderscore}{\kern0pt}of\ thm{\isacharparenright}{\kern0pt}{\isacharparenright}{\kern0pt}\ {\isacharbar}{\kern0pt}{\isachargreater}{\kern0pt}\ writeln\isanewline
{\isacartoucheclose}%
\endisatagML
{\isafoldML}%
%
\isadelimML
\ \isanewline
%
\endisadelimML
%
\isadelimtheory
\isanewline
%
\endisadelimtheory
%
\isatagtheory
\isacommand{end}\isamarkupfalse%
%
\endisatagtheory
{\isafoldtheory}%
%
\isadelimtheory
%
\endisadelimtheory
%
\end{isabellebody}%
\endinput
%:%file=~/Documents/Studies/VotingRuleGenerator/virage/src/test/resources/verifiedVotingRuleConstruction/theories/Compositional_Framework/Components/Compositional_Structures/Sequential_Composition.thy%:%
%:%6=3%:%
%:%11=4%:%
%:%12=5%:%
%:%14=8%:%
%:%30=10%:%
%:%31=10%:%
%:%32=11%:%
%:%33=12%:%
%:%34=13%:%
%:%35=14%:%
%:%44=17%:%
%:%45=18%:%
%:%46=19%:%
%:%47=20%:%
%:%56=22%:%
%:%66=24%:%
%:%67=24%:%
%:%68=25%:%
%:%69=26%:%
%:%74=31%:%
%:%75=32%:%
%:%76=33%:%
%:%77=33%:%
%:%78=34%:%
%:%79=35%:%
%:%80=36%:%
%:%81=37%:%
%:%82=38%:%
%:%83=38%:%
%:%84=39%:%
%:%85=40%:%
%:%86=41%:%
%:%87=42%:%
%:%94=43%:%
%:%95=43%:%
%:%96=44%:%
%:%97=44%:%
%:%98=45%:%
%:%99=45%:%
%:%100=46%:%
%:%101=46%:%
%:%102=46%:%
%:%103=47%:%
%:%104=47%:%
%:%105=48%:%
%:%106=48%:%
%:%107=49%:%
%:%108=49%:%
%:%109=49%:%
%:%110=50%:%
%:%111=51%:%
%:%112=51%:%
%:%113=52%:%
%:%114=52%:%
%:%115=53%:%
%:%116=53%:%
%:%117=53%:%
%:%118=54%:%
%:%119=55%:%
%:%120=55%:%
%:%121=56%:%
%:%122=57%:%
%:%123=57%:%
%:%124=58%:%
%:%125=58%:%
%:%126=59%:%
%:%127=59%:%
%:%128=60%:%
%:%129=61%:%
%:%130=61%:%
%:%131=62%:%
%:%132=63%:%
%:%133=64%:%
%:%134=65%:%
%:%135=66%:%
%:%136=66%:%
%:%137=67%:%
%:%138=67%:%
%:%139=68%:%
%:%140=68%:%
%:%141=69%:%
%:%142=70%:%
%:%143=70%:%
%:%144=71%:%
%:%145=72%:%
%:%146=73%:%
%:%147=74%:%
%:%148=74%:%
%:%149=75%:%
%:%150=75%:%
%:%151=76%:%
%:%152=76%:%
%:%153=77%:%
%:%154=78%:%
%:%155=78%:%
%:%156=79%:%
%:%157=80%:%
%:%158=81%:%
%:%159=81%:%
%:%160=82%:%
%:%161=82%:%
%:%162=82%:%
%:%163=83%:%
%:%164=84%:%
%:%165=85%:%
%:%166=85%:%
%:%167=86%:%
%:%168=87%:%
%:%169=88%:%
%:%170=88%:%
%:%171=89%:%
%:%172=89%:%
%:%173=89%:%
%:%174=89%:%
%:%175=90%:%
%:%176=91%:%
%:%177=92%:%
%:%178=92%:%
%:%179=93%:%
%:%180=93%:%
%:%181=94%:%
%:%182=94%:%
%:%183=94%:%
%:%184=95%:%
%:%185=95%:%
%:%186=96%:%
%:%187=97%:%
%:%188=98%:%
%:%189=98%:%
%:%190=99%:%
%:%191=100%:%
%:%192=101%:%
%:%193=101%:%
%:%194=102%:%
%:%195=102%:%
%:%196=102%:%
%:%197=103%:%
%:%199=105%:%
%:%200=106%:%
%:%201=106%:%
%:%202=107%:%
%:%203=107%:%
%:%204=108%:%
%:%205=108%:%
%:%206=109%:%
%:%207=109%:%
%:%208=110%:%
%:%214=110%:%
%:%217=111%:%
%:%218=112%:%
%:%219=112%:%
%:%220=113%:%
%:%221=114%:%
%:%222=115%:%
%:%223=116%:%
%:%230=117%:%
%:%231=117%:%
%:%232=118%:%
%:%233=118%:%
%:%234=119%:%
%:%235=119%:%
%:%236=120%:%
%:%237=120%:%
%:%238=120%:%
%:%239=121%:%
%:%240=121%:%
%:%241=122%:%
%:%242=122%:%
%:%243=122%:%
%:%244=123%:%
%:%245=124%:%
%:%246=124%:%
%:%247=125%:%
%:%248=125%:%
%:%249=125%:%
%:%250=126%:%
%:%251=127%:%
%:%252=127%:%
%:%253=128%:%
%:%254=128%:%
%:%255=129%:%
%:%256=129%:%
%:%257=129%:%
%:%258=130%:%
%:%260=132%:%
%:%261=133%:%
%:%262=133%:%
%:%263=134%:%
%:%264=134%:%
%:%265=135%:%
%:%266=135%:%
%:%267=135%:%
%:%268=136%:%
%:%270=138%:%
%:%271=139%:%
%:%272=139%:%
%:%273=140%:%
%:%274=140%:%
%:%275=141%:%
%:%278=144%:%
%:%279=145%:%
%:%280=145%:%
%:%281=146%:%
%:%282=146%:%
%:%283=147%:%
%:%284=147%:%
%:%285=148%:%
%:%286=148%:%
%:%287=149%:%
%:%302=151%:%
%:%312=153%:%
%:%313=153%:%
%:%314=154%:%
%:%315=155%:%
%:%316=156%:%
%:%319=157%:%
%:%323=157%:%
%:%324=157%:%
%:%325=158%:%
%:%326=158%:%
%:%327=159%:%
%:%328=159%:%
%:%329=160%:%
%:%330=161%:%
%:%331=162%:%
%:%332=162%:%
%:%333=163%:%
%:%334=164%:%
%:%335=165%:%
%:%336=165%:%
%:%337=166%:%
%:%338=167%:%
%:%339=167%:%
%:%340=168%:%
%:%341=168%:%
%:%342=169%:%
%:%343=169%:%
%:%344=170%:%
%:%345=171%:%
%:%346=171%:%
%:%347=172%:%
%:%362=174%:%
%:%372=176%:%
%:%373=176%:%
%:%374=177%:%
%:%375=178%:%
%:%376=179%:%
%:%377=180%:%
%:%378=181%:%
%:%379=182%:%
%:%382=183%:%
%:%386=183%:%
%:%387=183%:%
%:%388=184%:%
%:%389=185%:%
%:%390=186%:%
%:%391=187%:%
%:%392=187%:%
%:%397=187%:%
%:%400=188%:%
%:%401=189%:%
%:%402=189%:%
%:%403=190%:%
%:%404=191%:%
%:%405=192%:%
%:%406=193%:%
%:%407=194%:%
%:%408=195%:%
%:%415=196%:%
%:%416=196%:%
%:%417=197%:%
%:%418=197%:%
%:%419=198%:%
%:%420=198%:%
%:%421=198%:%
%:%422=199%:%
%:%423=199%:%
%:%424=200%:%
%:%425=201%:%
%:%426=201%:%
%:%427=202%:%
%:%428=202%:%
%:%429=203%:%
%:%430=203%:%
%:%431=204%:%
%:%432=204%:%
%:%433=204%:%
%:%434=205%:%
%:%435=205%:%
%:%436=206%:%
%:%437=206%:%
%:%438=207%:%
%:%439=207%:%
%:%440=207%:%
%:%441=208%:%
%:%442=209%:%
%:%443=209%:%
%:%444=210%:%
%:%445=210%:%
%:%446=210%:%
%:%447=211%:%
%:%448=212%:%
%:%449=212%:%
%:%450=213%:%
%:%451=214%:%
%:%452=214%:%
%:%453=215%:%
%:%454=215%:%
%:%455=215%:%
%:%456=216%:%
%:%457=217%:%
%:%458=217%:%
%:%459=218%:%
%:%460=218%:%
%:%461=219%:%
%:%462=219%:%
%:%463=219%:%
%:%464=220%:%
%:%465=221%:%
%:%466=221%:%
%:%467=222%:%
%:%468=223%:%
%:%469=223%:%
%:%470=224%:%
%:%471=224%:%
%:%472=225%:%
%:%473=226%:%
%:%474=227%:%
%:%475=227%:%
%:%476=228%:%
%:%477=229%:%
%:%478=229%:%
%:%479=230%:%
%:%480=230%:%
%:%481=230%:%
%:%482=231%:%
%:%483=232%:%
%:%484=232%:%
%:%485=233%:%
%:%486=234%:%
%:%487=235%:%
%:%488=235%:%
%:%489=236%:%
%:%490=236%:%
%:%491=237%:%
%:%492=237%:%
%:%493=238%:%
%:%494=238%:%
%:%495=239%:%
%:%496=240%:%
%:%497=241%:%
%:%498=242%:%
%:%499=242%:%
%:%500=243%:%
%:%506=243%:%
%:%509=244%:%
%:%510=245%:%
%:%511=245%:%
%:%512=246%:%
%:%513=247%:%
%:%514=248%:%
%:%515=249%:%
%:%516=250%:%
%:%519=251%:%
%:%523=251%:%
%:%524=251%:%
%:%525=252%:%
%:%526=253%:%
%:%527=253%:%
%:%532=253%:%
%:%535=254%:%
%:%536=255%:%
%:%537=255%:%
%:%538=256%:%
%:%539=257%:%
%:%540=258%:%
%:%541=259%:%
%:%542=260%:%
%:%545=261%:%
%:%549=261%:%
%:%550=261%:%
%:%551=262%:%
%:%552=263%:%
%:%553=263%:%
%:%558=263%:%
%:%561=264%:%
%:%562=265%:%
%:%563=265%:%
%:%564=266%:%
%:%565=267%:%
%:%566=268%:%
%:%567=269%:%
%:%568=270%:%
%:%569=271%:%
%:%570=272%:%
%:%573=273%:%
%:%577=273%:%
%:%578=273%:%
%:%579=274%:%
%:%580=274%:%
%:%585=274%:%
%:%588=275%:%
%:%589=276%:%
%:%590=276%:%
%:%591=277%:%
%:%592=278%:%
%:%593=279%:%
%:%594=280%:%
%:%595=281%:%
%:%596=282%:%
%:%598=284%:%
%:%601=285%:%
%:%605=285%:%
%:%606=285%:%
%:%607=286%:%
%:%608=286%:%
%:%613=286%:%
%:%616=287%:%
%:%617=288%:%
%:%618=288%:%
%:%619=289%:%
%:%620=290%:%
%:%621=291%:%
%:%622=292%:%
%:%623=293%:%
%:%624=294%:%
%:%627=295%:%
%:%631=295%:%
%:%632=295%:%
%:%633=296%:%
%:%634=297%:%
%:%635=298%:%
%:%636=298%:%
%:%641=298%:%
%:%644=299%:%
%:%645=300%:%
%:%646=300%:%
%:%647=301%:%
%:%648=302%:%
%:%649=303%:%
%:%650=304%:%
%:%651=305%:%
%:%658=306%:%
%:%659=306%:%
%:%660=307%:%
%:%661=307%:%
%:%662=308%:%
%:%663=308%:%
%:%664=309%:%
%:%665=309%:%
%:%666=310%:%
%:%667=310%:%
%:%668=311%:%
%:%670=313%:%
%:%671=314%:%
%:%672=314%:%
%:%673=315%:%
%:%674=315%:%
%:%675=316%:%
%:%676=317%:%
%:%677=318%:%
%:%678=318%:%
%:%679=319%:%
%:%680=319%:%
%:%681=320%:%
%:%682=320%:%
%:%683=321%:%
%:%684=321%:%
%:%685=322%:%
%:%686=322%:%
%:%687=323%:%
%:%693=323%:%
%:%696=324%:%
%:%697=325%:%
%:%698=325%:%
%:%699=326%:%
%:%700=327%:%
%:%701=328%:%
%:%702=329%:%
%:%703=330%:%
%:%704=331%:%
%:%706=333%:%
%:%709=334%:%
%:%713=334%:%
%:%714=334%:%
%:%715=335%:%
%:%716=336%:%
%:%717=336%:%
%:%722=336%:%
%:%727=337%:%
%:%732=338%:%
%:%733=338%:%
%:%750=355%:%
%:%751=356%:%
%:%752=357%:%
%:%753=357%:%
%:%754=358%:%
%:%755=359%:%
%:%756=360%:%
%:%757=360%:%
%:%821=424%:%
%:%822=425%:%
%:%823=426%:%
%:%824=426%:%
%:%827=429%:%
%:%828=430%:%
%:%829=431%:%
%:%830=432%:%
%:%831=432%:%
%:%840=436%:%
%:%845=437%:%
%:%850=438%:%
%
\begin{isabellebody}%
\setisabellecontext{Parallel{\isacharunderscore}{\kern0pt}Composition}%
%
\isadelimdocument
\isanewline
%
\endisadelimdocument
%
\isatagdocument
\isanewline
\isanewline
%
\isamarkupsection{Parallel Composition%
}
\isamarkuptrue%
%
\endisatagdocument
{\isafolddocument}%
%
\isadelimdocument
%
\endisadelimdocument
%
\isadelimtheory
%
\endisadelimtheory
%
\isatagtheory
\isacommand{theory}\isamarkupfalse%
\ Parallel{\isacharunderscore}{\kern0pt}Composition\isanewline
\ \ \isakeyword{imports}\ {\isachardoublequoteopen}{\isachardot}{\kern0pt}{\isachardot}{\kern0pt}{\isacharslash}{\kern0pt}Aggregator{\isachardoublequoteclose}\isanewline
\ \ \ \ \ \ \ \ \ \ {\isachardoublequoteopen}{\isachardot}{\kern0pt}{\isachardot}{\kern0pt}{\isacharslash}{\kern0pt}Electoral{\isacharunderscore}{\kern0pt}Module{\isachardoublequoteclose}\isanewline
\isakeyword{begin}%
\endisatagtheory
{\isafoldtheory}%
%
\isadelimtheory
%
\endisadelimtheory
%
\begin{isamarkuptext}%
The parallel composition composes a new electoral module from
two electoral modules combined with an aggregator.
Therein, the two modules each make a decision and the aggregator combines
them to a single (aggregated) result.%
\end{isamarkuptext}\isamarkuptrue%
%
\isadelimdocument
%
\endisadelimdocument
%
\isatagdocument
%
\isamarkupsubsection{Definition%
}
\isamarkuptrue%
%
\endisatagdocument
{\isafolddocument}%
%
\isadelimdocument
%
\endisadelimdocument
\isacommand{fun}\isamarkupfalse%
\ parallel{\isacharunderscore}{\kern0pt}composition\ {\isacharcolon}{\kern0pt}{\isacharcolon}{\kern0pt}\ {\isachardoublequoteopen}{\isacharprime}{\kern0pt}a\ Electoral{\isacharunderscore}{\kern0pt}Module\ {\isasymRightarrow}\ {\isacharprime}{\kern0pt}a\ Electoral{\isacharunderscore}{\kern0pt}Module\ {\isasymRightarrow}\isanewline
\ \ \ \ \ \ \ \ {\isacharprime}{\kern0pt}a\ Aggregator\ {\isasymRightarrow}\ {\isacharprime}{\kern0pt}a\ Electoral{\isacharunderscore}{\kern0pt}Module{\isachardoublequoteclose}\ \isakeyword{where}\isanewline
\ \ {\isachardoublequoteopen}parallel{\isacharunderscore}{\kern0pt}composition\ m\ n\ agg\ A\ p\ {\isacharequal}{\kern0pt}\ agg\ A\ {\isacharparenleft}{\kern0pt}m\ A\ p{\isacharparenright}{\kern0pt}\ {\isacharparenleft}{\kern0pt}n\ A\ p{\isacharparenright}{\kern0pt}{\isachardoublequoteclose}\isanewline
\isanewline
\isacommand{abbreviation}\isamarkupfalse%
\ parallel\ {\isacharcolon}{\kern0pt}{\isacharcolon}{\kern0pt}\ {\isachardoublequoteopen}{\isacharprime}{\kern0pt}a\ Electoral{\isacharunderscore}{\kern0pt}Module\ {\isasymRightarrow}\ {\isacharprime}{\kern0pt}a\ Aggregator\ {\isasymRightarrow}\isanewline
\ \ \ \ \ \ \ \ {\isacharprime}{\kern0pt}a\ Electoral{\isacharunderscore}{\kern0pt}Module\ {\isasymRightarrow}\ {\isacharprime}{\kern0pt}a\ Electoral{\isacharunderscore}{\kern0pt}Module{\isachardoublequoteclose}\isanewline
\ \ \ \ \ \ {\isacharparenleft}{\kern0pt}{\isachardoublequoteopen}{\isacharunderscore}{\kern0pt}\ {\isasymparallel}\isactrlsub {\isacharunderscore}{\kern0pt}\ {\isacharunderscore}{\kern0pt}{\isachardoublequoteclose}\ {\isacharbrackleft}{\kern0pt}{\isadigit{5}}{\isadigit{0}}{\isacharcomma}{\kern0pt}\ {\isadigit{1}}{\isadigit{0}}{\isadigit{0}}{\isadigit{0}}{\isacharcomma}{\kern0pt}\ {\isadigit{5}}{\isadigit{1}}{\isacharbrackright}{\kern0pt}\ {\isadigit{5}}{\isadigit{0}}{\isacharparenright}{\kern0pt}\ \isakeyword{where}\isanewline
\ \ {\isachardoublequoteopen}m\ {\isasymparallel}\isactrlsub a\ n\ {\isacharequal}{\kern0pt}{\isacharequal}{\kern0pt}\ parallel{\isacharunderscore}{\kern0pt}composition\ m\ n\ a{\isachardoublequoteclose}%
\isadelimdocument
%
\endisadelimdocument
%
\isatagdocument
%
\isamarkupsubsection{Soundness%
}
\isamarkuptrue%
%
\endisatagdocument
{\isafolddocument}%
%
\isadelimdocument
%
\endisadelimdocument
\isacommand{theorem}\isamarkupfalse%
\ par{\isacharunderscore}{\kern0pt}comp{\isacharunderscore}{\kern0pt}sound{\isacharbrackleft}{\kern0pt}simp{\isacharbrackright}{\kern0pt}{\isacharcolon}{\kern0pt}\isanewline
\ \ \isakeyword{assumes}\isanewline
\ \ \ \ mod{\isacharunderscore}{\kern0pt}m{\isacharcolon}{\kern0pt}\ {\isachardoublequoteopen}electoral{\isacharunderscore}{\kern0pt}module\ m{\isachardoublequoteclose}\ \isakeyword{and}\isanewline
\ \ \ \ mod{\isacharunderscore}{\kern0pt}n{\isacharcolon}{\kern0pt}\ {\isachardoublequoteopen}electoral{\isacharunderscore}{\kern0pt}module\ n{\isachardoublequoteclose}\ \isakeyword{and}\isanewline
\ \ \ \ agg{\isacharunderscore}{\kern0pt}a{\isacharcolon}{\kern0pt}\ {\isachardoublequoteopen}aggregator\ a{\isachardoublequoteclose}\isanewline
\ \ \isakeyword{shows}\ {\isachardoublequoteopen}electoral{\isacharunderscore}{\kern0pt}module\ {\isacharparenleft}{\kern0pt}m\ {\isasymparallel}\isactrlsub a\ n{\isacharparenright}{\kern0pt}{\isachardoublequoteclose}\isanewline
%
\isadelimproof
\ \ %
\endisadelimproof
%
\isatagproof
\isacommand{unfolding}\isamarkupfalse%
\ electoral{\isacharunderscore}{\kern0pt}module{\isacharunderscore}{\kern0pt}def\isanewline
\isacommand{proof}\isamarkupfalse%
\ {\isacharparenleft}{\kern0pt}safe{\isacharparenright}{\kern0pt}\isanewline
\ \ \isacommand{fix}\isamarkupfalse%
\isanewline
\ \ \ \ A\ {\isacharcolon}{\kern0pt}{\isacharcolon}{\kern0pt}\ {\isachardoublequoteopen}{\isacharprime}{\kern0pt}a\ set{\isachardoublequoteclose}\ \isakeyword{and}\isanewline
\ \ \ \ p\ {\isacharcolon}{\kern0pt}{\isacharcolon}{\kern0pt}\ {\isachardoublequoteopen}{\isacharprime}{\kern0pt}a\ Profile{\isachardoublequoteclose}\isanewline
\ \ \isacommand{assume}\isamarkupfalse%
\isanewline
\ \ \ \ fin{\isacharunderscore}{\kern0pt}A{\isacharcolon}{\kern0pt}\ {\isachardoublequoteopen}finite\ A{\isachardoublequoteclose}\ \isakeyword{and}\isanewline
\ \ \ \ prof{\isacharunderscore}{\kern0pt}A{\isacharcolon}{\kern0pt}\ {\isachardoublequoteopen}profile\ A\ p{\isachardoublequoteclose}\isanewline
\ \ \isacommand{have}\isamarkupfalse%
\ {\isachardoublequoteopen}well{\isacharunderscore}{\kern0pt}formed\ A\ {\isacharparenleft}{\kern0pt}a\ A\ {\isacharparenleft}{\kern0pt}m\ A\ p{\isacharparenright}{\kern0pt}\ {\isacharparenleft}{\kern0pt}n\ A\ p{\isacharparenright}{\kern0pt}{\isacharparenright}{\kern0pt}{\isachardoublequoteclose}\isanewline
\ \ \ \ \isacommand{using}\isamarkupfalse%
\ aggregator{\isacharunderscore}{\kern0pt}def\ combine{\isacharunderscore}{\kern0pt}ele{\isacharunderscore}{\kern0pt}rej{\isacharunderscore}{\kern0pt}def\ par{\isacharunderscore}{\kern0pt}comp{\isacharunderscore}{\kern0pt}result{\isacharunderscore}{\kern0pt}sound\isanewline
\ \ \ \ \ \ \ \ \ \ electoral{\isacharunderscore}{\kern0pt}module{\isacharunderscore}{\kern0pt}def\ mod{\isacharunderscore}{\kern0pt}m\ mod{\isacharunderscore}{\kern0pt}n\ fin{\isacharunderscore}{\kern0pt}A\ prof{\isacharunderscore}{\kern0pt}A\ agg{\isacharunderscore}{\kern0pt}a\isanewline
\ \ \ \ \isacommand{by}\isamarkupfalse%
\ {\isacharparenleft}{\kern0pt}smt\ {\isacharparenleft}{\kern0pt}verit{\isacharcomma}{\kern0pt}\ ccfv{\isacharunderscore}{\kern0pt}threshold{\isacharparenright}{\kern0pt}{\isacharparenright}{\kern0pt}\isanewline
\ \ \isacommand{thus}\isamarkupfalse%
\ {\isachardoublequoteopen}well{\isacharunderscore}{\kern0pt}formed\ A\ {\isacharparenleft}{\kern0pt}{\isacharparenleft}{\kern0pt}m\ {\isasymparallel}\isactrlsub a\ n{\isacharparenright}{\kern0pt}\ A\ p{\isacharparenright}{\kern0pt}{\isachardoublequoteclose}\isanewline
\ \ \ \ \isacommand{by}\isamarkupfalse%
\ simp\isanewline
\isacommand{qed}\isamarkupfalse%
%
\endisatagproof
{\isafoldproof}%
%
\isadelimproof
\isanewline
%
\endisadelimproof
%
\isadelimtheory
\isanewline
%
\endisadelimtheory
%
\isatagtheory
\isacommand{end}\isamarkupfalse%
%
\endisatagtheory
{\isafoldtheory}%
%
\isadelimtheory
%
\endisadelimtheory
%
\end{isabellebody}%
\endinput
%:%file=~/Documents/Studies/VotingRuleGenerator/virage/src/test/resources/verifiedVotingRuleConstruction/theories/Compositional_Framework/Components/Compositional_Structures/Parallel_Composition.thy%:%
%:%6=3%:%
%:%11=4%:%
%:%12=5%:%
%:%14=8%:%
%:%30=10%:%
%:%31=10%:%
%:%32=11%:%
%:%33=12%:%
%:%34=13%:%
%:%43=16%:%
%:%44=17%:%
%:%45=18%:%
%:%46=19%:%
%:%55=21%:%
%:%65=23%:%
%:%66=23%:%
%:%67=24%:%
%:%68=25%:%
%:%69=26%:%
%:%70=27%:%
%:%71=27%:%
%:%72=28%:%
%:%73=29%:%
%:%74=30%:%
%:%81=32%:%
%:%91=34%:%
%:%92=34%:%
%:%93=35%:%
%:%94=36%:%
%:%95=37%:%
%:%96=38%:%
%:%97=39%:%
%:%100=40%:%
%:%104=40%:%
%:%105=40%:%
%:%106=41%:%
%:%107=41%:%
%:%108=42%:%
%:%109=42%:%
%:%110=43%:%
%:%111=44%:%
%:%112=45%:%
%:%113=45%:%
%:%114=46%:%
%:%115=47%:%
%:%116=48%:%
%:%117=48%:%
%:%118=49%:%
%:%119=49%:%
%:%120=50%:%
%:%121=51%:%
%:%122=51%:%
%:%123=52%:%
%:%124=52%:%
%:%125=53%:%
%:%126=53%:%
%:%127=54%:%
%:%133=54%:%
%:%138=55%:%
%:%143=56%:%
%
\begin{isabellebody}%
\setisabellecontext{Loop{\isacharunderscore}{\kern0pt}Composition}%
%
\isadelimdocument
\isanewline
%
\endisadelimdocument
%
\isatagdocument
\isanewline
\isanewline
%
\isamarkupsection{Loop Composition%
}
\isamarkuptrue%
%
\endisatagdocument
{\isafolddocument}%
%
\isadelimdocument
%
\endisadelimdocument
%
\isadelimtheory
%
\endisadelimtheory
%
\isatagtheory
\isacommand{theory}\isamarkupfalse%
\ Loop{\isacharunderscore}{\kern0pt}Composition\isanewline
\ \ \isakeyword{imports}\ {\isachardoublequoteopen}{\isachardot}{\kern0pt}{\isachardot}{\kern0pt}{\isacharslash}{\kern0pt}Termination{\isacharunderscore}{\kern0pt}Condition{\isachardoublequoteclose}\isanewline
\ \ \ \ \ \ \ \ \ \ {\isachardoublequoteopen}{\isachardot}{\kern0pt}{\isachardot}{\kern0pt}{\isacharslash}{\kern0pt}Basic{\isacharunderscore}{\kern0pt}Modules{\isacharslash}{\kern0pt}Defer{\isacharunderscore}{\kern0pt}Module{\isachardoublequoteclose}\isanewline
\ \ \ \ \ \ \ \ \ \ Sequential{\isacharunderscore}{\kern0pt}Composition\isanewline
\isanewline
\isakeyword{begin}%
\endisatagtheory
{\isafoldtheory}%
%
\isadelimtheory
%
\endisadelimtheory
%
\begin{isamarkuptext}%
The loop composition uses the same module in sequence,
combined with a termination condition, until either
  (1) the termination condition is met or
  (2) no new decisions are made (i.e., a fixed point is reached).%
\end{isamarkuptext}\isamarkuptrue%
%
\isadelimdocument
%
\endisadelimdocument
%
\isatagdocument
%
\isamarkupsubsection{Definition%
}
\isamarkuptrue%
%
\endisatagdocument
{\isafolddocument}%
%
\isadelimdocument
%
\endisadelimdocument
\isacommand{lemma}\isamarkupfalse%
\ loop{\isacharunderscore}{\kern0pt}termination{\isacharunderscore}{\kern0pt}helper{\isacharcolon}{\kern0pt}\isanewline
\ \ \isakeyword{assumes}\isanewline
\ \ \ \ not{\isacharunderscore}{\kern0pt}term{\isacharcolon}{\kern0pt}\ {\isachardoublequoteopen}{\isasymnot}t\ {\isacharparenleft}{\kern0pt}acc\ A\ p{\isacharparenright}{\kern0pt}{\isachardoublequoteclose}\ \isakeyword{and}\isanewline
\ \ \ \ subset{\isacharcolon}{\kern0pt}\ {\isachardoublequoteopen}defer\ {\isacharparenleft}{\kern0pt}acc\ {\isasymtriangleright}\ m{\isacharparenright}{\kern0pt}\ A\ p\ {\isasymsubset}\ defer\ acc\ A\ p{\isachardoublequoteclose}\ \isakeyword{and}\isanewline
\ \ \ \ not{\isacharunderscore}{\kern0pt}inf{\isacharcolon}{\kern0pt}\ {\isachardoublequoteopen}{\isasymnot}infinite\ {\isacharparenleft}{\kern0pt}defer\ acc\ A\ p{\isacharparenright}{\kern0pt}{\isachardoublequoteclose}\isanewline
\ \ \isakeyword{shows}\isanewline
\ \ \ \ {\isachardoublequoteopen}{\isacharparenleft}{\kern0pt}{\isacharparenleft}{\kern0pt}acc\ {\isasymtriangleright}\ m{\isacharcomma}{\kern0pt}\ m{\isacharcomma}{\kern0pt}\ t{\isacharcomma}{\kern0pt}\ A{\isacharcomma}{\kern0pt}\ p{\isacharparenright}{\kern0pt}{\isacharcomma}{\kern0pt}\ {\isacharparenleft}{\kern0pt}acc{\isacharcomma}{\kern0pt}\ m{\isacharcomma}{\kern0pt}\ t{\isacharcomma}{\kern0pt}\ A{\isacharcomma}{\kern0pt}\ p{\isacharparenright}{\kern0pt}{\isacharparenright}{\kern0pt}\ {\isasymin}\isanewline
\ \ \ \ \ \ \ \ measure\ {\isacharparenleft}{\kern0pt}{\isasymlambda}{\isacharparenleft}{\kern0pt}acc{\isacharcomma}{\kern0pt}\ m{\isacharcomma}{\kern0pt}\ t{\isacharcomma}{\kern0pt}\ A{\isacharcomma}{\kern0pt}\ p{\isacharparenright}{\kern0pt}{\isachardot}{\kern0pt}\ card\ {\isacharparenleft}{\kern0pt}defer\ acc\ A\ p{\isacharparenright}{\kern0pt}{\isacharparenright}{\kern0pt}{\isachardoublequoteclose}\isanewline
%
\isadelimproof
\ \ %
\endisadelimproof
%
\isatagproof
\isacommand{using}\isamarkupfalse%
\ assms\ psubset{\isacharunderscore}{\kern0pt}card{\isacharunderscore}{\kern0pt}mono\isanewline
\ \ \isacommand{by}\isamarkupfalse%
\ auto%
\endisatagproof
{\isafoldproof}%
%
\isadelimproof
\isanewline
%
\endisadelimproof
\isanewline
\isanewline
\isacommand{function}\isamarkupfalse%
\ loop{\isacharunderscore}{\kern0pt}comp{\isacharunderscore}{\kern0pt}helper\ {\isacharcolon}{\kern0pt}{\isacharcolon}{\kern0pt}\isanewline
\ \ \ \ {\isachardoublequoteopen}{\isacharprime}{\kern0pt}a\ Electoral{\isacharunderscore}{\kern0pt}Module\ {\isasymRightarrow}\ {\isacharprime}{\kern0pt}a\ Electoral{\isacharunderscore}{\kern0pt}Module\ {\isasymRightarrow}\isanewline
\ \ \ \ \ \ \ \ {\isacharprime}{\kern0pt}a\ Termination{\isacharunderscore}{\kern0pt}Condition\ {\isasymRightarrow}\ {\isacharprime}{\kern0pt}a\ Electoral{\isacharunderscore}{\kern0pt}Module{\isachardoublequoteclose}\ \isakeyword{where}\isanewline
\ \ {\isachardoublequoteopen}t\ {\isacharparenleft}{\kern0pt}acc\ A\ p{\isacharparenright}{\kern0pt}\ {\isasymor}\ {\isasymnot}{\isacharparenleft}{\kern0pt}{\isacharparenleft}{\kern0pt}defer\ {\isacharparenleft}{\kern0pt}acc\ {\isasymtriangleright}\ m{\isacharparenright}{\kern0pt}\ A\ p{\isacharparenright}{\kern0pt}\ {\isasymsubset}\ {\isacharparenleft}{\kern0pt}defer\ acc\ A\ p{\isacharparenright}{\kern0pt}{\isacharparenright}{\kern0pt}\ {\isasymor}\isanewline
\ \ \ \ infinite\ {\isacharparenleft}{\kern0pt}defer\ acc\ A\ p{\isacharparenright}{\kern0pt}\ {\isasymLongrightarrow}\isanewline
\ \ \ \ \ \ loop{\isacharunderscore}{\kern0pt}comp{\isacharunderscore}{\kern0pt}helper\ acc\ m\ t\ A\ p\ {\isacharequal}{\kern0pt}\ acc\ A\ p{\isachardoublequoteclose}\ {\isacharbar}{\kern0pt}\isanewline
\ \ {\isachardoublequoteopen}{\isasymnot}{\isacharparenleft}{\kern0pt}t\ {\isacharparenleft}{\kern0pt}acc\ A\ p{\isacharparenright}{\kern0pt}\ {\isasymor}\ {\isasymnot}{\isacharparenleft}{\kern0pt}{\isacharparenleft}{\kern0pt}defer\ {\isacharparenleft}{\kern0pt}acc\ {\isasymtriangleright}\ m{\isacharparenright}{\kern0pt}\ A\ p{\isacharparenright}{\kern0pt}\ {\isasymsubset}\ {\isacharparenleft}{\kern0pt}defer\ acc\ A\ p{\isacharparenright}{\kern0pt}{\isacharparenright}{\kern0pt}\ {\isasymor}\isanewline
\ \ \ \ infinite\ {\isacharparenleft}{\kern0pt}defer\ acc\ A\ p{\isacharparenright}{\kern0pt}{\isacharparenright}{\kern0pt}\ {\isasymLongrightarrow}\isanewline
\ \ \ \ \ \ loop{\isacharunderscore}{\kern0pt}comp{\isacharunderscore}{\kern0pt}helper\ acc\ m\ t\ A\ p\ {\isacharequal}{\kern0pt}\ loop{\isacharunderscore}{\kern0pt}comp{\isacharunderscore}{\kern0pt}helper\ {\isacharparenleft}{\kern0pt}acc\ {\isasymtriangleright}\ m{\isacharparenright}{\kern0pt}\ m\ t\ A\ p{\isachardoublequoteclose}\isanewline
%
\isadelimproof
%
\endisadelimproof
%
\isatagproof
\isacommand{proof}\isamarkupfalse%
\ {\isacharminus}{\kern0pt}\isanewline
\ \ \isacommand{fix}\isamarkupfalse%
\isanewline
\ \ \ \ P\ {\isacharcolon}{\kern0pt}{\isacharcolon}{\kern0pt}\ bool\ \isakeyword{and}\isanewline
\ \ \ \ x\ {\isacharcolon}{\kern0pt}{\isacharcolon}{\kern0pt}\ {\isachardoublequoteopen}{\isacharparenleft}{\kern0pt}{\isacharprime}{\kern0pt}a\ Electoral{\isacharunderscore}{\kern0pt}Module{\isacharparenright}{\kern0pt}\ {\isasymtimes}\ {\isacharparenleft}{\kern0pt}{\isacharprime}{\kern0pt}a\ Electoral{\isacharunderscore}{\kern0pt}Module{\isacharparenright}{\kern0pt}\ {\isasymtimes}\isanewline
\ \ \ \ \ \ \ \ \ \ {\isacharparenleft}{\kern0pt}{\isacharprime}{\kern0pt}a\ Termination{\isacharunderscore}{\kern0pt}Condition{\isacharparenright}{\kern0pt}\ {\isasymtimes}\ {\isacharprime}{\kern0pt}a\ set\ {\isasymtimes}\ {\isacharprime}{\kern0pt}a\ Profile{\isachardoublequoteclose}\isanewline
\ \ \isacommand{assume}\isamarkupfalse%
\isanewline
\ \ \ \ a{\isadigit{1}}{\isacharcolon}{\kern0pt}\ {\isachardoublequoteopen}{\isasymAnd}t\ acc\ A\ p\ m{\isachardot}{\kern0pt}\isanewline
\ \ \ \ \ \ \ \ \ \ {\isasymlbrakk}t\ {\isacharparenleft}{\kern0pt}acc\ A\ p{\isacharparenright}{\kern0pt}\ {\isasymor}\ {\isasymnot}\ defer\ {\isacharparenleft}{\kern0pt}acc\ {\isasymtriangleright}\ m{\isacharparenright}{\kern0pt}\ A\ p\ {\isasymsubset}\ defer\ acc\ A\ p\ {\isasymor}\isanewline
\ \ \ \ \ \ \ \ \ \ \ \ \ \ infinite\ {\isacharparenleft}{\kern0pt}defer\ acc\ A\ p{\isacharparenright}{\kern0pt}{\isacharsemicolon}{\kern0pt}\isanewline
\ \ \ \ \ \ \ \ \ \ \ \ x\ {\isacharequal}{\kern0pt}\ {\isacharparenleft}{\kern0pt}acc{\isacharcomma}{\kern0pt}\ m{\isacharcomma}{\kern0pt}\ t{\isacharcomma}{\kern0pt}\ A{\isacharcomma}{\kern0pt}\ p{\isacharparenright}{\kern0pt}{\isasymrbrakk}\ {\isasymLongrightarrow}\ P{\isachardoublequoteclose}\ \isakeyword{and}\isanewline
\ \ \ \ a{\isadigit{2}}{\isacharcolon}{\kern0pt}\ {\isachardoublequoteopen}{\isasymAnd}t\ acc\ A\ p\ m{\isachardot}{\kern0pt}\isanewline
\ \ \ \ \ \ \ \ \ \ {\isasymlbrakk}{\isasymnot}\ {\isacharparenleft}{\kern0pt}t\ {\isacharparenleft}{\kern0pt}acc\ A\ p{\isacharparenright}{\kern0pt}\ {\isasymor}\ {\isasymnot}\ defer\ {\isacharparenleft}{\kern0pt}acc\ {\isasymtriangleright}\ m{\isacharparenright}{\kern0pt}\ A\ p\ {\isasymsubset}\ defer\ acc\ A\ p\ {\isasymor}\isanewline
\ \ \ \ \ \ \ \ \ \ \ \ \ \ infinite\ {\isacharparenleft}{\kern0pt}defer\ acc\ A\ p{\isacharparenright}{\kern0pt}{\isacharparenright}{\kern0pt}{\isacharsemicolon}{\kern0pt}\isanewline
\ \ \ \ \ \ \ \ \ \ \ \ x\ {\isacharequal}{\kern0pt}\ {\isacharparenleft}{\kern0pt}acc{\isacharcomma}{\kern0pt}\ m{\isacharcomma}{\kern0pt}\ t{\isacharcomma}{\kern0pt}\ A{\isacharcomma}{\kern0pt}\ p{\isacharparenright}{\kern0pt}{\isasymrbrakk}\ {\isasymLongrightarrow}\ P{\isachardoublequoteclose}\isanewline
\ \ \isacommand{have}\isamarkupfalse%
\ {\isachardoublequoteopen}{\isasymexists}f\ A\ p\ rs\ fa{\isachardot}{\kern0pt}\ {\isacharparenleft}{\kern0pt}fa{\isacharcomma}{\kern0pt}\ f{\isacharcomma}{\kern0pt}\ p{\isacharcomma}{\kern0pt}\ A{\isacharcomma}{\kern0pt}\ rs{\isacharparenright}{\kern0pt}\ {\isacharequal}{\kern0pt}\ x{\isachardoublequoteclose}\isanewline
\ \ \ \ \isacommand{using}\isamarkupfalse%
\ prod{\isacharunderscore}{\kern0pt}cases{\isadigit{5}}\isanewline
\ \ \ \ \isacommand{by}\isamarkupfalse%
\ metis\isanewline
\ \ \isacommand{then}\isamarkupfalse%
\ \isacommand{show}\isamarkupfalse%
\ P\isanewline
\ \ \ \ \isacommand{using}\isamarkupfalse%
\ a{\isadigit{2}}\ a{\isadigit{1}}\isanewline
\ \ \ \ \isacommand{by}\isamarkupfalse%
\ {\isacharparenleft}{\kern0pt}metis\ {\isacharparenleft}{\kern0pt}no{\isacharunderscore}{\kern0pt}types{\isacharparenright}{\kern0pt}{\isacharparenright}{\kern0pt}\isanewline
\isacommand{next}\isamarkupfalse%
\isanewline
\ \ \isacommand{show}\isamarkupfalse%
\isanewline
\ \ \ \ {\isachardoublequoteopen}{\isasymAnd}t\ acc\ A\ p\ m\ ta\ acca\ Aa\ pa\ ma{\isachardot}{\kern0pt}\isanewline
\ \ \ \ \ \ \ t\ {\isacharparenleft}{\kern0pt}acc\ A\ p{\isacharparenright}{\kern0pt}\ {\isasymor}\ {\isasymnot}\ defer\ {\isacharparenleft}{\kern0pt}acc\ {\isasymtriangleright}\ m{\isacharparenright}{\kern0pt}\ A\ p\ {\isasymsubset}\ defer\ acc\ A\ p\ {\isasymor}\isanewline
\ \ \ \ \ \ \ \ infinite\ {\isacharparenleft}{\kern0pt}defer\ acc\ A\ p{\isacharparenright}{\kern0pt}\ {\isasymLongrightarrow}\isanewline
\ \ \ \ \ \ \ \ \ \ ta\ {\isacharparenleft}{\kern0pt}acca\ Aa\ pa{\isacharparenright}{\kern0pt}\ {\isasymor}\ {\isasymnot}\ defer\ {\isacharparenleft}{\kern0pt}acca\ {\isasymtriangleright}\ ma{\isacharparenright}{\kern0pt}\ Aa\ pa\ {\isasymsubset}\ defer\ acca\ Aa\ pa\ {\isasymor}\isanewline
\ \ \ \ \ \ \ \ \ \ infinite\ {\isacharparenleft}{\kern0pt}defer\ acca\ Aa\ pa{\isacharparenright}{\kern0pt}\ {\isasymLongrightarrow}\isanewline
\ \ \ \ \ \ \ \ \ \ \ {\isacharparenleft}{\kern0pt}acc{\isacharcomma}{\kern0pt}\ m{\isacharcomma}{\kern0pt}\ t{\isacharcomma}{\kern0pt}\ A{\isacharcomma}{\kern0pt}\ p{\isacharparenright}{\kern0pt}\ {\isacharequal}{\kern0pt}\ {\isacharparenleft}{\kern0pt}acca{\isacharcomma}{\kern0pt}\ ma{\isacharcomma}{\kern0pt}\ ta{\isacharcomma}{\kern0pt}\ Aa{\isacharcomma}{\kern0pt}\ pa{\isacharparenright}{\kern0pt}\ {\isasymLongrightarrow}\isanewline
\ \ \ \ \ \ \ \ \ \ \ \ \ \ acc\ A\ p\ {\isacharequal}{\kern0pt}\ acca\ Aa\ pa{\isachardoublequoteclose}\isanewline
\ \ \ \ \isacommand{by}\isamarkupfalse%
\ fastforce\isanewline
\isacommand{next}\isamarkupfalse%
\isanewline
\ \ \isacommand{show}\isamarkupfalse%
\isanewline
\ \ \ \ {\isachardoublequoteopen}{\isasymAnd}t\ acc\ A\ p\ m\ ta\ acca\ Aa\ pa\ ma{\isachardot}{\kern0pt}\isanewline
\ \ \ \ \ \ \ t\ {\isacharparenleft}{\kern0pt}acc\ A\ p{\isacharparenright}{\kern0pt}\ {\isasymor}\ {\isasymnot}\ defer\ {\isacharparenleft}{\kern0pt}acc\ {\isasymtriangleright}\ m{\isacharparenright}{\kern0pt}\ A\ p\ {\isasymsubset}\ defer\ acc\ A\ p\ {\isasymor}\isanewline
\ \ \ \ \ \ \ \ infinite\ {\isacharparenleft}{\kern0pt}defer\ acc\ A\ p{\isacharparenright}{\kern0pt}\ {\isasymLongrightarrow}\isanewline
\ \ \ \ \ \ \ \ \ \ {\isasymnot}\ {\isacharparenleft}{\kern0pt}ta\ {\isacharparenleft}{\kern0pt}acca\ Aa\ pa{\isacharparenright}{\kern0pt}\ {\isasymor}\ {\isasymnot}\ defer\ {\isacharparenleft}{\kern0pt}acca\ {\isasymtriangleright}\ ma{\isacharparenright}{\kern0pt}\ Aa\ pa\ {\isasymsubset}\ defer\ acca\ Aa\ pa\ {\isasymor}\isanewline
\ \ \ \ \ \ \ \ \ \ infinite\ {\isacharparenleft}{\kern0pt}defer\ acca\ Aa\ pa{\isacharparenright}{\kern0pt}{\isacharparenright}{\kern0pt}\ {\isasymLongrightarrow}\isanewline
\ \ \ \ \ \ \ \ \ \ \ {\isacharparenleft}{\kern0pt}acc{\isacharcomma}{\kern0pt}\ m{\isacharcomma}{\kern0pt}\ t{\isacharcomma}{\kern0pt}\ A{\isacharcomma}{\kern0pt}\ p{\isacharparenright}{\kern0pt}\ {\isacharequal}{\kern0pt}\ {\isacharparenleft}{\kern0pt}acca{\isacharcomma}{\kern0pt}\ ma{\isacharcomma}{\kern0pt}\ ta{\isacharcomma}{\kern0pt}\ Aa{\isacharcomma}{\kern0pt}\ pa{\isacharparenright}{\kern0pt}\ {\isasymLongrightarrow}\isanewline
\ \ \ \ \ \ \ \ \ \ \ \ \ \ acc\ A\ p\ {\isacharequal}{\kern0pt}\ loop{\isacharunderscore}{\kern0pt}comp{\isacharunderscore}{\kern0pt}helper{\isacharunderscore}{\kern0pt}sumC\ {\isacharparenleft}{\kern0pt}acca\ {\isasymtriangleright}\ ma{\isacharcomma}{\kern0pt}\ ma{\isacharcomma}{\kern0pt}\ ta{\isacharcomma}{\kern0pt}\ Aa{\isacharcomma}{\kern0pt}\ pa{\isacharparenright}{\kern0pt}{\isachardoublequoteclose}\isanewline
\ \ \isacommand{proof}\isamarkupfalse%
\ {\isacharminus}{\kern0pt}\isanewline
\ \ \ \ \isacommand{fix}\isamarkupfalse%
\isanewline
\ \ \ \ \ \ t\ {\isacharcolon}{\kern0pt}{\isacharcolon}{\kern0pt}\ {\isachardoublequoteopen}{\isacharprime}{\kern0pt}a\ Termination{\isacharunderscore}{\kern0pt}Condition{\isachardoublequoteclose}\ \isakeyword{and}\isanewline
\ \ \ \ \ \ acc\ {\isacharcolon}{\kern0pt}{\isacharcolon}{\kern0pt}\ {\isachardoublequoteopen}{\isacharprime}{\kern0pt}a\ Electoral{\isacharunderscore}{\kern0pt}Module{\isachardoublequoteclose}\ \isakeyword{and}\isanewline
\ \ \ \ \ \ A\ {\isacharcolon}{\kern0pt}{\isacharcolon}{\kern0pt}\ {\isachardoublequoteopen}{\isacharprime}{\kern0pt}a\ set{\isachardoublequoteclose}\ \isakeyword{and}\isanewline
\ \ \ \ \ \ p\ {\isacharcolon}{\kern0pt}{\isacharcolon}{\kern0pt}\ {\isachardoublequoteopen}{\isacharprime}{\kern0pt}a\ Profile{\isachardoublequoteclose}\ \isakeyword{and}\isanewline
\ \ \ \ \ \ m\ {\isacharcolon}{\kern0pt}{\isacharcolon}{\kern0pt}\ {\isachardoublequoteopen}{\isacharprime}{\kern0pt}a\ Electoral{\isacharunderscore}{\kern0pt}Module{\isachardoublequoteclose}\ \isakeyword{and}\isanewline
\ \ \ \ \ \ ta\ {\isacharcolon}{\kern0pt}{\isacharcolon}{\kern0pt}\ {\isachardoublequoteopen}{\isacharprime}{\kern0pt}a\ Termination{\isacharunderscore}{\kern0pt}Condition{\isachardoublequoteclose}\ \isakeyword{and}\isanewline
\ \ \ \ \ \ acca\ {\isacharcolon}{\kern0pt}{\isacharcolon}{\kern0pt}\ {\isachardoublequoteopen}{\isacharprime}{\kern0pt}a\ Electoral{\isacharunderscore}{\kern0pt}Module{\isachardoublequoteclose}\ \isakeyword{and}\isanewline
\ \ \ \ \ \ Aa\ {\isacharcolon}{\kern0pt}{\isacharcolon}{\kern0pt}\ {\isachardoublequoteopen}{\isacharprime}{\kern0pt}a\ set{\isachardoublequoteclose}\ \isakeyword{and}\isanewline
\ \ \ \ \ \ pa\ {\isacharcolon}{\kern0pt}{\isacharcolon}{\kern0pt}\ {\isachardoublequoteopen}{\isacharprime}{\kern0pt}a\ Profile{\isachardoublequoteclose}\ \isakeyword{and}\isanewline
\ \ \ \ \ \ ma\ {\isacharcolon}{\kern0pt}{\isacharcolon}{\kern0pt}\ {\isachardoublequoteopen}{\isacharprime}{\kern0pt}a\ Electoral{\isacharunderscore}{\kern0pt}Module{\isachardoublequoteclose}\isanewline
\ \ \ \ \isacommand{assume}\isamarkupfalse%
\isanewline
\ \ \ \ \ \ a{\isadigit{1}}{\isacharcolon}{\kern0pt}\ {\isachardoublequoteopen}t\ {\isacharparenleft}{\kern0pt}acc\ A\ p{\isacharparenright}{\kern0pt}\ {\isasymor}\ {\isasymnot}\ defer\ {\isacharparenleft}{\kern0pt}acc\ {\isasymtriangleright}\ m{\isacharparenright}{\kern0pt}\ A\ p\ {\isasymsubset}\ defer\ acc\ A\ p\ {\isasymor}\isanewline
\ \ \ \ \ \ \ \ \ \ \ \ infinite\ {\isacharparenleft}{\kern0pt}defer\ acc\ A\ p{\isacharparenright}{\kern0pt}{\isachardoublequoteclose}\ \isakeyword{and}\isanewline
\ \ \ \ \ \ a{\isadigit{2}}{\isacharcolon}{\kern0pt}\ {\isachardoublequoteopen}{\isasymnot}\ {\isacharparenleft}{\kern0pt}ta\ {\isacharparenleft}{\kern0pt}acca\ Aa\ pa{\isacharparenright}{\kern0pt}\ {\isasymor}\ {\isasymnot}\ defer\ {\isacharparenleft}{\kern0pt}acca\ {\isasymtriangleright}\ ma{\isacharparenright}{\kern0pt}\ Aa\ pa\ {\isasymsubset}\ defer\ acca\ Aa\ pa\ {\isasymor}\isanewline
\ \ \ \ \ \ \ \ \ \ \ \ infinite\ {\isacharparenleft}{\kern0pt}defer\ acca\ Aa\ pa{\isacharparenright}{\kern0pt}{\isacharparenright}{\kern0pt}{\isachardoublequoteclose}\ \isakeyword{and}\isanewline
\ \ \ \ \ \ {\isachardoublequoteopen}{\isacharparenleft}{\kern0pt}acc{\isacharcomma}{\kern0pt}\ m{\isacharcomma}{\kern0pt}\ t{\isacharcomma}{\kern0pt}\ A{\isacharcomma}{\kern0pt}\ p{\isacharparenright}{\kern0pt}\ {\isacharequal}{\kern0pt}\ {\isacharparenleft}{\kern0pt}acca{\isacharcomma}{\kern0pt}\ ma{\isacharcomma}{\kern0pt}\ ta{\isacharcomma}{\kern0pt}\ Aa{\isacharcomma}{\kern0pt}\ pa{\isacharparenright}{\kern0pt}{\isachardoublequoteclose}\isanewline
\ \ \ \ \isacommand{hence}\isamarkupfalse%
\ False\isanewline
\ \ \ \ \ \ \isacommand{using}\isamarkupfalse%
\ a{\isadigit{2}}\ a{\isadigit{1}}\isanewline
\ \ \ \ \ \ \isacommand{by}\isamarkupfalse%
\ force\isanewline
\ \ \isacommand{thus}\isamarkupfalse%
\ {\isachardoublequoteopen}acc\ A\ p\ {\isacharequal}{\kern0pt}\ loop{\isacharunderscore}{\kern0pt}comp{\isacharunderscore}{\kern0pt}helper{\isacharunderscore}{\kern0pt}sumC\ {\isacharparenleft}{\kern0pt}acca\ {\isasymtriangleright}\ ma{\isacharcomma}{\kern0pt}\ ma{\isacharcomma}{\kern0pt}\ ta{\isacharcomma}{\kern0pt}\ Aa{\isacharcomma}{\kern0pt}\ pa{\isacharparenright}{\kern0pt}{\isachardoublequoteclose}\isanewline
\ \ \ \ \isacommand{by}\isamarkupfalse%
\ auto\isanewline
\isacommand{qed}\isamarkupfalse%
\isanewline
\isacommand{next}\isamarkupfalse%
\isanewline
\ \ \isacommand{show}\isamarkupfalse%
\isanewline
\ \ \ \ {\isachardoublequoteopen}{\isasymAnd}t\ acc\ A\ p\ m\ ta\ acca\ Aa\ pa\ ma{\isachardot}{\kern0pt}\isanewline
\ \ \ \ \ \ \ {\isasymnot}\ {\isacharparenleft}{\kern0pt}t\ {\isacharparenleft}{\kern0pt}acc\ A\ p{\isacharparenright}{\kern0pt}\ {\isasymor}\ {\isasymnot}\ defer\ {\isacharparenleft}{\kern0pt}acc\ {\isasymtriangleright}\ m{\isacharparenright}{\kern0pt}\ A\ p\ {\isasymsubset}\ defer\ acc\ A\ p\ {\isasymor}\isanewline
\ \ \ \ \ \ \ \ \ \ infinite\ {\isacharparenleft}{\kern0pt}defer\ acc\ A\ p{\isacharparenright}{\kern0pt}{\isacharparenright}{\kern0pt}\ {\isasymLongrightarrow}\isanewline
\ \ \ \ \ \ \ \ \ \ \ {\isasymnot}\ {\isacharparenleft}{\kern0pt}ta\ {\isacharparenleft}{\kern0pt}acca\ Aa\ pa{\isacharparenright}{\kern0pt}\ {\isasymor}\ {\isasymnot}\ defer\ {\isacharparenleft}{\kern0pt}acca\ {\isasymtriangleright}\ ma{\isacharparenright}{\kern0pt}\ Aa\ pa\ {\isasymsubset}\ defer\ acca\ Aa\ pa\ {\isasymor}\isanewline
\ \ \ \ \ \ \ \ \ \ \ \ infinite\ {\isacharparenleft}{\kern0pt}defer\ acca\ Aa\ pa{\isacharparenright}{\kern0pt}{\isacharparenright}{\kern0pt}\ {\isasymLongrightarrow}\isanewline
\ \ \ \ \ \ \ \ \ \ \ \ \ {\isacharparenleft}{\kern0pt}acc{\isacharcomma}{\kern0pt}\ m{\isacharcomma}{\kern0pt}\ t{\isacharcomma}{\kern0pt}\ A{\isacharcomma}{\kern0pt}\ p{\isacharparenright}{\kern0pt}\ {\isacharequal}{\kern0pt}\ {\isacharparenleft}{\kern0pt}acca{\isacharcomma}{\kern0pt}\ ma{\isacharcomma}{\kern0pt}\ ta{\isacharcomma}{\kern0pt}\ Aa{\isacharcomma}{\kern0pt}\ pa{\isacharparenright}{\kern0pt}\ {\isasymLongrightarrow}\isanewline
\ \ \ \ \ \ \ \ \ \ \ \ \ \ \ \ loop{\isacharunderscore}{\kern0pt}comp{\isacharunderscore}{\kern0pt}helper{\isacharunderscore}{\kern0pt}sumC\ {\isacharparenleft}{\kern0pt}acc\ {\isasymtriangleright}\ m{\isacharcomma}{\kern0pt}\ m{\isacharcomma}{\kern0pt}\ t{\isacharcomma}{\kern0pt}\ A{\isacharcomma}{\kern0pt}\ p{\isacharparenright}{\kern0pt}\ {\isacharequal}{\kern0pt}\isanewline
\ \ \ \ \ \ \ \ \ \ \ \ \ \ \ \ \ \ loop{\isacharunderscore}{\kern0pt}comp{\isacharunderscore}{\kern0pt}helper{\isacharunderscore}{\kern0pt}sumC\ {\isacharparenleft}{\kern0pt}acca\ {\isasymtriangleright}\ ma{\isacharcomma}{\kern0pt}\ ma{\isacharcomma}{\kern0pt}\ ta{\isacharcomma}{\kern0pt}\ Aa{\isacharcomma}{\kern0pt}\ pa{\isacharparenright}{\kern0pt}{\isachardoublequoteclose}\isanewline
\ \ \ \ \isacommand{by}\isamarkupfalse%
\ force\isanewline
\isacommand{qed}\isamarkupfalse%
%
\endisatagproof
{\isafoldproof}%
%
\isadelimproof
\isanewline
%
\endisadelimproof
\isacommand{termination}\isamarkupfalse%
\isanewline
%
\isadelimproof
%
\endisadelimproof
%
\isatagproof
\isacommand{proof}\isamarkupfalse%
\ {\isacharminus}{\kern0pt}\isanewline
\ \ \isacommand{have}\isamarkupfalse%
\ f{\isadigit{0}}{\isacharcolon}{\kern0pt}\isanewline
\ \ \ \ {\isachardoublequoteopen}{\isasymexists}r{\isachardot}{\kern0pt}\ wf\ r\ {\isasymand}\isanewline
\ \ \ \ \ \ \ \ {\isacharparenleft}{\kern0pt}{\isasymforall}p\ f\ A\ rs\ fa{\isachardot}{\kern0pt}\isanewline
\ \ \ \ \ \ \ \ \ \ p\ {\isacharparenleft}{\kern0pt}f\ {\isacharparenleft}{\kern0pt}A{\isacharcolon}{\kern0pt}{\isacharcolon}{\kern0pt}{\isacharprime}{\kern0pt}a\ set{\isacharparenright}{\kern0pt}\ rs{\isacharparenright}{\kern0pt}\ {\isasymor}\isanewline
\ \ \ \ \ \ \ \ \ \ {\isasymnot}\ defer\ {\isacharparenleft}{\kern0pt}f\ {\isasymtriangleright}\ fa{\isacharparenright}{\kern0pt}\ A\ rs\ {\isasymsubset}\ defer\ f\ A\ rs\ {\isasymor}\isanewline
\ \ \ \ \ \ \ \ \ \ infinite\ {\isacharparenleft}{\kern0pt}defer\ f\ A\ rs{\isacharparenright}{\kern0pt}\ {\isasymor}\isanewline
\ \ \ \ \ \ \ \ \ \ {\isacharparenleft}{\kern0pt}{\isacharparenleft}{\kern0pt}f\ {\isasymtriangleright}\ fa{\isacharcomma}{\kern0pt}\ fa{\isacharcomma}{\kern0pt}\ p{\isacharcomma}{\kern0pt}\ A{\isacharcomma}{\kern0pt}\ rs{\isacharparenright}{\kern0pt}{\isacharcomma}{\kern0pt}\ {\isacharparenleft}{\kern0pt}f{\isacharcomma}{\kern0pt}\ fa{\isacharcomma}{\kern0pt}\ p{\isacharcomma}{\kern0pt}\ A{\isacharcomma}{\kern0pt}\ rs{\isacharparenright}{\kern0pt}{\isacharparenright}{\kern0pt}\ {\isasymin}\ r{\isacharparenright}{\kern0pt}{\isachardoublequoteclose}\isanewline
\ \ \ \ \isacommand{using}\isamarkupfalse%
\ loop{\isacharunderscore}{\kern0pt}termination{\isacharunderscore}{\kern0pt}helper\ wf{\isacharunderscore}{\kern0pt}measure\ {\isachardoublequoteopen}termination{\isachardoublequoteclose}\isanewline
\ \ \ \ \isacommand{by}\isamarkupfalse%
\ {\isacharparenleft}{\kern0pt}metis\ {\isacharparenleft}{\kern0pt}no{\isacharunderscore}{\kern0pt}types{\isacharparenright}{\kern0pt}{\isacharparenright}{\kern0pt}\isanewline
\ \ \isacommand{hence}\isamarkupfalse%
\isanewline
\ \ \ \ {\isachardoublequoteopen}{\isasymforall}r\ p{\isachardot}{\kern0pt}\isanewline
\ \ \ \ \ \ Ex\ {\isacharparenleft}{\kern0pt}{\isacharparenleft}{\kern0pt}{\isasymlambda}ra{\isachardot}{\kern0pt}\ {\isasymforall}f\ A\ rs\ pa\ fa{\isachardot}{\kern0pt}\ {\isasymexists}ra\ pb\ rb\ pc\ pd\ fb\ Aa\ rsa\ fc\ pe{\isachardot}{\kern0pt}\isanewline
\ \ \ \ \ \ \ \ {\isasymnot}\ wf\ r\ {\isasymor}\isanewline
\ \ \ \ \ \ \ \ \ \ loop{\isacharunderscore}{\kern0pt}comp{\isacharunderscore}{\kern0pt}helper{\isacharunderscore}{\kern0pt}dom\isanewline
\ \ \ \ \ \ \ \ \ \ \ \ {\isacharparenleft}{\kern0pt}p{\isacharcolon}{\kern0pt}{\isacharcolon}{\kern0pt}{\isacharparenleft}{\kern0pt}{\isacharprime}{\kern0pt}a\ Electoral{\isacharunderscore}{\kern0pt}Module{\isacharparenright}{\kern0pt}\ {\isasymtimes}\ {\isacharparenleft}{\kern0pt}{\isacharunderscore}{\kern0pt}\ Electoral{\isacharunderscore}{\kern0pt}Module{\isacharparenright}{\kern0pt}\ {\isasymtimes}\isanewline
\ \ \ \ \ \ \ \ \ \ \ \ \ \ {\isacharparenleft}{\kern0pt}{\isacharunderscore}{\kern0pt}\ Termination{\isacharunderscore}{\kern0pt}Condition{\isacharparenright}{\kern0pt}\ {\isasymtimes}\ {\isacharunderscore}{\kern0pt}\ set\ {\isasymtimes}\ {\isacharunderscore}{\kern0pt}\ Profile{\isacharparenright}{\kern0pt}\ {\isasymor}\isanewline
\ \ \ \ \ \ \ \ \ \ infinite\ {\isacharparenleft}{\kern0pt}defer\ f\ {\isacharparenleft}{\kern0pt}A{\isacharcolon}{\kern0pt}{\isacharcolon}{\kern0pt}{\isacharprime}{\kern0pt}a\ set{\isacharparenright}{\kern0pt}\ rs{\isacharparenright}{\kern0pt}\ {\isasymor}\isanewline
\ \ \ \ \ \ \ \ \ \ pa\ {\isacharparenleft}{\kern0pt}f\ A\ rs{\isacharparenright}{\kern0pt}\ {\isasymand}\isanewline
\ \ \ \ \ \ \ \ \ \ \ \ wf\isanewline
\ \ \ \ \ \ \ \ \ \ \ \ \ \ {\isacharparenleft}{\kern0pt}ra{\isacharcolon}{\kern0pt}{\isacharcolon}{\kern0pt}{\isacharparenleft}{\kern0pt}{\isacharparenleft}{\kern0pt}\isanewline
\ \ \ \ \ \ \ \ \ \ \ \ \ \ \ \ {\isacharparenleft}{\kern0pt}{\isacharprime}{\kern0pt}a\ Electoral{\isacharunderscore}{\kern0pt}Module{\isacharparenright}{\kern0pt}\ {\isasymtimes}\ {\isacharparenleft}{\kern0pt}{\isacharprime}{\kern0pt}a\ Electoral{\isacharunderscore}{\kern0pt}Module{\isacharparenright}{\kern0pt}\ {\isasymtimes}\isanewline
\ \ \ \ \ \ \ \ \ \ \ \ \ \ \ \ {\isacharparenleft}{\kern0pt}{\isacharprime}{\kern0pt}a\ Termination{\isacharunderscore}{\kern0pt}Condition{\isacharparenright}{\kern0pt}\ {\isasymtimes}\ {\isacharprime}{\kern0pt}a\ set\ {\isasymtimes}\ {\isacharprime}{\kern0pt}a\ Profile{\isacharparenright}{\kern0pt}\ {\isasymtimes}\ {\isacharunderscore}{\kern0pt}{\isacharparenright}{\kern0pt}\ set{\isacharparenright}{\kern0pt}\ {\isasymand}\isanewline
\ \ \ \ \ \ \ \ \ \ \ \ {\isasymnot}\ loop{\isacharunderscore}{\kern0pt}comp{\isacharunderscore}{\kern0pt}helper{\isacharunderscore}{\kern0pt}dom\ {\isacharparenleft}{\kern0pt}pb{\isacharcolon}{\kern0pt}{\isacharcolon}{\kern0pt}\isanewline
\ \ \ \ \ \ \ \ \ \ \ \ \ \ \ \ {\isacharparenleft}{\kern0pt}{\isacharprime}{\kern0pt}a\ Electoral{\isacharunderscore}{\kern0pt}Module{\isacharparenright}{\kern0pt}\ {\isasymtimes}\ {\isacharparenleft}{\kern0pt}{\isacharunderscore}{\kern0pt}\ Electoral{\isacharunderscore}{\kern0pt}Module{\isacharparenright}{\kern0pt}\ {\isasymtimes}\isanewline
\ \ \ \ \ \ \ \ \ \ \ \ \ \ \ \ {\isacharparenleft}{\kern0pt}{\isacharunderscore}{\kern0pt}\ Termination{\isacharunderscore}{\kern0pt}Condition{\isacharparenright}{\kern0pt}\ {\isasymtimes}\ {\isacharunderscore}{\kern0pt}\ set\ {\isasymtimes}\ {\isacharunderscore}{\kern0pt}\ Profile{\isacharparenright}{\kern0pt}\ {\isasymor}\isanewline
\ \ \ \ \ \ \ \ \ \ wf\ rb\ {\isasymand}\ {\isasymnot}\ defer\ {\isacharparenleft}{\kern0pt}f\ {\isasymtriangleright}\ fa{\isacharparenright}{\kern0pt}\ A\ rs\ {\isasymsubset}\ defer\ f\ A\ rs\ {\isasymand}\isanewline
\ \ \ \ \ \ \ \ \ \ \ \ {\isasymnot}\ loop{\isacharunderscore}{\kern0pt}comp{\isacharunderscore}{\kern0pt}helper{\isacharunderscore}{\kern0pt}dom\isanewline
\ \ \ \ \ \ \ \ \ \ \ \ \ \ \ \ {\isacharparenleft}{\kern0pt}pc{\isacharcolon}{\kern0pt}{\isacharcolon}{\kern0pt}{\isacharparenleft}{\kern0pt}{\isacharprime}{\kern0pt}a\ Electoral{\isacharunderscore}{\kern0pt}Module{\isacharparenright}{\kern0pt}\ {\isasymtimes}\ {\isacharparenleft}{\kern0pt}{\isacharunderscore}{\kern0pt}\ Electoral{\isacharunderscore}{\kern0pt}Module{\isacharparenright}{\kern0pt}\ {\isasymtimes}\isanewline
\ \ \ \ \ \ \ \ \ \ \ \ \ \ \ \ \ \ {\isacharparenleft}{\kern0pt}{\isacharunderscore}{\kern0pt}\ Termination{\isacharunderscore}{\kern0pt}Condition{\isacharparenright}{\kern0pt}\ {\isasymtimes}\ {\isacharunderscore}{\kern0pt}\ set\ {\isasymtimes}\ {\isacharunderscore}{\kern0pt}\ Profile{\isacharparenright}{\kern0pt}\ {\isasymor}\isanewline
\ \ \ \ \ \ \ \ \ \ \ \ {\isacharparenleft}{\kern0pt}{\isacharparenleft}{\kern0pt}f\ {\isasymtriangleright}\ fa{\isacharcomma}{\kern0pt}\ fa{\isacharcomma}{\kern0pt}\ pa{\isacharcomma}{\kern0pt}\ A{\isacharcomma}{\kern0pt}\ rs{\isacharparenright}{\kern0pt}{\isacharcomma}{\kern0pt}\ f{\isacharcomma}{\kern0pt}\ fa{\isacharcomma}{\kern0pt}\ pa{\isacharcomma}{\kern0pt}\ A{\isacharcomma}{\kern0pt}\ rs{\isacharparenright}{\kern0pt}\ {\isasymin}\ rb\ {\isasymand}\ wf\ rb\ {\isasymand}\isanewline
\ \ \ \ \ \ \ \ \ \ \ \ {\isasymnot}\ loop{\isacharunderscore}{\kern0pt}comp{\isacharunderscore}{\kern0pt}helper{\isacharunderscore}{\kern0pt}dom\isanewline
\ \ \ \ \ \ \ \ \ \ \ \ \ \ \ \ {\isacharparenleft}{\kern0pt}pd{\isacharcolon}{\kern0pt}{\isacharcolon}{\kern0pt}{\isacharparenleft}{\kern0pt}{\isacharprime}{\kern0pt}a\ Electoral{\isacharunderscore}{\kern0pt}Module{\isacharparenright}{\kern0pt}\ {\isasymtimes}\ {\isacharparenleft}{\kern0pt}{\isacharunderscore}{\kern0pt}\ Electoral{\isacharunderscore}{\kern0pt}Module{\isacharparenright}{\kern0pt}\ {\isasymtimes}\isanewline
\ \ \ \ \ \ \ \ \ \ \ \ \ \ \ \ \ \ {\isacharparenleft}{\kern0pt}{\isacharunderscore}{\kern0pt}\ Termination{\isacharunderscore}{\kern0pt}Condition{\isacharparenright}{\kern0pt}\ {\isasymtimes}\ {\isacharunderscore}{\kern0pt}\ set\ {\isasymtimes}\ {\isacharunderscore}{\kern0pt}\ Profile{\isacharparenright}{\kern0pt}\ {\isasymor}\isanewline
\ \ \ \ \ \ \ \ \ \ \ \ finite\ {\isacharparenleft}{\kern0pt}defer\ fb\ {\isacharparenleft}{\kern0pt}Aa{\isacharcolon}{\kern0pt}{\isacharcolon}{\kern0pt}{\isacharprime}{\kern0pt}a\ set{\isacharparenright}{\kern0pt}\ rsa{\isacharparenright}{\kern0pt}\ {\isasymand}\isanewline
\ \ \ \ \ \ \ \ \ \ \ \ defer\ {\isacharparenleft}{\kern0pt}fb\ {\isasymtriangleright}\ fc{\isacharparenright}{\kern0pt}\ Aa\ rsa\ {\isasymsubset}\ defer\ fb\ Aa\ rsa\ {\isasymand}\isanewline
\ \ \ \ \ \ \ \ \ \ \ \ {\isasymnot}\ pe\ {\isacharparenleft}{\kern0pt}fb\ Aa\ rsa{\isacharparenright}{\kern0pt}\ {\isasymand}\isanewline
\ \ \ \ \ \ \ \ \ \ \ \ {\isacharparenleft}{\kern0pt}{\isacharparenleft}{\kern0pt}fb\ {\isasymtriangleright}\ fc{\isacharcomma}{\kern0pt}\ fc{\isacharcomma}{\kern0pt}\ pe{\isacharcomma}{\kern0pt}\ Aa{\isacharcomma}{\kern0pt}\ rsa{\isacharparenright}{\kern0pt}{\isacharcomma}{\kern0pt}\ fb{\isacharcomma}{\kern0pt}\ fc{\isacharcomma}{\kern0pt}\ pe{\isacharcomma}{\kern0pt}\ Aa{\isacharcomma}{\kern0pt}\ rsa{\isacharparenright}{\kern0pt}\ {\isasymnotin}\ r{\isacharparenright}{\kern0pt}{\isacharcolon}{\kern0pt}{\isacharcolon}{\kern0pt}\isanewline
\ \ \ \ \ \ \ \ \ \ {\isacharparenleft}{\kern0pt}{\isacharparenleft}{\kern0pt}{\isacharparenleft}{\kern0pt}{\isacharprime}{\kern0pt}a\ Electoral{\isacharunderscore}{\kern0pt}Module{\isacharparenright}{\kern0pt}\ {\isasymtimes}\ {\isacharparenleft}{\kern0pt}{\isacharprime}{\kern0pt}a\ Electoral{\isacharunderscore}{\kern0pt}Module{\isacharparenright}{\kern0pt}\ {\isasymtimes}\isanewline
\ \ \ \ \ \ \ \ \ \ \ \ {\isacharparenleft}{\kern0pt}{\isacharprime}{\kern0pt}a\ Termination{\isacharunderscore}{\kern0pt}Condition{\isacharparenright}{\kern0pt}\ {\isasymtimes}\ {\isacharprime}{\kern0pt}a\ set\ {\isasymtimes}\ {\isacharprime}{\kern0pt}a\ Profile{\isacharparenright}{\kern0pt}\ {\isasymtimes}\isanewline
\ \ \ \ \ \ \ \ \ \ \ \ {\isacharparenleft}{\kern0pt}{\isacharprime}{\kern0pt}a\ Electoral{\isacharunderscore}{\kern0pt}Module{\isacharparenright}{\kern0pt}\ {\isasymtimes}\ {\isacharparenleft}{\kern0pt}{\isacharprime}{\kern0pt}a\ Electoral{\isacharunderscore}{\kern0pt}Module{\isacharparenright}{\kern0pt}\ {\isasymtimes}\isanewline
\ \ \ \ \ \ \ \ \ \ \ \ {\isacharparenleft}{\kern0pt}{\isacharprime}{\kern0pt}a\ Termination{\isacharunderscore}{\kern0pt}Condition{\isacharparenright}{\kern0pt}\ {\isasymtimes}\ {\isacharprime}{\kern0pt}a\ set\ {\isasymtimes}\ {\isacharprime}{\kern0pt}a\ Profile{\isacharparenright}{\kern0pt}\ set\ {\isasymRightarrow}\ bool{\isacharparenright}{\kern0pt}{\isachardoublequoteclose}\isanewline
\ \ \ \ \isacommand{by}\isamarkupfalse%
\ metis\isanewline
\ \ \isacommand{obtain}\isamarkupfalse%
\isanewline
\ \ \ \ rr\ {\isacharcolon}{\kern0pt}{\isacharcolon}{\kern0pt}\ \ {\isachardoublequoteopen}{\isacharparenleft}{\kern0pt}{\isacharparenleft}{\kern0pt}{\isacharparenleft}{\kern0pt}{\isacharprime}{\kern0pt}a\ Electoral{\isacharunderscore}{\kern0pt}Module{\isacharparenright}{\kern0pt}\ {\isasymtimes}\ {\isacharparenleft}{\kern0pt}{\isacharprime}{\kern0pt}a\ Electoral{\isacharunderscore}{\kern0pt}Module{\isacharparenright}{\kern0pt}\ {\isasymtimes}\isanewline
\ \ \ \ \ \ \ \ \ \ \ \ \ {\isacharparenleft}{\kern0pt}{\isacharprime}{\kern0pt}a\ Termination{\isacharunderscore}{\kern0pt}Condition{\isacharparenright}{\kern0pt}\ {\isasymtimes}\ {\isacharprime}{\kern0pt}a\ set\ {\isasymtimes}\ {\isacharprime}{\kern0pt}a\ Profile{\isacharparenright}{\kern0pt}\ {\isasymtimes}\isanewline
\ \ \ \ \ \ \ \ \ \ \ \ \ {\isacharparenleft}{\kern0pt}{\isacharprime}{\kern0pt}a\ Electoral{\isacharunderscore}{\kern0pt}Module{\isacharparenright}{\kern0pt}\ {\isasymtimes}\ {\isacharparenleft}{\kern0pt}{\isacharprime}{\kern0pt}a\ Electoral{\isacharunderscore}{\kern0pt}Module{\isacharparenright}{\kern0pt}\ {\isasymtimes}\isanewline
\ \ \ \ \ \ \ \ \ \ \ \ \ {\isacharparenleft}{\kern0pt}{\isacharprime}{\kern0pt}a\ Termination{\isacharunderscore}{\kern0pt}Condition{\isacharparenright}{\kern0pt}\ {\isasymtimes}\ {\isacharprime}{\kern0pt}a\ set\ {\isasymtimes}\ {\isacharprime}{\kern0pt}a\ Profile{\isacharparenright}{\kern0pt}\ set{\isachardoublequoteclose}\ \isakeyword{where}\isanewline
\ \ \ \ \ \ {\isachardoublequoteopen}wf\ rr\ {\isasymand}\isanewline
\ \ \ \ \ \ \ \ {\isacharparenleft}{\kern0pt}{\isasymforall}p\ f\ A\ rs\ fa{\isachardot}{\kern0pt}\ p\ {\isacharparenleft}{\kern0pt}f\ A\ rs{\isacharparenright}{\kern0pt}\ {\isasymor}\isanewline
\ \ \ \ \ \ \ \ \ \ {\isasymnot}\ defer\ {\isacharparenleft}{\kern0pt}f\ {\isasymtriangleright}\ fa{\isacharparenright}{\kern0pt}\ A\ rs\ {\isasymsubset}\ defer\ f\ A\ rs\ {\isasymor}\isanewline
\ \ \ \ \ \ \ \ \ \ infinite\ {\isacharparenleft}{\kern0pt}defer\ f\ A\ rs{\isacharparenright}{\kern0pt}\ {\isasymor}\isanewline
\ \ \ \ \ \ \ \ \ \ {\isacharparenleft}{\kern0pt}{\isacharparenleft}{\kern0pt}f\ {\isasymtriangleright}\ fa{\isacharcomma}{\kern0pt}\ fa{\isacharcomma}{\kern0pt}\ p{\isacharcomma}{\kern0pt}\ A{\isacharcomma}{\kern0pt}\ rs{\isacharparenright}{\kern0pt}{\isacharcomma}{\kern0pt}\ f{\isacharcomma}{\kern0pt}\ fa{\isacharcomma}{\kern0pt}\ p{\isacharcomma}{\kern0pt}\ A{\isacharcomma}{\kern0pt}\ rs{\isacharparenright}{\kern0pt}\ {\isasymin}\ rr{\isacharparenright}{\kern0pt}{\isachardoublequoteclose}\isanewline
\ \ \ \ \isacommand{using}\isamarkupfalse%
\ f{\isadigit{0}}\isanewline
\ \ \ \ \isacommand{by}\isamarkupfalse%
\ presburger\isanewline
\ \ \isacommand{thus}\isamarkupfalse%
\ {\isacharquery}{\kern0pt}thesis\isanewline
\ \ \ \ \isacommand{using}\isamarkupfalse%
\ {\isachardoublequoteopen}termination{\isachardoublequoteclose}\isanewline
\ \ \ \ \isacommand{by}\isamarkupfalse%
\ metis\isanewline
\isacommand{qed}\isamarkupfalse%
%
\endisatagproof
{\isafoldproof}%
%
\isadelimproof
\isanewline
%
\endisadelimproof
\isanewline
\isacommand{lemma}\isamarkupfalse%
\ loop{\isacharunderscore}{\kern0pt}comp{\isacharunderscore}{\kern0pt}code{\isacharunderscore}{\kern0pt}helper{\isacharbrackleft}{\kern0pt}code{\isacharbrackright}{\kern0pt}{\isacharcolon}{\kern0pt}\isanewline
\ \ {\isachardoublequoteopen}loop{\isacharunderscore}{\kern0pt}comp{\isacharunderscore}{\kern0pt}helper\ acc\ m\ t\ A\ p\ {\isacharequal}{\kern0pt}\isanewline
\ \ \ \ {\isacharparenleft}{\kern0pt}if\ {\isacharparenleft}{\kern0pt}t\ {\isacharparenleft}{\kern0pt}acc\ A\ p{\isacharparenright}{\kern0pt}\ {\isasymor}\ {\isasymnot}{\isacharparenleft}{\kern0pt}{\isacharparenleft}{\kern0pt}defer\ {\isacharparenleft}{\kern0pt}acc\ {\isasymtriangleright}\ m{\isacharparenright}{\kern0pt}\ A\ p{\isacharparenright}{\kern0pt}\ {\isasymsubset}\ {\isacharparenleft}{\kern0pt}defer\ acc\ A\ p{\isacharparenright}{\kern0pt}{\isacharparenright}{\kern0pt}\ {\isasymor}\isanewline
\ \ \ \ \ \ infinite\ {\isacharparenleft}{\kern0pt}defer\ acc\ A\ p{\isacharparenright}{\kern0pt}{\isacharparenright}{\kern0pt}\isanewline
\ \ \ \ then\ {\isacharparenleft}{\kern0pt}acc\ A\ p{\isacharparenright}{\kern0pt}\ else\ {\isacharparenleft}{\kern0pt}loop{\isacharunderscore}{\kern0pt}comp{\isacharunderscore}{\kern0pt}helper\ {\isacharparenleft}{\kern0pt}acc\ {\isasymtriangleright}\ m{\isacharparenright}{\kern0pt}\ m\ t\ A\ p{\isacharparenright}{\kern0pt}{\isacharparenright}{\kern0pt}{\isachardoublequoteclose}\isanewline
%
\isadelimproof
\ \ %
\endisadelimproof
%
\isatagproof
\isacommand{by}\isamarkupfalse%
\ simp%
\endisatagproof
{\isafoldproof}%
%
\isadelimproof
\isanewline
%
\endisadelimproof
\isanewline
\isacommand{function}\isamarkupfalse%
\ loop{\isacharunderscore}{\kern0pt}composition\ {\isacharcolon}{\kern0pt}{\isacharcolon}{\kern0pt}\isanewline
\ \ \ \ {\isachardoublequoteopen}{\isacharprime}{\kern0pt}a\ Electoral{\isacharunderscore}{\kern0pt}Module\ {\isasymRightarrow}\ {\isacharprime}{\kern0pt}a\ Termination{\isacharunderscore}{\kern0pt}Condition\ {\isasymRightarrow}\isanewline
\ \ \ \ \ \ \ \ {\isacharprime}{\kern0pt}a\ Electoral{\isacharunderscore}{\kern0pt}Module{\isachardoublequoteclose}\ \isakeyword{where}\isanewline
\ \ {\isachardoublequoteopen}t\ {\isacharparenleft}{\kern0pt}{\isacharbraceleft}{\kern0pt}{\isacharbraceright}{\kern0pt}{\isacharcomma}{\kern0pt}\ {\isacharbraceleft}{\kern0pt}{\isacharbraceright}{\kern0pt}{\isacharcomma}{\kern0pt}\ A{\isacharparenright}{\kern0pt}\ {\isasymLongrightarrow}\isanewline
\ \ \ \ loop{\isacharunderscore}{\kern0pt}composition\ m\ t\ A\ p\ {\isacharequal}{\kern0pt}\ defer{\isacharunderscore}{\kern0pt}module\ A\ p{\isachardoublequoteclose}\ {\isacharbar}{\kern0pt}\isanewline
\ \ {\isachardoublequoteopen}{\isasymnot}{\isacharparenleft}{\kern0pt}t\ {\isacharparenleft}{\kern0pt}{\isacharbraceleft}{\kern0pt}{\isacharbraceright}{\kern0pt}{\isacharcomma}{\kern0pt}\ {\isacharbraceleft}{\kern0pt}{\isacharbraceright}{\kern0pt}{\isacharcomma}{\kern0pt}\ A{\isacharparenright}{\kern0pt}{\isacharparenright}{\kern0pt}\ {\isasymLongrightarrow}\isanewline
\ \ \ \ loop{\isacharunderscore}{\kern0pt}composition\ m\ t\ A\ p\ {\isacharequal}{\kern0pt}\ {\isacharparenleft}{\kern0pt}loop{\isacharunderscore}{\kern0pt}comp{\isacharunderscore}{\kern0pt}helper\ m\ m\ t{\isacharparenright}{\kern0pt}\ A\ p{\isachardoublequoteclose}\isanewline
%
\isadelimproof
\ \ %
\endisadelimproof
%
\isatagproof
\isacommand{by}\isamarkupfalse%
\ {\isacharparenleft}{\kern0pt}fastforce{\isacharcomma}{\kern0pt}\ simp{\isacharunderscore}{\kern0pt}all{\isacharparenright}{\kern0pt}%
\endisatagproof
{\isafoldproof}%
%
\isadelimproof
\isanewline
%
\endisadelimproof
\isacommand{termination}\isamarkupfalse%
\isanewline
%
\isadelimproof
\ \ %
\endisadelimproof
%
\isatagproof
\isacommand{using}\isamarkupfalse%
\ \ {\isachardoublequoteopen}termination{\isachardoublequoteclose}\ wf{\isacharunderscore}{\kern0pt}empty\isanewline
\ \ \isacommand{by}\isamarkupfalse%
\ blast%
\endisatagproof
{\isafoldproof}%
%
\isadelimproof
\isanewline
%
\endisadelimproof
\isanewline
\isacommand{abbreviation}\isamarkupfalse%
\ loop\ {\isacharcolon}{\kern0pt}{\isacharcolon}{\kern0pt}\isanewline
\ \ {\isachardoublequoteopen}{\isacharprime}{\kern0pt}a\ Electoral{\isacharunderscore}{\kern0pt}Module\ {\isasymRightarrow}\ {\isacharprime}{\kern0pt}a\ Termination{\isacharunderscore}{\kern0pt}Condition\ {\isasymRightarrow}\ {\isacharprime}{\kern0pt}a\ Electoral{\isacharunderscore}{\kern0pt}Module{\isachardoublequoteclose}\isanewline
\ \ \ \ {\isacharparenleft}{\kern0pt}{\isachardoublequoteopen}{\isacharunderscore}{\kern0pt}\ {\isasymcirclearrowleft}\isactrlsub {\isacharunderscore}{\kern0pt}{\isachardoublequoteclose}\ {\isadigit{5}}{\isadigit{0}}{\isacharparenright}{\kern0pt}\ \isakeyword{where}\isanewline
\ \ {\isachardoublequoteopen}m\ {\isasymcirclearrowleft}\isactrlsub t\ {\isasymequiv}\ loop{\isacharunderscore}{\kern0pt}composition\ m\ t{\isachardoublequoteclose}\isanewline
\isanewline
\isacommand{lemma}\isamarkupfalse%
\ loop{\isacharunderscore}{\kern0pt}comp{\isacharunderscore}{\kern0pt}code{\isacharbrackleft}{\kern0pt}code{\isacharbrackright}{\kern0pt}{\isacharcolon}{\kern0pt}\isanewline
\ \ {\isachardoublequoteopen}loop{\isacharunderscore}{\kern0pt}composition\ m\ t\ A\ p\ {\isacharequal}{\kern0pt}\isanewline
\ \ \ \ {\isacharparenleft}{\kern0pt}if\ {\isacharparenleft}{\kern0pt}t\ {\isacharparenleft}{\kern0pt}{\isacharbraceleft}{\kern0pt}{\isacharbraceright}{\kern0pt}{\isacharcomma}{\kern0pt}{\isacharbraceleft}{\kern0pt}{\isacharbraceright}{\kern0pt}{\isacharcomma}{\kern0pt}A{\isacharparenright}{\kern0pt}{\isacharparenright}{\kern0pt}\isanewline
\ \ \ \ then\ {\isacharparenleft}{\kern0pt}defer{\isacharunderscore}{\kern0pt}module\ A\ p{\isacharparenright}{\kern0pt}\ else\ {\isacharparenleft}{\kern0pt}loop{\isacharunderscore}{\kern0pt}comp{\isacharunderscore}{\kern0pt}helper\ m\ m\ t{\isacharparenright}{\kern0pt}\ A\ p{\isacharparenright}{\kern0pt}{\isachardoublequoteclose}\isanewline
%
\isadelimproof
\ \ %
\endisadelimproof
%
\isatagproof
\isacommand{by}\isamarkupfalse%
\ simp%
\endisatagproof
{\isafoldproof}%
%
\isadelimproof
\isanewline
%
\endisadelimproof
\isanewline
\isacommand{lemma}\isamarkupfalse%
\ loop{\isacharunderscore}{\kern0pt}comp{\isacharunderscore}{\kern0pt}helper{\isacharunderscore}{\kern0pt}imp{\isacharunderscore}{\kern0pt}partit{\isacharcolon}{\kern0pt}\isanewline
\ \ \isakeyword{assumes}\isanewline
\ \ \ \ module{\isacharunderscore}{\kern0pt}m{\isacharcolon}{\kern0pt}\ {\isachardoublequoteopen}electoral{\isacharunderscore}{\kern0pt}module\ m{\isachardoublequoteclose}\ \isakeyword{and}\isanewline
\ \ \ \ profile{\isacharcolon}{\kern0pt}\ {\isachardoublequoteopen}finite{\isacharunderscore}{\kern0pt}profile\ A\ p{\isachardoublequoteclose}\isanewline
\ \ \isakeyword{shows}\isanewline
\ \ \ \ {\isachardoublequoteopen}electoral{\isacharunderscore}{\kern0pt}module\ acc\ {\isasymand}\ {\isacharparenleft}{\kern0pt}n\ {\isacharequal}{\kern0pt}\ card\ {\isacharparenleft}{\kern0pt}defer\ acc\ A\ p{\isacharparenright}{\kern0pt}{\isacharparenright}{\kern0pt}\ {\isasymLongrightarrow}\isanewline
\ \ \ \ \ \ \ \ well{\isacharunderscore}{\kern0pt}formed\ A\ {\isacharparenleft}{\kern0pt}loop{\isacharunderscore}{\kern0pt}comp{\isacharunderscore}{\kern0pt}helper\ acc\ m\ t\ A\ p{\isacharparenright}{\kern0pt}{\isachardoublequoteclose}\isanewline
%
\isadelimproof
%
\endisadelimproof
%
\isatagproof
\isacommand{proof}\isamarkupfalse%
\ {\isacharparenleft}{\kern0pt}induct\ arbitrary{\isacharcolon}{\kern0pt}\ acc\ rule{\isacharcolon}{\kern0pt}\ less{\isacharunderscore}{\kern0pt}induct{\isacharparenright}{\kern0pt}\isanewline
\ \ \isacommand{case}\isamarkupfalse%
\ {\isacharparenleft}{\kern0pt}less{\isacharparenright}{\kern0pt}\isanewline
\ \ \isacommand{thus}\isamarkupfalse%
\ {\isacharquery}{\kern0pt}case\isanewline
\ \ \ \ \isacommand{using}\isamarkupfalse%
\ electoral{\isacharunderscore}{\kern0pt}module{\isacharunderscore}{\kern0pt}def\ loop{\isacharunderscore}{\kern0pt}comp{\isacharunderscore}{\kern0pt}helper{\isachardot}{\kern0pt}simps{\isacharparenleft}{\kern0pt}{\isadigit{1}}{\isacharparenright}{\kern0pt}\isanewline
\ \ \ \ \ \ \ \ \ \ loop{\isacharunderscore}{\kern0pt}comp{\isacharunderscore}{\kern0pt}helper{\isachardot}{\kern0pt}simps{\isacharparenleft}{\kern0pt}{\isadigit{2}}{\isacharparenright}{\kern0pt}\ module{\isacharunderscore}{\kern0pt}m\ profile\isanewline
\ \ \ \ \ \ \ \ \ \ psubset{\isacharunderscore}{\kern0pt}card{\isacharunderscore}{\kern0pt}mono\ seq{\isacharunderscore}{\kern0pt}comp{\isacharunderscore}{\kern0pt}sound\isanewline
\ \ \ \ \isacommand{by}\isamarkupfalse%
\ {\isacharparenleft}{\kern0pt}smt\ {\isacharparenleft}{\kern0pt}verit{\isacharparenright}{\kern0pt}{\isacharparenright}{\kern0pt}\isanewline
\isacommand{qed}\isamarkupfalse%
%
\endisatagproof
{\isafoldproof}%
%
\isadelimproof
%
\endisadelimproof
%
\isadelimdocument
%
\endisadelimdocument
%
\isatagdocument
%
\isamarkupsubsection{Soundness%
}
\isamarkuptrue%
%
\endisatagdocument
{\isafolddocument}%
%
\isadelimdocument
%
\endisadelimdocument
\isacommand{theorem}\isamarkupfalse%
\ loop{\isacharunderscore}{\kern0pt}comp{\isacharunderscore}{\kern0pt}sound{\isacharcolon}{\kern0pt}\isanewline
\ \ \isakeyword{assumes}\ m{\isacharunderscore}{\kern0pt}module{\isacharcolon}{\kern0pt}\ {\isachardoublequoteopen}electoral{\isacharunderscore}{\kern0pt}module\ m{\isachardoublequoteclose}\isanewline
\ \ \isakeyword{shows}\ {\isachardoublequoteopen}electoral{\isacharunderscore}{\kern0pt}module\ {\isacharparenleft}{\kern0pt}m\ {\isasymcirclearrowleft}\isactrlsub t{\isacharparenright}{\kern0pt}{\isachardoublequoteclose}\isanewline
%
\isadelimproof
\ \ %
\endisadelimproof
%
\isatagproof
\isacommand{using}\isamarkupfalse%
\ def{\isacharunderscore}{\kern0pt}mod{\isacharunderscore}{\kern0pt}sound\ electoral{\isacharunderscore}{\kern0pt}module{\isacharunderscore}{\kern0pt}def\ loop{\isacharunderscore}{\kern0pt}composition{\isachardot}{\kern0pt}simps{\isacharparenleft}{\kern0pt}{\isadigit{1}}{\isacharparenright}{\kern0pt}\isanewline
\ \ \ \ \ \ \ \ loop{\isacharunderscore}{\kern0pt}composition{\isachardot}{\kern0pt}simps{\isacharparenleft}{\kern0pt}{\isadigit{2}}{\isacharparenright}{\kern0pt}\ loop{\isacharunderscore}{\kern0pt}comp{\isacharunderscore}{\kern0pt}helper{\isacharunderscore}{\kern0pt}imp{\isacharunderscore}{\kern0pt}partit\ m{\isacharunderscore}{\kern0pt}module\isanewline
\ \ \isacommand{by}\isamarkupfalse%
\ metis%
\endisatagproof
{\isafoldproof}%
%
\isadelimproof
\isanewline
%
\endisadelimproof
\isanewline
\isacommand{lemma}\isamarkupfalse%
\ loop{\isacharunderscore}{\kern0pt}comp{\isacharunderscore}{\kern0pt}helper{\isacharunderscore}{\kern0pt}imp{\isacharunderscore}{\kern0pt}no{\isacharunderscore}{\kern0pt}def{\isacharunderscore}{\kern0pt}incr{\isacharcolon}{\kern0pt}\isanewline
\ \ \isakeyword{assumes}\isanewline
\ \ \ \ module{\isacharunderscore}{\kern0pt}m{\isacharcolon}{\kern0pt}\ {\isachardoublequoteopen}electoral{\isacharunderscore}{\kern0pt}module\ m{\isachardoublequoteclose}\ \isakeyword{and}\isanewline
\ \ \ \ profile{\isacharcolon}{\kern0pt}\ {\isachardoublequoteopen}finite{\isacharunderscore}{\kern0pt}profile\ A\ p{\isachardoublequoteclose}\isanewline
\ \ \isakeyword{shows}\isanewline
\ \ \ \ {\isachardoublequoteopen}{\isacharparenleft}{\kern0pt}electoral{\isacharunderscore}{\kern0pt}module\ acc\ {\isasymand}\ n\ {\isacharequal}{\kern0pt}\ card\ {\isacharparenleft}{\kern0pt}defer\ acc\ A\ p{\isacharparenright}{\kern0pt}{\isacharparenright}{\kern0pt}\ {\isasymLongrightarrow}\isanewline
\ \ \ \ \ \ \ \ defer\ {\isacharparenleft}{\kern0pt}loop{\isacharunderscore}{\kern0pt}comp{\isacharunderscore}{\kern0pt}helper\ acc\ m\ t{\isacharparenright}{\kern0pt}\ A\ p\ {\isasymsubseteq}\ defer\ acc\ A\ p{\isachardoublequoteclose}\isanewline
%
\isadelimproof
%
\endisadelimproof
%
\isatagproof
\isacommand{proof}\isamarkupfalse%
\ {\isacharparenleft}{\kern0pt}induct\ arbitrary{\isacharcolon}{\kern0pt}\ acc\ rule{\isacharcolon}{\kern0pt}\ less{\isacharunderscore}{\kern0pt}induct{\isacharparenright}{\kern0pt}\isanewline
\ \ \isacommand{case}\isamarkupfalse%
\ {\isacharparenleft}{\kern0pt}less{\isacharparenright}{\kern0pt}\isanewline
\ \ \isacommand{thus}\isamarkupfalse%
\ {\isacharquery}{\kern0pt}case\isanewline
\ \ \ \ \isacommand{using}\isamarkupfalse%
\ dual{\isacharunderscore}{\kern0pt}order{\isachardot}{\kern0pt}trans\ eq{\isacharunderscore}{\kern0pt}iff\ less{\isacharunderscore}{\kern0pt}imp{\isacharunderscore}{\kern0pt}le\ loop{\isacharunderscore}{\kern0pt}comp{\isacharunderscore}{\kern0pt}helper{\isachardot}{\kern0pt}simps{\isacharparenleft}{\kern0pt}{\isadigit{1}}{\isacharparenright}{\kern0pt}\isanewline
\ \ \ \ \ \ \ \ \ \ loop{\isacharunderscore}{\kern0pt}comp{\isacharunderscore}{\kern0pt}helper{\isachardot}{\kern0pt}simps{\isacharparenleft}{\kern0pt}{\isadigit{2}}{\isacharparenright}{\kern0pt}\ module{\isacharunderscore}{\kern0pt}m\ psubset{\isacharunderscore}{\kern0pt}card{\isacharunderscore}{\kern0pt}mono\isanewline
\ \ \ \ \ \ \ \ \ \ seq{\isacharunderscore}{\kern0pt}comp{\isacharunderscore}{\kern0pt}sound\isanewline
\ \ \ \ \isacommand{by}\isamarkupfalse%
\ {\isacharparenleft}{\kern0pt}smt\ {\isacharparenleft}{\kern0pt}verit{\isacharcomma}{\kern0pt}\ ccfv{\isacharunderscore}{\kern0pt}SIG{\isacharparenright}{\kern0pt}{\isacharparenright}{\kern0pt}\isanewline
\isacommand{qed}\isamarkupfalse%
%
\endisatagproof
{\isafoldproof}%
%
\isadelimproof
%
\endisadelimproof
%
\isadelimdocument
%
\endisadelimdocument
%
\isatagdocument
%
\isamarkupsubsection{Lemmata%
}
\isamarkuptrue%
%
\endisatagdocument
{\isafolddocument}%
%
\isadelimdocument
%
\endisadelimdocument
%
\isadelimtheory
%
\endisadelimtheory
%
\isatagtheory
\isacommand{end}\isamarkupfalse%
%
\endisatagtheory
{\isafoldtheory}%
%
\isadelimtheory
%
\endisadelimtheory
%
\end{isabellebody}%
\endinput
%:%file=~/Documents/Studies/VotingRuleGenerator/virage/src/test/resources/verifiedVotingRuleConstruction/theories/Compositional_Framework/Components/Compositional_Structures/Loop_Composition.thy%:%
%:%6=3%:%
%:%11=4%:%
%:%12=5%:%
%:%14=8%:%
%:%30=10%:%
%:%31=10%:%
%:%32=11%:%
%:%33=12%:%
%:%34=13%:%
%:%35=14%:%
%:%36=15%:%
%:%45=18%:%
%:%46=19%:%
%:%47=20%:%
%:%48=21%:%
%:%57=23%:%
%:%67=25%:%
%:%68=25%:%
%:%69=26%:%
%:%70=27%:%
%:%71=28%:%
%:%72=29%:%
%:%73=30%:%
%:%74=31%:%
%:%75=32%:%
%:%78=33%:%
%:%82=33%:%
%:%83=33%:%
%:%84=34%:%
%:%85=34%:%
%:%90=34%:%
%:%93=35%:%
%:%94=39%:%
%:%95=40%:%
%:%96=40%:%
%:%97=41%:%
%:%98=42%:%
%:%99=43%:%
%:%101=45%:%
%:%102=46%:%
%:%104=48%:%
%:%111=49%:%
%:%112=49%:%
%:%113=50%:%
%:%114=50%:%
%:%115=51%:%
%:%116=52%:%
%:%117=53%:%
%:%118=54%:%
%:%119=54%:%
%:%120=55%:%
%:%123=58%:%
%:%124=59%:%
%:%127=62%:%
%:%128=63%:%
%:%129=63%:%
%:%130=64%:%
%:%131=64%:%
%:%132=65%:%
%:%133=65%:%
%:%134=66%:%
%:%135=66%:%
%:%136=66%:%
%:%137=67%:%
%:%138=67%:%
%:%139=68%:%
%:%140=68%:%
%:%141=69%:%
%:%142=69%:%
%:%143=70%:%
%:%144=70%:%
%:%145=71%:%
%:%151=77%:%
%:%152=78%:%
%:%153=78%:%
%:%154=79%:%
%:%155=79%:%
%:%156=80%:%
%:%157=80%:%
%:%158=81%:%
%:%164=87%:%
%:%165=88%:%
%:%166=88%:%
%:%167=89%:%
%:%168=89%:%
%:%169=90%:%
%:%170=91%:%
%:%171=92%:%
%:%172=93%:%
%:%173=94%:%
%:%174=95%:%
%:%175=96%:%
%:%176=97%:%
%:%177=98%:%
%:%178=99%:%
%:%179=100%:%
%:%180=100%:%
%:%181=101%:%
%:%182=102%:%
%:%183=103%:%
%:%184=104%:%
%:%185=105%:%
%:%186=106%:%
%:%187=106%:%
%:%188=107%:%
%:%189=107%:%
%:%190=108%:%
%:%191=108%:%
%:%192=109%:%
%:%193=109%:%
%:%194=110%:%
%:%195=110%:%
%:%196=111%:%
%:%197=111%:%
%:%198=112%:%
%:%199=112%:%
%:%200=113%:%
%:%201=113%:%
%:%202=114%:%
%:%209=121%:%
%:%210=122%:%
%:%211=122%:%
%:%212=123%:%
%:%218=123%:%
%:%221=124%:%
%:%222=124%:%
%:%229=125%:%
%:%230=125%:%
%:%231=126%:%
%:%232=126%:%
%:%233=127%:%
%:%238=132%:%
%:%239=133%:%
%:%240=133%:%
%:%241=134%:%
%:%242=134%:%
%:%243=135%:%
%:%244=135%:%
%:%245=136%:%
%:%275=166%:%
%:%276=167%:%
%:%277=167%:%
%:%278=168%:%
%:%279=168%:%
%:%280=169%:%
%:%283=172%:%
%:%284=173%:%
%:%288=177%:%
%:%289=178%:%
%:%290=178%:%
%:%291=179%:%
%:%292=179%:%
%:%293=180%:%
%:%294=180%:%
%:%295=181%:%
%:%296=181%:%
%:%297=182%:%
%:%298=182%:%
%:%299=183%:%
%:%305=183%:%
%:%308=184%:%
%:%309=185%:%
%:%310=185%:%
%:%311=186%:%
%:%314=189%:%
%:%317=190%:%
%:%321=190%:%
%:%322=190%:%
%:%327=190%:%
%:%330=191%:%
%:%331=192%:%
%:%332=192%:%
%:%333=193%:%
%:%334=194%:%
%:%335=195%:%
%:%336=196%:%
%:%337=197%:%
%:%338=198%:%
%:%341=199%:%
%:%345=199%:%
%:%346=199%:%
%:%351=199%:%
%:%354=200%:%
%:%355=200%:%
%:%358=201%:%
%:%362=201%:%
%:%363=201%:%
%:%364=202%:%
%:%365=202%:%
%:%370=202%:%
%:%373=203%:%
%:%374=204%:%
%:%375=204%:%
%:%376=205%:%
%:%377=206%:%
%:%378=207%:%
%:%379=208%:%
%:%380=209%:%
%:%381=209%:%
%:%382=210%:%
%:%384=212%:%
%:%387=213%:%
%:%391=213%:%
%:%392=213%:%
%:%397=213%:%
%:%400=214%:%
%:%401=215%:%
%:%402=215%:%
%:%403=216%:%
%:%404=217%:%
%:%405=218%:%
%:%406=219%:%
%:%407=220%:%
%:%408=221%:%
%:%415=222%:%
%:%416=222%:%
%:%417=223%:%
%:%418=223%:%
%:%419=224%:%
%:%420=224%:%
%:%421=225%:%
%:%422=225%:%
%:%423=226%:%
%:%424=227%:%
%:%425=228%:%
%:%426=228%:%
%:%427=229%:%
%:%442=231%:%
%:%452=233%:%
%:%453=233%:%
%:%454=234%:%
%:%455=235%:%
%:%458=236%:%
%:%462=236%:%
%:%463=236%:%
%:%464=237%:%
%:%465=238%:%
%:%466=238%:%
%:%471=238%:%
%:%474=239%:%
%:%475=240%:%
%:%476=240%:%
%:%477=241%:%
%:%478=242%:%
%:%479=243%:%
%:%480=244%:%
%:%481=245%:%
%:%482=246%:%
%:%489=247%:%
%:%490=247%:%
%:%491=248%:%
%:%492=248%:%
%:%493=249%:%
%:%494=249%:%
%:%495=250%:%
%:%496=250%:%
%:%497=251%:%
%:%498=252%:%
%:%499=253%:%
%:%500=253%:%
%:%501=254%:%
%:%516=256%:%
%:%532=260%:%
%
\begin{isabellebody}%
\setisabellecontext{Aggregator{\isacharunderscore}{\kern0pt}Properties}%
%
\isadelimtheory
%
\endisadelimtheory
%
\isatagtheory
\isacommand{theory}\isamarkupfalse%
\ Aggregator{\isacharunderscore}{\kern0pt}Properties\isanewline
\ \ \isakeyword{imports}\ {\isachardoublequoteopen}{\isachardot}{\kern0pt}{\isachardot}{\kern0pt}{\isacharslash}{\kern0pt}Components{\isacharslash}{\kern0pt}Aggregator{\isachardoublequoteclose}\isanewline
\isanewline
\isakeyword{begin}%
\endisatagtheory
{\isafoldtheory}%
%
\isadelimtheory
\isanewline
%
\endisadelimtheory
\isanewline
\isacommand{definition}\isamarkupfalse%
\ agg{\isacharunderscore}{\kern0pt}commutative\ {\isacharcolon}{\kern0pt}{\isacharcolon}{\kern0pt}\ {\isachardoublequoteopen}{\isacharprime}{\kern0pt}a\ Aggregator\ {\isasymRightarrow}\ bool{\isachardoublequoteclose}\ \isakeyword{where}\isanewline
\ \ {\isachardoublequoteopen}agg{\isacharunderscore}{\kern0pt}commutative\ agg\ {\isasymequiv}\isanewline
\ \ \ \ aggregator\ agg\ {\isasymand}\ {\isacharparenleft}{\kern0pt}{\isasymforall}A\ e{\isadigit{1}}\ e{\isadigit{2}}\ d{\isadigit{1}}\ d{\isadigit{2}}\ r{\isadigit{1}}\ r{\isadigit{2}}{\isachardot}{\kern0pt}\isanewline
\ \ \ \ \ \ agg\ A\ {\isacharparenleft}{\kern0pt}e{\isadigit{1}}{\isacharcomma}{\kern0pt}\ r{\isadigit{1}}{\isacharcomma}{\kern0pt}\ d{\isadigit{1}}{\isacharparenright}{\kern0pt}\ {\isacharparenleft}{\kern0pt}e{\isadigit{2}}{\isacharcomma}{\kern0pt}\ r{\isadigit{2}}{\isacharcomma}{\kern0pt}\ d{\isadigit{2}}{\isacharparenright}{\kern0pt}\ {\isacharequal}{\kern0pt}\ agg\ A\ {\isacharparenleft}{\kern0pt}e{\isadigit{2}}{\isacharcomma}{\kern0pt}\ r{\isadigit{2}}{\isacharcomma}{\kern0pt}\ d{\isadigit{2}}{\isacharparenright}{\kern0pt}\ {\isacharparenleft}{\kern0pt}e{\isadigit{1}}{\isacharcomma}{\kern0pt}\ r{\isadigit{1}}{\isacharcomma}{\kern0pt}\ d{\isadigit{1}}{\isacharparenright}{\kern0pt}{\isacharparenright}{\kern0pt}{\isachardoublequoteclose}\isanewline
\isanewline
\isacommand{definition}\isamarkupfalse%
\ agg{\isacharunderscore}{\kern0pt}conservative\ {\isacharcolon}{\kern0pt}{\isacharcolon}{\kern0pt}\ {\isachardoublequoteopen}{\isacharprime}{\kern0pt}a\ Aggregator\ {\isasymRightarrow}\ bool{\isachardoublequoteclose}\ \isakeyword{where}\isanewline
\ \ {\isachardoublequoteopen}agg{\isacharunderscore}{\kern0pt}conservative\ agg\ {\isasymequiv}\isanewline
\ \ \ \ aggregator\ agg\ {\isasymand}\isanewline
\ \ \ \ {\isacharparenleft}{\kern0pt}{\isasymforall}A\ e{\isadigit{1}}\ e{\isadigit{2}}\ d{\isadigit{1}}\ d{\isadigit{2}}\ r{\isadigit{1}}\ r{\isadigit{2}}{\isachardot}{\kern0pt}\isanewline
\ \ \ \ \ \ {\isacharparenleft}{\kern0pt}{\isacharparenleft}{\kern0pt}well{\isacharunderscore}{\kern0pt}formed\ A\ {\isacharparenleft}{\kern0pt}e{\isadigit{1}}{\isacharcomma}{\kern0pt}\ r{\isadigit{1}}{\isacharcomma}{\kern0pt}\ d{\isadigit{1}}{\isacharparenright}{\kern0pt}\ {\isasymand}\ well{\isacharunderscore}{\kern0pt}formed\ A\ {\isacharparenleft}{\kern0pt}e{\isadigit{2}}{\isacharcomma}{\kern0pt}\ r{\isadigit{2}}{\isacharcomma}{\kern0pt}\ d{\isadigit{2}}{\isacharparenright}{\kern0pt}{\isacharparenright}{\kern0pt}\ {\isasymlongrightarrow}\isanewline
\ \ \ \ \ \ \ \ elect{\isacharunderscore}{\kern0pt}r\ {\isacharparenleft}{\kern0pt}agg\ A\ {\isacharparenleft}{\kern0pt}e{\isadigit{1}}{\isacharcomma}{\kern0pt}\ r{\isadigit{1}}{\isacharcomma}{\kern0pt}\ d{\isadigit{1}}{\isacharparenright}{\kern0pt}\ {\isacharparenleft}{\kern0pt}e{\isadigit{2}}{\isacharcomma}{\kern0pt}\ r{\isadigit{2}}{\isacharcomma}{\kern0pt}\ d{\isadigit{2}}{\isacharparenright}{\kern0pt}{\isacharparenright}{\kern0pt}\ {\isasymsubseteq}\ {\isacharparenleft}{\kern0pt}e{\isadigit{1}}\ {\isasymunion}\ e{\isadigit{2}}{\isacharparenright}{\kern0pt}\ {\isasymand}\isanewline
\ \ \ \ \ \ \ \ reject{\isacharunderscore}{\kern0pt}r\ {\isacharparenleft}{\kern0pt}agg\ A\ {\isacharparenleft}{\kern0pt}e{\isadigit{1}}{\isacharcomma}{\kern0pt}\ r{\isadigit{1}}{\isacharcomma}{\kern0pt}\ d{\isadigit{1}}{\isacharparenright}{\kern0pt}\ {\isacharparenleft}{\kern0pt}e{\isadigit{2}}{\isacharcomma}{\kern0pt}\ r{\isadigit{2}}{\isacharcomma}{\kern0pt}\ d{\isadigit{2}}{\isacharparenright}{\kern0pt}{\isacharparenright}{\kern0pt}\ {\isasymsubseteq}\ {\isacharparenleft}{\kern0pt}r{\isadigit{1}}\ {\isasymunion}\ r{\isadigit{2}}{\isacharparenright}{\kern0pt}\ {\isasymand}\isanewline
\ \ \ \ \ \ \ \ defer{\isacharunderscore}{\kern0pt}r\ {\isacharparenleft}{\kern0pt}agg\ A\ {\isacharparenleft}{\kern0pt}e{\isadigit{1}}{\isacharcomma}{\kern0pt}\ r{\isadigit{1}}{\isacharcomma}{\kern0pt}\ d{\isadigit{1}}{\isacharparenright}{\kern0pt}\ {\isacharparenleft}{\kern0pt}e{\isadigit{2}}{\isacharcomma}{\kern0pt}\ r{\isadigit{2}}{\isacharcomma}{\kern0pt}\ d{\isadigit{2}}{\isacharparenright}{\kern0pt}{\isacharparenright}{\kern0pt}\ {\isasymsubseteq}\ {\isacharparenleft}{\kern0pt}d{\isadigit{1}}\ {\isasymunion}\ d{\isadigit{2}}{\isacharparenright}{\kern0pt}{\isacharparenright}{\kern0pt}{\isacharparenright}{\kern0pt}{\isachardoublequoteclose}\isanewline
%
\isadelimtheory
\isanewline
%
\endisadelimtheory
%
\isatagtheory
\isacommand{end}\isamarkupfalse%
%
\endisatagtheory
{\isafoldtheory}%
%
\isadelimtheory
%
\endisadelimtheory
%
\end{isabellebody}%
\endinput
%:%file=~/Documents/Studies/VotingRuleGenerator/virage/src/test/resources/verifiedVotingRuleConstruction/theories/Compositional_Framework/Properties/Aggregator_Properties.thy%:%
%:%10=1%:%
%:%11=1%:%
%:%12=2%:%
%:%13=3%:%
%:%14=4%:%
%:%19=4%:%
%:%22=5%:%
%:%23=6%:%
%:%24=6%:%
%:%25=7%:%
%:%27=9%:%
%:%28=10%:%
%:%29=11%:%
%:%30=11%:%
%:%31=12%:%
%:%37=18%:%
%:%40=19%:%
%:%45=20%:%
%
\begin{isabellebody}%
\setisabellecontext{Indep{\isacharunderscore}{\kern0pt}Of{\isacharunderscore}{\kern0pt}Alt}%
%
\isadelimtheory
%
\endisadelimtheory
%
\isatagtheory
\isacommand{theory}\isamarkupfalse%
\ Indep{\isacharunderscore}{\kern0pt}Of{\isacharunderscore}{\kern0pt}Alt\isanewline
\ \ \isakeyword{imports}\ {\isachardoublequoteopen}{\isachardot}{\kern0pt}{\isachardot}{\kern0pt}{\isacharslash}{\kern0pt}Components{\isacharslash}{\kern0pt}Electoral{\isacharunderscore}{\kern0pt}Module{\isachardoublequoteclose}\isanewline
\isanewline
\isakeyword{begin}%
\endisatagtheory
{\isafoldtheory}%
%
\isadelimtheory
\isanewline
%
\endisadelimtheory
\isanewline
\isanewline
\isacommand{definition}\isamarkupfalse%
\ indep{\isacharunderscore}{\kern0pt}of{\isacharunderscore}{\kern0pt}alt\ {\isacharcolon}{\kern0pt}{\isacharcolon}{\kern0pt}\ {\isachardoublequoteopen}{\isacharprime}{\kern0pt}a\ Electoral{\isacharunderscore}{\kern0pt}Module\ {\isasymRightarrow}\ {\isacharprime}{\kern0pt}a\ set\ {\isasymRightarrow}\ {\isacharprime}{\kern0pt}a\ {\isasymRightarrow}\ bool{\isachardoublequoteclose}\ \isakeyword{where}\isanewline
\ \ {\isachardoublequoteopen}indep{\isacharunderscore}{\kern0pt}of{\isacharunderscore}{\kern0pt}alt\ m\ A\ a\ {\isasymequiv}\isanewline
\ \ \ \ electoral{\isacharunderscore}{\kern0pt}module\ m\ {\isasymand}\ {\isacharparenleft}{\kern0pt}{\isasymforall}p\ q{\isachardot}{\kern0pt}\ equiv{\isacharunderscore}{\kern0pt}prof{\isacharunderscore}{\kern0pt}except{\isacharunderscore}{\kern0pt}a\ A\ p\ q\ a\ {\isasymlongrightarrow}\ m\ A\ p\ {\isacharequal}{\kern0pt}\ m\ A\ q{\isacharparenright}{\kern0pt}{\isachardoublequoteclose}\isanewline
%
\isadelimtheory
\isanewline
%
\endisadelimtheory
%
\isatagtheory
\isacommand{end}\isamarkupfalse%
%
\endisatagtheory
{\isafoldtheory}%
%
\isadelimtheory
%
\endisadelimtheory
%
\end{isabellebody}%
\endinput
%:%file=~/Documents/Studies/VotingRuleGenerator/virage/src/test/resources/verifiedVotingRuleConstruction/theories/Compositional_Framework/Properties/Indep_Of_Alt.thy%:%
%:%10=1%:%
%:%11=1%:%
%:%12=2%:%
%:%13=3%:%
%:%14=4%:%
%:%19=4%:%
%:%22=5%:%
%:%23=9%:%
%:%24=10%:%
%:%25=10%:%
%:%26=11%:%
%:%27=12%:%
%:%30=13%:%
%:%35=14%:%
%
\begin{isabellebody}%
\setisabellecontext{Disjoint{\isacharunderscore}{\kern0pt}Compatibility}%
%
\isadelimtheory
%
\endisadelimtheory
%
\isatagtheory
\isacommand{theory}\isamarkupfalse%
\ Disjoint{\isacharunderscore}{\kern0pt}Compatibility\isanewline
\ \ \isakeyword{imports}\ {\isachardoublequoteopen}{\isachardot}{\kern0pt}{\isachardot}{\kern0pt}{\isacharslash}{\kern0pt}Components{\isacharslash}{\kern0pt}Electoral{\isacharunderscore}{\kern0pt}Module{\isachardoublequoteclose}\isanewline
\ \ \ \ \ \ \ \ \ \ Indep{\isacharunderscore}{\kern0pt}Of{\isacharunderscore}{\kern0pt}Alt\isanewline
\isanewline
\isakeyword{begin}%
\endisatagtheory
{\isafoldtheory}%
%
\isadelimtheory
\isanewline
%
\endisadelimtheory
\isanewline
\isanewline
\isacommand{definition}\isamarkupfalse%
\ disjoint{\isacharunderscore}{\kern0pt}compatibility\ {\isacharcolon}{\kern0pt}{\isacharcolon}{\kern0pt}\ {\isachardoublequoteopen}{\isacharprime}{\kern0pt}a\ Electoral{\isacharunderscore}{\kern0pt}Module\ {\isasymRightarrow}\isanewline
\ \ \ \ \ \ \ \ \ \ \ \ \ \ \ \ \ \ \ \ \ \ \ \ \ \ \ \ \ \ \ \ \ \ \ \ \ \ \ \ \ {\isacharprime}{\kern0pt}a\ Electoral{\isacharunderscore}{\kern0pt}Module\ {\isasymRightarrow}\ bool{\isachardoublequoteclose}\ \isakeyword{where}\isanewline
\ \ {\isachardoublequoteopen}disjoint{\isacharunderscore}{\kern0pt}compatibility\ m\ n\ {\isasymequiv}\isanewline
\ \ \ \ electoral{\isacharunderscore}{\kern0pt}module\ m\ {\isasymand}\ electoral{\isacharunderscore}{\kern0pt}module\ n\ {\isasymand}\isanewline
\ \ \ \ \ \ \ \ {\isacharparenleft}{\kern0pt}{\isasymforall}S{\isachardot}{\kern0pt}\ finite\ S\ {\isasymlongrightarrow}\isanewline
\ \ \ \ \ \ \ \ \ \ {\isacharparenleft}{\kern0pt}{\isasymexists}A\ {\isasymsubseteq}\ S{\isachardot}{\kern0pt}\isanewline
\ \ \ \ \ \ \ \ \ \ \ \ {\isacharparenleft}{\kern0pt}{\isasymforall}a\ {\isasymin}\ A{\isachardot}{\kern0pt}\ indep{\isacharunderscore}{\kern0pt}of{\isacharunderscore}{\kern0pt}alt\ m\ S\ a\ {\isasymand}\isanewline
\ \ \ \ \ \ \ \ \ \ \ \ \ \ {\isacharparenleft}{\kern0pt}{\isasymforall}p{\isachardot}{\kern0pt}\ finite{\isacharunderscore}{\kern0pt}profile\ S\ p\ {\isasymlongrightarrow}\ a\ {\isasymin}\ reject\ m\ S\ p{\isacharparenright}{\kern0pt}{\isacharparenright}{\kern0pt}\ {\isasymand}\isanewline
\ \ \ \ \ \ \ \ \ \ \ \ {\isacharparenleft}{\kern0pt}{\isasymforall}a\ {\isasymin}\ S{\isacharminus}{\kern0pt}A{\isachardot}{\kern0pt}\ indep{\isacharunderscore}{\kern0pt}of{\isacharunderscore}{\kern0pt}alt\ n\ S\ a\ {\isasymand}\isanewline
\ \ \ \ \ \ \ \ \ \ \ \ \ \ {\isacharparenleft}{\kern0pt}{\isasymforall}p{\isachardot}{\kern0pt}\ finite{\isacharunderscore}{\kern0pt}profile\ S\ p\ {\isasymlongrightarrow}\ a\ {\isasymin}\ reject\ n\ S\ p{\isacharparenright}{\kern0pt}{\isacharparenright}{\kern0pt}{\isacharparenright}{\kern0pt}{\isacharparenright}{\kern0pt}{\isachardoublequoteclose}\isanewline
%
\isadelimtheory
\isanewline
%
\endisadelimtheory
%
\isatagtheory
\isacommand{end}\isamarkupfalse%
%
\endisatagtheory
{\isafoldtheory}%
%
\isadelimtheory
%
\endisadelimtheory
%
\end{isabellebody}%
\endinput
%:%file=~/Documents/Studies/VotingRuleGenerator/virage/src/test/resources/verifiedVotingRuleConstruction/theories/Compositional_Framework/Properties/Disjoint_Compatibility.thy%:%
%:%10=1%:%
%:%11=1%:%
%:%12=2%:%
%:%13=3%:%
%:%14=4%:%
%:%15=5%:%
%:%20=5%:%
%:%23=6%:%
%:%24=11%:%
%:%25=12%:%
%:%26=12%:%
%:%27=13%:%
%:%28=14%:%
%:%35=21%:%
%:%38=22%:%
%:%43=23%:%
%
\begin{isabellebody}%
\setisabellecontext{Aggregator{\isacharunderscore}{\kern0pt}Facts}%
%
\isadelimtheory
%
\endisadelimtheory
%
\isatagtheory
\isacommand{theory}\isamarkupfalse%
\ Aggregator{\isacharunderscore}{\kern0pt}Facts\isanewline
\ \ \isakeyword{imports}\ {\isachardoublequoteopen}{\isachardot}{\kern0pt}{\isachardot}{\kern0pt}{\isacharslash}{\kern0pt}Properties{\isacharslash}{\kern0pt}Aggregator{\isacharunderscore}{\kern0pt}Properties{\isachardoublequoteclose}\isanewline
\ \ \ \ \ \ \ \ \ \ {\isachardoublequoteopen}{\isachardot}{\kern0pt}{\isachardot}{\kern0pt}{\isacharslash}{\kern0pt}Components{\isacharslash}{\kern0pt}Basic{\isacharunderscore}{\kern0pt}Modules{\isacharslash}{\kern0pt}Maximum{\isacharunderscore}{\kern0pt}Aggregator{\isachardoublequoteclose}\isanewline
\isanewline
\isakeyword{begin}%
\endisatagtheory
{\isafoldtheory}%
%
\isadelimtheory
\isanewline
%
\endisadelimtheory
\isanewline
\isanewline
\isacommand{theorem}\isamarkupfalse%
\ max{\isacharunderscore}{\kern0pt}agg{\isacharunderscore}{\kern0pt}comm{\isacharbrackleft}{\kern0pt}simp{\isacharbrackright}{\kern0pt}{\isacharcolon}{\kern0pt}\ {\isachardoublequoteopen}agg{\isacharunderscore}{\kern0pt}commutative\ max{\isacharunderscore}{\kern0pt}aggregator{\isachardoublequoteclose}\isanewline
%
\isadelimproof
\ \ %
\endisadelimproof
%
\isatagproof
\isacommand{unfolding}\isamarkupfalse%
\ agg{\isacharunderscore}{\kern0pt}commutative{\isacharunderscore}{\kern0pt}def\isanewline
\isacommand{proof}\isamarkupfalse%
\ {\isacharparenleft}{\kern0pt}safe{\isacharparenright}{\kern0pt}\isanewline
\ \ \isacommand{show}\isamarkupfalse%
\ {\isachardoublequoteopen}aggregator\ max{\isacharunderscore}{\kern0pt}aggregator{\isachardoublequoteclose}\isanewline
\ \ \ \ \isacommand{by}\isamarkupfalse%
\ simp\isanewline
\isacommand{next}\isamarkupfalse%
\isanewline
\ \ \isacommand{fix}\isamarkupfalse%
\isanewline
\ \ \ \ A\ {\isacharcolon}{\kern0pt}{\isacharcolon}{\kern0pt}\ {\isachardoublequoteopen}{\isacharprime}{\kern0pt}a\ set{\isachardoublequoteclose}\ \isakeyword{and}\isanewline
\ \ \ \ e{\isadigit{1}}\ {\isacharcolon}{\kern0pt}{\isacharcolon}{\kern0pt}\ {\isachardoublequoteopen}{\isacharprime}{\kern0pt}a\ set{\isachardoublequoteclose}\ \isakeyword{and}\isanewline
\ \ \ \ e{\isadigit{2}}\ {\isacharcolon}{\kern0pt}{\isacharcolon}{\kern0pt}\ {\isachardoublequoteopen}{\isacharprime}{\kern0pt}a\ set{\isachardoublequoteclose}\ \isakeyword{and}\isanewline
\ \ \ \ d{\isadigit{1}}\ {\isacharcolon}{\kern0pt}{\isacharcolon}{\kern0pt}\ {\isachardoublequoteopen}{\isacharprime}{\kern0pt}a\ set{\isachardoublequoteclose}\ \isakeyword{and}\isanewline
\ \ \ \ d{\isadigit{2}}\ {\isacharcolon}{\kern0pt}{\isacharcolon}{\kern0pt}\ {\isachardoublequoteopen}{\isacharprime}{\kern0pt}a\ set{\isachardoublequoteclose}\ \isakeyword{and}\isanewline
\ \ \ \ r{\isadigit{1}}\ {\isacharcolon}{\kern0pt}{\isacharcolon}{\kern0pt}\ {\isachardoublequoteopen}{\isacharprime}{\kern0pt}a\ set{\isachardoublequoteclose}\ \isakeyword{and}\isanewline
\ \ \ \ r{\isadigit{2}}\ {\isacharcolon}{\kern0pt}{\isacharcolon}{\kern0pt}\ {\isachardoublequoteopen}{\isacharprime}{\kern0pt}a\ set{\isachardoublequoteclose}\isanewline
\ \ \isacommand{show}\isamarkupfalse%
\isanewline
\ \ \ \ {\isachardoublequoteopen}max{\isacharunderscore}{\kern0pt}aggregator\ A\ {\isacharparenleft}{\kern0pt}e{\isadigit{1}}{\isacharcomma}{\kern0pt}\ r{\isadigit{1}}{\isacharcomma}{\kern0pt}\ d{\isadigit{1}}{\isacharparenright}{\kern0pt}\ {\isacharparenleft}{\kern0pt}e{\isadigit{2}}{\isacharcomma}{\kern0pt}\ r{\isadigit{2}}{\isacharcomma}{\kern0pt}\ d{\isadigit{2}}{\isacharparenright}{\kern0pt}\ {\isacharequal}{\kern0pt}\isanewline
\ \ \ \ \ \ max{\isacharunderscore}{\kern0pt}aggregator\ A\ {\isacharparenleft}{\kern0pt}e{\isadigit{2}}{\isacharcomma}{\kern0pt}\ r{\isadigit{2}}{\isacharcomma}{\kern0pt}\ d{\isadigit{2}}{\isacharparenright}{\kern0pt}\ {\isacharparenleft}{\kern0pt}e{\isadigit{1}}{\isacharcomma}{\kern0pt}\ r{\isadigit{1}}{\isacharcomma}{\kern0pt}\ d{\isadigit{1}}{\isacharparenright}{\kern0pt}{\isachardoublequoteclose}\isanewline
\ \ \isacommand{by}\isamarkupfalse%
\ auto\isanewline
\isacommand{qed}\isamarkupfalse%
%
\endisatagproof
{\isafoldproof}%
%
\isadelimproof
\isanewline
%
\endisadelimproof
\isanewline
\isanewline
\isanewline
\isacommand{theorem}\isamarkupfalse%
\ max{\isacharunderscore}{\kern0pt}agg{\isacharunderscore}{\kern0pt}consv{\isacharbrackleft}{\kern0pt}simp{\isacharbrackright}{\kern0pt}{\isacharcolon}{\kern0pt}\ {\isachardoublequoteopen}agg{\isacharunderscore}{\kern0pt}conservative\ max{\isacharunderscore}{\kern0pt}aggregator{\isachardoublequoteclose}\isanewline
%
\isadelimproof
%
\endisadelimproof
%
\isatagproof
\isacommand{proof}\isamarkupfalse%
\ {\isacharminus}{\kern0pt}\isanewline
\ \ \isacommand{have}\isamarkupfalse%
\isanewline
\ \ \ \ {\isachardoublequoteopen}{\isasymforall}A\ e{\isadigit{1}}\ e{\isadigit{2}}\ d{\isadigit{1}}\ d{\isadigit{2}}\ r{\isadigit{1}}\ r{\isadigit{2}}{\isachardot}{\kern0pt}\isanewline
\ \ \ \ \ \ \ \ \ \ {\isacharparenleft}{\kern0pt}well{\isacharunderscore}{\kern0pt}formed\ A\ {\isacharparenleft}{\kern0pt}e{\isadigit{1}}{\isacharcomma}{\kern0pt}\ r{\isadigit{1}}{\isacharcomma}{\kern0pt}\ d{\isadigit{1}}{\isacharparenright}{\kern0pt}\ {\isasymand}\ well{\isacharunderscore}{\kern0pt}formed\ A\ {\isacharparenleft}{\kern0pt}e{\isadigit{2}}{\isacharcomma}{\kern0pt}\ r{\isadigit{2}}{\isacharcomma}{\kern0pt}\ d{\isadigit{2}}{\isacharparenright}{\kern0pt}{\isacharparenright}{\kern0pt}\ {\isasymlongrightarrow}\isanewline
\ \ \ \ \ \ reject{\isacharunderscore}{\kern0pt}r\ {\isacharparenleft}{\kern0pt}max{\isacharunderscore}{\kern0pt}aggregator\ A\ {\isacharparenleft}{\kern0pt}e{\isadigit{1}}{\isacharcomma}{\kern0pt}\ r{\isadigit{1}}{\isacharcomma}{\kern0pt}\ d{\isadigit{1}}{\isacharparenright}{\kern0pt}\ {\isacharparenleft}{\kern0pt}e{\isadigit{2}}{\isacharcomma}{\kern0pt}\ r{\isadigit{2}}{\isacharcomma}{\kern0pt}\ d{\isadigit{2}}{\isacharparenright}{\kern0pt}{\isacharparenright}{\kern0pt}\ {\isacharequal}{\kern0pt}\ r{\isadigit{1}}\ {\isasyminter}\ r{\isadigit{2}}{\isachardoublequoteclose}\isanewline
\ \ \ \ \isacommand{using}\isamarkupfalse%
\ max{\isacharunderscore}{\kern0pt}agg{\isacharunderscore}{\kern0pt}rej{\isacharunderscore}{\kern0pt}set\isanewline
\ \ \ \ \isacommand{by}\isamarkupfalse%
\ blast\isanewline
\ \ \isacommand{hence}\isamarkupfalse%
\isanewline
\ \ \ \ {\isachardoublequoteopen}{\isasymforall}A\ e{\isadigit{1}}\ e{\isadigit{2}}\ d{\isadigit{1}}\ d{\isadigit{2}}\ r{\isadigit{1}}\ r{\isadigit{2}}{\isachardot}{\kern0pt}\isanewline
\ \ \ \ \ \ \ \ \ \ \ \ {\isacharparenleft}{\kern0pt}well{\isacharunderscore}{\kern0pt}formed\ A\ {\isacharparenleft}{\kern0pt}e{\isadigit{1}}{\isacharcomma}{\kern0pt}\ r{\isadigit{1}}{\isacharcomma}{\kern0pt}\ d{\isadigit{1}}{\isacharparenright}{\kern0pt}\ {\isasymand}\ well{\isacharunderscore}{\kern0pt}formed\ A\ {\isacharparenleft}{\kern0pt}e{\isadigit{2}}{\isacharcomma}{\kern0pt}\ r{\isadigit{2}}{\isacharcomma}{\kern0pt}\ d{\isadigit{2}}{\isacharparenright}{\kern0pt}{\isacharparenright}{\kern0pt}\ {\isasymlongrightarrow}\isanewline
\ \ \ \ \ \ \ \ reject{\isacharunderscore}{\kern0pt}r\ {\isacharparenleft}{\kern0pt}max{\isacharunderscore}{\kern0pt}aggregator\ A\ {\isacharparenleft}{\kern0pt}e{\isadigit{1}}{\isacharcomma}{\kern0pt}\ r{\isadigit{1}}{\isacharcomma}{\kern0pt}\ d{\isadigit{1}}{\isacharparenright}{\kern0pt}\ {\isacharparenleft}{\kern0pt}e{\isadigit{2}}{\isacharcomma}{\kern0pt}\ r{\isadigit{2}}{\isacharcomma}{\kern0pt}\ d{\isadigit{2}}{\isacharparenright}{\kern0pt}{\isacharparenright}{\kern0pt}\ {\isasymsubseteq}\ r{\isadigit{1}}\ {\isasyminter}\ r{\isadigit{2}}{\isachardoublequoteclose}\isanewline
\ \ \ \ \isacommand{by}\isamarkupfalse%
\ blast\isanewline
\ \ \isacommand{moreover}\isamarkupfalse%
\ \isacommand{have}\isamarkupfalse%
\isanewline
\ \ \ \ {\isachardoublequoteopen}{\isasymforall}A\ e{\isadigit{1}}\ e{\isadigit{2}}\ d{\isadigit{1}}\ d{\isadigit{2}}\ r{\isadigit{1}}\ r{\isadigit{2}}{\isachardot}{\kern0pt}\isanewline
\ \ \ \ \ \ \ \ {\isacharparenleft}{\kern0pt}well{\isacharunderscore}{\kern0pt}formed\ A\ {\isacharparenleft}{\kern0pt}e{\isadigit{1}}{\isacharcomma}{\kern0pt}\ r{\isadigit{1}}{\isacharcomma}{\kern0pt}\ d{\isadigit{1}}{\isacharparenright}{\kern0pt}\ {\isasymand}\ well{\isacharunderscore}{\kern0pt}formed\ A\ {\isacharparenleft}{\kern0pt}e{\isadigit{2}}{\isacharcomma}{\kern0pt}\ r{\isadigit{2}}{\isacharcomma}{\kern0pt}\ d{\isadigit{2}}{\isacharparenright}{\kern0pt}{\isacharparenright}{\kern0pt}\ {\isasymlongrightarrow}\isanewline
\ \ \ \ \ \ \ \ \ \ \ \ elect{\isacharunderscore}{\kern0pt}r\ {\isacharparenleft}{\kern0pt}max{\isacharunderscore}{\kern0pt}aggregator\ A\ {\isacharparenleft}{\kern0pt}e{\isadigit{1}}{\isacharcomma}{\kern0pt}\ r{\isadigit{1}}{\isacharcomma}{\kern0pt}\ d{\isadigit{1}}{\isacharparenright}{\kern0pt}\ {\isacharparenleft}{\kern0pt}e{\isadigit{2}}{\isacharcomma}{\kern0pt}\ r{\isadigit{2}}{\isacharcomma}{\kern0pt}\ d{\isadigit{2}}{\isacharparenright}{\kern0pt}{\isacharparenright}{\kern0pt}\ {\isasymsubseteq}\ {\isacharparenleft}{\kern0pt}e{\isadigit{1}}\ {\isasymunion}\ e{\isadigit{2}}{\isacharparenright}{\kern0pt}{\isachardoublequoteclose}\isanewline
\ \ \ \ \isacommand{by}\isamarkupfalse%
\ {\isacharparenleft}{\kern0pt}simp\ add{\isacharcolon}{\kern0pt}\ subset{\isacharunderscore}{\kern0pt}eq{\isacharparenright}{\kern0pt}\isanewline
\ \ \isacommand{ultimately}\isamarkupfalse%
\ \isacommand{have}\isamarkupfalse%
\isanewline
\ \ \ \ {\isachardoublequoteopen}{\isasymforall}A\ e{\isadigit{1}}\ e{\isadigit{2}}\ d{\isadigit{1}}\ d{\isadigit{2}}\ r{\isadigit{1}}\ r{\isadigit{2}}{\isachardot}{\kern0pt}\isanewline
\ \ \ \ \ \ \ \ {\isacharparenleft}{\kern0pt}well{\isacharunderscore}{\kern0pt}formed\ A\ {\isacharparenleft}{\kern0pt}e{\isadigit{1}}{\isacharcomma}{\kern0pt}\ r{\isadigit{1}}{\isacharcomma}{\kern0pt}\ d{\isadigit{1}}{\isacharparenright}{\kern0pt}\ {\isasymand}\ well{\isacharunderscore}{\kern0pt}formed\ A\ {\isacharparenleft}{\kern0pt}e{\isadigit{2}}{\isacharcomma}{\kern0pt}\ r{\isadigit{2}}{\isacharcomma}{\kern0pt}\ d{\isadigit{2}}{\isacharparenright}{\kern0pt}{\isacharparenright}{\kern0pt}\ {\isasymlongrightarrow}\isanewline
\ \ \ \ \ \ \ \ \ \ \ \ {\isacharparenleft}{\kern0pt}elect{\isacharunderscore}{\kern0pt}r\ {\isacharparenleft}{\kern0pt}max{\isacharunderscore}{\kern0pt}aggregator\ A\ {\isacharparenleft}{\kern0pt}e{\isadigit{1}}{\isacharcomma}{\kern0pt}\ r{\isadigit{1}}{\isacharcomma}{\kern0pt}\ d{\isadigit{1}}{\isacharparenright}{\kern0pt}\ {\isacharparenleft}{\kern0pt}e{\isadigit{2}}{\isacharcomma}{\kern0pt}\ r{\isadigit{2}}{\isacharcomma}{\kern0pt}\ d{\isadigit{2}}{\isacharparenright}{\kern0pt}{\isacharparenright}{\kern0pt}\ {\isasymsubseteq}\ {\isacharparenleft}{\kern0pt}e{\isadigit{1}}\ {\isasymunion}\ e{\isadigit{2}}{\isacharparenright}{\kern0pt}\ {\isasymand}\isanewline
\ \ \ \ \ \ \ \ \ \ \ \ \ reject{\isacharunderscore}{\kern0pt}r\ {\isacharparenleft}{\kern0pt}max{\isacharunderscore}{\kern0pt}aggregator\ A\ {\isacharparenleft}{\kern0pt}e{\isadigit{1}}{\isacharcomma}{\kern0pt}\ r{\isadigit{1}}{\isacharcomma}{\kern0pt}\ d{\isadigit{1}}{\isacharparenright}{\kern0pt}\ {\isacharparenleft}{\kern0pt}e{\isadigit{2}}{\isacharcomma}{\kern0pt}\ r{\isadigit{2}}{\isacharcomma}{\kern0pt}\ d{\isadigit{2}}{\isacharparenright}{\kern0pt}{\isacharparenright}{\kern0pt}\ {\isasymsubseteq}\ {\isacharparenleft}{\kern0pt}r{\isadigit{1}}\ {\isasymunion}\ r{\isadigit{2}}{\isacharparenright}{\kern0pt}{\isacharparenright}{\kern0pt}{\isachardoublequoteclose}\isanewline
\ \ \ \ \isacommand{by}\isamarkupfalse%
\ blast\isanewline
\ \ \isacommand{moreover}\isamarkupfalse%
\ \isacommand{have}\isamarkupfalse%
\isanewline
\ \ \ \ {\isachardoublequoteopen}{\isasymforall}A\ e{\isadigit{1}}\ e{\isadigit{2}}\ d{\isadigit{1}}\ d{\isadigit{2}}\ r{\isadigit{1}}\ r{\isadigit{2}}{\isachardot}{\kern0pt}\isanewline
\ \ \ \ \ \ \ \ {\isacharparenleft}{\kern0pt}well{\isacharunderscore}{\kern0pt}formed\ A\ {\isacharparenleft}{\kern0pt}e{\isadigit{1}}{\isacharcomma}{\kern0pt}\ r{\isadigit{1}}{\isacharcomma}{\kern0pt}\ d{\isadigit{1}}{\isacharparenright}{\kern0pt}\ {\isasymand}\ well{\isacharunderscore}{\kern0pt}formed\ A\ {\isacharparenleft}{\kern0pt}e{\isadigit{2}}{\isacharcomma}{\kern0pt}\ r{\isadigit{2}}{\isacharcomma}{\kern0pt}\ d{\isadigit{2}}{\isacharparenright}{\kern0pt}{\isacharparenright}{\kern0pt}\ {\isasymlongrightarrow}\isanewline
\ \ \ \ \ \ \ \ \ \ \ \ defer{\isacharunderscore}{\kern0pt}r\ {\isacharparenleft}{\kern0pt}max{\isacharunderscore}{\kern0pt}aggregator\ A\ {\isacharparenleft}{\kern0pt}e{\isadigit{1}}{\isacharcomma}{\kern0pt}\ r{\isadigit{1}}{\isacharcomma}{\kern0pt}\ d{\isadigit{1}}{\isacharparenright}{\kern0pt}\ {\isacharparenleft}{\kern0pt}e{\isadigit{2}}{\isacharcomma}{\kern0pt}\ r{\isadigit{2}}{\isacharcomma}{\kern0pt}\ d{\isadigit{2}}{\isacharparenright}{\kern0pt}{\isacharparenright}{\kern0pt}\ {\isasymsubseteq}\ {\isacharparenleft}{\kern0pt}d{\isadigit{1}}\ {\isasymunion}\ d{\isadigit{2}}{\isacharparenright}{\kern0pt}{\isachardoublequoteclose}\isanewline
\ \ \ \ \isacommand{by}\isamarkupfalse%
\ auto\isanewline
\ \ \isacommand{ultimately}\isamarkupfalse%
\ \isacommand{have}\isamarkupfalse%
\isanewline
\ \ \ \ {\isachardoublequoteopen}{\isasymforall}A\ e{\isadigit{1}}\ e{\isadigit{2}}\ d{\isadigit{1}}\ d{\isadigit{2}}\ r{\isadigit{1}}\ r{\isadigit{2}}{\isachardot}{\kern0pt}\isanewline
\ \ \ \ \ \ \ \ {\isacharparenleft}{\kern0pt}well{\isacharunderscore}{\kern0pt}formed\ A\ {\isacharparenleft}{\kern0pt}e{\isadigit{1}}{\isacharcomma}{\kern0pt}\ r{\isadigit{1}}{\isacharcomma}{\kern0pt}\ d{\isadigit{1}}{\isacharparenright}{\kern0pt}\ {\isasymand}\ well{\isacharunderscore}{\kern0pt}formed\ A\ {\isacharparenleft}{\kern0pt}e{\isadigit{2}}{\isacharcomma}{\kern0pt}\ r{\isadigit{2}}{\isacharcomma}{\kern0pt}\ d{\isadigit{2}}{\isacharparenright}{\kern0pt}{\isacharparenright}{\kern0pt}\ {\isasymlongrightarrow}\isanewline
\ \ \ \ \ \ \ \ \ \ \ \ {\isacharparenleft}{\kern0pt}elect{\isacharunderscore}{\kern0pt}r\ {\isacharparenleft}{\kern0pt}max{\isacharunderscore}{\kern0pt}aggregator\ A\ {\isacharparenleft}{\kern0pt}e{\isadigit{1}}{\isacharcomma}{\kern0pt}\ r{\isadigit{1}}{\isacharcomma}{\kern0pt}\ d{\isadigit{1}}{\isacharparenright}{\kern0pt}\ {\isacharparenleft}{\kern0pt}e{\isadigit{2}}{\isacharcomma}{\kern0pt}\ r{\isadigit{2}}{\isacharcomma}{\kern0pt}\ d{\isadigit{2}}{\isacharparenright}{\kern0pt}{\isacharparenright}{\kern0pt}\ {\isasymsubseteq}\ {\isacharparenleft}{\kern0pt}e{\isadigit{1}}\ {\isasymunion}\ e{\isadigit{2}}{\isacharparenright}{\kern0pt}\ {\isasymand}\isanewline
\ \ \ \ \ \ \ \ \ \ \ \ reject{\isacharunderscore}{\kern0pt}r\ {\isacharparenleft}{\kern0pt}max{\isacharunderscore}{\kern0pt}aggregator\ A\ {\isacharparenleft}{\kern0pt}e{\isadigit{1}}{\isacharcomma}{\kern0pt}\ r{\isadigit{1}}{\isacharcomma}{\kern0pt}\ d{\isadigit{1}}{\isacharparenright}{\kern0pt}\ {\isacharparenleft}{\kern0pt}e{\isadigit{2}}{\isacharcomma}{\kern0pt}\ r{\isadigit{2}}{\isacharcomma}{\kern0pt}\ d{\isadigit{2}}{\isacharparenright}{\kern0pt}{\isacharparenright}{\kern0pt}\ {\isasymsubseteq}\ {\isacharparenleft}{\kern0pt}r{\isadigit{1}}\ {\isasymunion}\ r{\isadigit{2}}{\isacharparenright}{\kern0pt}\ {\isasymand}\isanewline
\ \ \ \ \ \ \ \ \ \ \ \ defer{\isacharunderscore}{\kern0pt}r\ {\isacharparenleft}{\kern0pt}max{\isacharunderscore}{\kern0pt}aggregator\ A\ {\isacharparenleft}{\kern0pt}e{\isadigit{1}}{\isacharcomma}{\kern0pt}\ r{\isadigit{1}}{\isacharcomma}{\kern0pt}\ d{\isadigit{1}}{\isacharparenright}{\kern0pt}\ {\isacharparenleft}{\kern0pt}e{\isadigit{2}}{\isacharcomma}{\kern0pt}\ r{\isadigit{2}}{\isacharcomma}{\kern0pt}\ d{\isadigit{2}}{\isacharparenright}{\kern0pt}{\isacharparenright}{\kern0pt}\ {\isasymsubseteq}\ {\isacharparenleft}{\kern0pt}d{\isadigit{1}}\ {\isasymunion}\ d{\isadigit{2}}{\isacharparenright}{\kern0pt}{\isacharparenright}{\kern0pt}{\isachardoublequoteclose}\isanewline
\ \ \ \ \isacommand{by}\isamarkupfalse%
\ blast\isanewline
\ \ \isacommand{thus}\isamarkupfalse%
\ {\isacharquery}{\kern0pt}thesis\isanewline
\ \ \ \ \isacommand{by}\isamarkupfalse%
\ {\isacharparenleft}{\kern0pt}simp\ add{\isacharcolon}{\kern0pt}\ agg{\isacharunderscore}{\kern0pt}conservative{\isacharunderscore}{\kern0pt}def{\isacharparenright}{\kern0pt}\isanewline
\isacommand{qed}\isamarkupfalse%
%
\endisatagproof
{\isafoldproof}%
%
\isadelimproof
\isanewline
%
\endisadelimproof
%
\isadelimtheory
\isanewline
%
\endisadelimtheory
%
\isatagtheory
\isacommand{end}\isamarkupfalse%
%
\endisatagtheory
{\isafoldtheory}%
%
\isadelimtheory
%
\endisadelimtheory
%
\end{isabellebody}%
\endinput
%:%file=~/Documents/Studies/VotingRuleGenerator/virage/src/test/resources/verifiedVotingRuleConstruction/theories/Compositional_Framework/Composition_Rules/Aggregator_Facts.thy%:%
%:%10=1%:%
%:%11=1%:%
%:%12=2%:%
%:%13=3%:%
%:%14=4%:%
%:%15=5%:%
%:%20=5%:%
%:%23=6%:%
%:%24=7%:%
%:%25=8%:%
%:%26=8%:%
%:%29=9%:%
%:%33=9%:%
%:%34=9%:%
%:%35=10%:%
%:%36=10%:%
%:%37=11%:%
%:%38=11%:%
%:%39=12%:%
%:%40=12%:%
%:%41=13%:%
%:%42=13%:%
%:%43=14%:%
%:%44=14%:%
%:%45=15%:%
%:%46=16%:%
%:%47=17%:%
%:%48=18%:%
%:%49=19%:%
%:%50=20%:%
%:%51=21%:%
%:%52=22%:%
%:%53=22%:%
%:%54=23%:%
%:%55=24%:%
%:%56=25%:%
%:%57=25%:%
%:%58=26%:%
%:%64=26%:%
%:%67=27%:%
%:%68=28%:%
%:%69=29%:%
%:%70=30%:%
%:%71=30%:%
%:%78=31%:%
%:%79=31%:%
%:%80=32%:%
%:%81=32%:%
%:%82=33%:%
%:%84=35%:%
%:%85=36%:%
%:%86=36%:%
%:%87=37%:%
%:%88=37%:%
%:%89=38%:%
%:%90=38%:%
%:%91=39%:%
%:%93=41%:%
%:%94=42%:%
%:%95=42%:%
%:%96=43%:%
%:%97=43%:%
%:%98=43%:%
%:%99=44%:%
%:%101=46%:%
%:%102=47%:%
%:%103=47%:%
%:%104=48%:%
%:%105=48%:%
%:%106=48%:%
%:%107=49%:%
%:%110=52%:%
%:%111=53%:%
%:%112=53%:%
%:%113=54%:%
%:%114=54%:%
%:%115=54%:%
%:%116=55%:%
%:%118=57%:%
%:%119=58%:%
%:%120=58%:%
%:%121=59%:%
%:%122=59%:%
%:%123=59%:%
%:%124=60%:%
%:%128=64%:%
%:%129=65%:%
%:%130=65%:%
%:%131=66%:%
%:%132=66%:%
%:%133=67%:%
%:%134=67%:%
%:%135=68%:%
%:%141=68%:%
%:%146=69%:%
%:%151=70%:%
%
\begin{isabellebody}%
\setisabellecontext{Composite{\isacharunderscore}{\kern0pt}Structures}%
%
\isadelimtheory
%
\endisadelimtheory
%
\isatagtheory
\isacommand{theory}\isamarkupfalse%
\ Composite{\isacharunderscore}{\kern0pt}Structures\isanewline
\ \ \isakeyword{imports}\ {\isachardoublequoteopen}{\isachardot}{\kern0pt}{\isachardot}{\kern0pt}{\isacharslash}{\kern0pt}Electoral{\isacharunderscore}{\kern0pt}Module{\isachardoublequoteclose}\isanewline
\ \ \ \ \ \ \ \ \ \ {\isachardoublequoteopen}{\isachardot}{\kern0pt}{\isachardot}{\kern0pt}{\isacharslash}{\kern0pt}Basic{\isacharunderscore}{\kern0pt}Modules{\isacharslash}{\kern0pt}Elect{\isacharunderscore}{\kern0pt}Module{\isachardoublequoteclose}\isanewline
\ \ \ \ \ \ \ \ \ \ {\isachardoublequoteopen}{\isachardot}{\kern0pt}{\isachardot}{\kern0pt}{\isacharslash}{\kern0pt}Basic{\isacharunderscore}{\kern0pt}Modules{\isacharslash}{\kern0pt}Maximum{\isacharunderscore}{\kern0pt}Aggregator{\isachardoublequoteclose}\isanewline
\ \ \ \ \ \ \ \ \ \ {\isachardoublequoteopen}{\isachardot}{\kern0pt}{\isachardot}{\kern0pt}{\isacharslash}{\kern0pt}Basic{\isacharunderscore}{\kern0pt}Modules{\isacharslash}{\kern0pt}Defer{\isacharunderscore}{\kern0pt}Equal{\isacharunderscore}{\kern0pt}Condition{\isachardoublequoteclose}\isanewline
\ \ \ \ \ \ \ \ \ \ {\isachardoublequoteopen}{\isachardot}{\kern0pt}{\isachardot}{\kern0pt}{\isacharslash}{\kern0pt}Compositional{\isacharunderscore}{\kern0pt}Structures{\isacharslash}{\kern0pt}Sequential{\isacharunderscore}{\kern0pt}Composition{\isachardoublequoteclose}\isanewline
\ \ \ \ \ \ \ \ \ \ {\isachardoublequoteopen}{\isachardot}{\kern0pt}{\isachardot}{\kern0pt}{\isacharslash}{\kern0pt}Compositional{\isacharunderscore}{\kern0pt}Structures{\isacharslash}{\kern0pt}Parallel{\isacharunderscore}{\kern0pt}Composition{\isachardoublequoteclose}\isanewline
\ \ \ \ \ \ \ \ \ \ {\isachardoublequoteopen}{\isachardot}{\kern0pt}{\isachardot}{\kern0pt}{\isacharslash}{\kern0pt}Compositional{\isacharunderscore}{\kern0pt}Structures{\isacharslash}{\kern0pt}Loop{\isacharunderscore}{\kern0pt}Composition{\isachardoublequoteclose}\isanewline
\ \ \ \ \ \ \ \ \ \ {\isachardoublequoteopen}{\isachardot}{\kern0pt}{\isachardot}{\kern0pt}{\isacharslash}{\kern0pt}{\isachardot}{\kern0pt}{\isachardot}{\kern0pt}{\isacharslash}{\kern0pt}Properties{\isacharslash}{\kern0pt}Aggregator{\isacharunderscore}{\kern0pt}Properties{\isachardoublequoteclose}\isanewline
\ \ \ \ \ \ \ \ \ \ {\isachardoublequoteopen}{\isachardot}{\kern0pt}{\isachardot}{\kern0pt}{\isacharslash}{\kern0pt}{\isachardot}{\kern0pt}{\isachardot}{\kern0pt}{\isacharslash}{\kern0pt}Properties{\isacharslash}{\kern0pt}Disjoint{\isacharunderscore}{\kern0pt}Compatibility{\isachardoublequoteclose}\isanewline
\ \ \ \ \ \ \ \ \ \ {\isachardoublequoteopen}{\isachardot}{\kern0pt}{\isachardot}{\kern0pt}{\isacharslash}{\kern0pt}{\isachardot}{\kern0pt}{\isachardot}{\kern0pt}{\isacharslash}{\kern0pt}Composition{\isacharunderscore}{\kern0pt}Rules{\isacharslash}{\kern0pt}Aggregator{\isacharunderscore}{\kern0pt}Facts{\isachardoublequoteclose}\isanewline
\isanewline
\isakeyword{begin}%
\endisatagtheory
{\isafoldtheory}%
%
\isadelimtheory
%
\endisadelimtheory
%
\isadelimdocument
%
\endisadelimdocument
%
\isatagdocument
%
\isamarkupsection{Elect Composition%
}
\isamarkuptrue%
%
\endisatagdocument
{\isafolddocument}%
%
\isadelimdocument
%
\endisadelimdocument
%
\begin{isamarkuptext}%
The elect composition sequences an electoral module and the elect
module. It finalizes the module's decision as it simply elects all their
non-rejected alternatives. Thereby, any such elect-composed module induces
a proper voting rule in the social choice sense, as all alternatives are either
rejected or elected.%
\end{isamarkuptext}\isamarkuptrue%
%
\isadelimdocument
%
\endisadelimdocument
%
\isatagdocument
%
\isamarkupsubsection{Definition%
}
\isamarkuptrue%
%
\endisatagdocument
{\isafolddocument}%
%
\isadelimdocument
%
\endisadelimdocument
\isacommand{fun}\isamarkupfalse%
\ elector\ {\isacharcolon}{\kern0pt}{\isacharcolon}{\kern0pt}\ {\isachardoublequoteopen}{\isacharprime}{\kern0pt}a\ Electoral{\isacharunderscore}{\kern0pt}Module\ {\isasymRightarrow}\ {\isacharprime}{\kern0pt}a\ Electoral{\isacharunderscore}{\kern0pt}Module{\isachardoublequoteclose}\ \isakeyword{where}\isanewline
\ \ {\isachardoublequoteopen}elector\ m\ {\isacharequal}{\kern0pt}\ {\isacharparenleft}{\kern0pt}m\ {\isasymtriangleright}\ elect{\isacharunderscore}{\kern0pt}module{\isacharparenright}{\kern0pt}{\isachardoublequoteclose}%
\isadelimdocument
%
\endisadelimdocument
%
\isatagdocument
%
\isamarkupsubsection{Soundness%
}
\isamarkuptrue%
%
\endisatagdocument
{\isafolddocument}%
%
\isadelimdocument
%
\endisadelimdocument
\isacommand{theorem}\isamarkupfalse%
\ elector{\isacharunderscore}{\kern0pt}sound{\isacharbrackleft}{\kern0pt}simp{\isacharbrackright}{\kern0pt}{\isacharcolon}{\kern0pt}\isanewline
\ \ \isakeyword{assumes}\ module{\isacharunderscore}{\kern0pt}m{\isacharcolon}{\kern0pt}\ {\isachardoublequoteopen}electoral{\isacharunderscore}{\kern0pt}module\ m{\isachardoublequoteclose}\isanewline
\ \ \isakeyword{shows}\ {\isachardoublequoteopen}electoral{\isacharunderscore}{\kern0pt}module\ {\isacharparenleft}{\kern0pt}elector\ m{\isacharparenright}{\kern0pt}{\isachardoublequoteclose}\isanewline
%
\isadelimproof
\ \ %
\endisadelimproof
%
\isatagproof
\isacommand{by}\isamarkupfalse%
\ {\isacharparenleft}{\kern0pt}simp\ add{\isacharcolon}{\kern0pt}\ module{\isacharunderscore}{\kern0pt}m{\isacharparenright}{\kern0pt}%
\endisatagproof
{\isafoldproof}%
%
\isadelimproof
%
\endisadelimproof
%
\isadelimdocument
%
\endisadelimdocument
%
\isatagdocument
%
\isamarkupsection{Defer One Loop Composition%
}
\isamarkuptrue%
%
\endisatagdocument
{\isafolddocument}%
%
\isadelimdocument
%
\endisadelimdocument
%
\begin{isamarkuptext}%
This is a family of loop compositions. It uses the same module in sequence
until either no new decisions are made or only one alternative is remaining
in the defer-set. The second family herein uses the above family and
subsequently elects the remaining alternative.%
\end{isamarkuptext}\isamarkuptrue%
%
\isadelimdocument
%
\endisadelimdocument
%
\isatagdocument
%
\isamarkupsubsection{Definition%
}
\isamarkuptrue%
%
\endisatagdocument
{\isafolddocument}%
%
\isadelimdocument
%
\endisadelimdocument
\isacommand{fun}\isamarkupfalse%
\ iter\ {\isacharcolon}{\kern0pt}{\isacharcolon}{\kern0pt}\ {\isachardoublequoteopen}{\isacharprime}{\kern0pt}a\ Electoral{\isacharunderscore}{\kern0pt}Module\ {\isasymRightarrow}\ {\isacharprime}{\kern0pt}a\ Electoral{\isacharunderscore}{\kern0pt}Module{\isachardoublequoteclose}\ \isakeyword{where}\isanewline
\ \ {\isachardoublequoteopen}iter\ m\ {\isacharequal}{\kern0pt}\isanewline
\ \ \ \ {\isacharparenleft}{\kern0pt}let\ t\ {\isacharequal}{\kern0pt}\ defer{\isacharunderscore}{\kern0pt}equal{\isacharunderscore}{\kern0pt}condition\ {\isadigit{1}}\ in\isanewline
\ \ \ \ \ \ {\isacharparenleft}{\kern0pt}m\ {\isasymcirclearrowleft}\isactrlsub t{\isacharparenright}{\kern0pt}{\isacharparenright}{\kern0pt}{\isachardoublequoteclose}\isanewline
\isanewline
\isacommand{abbreviation}\isamarkupfalse%
\ defer{\isacharunderscore}{\kern0pt}one{\isacharunderscore}{\kern0pt}loop\ {\isacharcolon}{\kern0pt}{\isacharcolon}{\kern0pt}\isanewline
\ \ {\isachardoublequoteopen}{\isacharprime}{\kern0pt}a\ Electoral{\isacharunderscore}{\kern0pt}Module\ {\isasymRightarrow}\ {\isacharprime}{\kern0pt}a\ Electoral{\isacharunderscore}{\kern0pt}Module{\isachardoublequoteclose}\isanewline
\ \ \ \ {\isacharparenleft}{\kern0pt}{\isachardoublequoteopen}{\isacharunderscore}{\kern0pt}{\isasymcirclearrowleft}\isactrlsub {\isasymexists}\isactrlsub {\isacharbang}{\kern0pt}\isactrlsub d{\isachardoublequoteclose}\ {\isadigit{5}}{\isadigit{0}}{\isacharparenright}{\kern0pt}\ \isakeyword{where}\isanewline
\ \ {\isachardoublequoteopen}m\ {\isasymcirclearrowleft}\isactrlsub {\isasymexists}\isactrlsub {\isacharbang}{\kern0pt}\isactrlsub d\ {\isasymequiv}\ iter\ m{\isachardoublequoteclose}\isanewline
\isanewline
\isacommand{fun}\isamarkupfalse%
\ iterelect\ {\isacharcolon}{\kern0pt}{\isacharcolon}{\kern0pt}\ {\isachardoublequoteopen}{\isacharprime}{\kern0pt}a\ Electoral{\isacharunderscore}{\kern0pt}Module\ {\isasymRightarrow}\ {\isacharprime}{\kern0pt}a\ Electoral{\isacharunderscore}{\kern0pt}Module{\isachardoublequoteclose}\ \isakeyword{where}\isanewline
\ \ {\isachardoublequoteopen}iterelect\ m\ {\isacharequal}{\kern0pt}\ elector\ {\isacharparenleft}{\kern0pt}m\ {\isasymcirclearrowleft}\isactrlsub {\isasymexists}\isactrlsub {\isacharbang}{\kern0pt}\isactrlsub d{\isacharparenright}{\kern0pt}{\isachardoublequoteclose}%
\isadelimdocument
%
\endisadelimdocument
%
\isatagdocument
%
\isamarkupsection{Maximum Parallel Composition%
}
\isamarkuptrue%
%
\endisatagdocument
{\isafolddocument}%
%
\isadelimdocument
%
\endisadelimdocument
%
\begin{isamarkuptext}%
This is a family of parallel compositions. It composes a new electoral module
from two electoral modules combined with the maximum aggregator. Therein, the
two modules each make a decision and then a partition is returned where every
alternative receives the maximum result of the two input partitions. This means
that, if any alternative is elected by at least one of the modules, then it
gets elected, if any non-elected alternative is deferred by at least one of the
modules, then it gets deferred, only alternatives rejected by both modules get
rejected.%
\end{isamarkuptext}\isamarkuptrue%
%
\isadelimdocument
%
\endisadelimdocument
%
\isatagdocument
%
\isamarkupsubsection{Definition%
}
\isamarkuptrue%
%
\endisatagdocument
{\isafolddocument}%
%
\isadelimdocument
%
\endisadelimdocument
\isacommand{fun}\isamarkupfalse%
\ maximum{\isacharunderscore}{\kern0pt}parallel{\isacharunderscore}{\kern0pt}composition\ {\isacharcolon}{\kern0pt}{\isacharcolon}{\kern0pt}\ {\isachardoublequoteopen}{\isacharprime}{\kern0pt}a\ Electoral{\isacharunderscore}{\kern0pt}Module\ {\isasymRightarrow}\isanewline
\ \ \ \ \ \ \ \ {\isacharprime}{\kern0pt}a\ Electoral{\isacharunderscore}{\kern0pt}Module\ {\isasymRightarrow}\ {\isacharprime}{\kern0pt}a\ Electoral{\isacharunderscore}{\kern0pt}Module{\isachardoublequoteclose}\ \isakeyword{where}\isanewline
\ \ {\isachardoublequoteopen}maximum{\isacharunderscore}{\kern0pt}parallel{\isacharunderscore}{\kern0pt}composition\ m\ n\ {\isacharequal}{\kern0pt}\isanewline
\ \ \ \ {\isacharparenleft}{\kern0pt}let\ a\ {\isacharequal}{\kern0pt}\ max{\isacharunderscore}{\kern0pt}aggregator\ in\ {\isacharparenleft}{\kern0pt}m\ {\isasymparallel}\isactrlsub a\ n{\isacharparenright}{\kern0pt}{\isacharparenright}{\kern0pt}{\isachardoublequoteclose}\isanewline
\isanewline
\isacommand{abbreviation}\isamarkupfalse%
\ max{\isacharunderscore}{\kern0pt}parallel\ {\isacharcolon}{\kern0pt}{\isacharcolon}{\kern0pt}\ {\isachardoublequoteopen}{\isacharprime}{\kern0pt}a\ Electoral{\isacharunderscore}{\kern0pt}Module\ {\isasymRightarrow}\ {\isacharprime}{\kern0pt}a\ Electoral{\isacharunderscore}{\kern0pt}Module\ {\isasymRightarrow}\isanewline
\ \ \ \ \ \ \ \ {\isacharprime}{\kern0pt}a\ Electoral{\isacharunderscore}{\kern0pt}Module{\isachardoublequoteclose}\ {\isacharparenleft}{\kern0pt}\isakeyword{infix}\ {\isachardoublequoteopen}{\isasymparallel}\isactrlsub {\isasymup}{\isachardoublequoteclose}\ {\isadigit{5}}{\isadigit{0}}{\isacharparenright}{\kern0pt}\ \isakeyword{where}\isanewline
\ \ {\isachardoublequoteopen}m\ {\isasymparallel}\isactrlsub {\isasymup}\ n\ {\isacharequal}{\kern0pt}{\isacharequal}{\kern0pt}\ maximum{\isacharunderscore}{\kern0pt}parallel{\isacharunderscore}{\kern0pt}composition\ m\ n{\isachardoublequoteclose}%
\isadelimdocument
%
\endisadelimdocument
%
\isatagdocument
%
\isamarkupsubsection{Soundness%
}
\isamarkuptrue%
%
\endisatagdocument
{\isafolddocument}%
%
\isadelimdocument
%
\endisadelimdocument
\isacommand{theorem}\isamarkupfalse%
\ max{\isacharunderscore}{\kern0pt}par{\isacharunderscore}{\kern0pt}comp{\isacharunderscore}{\kern0pt}sound{\isacharcolon}{\kern0pt}\isanewline
\ \ \isakeyword{assumes}\isanewline
\ \ \ \ mod{\isacharunderscore}{\kern0pt}m{\isacharcolon}{\kern0pt}\ {\isachardoublequoteopen}electoral{\isacharunderscore}{\kern0pt}module\ m{\isachardoublequoteclose}\ \isakeyword{and}\isanewline
\ \ \ \ mod{\isacharunderscore}{\kern0pt}n{\isacharcolon}{\kern0pt}\ {\isachardoublequoteopen}electoral{\isacharunderscore}{\kern0pt}module\ n{\isachardoublequoteclose}\isanewline
\ \ \isakeyword{shows}\ {\isachardoublequoteopen}electoral{\isacharunderscore}{\kern0pt}module\ {\isacharparenleft}{\kern0pt}m\ {\isasymparallel}\isactrlsub {\isasymup}\ n{\isacharparenright}{\kern0pt}{\isachardoublequoteclose}\isanewline
%
\isadelimproof
\ \ %
\endisadelimproof
%
\isatagproof
\isacommand{using}\isamarkupfalse%
\ mod{\isacharunderscore}{\kern0pt}m\ mod{\isacharunderscore}{\kern0pt}n\isanewline
\ \ \isacommand{by}\isamarkupfalse%
\ simp%
\endisatagproof
{\isafoldproof}%
%
\isadelimproof
%
\endisadelimproof
%
\isadelimdocument
%
\endisadelimdocument
%
\isatagdocument
%
\isamarkupsubsection{Lemmata%
}
\isamarkuptrue%
%
\endisatagdocument
{\isafolddocument}%
%
\isadelimdocument
%
\endisadelimdocument
\isacommand{lemma}\isamarkupfalse%
\ max{\isacharunderscore}{\kern0pt}agg{\isacharunderscore}{\kern0pt}eq{\isacharunderscore}{\kern0pt}result{\isacharcolon}{\kern0pt}\isanewline
\ \ \isakeyword{assumes}\isanewline
\ \ \ \ module{\isacharunderscore}{\kern0pt}m{\isacharcolon}{\kern0pt}\ {\isachardoublequoteopen}electoral{\isacharunderscore}{\kern0pt}module\ m{\isachardoublequoteclose}\ \isakeyword{and}\isanewline
\ \ \ \ module{\isacharunderscore}{\kern0pt}n{\isacharcolon}{\kern0pt}\ {\isachardoublequoteopen}electoral{\isacharunderscore}{\kern0pt}module\ n{\isachardoublequoteclose}\ \isakeyword{and}\isanewline
\ \ \ \ f{\isacharunderscore}{\kern0pt}prof{\isacharcolon}{\kern0pt}\ {\isachardoublequoteopen}finite{\isacharunderscore}{\kern0pt}profile\ A\ p{\isachardoublequoteclose}\ \isakeyword{and}\isanewline
\ \ \ \ in{\isacharunderscore}{\kern0pt}A{\isacharcolon}{\kern0pt}\ {\isachardoublequoteopen}x\ {\isasymin}\ A{\isachardoublequoteclose}\isanewline
\ \ \isakeyword{shows}\isanewline
\ \ \ \ {\isachardoublequoteopen}mod{\isacharunderscore}{\kern0pt}contains{\isacharunderscore}{\kern0pt}result\ {\isacharparenleft}{\kern0pt}m\ {\isasymparallel}\isactrlsub {\isasymup}\ n{\isacharparenright}{\kern0pt}\ m\ A\ p\ x\ {\isasymor}\isanewline
\ \ \ \ \ \ mod{\isacharunderscore}{\kern0pt}contains{\isacharunderscore}{\kern0pt}result\ {\isacharparenleft}{\kern0pt}m\ {\isasymparallel}\isactrlsub {\isasymup}\ n{\isacharparenright}{\kern0pt}\ n\ A\ p\ x{\isachardoublequoteclose}\isanewline
%
\isadelimproof
%
\endisadelimproof
%
\isatagproof
\isacommand{proof}\isamarkupfalse%
\ cases\isanewline
\ \ \isacommand{assume}\isamarkupfalse%
\ a{\isadigit{1}}{\isacharcolon}{\kern0pt}\ {\isachardoublequoteopen}x\ {\isasymin}\ elect\ {\isacharparenleft}{\kern0pt}m\ {\isasymparallel}\isactrlsub {\isasymup}\ n{\isacharparenright}{\kern0pt}\ A\ p{\isachardoublequoteclose}\isanewline
\ \ \isacommand{hence}\isamarkupfalse%
\isanewline
\ \ \ \ {\isachardoublequoteopen}let\ {\isacharparenleft}{\kern0pt}e{\isadigit{1}}{\isacharcomma}{\kern0pt}\ r{\isadigit{1}}{\isacharcomma}{\kern0pt}\ d{\isadigit{1}}{\isacharparenright}{\kern0pt}\ {\isacharequal}{\kern0pt}\ m\ A\ p{\isacharsemicolon}{\kern0pt}\isanewline
\ \ \ \ \ \ \ \ {\isacharparenleft}{\kern0pt}e{\isadigit{2}}{\isacharcomma}{\kern0pt}\ r{\isadigit{2}}{\isacharcomma}{\kern0pt}\ d{\isadigit{2}}{\isacharparenright}{\kern0pt}\ {\isacharequal}{\kern0pt}\ n\ A\ p\ in\isanewline
\ \ \ \ \ \ x\ {\isasymin}\ e{\isadigit{1}}\ {\isasymunion}\ e{\isadigit{2}}{\isachardoublequoteclose}\isanewline
\ \ \ \ \isacommand{by}\isamarkupfalse%
\ auto\isanewline
\ \ \isacommand{hence}\isamarkupfalse%
\ {\isachardoublequoteopen}x\ {\isasymin}\ {\isacharparenleft}{\kern0pt}elect\ m\ A\ p{\isacharparenright}{\kern0pt}\ {\isasymunion}\ {\isacharparenleft}{\kern0pt}elect\ n\ A\ p{\isacharparenright}{\kern0pt}{\isachardoublequoteclose}\isanewline
\ \ \ \ \isacommand{by}\isamarkupfalse%
\ auto\isanewline
\ \ \isacommand{thus}\isamarkupfalse%
\ {\isacharquery}{\kern0pt}thesis\isanewline
\ \ \ \ \isacommand{using}\isamarkupfalse%
\ IntI\ Un{\isacharunderscore}{\kern0pt}iff\ a{\isadigit{1}}\ empty{\isacharunderscore}{\kern0pt}iff\ mod{\isacharunderscore}{\kern0pt}contains{\isacharunderscore}{\kern0pt}result{\isacharunderscore}{\kern0pt}def\isanewline
\ \ \ \ \ \ \ \ \ \ in{\isacharunderscore}{\kern0pt}A\ max{\isacharunderscore}{\kern0pt}agg{\isacharunderscore}{\kern0pt}sound\ module{\isacharunderscore}{\kern0pt}m\ module{\isacharunderscore}{\kern0pt}n\ par{\isacharunderscore}{\kern0pt}comp{\isacharunderscore}{\kern0pt}sound\isanewline
\ \ \ \ \ \ \ \ \ \ f{\isacharunderscore}{\kern0pt}prof\ result{\isacharunderscore}{\kern0pt}disj\ maximum{\isacharunderscore}{\kern0pt}parallel{\isacharunderscore}{\kern0pt}composition{\isachardot}{\kern0pt}simps\isanewline
\ \ \ \ \isacommand{by}\isamarkupfalse%
\ {\isacharparenleft}{\kern0pt}smt\ {\isacharparenleft}{\kern0pt}verit{\isacharcomma}{\kern0pt}\ ccfv{\isacharunderscore}{\kern0pt}threshold{\isacharparenright}{\kern0pt}{\isacharparenright}{\kern0pt}\isanewline
\isacommand{next}\isamarkupfalse%
\isanewline
\ \ \isacommand{assume}\isamarkupfalse%
\ not{\isacharunderscore}{\kern0pt}a{\isadigit{1}}{\isacharcolon}{\kern0pt}\ {\isachardoublequoteopen}x\ {\isasymnotin}\ elect\ {\isacharparenleft}{\kern0pt}m\ {\isasymparallel}\isactrlsub {\isasymup}\ n{\isacharparenright}{\kern0pt}\ A\ p{\isachardoublequoteclose}\isanewline
\ \ \isacommand{thus}\isamarkupfalse%
\ {\isacharquery}{\kern0pt}thesis\isanewline
\ \ \isacommand{proof}\isamarkupfalse%
\ cases\isanewline
\ \ \ \ \isacommand{assume}\isamarkupfalse%
\ a{\isadigit{2}}{\isacharcolon}{\kern0pt}\ {\isachardoublequoteopen}x\ {\isasymin}\ defer\ {\isacharparenleft}{\kern0pt}m\ {\isasymparallel}\isactrlsub {\isasymup}\ n{\isacharparenright}{\kern0pt}\ A\ p{\isachardoublequoteclose}\isanewline
\ \ \ \ \isacommand{thus}\isamarkupfalse%
\ {\isacharquery}{\kern0pt}thesis\isanewline
\ \ \ \ \ \ \isacommand{using}\isamarkupfalse%
\ CollectD\ DiffD{\isadigit{1}}\ DiffD{\isadigit{2}}\ max{\isacharunderscore}{\kern0pt}aggregator{\isachardot}{\kern0pt}simps\ Un{\isacharunderscore}{\kern0pt}iff\isanewline
\ \ \ \ \ \ \ \ \ \ \ \ case{\isacharunderscore}{\kern0pt}prod{\isacharunderscore}{\kern0pt}conv\ defer{\isacharunderscore}{\kern0pt}not{\isacharunderscore}{\kern0pt}elec{\isacharunderscore}{\kern0pt}or{\isacharunderscore}{\kern0pt}rej\ max{\isacharunderscore}{\kern0pt}agg{\isacharunderscore}{\kern0pt}sound\isanewline
\ \ \ \ \ \ \ \ \ \ \ \ mod{\isacharunderscore}{\kern0pt}contains{\isacharunderscore}{\kern0pt}result{\isacharunderscore}{\kern0pt}def\ module{\isacharunderscore}{\kern0pt}m\ module{\isacharunderscore}{\kern0pt}n\ par{\isacharunderscore}{\kern0pt}comp{\isacharunderscore}{\kern0pt}sound\isanewline
\ \ \ \ \ \ \ \ \ \ \ \ parallel{\isacharunderscore}{\kern0pt}composition{\isachardot}{\kern0pt}simps\ prod{\isachardot}{\kern0pt}collapse\ f{\isacharunderscore}{\kern0pt}prof\ sndI\isanewline
\ \ \ \ \ \ \ \ \ \ \ \ Int{\isacharunderscore}{\kern0pt}iff\ electoral{\isacharunderscore}{\kern0pt}mod{\isacharunderscore}{\kern0pt}defer{\isacharunderscore}{\kern0pt}elem\ electoral{\isacharunderscore}{\kern0pt}module{\isacharunderscore}{\kern0pt}def\isanewline
\ \ \ \ \ \ \ \ \ \ \ \ max{\isacharunderscore}{\kern0pt}agg{\isacharunderscore}{\kern0pt}rej{\isacharunderscore}{\kern0pt}set\ prod{\isachardot}{\kern0pt}sel{\isacharparenleft}{\kern0pt}{\isadigit{1}}{\isacharparenright}{\kern0pt}\ maximum{\isacharunderscore}{\kern0pt}parallel{\isacharunderscore}{\kern0pt}composition{\isachardot}{\kern0pt}simps\isanewline
\ \ \ \ \ \ \isacommand{by}\isamarkupfalse%
\ {\isacharparenleft}{\kern0pt}smt\ {\isacharparenleft}{\kern0pt}verit{\isacharcomma}{\kern0pt}\ del{\isacharunderscore}{\kern0pt}insts{\isacharparenright}{\kern0pt}{\isacharparenright}{\kern0pt}\isanewline
\ \ \isacommand{next}\isamarkupfalse%
\isanewline
\ \ \ \ \isacommand{assume}\isamarkupfalse%
\ not{\isacharunderscore}{\kern0pt}a{\isadigit{2}}{\isacharcolon}{\kern0pt}\ {\isachardoublequoteopen}x\ {\isasymnotin}\ defer\ {\isacharparenleft}{\kern0pt}m\ {\isasymparallel}\isactrlsub {\isasymup}\ n{\isacharparenright}{\kern0pt}\ A\ p{\isachardoublequoteclose}\isanewline
\ \ \ \ \isacommand{with}\isamarkupfalse%
\ not{\isacharunderscore}{\kern0pt}a{\isadigit{1}}\ \isacommand{have}\isamarkupfalse%
\ a{\isadigit{3}}{\isacharcolon}{\kern0pt}\isanewline
\ \ \ \ \ \ {\isachardoublequoteopen}x\ {\isasymin}\ reject\ {\isacharparenleft}{\kern0pt}m\ {\isasymparallel}\isactrlsub {\isasymup}\ n{\isacharparenright}{\kern0pt}\ A\ p{\isachardoublequoteclose}\isanewline
\ \ \ \ \ \ \isacommand{using}\isamarkupfalse%
\ electoral{\isacharunderscore}{\kern0pt}mod{\isacharunderscore}{\kern0pt}defer{\isacharunderscore}{\kern0pt}elem\ in{\isacharunderscore}{\kern0pt}A\ max{\isacharunderscore}{\kern0pt}agg{\isacharunderscore}{\kern0pt}sound\ module{\isacharunderscore}{\kern0pt}m\ module{\isacharunderscore}{\kern0pt}n\isanewline
\ \ \ \ \ \ \ \ \ \ \ \ par{\isacharunderscore}{\kern0pt}comp{\isacharunderscore}{\kern0pt}sound\ f{\isacharunderscore}{\kern0pt}prof\ maximum{\isacharunderscore}{\kern0pt}parallel{\isacharunderscore}{\kern0pt}composition{\isachardot}{\kern0pt}simps\isanewline
\ \ \ \ \ \ \isacommand{by}\isamarkupfalse%
\ metis\isanewline
\ \ \ \ \isacommand{hence}\isamarkupfalse%
\isanewline
\ \ \ \ \ \ {\isachardoublequoteopen}let\ {\isacharparenleft}{\kern0pt}e{\isadigit{1}}{\isacharcomma}{\kern0pt}\ r{\isadigit{1}}{\isacharcomma}{\kern0pt}\ d{\isadigit{1}}{\isacharparenright}{\kern0pt}\ {\isacharequal}{\kern0pt}\ m\ A\ p{\isacharsemicolon}{\kern0pt}\isanewline
\ \ \ \ \ \ \ \ \ \ {\isacharparenleft}{\kern0pt}e{\isadigit{2}}{\isacharcomma}{\kern0pt}\ r{\isadigit{2}}{\isacharcomma}{\kern0pt}\ d{\isadigit{2}}{\isacharparenright}{\kern0pt}\ {\isacharequal}{\kern0pt}\ n\ A\ p\ in\isanewline
\ \ \ \ \ \ \ \ x\ {\isasymin}\ fst\ {\isacharparenleft}{\kern0pt}snd\ {\isacharparenleft}{\kern0pt}max{\isacharunderscore}{\kern0pt}aggregator\ A\ {\isacharparenleft}{\kern0pt}e{\isadigit{1}}{\isacharcomma}{\kern0pt}\ r{\isadigit{1}}{\isacharcomma}{\kern0pt}\ d{\isadigit{1}}{\isacharparenright}{\kern0pt}\ {\isacharparenleft}{\kern0pt}e{\isadigit{2}}{\isacharcomma}{\kern0pt}\ r{\isadigit{2}}{\isacharcomma}{\kern0pt}\ d{\isadigit{2}}{\isacharparenright}{\kern0pt}{\isacharparenright}{\kern0pt}{\isacharparenright}{\kern0pt}{\isachardoublequoteclose}\isanewline
\ \ \ \ \ \ \isacommand{using}\isamarkupfalse%
\ case{\isacharunderscore}{\kern0pt}prod{\isacharunderscore}{\kern0pt}unfold\ parallel{\isacharunderscore}{\kern0pt}composition{\isachardot}{\kern0pt}simps\isanewline
\ \ \ \ \ \ \ \ \ \ \ \ surjective{\isacharunderscore}{\kern0pt}pairing\ maximum{\isacharunderscore}{\kern0pt}parallel{\isacharunderscore}{\kern0pt}composition{\isachardot}{\kern0pt}simps\isanewline
\ \ \ \ \ \ \isacommand{by}\isamarkupfalse%
\ {\isacharparenleft}{\kern0pt}smt\ {\isacharparenleft}{\kern0pt}verit{\isacharcomma}{\kern0pt}\ ccfv{\isacharunderscore}{\kern0pt}threshold{\isacharparenright}{\kern0pt}{\isacharparenright}{\kern0pt}\isanewline
\ \ \ \ \isacommand{hence}\isamarkupfalse%
\isanewline
\ \ \ \ \ \ {\isachardoublequoteopen}let\ {\isacharparenleft}{\kern0pt}e{\isadigit{1}}{\isacharcomma}{\kern0pt}\ r{\isadigit{1}}{\isacharcomma}{\kern0pt}\ d{\isadigit{1}}{\isacharparenright}{\kern0pt}\ {\isacharequal}{\kern0pt}\ m\ A\ p{\isacharsemicolon}{\kern0pt}\isanewline
\ \ \ \ \ \ \ \ \ \ {\isacharparenleft}{\kern0pt}e{\isadigit{2}}{\isacharcomma}{\kern0pt}\ r{\isadigit{2}}{\isacharcomma}{\kern0pt}\ d{\isadigit{2}}{\isacharparenright}{\kern0pt}\ {\isacharequal}{\kern0pt}\ n\ A\ p\ in\isanewline
\ \ \ \ \ \ \ \ x\ {\isasymin}\ A\ {\isacharminus}{\kern0pt}\ {\isacharparenleft}{\kern0pt}e{\isadigit{1}}\ {\isasymunion}\ e{\isadigit{2}}\ {\isasymunion}\ d{\isadigit{1}}\ {\isasymunion}\ d{\isadigit{2}}{\isacharparenright}{\kern0pt}{\isachardoublequoteclose}\isanewline
\ \ \ \ \ \ \isacommand{by}\isamarkupfalse%
\ simp\isanewline
\ \ \ \ \isacommand{thus}\isamarkupfalse%
\ {\isacharquery}{\kern0pt}thesis\isanewline
\ \ \ \ \ \ \isacommand{using}\isamarkupfalse%
\ Un{\isacharunderscore}{\kern0pt}iff\ combine{\isacharunderscore}{\kern0pt}ele{\isacharunderscore}{\kern0pt}rej{\isacharunderscore}{\kern0pt}def\ agg{\isacharunderscore}{\kern0pt}conservative{\isacharunderscore}{\kern0pt}def\isanewline
\ \ \ \ \ \ \ \ \ \ \ \ contra{\isacharunderscore}{\kern0pt}subsetD\ disjoint{\isacharunderscore}{\kern0pt}iff{\isacharunderscore}{\kern0pt}not{\isacharunderscore}{\kern0pt}equal\ in{\isacharunderscore}{\kern0pt}A\isanewline
\ \ \ \ \ \ \ \ \ \ \ \ electoral{\isacharunderscore}{\kern0pt}module{\isacharunderscore}{\kern0pt}def\ mod{\isacharunderscore}{\kern0pt}contains{\isacharunderscore}{\kern0pt}result{\isacharunderscore}{\kern0pt}def\isanewline
\ \ \ \ \ \ \ \ \ \ \ \ max{\isacharunderscore}{\kern0pt}agg{\isacharunderscore}{\kern0pt}consv\ module{\isacharunderscore}{\kern0pt}m\ module{\isacharunderscore}{\kern0pt}n\ par{\isacharunderscore}{\kern0pt}comp{\isacharunderscore}{\kern0pt}sound\isanewline
\ \ \ \ \ \ \ \ \ \ \ \ parallel{\isacharunderscore}{\kern0pt}composition{\isachardot}{\kern0pt}simps\ f{\isacharunderscore}{\kern0pt}prof\ result{\isacharunderscore}{\kern0pt}disj\isanewline
\ \ \ \ \ \ \ \ \ \ \ \ max{\isacharunderscore}{\kern0pt}agg{\isacharunderscore}{\kern0pt}rej{\isacharunderscore}{\kern0pt}set\ not{\isacharunderscore}{\kern0pt}a{\isadigit{1}}\ not{\isacharunderscore}{\kern0pt}a{\isadigit{2}}\ Int{\isacharunderscore}{\kern0pt}iff\isanewline
\ \ \ \ \ \ \ \ \ \ \ \ maximum{\isacharunderscore}{\kern0pt}parallel{\isacharunderscore}{\kern0pt}composition{\isachardot}{\kern0pt}simps\isanewline
\ \ \ \ \ \ \isacommand{by}\isamarkupfalse%
\ {\isacharparenleft}{\kern0pt}smt\ {\isacharparenleft}{\kern0pt}verit{\isacharcomma}{\kern0pt}\ del{\isacharunderscore}{\kern0pt}insts{\isacharparenright}{\kern0pt}{\isacharparenright}{\kern0pt}\isanewline
\ \ \isacommand{qed}\isamarkupfalse%
\isanewline
\isacommand{qed}\isamarkupfalse%
%
\endisatagproof
{\isafoldproof}%
%
\isadelimproof
\isanewline
%
\endisadelimproof
\isanewline
\isacommand{lemma}\isamarkupfalse%
\ max{\isacharunderscore}{\kern0pt}agg{\isacharunderscore}{\kern0pt}rej{\isacharunderscore}{\kern0pt}iff{\isacharunderscore}{\kern0pt}both{\isacharunderscore}{\kern0pt}reject{\isacharcolon}{\kern0pt}\isanewline
\ \ \isakeyword{assumes}\isanewline
\ \ \ \ f{\isacharunderscore}{\kern0pt}prof{\isacharcolon}{\kern0pt}\ {\isachardoublequoteopen}finite{\isacharunderscore}{\kern0pt}profile\ A\ p{\isachardoublequoteclose}\ \isakeyword{and}\isanewline
\ \ \ \ module{\isacharunderscore}{\kern0pt}m{\isacharcolon}{\kern0pt}\ {\isachardoublequoteopen}electoral{\isacharunderscore}{\kern0pt}module\ m{\isachardoublequoteclose}\ \isakeyword{and}\isanewline
\ \ \ \ module{\isacharunderscore}{\kern0pt}n{\isacharcolon}{\kern0pt}\ {\isachardoublequoteopen}electoral{\isacharunderscore}{\kern0pt}module\ n{\isachardoublequoteclose}\isanewline
\ \ \isakeyword{shows}\isanewline
\ \ \ \ {\isachardoublequoteopen}x\ {\isasymin}\ reject\ {\isacharparenleft}{\kern0pt}m\ {\isasymparallel}\isactrlsub {\isasymup}\ n{\isacharparenright}{\kern0pt}\ A\ p\ {\isasymlongleftrightarrow}\isanewline
\ \ \ \ \ \ {\isacharparenleft}{\kern0pt}x\ {\isasymin}\ reject\ m\ A\ p\ {\isasymand}\ x\ {\isasymin}\ reject\ n\ A\ p{\isacharparenright}{\kern0pt}{\isachardoublequoteclose}\isanewline
%
\isadelimproof
%
\endisadelimproof
%
\isatagproof
\isacommand{proof}\isamarkupfalse%
\ {\isacharminus}{\kern0pt}\isanewline
\ \ \isacommand{have}\isamarkupfalse%
\isanewline
\ \ \ \ {\isachardoublequoteopen}x\ {\isasymin}\ reject\ {\isacharparenleft}{\kern0pt}m\ {\isasymparallel}\isactrlsub {\isasymup}\ n{\isacharparenright}{\kern0pt}\ A\ p\ {\isasymlongrightarrow}\isanewline
\ \ \ \ \ \ {\isacharparenleft}{\kern0pt}x\ {\isasymin}\ reject\ m\ A\ p\ {\isasymand}\ x\ {\isasymin}\ reject\ n\ A\ p{\isacharparenright}{\kern0pt}{\isachardoublequoteclose}\isanewline
\ \ \isacommand{proof}\isamarkupfalse%
\isanewline
\ \ \ \ \isacommand{assume}\isamarkupfalse%
\ a{\isacharcolon}{\kern0pt}\ {\isachardoublequoteopen}x\ {\isasymin}\ reject\ {\isacharparenleft}{\kern0pt}m\ {\isasymparallel}\isactrlsub {\isasymup}\ n{\isacharparenright}{\kern0pt}\ A\ p{\isachardoublequoteclose}\isanewline
\ \ \ \ \isacommand{hence}\isamarkupfalse%
\isanewline
\ \ \ \ \ \ {\isachardoublequoteopen}let\ {\isacharparenleft}{\kern0pt}e{\isadigit{1}}{\isacharcomma}{\kern0pt}\ r{\isadigit{1}}{\isacharcomma}{\kern0pt}\ d{\isadigit{1}}{\isacharparenright}{\kern0pt}\ {\isacharequal}{\kern0pt}\ m\ A\ p{\isacharsemicolon}{\kern0pt}\isanewline
\ \ \ \ \ \ \ \ \ \ {\isacharparenleft}{\kern0pt}e{\isadigit{2}}{\isacharcomma}{\kern0pt}\ r{\isadigit{2}}{\isacharcomma}{\kern0pt}\ d{\isadigit{2}}{\isacharparenright}{\kern0pt}\ {\isacharequal}{\kern0pt}\ n\ A\ p\ in\isanewline
\ \ \ \ \ \ \ \ x\ {\isasymin}\ fst\ {\isacharparenleft}{\kern0pt}snd\ {\isacharparenleft}{\kern0pt}max{\isacharunderscore}{\kern0pt}aggregator\ A\ {\isacharparenleft}{\kern0pt}e{\isadigit{1}}{\isacharcomma}{\kern0pt}\ r{\isadigit{1}}{\isacharcomma}{\kern0pt}\ d{\isadigit{1}}{\isacharparenright}{\kern0pt}\ {\isacharparenleft}{\kern0pt}e{\isadigit{2}}{\isacharcomma}{\kern0pt}\ r{\isadigit{2}}{\isacharcomma}{\kern0pt}\ d{\isadigit{2}}{\isacharparenright}{\kern0pt}{\isacharparenright}{\kern0pt}{\isacharparenright}{\kern0pt}{\isachardoublequoteclose}\isanewline
\ \ \ \ \ \ \isacommand{using}\isamarkupfalse%
\ case{\isacharunderscore}{\kern0pt}prodI{\isadigit{2}}\ maximum{\isacharunderscore}{\kern0pt}parallel{\isacharunderscore}{\kern0pt}composition{\isachardot}{\kern0pt}simps\ split{\isacharunderscore}{\kern0pt}def\isanewline
\ \ \ \ \ \ \ \ \ \ \ \ parallel{\isacharunderscore}{\kern0pt}composition{\isachardot}{\kern0pt}simps\ prod{\isachardot}{\kern0pt}collapse\ split{\isacharunderscore}{\kern0pt}beta\isanewline
\ \ \ \ \ \ \isacommand{by}\isamarkupfalse%
\ {\isacharparenleft}{\kern0pt}smt\ {\isacharparenleft}{\kern0pt}verit{\isacharcomma}{\kern0pt}\ ccfv{\isacharunderscore}{\kern0pt}threshold{\isacharparenright}{\kern0pt}{\isacharparenright}{\kern0pt}\isanewline
\ \ \ \ \isacommand{hence}\isamarkupfalse%
\isanewline
\ \ \ \ \ \ {\isachardoublequoteopen}let\ {\isacharparenleft}{\kern0pt}e{\isadigit{1}}{\isacharcomma}{\kern0pt}\ r{\isadigit{1}}{\isacharcomma}{\kern0pt}\ d{\isadigit{1}}{\isacharparenright}{\kern0pt}\ {\isacharequal}{\kern0pt}\ m\ A\ p{\isacharsemicolon}{\kern0pt}\isanewline
\ \ \ \ \ \ \ \ \ \ {\isacharparenleft}{\kern0pt}e{\isadigit{2}}{\isacharcomma}{\kern0pt}\ r{\isadigit{2}}{\isacharcomma}{\kern0pt}\ d{\isadigit{2}}{\isacharparenright}{\kern0pt}\ {\isacharequal}{\kern0pt}\ n\ A\ p\ in\isanewline
\ \ \ \ \ \ \ \ x\ {\isasymin}\ A\ {\isacharminus}{\kern0pt}\ {\isacharparenleft}{\kern0pt}e{\isadigit{1}}\ {\isasymunion}\ e{\isadigit{2}}\ {\isasymunion}\ d{\isadigit{1}}\ {\isasymunion}\ d{\isadigit{2}}{\isacharparenright}{\kern0pt}{\isachardoublequoteclose}\isanewline
\ \ \ \ \ \ \isacommand{by}\isamarkupfalse%
\ simp\isanewline
\ \ \ \ \isacommand{thus}\isamarkupfalse%
\ {\isachardoublequoteopen}x\ {\isasymin}\ reject\ m\ A\ p\ {\isasymand}\ x\ {\isasymin}\ reject\ n\ A\ p{\isachardoublequoteclose}\isanewline
\ \ \ \ \ \ \isacommand{using}\isamarkupfalse%
\ Int{\isacharunderscore}{\kern0pt}iff\ a\ electoral{\isacharunderscore}{\kern0pt}module{\isacharunderscore}{\kern0pt}def\ max{\isacharunderscore}{\kern0pt}agg{\isacharunderscore}{\kern0pt}rej{\isacharunderscore}{\kern0pt}set\ module{\isacharunderscore}{\kern0pt}m\isanewline
\ \ \ \ \ \ \ \ \ \ \ \ module{\isacharunderscore}{\kern0pt}n\ parallel{\isacharunderscore}{\kern0pt}composition{\isachardot}{\kern0pt}simps\ surjective{\isacharunderscore}{\kern0pt}pairing\isanewline
\ \ \ \ \ \ \ \ \ \ \ \ maximum{\isacharunderscore}{\kern0pt}parallel{\isacharunderscore}{\kern0pt}composition{\isachardot}{\kern0pt}simps\ f{\isacharunderscore}{\kern0pt}prof\isanewline
\ \ \ \ \ \ \isacommand{by}\isamarkupfalse%
\ {\isacharparenleft}{\kern0pt}smt\ {\isacharparenleft}{\kern0pt}verit{\isacharcomma}{\kern0pt}\ best{\isacharparenright}{\kern0pt}{\isacharparenright}{\kern0pt}\isanewline
\ \ \isacommand{qed}\isamarkupfalse%
\isanewline
\ \ \isacommand{moreover}\isamarkupfalse%
\ \isacommand{have}\isamarkupfalse%
\isanewline
\ \ \ \ {\isachardoublequoteopen}{\isacharparenleft}{\kern0pt}x\ {\isasymin}\ reject\ m\ A\ p\ {\isasymand}\ x\ {\isasymin}\ reject\ n\ A\ p{\isacharparenright}{\kern0pt}\ {\isasymlongrightarrow}\isanewline
\ \ \ \ \ \ \ \ x\ {\isasymin}\ reject\ {\isacharparenleft}{\kern0pt}m\ {\isasymparallel}\isactrlsub {\isasymup}\ n{\isacharparenright}{\kern0pt}\ A\ p{\isachardoublequoteclose}\isanewline
\ \ \isacommand{proof}\isamarkupfalse%
\isanewline
\ \ \ \ \isacommand{assume}\isamarkupfalse%
\ a{\isacharcolon}{\kern0pt}\ {\isachardoublequoteopen}x\ {\isasymin}\ reject\ m\ A\ p\ {\isasymand}\ x\ {\isasymin}\ reject\ n\ A\ p{\isachardoublequoteclose}\isanewline
\ \ \ \ \isacommand{hence}\isamarkupfalse%
\isanewline
\ \ \ \ \ \ {\isachardoublequoteopen}x\ {\isasymnotin}\ elect\ m\ A\ p\ {\isasymand}\ x\ {\isasymnotin}\ defer\ m\ A\ p\ {\isasymand}\isanewline
\ \ \ \ \ \ \ \ x\ {\isasymnotin}\ elect\ n\ A\ p\ {\isasymand}\ x\ {\isasymnotin}\ defer\ n\ A\ p{\isachardoublequoteclose}\isanewline
\ \ \ \ \ \ \isacommand{using}\isamarkupfalse%
\ IntI\ empty{\isacharunderscore}{\kern0pt}iff\ module{\isacharunderscore}{\kern0pt}m\ module{\isacharunderscore}{\kern0pt}n\ f{\isacharunderscore}{\kern0pt}prof\ result{\isacharunderscore}{\kern0pt}disj\isanewline
\ \ \ \ \ \ \isacommand{by}\isamarkupfalse%
\ metis\isanewline
\ \ \ \ \isacommand{thus}\isamarkupfalse%
\ {\isachardoublequoteopen}x\ {\isasymin}\ reject\ {\isacharparenleft}{\kern0pt}m\ {\isasymparallel}\isactrlsub {\isasymup}\ n{\isacharparenright}{\kern0pt}\ A\ p{\isachardoublequoteclose}\isanewline
\ \ \ \ \ \ \isacommand{using}\isamarkupfalse%
\ CollectD\ DiffD{\isadigit{1}}\ max{\isacharunderscore}{\kern0pt}aggregator{\isachardot}{\kern0pt}simps\ Un{\isacharunderscore}{\kern0pt}iff\ a\isanewline
\ \ \ \ \ \ \ \ \ \ \ \ electoral{\isacharunderscore}{\kern0pt}mod{\isacharunderscore}{\kern0pt}defer{\isacharunderscore}{\kern0pt}elem\ prod{\isachardot}{\kern0pt}simps\ max{\isacharunderscore}{\kern0pt}agg{\isacharunderscore}{\kern0pt}sound\isanewline
\ \ \ \ \ \ \ \ \ \ \ \ module{\isacharunderscore}{\kern0pt}m\ module{\isacharunderscore}{\kern0pt}n\ f{\isacharunderscore}{\kern0pt}prof\ old{\isachardot}{\kern0pt}prod{\isachardot}{\kern0pt}inject\ par{\isacharunderscore}{\kern0pt}comp{\isacharunderscore}{\kern0pt}sound\isanewline
\ \ \ \ \ \ \ \ \ \ \ \ prod{\isachardot}{\kern0pt}collapse\ parallel{\isacharunderscore}{\kern0pt}composition{\isachardot}{\kern0pt}simps\isanewline
\ \ \ \ \ \ \ \ \ \ \ \ reject{\isacharunderscore}{\kern0pt}not{\isacharunderscore}{\kern0pt}elec{\isacharunderscore}{\kern0pt}or{\isacharunderscore}{\kern0pt}def\ maximum{\isacharunderscore}{\kern0pt}parallel{\isacharunderscore}{\kern0pt}composition{\isachardot}{\kern0pt}simps\isanewline
\ \ \ \ \ \ \isacommand{by}\isamarkupfalse%
\ {\isacharparenleft}{\kern0pt}smt\ {\isacharparenleft}{\kern0pt}verit{\isacharcomma}{\kern0pt}\ ccfv{\isacharunderscore}{\kern0pt}threshold{\isacharparenright}{\kern0pt}{\isacharparenright}{\kern0pt}\isanewline
\ \ \isacommand{qed}\isamarkupfalse%
\isanewline
\ \ \isacommand{ultimately}\isamarkupfalse%
\ \isacommand{show}\isamarkupfalse%
\ {\isacharquery}{\kern0pt}thesis\isanewline
\ \ \ \ \isacommand{by}\isamarkupfalse%
\ blast\isanewline
\isacommand{qed}\isamarkupfalse%
%
\endisatagproof
{\isafoldproof}%
%
\isadelimproof
\isanewline
%
\endisadelimproof
\isanewline
\isacommand{lemma}\isamarkupfalse%
\ max{\isacharunderscore}{\kern0pt}agg{\isacharunderscore}{\kern0pt}rej{\isadigit{1}}{\isacharcolon}{\kern0pt}\isanewline
\ \ \isakeyword{assumes}\isanewline
\ \ \ \ f{\isacharunderscore}{\kern0pt}prof{\isacharcolon}{\kern0pt}\ {\isachardoublequoteopen}finite{\isacharunderscore}{\kern0pt}profile\ A\ p{\isachardoublequoteclose}\ \isakeyword{and}\isanewline
\ \ \ \ module{\isacharunderscore}{\kern0pt}m{\isacharcolon}{\kern0pt}\ {\isachardoublequoteopen}electoral{\isacharunderscore}{\kern0pt}module\ m{\isachardoublequoteclose}\ \isakeyword{and}\isanewline
\ \ \ \ module{\isacharunderscore}{\kern0pt}n{\isacharcolon}{\kern0pt}\ {\isachardoublequoteopen}electoral{\isacharunderscore}{\kern0pt}module\ n{\isachardoublequoteclose}\ \isakeyword{and}\isanewline
\ \ \ \ rejected{\isacharcolon}{\kern0pt}\ {\isachardoublequoteopen}x\ {\isasymin}\ reject\ n\ A\ p{\isachardoublequoteclose}\isanewline
\ \ \isakeyword{shows}\isanewline
\ \ \ \ {\isachardoublequoteopen}mod{\isacharunderscore}{\kern0pt}contains{\isacharunderscore}{\kern0pt}result\ m\ {\isacharparenleft}{\kern0pt}m\ {\isasymparallel}\isactrlsub {\isasymup}\ n{\isacharparenright}{\kern0pt}\ A\ p\ x{\isachardoublequoteclose}\isanewline
%
\isadelimproof
\ \ %
\endisadelimproof
%
\isatagproof
\isacommand{using}\isamarkupfalse%
\ Set{\isachardot}{\kern0pt}set{\isacharunderscore}{\kern0pt}insert\ contra{\isacharunderscore}{\kern0pt}subsetD\ disjoint{\isacharunderscore}{\kern0pt}insert\isanewline
\ \ \ \ \ \ \ \ mod{\isacharunderscore}{\kern0pt}contains{\isacharunderscore}{\kern0pt}result{\isacharunderscore}{\kern0pt}comm\ mod{\isacharunderscore}{\kern0pt}contains{\isacharunderscore}{\kern0pt}result{\isacharunderscore}{\kern0pt}def\isanewline
\ \ \ \ \ \ \ \ max{\isacharunderscore}{\kern0pt}agg{\isacharunderscore}{\kern0pt}eq{\isacharunderscore}{\kern0pt}result\ max{\isacharunderscore}{\kern0pt}agg{\isacharunderscore}{\kern0pt}rej{\isacharunderscore}{\kern0pt}iff{\isacharunderscore}{\kern0pt}both{\isacharunderscore}{\kern0pt}reject\isanewline
\ \ \ \ \ \ \ \ module{\isacharunderscore}{\kern0pt}m\ module{\isacharunderscore}{\kern0pt}n\ f{\isacharunderscore}{\kern0pt}prof\ reject{\isacharunderscore}{\kern0pt}in{\isacharunderscore}{\kern0pt}alts\ rejected\isanewline
\ \ \ \ \ \ \ \ result{\isacharunderscore}{\kern0pt}disj\isanewline
\ \ \isacommand{by}\isamarkupfalse%
\ {\isacharparenleft}{\kern0pt}smt\ {\isacharparenleft}{\kern0pt}verit{\isacharcomma}{\kern0pt}\ best{\isacharparenright}{\kern0pt}{\isacharparenright}{\kern0pt}%
\endisatagproof
{\isafoldproof}%
%
\isadelimproof
\isanewline
%
\endisadelimproof
\isanewline
\isacommand{lemma}\isamarkupfalse%
\ max{\isacharunderscore}{\kern0pt}agg{\isacharunderscore}{\kern0pt}rej{\isadigit{2}}{\isacharcolon}{\kern0pt}\isanewline
\ \ \isakeyword{assumes}\isanewline
\ \ \ \ f{\isacharunderscore}{\kern0pt}prof{\isacharcolon}{\kern0pt}\ {\isachardoublequoteopen}finite{\isacharunderscore}{\kern0pt}profile\ A\ p{\isachardoublequoteclose}\ \isakeyword{and}\isanewline
\ \ \ \ module{\isacharunderscore}{\kern0pt}m{\isacharcolon}{\kern0pt}\ {\isachardoublequoteopen}electoral{\isacharunderscore}{\kern0pt}module\ m{\isachardoublequoteclose}\ \isakeyword{and}\isanewline
\ \ \ \ module{\isacharunderscore}{\kern0pt}n{\isacharcolon}{\kern0pt}\ {\isachardoublequoteopen}electoral{\isacharunderscore}{\kern0pt}module\ n{\isachardoublequoteclose}\ \isakeyword{and}\isanewline
\ \ \ \ rejected{\isacharcolon}{\kern0pt}\ {\isachardoublequoteopen}x\ {\isasymin}\ reject\ n\ A\ p{\isachardoublequoteclose}\isanewline
\ \ \isakeyword{shows}\isanewline
\ \ \ \ {\isachardoublequoteopen}mod{\isacharunderscore}{\kern0pt}contains{\isacharunderscore}{\kern0pt}result\ {\isacharparenleft}{\kern0pt}m\ {\isasymparallel}\isactrlsub {\isasymup}\ n{\isacharparenright}{\kern0pt}\ m\ A\ p\ x{\isachardoublequoteclose}\isanewline
%
\isadelimproof
\ \ %
\endisadelimproof
%
\isatagproof
\isacommand{using}\isamarkupfalse%
\ mod{\isacharunderscore}{\kern0pt}contains{\isacharunderscore}{\kern0pt}result{\isacharunderscore}{\kern0pt}comm\ max{\isacharunderscore}{\kern0pt}agg{\isacharunderscore}{\kern0pt}rej{\isadigit{1}}\isanewline
\ \ \ \ \ \ \ \ module{\isacharunderscore}{\kern0pt}m\ module{\isacharunderscore}{\kern0pt}n\ f{\isacharunderscore}{\kern0pt}prof\ rejected\isanewline
\ \ \isacommand{by}\isamarkupfalse%
\ metis%
\endisatagproof
{\isafoldproof}%
%
\isadelimproof
\isanewline
%
\endisadelimproof
\isanewline
\isacommand{lemma}\isamarkupfalse%
\ max{\isacharunderscore}{\kern0pt}agg{\isacharunderscore}{\kern0pt}rej{\isadigit{3}}{\isacharcolon}{\kern0pt}\isanewline
\ \ \isakeyword{assumes}\isanewline
\ \ \ \ f{\isacharunderscore}{\kern0pt}prof{\isacharcolon}{\kern0pt}\ \ {\isachardoublequoteopen}finite{\isacharunderscore}{\kern0pt}profile\ A\ p{\isachardoublequoteclose}\ \isakeyword{and}\isanewline
\ \ \ \ module{\isacharunderscore}{\kern0pt}m{\isacharcolon}{\kern0pt}\ {\isachardoublequoteopen}electoral{\isacharunderscore}{\kern0pt}module\ m{\isachardoublequoteclose}\ \isakeyword{and}\isanewline
\ \ \ \ module{\isacharunderscore}{\kern0pt}n{\isacharcolon}{\kern0pt}\ {\isachardoublequoteopen}electoral{\isacharunderscore}{\kern0pt}module\ n{\isachardoublequoteclose}\ \isakeyword{and}\isanewline
\ \ \ \ rejected{\isacharcolon}{\kern0pt}\ {\isachardoublequoteopen}x\ {\isasymin}\ reject\ m\ A\ p{\isachardoublequoteclose}\isanewline
\ \ \isakeyword{shows}\isanewline
\ \ \ \ {\isachardoublequoteopen}mod{\isacharunderscore}{\kern0pt}contains{\isacharunderscore}{\kern0pt}result\ n\ {\isacharparenleft}{\kern0pt}m\ {\isasymparallel}\isactrlsub {\isasymup}\ n{\isacharparenright}{\kern0pt}\ A\ p\ x{\isachardoublequoteclose}\isanewline
%
\isadelimproof
\ \ %
\endisadelimproof
%
\isatagproof
\isacommand{using}\isamarkupfalse%
\ contra{\isacharunderscore}{\kern0pt}subsetD\ disjoint{\isacharunderscore}{\kern0pt}iff{\isacharunderscore}{\kern0pt}not{\isacharunderscore}{\kern0pt}equal\ result{\isacharunderscore}{\kern0pt}disj\isanewline
\ \ \ \ \ \ \ \ mod{\isacharunderscore}{\kern0pt}contains{\isacharunderscore}{\kern0pt}result{\isacharunderscore}{\kern0pt}comm\ mod{\isacharunderscore}{\kern0pt}contains{\isacharunderscore}{\kern0pt}result{\isacharunderscore}{\kern0pt}def\isanewline
\ \ \ \ \ \ \ \ max{\isacharunderscore}{\kern0pt}agg{\isacharunderscore}{\kern0pt}eq{\isacharunderscore}{\kern0pt}result\ max{\isacharunderscore}{\kern0pt}agg{\isacharunderscore}{\kern0pt}rej{\isacharunderscore}{\kern0pt}iff{\isacharunderscore}{\kern0pt}both{\isacharunderscore}{\kern0pt}reject\isanewline
\ \ \ \ \ \ \ \ module{\isacharunderscore}{\kern0pt}m\ module{\isacharunderscore}{\kern0pt}n\ f{\isacharunderscore}{\kern0pt}prof\ reject{\isacharunderscore}{\kern0pt}in{\isacharunderscore}{\kern0pt}alts\ rejected\isanewline
\ \ \isacommand{by}\isamarkupfalse%
\ {\isacharparenleft}{\kern0pt}smt\ {\isacharparenleft}{\kern0pt}verit{\isacharcomma}{\kern0pt}\ ccfv{\isacharunderscore}{\kern0pt}SIG{\isacharparenright}{\kern0pt}{\isacharparenright}{\kern0pt}%
\endisatagproof
{\isafoldproof}%
%
\isadelimproof
\isanewline
%
\endisadelimproof
\isanewline
\isacommand{lemma}\isamarkupfalse%
\ max{\isacharunderscore}{\kern0pt}agg{\isacharunderscore}{\kern0pt}rej{\isadigit{4}}{\isacharcolon}{\kern0pt}\isanewline
\ \ \isakeyword{assumes}\isanewline
\ \ \ \ f{\isacharunderscore}{\kern0pt}prof{\isacharcolon}{\kern0pt}\ {\isachardoublequoteopen}finite{\isacharunderscore}{\kern0pt}profile\ A\ p{\isachardoublequoteclose}\ \isakeyword{and}\isanewline
\ \ \ \ module{\isacharunderscore}{\kern0pt}m{\isacharcolon}{\kern0pt}\ {\isachardoublequoteopen}electoral{\isacharunderscore}{\kern0pt}module\ m{\isachardoublequoteclose}\ \isakeyword{and}\isanewline
\ \ \ \ module{\isacharunderscore}{\kern0pt}n{\isacharcolon}{\kern0pt}\ {\isachardoublequoteopen}electoral{\isacharunderscore}{\kern0pt}module\ n{\isachardoublequoteclose}\ \isakeyword{and}\isanewline
\ \ \ \ rejected{\isacharcolon}{\kern0pt}\ {\isachardoublequoteopen}x\ {\isasymin}\ reject\ m\ A\ p{\isachardoublequoteclose}\isanewline
\ \ \isakeyword{shows}\isanewline
\ \ \ \ {\isachardoublequoteopen}mod{\isacharunderscore}{\kern0pt}contains{\isacharunderscore}{\kern0pt}result\ {\isacharparenleft}{\kern0pt}m\ {\isasymparallel}\isactrlsub {\isasymup}\ n{\isacharparenright}{\kern0pt}\ n\ A\ p\ x{\isachardoublequoteclose}\isanewline
%
\isadelimproof
\ \ %
\endisadelimproof
%
\isatagproof
\isacommand{using}\isamarkupfalse%
\ mod{\isacharunderscore}{\kern0pt}contains{\isacharunderscore}{\kern0pt}result{\isacharunderscore}{\kern0pt}comm\ max{\isacharunderscore}{\kern0pt}agg{\isacharunderscore}{\kern0pt}rej{\isadigit{3}}\isanewline
\ \ \ \ \ \ \ \ module{\isacharunderscore}{\kern0pt}m\ module{\isacharunderscore}{\kern0pt}n\ f{\isacharunderscore}{\kern0pt}prof\ rejected\isanewline
\ \ \isacommand{by}\isamarkupfalse%
\ metis%
\endisatagproof
{\isafoldproof}%
%
\isadelimproof
\isanewline
%
\endisadelimproof
\isanewline
\isacommand{lemma}\isamarkupfalse%
\ max{\isacharunderscore}{\kern0pt}agg{\isacharunderscore}{\kern0pt}rej{\isacharunderscore}{\kern0pt}intersect{\isacharcolon}{\kern0pt}\isanewline
\ \ \isakeyword{assumes}\isanewline
\ \ \ \ module{\isacharunderscore}{\kern0pt}m{\isacharcolon}{\kern0pt}\ {\isachardoublequoteopen}electoral{\isacharunderscore}{\kern0pt}module\ m{\isachardoublequoteclose}\ \isakeyword{and}\isanewline
\ \ \ \ module{\isacharunderscore}{\kern0pt}n{\isacharcolon}{\kern0pt}\ {\isachardoublequoteopen}electoral{\isacharunderscore}{\kern0pt}module\ n{\isachardoublequoteclose}\ \isakeyword{and}\isanewline
\ \ \ \ f{\isacharunderscore}{\kern0pt}prof{\isacharcolon}{\kern0pt}\ {\isachardoublequoteopen}finite{\isacharunderscore}{\kern0pt}profile\ A\ p{\isachardoublequoteclose}\isanewline
\ \ \isakeyword{shows}\isanewline
\ \ \ \ {\isachardoublequoteopen}reject\ {\isacharparenleft}{\kern0pt}m\ {\isasymparallel}\isactrlsub {\isasymup}\ n{\isacharparenright}{\kern0pt}\ A\ p\ {\isacharequal}{\kern0pt}\isanewline
\ \ \ \ \ \ {\isacharparenleft}{\kern0pt}reject\ m\ A\ p{\isacharparenright}{\kern0pt}\ {\isasyminter}\ {\isacharparenleft}{\kern0pt}reject\ n\ A\ p{\isacharparenright}{\kern0pt}{\isachardoublequoteclose}\isanewline
%
\isadelimproof
%
\endisadelimproof
%
\isatagproof
\isacommand{proof}\isamarkupfalse%
\ {\isacharminus}{\kern0pt}\isanewline
\ \ \isacommand{have}\isamarkupfalse%
\isanewline
\ \ \ \ {\isachardoublequoteopen}A\ {\isacharequal}{\kern0pt}\ {\isacharparenleft}{\kern0pt}elect\ m\ A\ p{\isacharparenright}{\kern0pt}\ {\isasymunion}\ {\isacharparenleft}{\kern0pt}reject\ m\ A\ p{\isacharparenright}{\kern0pt}\ {\isasymunion}\ {\isacharparenleft}{\kern0pt}defer\ m\ A\ p{\isacharparenright}{\kern0pt}\ {\isasymand}\isanewline
\ \ \ \ \ \ A\ {\isacharequal}{\kern0pt}\ {\isacharparenleft}{\kern0pt}elect\ n\ A\ p{\isacharparenright}{\kern0pt}\ {\isasymunion}\ {\isacharparenleft}{\kern0pt}reject\ n\ A\ p{\isacharparenright}{\kern0pt}\ {\isasymunion}\ {\isacharparenleft}{\kern0pt}defer\ n\ A\ p{\isacharparenright}{\kern0pt}{\isachardoublequoteclose}\isanewline
\ \ \ \ \isacommand{by}\isamarkupfalse%
\ {\isacharparenleft}{\kern0pt}simp\ add{\isacharcolon}{\kern0pt}\ module{\isacharunderscore}{\kern0pt}m\ module{\isacharunderscore}{\kern0pt}n\ f{\isacharunderscore}{\kern0pt}prof\ result{\isacharunderscore}{\kern0pt}presv{\isacharunderscore}{\kern0pt}alts{\isacharparenright}{\kern0pt}\isanewline
\ \ \isacommand{hence}\isamarkupfalse%
\isanewline
\ \ \ \ {\isachardoublequoteopen}A\ {\isacharminus}{\kern0pt}\ {\isacharparenleft}{\kern0pt}{\isacharparenleft}{\kern0pt}elect\ m\ A\ p{\isacharparenright}{\kern0pt}\ {\isasymunion}\ {\isacharparenleft}{\kern0pt}defer\ m\ A\ p{\isacharparenright}{\kern0pt}{\isacharparenright}{\kern0pt}\ {\isacharequal}{\kern0pt}\ {\isacharparenleft}{\kern0pt}reject\ m\ A\ p{\isacharparenright}{\kern0pt}\ {\isasymand}\isanewline
\ \ \ \ \ \ A\ {\isacharminus}{\kern0pt}\ {\isacharparenleft}{\kern0pt}{\isacharparenleft}{\kern0pt}elect\ n\ A\ p{\isacharparenright}{\kern0pt}\ {\isasymunion}\ {\isacharparenleft}{\kern0pt}defer\ n\ A\ p{\isacharparenright}{\kern0pt}{\isacharparenright}{\kern0pt}\ {\isacharequal}{\kern0pt}\ {\isacharparenleft}{\kern0pt}reject\ n\ A\ p{\isacharparenright}{\kern0pt}{\isachardoublequoteclose}\isanewline
\ \ \ \ \isacommand{using}\isamarkupfalse%
\ module{\isacharunderscore}{\kern0pt}m\ module{\isacharunderscore}{\kern0pt}n\ f{\isacharunderscore}{\kern0pt}prof\ reject{\isacharunderscore}{\kern0pt}not{\isacharunderscore}{\kern0pt}elec{\isacharunderscore}{\kern0pt}or{\isacharunderscore}{\kern0pt}def\isanewline
\ \ \ \ \isacommand{by}\isamarkupfalse%
\ auto\isanewline
\ \ \isacommand{hence}\isamarkupfalse%
\isanewline
\ \ \ \ {\isachardoublequoteopen}A\ {\isacharminus}{\kern0pt}\ {\isacharparenleft}{\kern0pt}{\isacharparenleft}{\kern0pt}elect\ m\ A\ p{\isacharparenright}{\kern0pt}\ {\isasymunion}\ {\isacharparenleft}{\kern0pt}elect\ n\ A\ p{\isacharparenright}{\kern0pt}\ {\isasymunion}\ {\isacharparenleft}{\kern0pt}defer\ m\ A\ p{\isacharparenright}{\kern0pt}\ {\isasymunion}\ {\isacharparenleft}{\kern0pt}defer\ n\ A\ p{\isacharparenright}{\kern0pt}{\isacharparenright}{\kern0pt}\ {\isacharequal}{\kern0pt}\isanewline
\ \ \ \ \ \ {\isacharparenleft}{\kern0pt}reject\ m\ A\ p{\isacharparenright}{\kern0pt}\ {\isasyminter}\ {\isacharparenleft}{\kern0pt}reject\ n\ A\ p{\isacharparenright}{\kern0pt}{\isachardoublequoteclose}\isanewline
\ \ \ \ \isacommand{by}\isamarkupfalse%
\ blast\isanewline
\ \ \isacommand{hence}\isamarkupfalse%
\isanewline
\ \ \ \ {\isachardoublequoteopen}let\ {\isacharparenleft}{\kern0pt}e{\isadigit{1}}{\isacharcomma}{\kern0pt}\ r{\isadigit{1}}{\isacharcomma}{\kern0pt}\ d{\isadigit{1}}{\isacharparenright}{\kern0pt}\ {\isacharequal}{\kern0pt}\ m\ A\ p{\isacharsemicolon}{\kern0pt}\isanewline
\ \ \ \ \ \ \ \ {\isacharparenleft}{\kern0pt}e{\isadigit{2}}{\isacharcomma}{\kern0pt}\ r{\isadigit{2}}{\isacharcomma}{\kern0pt}\ d{\isadigit{2}}{\isacharparenright}{\kern0pt}\ {\isacharequal}{\kern0pt}\ n\ A\ p\ in\isanewline
\ \ \ \ \ \ A\ {\isacharminus}{\kern0pt}\ {\isacharparenleft}{\kern0pt}e{\isadigit{1}}\ {\isasymunion}\ e{\isadigit{2}}\ {\isasymunion}\ d{\isadigit{1}}\ {\isasymunion}\ d{\isadigit{2}}{\isacharparenright}{\kern0pt}\ {\isacharequal}{\kern0pt}\ r{\isadigit{1}}\ {\isasyminter}\ r{\isadigit{2}}{\isachardoublequoteclose}\isanewline
\ \ \ \ \isacommand{by}\isamarkupfalse%
\ fastforce\isanewline
\ \ \isacommand{thus}\isamarkupfalse%
\ {\isacharquery}{\kern0pt}thesis\isanewline
\ \ \ \ \isacommand{by}\isamarkupfalse%
\ auto\isanewline
\isacommand{qed}\isamarkupfalse%
%
\endisatagproof
{\isafoldproof}%
%
\isadelimproof
\isanewline
%
\endisadelimproof
\isanewline
\isacommand{lemma}\isamarkupfalse%
\ dcompat{\isacharunderscore}{\kern0pt}dec{\isacharunderscore}{\kern0pt}by{\isacharunderscore}{\kern0pt}one{\isacharunderscore}{\kern0pt}mod{\isacharcolon}{\kern0pt}\isanewline
\ \ \isakeyword{assumes}\isanewline
\ \ \ \ compatible{\isacharcolon}{\kern0pt}\ {\isachardoublequoteopen}disjoint{\isacharunderscore}{\kern0pt}compatibility\ m\ n{\isachardoublequoteclose}\ \isakeyword{and}\isanewline
\ \ \ \ in{\isacharunderscore}{\kern0pt}A{\isacharcolon}{\kern0pt}\ {\isachardoublequoteopen}x\ {\isasymin}\ A{\isachardoublequoteclose}\isanewline
\ \ \isakeyword{shows}\isanewline
\ \ \ \ {\isachardoublequoteopen}{\isacharparenleft}{\kern0pt}{\isasymforall}p{\isachardot}{\kern0pt}\ finite{\isacharunderscore}{\kern0pt}profile\ A\ p\ {\isasymlongrightarrow}\isanewline
\ \ \ \ \ \ \ \ \ \ mod{\isacharunderscore}{\kern0pt}contains{\isacharunderscore}{\kern0pt}result\ m\ {\isacharparenleft}{\kern0pt}m\ {\isasymparallel}\isactrlsub {\isasymup}\ n{\isacharparenright}{\kern0pt}\ A\ p\ x{\isacharparenright}{\kern0pt}\ {\isasymor}\isanewline
\ \ \ \ \ \ \ \ {\isacharparenleft}{\kern0pt}{\isasymforall}p{\isachardot}{\kern0pt}\ finite{\isacharunderscore}{\kern0pt}profile\ A\ p\ {\isasymlongrightarrow}\isanewline
\ \ \ \ \ \ \ \ \ \ mod{\isacharunderscore}{\kern0pt}contains{\isacharunderscore}{\kern0pt}result\ n\ {\isacharparenleft}{\kern0pt}m\ {\isasymparallel}\isactrlsub {\isasymup}\ n{\isacharparenright}{\kern0pt}\ A\ p\ x{\isacharparenright}{\kern0pt}{\isachardoublequoteclose}\isanewline
%
\isadelimproof
\ \ %
\endisadelimproof
%
\isatagproof
\isacommand{using}\isamarkupfalse%
\ DiffI\ compatible\ disjoint{\isacharunderscore}{\kern0pt}compatibility{\isacharunderscore}{\kern0pt}def\isanewline
\ \ \ \ \ \ \ \ in{\isacharunderscore}{\kern0pt}A\ max{\isacharunderscore}{\kern0pt}agg{\isacharunderscore}{\kern0pt}rej{\isadigit{1}}\ max{\isacharunderscore}{\kern0pt}agg{\isacharunderscore}{\kern0pt}rej{\isadigit{3}}\isanewline
\ \ \isacommand{by}\isamarkupfalse%
\ metis%
\endisatagproof
{\isafoldproof}%
%
\isadelimproof
\isanewline
%
\endisadelimproof
\isanewline
\isacommand{lemma}\isamarkupfalse%
\ par{\isacharunderscore}{\kern0pt}comp{\isacharunderscore}{\kern0pt}rej{\isacharunderscore}{\kern0pt}card{\isacharcolon}{\kern0pt}\isanewline
\ \ \isakeyword{assumes}\isanewline
\ \ \ \ compatible{\isacharcolon}{\kern0pt}\ {\isachardoublequoteopen}disjoint{\isacharunderscore}{\kern0pt}compatibility\ x\ y{\isachardoublequoteclose}\ \isakeyword{and}\isanewline
\ \ \ \ f{\isacharunderscore}{\kern0pt}prof{\isacharcolon}{\kern0pt}\ {\isachardoublequoteopen}finite{\isacharunderscore}{\kern0pt}profile\ S\ p{\isachardoublequoteclose}\ \isakeyword{and}\isanewline
\ \ \ \ reject{\isacharunderscore}{\kern0pt}sum{\isacharcolon}{\kern0pt}\ {\isachardoublequoteopen}card\ {\isacharparenleft}{\kern0pt}reject\ x\ S\ p{\isacharparenright}{\kern0pt}\ {\isacharplus}{\kern0pt}\ card\ {\isacharparenleft}{\kern0pt}reject\ y\ S\ p{\isacharparenright}{\kern0pt}\ {\isacharequal}{\kern0pt}\ card\ S\ {\isacharplus}{\kern0pt}\ n{\isachardoublequoteclose}\isanewline
\ \ \isakeyword{shows}\ {\isachardoublequoteopen}card\ {\isacharparenleft}{\kern0pt}reject\ {\isacharparenleft}{\kern0pt}x\ {\isasymparallel}\isactrlsub {\isasymup}\ y{\isacharparenright}{\kern0pt}\ S\ p{\isacharparenright}{\kern0pt}\ {\isacharequal}{\kern0pt}\ n{\isachardoublequoteclose}\isanewline
%
\isadelimproof
%
\endisadelimproof
%
\isatagproof
\isacommand{proof}\isamarkupfalse%
\ {\isacharminus}{\kern0pt}\isanewline
\ \ \isacommand{from}\isamarkupfalse%
\ compatible\ \isacommand{obtain}\isamarkupfalse%
\ A\ \isakeyword{where}\ A{\isacharcolon}{\kern0pt}\isanewline
\ \ \ \ {\isachardoublequoteopen}A\ {\isasymsubseteq}\ S\ {\isasymand}\isanewline
\ \ \ \ \ \ {\isacharparenleft}{\kern0pt}{\isasymforall}a\ {\isasymin}\ A{\isachardot}{\kern0pt}\ indep{\isacharunderscore}{\kern0pt}of{\isacharunderscore}{\kern0pt}alt\ x\ S\ a\ {\isasymand}\isanewline
\ \ \ \ \ \ \ \ \ \ {\isacharparenleft}{\kern0pt}{\isasymforall}p{\isachardot}{\kern0pt}\ finite{\isacharunderscore}{\kern0pt}profile\ S\ p\ {\isasymlongrightarrow}\ a\ {\isasymin}\ reject\ x\ S\ p{\isacharparenright}{\kern0pt}{\isacharparenright}{\kern0pt}\ {\isasymand}\isanewline
\ \ \ \ \ \ {\isacharparenleft}{\kern0pt}{\isasymforall}a\ {\isasymin}\ S{\isacharminus}{\kern0pt}A{\isachardot}{\kern0pt}\ indep{\isacharunderscore}{\kern0pt}of{\isacharunderscore}{\kern0pt}alt\ y\ S\ a\ {\isasymand}\isanewline
\ \ \ \ \ \ \ \ \ \ {\isacharparenleft}{\kern0pt}{\isasymforall}p{\isachardot}{\kern0pt}\ finite{\isacharunderscore}{\kern0pt}profile\ S\ p\ {\isasymlongrightarrow}\ a\ {\isasymin}\ reject\ y\ S\ p{\isacharparenright}{\kern0pt}{\isacharparenright}{\kern0pt}{\isachardoublequoteclose}\isanewline
\ \ \ \ \isacommand{using}\isamarkupfalse%
\ disjoint{\isacharunderscore}{\kern0pt}compatibility{\isacharunderscore}{\kern0pt}def\ f{\isacharunderscore}{\kern0pt}prof\isanewline
\ \ \ \ \isacommand{by}\isamarkupfalse%
\ metis\isanewline
\ \ \isacommand{from}\isamarkupfalse%
\ f{\isacharunderscore}{\kern0pt}prof\ compatible\isanewline
\ \ \isacommand{have}\isamarkupfalse%
\ reject{\isacharunderscore}{\kern0pt}representation{\isacharcolon}{\kern0pt}\isanewline
\ \ \ \ {\isachardoublequoteopen}reject\ {\isacharparenleft}{\kern0pt}x\ {\isasymparallel}\isactrlsub {\isasymup}\ y{\isacharparenright}{\kern0pt}\ S\ p\ {\isacharequal}{\kern0pt}\ {\isacharparenleft}{\kern0pt}reject\ x\ S\ p{\isacharparenright}{\kern0pt}\ {\isasyminter}\ {\isacharparenleft}{\kern0pt}reject\ y\ S\ p{\isacharparenright}{\kern0pt}{\isachardoublequoteclose}\isanewline
\ \ \ \ \isacommand{using}\isamarkupfalse%
\ max{\isacharunderscore}{\kern0pt}agg{\isacharunderscore}{\kern0pt}rej{\isacharunderscore}{\kern0pt}intersect\ disjoint{\isacharunderscore}{\kern0pt}compatibility{\isacharunderscore}{\kern0pt}def\isanewline
\ \ \ \ \isacommand{by}\isamarkupfalse%
\ blast\isanewline
\ \ \isacommand{have}\isamarkupfalse%
\ {\isachardoublequoteopen}electoral{\isacharunderscore}{\kern0pt}module\ x\ {\isasymand}\ electoral{\isacharunderscore}{\kern0pt}module\ y{\isachardoublequoteclose}\isanewline
\ \ \ \ \isacommand{using}\isamarkupfalse%
\ compatible\ disjoint{\isacharunderscore}{\kern0pt}compatibility{\isacharunderscore}{\kern0pt}def\isanewline
\ \ \ \ \isacommand{by}\isamarkupfalse%
\ auto\isanewline
\ \ \isacommand{hence}\isamarkupfalse%
\ subsets{\isacharcolon}{\kern0pt}\ {\isachardoublequoteopen}{\isacharparenleft}{\kern0pt}reject\ x\ S\ p{\isacharparenright}{\kern0pt}\ {\isasymsubseteq}\ S\ {\isasymand}\ {\isacharparenleft}{\kern0pt}reject\ y\ S\ p{\isacharparenright}{\kern0pt}\ {\isasymsubseteq}\ S{\isachardoublequoteclose}\isanewline
\ \ \ \ \isacommand{by}\isamarkupfalse%
\ {\isacharparenleft}{\kern0pt}simp\ add{\isacharcolon}{\kern0pt}\ f{\isacharunderscore}{\kern0pt}prof\ reject{\isacharunderscore}{\kern0pt}in{\isacharunderscore}{\kern0pt}alts{\isacharparenright}{\kern0pt}\isanewline
\ \ \isacommand{hence}\isamarkupfalse%
\ {\isachardoublequoteopen}finite\ {\isacharparenleft}{\kern0pt}reject\ x\ S\ p{\isacharparenright}{\kern0pt}\ {\isasymand}\ finite\ {\isacharparenleft}{\kern0pt}reject\ y\ S\ p{\isacharparenright}{\kern0pt}{\isachardoublequoteclose}\isanewline
\ \ \ \ \isacommand{using}\isamarkupfalse%
\ rev{\isacharunderscore}{\kern0pt}finite{\isacharunderscore}{\kern0pt}subset\ f{\isacharunderscore}{\kern0pt}prof\ reject{\isacharunderscore}{\kern0pt}in{\isacharunderscore}{\kern0pt}alts\isanewline
\ \ \ \ \isacommand{by}\isamarkupfalse%
\ auto\isanewline
\ \ \isacommand{hence}\isamarkupfalse%
\ {\isadigit{0}}{\isacharcolon}{\kern0pt}\isanewline
\ \ \ \ {\isachardoublequoteopen}card\ {\isacharparenleft}{\kern0pt}reject\ {\isacharparenleft}{\kern0pt}x\ {\isasymparallel}\isactrlsub {\isasymup}\ y{\isacharparenright}{\kern0pt}\ S\ p{\isacharparenright}{\kern0pt}\ {\isacharequal}{\kern0pt}\isanewline
\ \ \ \ \ \ \ \ card\ S\ {\isacharplus}{\kern0pt}\ n\ {\isacharminus}{\kern0pt}\isanewline
\ \ \ \ \ \ \ \ \ \ card\ {\isacharparenleft}{\kern0pt}{\isacharparenleft}{\kern0pt}reject\ x\ S\ p{\isacharparenright}{\kern0pt}\ {\isasymunion}\ {\isacharparenleft}{\kern0pt}reject\ y\ S\ p{\isacharparenright}{\kern0pt}{\isacharparenright}{\kern0pt}{\isachardoublequoteclose}\isanewline
\ \ \ \ \isacommand{using}\isamarkupfalse%
\ card{\isacharunderscore}{\kern0pt}Un{\isacharunderscore}{\kern0pt}Int\ reject{\isacharunderscore}{\kern0pt}representation\ reject{\isacharunderscore}{\kern0pt}sum\isanewline
\ \ \ \ \isacommand{by}\isamarkupfalse%
\ fastforce\isanewline
\ \ \isacommand{have}\isamarkupfalse%
\ {\isachardoublequoteopen}{\isasymforall}a\ {\isasymin}\ S{\isachardot}{\kern0pt}\ a\ {\isasymin}\ {\isacharparenleft}{\kern0pt}reject\ x\ S\ p{\isacharparenright}{\kern0pt}\ {\isasymor}\ a\ {\isasymin}\ {\isacharparenleft}{\kern0pt}reject\ y\ S\ p{\isacharparenright}{\kern0pt}{\isachardoublequoteclose}\isanewline
\ \ \ \ \isacommand{using}\isamarkupfalse%
\ A\ f{\isacharunderscore}{\kern0pt}prof\isanewline
\ \ \ \ \isacommand{by}\isamarkupfalse%
\ blast\isanewline
\ \ \isacommand{hence}\isamarkupfalse%
\ {\isadigit{1}}{\isacharcolon}{\kern0pt}\ {\isachardoublequoteopen}card\ {\isacharparenleft}{\kern0pt}{\isacharparenleft}{\kern0pt}reject\ x\ S\ p{\isacharparenright}{\kern0pt}\ {\isasymunion}\ {\isacharparenleft}{\kern0pt}reject\ y\ S\ p{\isacharparenright}{\kern0pt}{\isacharparenright}{\kern0pt}\ {\isacharequal}{\kern0pt}\ card\ S{\isachardoublequoteclose}\isanewline
\ \ \ \ \isacommand{using}\isamarkupfalse%
\ subsets\ subset{\isacharunderscore}{\kern0pt}eq\ sup{\isachardot}{\kern0pt}absorb{\isacharunderscore}{\kern0pt}iff{\isadigit{1}}\isanewline
\ \ \ \ \ \ \ \ \ \ sup{\isachardot}{\kern0pt}cobounded{\isadigit{1}}\ sup{\isacharunderscore}{\kern0pt}left{\isacharunderscore}{\kern0pt}commute\isanewline
\ \ \ \ \isacommand{by}\isamarkupfalse%
\ {\isacharparenleft}{\kern0pt}smt\ {\isacharparenleft}{\kern0pt}verit{\isacharcomma}{\kern0pt}\ best{\isacharparenright}{\kern0pt}{\isacharparenright}{\kern0pt}\isanewline
\ \ \isacommand{from}\isamarkupfalse%
\ {\isadigit{0}}\ {\isadigit{1}}\isanewline
\ \ \isacommand{show}\isamarkupfalse%
\ {\isachardoublequoteopen}card\ {\isacharparenleft}{\kern0pt}reject\ {\isacharparenleft}{\kern0pt}x\ {\isasymparallel}\isactrlsub {\isasymup}\ y{\isacharparenright}{\kern0pt}\ S\ p{\isacharparenright}{\kern0pt}\ {\isacharequal}{\kern0pt}\ n{\isachardoublequoteclose}\isanewline
\ \ \ \ \isacommand{by}\isamarkupfalse%
\ simp\isanewline
\isacommand{qed}\isamarkupfalse%
%
\endisatagproof
{\isafoldproof}%
%
\isadelimproof
\isanewline
%
\endisadelimproof
%
\isadelimtheory
\isanewline
%
\endisadelimtheory
%
\isatagtheory
\isacommand{end}\isamarkupfalse%
%
\endisatagtheory
{\isafoldtheory}%
%
\isadelimtheory
%
\endisadelimtheory
%
\end{isabellebody}%
\endinput
%:%file=~/Documents/Studies/VotingRuleGenerator/virage/src/test/resources/verifiedVotingRuleConstruction/theories/Compositional_Framework/Components/Composites/Composite_Structures.thy%:%
%:%10=1%:%
%:%11=1%:%
%:%12=2%:%
%:%13=3%:%
%:%14=4%:%
%:%15=5%:%
%:%16=6%:%
%:%17=7%:%
%:%18=8%:%
%:%19=9%:%
%:%20=10%:%
%:%21=11%:%
%:%22=12%:%
%:%23=13%:%
%:%37=15%:%
%:%49=18%:%
%:%50=19%:%
%:%51=20%:%
%:%52=21%:%
%:%53=22%:%
%:%62=24%:%
%:%72=26%:%
%:%73=26%:%
%:%74=27%:%
%:%81=29%:%
%:%91=31%:%
%:%92=31%:%
%:%93=32%:%
%:%94=33%:%
%:%97=34%:%
%:%101=34%:%
%:%102=34%:%
%:%116=36%:%
%:%128=39%:%
%:%129=40%:%
%:%130=41%:%
%:%131=42%:%
%:%140=44%:%
%:%150=46%:%
%:%151=46%:%
%:%152=47%:%
%:%154=49%:%
%:%155=50%:%
%:%156=51%:%
%:%157=51%:%
%:%158=52%:%
%:%159=53%:%
%:%160=54%:%
%:%161=55%:%
%:%162=56%:%
%:%163=56%:%
%:%164=57%:%
%:%171=59%:%
%:%183=62%:%
%:%184=63%:%
%:%185=64%:%
%:%186=65%:%
%:%187=66%:%
%:%188=67%:%
%:%189=68%:%
%:%190=69%:%
%:%199=71%:%
%:%209=73%:%
%:%210=73%:%
%:%211=74%:%
%:%212=75%:%
%:%213=76%:%
%:%214=77%:%
%:%215=78%:%
%:%216=78%:%
%:%217=79%:%
%:%218=80%:%
%:%225=82%:%
%:%235=84%:%
%:%236=84%:%
%:%237=85%:%
%:%238=86%:%
%:%239=87%:%
%:%240=88%:%
%:%243=89%:%
%:%247=89%:%
%:%248=89%:%
%:%249=90%:%
%:%250=90%:%
%:%264=92%:%
%:%274=94%:%
%:%275=94%:%
%:%276=95%:%
%:%277=96%:%
%:%278=97%:%
%:%279=98%:%
%:%280=99%:%
%:%281=100%:%
%:%282=101%:%
%:%283=102%:%
%:%290=103%:%
%:%291=103%:%
%:%292=104%:%
%:%293=104%:%
%:%294=105%:%
%:%295=105%:%
%:%296=106%:%
%:%298=108%:%
%:%299=109%:%
%:%300=109%:%
%:%301=110%:%
%:%302=110%:%
%:%303=111%:%
%:%304=111%:%
%:%305=112%:%
%:%306=112%:%
%:%307=113%:%
%:%308=113%:%
%:%309=114%:%
%:%310=115%:%
%:%311=116%:%
%:%312=116%:%
%:%313=117%:%
%:%314=117%:%
%:%315=118%:%
%:%316=118%:%
%:%317=119%:%
%:%318=119%:%
%:%319=120%:%
%:%320=120%:%
%:%321=121%:%
%:%322=121%:%
%:%323=122%:%
%:%324=122%:%
%:%325=123%:%
%:%326=123%:%
%:%327=124%:%
%:%328=125%:%
%:%329=126%:%
%:%330=127%:%
%:%331=128%:%
%:%332=129%:%
%:%333=129%:%
%:%334=130%:%
%:%335=130%:%
%:%336=131%:%
%:%337=131%:%
%:%338=132%:%
%:%339=132%:%
%:%340=132%:%
%:%341=133%:%
%:%342=134%:%
%:%343=134%:%
%:%344=135%:%
%:%345=136%:%
%:%346=136%:%
%:%347=137%:%
%:%348=137%:%
%:%349=138%:%
%:%351=140%:%
%:%352=141%:%
%:%353=141%:%
%:%354=142%:%
%:%355=143%:%
%:%356=143%:%
%:%357=144%:%
%:%358=144%:%
%:%359=145%:%
%:%361=147%:%
%:%362=148%:%
%:%363=148%:%
%:%364=149%:%
%:%365=149%:%
%:%366=150%:%
%:%367=150%:%
%:%368=151%:%
%:%369=152%:%
%:%370=153%:%
%:%371=154%:%
%:%372=155%:%
%:%373=156%:%
%:%374=157%:%
%:%375=157%:%
%:%376=158%:%
%:%377=158%:%
%:%378=159%:%
%:%384=159%:%
%:%387=160%:%
%:%388=161%:%
%:%389=161%:%
%:%390=162%:%
%:%391=163%:%
%:%392=164%:%
%:%393=165%:%
%:%394=166%:%
%:%395=167%:%
%:%396=168%:%
%:%403=169%:%
%:%404=169%:%
%:%405=170%:%
%:%406=170%:%
%:%407=171%:%
%:%408=172%:%
%:%409=173%:%
%:%410=173%:%
%:%411=174%:%
%:%412=174%:%
%:%413=175%:%
%:%414=175%:%
%:%415=176%:%
%:%417=178%:%
%:%418=179%:%
%:%419=179%:%
%:%420=180%:%
%:%421=181%:%
%:%422=181%:%
%:%423=182%:%
%:%424=182%:%
%:%425=183%:%
%:%427=185%:%
%:%428=186%:%
%:%429=186%:%
%:%430=187%:%
%:%431=187%:%
%:%432=188%:%
%:%433=188%:%
%:%434=189%:%
%:%435=190%:%
%:%436=191%:%
%:%437=191%:%
%:%438=192%:%
%:%439=192%:%
%:%440=193%:%
%:%441=193%:%
%:%442=193%:%
%:%443=194%:%
%:%444=195%:%
%:%445=196%:%
%:%446=196%:%
%:%447=197%:%
%:%448=197%:%
%:%449=198%:%
%:%450=198%:%
%:%451=199%:%
%:%452=200%:%
%:%453=201%:%
%:%454=201%:%
%:%455=202%:%
%:%456=202%:%
%:%457=203%:%
%:%458=203%:%
%:%459=204%:%
%:%460=204%:%
%:%461=205%:%
%:%462=206%:%
%:%463=207%:%
%:%464=208%:%
%:%465=209%:%
%:%466=209%:%
%:%467=210%:%
%:%468=210%:%
%:%469=211%:%
%:%470=211%:%
%:%471=211%:%
%:%472=212%:%
%:%473=212%:%
%:%474=213%:%
%:%480=213%:%
%:%483=214%:%
%:%484=215%:%
%:%485=215%:%
%:%486=216%:%
%:%487=217%:%
%:%488=218%:%
%:%489=219%:%
%:%490=220%:%
%:%491=221%:%
%:%492=222%:%
%:%495=223%:%
%:%499=223%:%
%:%500=223%:%
%:%501=224%:%
%:%502=225%:%
%:%503=226%:%
%:%504=227%:%
%:%505=228%:%
%:%506=228%:%
%:%511=228%:%
%:%514=229%:%
%:%515=230%:%
%:%516=230%:%
%:%517=231%:%
%:%518=232%:%
%:%519=233%:%
%:%520=234%:%
%:%521=235%:%
%:%522=236%:%
%:%523=237%:%
%:%526=238%:%
%:%530=238%:%
%:%531=238%:%
%:%532=239%:%
%:%533=240%:%
%:%534=240%:%
%:%539=240%:%
%:%542=241%:%
%:%543=242%:%
%:%544=242%:%
%:%545=243%:%
%:%546=244%:%
%:%547=245%:%
%:%548=246%:%
%:%549=247%:%
%:%550=248%:%
%:%551=249%:%
%:%554=250%:%
%:%558=250%:%
%:%559=250%:%
%:%560=251%:%
%:%561=252%:%
%:%562=253%:%
%:%563=254%:%
%:%564=254%:%
%:%569=254%:%
%:%572=255%:%
%:%573=256%:%
%:%574=256%:%
%:%575=257%:%
%:%576=258%:%
%:%577=259%:%
%:%578=260%:%
%:%579=261%:%
%:%580=262%:%
%:%581=263%:%
%:%584=264%:%
%:%588=264%:%
%:%589=264%:%
%:%590=265%:%
%:%591=266%:%
%:%592=266%:%
%:%597=266%:%
%:%600=267%:%
%:%601=268%:%
%:%602=268%:%
%:%603=269%:%
%:%604=270%:%
%:%605=271%:%
%:%606=272%:%
%:%607=273%:%
%:%608=274%:%
%:%609=275%:%
%:%616=276%:%
%:%617=276%:%
%:%618=277%:%
%:%619=277%:%
%:%620=278%:%
%:%621=279%:%
%:%622=280%:%
%:%623=280%:%
%:%624=281%:%
%:%625=281%:%
%:%626=282%:%
%:%627=283%:%
%:%628=284%:%
%:%629=284%:%
%:%630=285%:%
%:%631=285%:%
%:%632=286%:%
%:%633=286%:%
%:%634=287%:%
%:%635=288%:%
%:%636=289%:%
%:%637=289%:%
%:%638=290%:%
%:%639=290%:%
%:%640=291%:%
%:%642=293%:%
%:%643=294%:%
%:%644=294%:%
%:%645=295%:%
%:%646=295%:%
%:%647=296%:%
%:%648=296%:%
%:%649=297%:%
%:%655=297%:%
%:%658=298%:%
%:%659=299%:%
%:%660=299%:%
%:%661=300%:%
%:%662=301%:%
%:%663=302%:%
%:%664=303%:%
%:%665=304%:%
%:%668=307%:%
%:%671=308%:%
%:%675=308%:%
%:%676=308%:%
%:%677=309%:%
%:%678=310%:%
%:%679=310%:%
%:%684=310%:%
%:%687=311%:%
%:%688=312%:%
%:%689=312%:%
%:%690=313%:%
%:%691=314%:%
%:%692=315%:%
%:%693=316%:%
%:%694=317%:%
%:%701=318%:%
%:%702=318%:%
%:%703=319%:%
%:%704=319%:%
%:%705=319%:%
%:%706=320%:%
%:%710=324%:%
%:%711=325%:%
%:%712=325%:%
%:%713=326%:%
%:%714=326%:%
%:%715=327%:%
%:%716=327%:%
%:%717=328%:%
%:%718=328%:%
%:%719=329%:%
%:%720=330%:%
%:%721=330%:%
%:%722=331%:%
%:%723=331%:%
%:%724=332%:%
%:%725=332%:%
%:%726=333%:%
%:%727=333%:%
%:%728=334%:%
%:%729=334%:%
%:%730=335%:%
%:%731=335%:%
%:%732=336%:%
%:%733=336%:%
%:%734=337%:%
%:%735=337%:%
%:%736=338%:%
%:%737=338%:%
%:%738=339%:%
%:%739=339%:%
%:%740=340%:%
%:%741=340%:%
%:%742=341%:%
%:%744=343%:%
%:%745=344%:%
%:%746=344%:%
%:%747=345%:%
%:%748=345%:%
%:%749=346%:%
%:%750=346%:%
%:%751=347%:%
%:%752=347%:%
%:%753=348%:%
%:%754=348%:%
%:%755=349%:%
%:%756=349%:%
%:%757=350%:%
%:%758=350%:%
%:%759=351%:%
%:%760=352%:%
%:%761=352%:%
%:%762=353%:%
%:%763=353%:%
%:%764=354%:%
%:%765=354%:%
%:%766=355%:%
%:%767=355%:%
%:%768=356%:%
%:%774=356%:%
%:%779=357%:%
%:%784=358%:%
%
\begin{isabellebody}%
\setisabellecontext{Revision{\isacharunderscore}{\kern0pt}Composition}%
%
\isadelimdocument
\isanewline
%
\endisadelimdocument
%
\isatagdocument
\isanewline
\isanewline
%
\isamarkupsection{Revision Composition%
}
\isamarkuptrue%
%
\endisatagdocument
{\isafolddocument}%
%
\isadelimdocument
%
\endisadelimdocument
%
\isadelimtheory
%
\endisadelimtheory
%
\isatagtheory
\isacommand{theory}\isamarkupfalse%
\ Revision{\isacharunderscore}{\kern0pt}Composition\isanewline
\ \ \isakeyword{imports}\ {\isachardoublequoteopen}Basic{\isacharunderscore}{\kern0pt}Modules{\isacharslash}{\kern0pt}Component{\isacharunderscore}{\kern0pt}Types{\isacharslash}{\kern0pt}Electoral{\isacharunderscore}{\kern0pt}Module{\isachardoublequoteclose}\isanewline
\isakeyword{begin}%
\endisatagtheory
{\isafoldtheory}%
%
\isadelimtheory
%
\endisadelimtheory
%
\begin{isamarkuptext}%
A revised electoral module rejects all originally rejected or deferred
alternatives, and defers the originally elected alternatives.
It does not elect any alternatives.%
\end{isamarkuptext}\isamarkuptrue%
%
\isadelimdocument
%
\endisadelimdocument
%
\isatagdocument
%
\isamarkupsubsection{Definition%
}
\isamarkuptrue%
%
\endisatagdocument
{\isafolddocument}%
%
\isadelimdocument
%
\endisadelimdocument
\isacommand{fun}\isamarkupfalse%
\ revision{\isacharunderscore}{\kern0pt}composition\ {\isacharcolon}{\kern0pt}{\isacharcolon}{\kern0pt}\ {\isachardoublequoteopen}{\isacharprime}{\kern0pt}a\ Electoral{\isacharunderscore}{\kern0pt}Module\ {\isasymRightarrow}\ {\isacharprime}{\kern0pt}a\ Electoral{\isacharunderscore}{\kern0pt}Module{\isachardoublequoteclose}\ \isakeyword{where}\isanewline
\ \ {\isachardoublequoteopen}revision{\isacharunderscore}{\kern0pt}composition\ m\ A\ p\ {\isacharequal}{\kern0pt}\ {\isacharparenleft}{\kern0pt}{\isacharbraceleft}{\kern0pt}{\isacharbraceright}{\kern0pt}{\isacharcomma}{\kern0pt}\ A\ {\isacharminus}{\kern0pt}\ elect\ m\ A\ p{\isacharcomma}{\kern0pt}\ elect\ m\ A\ p{\isacharparenright}{\kern0pt}{\isachardoublequoteclose}\isanewline
\isanewline
\isacommand{abbreviation}\isamarkupfalse%
\ rev\ {\isacharcolon}{\kern0pt}{\isacharcolon}{\kern0pt}\isanewline
{\isachardoublequoteopen}{\isacharprime}{\kern0pt}a\ Electoral{\isacharunderscore}{\kern0pt}Module\ {\isasymRightarrow}\ {\isacharprime}{\kern0pt}a\ Electoral{\isacharunderscore}{\kern0pt}Module{\isachardoublequoteclose}\ {\isacharparenleft}{\kern0pt}{\isachardoublequoteopen}{\isacharunderscore}{\kern0pt}{\isasymdown}{\isachardoublequoteclose}\ {\isadigit{5}}{\isadigit{0}}{\isacharparenright}{\kern0pt}\ \isakeyword{where}\isanewline
\ \ {\isachardoublequoteopen}m{\isasymdown}\ {\isacharequal}{\kern0pt}{\isacharequal}{\kern0pt}\ revision{\isacharunderscore}{\kern0pt}composition\ m{\isachardoublequoteclose}%
\isadelimdocument
%
\endisadelimdocument
%
\isatagdocument
%
\isamarkupsubsection{Soundness%
}
\isamarkuptrue%
%
\endisatagdocument
{\isafolddocument}%
%
\isadelimdocument
%
\endisadelimdocument
\isacommand{theorem}\isamarkupfalse%
\ rev{\isacharunderscore}{\kern0pt}comp{\isacharunderscore}{\kern0pt}sound{\isacharbrackleft}{\kern0pt}simp{\isacharbrackright}{\kern0pt}{\isacharcolon}{\kern0pt}\isanewline
\ \ \isakeyword{assumes}\ module{\isacharcolon}{\kern0pt}\ {\isachardoublequoteopen}electoral{\isacharunderscore}{\kern0pt}module\ m{\isachardoublequoteclose}\isanewline
\ \ \isakeyword{shows}\ {\isachardoublequoteopen}electoral{\isacharunderscore}{\kern0pt}module\ {\isacharparenleft}{\kern0pt}revision{\isacharunderscore}{\kern0pt}composition\ m{\isacharparenright}{\kern0pt}{\isachardoublequoteclose}\isanewline
%
\isadelimproof
%
\endisadelimproof
%
\isatagproof
\isacommand{proof}\isamarkupfalse%
\ {\isacharminus}{\kern0pt}\isanewline
\ \ \isacommand{from}\isamarkupfalse%
\ module\ \isacommand{have}\isamarkupfalse%
\ {\isachardoublequoteopen}{\isasymforall}A\ p{\isachardot}{\kern0pt}\ finite{\isacharunderscore}{\kern0pt}profile\ A\ p\ {\isasymlongrightarrow}\ elect\ m\ A\ p\ {\isasymsubseteq}\ A{\isachardoublequoteclose}\isanewline
\ \ \ \ \isacommand{using}\isamarkupfalse%
\ elect{\isacharunderscore}{\kern0pt}in{\isacharunderscore}{\kern0pt}alts\isanewline
\ \ \ \ \isacommand{by}\isamarkupfalse%
\ auto\isanewline
\ \ \isacommand{hence}\isamarkupfalse%
\ {\isachardoublequoteopen}{\isasymforall}A\ p{\isachardot}{\kern0pt}\ finite{\isacharunderscore}{\kern0pt}profile\ A\ p\ {\isasymlongrightarrow}\ {\isacharparenleft}{\kern0pt}A\ {\isacharminus}{\kern0pt}\ elect\ m\ A\ p{\isacharparenright}{\kern0pt}\ {\isasymunion}\ elect\ m\ A\ p\ {\isacharequal}{\kern0pt}\ A{\isachardoublequoteclose}\isanewline
\ \ \ \ \isacommand{by}\isamarkupfalse%
\ blast\isanewline
\ \ \isacommand{hence}\isamarkupfalse%
\ unity{\isacharcolon}{\kern0pt}\isanewline
\ \ \ \ {\isachardoublequoteopen}{\isasymforall}A\ p{\isachardot}{\kern0pt}\ finite{\isacharunderscore}{\kern0pt}profile\ A\ p\ {\isasymlongrightarrow}\isanewline
\ \ \ \ \ \ set{\isacharunderscore}{\kern0pt}equals{\isacharunderscore}{\kern0pt}partition\ A\ {\isacharparenleft}{\kern0pt}revision{\isacharunderscore}{\kern0pt}composition\ m\ A\ p{\isacharparenright}{\kern0pt}{\isachardoublequoteclose}\isanewline
\ \ \ \ \isacommand{by}\isamarkupfalse%
\ simp\isanewline
\ \ \isacommand{have}\isamarkupfalse%
\ {\isachardoublequoteopen}{\isasymforall}A\ p{\isachardot}{\kern0pt}\ finite{\isacharunderscore}{\kern0pt}profile\ A\ p\ {\isasymlongrightarrow}\ {\isacharparenleft}{\kern0pt}A\ {\isacharminus}{\kern0pt}\ elect\ m\ A\ p{\isacharparenright}{\kern0pt}\ {\isasyminter}\ elect\ m\ A\ p\ {\isacharequal}{\kern0pt}\ {\isacharbraceleft}{\kern0pt}{\isacharbraceright}{\kern0pt}{\isachardoublequoteclose}\isanewline
\ \ \ \ \isacommand{by}\isamarkupfalse%
\ blast\isanewline
\ \ \isacommand{hence}\isamarkupfalse%
\ disjoint{\isacharcolon}{\kern0pt}\isanewline
\ \ \ \ {\isachardoublequoteopen}{\isasymforall}A\ p{\isachardot}{\kern0pt}\ finite{\isacharunderscore}{\kern0pt}profile\ A\ p\ {\isasymlongrightarrow}\ disjoint{\isadigit{3}}\ {\isacharparenleft}{\kern0pt}revision{\isacharunderscore}{\kern0pt}composition\ m\ A\ p{\isacharparenright}{\kern0pt}{\isachardoublequoteclose}\isanewline
\ \ \ \ \isacommand{by}\isamarkupfalse%
\ simp\isanewline
\ \ \isacommand{from}\isamarkupfalse%
\ unity\ disjoint\ \isacommand{show}\isamarkupfalse%
\ {\isacharquery}{\kern0pt}thesis\isanewline
\ \ \ \ \isacommand{by}\isamarkupfalse%
\ {\isacharparenleft}{\kern0pt}simp\ add{\isacharcolon}{\kern0pt}\ electoral{\isacharunderscore}{\kern0pt}modI{\isacharparenright}{\kern0pt}\isanewline
\isacommand{qed}\isamarkupfalse%
%
\endisatagproof
{\isafoldproof}%
%
\isadelimproof
%
\endisadelimproof
%
\isadelimdocument
%
\endisadelimdocument
%
\isatagdocument
%
\isamarkupsubsection{Composition Rules%
}
\isamarkuptrue%
%
\endisatagdocument
{\isafolddocument}%
%
\isadelimdocument
%
\endisadelimdocument
\isacommand{theorem}\isamarkupfalse%
\ rev{\isacharunderscore}{\kern0pt}comp{\isacharunderscore}{\kern0pt}non{\isacharunderscore}{\kern0pt}electing{\isacharbrackleft}{\kern0pt}simp{\isacharbrackright}{\kern0pt}{\isacharcolon}{\kern0pt}\isanewline
\ \ \isakeyword{assumes}\ {\isachardoublequoteopen}electoral{\isacharunderscore}{\kern0pt}module\ m{\isachardoublequoteclose}\isanewline
\ \ \isakeyword{shows}\ {\isachardoublequoteopen}non{\isacharunderscore}{\kern0pt}electing\ {\isacharparenleft}{\kern0pt}m{\isasymdown}{\isacharparenright}{\kern0pt}{\isachardoublequoteclose}\isanewline
%
\isadelimproof
\ \ %
\endisadelimproof
%
\isatagproof
\isacommand{by}\isamarkupfalse%
\ {\isacharparenleft}{\kern0pt}simp\ add{\isacharcolon}{\kern0pt}\ assms\ non{\isacharunderscore}{\kern0pt}electing{\isacharunderscore}{\kern0pt}def{\isacharparenright}{\kern0pt}%
\endisatagproof
{\isafoldproof}%
%
\isadelimproof
\isanewline
%
\endisadelimproof
\isanewline
\isanewline
\isacommand{theorem}\isamarkupfalse%
\ rev{\isacharunderscore}{\kern0pt}comp{\isacharunderscore}{\kern0pt}non{\isacharunderscore}{\kern0pt}blocking{\isacharbrackleft}{\kern0pt}simp{\isacharbrackright}{\kern0pt}{\isacharcolon}{\kern0pt}\isanewline
\ \ \isakeyword{assumes}\ {\isachardoublequoteopen}electing\ m{\isachardoublequoteclose}\isanewline
\ \ \isakeyword{shows}\ {\isachardoublequoteopen}non{\isacharunderscore}{\kern0pt}blocking\ {\isacharparenleft}{\kern0pt}m{\isasymdown}{\isacharparenright}{\kern0pt}{\isachardoublequoteclose}\isanewline
%
\isadelimproof
\ \ %
\endisadelimproof
%
\isatagproof
\isacommand{unfolding}\isamarkupfalse%
\ non{\isacharunderscore}{\kern0pt}blocking{\isacharunderscore}{\kern0pt}def\isanewline
\isacommand{proof}\isamarkupfalse%
\ {\isacharparenleft}{\kern0pt}safe{\isacharcomma}{\kern0pt}\ simp{\isacharunderscore}{\kern0pt}all{\isacharparenright}{\kern0pt}\isanewline
\ \ \isacommand{show}\isamarkupfalse%
\ {\isachardoublequoteopen}electoral{\isacharunderscore}{\kern0pt}module\ {\isacharparenleft}{\kern0pt}m{\isasymdown}{\isacharparenright}{\kern0pt}{\isachardoublequoteclose}\isanewline
\ \ \ \ \isacommand{using}\isamarkupfalse%
\ assms\ electing{\isacharunderscore}{\kern0pt}def\ rev{\isacharunderscore}{\kern0pt}comp{\isacharunderscore}{\kern0pt}sound\isanewline
\ \ \ \ \isacommand{by}\isamarkupfalse%
\ {\isacharparenleft}{\kern0pt}metis\ {\isacharparenleft}{\kern0pt}no{\isacharunderscore}{\kern0pt}types{\isacharcomma}{\kern0pt}\ lifting{\isacharparenright}{\kern0pt}{\isacharparenright}{\kern0pt}\isanewline
\isacommand{next}\isamarkupfalse%
\isanewline
\ \ \isacommand{fix}\isamarkupfalse%
\isanewline
\ \ \ \ A\ {\isacharcolon}{\kern0pt}{\isacharcolon}{\kern0pt}\ {\isachardoublequoteopen}{\isacharprime}{\kern0pt}a\ set{\isachardoublequoteclose}\ \isakeyword{and}\isanewline
\ \ \ \ p\ {\isacharcolon}{\kern0pt}{\isacharcolon}{\kern0pt}\ {\isachardoublequoteopen}{\isacharprime}{\kern0pt}a\ Profile{\isachardoublequoteclose}\ \isakeyword{and}\isanewline
\ \ \ \ x\ {\isacharcolon}{\kern0pt}{\isacharcolon}{\kern0pt}\ {\isachardoublequoteopen}{\isacharprime}{\kern0pt}a{\isachardoublequoteclose}\isanewline
\ \ \isacommand{assume}\isamarkupfalse%
\isanewline
\ \ \ \ fin{\isacharunderscore}{\kern0pt}A{\isacharcolon}{\kern0pt}\ {\isachardoublequoteopen}finite\ A{\isachardoublequoteclose}\ \isakeyword{and}\isanewline
\ \ \ \ prof{\isacharunderscore}{\kern0pt}A{\isacharcolon}{\kern0pt}\ {\isachardoublequoteopen}profile\ A\ p{\isachardoublequoteclose}\ \isakeyword{and}\isanewline
\ \ \ \ no{\isacharunderscore}{\kern0pt}elect{\isacharcolon}{\kern0pt}\ {\isachardoublequoteopen}A\ {\isacharminus}{\kern0pt}\ elect\ m\ A\ p\ {\isacharequal}{\kern0pt}\ A{\isachardoublequoteclose}\ \isakeyword{and}\isanewline
\ \ \ \ x{\isacharunderscore}{\kern0pt}in{\isacharunderscore}{\kern0pt}A{\isacharcolon}{\kern0pt}\ {\isachardoublequoteopen}x\ {\isasymin}\ A{\isachardoublequoteclose}\isanewline
\ \ \isacommand{from}\isamarkupfalse%
\ no{\isacharunderscore}{\kern0pt}elect\ \isacommand{have}\isamarkupfalse%
\ non{\isacharunderscore}{\kern0pt}elect{\isacharcolon}{\kern0pt}\isanewline
\ \ \ \ {\isachardoublequoteopen}non{\isacharunderscore}{\kern0pt}electing\ m{\isachardoublequoteclose}\isanewline
\ \ \ \ \isacommand{using}\isamarkupfalse%
\ assms\ prof{\isacharunderscore}{\kern0pt}A\ x{\isacharunderscore}{\kern0pt}in{\isacharunderscore}{\kern0pt}A\ fin{\isacharunderscore}{\kern0pt}A\ electing{\isacharunderscore}{\kern0pt}def\ empty{\isacharunderscore}{\kern0pt}iff\isanewline
\ \ \ \ \ \ \ \ \ \ Diff{\isacharunderscore}{\kern0pt}disjoint\ Int{\isacharunderscore}{\kern0pt}absorb{\isadigit{2}}\ elect{\isacharunderscore}{\kern0pt}in{\isacharunderscore}{\kern0pt}alts\isanewline
\ \ \ \ \isacommand{by}\isamarkupfalse%
\ {\isacharparenleft}{\kern0pt}metis\ {\isacharparenleft}{\kern0pt}no{\isacharunderscore}{\kern0pt}types{\isacharcomma}{\kern0pt}\ lifting{\isacharparenright}{\kern0pt}{\isacharparenright}{\kern0pt}\isanewline
\ \ \isacommand{show}\isamarkupfalse%
\ {\isachardoublequoteopen}False{\isachardoublequoteclose}\isanewline
\ \ \ \ \isacommand{using}\isamarkupfalse%
\ non{\isacharunderscore}{\kern0pt}elect\ assms\ electing{\isacharunderscore}{\kern0pt}def\ empty{\isacharunderscore}{\kern0pt}iff\ fin{\isacharunderscore}{\kern0pt}A\isanewline
\ \ \ \ \ \ \ \ \ \ non{\isacharunderscore}{\kern0pt}electing{\isacharunderscore}{\kern0pt}def\ prof{\isacharunderscore}{\kern0pt}A\ x{\isacharunderscore}{\kern0pt}in{\isacharunderscore}{\kern0pt}A\isanewline
\ \ \ \ \isacommand{by}\isamarkupfalse%
\ {\isacharparenleft}{\kern0pt}metis\ {\isacharparenleft}{\kern0pt}no{\isacharunderscore}{\kern0pt}types{\isacharcomma}{\kern0pt}\ lifting{\isacharparenright}{\kern0pt}{\isacharparenright}{\kern0pt}\isanewline
\isacommand{qed}\isamarkupfalse%
%
\endisatagproof
{\isafoldproof}%
%
\isadelimproof
\isanewline
%
\endisadelimproof
\isanewline
\isanewline
\isacommand{theorem}\isamarkupfalse%
\ rev{\isacharunderscore}{\kern0pt}comp{\isacharunderscore}{\kern0pt}def{\isacharunderscore}{\kern0pt}inv{\isacharunderscore}{\kern0pt}mono{\isacharbrackleft}{\kern0pt}simp{\isacharbrackright}{\kern0pt}{\isacharcolon}{\kern0pt}\isanewline
\ \ \isakeyword{assumes}\ {\isachardoublequoteopen}invariant{\isacharunderscore}{\kern0pt}monotonicity\ m{\isachardoublequoteclose}\isanewline
\ \ \isakeyword{shows}\ {\isachardoublequoteopen}defer{\isacharunderscore}{\kern0pt}invariant{\isacharunderscore}{\kern0pt}monotonicity\ {\isacharparenleft}{\kern0pt}m{\isasymdown}{\isacharparenright}{\kern0pt}{\isachardoublequoteclose}\isanewline
%
\isadelimproof
%
\endisadelimproof
%
\isatagproof
\isacommand{proof}\isamarkupfalse%
\ {\isacharminus}{\kern0pt}\isanewline
\ \ \isacommand{have}\isamarkupfalse%
\ {\isachardoublequoteopen}{\isasymforall}A\ p\ q\ w{\isachardot}{\kern0pt}\ {\isacharparenleft}{\kern0pt}w\ {\isasymin}\ defer\ {\isacharparenleft}{\kern0pt}m{\isasymdown}{\isacharparenright}{\kern0pt}\ A\ p\ {\isasymand}\ lifted\ A\ p\ q\ w{\isacharparenright}{\kern0pt}\ {\isasymlongrightarrow}\isanewline
\ \ \ \ \ \ \ \ \ \ \ \ \ \ \ \ \ \ {\isacharparenleft}{\kern0pt}defer\ {\isacharparenleft}{\kern0pt}m{\isasymdown}{\isacharparenright}{\kern0pt}\ A\ q\ {\isacharequal}{\kern0pt}\ defer\ {\isacharparenleft}{\kern0pt}m{\isasymdown}{\isacharparenright}{\kern0pt}\ A\ p\ {\isasymor}\ defer\ {\isacharparenleft}{\kern0pt}m{\isasymdown}{\isacharparenright}{\kern0pt}\ A\ q\ {\isacharequal}{\kern0pt}\ {\isacharbraceleft}{\kern0pt}w{\isacharbraceright}{\kern0pt}{\isacharparenright}{\kern0pt}{\isachardoublequoteclose}\isanewline
\ \ \ \ \isacommand{using}\isamarkupfalse%
\ assms\isanewline
\ \ \ \ \isacommand{by}\isamarkupfalse%
\ {\isacharparenleft}{\kern0pt}simp\ add{\isacharcolon}{\kern0pt}\ invariant{\isacharunderscore}{\kern0pt}monotonicity{\isacharunderscore}{\kern0pt}def{\isacharparenright}{\kern0pt}\isanewline
\ \ \isacommand{moreover}\isamarkupfalse%
\ \isacommand{have}\isamarkupfalse%
\ {\isachardoublequoteopen}electoral{\isacharunderscore}{\kern0pt}module\ {\isacharparenleft}{\kern0pt}m{\isasymdown}{\isacharparenright}{\kern0pt}{\isachardoublequoteclose}\isanewline
\ \ \ \ \isacommand{using}\isamarkupfalse%
\ assms\ rev{\isacharunderscore}{\kern0pt}comp{\isacharunderscore}{\kern0pt}sound\ invariant{\isacharunderscore}{\kern0pt}monotonicity{\isacharunderscore}{\kern0pt}def\isanewline
\ \ \ \ \isacommand{by}\isamarkupfalse%
\ auto\isanewline
\ \ \isacommand{moreover}\isamarkupfalse%
\ \isacommand{have}\isamarkupfalse%
\ {\isachardoublequoteopen}non{\isacharunderscore}{\kern0pt}electing\ {\isacharparenleft}{\kern0pt}m{\isasymdown}{\isacharparenright}{\kern0pt}{\isachardoublequoteclose}\isanewline
\ \ \ \ \isacommand{using}\isamarkupfalse%
\ assms\ rev{\isacharunderscore}{\kern0pt}comp{\isacharunderscore}{\kern0pt}non{\isacharunderscore}{\kern0pt}electing\ invariant{\isacharunderscore}{\kern0pt}monotonicity{\isacharunderscore}{\kern0pt}def\isanewline
\ \ \ \ \isacommand{by}\isamarkupfalse%
\ auto\isanewline
\ \ \isacommand{ultimately}\isamarkupfalse%
\ \isacommand{have}\isamarkupfalse%
\ {\isachardoublequoteopen}electoral{\isacharunderscore}{\kern0pt}module\ {\isacharparenleft}{\kern0pt}m{\isasymdown}{\isacharparenright}{\kern0pt}\ {\isasymand}\ non{\isacharunderscore}{\kern0pt}electing\ {\isacharparenleft}{\kern0pt}m{\isasymdown}{\isacharparenright}{\kern0pt}\ {\isasymand}\isanewline
\ \ \ \ \ \ {\isacharparenleft}{\kern0pt}{\isasymforall}A\ p\ q\ w{\isachardot}{\kern0pt}\ {\isacharparenleft}{\kern0pt}w\ {\isasymin}\ defer\ {\isacharparenleft}{\kern0pt}m{\isasymdown}{\isacharparenright}{\kern0pt}\ A\ p\ {\isasymand}\ lifted\ A\ p\ q\ w{\isacharparenright}{\kern0pt}\ {\isasymlongrightarrow}\isanewline
\ \ \ \ \ \ \ \ \ \ \ \ \ \ \ \ \ {\isacharparenleft}{\kern0pt}defer\ {\isacharparenleft}{\kern0pt}m{\isasymdown}{\isacharparenright}{\kern0pt}\ A\ q\ {\isacharequal}{\kern0pt}\ defer\ {\isacharparenleft}{\kern0pt}m{\isasymdown}{\isacharparenright}{\kern0pt}\ A\ p\ {\isasymor}\ defer\ {\isacharparenleft}{\kern0pt}m{\isasymdown}{\isacharparenright}{\kern0pt}\ A\ q\ {\isacharequal}{\kern0pt}\ {\isacharbraceleft}{\kern0pt}w{\isacharbraceright}{\kern0pt}{\isacharparenright}{\kern0pt}{\isacharparenright}{\kern0pt}{\isachardoublequoteclose}\isanewline
\ \ \ \ \isacommand{by}\isamarkupfalse%
\ blast\isanewline
\ \ \isacommand{thus}\isamarkupfalse%
\ {\isacharquery}{\kern0pt}thesis\isanewline
\ \ \ \ \isacommand{using}\isamarkupfalse%
\ defer{\isacharunderscore}{\kern0pt}invariant{\isacharunderscore}{\kern0pt}monotonicity{\isacharunderscore}{\kern0pt}def\isanewline
\ \ \ \ \isacommand{by}\isamarkupfalse%
\ {\isacharparenleft}{\kern0pt}simp\ add{\isacharcolon}{\kern0pt}\ defer{\isacharunderscore}{\kern0pt}invariant{\isacharunderscore}{\kern0pt}monotonicity{\isacharunderscore}{\kern0pt}def{\isacharparenright}{\kern0pt}\isanewline
\isacommand{qed}\isamarkupfalse%
%
\endisatagproof
{\isafoldproof}%
%
\isadelimproof
\isanewline
%
\endisadelimproof
%
\isadelimtheory
\isanewline
%
\endisadelimtheory
%
\isatagtheory
\isacommand{end}\isamarkupfalse%
%
\endisatagtheory
{\isafoldtheory}%
%
\isadelimtheory
%
\endisadelimtheory
%
\end{isabellebody}%
\endinput
%:%file=~/Documents/Studies/VotingRuleGenerator/virage/src/test/resources/old_theories/Compositional_Structures/Revision_Composition.thy%:%
%:%6=3%:%
%:%11=4%:%
%:%12=5%:%
%:%14=8%:%
%:%30=10%:%
%:%31=10%:%
%:%32=11%:%
%:%33=12%:%
%:%42=15%:%
%:%43=16%:%
%:%44=17%:%
%:%53=19%:%
%:%63=21%:%
%:%64=21%:%
%:%65=22%:%
%:%66=23%:%
%:%67=24%:%
%:%68=24%:%
%:%69=25%:%
%:%70=26%:%
%:%77=28%:%
%:%87=30%:%
%:%88=30%:%
%:%89=31%:%
%:%90=32%:%
%:%97=33%:%
%:%98=33%:%
%:%99=34%:%
%:%100=34%:%
%:%101=34%:%
%:%102=35%:%
%:%103=35%:%
%:%104=36%:%
%:%105=36%:%
%:%106=37%:%
%:%107=37%:%
%:%108=38%:%
%:%109=38%:%
%:%110=39%:%
%:%111=39%:%
%:%112=40%:%
%:%113=41%:%
%:%114=42%:%
%:%115=42%:%
%:%116=43%:%
%:%117=43%:%
%:%118=44%:%
%:%119=44%:%
%:%120=45%:%
%:%121=45%:%
%:%122=46%:%
%:%123=47%:%
%:%124=47%:%
%:%125=48%:%
%:%126=48%:%
%:%127=48%:%
%:%128=49%:%
%:%129=49%:%
%:%130=50%:%
%:%145=52%:%
%:%155=55%:%
%:%156=55%:%
%:%157=56%:%
%:%158=57%:%
%:%161=58%:%
%:%165=58%:%
%:%166=58%:%
%:%171=58%:%
%:%174=59%:%
%:%175=63%:%
%:%176=64%:%
%:%177=64%:%
%:%178=65%:%
%:%179=66%:%
%:%182=67%:%
%:%186=67%:%
%:%187=67%:%
%:%188=68%:%
%:%189=68%:%
%:%190=69%:%
%:%191=69%:%
%:%192=70%:%
%:%193=70%:%
%:%194=71%:%
%:%195=71%:%
%:%196=72%:%
%:%197=72%:%
%:%198=73%:%
%:%199=73%:%
%:%200=74%:%
%:%201=75%:%
%:%202=76%:%
%:%203=77%:%
%:%204=77%:%
%:%205=78%:%
%:%206=79%:%
%:%207=80%:%
%:%208=81%:%
%:%209=82%:%
%:%210=82%:%
%:%211=82%:%
%:%212=83%:%
%:%213=84%:%
%:%214=84%:%
%:%215=85%:%
%:%216=86%:%
%:%217=86%:%
%:%218=87%:%
%:%219=87%:%
%:%220=88%:%
%:%221=88%:%
%:%222=89%:%
%:%223=90%:%
%:%224=90%:%
%:%225=91%:%
%:%231=91%:%
%:%234=92%:%
%:%235=96%:%
%:%236=97%:%
%:%237=97%:%
%:%238=98%:%
%:%239=99%:%
%:%246=100%:%
%:%247=100%:%
%:%248=101%:%
%:%249=101%:%
%:%250=102%:%
%:%251=103%:%
%:%252=103%:%
%:%253=104%:%
%:%254=104%:%
%:%255=105%:%
%:%256=105%:%
%:%257=105%:%
%:%258=106%:%
%:%259=106%:%
%:%260=107%:%
%:%261=107%:%
%:%262=108%:%
%:%263=108%:%
%:%264=108%:%
%:%265=109%:%
%:%266=109%:%
%:%267=110%:%
%:%268=110%:%
%:%269=111%:%
%:%270=111%:%
%:%271=111%:%
%:%273=113%:%
%:%274=114%:%
%:%275=114%:%
%:%276=115%:%
%:%277=115%:%
%:%278=116%:%
%:%279=116%:%
%:%280=117%:%
%:%281=117%:%
%:%282=118%:%
%:%288=118%:%
%:%293=119%:%
%:%298=120%:%
%
\begin{isabellebody}%
\setisabellecontext{Borda{\isacharunderscore}{\kern0pt}Rule}%
%
\isadelimdocument
\isanewline
%
\endisadelimdocument
%
\isatagdocument
\isanewline
%
\isamarkupchapter{Voting Rules%
}
\isamarkuptrue%
%
\isamarkupsection{Borda Rule%
}
\isamarkuptrue%
%
\endisatagdocument
{\isafolddocument}%
%
\isadelimdocument
%
\endisadelimdocument
%
\isadelimtheory
%
\endisadelimtheory
%
\isatagtheory
\isacommand{theory}\isamarkupfalse%
\ Borda{\isacharunderscore}{\kern0pt}Rule\isanewline
\ \ \isakeyword{imports}\ {\isachardoublequoteopen}{\isachardot}{\kern0pt}{\isachardot}{\kern0pt}{\isacharslash}{\kern0pt}Compositional{\isacharunderscore}{\kern0pt}Framework{\isacharslash}{\kern0pt}Components{\isacharslash}{\kern0pt}Composites{\isacharslash}{\kern0pt}Composite{\isacharunderscore}{\kern0pt}Elimination{\isacharunderscore}{\kern0pt}Modules{\isachardoublequoteclose}\isanewline
\ \ \ \ \ \ \ \ \ \ {\isachardoublequoteopen}{\isachardot}{\kern0pt}{\isachardot}{\kern0pt}{\isacharslash}{\kern0pt}Compositional{\isacharunderscore}{\kern0pt}Framework{\isacharslash}{\kern0pt}Components{\isacharslash}{\kern0pt}Composites{\isacharslash}{\kern0pt}Composite{\isacharunderscore}{\kern0pt}Structures{\isachardoublequoteclose}\isanewline
\isanewline
\isakeyword{begin}%
\endisatagtheory
{\isafoldtheory}%
%
\isadelimtheory
%
\endisadelimtheory
%
\begin{isamarkuptext}%
This is the Borda rule. On each ballot, each alternative is assigned a score
that depends on how many alternatives are ranked below. The sum of all such
scores for an alternative is hence called their Borda score. The alternative
with the highest Borda score is elected.%
\end{isamarkuptext}\isamarkuptrue%
%
\isadelimdocument
%
\endisadelimdocument
%
\isatagdocument
%
\isamarkupsubsection{Definition%
}
\isamarkuptrue%
%
\endisatagdocument
{\isafolddocument}%
%
\isadelimdocument
%
\endisadelimdocument
\isacommand{fun}\isamarkupfalse%
\ borda{\isacharunderscore}{\kern0pt}rule\ {\isacharcolon}{\kern0pt}{\isacharcolon}{\kern0pt}\ {\isachardoublequoteopen}{\isacharprime}{\kern0pt}a\ Electoral{\isacharunderscore}{\kern0pt}Module{\isachardoublequoteclose}\ \isakeyword{where}\isanewline
\ \ {\isachardoublequoteopen}borda{\isacharunderscore}{\kern0pt}rule\ A\ p\ {\isacharequal}{\kern0pt}\ elector\ borda\ A\ p{\isachardoublequoteclose}\isanewline
%
\isadelimtheory
\isanewline
%
\endisadelimtheory
%
\isatagtheory
\isacommand{end}\isamarkupfalse%
%
\endisatagtheory
{\isafoldtheory}%
%
\isadelimtheory
%
\endisadelimtheory
%
\end{isabellebody}%
\endinput
%:%file=~/Documents/Studies/VotingRuleGenerator/virage/src/test/resources/verifiedVotingRuleConstruction/theories/Voting_Rules/Borda_Rule.thy%:%
%:%6=3%:%
%:%11=4%:%
%:%13=7%:%
%:%17=9%:%
%:%33=11%:%
%:%34=11%:%
%:%35=12%:%
%:%36=13%:%
%:%37=14%:%
%:%38=15%:%
%:%47=18%:%
%:%48=19%:%
%:%49=20%:%
%:%50=21%:%
%:%59=23%:%
%:%69=25%:%
%:%70=25%:%
%:%71=26%:%
%:%74=27%:%
%:%79=28%:%
%
\begin{isabellebody}%
\setisabellecontext{Condorcet{\isacharunderscore}{\kern0pt}Properties}%
%
\isadelimtheory
%
\endisadelimtheory
%
\isatagtheory
\isacommand{theory}\isamarkupfalse%
\ Condorcet{\isacharunderscore}{\kern0pt}Properties\isanewline
\ \ \isakeyword{imports}\ {\isachardoublequoteopen}{\isachardot}{\kern0pt}{\isachardot}{\kern0pt}{\isacharslash}{\kern0pt}Components{\isacharslash}{\kern0pt}Electoral{\isacharunderscore}{\kern0pt}Module{\isachardoublequoteclose}\isanewline
\ \ \ \ \ \ \ \ \ \ {\isachardoublequoteopen}{\isachardot}{\kern0pt}{\isachardot}{\kern0pt}{\isacharslash}{\kern0pt}Components{\isacharslash}{\kern0pt}Evaluation{\isacharunderscore}{\kern0pt}Function{\isachardoublequoteclose}\isanewline
\isanewline
\isakeyword{begin}%
\endisatagtheory
{\isafoldtheory}%
%
\isadelimtheory
\isanewline
%
\endisadelimtheory
\isanewline
\isacommand{definition}\isamarkupfalse%
\ condorcet{\isacharunderscore}{\kern0pt}compatibility\ {\isacharcolon}{\kern0pt}{\isacharcolon}{\kern0pt}\ {\isachardoublequoteopen}{\isacharprime}{\kern0pt}a\ Electoral{\isacharunderscore}{\kern0pt}Module\ {\isasymRightarrow}\ bool{\isachardoublequoteclose}\ \isakeyword{where}\isanewline
\ \ {\isachardoublequoteopen}condorcet{\isacharunderscore}{\kern0pt}compatibility\ m\ {\isasymequiv}\isanewline
\ \ \ \ electoral{\isacharunderscore}{\kern0pt}module\ m\ {\isasymand}\isanewline
\ \ \ \ {\isacharparenleft}{\kern0pt}{\isasymforall}\ A\ p\ w{\isachardot}{\kern0pt}\ condorcet{\isacharunderscore}{\kern0pt}winner\ A\ p\ w\ {\isasymand}\ finite\ A\ {\isasymlongrightarrow}\isanewline
\ \ \ \ \ \ {\isacharparenleft}{\kern0pt}w\ {\isasymnotin}\ reject\ m\ A\ p\ {\isasymand}\isanewline
\ \ \ \ \ \ \ \ {\isacharparenleft}{\kern0pt}{\isasymforall}l{\isachardot}{\kern0pt}\ {\isasymnot}condorcet{\isacharunderscore}{\kern0pt}winner\ A\ p\ l\ {\isasymlongrightarrow}\ l\ {\isasymnotin}\ elect\ m\ A\ p{\isacharparenright}{\kern0pt}\ {\isasymand}\isanewline
\ \ \ \ \ \ \ \ \ \ {\isacharparenleft}{\kern0pt}w\ {\isasymin}\ elect\ m\ A\ p\ {\isasymlongrightarrow}\isanewline
\ \ \ \ \ \ \ \ \ \ \ \ {\isacharparenleft}{\kern0pt}{\isasymforall}l{\isachardot}{\kern0pt}\ {\isasymnot}condorcet{\isacharunderscore}{\kern0pt}winner\ A\ p\ l\ {\isasymlongrightarrow}\ l\ {\isasymin}\ reject\ m\ A\ p{\isacharparenright}{\kern0pt}{\isacharparenright}{\kern0pt}{\isacharparenright}{\kern0pt}{\isacharparenright}{\kern0pt}{\isachardoublequoteclose}\isanewline
\isanewline
\isanewline
\isacommand{definition}\isamarkupfalse%
\ condorcet{\isacharunderscore}{\kern0pt}rating\ {\isacharcolon}{\kern0pt}{\isacharcolon}{\kern0pt}\ {\isachardoublequoteopen}{\isacharprime}{\kern0pt}a\ Evaluation{\isacharunderscore}{\kern0pt}Function\ {\isasymRightarrow}\ bool{\isachardoublequoteclose}\ \isakeyword{where}\isanewline
\ \ {\isachardoublequoteopen}condorcet{\isacharunderscore}{\kern0pt}rating\ f\ {\isasymequiv}\isanewline
\ \ \ \ {\isasymforall}A\ p\ w\ {\isachardot}{\kern0pt}\ condorcet{\isacharunderscore}{\kern0pt}winner\ A\ p\ w\ {\isasymlongrightarrow}\isanewline
\ \ \ \ \ \ {\isacharparenleft}{\kern0pt}{\isasymforall}l\ {\isasymin}\ A\ {\isachardot}{\kern0pt}\ l\ {\isasymnoteq}\ w\ {\isasymlongrightarrow}\ f\ l\ A\ p\ {\isacharless}{\kern0pt}\ f\ w\ A\ p{\isacharparenright}{\kern0pt}{\isachardoublequoteclose}\isanewline
\isanewline
\isacommand{definition}\isamarkupfalse%
\ defer{\isacharunderscore}{\kern0pt}condorcet{\isacharunderscore}{\kern0pt}consistency\ {\isacharcolon}{\kern0pt}{\isacharcolon}{\kern0pt}\ {\isachardoublequoteopen}{\isacharprime}{\kern0pt}a\ Electoral{\isacharunderscore}{\kern0pt}Module\ {\isasymRightarrow}\ bool{\isachardoublequoteclose}\ \isakeyword{where}\isanewline
\ \ {\isachardoublequoteopen}defer{\isacharunderscore}{\kern0pt}condorcet{\isacharunderscore}{\kern0pt}consistency\ m\ {\isasymequiv}\isanewline
\ \ \ \ electoral{\isacharunderscore}{\kern0pt}module\ m\ {\isasymand}\isanewline
\ \ \ \ {\isacharparenleft}{\kern0pt}{\isasymforall}\ A\ p\ w{\isachardot}{\kern0pt}\ condorcet{\isacharunderscore}{\kern0pt}winner\ A\ p\ w\ {\isasymand}\ finite\ A\ {\isasymlongrightarrow}\isanewline
\ \ \ \ \ \ {\isacharparenleft}{\kern0pt}m\ A\ p\ {\isacharequal}{\kern0pt}\isanewline
\ \ \ \ \ \ \ \ {\isacharparenleft}{\kern0pt}{\isacharbraceleft}{\kern0pt}{\isacharbraceright}{\kern0pt}{\isacharcomma}{\kern0pt}\isanewline
\ \ \ \ \ \ \ \ A\ {\isacharminus}{\kern0pt}\ {\isacharparenleft}{\kern0pt}defer\ m\ A\ p{\isacharparenright}{\kern0pt}{\isacharcomma}{\kern0pt}\isanewline
\ \ \ \ \ \ \ \ {\isacharbraceleft}{\kern0pt}d\ {\isasymin}\ A{\isachardot}{\kern0pt}\ condorcet{\isacharunderscore}{\kern0pt}winner\ A\ p\ d{\isacharbraceright}{\kern0pt}{\isacharparenright}{\kern0pt}{\isacharparenright}{\kern0pt}{\isacharparenright}{\kern0pt}{\isachardoublequoteclose}\isanewline
%
\isadelimtheory
\isanewline
%
\endisadelimtheory
%
\isatagtheory
\isacommand{end}\isamarkupfalse%
%
\endisatagtheory
{\isafoldtheory}%
%
\isadelimtheory
%
\endisadelimtheory
%
\end{isabellebody}%
\endinput
%:%file=~/Documents/Studies/VotingRuleGenerator/virage/src/test/resources/verifiedVotingRuleConstruction/theories/Compositional_Framework/Properties/Condorcet_Properties.thy%:%
%:%10=1%:%
%:%11=1%:%
%:%12=2%:%
%:%13=3%:%
%:%14=4%:%
%:%15=5%:%
%:%20=5%:%
%:%23=6%:%
%:%24=7%:%
%:%25=7%:%
%:%26=8%:%
%:%32=14%:%
%:%33=15%:%
%:%34=19%:%
%:%35=20%:%
%:%36=20%:%
%:%37=21%:%
%:%39=23%:%
%:%40=24%:%
%:%41=25%:%
%:%42=25%:%
%:%43=26%:%
%:%49=32%:%
%:%52=33%:%
%:%57=34%:%
%
\begin{isabellebody}%
\setisabellecontext{Condorcet{\isacharunderscore}{\kern0pt}Consistency}%
%
\isadelimtheory
%
\endisadelimtheory
%
\isatagtheory
\isacommand{theory}\isamarkupfalse%
\ Condorcet{\isacharunderscore}{\kern0pt}Consistency\isanewline
\ \ \isakeyword{imports}\ {\isachardoublequoteopen}{\isachardot}{\kern0pt}{\isachardot}{\kern0pt}{\isacharslash}{\kern0pt}Compositional{\isacharunderscore}{\kern0pt}Framework{\isacharslash}{\kern0pt}Components{\isacharslash}{\kern0pt}Electoral{\isacharunderscore}{\kern0pt}Module{\isachardoublequoteclose}\isanewline
\isanewline
\isakeyword{begin}%
\endisatagtheory
{\isafoldtheory}%
%
\isadelimtheory
\isanewline
%
\endisadelimtheory
\isanewline
\isacommand{definition}\isamarkupfalse%
\ condorcet{\isacharunderscore}{\kern0pt}consistency\ {\isacharcolon}{\kern0pt}{\isacharcolon}{\kern0pt}\ {\isachardoublequoteopen}{\isacharprime}{\kern0pt}a\ Electoral{\isacharunderscore}{\kern0pt}Module\ {\isasymRightarrow}\ bool{\isachardoublequoteclose}\ \isakeyword{where}\isanewline
\ \ {\isachardoublequoteopen}condorcet{\isacharunderscore}{\kern0pt}consistency\ m\ {\isasymequiv}\isanewline
\ \ \ \ electoral{\isacharunderscore}{\kern0pt}module\ m\ {\isasymand}\isanewline
\ \ \ \ {\isacharparenleft}{\kern0pt}{\isasymforall}\ A\ p\ w{\isachardot}{\kern0pt}\ condorcet{\isacharunderscore}{\kern0pt}winner\ A\ p\ w\ {\isasymlongrightarrow}\isanewline
\ \ \ \ \ \ {\isacharparenleft}{\kern0pt}m\ A\ p\ {\isacharequal}{\kern0pt}\isanewline
\ \ \ \ \ \ \ \ {\isacharparenleft}{\kern0pt}{\isacharbraceleft}{\kern0pt}e\ {\isasymin}\ A{\isachardot}{\kern0pt}\ condorcet{\isacharunderscore}{\kern0pt}winner\ A\ p\ e{\isacharbraceright}{\kern0pt}{\isacharcomma}{\kern0pt}\isanewline
\ \ \ \ \ \ \ \ \ \ A\ {\isacharminus}{\kern0pt}\ {\isacharparenleft}{\kern0pt}elect\ m\ A\ p{\isacharparenright}{\kern0pt}{\isacharcomma}{\kern0pt}\isanewline
\ \ \ \ \ \ \ \ \ \ {\isacharbraceleft}{\kern0pt}{\isacharbraceright}{\kern0pt}{\isacharparenright}{\kern0pt}{\isacharparenright}{\kern0pt}{\isacharparenright}{\kern0pt}{\isachardoublequoteclose}\isanewline
%
\isadelimtheory
\isanewline
%
\endisadelimtheory
%
\isatagtheory
\isacommand{end}\isamarkupfalse%
%
\endisatagtheory
{\isafoldtheory}%
%
\isadelimtheory
%
\endisadelimtheory
%
\end{isabellebody}%
\endinput
%:%file=~/Documents/Studies/VotingRuleGenerator/virage/src/test/resources/verifiedVotingRuleConstruction/theories/Social_Choice_Properties/Condorcet_Consistency.thy%:%
%:%10=1%:%
%:%11=1%:%
%:%12=2%:%
%:%13=3%:%
%:%14=4%:%
%:%19=4%:%
%:%22=5%:%
%:%23=6%:%
%:%24=6%:%
%:%25=7%:%
%:%31=13%:%
%:%34=14%:%
%:%39=15%:%
%
\begin{isabellebody}%
\setisabellecontext{Condorcet{\isacharunderscore}{\kern0pt}Rules}%
%
\isadelimtheory
%
\endisadelimtheory
%
\isatagtheory
\isacommand{theory}\isamarkupfalse%
\ Condorcet{\isacharunderscore}{\kern0pt}Rules\isanewline
\ \ \isakeyword{imports}\ {\isachardoublequoteopen}{\isachardot}{\kern0pt}{\isachardot}{\kern0pt}{\isacharslash}{\kern0pt}Properties{\isacharslash}{\kern0pt}Condorcet{\isacharunderscore}{\kern0pt}Properties{\isachardoublequoteclose}\isanewline
\ \ \ \ \ \ \ \ \ \ {\isachardoublequoteopen}{\isachardot}{\kern0pt}{\isachardot}{\kern0pt}{\isacharslash}{\kern0pt}{\isachardot}{\kern0pt}{\isachardot}{\kern0pt}{\isacharslash}{\kern0pt}Social{\isacharunderscore}{\kern0pt}Choice{\isacharunderscore}{\kern0pt}Properties{\isacharslash}{\kern0pt}Condorcet{\isacharunderscore}{\kern0pt}Consistency{\isachardoublequoteclose}\isanewline
\ \ \ \ \ \ \ \ \ \ {\isachardoublequoteopen}{\isachardot}{\kern0pt}{\isachardot}{\kern0pt}{\isacharslash}{\kern0pt}Components{\isacharslash}{\kern0pt}Compositional{\isacharunderscore}{\kern0pt}Structures{\isacharslash}{\kern0pt}Sequential{\isacharunderscore}{\kern0pt}Composition{\isachardoublequoteclose}\isanewline
\ \ \ \ \ \ \ \ \ \ {\isachardoublequoteopen}{\isachardot}{\kern0pt}{\isachardot}{\kern0pt}{\isacharslash}{\kern0pt}Components{\isacharslash}{\kern0pt}Composites{\isacharslash}{\kern0pt}Composite{\isacharunderscore}{\kern0pt}Elimination{\isacharunderscore}{\kern0pt}Modules{\isachardoublequoteclose}\isanewline
\ \ \ \ \ \ \ \ \ \ {\isachardoublequoteopen}{\isachardot}{\kern0pt}{\isachardot}{\kern0pt}{\isacharslash}{\kern0pt}Components{\isacharslash}{\kern0pt}Composites{\isacharslash}{\kern0pt}Composite{\isacharunderscore}{\kern0pt}Structures{\isachardoublequoteclose}\isanewline
\ \ \ \ \ \ \ \ \ \ {\isachardoublequoteopen}{\isachardot}{\kern0pt}{\isachardot}{\kern0pt}{\isacharslash}{\kern0pt}Components{\isacharslash}{\kern0pt}Basic{\isacharunderscore}{\kern0pt}Modules{\isacharslash}{\kern0pt}Elect{\isacharunderscore}{\kern0pt}Module{\isachardoublequoteclose}\isanewline
\isakeyword{begin}%
\endisatagtheory
{\isafoldtheory}%
%
\isadelimtheory
\isanewline
%
\endisadelimtheory
\isanewline
\isanewline
\isacommand{theorem}\isamarkupfalse%
\ cond{\isacharunderscore}{\kern0pt}winner{\isacharunderscore}{\kern0pt}imp{\isacharunderscore}{\kern0pt}max{\isacharunderscore}{\kern0pt}eval{\isacharunderscore}{\kern0pt}val{\isacharcolon}{\kern0pt}\isanewline
\ \ \isakeyword{assumes}\isanewline
\ \ \ \ rating{\isacharcolon}{\kern0pt}\ {\isachardoublequoteopen}condorcet{\isacharunderscore}{\kern0pt}rating\ e{\isachardoublequoteclose}\ \isakeyword{and}\isanewline
\ \ \ \ f{\isacharunderscore}{\kern0pt}prof{\isacharcolon}{\kern0pt}\ {\isachardoublequoteopen}finite{\isacharunderscore}{\kern0pt}profile\ A\ p{\isachardoublequoteclose}\ \isakeyword{and}\isanewline
\ \ \ \ winner{\isacharcolon}{\kern0pt}\ {\isachardoublequoteopen}condorcet{\isacharunderscore}{\kern0pt}winner\ A\ p\ w{\isachardoublequoteclose}\isanewline
\ \ \isakeyword{shows}\ {\isachardoublequoteopen}e\ w\ A\ p\ {\isacharequal}{\kern0pt}\ Max\ {\isacharbraceleft}{\kern0pt}e\ a\ A\ p\ {\isacharbar}{\kern0pt}\ a{\isachardot}{\kern0pt}\ a\ {\isasymin}\ A{\isacharbraceright}{\kern0pt}{\isachardoublequoteclose}\isanewline
%
\isadelimproof
%
\endisadelimproof
%
\isatagproof
\isacommand{proof}\isamarkupfalse%
\ {\isacharminus}{\kern0pt}\isanewline
\ \ \isanewline
\ \ \isacommand{let}\isamarkupfalse%
\ {\isacharquery}{\kern0pt}set\ {\isacharequal}{\kern0pt}\ {\isachardoublequoteopen}{\isacharbraceleft}{\kern0pt}e\ a\ A\ p\ {\isacharbar}{\kern0pt}\ a{\isachardot}{\kern0pt}\ a\ {\isasymin}\ A{\isacharbraceright}{\kern0pt}{\isachardoublequoteclose}\ \isakeyword{and}\isanewline
\ \ \ \ \ \ {\isacharquery}{\kern0pt}eMax\ {\isacharequal}{\kern0pt}\ {\isachardoublequoteopen}Max\ {\isacharbraceleft}{\kern0pt}e\ a\ A\ p\ {\isacharbar}{\kern0pt}\ a{\isachardot}{\kern0pt}\ a\ {\isasymin}\ A{\isacharbraceright}{\kern0pt}{\isachardoublequoteclose}\ \isakeyword{and}\isanewline
\ \ \ \ \ \ {\isacharquery}{\kern0pt}eW\ {\isacharequal}{\kern0pt}\ {\isachardoublequoteopen}e\ w\ A\ p{\isachardoublequoteclose}\isanewline
\ \ \isanewline
\ \ \isacommand{from}\isamarkupfalse%
\ f{\isacharunderscore}{\kern0pt}prof\ \isacommand{have}\isamarkupfalse%
\ {\isadigit{0}}{\isacharcolon}{\kern0pt}\ {\isachardoublequoteopen}finite\ {\isacharquery}{\kern0pt}set{\isachardoublequoteclose}\isanewline
\ \ \ \ \isacommand{by}\isamarkupfalse%
\ simp\isanewline
\ \ \isanewline
\ \ \isacommand{have}\isamarkupfalse%
\ {\isadigit{1}}{\isacharcolon}{\kern0pt}\ {\isachardoublequoteopen}{\isacharquery}{\kern0pt}set\ {\isasymnoteq}\ {\isacharbraceleft}{\kern0pt}{\isacharbraceright}{\kern0pt}{\isachardoublequoteclose}\isanewline
\ \ \ \ \isacommand{using}\isamarkupfalse%
\ condorcet{\isacharunderscore}{\kern0pt}winner{\isachardot}{\kern0pt}simps\ winner\isanewline
\ \ \ \ \isacommand{by}\isamarkupfalse%
\ fastforce\isanewline
\ \ \isanewline
\ \ \isacommand{have}\isamarkupfalse%
\ {\isadigit{2}}{\isacharcolon}{\kern0pt}\ {\isachardoublequoteopen}{\isacharquery}{\kern0pt}eW\ {\isasymin}\ {\isacharquery}{\kern0pt}set{\isachardoublequoteclose}\isanewline
\ \ \ \ \isacommand{using}\isamarkupfalse%
\ CollectI\ condorcet{\isacharunderscore}{\kern0pt}winner{\isachardot}{\kern0pt}simps\ winner\isanewline
\ \ \ \ \isacommand{by}\isamarkupfalse%
\ {\isacharparenleft}{\kern0pt}metis\ {\isacharparenleft}{\kern0pt}mono{\isacharunderscore}{\kern0pt}tags{\isacharcomma}{\kern0pt}\ lifting{\isacharparenright}{\kern0pt}{\isacharparenright}{\kern0pt}\isanewline
\ \ \isanewline
\ \ \isacommand{have}\isamarkupfalse%
\ {\isadigit{3}}{\isacharcolon}{\kern0pt}\ {\isachardoublequoteopen}{\isasymforall}\ e\ {\isasymin}\ {\isacharquery}{\kern0pt}set\ {\isachardot}{\kern0pt}\ e\ {\isasymle}\ {\isacharquery}{\kern0pt}eW{\isachardoublequoteclose}\isanewline
\ \ \ \ \isacommand{using}\isamarkupfalse%
\ CollectD\ condorcet{\isacharunderscore}{\kern0pt}rating{\isacharunderscore}{\kern0pt}def\ eq{\isacharunderscore}{\kern0pt}iff\isanewline
\ \ \ \ \ \ \ \ \ \ order{\isachardot}{\kern0pt}strict{\isacharunderscore}{\kern0pt}implies{\isacharunderscore}{\kern0pt}order\ rating\ winner\isanewline
\ \ \ \ \isacommand{by}\isamarkupfalse%
\ {\isacharparenleft}{\kern0pt}smt\ {\isacharparenleft}{\kern0pt}verit{\isacharcomma}{\kern0pt}\ best{\isacharparenright}{\kern0pt}{\isacharparenright}{\kern0pt}\isanewline
\ \ \isanewline
\ \ \isacommand{from}\isamarkupfalse%
\ {\isadigit{2}}\ {\isadigit{3}}\ \isacommand{have}\isamarkupfalse%
\ {\isadigit{4}}{\isacharcolon}{\kern0pt}\isanewline
\ \ \ \ {\isachardoublequoteopen}{\isacharquery}{\kern0pt}eW\ {\isasymin}\ {\isacharquery}{\kern0pt}set\ {\isasymand}\ {\isacharparenleft}{\kern0pt}{\isasymforall}a\ {\isasymin}\ {\isacharquery}{\kern0pt}set{\isachardot}{\kern0pt}\ a\ {\isasymle}\ {\isacharquery}{\kern0pt}eW{\isacharparenright}{\kern0pt}{\isachardoublequoteclose}\isanewline
\ \ \ \ \isacommand{by}\isamarkupfalse%
\ blast\isanewline
\ \ \isacommand{from}\isamarkupfalse%
\ {\isadigit{0}}\ {\isadigit{1}}\ {\isadigit{4}}\ Max{\isacharunderscore}{\kern0pt}eq{\isacharunderscore}{\kern0pt}iff\ \isacommand{show}\isamarkupfalse%
\ {\isacharquery}{\kern0pt}thesis\isanewline
\ \ \ \ \isacommand{by}\isamarkupfalse%
\ {\isacharparenleft}{\kern0pt}metis\ {\isacharparenleft}{\kern0pt}no{\isacharunderscore}{\kern0pt}types{\isacharcomma}{\kern0pt}\ lifting{\isacharparenright}{\kern0pt}{\isacharparenright}{\kern0pt}\isanewline
\isacommand{qed}\isamarkupfalse%
%
\endisatagproof
{\isafoldproof}%
%
\isadelimproof
\isanewline
%
\endisadelimproof
\isanewline
\isanewline
\isacommand{theorem}\isamarkupfalse%
\ non{\isacharunderscore}{\kern0pt}cond{\isacharunderscore}{\kern0pt}winner{\isacharunderscore}{\kern0pt}not{\isacharunderscore}{\kern0pt}max{\isacharunderscore}{\kern0pt}eval{\isacharcolon}{\kern0pt}\isanewline
\ \ \isakeyword{assumes}\isanewline
\ \ \ \ rating{\isacharcolon}{\kern0pt}\ {\isachardoublequoteopen}condorcet{\isacharunderscore}{\kern0pt}rating\ e{\isachardoublequoteclose}\ \isakeyword{and}\isanewline
\ \ \ \ f{\isacharunderscore}{\kern0pt}prof{\isacharcolon}{\kern0pt}\ {\isachardoublequoteopen}finite{\isacharunderscore}{\kern0pt}profile\ A\ p{\isachardoublequoteclose}\ \isakeyword{and}\isanewline
\ \ \ \ winner{\isacharcolon}{\kern0pt}\ {\isachardoublequoteopen}condorcet{\isacharunderscore}{\kern0pt}winner\ A\ p\ w{\isachardoublequoteclose}\ \isakeyword{and}\isanewline
\ \ \ \ linA{\isacharcolon}{\kern0pt}\ {\isachardoublequoteopen}l\ {\isasymin}\ A{\isachardoublequoteclose}\ \isakeyword{and}\isanewline
\ \ \ \ loser{\isacharcolon}{\kern0pt}\ {\isachardoublequoteopen}w\ {\isasymnoteq}\ l{\isachardoublequoteclose}\isanewline
\ \ \isakeyword{shows}\ {\isachardoublequoteopen}e\ l\ A\ p\ {\isacharless}{\kern0pt}\ Max\ {\isacharbraceleft}{\kern0pt}e\ a\ A\ p\ {\isacharbar}{\kern0pt}\ a{\isachardot}{\kern0pt}\ a\ {\isasymin}\ A{\isacharbraceright}{\kern0pt}{\isachardoublequoteclose}\isanewline
%
\isadelimproof
%
\endisadelimproof
%
\isatagproof
\isacommand{proof}\isamarkupfalse%
\ {\isacharminus}{\kern0pt}\isanewline
\ \ \isacommand{have}\isamarkupfalse%
\ {\isachardoublequoteopen}e\ l\ A\ p\ {\isacharless}{\kern0pt}\ e\ w\ A\ p{\isachardoublequoteclose}\isanewline
\ \ \ \ \isacommand{using}\isamarkupfalse%
\ condorcet{\isacharunderscore}{\kern0pt}rating{\isacharunderscore}{\kern0pt}def\ linA\ loser\ rating\ winner\isanewline
\ \ \ \ \isacommand{by}\isamarkupfalse%
\ metis\isanewline
\ \ \isacommand{also}\isamarkupfalse%
\ \isacommand{have}\isamarkupfalse%
\ {\isachardoublequoteopen}e\ w\ A\ p\ {\isacharequal}{\kern0pt}\ Max\ {\isacharbraceleft}{\kern0pt}e\ a\ A\ p\ {\isacharbar}{\kern0pt}a{\isachardot}{\kern0pt}\ a\ {\isasymin}\ A{\isacharbraceright}{\kern0pt}{\isachardoublequoteclose}\isanewline
\ \ \ \ \isacommand{using}\isamarkupfalse%
\ cond{\isacharunderscore}{\kern0pt}winner{\isacharunderscore}{\kern0pt}imp{\isacharunderscore}{\kern0pt}max{\isacharunderscore}{\kern0pt}eval{\isacharunderscore}{\kern0pt}val\ f{\isacharunderscore}{\kern0pt}prof\ rating\ winner\isanewline
\ \ \ \ \isacommand{by}\isamarkupfalse%
\ fastforce\isanewline
\ \ \isacommand{finally}\isamarkupfalse%
\ \isacommand{show}\isamarkupfalse%
\ {\isacharquery}{\kern0pt}thesis\isanewline
\ \ \ \ \isacommand{by}\isamarkupfalse%
\ simp\isanewline
\isacommand{qed}\isamarkupfalse%
%
\endisatagproof
{\isafoldproof}%
%
\isadelimproof
\isanewline
%
\endisadelimproof
\isanewline
\isanewline
\isacommand{theorem}\isamarkupfalse%
\ cr{\isacharunderscore}{\kern0pt}eval{\isacharunderscore}{\kern0pt}imp{\isacharunderscore}{\kern0pt}ccomp{\isacharunderscore}{\kern0pt}max{\isacharunderscore}{\kern0pt}elim{\isacharbrackleft}{\kern0pt}simp{\isacharbrackright}{\kern0pt}{\isacharcolon}{\kern0pt}\isanewline
\ \ \isakeyword{assumes}\isanewline
\ \ \ \ profile{\isacharcolon}{\kern0pt}\ {\isachardoublequoteopen}finite{\isacharunderscore}{\kern0pt}profile\ A\ p{\isachardoublequoteclose}\ \isakeyword{and}\isanewline
\ \ \ \ rating{\isacharcolon}{\kern0pt}\ {\isachardoublequoteopen}condorcet{\isacharunderscore}{\kern0pt}rating\ e{\isachardoublequoteclose}\isanewline
\ \ \isakeyword{shows}\isanewline
\ \ \ \ {\isachardoublequoteopen}condorcet{\isacharunderscore}{\kern0pt}compatibility\ {\isacharparenleft}{\kern0pt}max{\isacharunderscore}{\kern0pt}eliminator\ e{\isacharparenright}{\kern0pt}{\isachardoublequoteclose}\isanewline
%
\isadelimproof
\ \ %
\endisadelimproof
%
\isatagproof
\isacommand{unfolding}\isamarkupfalse%
\ condorcet{\isacharunderscore}{\kern0pt}compatibility{\isacharunderscore}{\kern0pt}def\isanewline
\isacommand{proof}\isamarkupfalse%
\ {\isacharparenleft}{\kern0pt}auto{\isacharparenright}{\kern0pt}\isanewline
\ \ \isacommand{have}\isamarkupfalse%
\ f{\isadigit{1}}{\isacharcolon}{\kern0pt}\isanewline
\ \ \ \ {\isachardoublequoteopen}{\isasymAnd}A\ p\ w\ x{\isachardot}{\kern0pt}\ condorcet{\isacharunderscore}{\kern0pt}winner\ A\ p\ w\ {\isasymLongrightarrow}\isanewline
\ \ \ \ \ \ finite\ A\ {\isasymLongrightarrow}\ w\ {\isasymin}\ A\ {\isasymLongrightarrow}\ e\ w\ A\ p\ {\isacharless}{\kern0pt}\ Max\ {\isacharbraceleft}{\kern0pt}e\ x\ A\ p\ {\isacharbar}{\kern0pt}x{\isachardot}{\kern0pt}\ x\ {\isasymin}\ A{\isacharbraceright}{\kern0pt}\ {\isasymLongrightarrow}\isanewline
\ \ \ \ \ \ \ \ x\ {\isasymin}\ A\ {\isasymLongrightarrow}\ e\ x\ A\ p\ {\isacharless}{\kern0pt}\ Max\ {\isacharbraceleft}{\kern0pt}e\ x\ A\ p\ {\isacharbar}{\kern0pt}x{\isachardot}{\kern0pt}\ x\ {\isasymin}\ A{\isacharbraceright}{\kern0pt}{\isachardoublequoteclose}\isanewline
\ \ \ \ \isacommand{using}\isamarkupfalse%
\ rating\isanewline
\ \ \ \ \isacommand{by}\isamarkupfalse%
\ {\isacharparenleft}{\kern0pt}simp\ add{\isacharcolon}{\kern0pt}\ cond{\isacharunderscore}{\kern0pt}winner{\isacharunderscore}{\kern0pt}imp{\isacharunderscore}{\kern0pt}max{\isacharunderscore}{\kern0pt}eval{\isacharunderscore}{\kern0pt}val{\isacharparenright}{\kern0pt}\isanewline
\ \ \isacommand{thus}\isamarkupfalse%
\isanewline
\ \ \ \ {\isachardoublequoteopen}{\isasymAnd}A\ p\ w\ x{\isachardot}{\kern0pt}\isanewline
\ \ \ \ \ \ profile\ A\ p\ {\isasymLongrightarrow}\ w\ {\isasymin}\ A\ {\isasymLongrightarrow}\isanewline
\ \ \ \ \ \ \ \ {\isasymforall}x{\isasymin}A\ {\isacharminus}{\kern0pt}\ {\isacharbraceleft}{\kern0pt}w{\isacharbraceright}{\kern0pt}{\isachardot}{\kern0pt}\isanewline
\ \ \ \ \ \ \ \ \ \ card\ {\isacharbraceleft}{\kern0pt}i{\isachardot}{\kern0pt}\ i\ {\isacharless}{\kern0pt}\ length\ p\ {\isasymand}\ {\isacharparenleft}{\kern0pt}w{\isacharcomma}{\kern0pt}\ x{\isacharparenright}{\kern0pt}\ {\isasymin}\ {\isacharparenleft}{\kern0pt}p{\isacharbang}{\kern0pt}i{\isacharparenright}{\kern0pt}{\isacharbraceright}{\kern0pt}\ {\isacharless}{\kern0pt}\isanewline
\ \ \ \ \ \ \ \ \ \ \ \ card\ {\isacharbraceleft}{\kern0pt}i{\isachardot}{\kern0pt}\ i\ {\isacharless}{\kern0pt}\ length\ p\ {\isasymand}\ {\isacharparenleft}{\kern0pt}x{\isacharcomma}{\kern0pt}\ w{\isacharparenright}{\kern0pt}\ {\isasymin}\ {\isacharparenleft}{\kern0pt}p{\isacharbang}{\kern0pt}i{\isacharparenright}{\kern0pt}{\isacharbraceright}{\kern0pt}\ {\isasymLongrightarrow}\isanewline
\ \ \ \ \ \ \ \ \ \ \ \ \ \ finite\ A\ {\isasymLongrightarrow}\ e\ w\ A\ p\ {\isacharless}{\kern0pt}\ Max\ {\isacharbraceleft}{\kern0pt}e\ x\ A\ p\ {\isacharbar}{\kern0pt}\ x{\isachardot}{\kern0pt}\ x\ {\isasymin}\ A{\isacharbraceright}{\kern0pt}\ {\isasymLongrightarrow}\isanewline
\ \ \ \ \ \ \ \ \ \ \ \ \ \ \ \ x\ {\isasymin}\ A\ {\isasymLongrightarrow}\ e\ x\ A\ p\ {\isacharless}{\kern0pt}\ Max\ {\isacharbraceleft}{\kern0pt}e\ x\ A\ p\ {\isacharbar}{\kern0pt}\ x{\isachardot}{\kern0pt}\ x\ {\isasymin}\ A{\isacharbraceright}{\kern0pt}{\isachardoublequoteclose}\isanewline
\ \ \ \ \isacommand{by}\isamarkupfalse%
\ simp\isanewline
\isacommand{qed}\isamarkupfalse%
%
\endisatagproof
{\isafoldproof}%
%
\isadelimproof
\isanewline
%
\endisadelimproof
\isanewline
\isanewline
\isacommand{lemma}\isamarkupfalse%
\ dcc{\isacharunderscore}{\kern0pt}imp{\isacharunderscore}{\kern0pt}cc{\isacharunderscore}{\kern0pt}elector{\isacharcolon}{\kern0pt}\isanewline
\ \ \isakeyword{assumes}\ dcc{\isacharcolon}{\kern0pt}\ {\isachardoublequoteopen}defer{\isacharunderscore}{\kern0pt}condorcet{\isacharunderscore}{\kern0pt}consistency\ m{\isachardoublequoteclose}\isanewline
\ \ \isakeyword{shows}\ {\isachardoublequoteopen}condorcet{\isacharunderscore}{\kern0pt}consistency\ {\isacharparenleft}{\kern0pt}elector\ m{\isacharparenright}{\kern0pt}{\isachardoublequoteclose}\isanewline
%
\isadelimproof
%
\endisadelimproof
%
\isatagproof
\isacommand{proof}\isamarkupfalse%
\ {\isacharparenleft}{\kern0pt}unfold\ defer{\isacharunderscore}{\kern0pt}condorcet{\isacharunderscore}{\kern0pt}consistency{\isacharunderscore}{\kern0pt}def\isanewline
\ \ \ \ \ \ \ \ \ \ \ \ \ \ condorcet{\isacharunderscore}{\kern0pt}consistency{\isacharunderscore}{\kern0pt}def{\isacharcomma}{\kern0pt}\ auto{\isacharparenright}{\kern0pt}\isanewline
\ \ \isacommand{show}\isamarkupfalse%
\ {\isachardoublequoteopen}electoral{\isacharunderscore}{\kern0pt}module\ {\isacharparenleft}{\kern0pt}m\ {\isasymtriangleright}\ elect{\isacharunderscore}{\kern0pt}module{\isacharparenright}{\kern0pt}{\isachardoublequoteclose}\isanewline
\ \ \ \ \isacommand{using}\isamarkupfalse%
\ dcc\ defer{\isacharunderscore}{\kern0pt}condorcet{\isacharunderscore}{\kern0pt}consistency{\isacharunderscore}{\kern0pt}def\isanewline
\ \ \ \ \ \ \ \ \ \ elect{\isacharunderscore}{\kern0pt}mod{\isacharunderscore}{\kern0pt}sound\ seq{\isacharunderscore}{\kern0pt}comp{\isacharunderscore}{\kern0pt}sound\isanewline
\ \ \ \ \isacommand{by}\isamarkupfalse%
\ metis\isanewline
\isacommand{next}\isamarkupfalse%
\isanewline
\ \ \isacommand{show}\isamarkupfalse%
\isanewline
\ \ \ \ {\isachardoublequoteopen}{\isasymAnd}A\ p\ w\ x{\isachardot}{\kern0pt}\isanewline
\ \ \ \ \ \ \ finite\ A\ {\isasymLongrightarrow}\ profile\ A\ p\ {\isasymLongrightarrow}\ w\ {\isasymin}\ A\ {\isasymLongrightarrow}\isanewline
\ \ \ \ \ \ \ \ \ {\isasymforall}x{\isasymin}A\ {\isacharminus}{\kern0pt}\ {\isacharbraceleft}{\kern0pt}w{\isacharbraceright}{\kern0pt}{\isachardot}{\kern0pt}\ card\ {\isacharbraceleft}{\kern0pt}i{\isachardot}{\kern0pt}\ i\ {\isacharless}{\kern0pt}\ length\ p\ {\isasymand}\ {\isacharparenleft}{\kern0pt}w{\isacharcomma}{\kern0pt}\ x{\isacharparenright}{\kern0pt}\ {\isasymin}\ {\isacharparenleft}{\kern0pt}p{\isacharbang}{\kern0pt}i{\isacharparenright}{\kern0pt}{\isacharbraceright}{\kern0pt}\ {\isacharless}{\kern0pt}\isanewline
\ \ \ \ \ \ \ \ \ \ \ \ card\ {\isacharbraceleft}{\kern0pt}i{\isachardot}{\kern0pt}\ i\ {\isacharless}{\kern0pt}\ length\ p\ {\isasymand}\ {\isacharparenleft}{\kern0pt}x{\isacharcomma}{\kern0pt}\ w{\isacharparenright}{\kern0pt}\ {\isasymin}\ {\isacharparenleft}{\kern0pt}p{\isacharbang}{\kern0pt}i{\isacharparenright}{\kern0pt}{\isacharbraceright}{\kern0pt}\ {\isasymLongrightarrow}\isanewline
\ \ \ \ \ \ \ \ x\ {\isasymin}\ elect\ m\ A\ p\ {\isasymLongrightarrow}\ x\ {\isasymin}\ A{\isachardoublequoteclose}\isanewline
\ \ \isacommand{proof}\isamarkupfalse%
\ {\isacharminus}{\kern0pt}\isanewline
\ \ \ \ \isacommand{fix}\isamarkupfalse%
\isanewline
\ \ \ \ \ \ A\ {\isacharcolon}{\kern0pt}{\isacharcolon}{\kern0pt}\ {\isachardoublequoteopen}{\isacharprime}{\kern0pt}a\ set{\isachardoublequoteclose}\ \isakeyword{and}\isanewline
\ \ \ \ \ \ p\ {\isacharcolon}{\kern0pt}{\isacharcolon}{\kern0pt}\ {\isachardoublequoteopen}{\isacharprime}{\kern0pt}a\ Profile{\isachardoublequoteclose}\ \isakeyword{and}\isanewline
\ \ \ \ \ \ w\ {\isacharcolon}{\kern0pt}{\isacharcolon}{\kern0pt}\ {\isachardoublequoteopen}{\isacharprime}{\kern0pt}a{\isachardoublequoteclose}\ \isakeyword{and}\isanewline
\ \ \ \ \ \ x\ {\isacharcolon}{\kern0pt}{\isacharcolon}{\kern0pt}\ {\isachardoublequoteopen}{\isacharprime}{\kern0pt}a{\isachardoublequoteclose}\isanewline
\ \ \ \ \isacommand{assume}\isamarkupfalse%
\isanewline
\ \ \ \ \ \ finite{\isacharcolon}{\kern0pt}\ {\isachardoublequoteopen}finite\ A{\isachardoublequoteclose}\ \isakeyword{and}\isanewline
\ \ \ \ \ \ prof{\isacharunderscore}{\kern0pt}A{\isacharcolon}{\kern0pt}\ {\isachardoublequoteopen}profile\ A\ p{\isachardoublequoteclose}\isanewline
\ \ \ \ \isacommand{show}\isamarkupfalse%
\isanewline
\ \ \ \ \ \ {\isachardoublequoteopen}{\isasymforall}y{\isasymin}A\ {\isacharminus}{\kern0pt}\ {\isacharbraceleft}{\kern0pt}w{\isacharbraceright}{\kern0pt}{\isachardot}{\kern0pt}\isanewline
\ \ \ \ \ \ \ \ \ \ card\ {\isacharbraceleft}{\kern0pt}i{\isachardot}{\kern0pt}\ i\ {\isacharless}{\kern0pt}\ length\ p\ {\isasymand}\ {\isacharparenleft}{\kern0pt}w{\isacharcomma}{\kern0pt}\ y{\isacharparenright}{\kern0pt}\ {\isasymin}\ {\isacharparenleft}{\kern0pt}p{\isacharbang}{\kern0pt}i{\isacharparenright}{\kern0pt}{\isacharbraceright}{\kern0pt}\ {\isacharless}{\kern0pt}\isanewline
\ \ \ \ \ \ \ \ \ \ \ \ card\ {\isacharbraceleft}{\kern0pt}i{\isachardot}{\kern0pt}\ i\ {\isacharless}{\kern0pt}\ length\ p\ {\isasymand}\ {\isacharparenleft}{\kern0pt}y{\isacharcomma}{\kern0pt}\ w{\isacharparenright}{\kern0pt}\ {\isasymin}\ {\isacharparenleft}{\kern0pt}p{\isacharbang}{\kern0pt}i{\isacharparenright}{\kern0pt}{\isacharbraceright}{\kern0pt}\ {\isasymLongrightarrow}\isanewline
\ \ \ \ \ \ \ \ \ \ \ \ \ x\ {\isasymin}\ elect\ m\ A\ p\ {\isasymLongrightarrow}\ x\ {\isasymin}\ A{\isachardoublequoteclose}\isanewline
\ \ \ \ \ \ \isacommand{using}\isamarkupfalse%
\ dcc\ defer{\isacharunderscore}{\kern0pt}condorcet{\isacharunderscore}{\kern0pt}consistency{\isacharunderscore}{\kern0pt}def\isanewline
\ \ \ \ \ \ \ \ \ \ \ \ elect{\isacharunderscore}{\kern0pt}in{\isacharunderscore}{\kern0pt}alts\ subset{\isacharunderscore}{\kern0pt}eq\ finite\ prof{\isacharunderscore}{\kern0pt}A\isanewline
\ \ \ \ \ \ \isacommand{by}\isamarkupfalse%
\ metis\isanewline
\ \ \isacommand{qed}\isamarkupfalse%
\isanewline
\isacommand{next}\isamarkupfalse%
\isanewline
\ \ \isacommand{fix}\isamarkupfalse%
\isanewline
\ \ \ \ A\ {\isacharcolon}{\kern0pt}{\isacharcolon}{\kern0pt}\ {\isachardoublequoteopen}{\isacharprime}{\kern0pt}a\ set{\isachardoublequoteclose}\ \isakeyword{and}\isanewline
\ \ \ \ p\ {\isacharcolon}{\kern0pt}{\isacharcolon}{\kern0pt}\ {\isachardoublequoteopen}{\isacharprime}{\kern0pt}a\ Profile{\isachardoublequoteclose}\ \isakeyword{and}\isanewline
\ \ \ \ w\ {\isacharcolon}{\kern0pt}{\isacharcolon}{\kern0pt}\ {\isachardoublequoteopen}{\isacharprime}{\kern0pt}a{\isachardoublequoteclose}\ \isakeyword{and}\isanewline
\ \ \ \ x\ {\isacharcolon}{\kern0pt}{\isacharcolon}{\kern0pt}\ {\isachardoublequoteopen}{\isacharprime}{\kern0pt}a{\isachardoublequoteclose}\ \isakeyword{and}\isanewline
\ \ \ \ xa\ {\isacharcolon}{\kern0pt}{\isacharcolon}{\kern0pt}\ {\isachardoublequoteopen}{\isacharprime}{\kern0pt}a{\isachardoublequoteclose}\isanewline
\ \ \isacommand{assume}\isamarkupfalse%
\isanewline
\ \ \ \ finite{\isacharcolon}{\kern0pt}\ {\isachardoublequoteopen}finite\ A{\isachardoublequoteclose}\ \isakeyword{and}\isanewline
\ \ \ \ prof{\isacharunderscore}{\kern0pt}A{\isacharcolon}{\kern0pt}\ {\isachardoublequoteopen}profile\ A\ p{\isachardoublequoteclose}\ \isakeyword{and}\isanewline
\ \ \ \ w{\isacharunderscore}{\kern0pt}in{\isacharunderscore}{\kern0pt}A{\isacharcolon}{\kern0pt}\ {\isachardoublequoteopen}w\ {\isasymin}\ A{\isachardoublequoteclose}\ \isakeyword{and}\isanewline
\ \ \ \ {\isadigit{1}}{\isacharcolon}{\kern0pt}\ {\isachardoublequoteopen}x\ {\isasymin}\ elect\ m\ A\ p{\isachardoublequoteclose}\ \isakeyword{and}\isanewline
\ \ \ \ {\isadigit{2}}{\isacharcolon}{\kern0pt}\ {\isachardoublequoteopen}{\isasymforall}y{\isasymin}A\ {\isacharminus}{\kern0pt}\ {\isacharbraceleft}{\kern0pt}w{\isacharbraceright}{\kern0pt}{\isachardot}{\kern0pt}\isanewline
\ \ \ \ \ \ \ \ \ \ card\ {\isacharbraceleft}{\kern0pt}i{\isachardot}{\kern0pt}\ i\ {\isacharless}{\kern0pt}\ length\ p\ {\isasymand}\ {\isacharparenleft}{\kern0pt}w{\isacharcomma}{\kern0pt}\ y{\isacharparenright}{\kern0pt}\ {\isasymin}\ {\isacharparenleft}{\kern0pt}p{\isacharbang}{\kern0pt}i{\isacharparenright}{\kern0pt}{\isacharbraceright}{\kern0pt}\ {\isacharless}{\kern0pt}\isanewline
\ \ \ \ \ \ \ \ \ \ \ \ card\ {\isacharbraceleft}{\kern0pt}i{\isachardot}{\kern0pt}\ i\ {\isacharless}{\kern0pt}\ length\ p\ {\isasymand}\ {\isacharparenleft}{\kern0pt}y{\isacharcomma}{\kern0pt}\ w{\isacharparenright}{\kern0pt}\ {\isasymin}\ {\isacharparenleft}{\kern0pt}p{\isacharbang}{\kern0pt}i{\isacharparenright}{\kern0pt}{\isacharbraceright}{\kern0pt}{\isachardoublequoteclose}\isanewline
\ \ \isacommand{have}\isamarkupfalse%
\ {\isachardoublequoteopen}condorcet{\isacharunderscore}{\kern0pt}winner\ A\ p\ w{\isachardoublequoteclose}\isanewline
\ \ \ \ \isacommand{using}\isamarkupfalse%
\ finite\ prof{\isacharunderscore}{\kern0pt}A\ w{\isacharunderscore}{\kern0pt}in{\isacharunderscore}{\kern0pt}A\ {\isachardoublequoteopen}{\isadigit{2}}{\isachardoublequoteclose}\isanewline
\ \ \ \ \isacommand{by}\isamarkupfalse%
\ simp\isanewline
\ \ \isacommand{thus}\isamarkupfalse%
\ {\isachardoublequoteopen}xa\ {\isacharequal}{\kern0pt}\ x{\isachardoublequoteclose}\isanewline
\ \ \ \ \isacommand{using}\isamarkupfalse%
\ condorcet{\isacharunderscore}{\kern0pt}winner{\isachardot}{\kern0pt}simps\ dcc\ fst{\isacharunderscore}{\kern0pt}conv\ insert{\isacharunderscore}{\kern0pt}Diff\ {\isachardoublequoteopen}{\isadigit{1}}{\isachardoublequoteclose}\isanewline
\ \ \ \ \ \ \ \ \ \ defer{\isacharunderscore}{\kern0pt}condorcet{\isacharunderscore}{\kern0pt}consistency{\isacharunderscore}{\kern0pt}def\ insert{\isacharunderscore}{\kern0pt}not{\isacharunderscore}{\kern0pt}empty\isanewline
\ \ \ \ \isacommand{by}\isamarkupfalse%
\ {\isacharparenleft}{\kern0pt}metis\ {\isacharparenleft}{\kern0pt}no{\isacharunderscore}{\kern0pt}types{\isacharcomma}{\kern0pt}\ lifting{\isacharparenright}{\kern0pt}{\isacharparenright}{\kern0pt}\isanewline
\isacommand{next}\isamarkupfalse%
\isanewline
\ \ \isacommand{fix}\isamarkupfalse%
\isanewline
\ \ \ \ A\ {\isacharcolon}{\kern0pt}{\isacharcolon}{\kern0pt}\ {\isachardoublequoteopen}{\isacharprime}{\kern0pt}a\ set{\isachardoublequoteclose}\ \isakeyword{and}\isanewline
\ \ \ \ p\ {\isacharcolon}{\kern0pt}{\isacharcolon}{\kern0pt}\ {\isachardoublequoteopen}{\isacharprime}{\kern0pt}a\ Profile{\isachardoublequoteclose}\ \isakeyword{and}\isanewline
\ \ \ \ w\ {\isacharcolon}{\kern0pt}{\isacharcolon}{\kern0pt}\ {\isachardoublequoteopen}{\isacharprime}{\kern0pt}a{\isachardoublequoteclose}\ \isakeyword{and}\isanewline
\ \ \ \ x\ {\isacharcolon}{\kern0pt}{\isacharcolon}{\kern0pt}\ {\isachardoublequoteopen}{\isacharprime}{\kern0pt}a{\isachardoublequoteclose}\isanewline
\ \ \isacommand{assume}\isamarkupfalse%
\isanewline
\ \ \ \ finite{\isacharcolon}{\kern0pt}\ {\isachardoublequoteopen}finite\ A{\isachardoublequoteclose}\ \isakeyword{and}\isanewline
\ \ \ \ prof{\isacharunderscore}{\kern0pt}A{\isacharcolon}{\kern0pt}\ {\isachardoublequoteopen}profile\ A\ p{\isachardoublequoteclose}\ \isakeyword{and}\isanewline
\ \ \ \ w{\isacharunderscore}{\kern0pt}in{\isacharunderscore}{\kern0pt}A{\isacharcolon}{\kern0pt}\ {\isachardoublequoteopen}w\ {\isasymin}\ A{\isachardoublequoteclose}\ \isakeyword{and}\isanewline
\ \ \ \ {\isadigit{0}}{\isacharcolon}{\kern0pt}\ {\isachardoublequoteopen}{\isasymforall}y{\isasymin}A\ {\isacharminus}{\kern0pt}\ {\isacharbraceleft}{\kern0pt}w{\isacharbraceright}{\kern0pt}{\isachardot}{\kern0pt}\isanewline
\ \ \ \ \ \ \ \ \ \ card\ {\isacharbraceleft}{\kern0pt}i{\isachardot}{\kern0pt}\ i\ {\isacharless}{\kern0pt}\ length\ p\ {\isasymand}\ {\isacharparenleft}{\kern0pt}w{\isacharcomma}{\kern0pt}\ y{\isacharparenright}{\kern0pt}\ {\isasymin}\ {\isacharparenleft}{\kern0pt}p{\isacharbang}{\kern0pt}i{\isacharparenright}{\kern0pt}{\isacharbraceright}{\kern0pt}\ {\isacharless}{\kern0pt}\isanewline
\ \ \ \ \ \ \ \ \ \ \ \ card\ {\isacharbraceleft}{\kern0pt}i{\isachardot}{\kern0pt}\ i\ {\isacharless}{\kern0pt}\ length\ p\ {\isasymand}\ {\isacharparenleft}{\kern0pt}y{\isacharcomma}{\kern0pt}\ w{\isacharparenright}{\kern0pt}\ {\isasymin}\ {\isacharparenleft}{\kern0pt}p{\isacharbang}{\kern0pt}i{\isacharparenright}{\kern0pt}{\isacharbraceright}{\kern0pt}{\isachardoublequoteclose}\ \isakeyword{and}\isanewline
\ \ \ \ {\isadigit{1}}{\isacharcolon}{\kern0pt}\ {\isachardoublequoteopen}x\ {\isasymin}\ defer\ m\ A\ p{\isachardoublequoteclose}\isanewline
\ \ \isacommand{have}\isamarkupfalse%
\ {\isachardoublequoteopen}condorcet{\isacharunderscore}{\kern0pt}winner\ A\ p\ w{\isachardoublequoteclose}\isanewline
\ \ \ \ \isacommand{using}\isamarkupfalse%
\ finite\ prof{\isacharunderscore}{\kern0pt}A\ w{\isacharunderscore}{\kern0pt}in{\isacharunderscore}{\kern0pt}A\ {\isachardoublequoteopen}{\isadigit{0}}{\isachardoublequoteclose}\isanewline
\ \ \ \ \isacommand{by}\isamarkupfalse%
\ simp\isanewline
\ \ \isacommand{thus}\isamarkupfalse%
\ {\isachardoublequoteopen}x\ {\isasymin}\ A{\isachardoublequoteclose}\isanewline
\ \ \ \ \isacommand{using}\isamarkupfalse%
\ {\isachardoublequoteopen}{\isadigit{0}}{\isachardoublequoteclose}\ {\isachardoublequoteopen}{\isadigit{1}}{\isachardoublequoteclose}\ condorcet{\isacharunderscore}{\kern0pt}winner{\isachardot}{\kern0pt}simps\ dcc\ defer{\isacharunderscore}{\kern0pt}in{\isacharunderscore}{\kern0pt}alts\isanewline
\ \ \ \ \ \ \ \ \ \ defer{\isacharunderscore}{\kern0pt}condorcet{\isacharunderscore}{\kern0pt}consistency{\isacharunderscore}{\kern0pt}def\ order{\isacharunderscore}{\kern0pt}trans\isanewline
\ \ \ \ \ \ \ \ \ \ subset{\isacharunderscore}{\kern0pt}Compl{\isacharunderscore}{\kern0pt}singleton\isanewline
\ \ \ \ \isacommand{by}\isamarkupfalse%
\ {\isacharparenleft}{\kern0pt}metis\ {\isacharparenleft}{\kern0pt}no{\isacharunderscore}{\kern0pt}types{\isacharcomma}{\kern0pt}\ lifting{\isacharparenright}{\kern0pt}{\isacharparenright}{\kern0pt}\isanewline
\isacommand{next}\isamarkupfalse%
\isanewline
\ \ \isacommand{fix}\isamarkupfalse%
\isanewline
\ \ \ \ A\ {\isacharcolon}{\kern0pt}{\isacharcolon}{\kern0pt}\ {\isachardoublequoteopen}{\isacharprime}{\kern0pt}a\ set{\isachardoublequoteclose}\ \isakeyword{and}\isanewline
\ \ \ \ p\ {\isacharcolon}{\kern0pt}{\isacharcolon}{\kern0pt}\ {\isachardoublequoteopen}{\isacharprime}{\kern0pt}a\ Profile{\isachardoublequoteclose}\ \isakeyword{and}\isanewline
\ \ \ \ w\ {\isacharcolon}{\kern0pt}{\isacharcolon}{\kern0pt}\ {\isachardoublequoteopen}{\isacharprime}{\kern0pt}a{\isachardoublequoteclose}\ \isakeyword{and}\isanewline
\ \ \ \ x\ {\isacharcolon}{\kern0pt}{\isacharcolon}{\kern0pt}\ {\isachardoublequoteopen}{\isacharprime}{\kern0pt}a{\isachardoublequoteclose}\ \isakeyword{and}\isanewline
\ \ \ \ xa\ {\isacharcolon}{\kern0pt}{\isacharcolon}{\kern0pt}\ {\isachardoublequoteopen}{\isacharprime}{\kern0pt}a{\isachardoublequoteclose}\isanewline
\ \ \isacommand{assume}\isamarkupfalse%
\isanewline
\ \ \ \ finite{\isacharcolon}{\kern0pt}\ {\isachardoublequoteopen}finite\ A{\isachardoublequoteclose}\ \isakeyword{and}\isanewline
\ \ \ \ prof{\isacharunderscore}{\kern0pt}A{\isacharcolon}{\kern0pt}\ {\isachardoublequoteopen}profile\ A\ p{\isachardoublequoteclose}\ \isakeyword{and}\isanewline
\ \ \ \ w{\isacharunderscore}{\kern0pt}in{\isacharunderscore}{\kern0pt}A{\isacharcolon}{\kern0pt}\ {\isachardoublequoteopen}w\ {\isasymin}\ A{\isachardoublequoteclose}\ \isakeyword{and}\isanewline
\ \ \ \ {\isadigit{1}}{\isacharcolon}{\kern0pt}\ {\isachardoublequoteopen}x\ {\isasymin}\ defer\ m\ A\ p{\isachardoublequoteclose}\ \isakeyword{and}\isanewline
\ \ \ \ xa{\isacharunderscore}{\kern0pt}in{\isacharunderscore}{\kern0pt}A{\isacharcolon}{\kern0pt}\ {\isachardoublequoteopen}xa\ {\isasymin}\ A{\isachardoublequoteclose}\ \isakeyword{and}\isanewline
\ \ \ \ {\isadigit{2}}{\isacharcolon}{\kern0pt}\ {\isachardoublequoteopen}{\isasymforall}y{\isasymin}A\ {\isacharminus}{\kern0pt}\ {\isacharbraceleft}{\kern0pt}w{\isacharbraceright}{\kern0pt}{\isachardot}{\kern0pt}\isanewline
\ \ \ \ \ \ \ \ \ \ card\ {\isacharbraceleft}{\kern0pt}i{\isachardot}{\kern0pt}\ i\ {\isacharless}{\kern0pt}\ length\ p\ {\isasymand}\ {\isacharparenleft}{\kern0pt}w{\isacharcomma}{\kern0pt}\ y{\isacharparenright}{\kern0pt}\ {\isasymin}\ {\isacharparenleft}{\kern0pt}p{\isacharbang}{\kern0pt}i{\isacharparenright}{\kern0pt}{\isacharbraceright}{\kern0pt}\ {\isacharless}{\kern0pt}\isanewline
\ \ \ \ \ \ \ \ \ \ \ \ card\ {\isacharbraceleft}{\kern0pt}i{\isachardot}{\kern0pt}\ i\ {\isacharless}{\kern0pt}\ length\ p\ {\isasymand}\ {\isacharparenleft}{\kern0pt}y{\isacharcomma}{\kern0pt}\ w{\isacharparenright}{\kern0pt}\ {\isasymin}\ {\isacharparenleft}{\kern0pt}p{\isacharbang}{\kern0pt}i{\isacharparenright}{\kern0pt}{\isacharbraceright}{\kern0pt}{\isachardoublequoteclose}\ \isakeyword{and}\isanewline
\ \ \ \ {\isadigit{3}}{\isacharcolon}{\kern0pt}\ {\isachardoublequoteopen}{\isasymnot}\ card\ {\isacharbraceleft}{\kern0pt}i{\isachardot}{\kern0pt}\ i\ {\isacharless}{\kern0pt}\ length\ p\ {\isasymand}\ {\isacharparenleft}{\kern0pt}x{\isacharcomma}{\kern0pt}\ xa{\isacharparenright}{\kern0pt}\ {\isasymin}\ {\isacharparenleft}{\kern0pt}p{\isacharbang}{\kern0pt}i{\isacharparenright}{\kern0pt}{\isacharbraceright}{\kern0pt}\ {\isacharless}{\kern0pt}\isanewline
\ \ \ \ \ \ \ \ \ \ \ \ card\ {\isacharbraceleft}{\kern0pt}i{\isachardot}{\kern0pt}\ i\ {\isacharless}{\kern0pt}\ length\ p\ {\isasymand}\ {\isacharparenleft}{\kern0pt}xa{\isacharcomma}{\kern0pt}\ x{\isacharparenright}{\kern0pt}\ {\isasymin}\ {\isacharparenleft}{\kern0pt}p{\isacharbang}{\kern0pt}i{\isacharparenright}{\kern0pt}{\isacharbraceright}{\kern0pt}{\isachardoublequoteclose}\isanewline
\ \ \isacommand{have}\isamarkupfalse%
\ {\isachardoublequoteopen}condorcet{\isacharunderscore}{\kern0pt}winner\ A\ p\ w{\isachardoublequoteclose}\isanewline
\ \ \ \ \isacommand{using}\isamarkupfalse%
\ finite\ prof{\isacharunderscore}{\kern0pt}A\ w{\isacharunderscore}{\kern0pt}in{\isacharunderscore}{\kern0pt}A\ {\isachardoublequoteopen}{\isadigit{2}}{\isachardoublequoteclose}\isanewline
\ \ \ \ \isacommand{by}\isamarkupfalse%
\ simp\isanewline
\ \ \isacommand{thus}\isamarkupfalse%
\ {\isachardoublequoteopen}xa\ {\isacharequal}{\kern0pt}\ x{\isachardoublequoteclose}\isanewline
\ \ \ \ \isacommand{using}\isamarkupfalse%
\ {\isachardoublequoteopen}{\isadigit{1}}{\isachardoublequoteclose}\ {\isachardoublequoteopen}{\isadigit{2}}{\isachardoublequoteclose}\ condorcet{\isacharunderscore}{\kern0pt}winner{\isachardot}{\kern0pt}simps\ dcc\ empty{\isacharunderscore}{\kern0pt}iff\ xa{\isacharunderscore}{\kern0pt}in{\isacharunderscore}{\kern0pt}A\isanewline
\ \ \ \ \ \ \ \ \ \ defer{\isacharunderscore}{\kern0pt}condorcet{\isacharunderscore}{\kern0pt}consistency{\isacharunderscore}{\kern0pt}def\ {\isachardoublequoteopen}{\isadigit{3}}{\isachardoublequoteclose}\ DiffI\isanewline
\ \ \ \ \ \ \ \ \ \ cond{\isacharunderscore}{\kern0pt}winner{\isacharunderscore}{\kern0pt}unique{\isadigit{3}}\ insert{\isacharunderscore}{\kern0pt}iff\ prod{\isachardot}{\kern0pt}sel{\isacharparenleft}{\kern0pt}{\isadigit{2}}{\isacharparenright}{\kern0pt}\isanewline
\ \ \ \ \isacommand{by}\isamarkupfalse%
\ {\isacharparenleft}{\kern0pt}metis\ {\isacharparenleft}{\kern0pt}no{\isacharunderscore}{\kern0pt}types{\isacharcomma}{\kern0pt}\ lifting{\isacharparenright}{\kern0pt}{\isacharparenright}{\kern0pt}\isanewline
\isacommand{next}\isamarkupfalse%
\isanewline
\ \ \isacommand{fix}\isamarkupfalse%
\isanewline
\ \ \ \ A\ {\isacharcolon}{\kern0pt}{\isacharcolon}{\kern0pt}\ {\isachardoublequoteopen}{\isacharprime}{\kern0pt}a\ set{\isachardoublequoteclose}\ \isakeyword{and}\isanewline
\ \ \ \ p\ {\isacharcolon}{\kern0pt}{\isacharcolon}{\kern0pt}\ {\isachardoublequoteopen}{\isacharprime}{\kern0pt}a\ Profile{\isachardoublequoteclose}\ \isakeyword{and}\isanewline
\ \ \ \ w\ {\isacharcolon}{\kern0pt}{\isacharcolon}{\kern0pt}\ {\isachardoublequoteopen}{\isacharprime}{\kern0pt}a{\isachardoublequoteclose}\ \isakeyword{and}\isanewline
\ \ \ \ x\ {\isacharcolon}{\kern0pt}{\isacharcolon}{\kern0pt}\ {\isachardoublequoteopen}{\isacharprime}{\kern0pt}a{\isachardoublequoteclose}\isanewline
\ \ \isacommand{assume}\isamarkupfalse%
\isanewline
\ \ \ \ finite{\isacharcolon}{\kern0pt}\ {\isachardoublequoteopen}finite\ A{\isachardoublequoteclose}\ \isakeyword{and}\isanewline
\ \ \ \ prof{\isacharunderscore}{\kern0pt}A{\isacharcolon}{\kern0pt}\ {\isachardoublequoteopen}profile\ A\ p{\isachardoublequoteclose}\ \isakeyword{and}\isanewline
\ \ \ \ w{\isacharunderscore}{\kern0pt}in{\isacharunderscore}{\kern0pt}A{\isacharcolon}{\kern0pt}\ {\isachardoublequoteopen}w\ {\isasymin}\ A{\isachardoublequoteclose}\ \isakeyword{and}\isanewline
\ \ \ \ x{\isacharunderscore}{\kern0pt}in{\isacharunderscore}{\kern0pt}A{\isacharcolon}{\kern0pt}\ {\isachardoublequoteopen}x\ {\isasymin}\ A{\isachardoublequoteclose}\ \isakeyword{and}\isanewline
\ \ \ \ {\isadigit{1}}{\isacharcolon}{\kern0pt}\ {\isachardoublequoteopen}x\ {\isasymnotin}\ defer\ m\ A\ p{\isachardoublequoteclose}\ \isakeyword{and}\isanewline
\ \ \ \ {\isadigit{2}}{\isacharcolon}{\kern0pt}\ {\isachardoublequoteopen}{\isasymforall}y{\isasymin}A\ {\isacharminus}{\kern0pt}\ {\isacharbraceleft}{\kern0pt}w{\isacharbraceright}{\kern0pt}{\isachardot}{\kern0pt}\isanewline
\ \ \ \ \ \ \ \ \ \ card\ {\isacharbraceleft}{\kern0pt}i{\isachardot}{\kern0pt}\ i\ {\isacharless}{\kern0pt}\ length\ p\ {\isasymand}\ {\isacharparenleft}{\kern0pt}w{\isacharcomma}{\kern0pt}\ y{\isacharparenright}{\kern0pt}\ {\isasymin}\ {\isacharparenleft}{\kern0pt}p{\isacharbang}{\kern0pt}i{\isacharparenright}{\kern0pt}{\isacharbraceright}{\kern0pt}\ {\isacharless}{\kern0pt}\isanewline
\ \ \ \ \ \ \ \ \ \ \ \ card\ {\isacharbraceleft}{\kern0pt}i{\isachardot}{\kern0pt}\ i\ {\isacharless}{\kern0pt}\ length\ p\ {\isasymand}\ {\isacharparenleft}{\kern0pt}y{\isacharcomma}{\kern0pt}\ w{\isacharparenright}{\kern0pt}\ {\isasymin}\ {\isacharparenleft}{\kern0pt}p{\isacharbang}{\kern0pt}i{\isacharparenright}{\kern0pt}{\isacharbraceright}{\kern0pt}{\isachardoublequoteclose}\ \isakeyword{and}\isanewline
\ \ \ \ {\isadigit{3}}{\isacharcolon}{\kern0pt}\ {\isachardoublequoteopen}{\isasymforall}y{\isasymin}A\ {\isacharminus}{\kern0pt}\ {\isacharbraceleft}{\kern0pt}x{\isacharbraceright}{\kern0pt}{\isachardot}{\kern0pt}\isanewline
\ \ \ \ \ \ \ \ \ \ card\ {\isacharbraceleft}{\kern0pt}i{\isachardot}{\kern0pt}\ i\ {\isacharless}{\kern0pt}\ length\ p\ {\isasymand}\ {\isacharparenleft}{\kern0pt}x{\isacharcomma}{\kern0pt}\ y{\isacharparenright}{\kern0pt}\ {\isasymin}\ {\isacharparenleft}{\kern0pt}p{\isacharbang}{\kern0pt}i{\isacharparenright}{\kern0pt}{\isacharbraceright}{\kern0pt}\ {\isacharless}{\kern0pt}\isanewline
\ \ \ \ \ \ \ \ \ \ \ \ card\ {\isacharbraceleft}{\kern0pt}i{\isachardot}{\kern0pt}\ i\ {\isacharless}{\kern0pt}\ length\ p\ {\isasymand}\ {\isacharparenleft}{\kern0pt}y{\isacharcomma}{\kern0pt}\ x{\isacharparenright}{\kern0pt}\ {\isasymin}\ {\isacharparenleft}{\kern0pt}p{\isacharbang}{\kern0pt}i{\isacharparenright}{\kern0pt}{\isacharbraceright}{\kern0pt}{\isachardoublequoteclose}\isanewline
\ \ \isacommand{have}\isamarkupfalse%
\ {\isachardoublequoteopen}condorcet{\isacharunderscore}{\kern0pt}winner\ A\ p\ w{\isachardoublequoteclose}\isanewline
\ \ \ \ \isacommand{using}\isamarkupfalse%
\ finite\ prof{\isacharunderscore}{\kern0pt}A\ w{\isacharunderscore}{\kern0pt}in{\isacharunderscore}{\kern0pt}A\ {\isachardoublequoteopen}{\isadigit{2}}{\isachardoublequoteclose}\isanewline
\ \ \ \ \isacommand{by}\isamarkupfalse%
\ simp\isanewline
\ \ \isacommand{also}\isamarkupfalse%
\ \isacommand{have}\isamarkupfalse%
\ {\isachardoublequoteopen}condorcet{\isacharunderscore}{\kern0pt}winner\ A\ p\ x{\isachardoublequoteclose}\isanewline
\ \ \ \ \isacommand{using}\isamarkupfalse%
\ finite\ prof{\isacharunderscore}{\kern0pt}A\ x{\isacharunderscore}{\kern0pt}in{\isacharunderscore}{\kern0pt}A\ {\isachardoublequoteopen}{\isadigit{3}}{\isachardoublequoteclose}\isanewline
\ \ \ \ \isacommand{by}\isamarkupfalse%
\ simp\isanewline
\ \ \isacommand{ultimately}\isamarkupfalse%
\ \isacommand{show}\isamarkupfalse%
\ {\isachardoublequoteopen}x\ {\isasymin}\ elect\ m\ A\ p{\isachardoublequoteclose}\isanewline
\ \ \ \ \isacommand{using}\isamarkupfalse%
\ {\isachardoublequoteopen}{\isadigit{1}}{\isachardoublequoteclose}\ condorcet{\isacharunderscore}{\kern0pt}winner{\isachardot}{\kern0pt}simps\ dcc\isanewline
\ \ \ \ \ \ \ \ \ \ defer{\isacharunderscore}{\kern0pt}condorcet{\isacharunderscore}{\kern0pt}consistency{\isacharunderscore}{\kern0pt}def\isanewline
\ \ \ \ \ \ \ \ \ \ cond{\isacharunderscore}{\kern0pt}winner{\isacharunderscore}{\kern0pt}unique{\isadigit{3}}\ insert{\isacharunderscore}{\kern0pt}iff\ eq{\isacharunderscore}{\kern0pt}snd{\isacharunderscore}{\kern0pt}iff\isanewline
\ \ \ \ \isacommand{by}\isamarkupfalse%
\ {\isacharparenleft}{\kern0pt}metis\ {\isacharparenleft}{\kern0pt}no{\isacharunderscore}{\kern0pt}types{\isacharcomma}{\kern0pt}\ lifting{\isacharparenright}{\kern0pt}{\isacharparenright}{\kern0pt}\isanewline
\isacommand{next}\isamarkupfalse%
\isanewline
\ \ \isacommand{fix}\isamarkupfalse%
\isanewline
\ \ \ \ A\ {\isacharcolon}{\kern0pt}{\isacharcolon}{\kern0pt}\ {\isachardoublequoteopen}{\isacharprime}{\kern0pt}a\ set{\isachardoublequoteclose}\ \isakeyword{and}\isanewline
\ \ \ \ p\ {\isacharcolon}{\kern0pt}{\isacharcolon}{\kern0pt}\ {\isachardoublequoteopen}{\isacharprime}{\kern0pt}a\ Profile{\isachardoublequoteclose}\ \isakeyword{and}\isanewline
\ \ \ \ w\ {\isacharcolon}{\kern0pt}{\isacharcolon}{\kern0pt}\ {\isachardoublequoteopen}{\isacharprime}{\kern0pt}a{\isachardoublequoteclose}\ \isakeyword{and}\isanewline
\ \ \ \ x\ {\isacharcolon}{\kern0pt}{\isacharcolon}{\kern0pt}\ {\isachardoublequoteopen}{\isacharprime}{\kern0pt}a{\isachardoublequoteclose}\isanewline
\ \ \isacommand{assume}\isamarkupfalse%
\isanewline
\ \ \ \ finite{\isacharcolon}{\kern0pt}\ {\isachardoublequoteopen}finite\ A{\isachardoublequoteclose}\ \isakeyword{and}\isanewline
\ \ \ \ prof{\isacharunderscore}{\kern0pt}A{\isacharcolon}{\kern0pt}\ {\isachardoublequoteopen}profile\ A\ p{\isachardoublequoteclose}\ \isakeyword{and}\isanewline
\ \ \ \ w{\isacharunderscore}{\kern0pt}in{\isacharunderscore}{\kern0pt}A{\isacharcolon}{\kern0pt}\ {\isachardoublequoteopen}w\ {\isasymin}\ A{\isachardoublequoteclose}\ \isakeyword{and}\isanewline
\ \ \ \ {\isadigit{1}}{\isacharcolon}{\kern0pt}\ {\isachardoublequoteopen}x\ {\isasymin}\ reject\ m\ A\ p{\isachardoublequoteclose}\ \isakeyword{and}\isanewline
\ \ \ \ {\isadigit{2}}{\isacharcolon}{\kern0pt}\ {\isachardoublequoteopen}{\isasymforall}y{\isasymin}A\ {\isacharminus}{\kern0pt}\ {\isacharbraceleft}{\kern0pt}w{\isacharbraceright}{\kern0pt}{\isachardot}{\kern0pt}\isanewline
\ \ \ \ \ \ \ \ \ \ card\ {\isacharbraceleft}{\kern0pt}i{\isachardot}{\kern0pt}\ i\ {\isacharless}{\kern0pt}\ length\ p\ {\isasymand}\ {\isacharparenleft}{\kern0pt}w{\isacharcomma}{\kern0pt}\ y{\isacharparenright}{\kern0pt}\ {\isasymin}\ {\isacharparenleft}{\kern0pt}p{\isacharbang}{\kern0pt}i{\isacharparenright}{\kern0pt}{\isacharbraceright}{\kern0pt}\ {\isacharless}{\kern0pt}\isanewline
\ \ \ \ \ \ \ \ \ \ \ \ card\ {\isacharbraceleft}{\kern0pt}i{\isachardot}{\kern0pt}\ i\ {\isacharless}{\kern0pt}\ length\ p\ {\isasymand}\ {\isacharparenleft}{\kern0pt}y{\isacharcomma}{\kern0pt}\ w{\isacharparenright}{\kern0pt}\ {\isasymin}\ {\isacharparenleft}{\kern0pt}p{\isacharbang}{\kern0pt}i{\isacharparenright}{\kern0pt}{\isacharbraceright}{\kern0pt}{\isachardoublequoteclose}\isanewline
\ \ \isacommand{have}\isamarkupfalse%
\ {\isachardoublequoteopen}condorcet{\isacharunderscore}{\kern0pt}winner\ A\ p\ w{\isachardoublequoteclose}\isanewline
\ \ \ \ \isacommand{using}\isamarkupfalse%
\ finite\ prof{\isacharunderscore}{\kern0pt}A\ w{\isacharunderscore}{\kern0pt}in{\isacharunderscore}{\kern0pt}A\ {\isachardoublequoteopen}{\isadigit{2}}{\isachardoublequoteclose}\isanewline
\ \ \ \ \isacommand{by}\isamarkupfalse%
\ simp\isanewline
\ \ \isacommand{thus}\isamarkupfalse%
\ {\isachardoublequoteopen}x\ {\isasymin}\ A{\isachardoublequoteclose}\isanewline
\ \ \ \ \isacommand{using}\isamarkupfalse%
\ {\isachardoublequoteopen}{\isadigit{1}}{\isachardoublequoteclose}\ dcc\ defer{\isacharunderscore}{\kern0pt}condorcet{\isacharunderscore}{\kern0pt}consistency{\isacharunderscore}{\kern0pt}def\ finite\isanewline
\ \ \ \ \ \ \ \ \ \ prof{\isacharunderscore}{\kern0pt}A\ reject{\isacharunderscore}{\kern0pt}in{\isacharunderscore}{\kern0pt}alts\ subsetD\isanewline
\ \ \ \ \isacommand{by}\isamarkupfalse%
\ metis\isanewline
\isacommand{next}\isamarkupfalse%
\isanewline
\ \ \isacommand{fix}\isamarkupfalse%
\isanewline
\ \ \ \ A\ {\isacharcolon}{\kern0pt}{\isacharcolon}{\kern0pt}\ {\isachardoublequoteopen}{\isacharprime}{\kern0pt}a\ set{\isachardoublequoteclose}\ \isakeyword{and}\isanewline
\ \ \ \ p\ {\isacharcolon}{\kern0pt}{\isacharcolon}{\kern0pt}\ {\isachardoublequoteopen}{\isacharprime}{\kern0pt}a\ Profile{\isachardoublequoteclose}\ \isakeyword{and}\isanewline
\ \ \ \ w\ {\isacharcolon}{\kern0pt}{\isacharcolon}{\kern0pt}\ {\isachardoublequoteopen}{\isacharprime}{\kern0pt}a{\isachardoublequoteclose}\ \isakeyword{and}\isanewline
\ \ \ \ x\ {\isacharcolon}{\kern0pt}{\isacharcolon}{\kern0pt}\ {\isachardoublequoteopen}{\isacharprime}{\kern0pt}a{\isachardoublequoteclose}\isanewline
\ \ \isacommand{assume}\isamarkupfalse%
\isanewline
\ \ \ \ finite{\isacharcolon}{\kern0pt}\ {\isachardoublequoteopen}finite\ A{\isachardoublequoteclose}\ \isakeyword{and}\isanewline
\ \ \ \ prof{\isacharunderscore}{\kern0pt}A{\isacharcolon}{\kern0pt}\ {\isachardoublequoteopen}profile\ A\ p{\isachardoublequoteclose}\ \isakeyword{and}\isanewline
\ \ \ \ w{\isacharunderscore}{\kern0pt}in{\isacharunderscore}{\kern0pt}A{\isacharcolon}{\kern0pt}\ {\isachardoublequoteopen}w\ {\isasymin}\ A{\isachardoublequoteclose}\ \isakeyword{and}\isanewline
\ \ \ \ {\isadigit{0}}{\isacharcolon}{\kern0pt}\ {\isachardoublequoteopen}x\ {\isasymin}\ reject\ m\ A\ p{\isachardoublequoteclose}\ \isakeyword{and}\isanewline
\ \ \ \ {\isadigit{1}}{\isacharcolon}{\kern0pt}\ {\isachardoublequoteopen}x\ {\isasymin}\ elect\ m\ A\ p{\isachardoublequoteclose}\ \isakeyword{and}\isanewline
\ \ \ \ {\isadigit{2}}{\isacharcolon}{\kern0pt}\ {\isachardoublequoteopen}{\isasymforall}y{\isasymin}A\ {\isacharminus}{\kern0pt}\ {\isacharbraceleft}{\kern0pt}w{\isacharbraceright}{\kern0pt}{\isachardot}{\kern0pt}\isanewline
\ \ \ \ \ \ \ \ \ \ card\ {\isacharbraceleft}{\kern0pt}i{\isachardot}{\kern0pt}\ i\ {\isacharless}{\kern0pt}\ length\ p\ {\isasymand}\ {\isacharparenleft}{\kern0pt}w{\isacharcomma}{\kern0pt}\ y{\isacharparenright}{\kern0pt}\ {\isasymin}\ {\isacharparenleft}{\kern0pt}p{\isacharbang}{\kern0pt}i{\isacharparenright}{\kern0pt}{\isacharbraceright}{\kern0pt}\ {\isacharless}{\kern0pt}\isanewline
\ \ \ \ \ \ \ \ \ \ \ \ card\ {\isacharbraceleft}{\kern0pt}i{\isachardot}{\kern0pt}\ i\ {\isacharless}{\kern0pt}\ length\ p\ {\isasymand}\ {\isacharparenleft}{\kern0pt}y{\isacharcomma}{\kern0pt}\ w{\isacharparenright}{\kern0pt}\ {\isasymin}\ {\isacharparenleft}{\kern0pt}p{\isacharbang}{\kern0pt}i{\isacharparenright}{\kern0pt}{\isacharbraceright}{\kern0pt}{\isachardoublequoteclose}\isanewline
\ \ \isacommand{have}\isamarkupfalse%
\ {\isachardoublequoteopen}condorcet{\isacharunderscore}{\kern0pt}winner\ A\ p\ w{\isachardoublequoteclose}\isanewline
\ \ \ \ \isacommand{using}\isamarkupfalse%
\ finite\ prof{\isacharunderscore}{\kern0pt}A\ w{\isacharunderscore}{\kern0pt}in{\isacharunderscore}{\kern0pt}A\ {\isachardoublequoteopen}{\isadigit{2}}{\isachardoublequoteclose}\isanewline
\ \ \ \ \isacommand{by}\isamarkupfalse%
\ simp\isanewline
\ \ \isacommand{thus}\isamarkupfalse%
\ {\isachardoublequoteopen}False{\isachardoublequoteclose}\isanewline
\ \ \ \ \isacommand{using}\isamarkupfalse%
\ {\isachardoublequoteopen}{\isadigit{0}}{\isachardoublequoteclose}\ {\isachardoublequoteopen}{\isadigit{1}}{\isachardoublequoteclose}\ condorcet{\isacharunderscore}{\kern0pt}winner{\isachardot}{\kern0pt}simps\ dcc\ IntI\ empty{\isacharunderscore}{\kern0pt}iff\isanewline
\ \ \ \ \ \ \ \ \ \ defer{\isacharunderscore}{\kern0pt}condorcet{\isacharunderscore}{\kern0pt}consistency{\isacharunderscore}{\kern0pt}def\ result{\isacharunderscore}{\kern0pt}disj\isanewline
\ \ \ \ \isacommand{by}\isamarkupfalse%
\ {\isacharparenleft}{\kern0pt}metis\ {\isacharparenleft}{\kern0pt}no{\isacharunderscore}{\kern0pt}types{\isacharcomma}{\kern0pt}\ hide{\isacharunderscore}{\kern0pt}lams{\isacharparenright}{\kern0pt}{\isacharparenright}{\kern0pt}\isanewline
\isacommand{next}\isamarkupfalse%
\isanewline
\ \ \isacommand{fix}\isamarkupfalse%
\isanewline
\ \ \ \ A\ {\isacharcolon}{\kern0pt}{\isacharcolon}{\kern0pt}\ {\isachardoublequoteopen}{\isacharprime}{\kern0pt}a\ set{\isachardoublequoteclose}\ \isakeyword{and}\isanewline
\ \ \ \ p\ {\isacharcolon}{\kern0pt}{\isacharcolon}{\kern0pt}\ {\isachardoublequoteopen}{\isacharprime}{\kern0pt}a\ Profile{\isachardoublequoteclose}\ \isakeyword{and}\isanewline
\ \ \ \ w\ {\isacharcolon}{\kern0pt}{\isacharcolon}{\kern0pt}\ {\isachardoublequoteopen}{\isacharprime}{\kern0pt}a{\isachardoublequoteclose}\ \isakeyword{and}\isanewline
\ \ \ \ x\ {\isacharcolon}{\kern0pt}{\isacharcolon}{\kern0pt}\ {\isachardoublequoteopen}{\isacharprime}{\kern0pt}a{\isachardoublequoteclose}\isanewline
\ \ \isacommand{assume}\isamarkupfalse%
\isanewline
\ \ \ \ finite{\isacharcolon}{\kern0pt}\ {\isachardoublequoteopen}finite\ A{\isachardoublequoteclose}\ \isakeyword{and}\isanewline
\ \ \ \ prof{\isacharunderscore}{\kern0pt}A{\isacharcolon}{\kern0pt}\ {\isachardoublequoteopen}profile\ A\ p{\isachardoublequoteclose}\ \isakeyword{and}\isanewline
\ \ \ \ w{\isacharunderscore}{\kern0pt}in{\isacharunderscore}{\kern0pt}A{\isacharcolon}{\kern0pt}\ {\isachardoublequoteopen}w\ {\isasymin}\ A{\isachardoublequoteclose}\ \isakeyword{and}\isanewline
\ \ \ \ {\isadigit{0}}{\isacharcolon}{\kern0pt}\ {\isachardoublequoteopen}x\ {\isasymin}\ reject\ m\ A\ p{\isachardoublequoteclose}\ \isakeyword{and}\isanewline
\ \ \ \ {\isadigit{1}}{\isacharcolon}{\kern0pt}\ {\isachardoublequoteopen}x\ {\isasymin}\ defer\ m\ A\ p{\isachardoublequoteclose}\ \isakeyword{and}\isanewline
\ \ \ \ {\isadigit{2}}{\isacharcolon}{\kern0pt}\ {\isachardoublequoteopen}{\isasymforall}y{\isasymin}A\ {\isacharminus}{\kern0pt}\ {\isacharbraceleft}{\kern0pt}w{\isacharbraceright}{\kern0pt}{\isachardot}{\kern0pt}\isanewline
\ \ \ \ \ \ \ \ \ \ card\ {\isacharbraceleft}{\kern0pt}i{\isachardot}{\kern0pt}\ i\ {\isacharless}{\kern0pt}\ length\ p\ {\isasymand}\ {\isacharparenleft}{\kern0pt}w{\isacharcomma}{\kern0pt}\ y{\isacharparenright}{\kern0pt}\ {\isasymin}\ {\isacharparenleft}{\kern0pt}p{\isacharbang}{\kern0pt}i{\isacharparenright}{\kern0pt}{\isacharbraceright}{\kern0pt}\ {\isacharless}{\kern0pt}\isanewline
\ \ \ \ \ \ \ \ \ \ \ \ card\ {\isacharbraceleft}{\kern0pt}i{\isachardot}{\kern0pt}\ i\ {\isacharless}{\kern0pt}\ length\ p\ {\isasymand}\ {\isacharparenleft}{\kern0pt}y{\isacharcomma}{\kern0pt}\ w{\isacharparenright}{\kern0pt}\ {\isasymin}\ {\isacharparenleft}{\kern0pt}p{\isacharbang}{\kern0pt}i{\isacharparenright}{\kern0pt}{\isacharbraceright}{\kern0pt}{\isachardoublequoteclose}\isanewline
\ \ \isacommand{have}\isamarkupfalse%
\ {\isachardoublequoteopen}condorcet{\isacharunderscore}{\kern0pt}winner\ A\ p\ w{\isachardoublequoteclose}\isanewline
\ \ \ \ \isacommand{using}\isamarkupfalse%
\ finite\ prof{\isacharunderscore}{\kern0pt}A\ w{\isacharunderscore}{\kern0pt}in{\isacharunderscore}{\kern0pt}A\ {\isachardoublequoteopen}{\isadigit{2}}{\isachardoublequoteclose}\isanewline
\ \ \ \ \isacommand{by}\isamarkupfalse%
\ simp\isanewline
\ \ \isacommand{thus}\isamarkupfalse%
\ {\isachardoublequoteopen}False{\isachardoublequoteclose}\isanewline
\ \ \ \ \isacommand{using}\isamarkupfalse%
\ {\isachardoublequoteopen}{\isadigit{0}}{\isachardoublequoteclose}\ {\isachardoublequoteopen}{\isadigit{1}}{\isachardoublequoteclose}\ dcc\ defer{\isacharunderscore}{\kern0pt}condorcet{\isacharunderscore}{\kern0pt}consistency{\isacharunderscore}{\kern0pt}def\ IntI\isanewline
\ \ \ \ \ \ \ \ \ \ Diff{\isacharunderscore}{\kern0pt}empty\ Diff{\isacharunderscore}{\kern0pt}iff\ finite\ prof{\isacharunderscore}{\kern0pt}A\ result{\isacharunderscore}{\kern0pt}disj\isanewline
\ \ \ \ \isacommand{by}\isamarkupfalse%
\ {\isacharparenleft}{\kern0pt}metis\ {\isacharparenleft}{\kern0pt}no{\isacharunderscore}{\kern0pt}types{\isacharcomma}{\kern0pt}\ hide{\isacharunderscore}{\kern0pt}lams{\isacharparenright}{\kern0pt}{\isacharparenright}{\kern0pt}\isanewline
\isacommand{next}\isamarkupfalse%
\isanewline
\ \ \isacommand{fix}\isamarkupfalse%
\isanewline
\ \ \ \ A\ {\isacharcolon}{\kern0pt}{\isacharcolon}{\kern0pt}\ {\isachardoublequoteopen}{\isacharprime}{\kern0pt}a\ set{\isachardoublequoteclose}\ \isakeyword{and}\isanewline
\ \ \ \ p\ {\isacharcolon}{\kern0pt}{\isacharcolon}{\kern0pt}\ {\isachardoublequoteopen}{\isacharprime}{\kern0pt}a\ Profile{\isachardoublequoteclose}\ \isakeyword{and}\isanewline
\ \ \ \ w\ {\isacharcolon}{\kern0pt}{\isacharcolon}{\kern0pt}\ {\isachardoublequoteopen}{\isacharprime}{\kern0pt}a{\isachardoublequoteclose}\ \isakeyword{and}\isanewline
\ \ \ \ x\ {\isacharcolon}{\kern0pt}{\isacharcolon}{\kern0pt}\ {\isachardoublequoteopen}{\isacharprime}{\kern0pt}a{\isachardoublequoteclose}\isanewline
\ \ \isacommand{assume}\isamarkupfalse%
\isanewline
\ \ \ \ finite{\isacharcolon}{\kern0pt}\ {\isachardoublequoteopen}finite\ A{\isachardoublequoteclose}\ \isakeyword{and}\isanewline
\ \ \ \ prof{\isacharunderscore}{\kern0pt}A{\isacharcolon}{\kern0pt}\ {\isachardoublequoteopen}profile\ A\ p{\isachardoublequoteclose}\ \isakeyword{and}\isanewline
\ \ \ \ w{\isacharunderscore}{\kern0pt}in{\isacharunderscore}{\kern0pt}A{\isacharcolon}{\kern0pt}\ {\isachardoublequoteopen}w\ {\isasymin}\ A{\isachardoublequoteclose}\ \isakeyword{and}\isanewline
\ \ \ \ x{\isacharunderscore}{\kern0pt}in{\isacharunderscore}{\kern0pt}A{\isacharcolon}{\kern0pt}\ {\isachardoublequoteopen}x\ {\isasymin}\ A{\isachardoublequoteclose}\ \isakeyword{and}\isanewline
\ \ \ \ {\isadigit{0}}{\isacharcolon}{\kern0pt}\ {\isachardoublequoteopen}x\ {\isasymnotin}\ reject\ m\ A\ p{\isachardoublequoteclose}\ \isakeyword{and}\isanewline
\ \ \ \ {\isadigit{1}}{\isacharcolon}{\kern0pt}\ {\isachardoublequoteopen}x\ {\isasymnotin}\ defer\ m\ A\ p{\isachardoublequoteclose}\ \isakeyword{and}\isanewline
\ \ \ \ {\isadigit{2}}{\isacharcolon}{\kern0pt}\ {\isachardoublequoteopen}{\isasymforall}y{\isasymin}A\ {\isacharminus}{\kern0pt}\ {\isacharbraceleft}{\kern0pt}w{\isacharbraceright}{\kern0pt}{\isachardot}{\kern0pt}\isanewline
\ \ \ \ \ \ \ \ \ \ card\ {\isacharbraceleft}{\kern0pt}i{\isachardot}{\kern0pt}\ i\ {\isacharless}{\kern0pt}\ length\ p\ {\isasymand}\ {\isacharparenleft}{\kern0pt}w{\isacharcomma}{\kern0pt}\ y{\isacharparenright}{\kern0pt}\ {\isasymin}\ {\isacharparenleft}{\kern0pt}p{\isacharbang}{\kern0pt}i{\isacharparenright}{\kern0pt}{\isacharbraceright}{\kern0pt}\ {\isacharless}{\kern0pt}\isanewline
\ \ \ \ \ \ \ \ \ \ \ \ card\ {\isacharbraceleft}{\kern0pt}i{\isachardot}{\kern0pt}\ i\ {\isacharless}{\kern0pt}\ length\ p\ {\isasymand}\ {\isacharparenleft}{\kern0pt}y{\isacharcomma}{\kern0pt}\ w{\isacharparenright}{\kern0pt}\ {\isasymin}\ {\isacharparenleft}{\kern0pt}p{\isacharbang}{\kern0pt}i{\isacharparenright}{\kern0pt}{\isacharbraceright}{\kern0pt}{\isachardoublequoteclose}\isanewline
\ \ \isacommand{have}\isamarkupfalse%
\ {\isachardoublequoteopen}condorcet{\isacharunderscore}{\kern0pt}winner\ A\ p\ w{\isachardoublequoteclose}\isanewline
\ \ \ \ \isacommand{using}\isamarkupfalse%
\ finite\ prof{\isacharunderscore}{\kern0pt}A\ w{\isacharunderscore}{\kern0pt}in{\isacharunderscore}{\kern0pt}A\ {\isachardoublequoteopen}{\isadigit{2}}{\isachardoublequoteclose}\isanewline
\ \ \ \ \isacommand{by}\isamarkupfalse%
\ simp\isanewline
\ \ \isacommand{thus}\isamarkupfalse%
\ {\isachardoublequoteopen}x\ {\isasymin}\ elect\ m\ A\ p{\isachardoublequoteclose}\isanewline
\ \ \ \ \isacommand{using}\isamarkupfalse%
\ {\isachardoublequoteopen}{\isadigit{0}}{\isachardoublequoteclose}\ {\isachardoublequoteopen}{\isadigit{1}}{\isachardoublequoteclose}\ condorcet{\isacharunderscore}{\kern0pt}winner{\isachardot}{\kern0pt}simps\ dcc\ x{\isacharunderscore}{\kern0pt}in{\isacharunderscore}{\kern0pt}A\isanewline
\ \ \ \ \ \ \ \ \ \ defer{\isacharunderscore}{\kern0pt}condorcet{\isacharunderscore}{\kern0pt}consistency{\isacharunderscore}{\kern0pt}def\ electoral{\isacharunderscore}{\kern0pt}mod{\isacharunderscore}{\kern0pt}defer{\isacharunderscore}{\kern0pt}elem\isanewline
\ \ \ \ \isacommand{by}\isamarkupfalse%
\ {\isacharparenleft}{\kern0pt}metis\ {\isacharparenleft}{\kern0pt}no{\isacharunderscore}{\kern0pt}types{\isacharcomma}{\kern0pt}\ lifting{\isacharparenright}{\kern0pt}{\isacharparenright}{\kern0pt}\isanewline
\isacommand{qed}\isamarkupfalse%
%
\endisatagproof
{\isafoldproof}%
%
\isadelimproof
\isanewline
%
\endisadelimproof
\isanewline
\isacommand{lemma}\isamarkupfalse%
\ ccomp{\isacharunderscore}{\kern0pt}and{\isacharunderscore}{\kern0pt}dd{\isacharunderscore}{\kern0pt}imp{\isacharunderscore}{\kern0pt}def{\isacharunderscore}{\kern0pt}only{\isacharunderscore}{\kern0pt}winner{\isacharcolon}{\kern0pt}\isanewline
\ \ \isakeyword{assumes}\ ccomp{\isacharcolon}{\kern0pt}\ {\isachardoublequoteopen}condorcet{\isacharunderscore}{\kern0pt}compatibility\ m{\isachardoublequoteclose}\ \isakeyword{and}\isanewline
\ \ \ \ \ \ \ \ \ \ dd{\isacharcolon}{\kern0pt}\ {\isachardoublequoteopen}defer{\isacharunderscore}{\kern0pt}deciding\ m{\isachardoublequoteclose}\ \isakeyword{and}\isanewline
\ \ \ \ \ \ \ \ \ \ winner{\isacharcolon}{\kern0pt}\ {\isachardoublequoteopen}condorcet{\isacharunderscore}{\kern0pt}winner\ A\ p\ w{\isachardoublequoteclose}\isanewline
\ \ \isakeyword{shows}\ {\isachardoublequoteopen}defer\ m\ A\ p\ {\isacharequal}{\kern0pt}\ {\isacharbraceleft}{\kern0pt}w{\isacharbraceright}{\kern0pt}{\isachardoublequoteclose}\isanewline
%
\isadelimproof
%
\endisadelimproof
%
\isatagproof
\isacommand{proof}\isamarkupfalse%
\ {\isacharparenleft}{\kern0pt}rule\ ccontr{\isacharparenright}{\kern0pt}\isanewline
\ \ \isacommand{assume}\isamarkupfalse%
\ not{\isacharunderscore}{\kern0pt}w{\isacharcolon}{\kern0pt}\ {\isachardoublequoteopen}defer\ m\ A\ p\ {\isasymnoteq}\ {\isacharbraceleft}{\kern0pt}w{\isacharbraceright}{\kern0pt}{\isachardoublequoteclose}\isanewline
\ \ \isacommand{from}\isamarkupfalse%
\ dd\ \isacommand{have}\isamarkupfalse%
\ def{\isacharunderscore}{\kern0pt}{\isadigit{1}}{\isacharcolon}{\kern0pt}\isanewline
\ \ \ \ {\isachardoublequoteopen}defers\ {\isadigit{1}}\ m{\isachardoublequoteclose}\isanewline
\ \ \ \ \isacommand{using}\isamarkupfalse%
\ defer{\isacharunderscore}{\kern0pt}deciding{\isacharunderscore}{\kern0pt}def\isanewline
\ \ \ \ \isacommand{by}\isamarkupfalse%
\ metis\isanewline
\ \ \isacommand{hence}\isamarkupfalse%
\ c{\isacharunderscore}{\kern0pt}win{\isacharcolon}{\kern0pt}\isanewline
\ \ \ \ {\isachardoublequoteopen}finite{\isacharunderscore}{\kern0pt}profile\ A\ p\ {\isasymand}\ \ w\ {\isasymin}\ A\ {\isasymand}\ {\isacharparenleft}{\kern0pt}{\isasymforall}x\ {\isasymin}\ A\ {\isacharminus}{\kern0pt}\ {\isacharbraceleft}{\kern0pt}w{\isacharbraceright}{\kern0pt}\ {\isachardot}{\kern0pt}\ wins\ w\ p\ x{\isacharparenright}{\kern0pt}{\isachardoublequoteclose}\isanewline
\ \ \ \ \isacommand{using}\isamarkupfalse%
\ winner\isanewline
\ \ \ \ \isacommand{by}\isamarkupfalse%
\ simp\isanewline
\ \ \isacommand{hence}\isamarkupfalse%
\ {\isachardoublequoteopen}card\ {\isacharparenleft}{\kern0pt}defer\ m\ A\ p{\isacharparenright}{\kern0pt}\ {\isacharequal}{\kern0pt}\ {\isadigit{1}}{\isachardoublequoteclose}\isanewline
\ \ \ \ \isacommand{using}\isamarkupfalse%
\ One{\isacharunderscore}{\kern0pt}nat{\isacharunderscore}{\kern0pt}def\ Suc{\isacharunderscore}{\kern0pt}leI\ card{\isacharunderscore}{\kern0pt}gt{\isacharunderscore}{\kern0pt}{\isadigit{0}}{\isacharunderscore}{\kern0pt}iff\isanewline
\ \ \ \ \ \ \ \ \ \ def{\isacharunderscore}{\kern0pt}{\isadigit{1}}\ defers{\isacharunderscore}{\kern0pt}def\ equals{\isadigit{0}}D\isanewline
\ \ \ \ \isacommand{by}\isamarkupfalse%
\ metis\isanewline
\ \ \isacommand{hence}\isamarkupfalse%
\ {\isadigit{0}}{\isacharcolon}{\kern0pt}\ {\isachardoublequoteopen}{\isasymexists}x\ {\isasymin}\ A\ {\isachardot}{\kern0pt}\ defer\ m\ A\ p\ {\isacharequal}{\kern0pt}{\isacharbraceleft}{\kern0pt}x{\isacharbraceright}{\kern0pt}{\isachardoublequoteclose}\isanewline
\ \ \ \ \isacommand{using}\isamarkupfalse%
\ card{\isacharunderscore}{\kern0pt}{\isadigit{1}}{\isacharunderscore}{\kern0pt}singletonE\ dd\ defer{\isacharunderscore}{\kern0pt}deciding{\isacharunderscore}{\kern0pt}def\isanewline
\ \ \ \ \ \ \ \ \ \ defer{\isacharunderscore}{\kern0pt}in{\isacharunderscore}{\kern0pt}alts\ insert{\isacharunderscore}{\kern0pt}subset\ c{\isacharunderscore}{\kern0pt}win\isanewline
\ \ \ \ \isacommand{by}\isamarkupfalse%
\ metis\isanewline
\ \ \isacommand{with}\isamarkupfalse%
\ not{\isacharunderscore}{\kern0pt}w\ \isacommand{have}\isamarkupfalse%
\ {\isachardoublequoteopen}{\isasymexists}l\ {\isasymin}\ A\ {\isachardot}{\kern0pt}\ l\ {\isasymnoteq}\ w\ {\isasymand}\ defer\ m\ A\ p\ {\isacharequal}{\kern0pt}\ {\isacharbraceleft}{\kern0pt}l{\isacharbraceright}{\kern0pt}{\isachardoublequoteclose}\isanewline
\ \ \ \ \isacommand{by}\isamarkupfalse%
\ metis\isanewline
\ \ \isacommand{hence}\isamarkupfalse%
\ not{\isacharunderscore}{\kern0pt}in{\isacharunderscore}{\kern0pt}defer{\isacharcolon}{\kern0pt}\ {\isachardoublequoteopen}w\ {\isasymnotin}\ defer\ m\ A\ p{\isachardoublequoteclose}\isanewline
\ \ \ \ \isacommand{by}\isamarkupfalse%
\ auto\isanewline
\ \ \isacommand{have}\isamarkupfalse%
\ {\isachardoublequoteopen}non{\isacharunderscore}{\kern0pt}electing\ m{\isachardoublequoteclose}\isanewline
\ \ \ \ \isacommand{using}\isamarkupfalse%
\ dd\ defer{\isacharunderscore}{\kern0pt}deciding{\isacharunderscore}{\kern0pt}def\isanewline
\ \ \ \ \isacommand{by}\isamarkupfalse%
\ metis\isanewline
\ \ \isacommand{hence}\isamarkupfalse%
\ not{\isacharunderscore}{\kern0pt}in{\isacharunderscore}{\kern0pt}elect{\isacharcolon}{\kern0pt}\ {\isachardoublequoteopen}w\ {\isasymnotin}\ elect\ m\ A\ p{\isachardoublequoteclose}\isanewline
\ \ \ \ \isacommand{using}\isamarkupfalse%
\ c{\isacharunderscore}{\kern0pt}win\ equals{\isadigit{0}}D\ non{\isacharunderscore}{\kern0pt}electing{\isacharunderscore}{\kern0pt}def\isanewline
\ \ \ \ \isacommand{by}\isamarkupfalse%
\ metis\isanewline
\ \ \isacommand{from}\isamarkupfalse%
\ not{\isacharunderscore}{\kern0pt}in{\isacharunderscore}{\kern0pt}defer\ not{\isacharunderscore}{\kern0pt}in{\isacharunderscore}{\kern0pt}elect\ \isacommand{have}\isamarkupfalse%
\ one{\isacharunderscore}{\kern0pt}side{\isacharcolon}{\kern0pt}\isanewline
\ \ \ \ {\isachardoublequoteopen}w\ {\isasymin}\ reject\ m\ A\ p{\isachardoublequoteclose}\isanewline
\ \ \ \ \isacommand{using}\isamarkupfalse%
\ ccomp\ condorcet{\isacharunderscore}{\kern0pt}compatibility{\isacharunderscore}{\kern0pt}def\ c{\isacharunderscore}{\kern0pt}win\isanewline
\ \ \ \ \ \ \ \ \ \ electoral{\isacharunderscore}{\kern0pt}mod{\isacharunderscore}{\kern0pt}defer{\isacharunderscore}{\kern0pt}elem\isanewline
\ \ \ \ \isacommand{by}\isamarkupfalse%
\ metis\isanewline
\ \ \isacommand{from}\isamarkupfalse%
\ ccomp\ \isacommand{have}\isamarkupfalse%
\ other{\isacharunderscore}{\kern0pt}side{\isacharcolon}{\kern0pt}\ {\isachardoublequoteopen}w\ {\isasymnotin}\ reject\ m\ A\ p{\isachardoublequoteclose}\isanewline
\ \ \ \ \isacommand{using}\isamarkupfalse%
\ condorcet{\isacharunderscore}{\kern0pt}compatibility{\isacharunderscore}{\kern0pt}def\ c{\isacharunderscore}{\kern0pt}win\ winner\isanewline
\ \ \ \ \isacommand{by}\isamarkupfalse%
\ {\isacharparenleft}{\kern0pt}metis\ {\isacharparenleft}{\kern0pt}no{\isacharunderscore}{\kern0pt}types{\isacharcomma}{\kern0pt}\ hide{\isacharunderscore}{\kern0pt}lams{\isacharparenright}{\kern0pt}{\isacharparenright}{\kern0pt}\isanewline
\ \ \isacommand{thus}\isamarkupfalse%
\ False\isanewline
\ \ \ \ \isacommand{by}\isamarkupfalse%
\ {\isacharparenleft}{\kern0pt}simp\ add{\isacharcolon}{\kern0pt}\ one{\isacharunderscore}{\kern0pt}side{\isacharparenright}{\kern0pt}\isanewline
\isacommand{qed}\isamarkupfalse%
%
\endisatagproof
{\isafoldproof}%
%
\isadelimproof
\isanewline
%
\endisadelimproof
\isanewline
\isacommand{theorem}\isamarkupfalse%
\ ccomp{\isacharunderscore}{\kern0pt}and{\isacharunderscore}{\kern0pt}dd{\isacharunderscore}{\kern0pt}imp{\isacharunderscore}{\kern0pt}dcc{\isacharbrackleft}{\kern0pt}simp{\isacharbrackright}{\kern0pt}{\isacharcolon}{\kern0pt}\isanewline
\ \ \isakeyword{assumes}\ ccomp{\isacharcolon}{\kern0pt}\ {\isachardoublequoteopen}condorcet{\isacharunderscore}{\kern0pt}compatibility\ m{\isachardoublequoteclose}\ \isakeyword{and}\isanewline
\ \ \ \ \ \ \ \ \ \ dd{\isacharcolon}{\kern0pt}\ {\isachardoublequoteopen}defer{\isacharunderscore}{\kern0pt}deciding\ m{\isachardoublequoteclose}\isanewline
\ \ \isakeyword{shows}\ {\isachardoublequoteopen}defer{\isacharunderscore}{\kern0pt}condorcet{\isacharunderscore}{\kern0pt}consistency\ m{\isachardoublequoteclose}\isanewline
%
\isadelimproof
%
\endisadelimproof
%
\isatagproof
\isacommand{proof}\isamarkupfalse%
\ {\isacharparenleft}{\kern0pt}unfold\ defer{\isacharunderscore}{\kern0pt}condorcet{\isacharunderscore}{\kern0pt}consistency{\isacharunderscore}{\kern0pt}def{\isacharcomma}{\kern0pt}\ auto{\isacharparenright}{\kern0pt}\isanewline
\ \ \isacommand{show}\isamarkupfalse%
\ {\isachardoublequoteopen}electoral{\isacharunderscore}{\kern0pt}module\ m{\isachardoublequoteclose}\isanewline
\ \ \ \ \isacommand{using}\isamarkupfalse%
\ dd\ defer{\isacharunderscore}{\kern0pt}deciding{\isacharunderscore}{\kern0pt}def\isanewline
\ \ \ \ \isacommand{by}\isamarkupfalse%
\ metis\isanewline
\isacommand{next}\isamarkupfalse%
\isanewline
\ \ \isacommand{fix}\isamarkupfalse%
\isanewline
\ \ \ \ A\ {\isacharcolon}{\kern0pt}{\isacharcolon}{\kern0pt}\ {\isachardoublequoteopen}{\isacharprime}{\kern0pt}a\ set{\isachardoublequoteclose}\ \isakeyword{and}\isanewline
\ \ \ \ p\ {\isacharcolon}{\kern0pt}{\isacharcolon}{\kern0pt}\ {\isachardoublequoteopen}{\isacharprime}{\kern0pt}a\ Profile{\isachardoublequoteclose}\ \isakeyword{and}\isanewline
\ \ \ \ w\ {\isacharcolon}{\kern0pt}{\isacharcolon}{\kern0pt}\ {\isachardoublequoteopen}{\isacharprime}{\kern0pt}a{\isachardoublequoteclose}\isanewline
\ \ \isacommand{assume}\isamarkupfalse%
\isanewline
\ \ \ \ prof{\isacharunderscore}{\kern0pt}A{\isacharcolon}{\kern0pt}\ {\isachardoublequoteopen}profile\ A\ p{\isachardoublequoteclose}\ \isakeyword{and}\isanewline
\ \ \ \ w{\isacharunderscore}{\kern0pt}in{\isacharunderscore}{\kern0pt}A{\isacharcolon}{\kern0pt}\ {\isachardoublequoteopen}w\ {\isasymin}\ A{\isachardoublequoteclose}\ \isakeyword{and}\isanewline
\ \ \ \ finiteness{\isacharcolon}{\kern0pt}\ {\isachardoublequoteopen}finite\ A{\isachardoublequoteclose}\ \isakeyword{and}\isanewline
\ \ \ \ assm{\isacharcolon}{\kern0pt}\ {\isachardoublequoteopen}{\isasymforall}x{\isasymin}A\ {\isacharminus}{\kern0pt}\ {\isacharbraceleft}{\kern0pt}w{\isacharbraceright}{\kern0pt}{\isachardot}{\kern0pt}\isanewline
\ \ \ \ \ \ \ \ \ \ card\ {\isacharbraceleft}{\kern0pt}i{\isachardot}{\kern0pt}\ i\ {\isacharless}{\kern0pt}\ length\ p\ {\isasymand}\ {\isacharparenleft}{\kern0pt}w{\isacharcomma}{\kern0pt}\ x{\isacharparenright}{\kern0pt}\ {\isasymin}\ {\isacharparenleft}{\kern0pt}p{\isacharbang}{\kern0pt}i{\isacharparenright}{\kern0pt}{\isacharbraceright}{\kern0pt}\ {\isacharless}{\kern0pt}\isanewline
\ \ \ \ \ \ \ \ \ \ \ \ card\ {\isacharbraceleft}{\kern0pt}i{\isachardot}{\kern0pt}\ i\ {\isacharless}{\kern0pt}\ length\ p\ {\isasymand}\ {\isacharparenleft}{\kern0pt}x{\isacharcomma}{\kern0pt}\ w{\isacharparenright}{\kern0pt}\ {\isasymin}\ {\isacharparenleft}{\kern0pt}p{\isacharbang}{\kern0pt}i{\isacharparenright}{\kern0pt}{\isacharbraceright}{\kern0pt}{\isachardoublequoteclose}\isanewline
\ \ \isacommand{have}\isamarkupfalse%
\ winner{\isacharcolon}{\kern0pt}\ {\isachardoublequoteopen}condorcet{\isacharunderscore}{\kern0pt}winner\ A\ p\ w{\isachardoublequoteclose}\isanewline
\ \ \ \ \isacommand{using}\isamarkupfalse%
\ assm\ finiteness\ prof{\isacharunderscore}{\kern0pt}A\ w{\isacharunderscore}{\kern0pt}in{\isacharunderscore}{\kern0pt}A\isanewline
\ \ \ \ \isacommand{by}\isamarkupfalse%
\ simp\isanewline
\ \ \isacommand{hence}\isamarkupfalse%
\isanewline
\ \ \ \ {\isachardoublequoteopen}m\ A\ p\ {\isacharequal}{\kern0pt}\isanewline
\ \ \ \ \ \ {\isacharparenleft}{\kern0pt}{\isacharbraceleft}{\kern0pt}{\isacharbraceright}{\kern0pt}{\isacharcomma}{\kern0pt}\isanewline
\ \ \ \ \ \ \ \ A\ {\isacharminus}{\kern0pt}\ defer\ m\ A\ p{\isacharcomma}{\kern0pt}\isanewline
\ \ \ \ \ \ \ \ {\isacharbraceleft}{\kern0pt}d\ {\isasymin}\ A{\isachardot}{\kern0pt}\ condorcet{\isacharunderscore}{\kern0pt}winner\ A\ p\ d{\isacharbraceright}{\kern0pt}{\isacharparenright}{\kern0pt}{\isachardoublequoteclose}\isanewline
\ \ \isacommand{proof}\isamarkupfalse%
\ {\isacharminus}{\kern0pt}\isanewline
\ \ \ \ \isanewline
\ \ \ \ \isacommand{from}\isamarkupfalse%
\ dd\ \isacommand{have}\isamarkupfalse%
\ {\isadigit{0}}{\isacharcolon}{\kern0pt}\isanewline
\ \ \ \ \ \ {\isachardoublequoteopen}elect\ m\ A\ p\ {\isacharequal}{\kern0pt}\ {\isacharbraceleft}{\kern0pt}{\isacharbraceright}{\kern0pt}{\isachardoublequoteclose}\isanewline
\ \ \ \ \ \ \isacommand{using}\isamarkupfalse%
\ defer{\isacharunderscore}{\kern0pt}deciding{\isacharunderscore}{\kern0pt}def\ non{\isacharunderscore}{\kern0pt}electing{\isacharunderscore}{\kern0pt}def\isanewline
\ \ \ \ \ \ \ \ \ \ \ \ winner\isanewline
\ \ \ \ \ \ \isacommand{by}\isamarkupfalse%
\ fastforce\isanewline
\ \ \ \ \isanewline
\ \ \ \ \isacommand{from}\isamarkupfalse%
\ dd\ ccomp\ \isacommand{have}\isamarkupfalse%
\ {\isadigit{1}}{\isacharcolon}{\kern0pt}\ {\isachardoublequoteopen}defer\ m\ A\ p\ {\isacharequal}{\kern0pt}\ {\isacharbraceleft}{\kern0pt}w{\isacharbraceright}{\kern0pt}{\isachardoublequoteclose}\isanewline
\ \ \ \ \ \ \isacommand{using}\isamarkupfalse%
\ ccomp{\isacharunderscore}{\kern0pt}and{\isacharunderscore}{\kern0pt}dd{\isacharunderscore}{\kern0pt}imp{\isacharunderscore}{\kern0pt}def{\isacharunderscore}{\kern0pt}only{\isacharunderscore}{\kern0pt}winner\ winner\isanewline
\ \ \ \ \ \ \isacommand{by}\isamarkupfalse%
\ simp\isanewline
\ \ \ \ \isanewline
\ \ \ \ \isacommand{from}\isamarkupfalse%
\ {\isadigit{0}}\ {\isadigit{1}}\ \isacommand{have}\isamarkupfalse%
\ {\isadigit{2}}{\isacharcolon}{\kern0pt}\ {\isachardoublequoteopen}reject\ m\ A\ p\ {\isacharequal}{\kern0pt}\ A\ {\isacharminus}{\kern0pt}\ defer\ m\ A\ p{\isachardoublequoteclose}\isanewline
\ \ \ \ \ \ \isacommand{using}\isamarkupfalse%
\ Diff{\isacharunderscore}{\kern0pt}empty\ dd\ defer{\isacharunderscore}{\kern0pt}deciding{\isacharunderscore}{\kern0pt}def\isanewline
\ \ \ \ \ \ \ \ \ \ \ \ reject{\isacharunderscore}{\kern0pt}not{\isacharunderscore}{\kern0pt}elec{\isacharunderscore}{\kern0pt}or{\isacharunderscore}{\kern0pt}def\ winner\isanewline
\ \ \ \ \ \ \isacommand{by}\isamarkupfalse%
\ fastforce\isanewline
\ \ \ \ \isacommand{from}\isamarkupfalse%
\ {\isadigit{0}}\ {\isadigit{1}}\ {\isadigit{2}}\ \isacommand{have}\isamarkupfalse%
\ {\isadigit{3}}{\isacharcolon}{\kern0pt}\ {\isachardoublequoteopen}m\ A\ p\ {\isacharequal}{\kern0pt}\ {\isacharparenleft}{\kern0pt}{\isacharbraceleft}{\kern0pt}{\isacharbraceright}{\kern0pt}{\isacharcomma}{\kern0pt}\ A\ {\isacharminus}{\kern0pt}\ defer\ m\ A\ p{\isacharcomma}{\kern0pt}\ {\isacharbraceleft}{\kern0pt}w{\isacharbraceright}{\kern0pt}{\isacharparenright}{\kern0pt}{\isachardoublequoteclose}\isanewline
\ \ \ \ \ \ \isacommand{using}\isamarkupfalse%
\ combine{\isacharunderscore}{\kern0pt}ele{\isacharunderscore}{\kern0pt}rej{\isacharunderscore}{\kern0pt}def\isanewline
\ \ \ \ \ \ \isacommand{by}\isamarkupfalse%
\ metis\isanewline
\ \ \ \ \isacommand{have}\isamarkupfalse%
\ {\isachardoublequoteopen}{\isacharbraceleft}{\kern0pt}w{\isacharbraceright}{\kern0pt}\ {\isacharequal}{\kern0pt}\ {\isacharbraceleft}{\kern0pt}d\ {\isasymin}\ A{\isachardot}{\kern0pt}\ condorcet{\isacharunderscore}{\kern0pt}winner\ A\ p\ d{\isacharbraceright}{\kern0pt}{\isachardoublequoteclose}\isanewline
\ \ \ \ \ \ \isacommand{using}\isamarkupfalse%
\ cond{\isacharunderscore}{\kern0pt}winner{\isacharunderscore}{\kern0pt}unique{\isadigit{3}}\ winner\isanewline
\ \ \ \ \ \ \isacommand{by}\isamarkupfalse%
\ metis\isanewline
\ \ \ \ \isacommand{thus}\isamarkupfalse%
\ {\isacharquery}{\kern0pt}thesis\isanewline
\ \ \ \ \ \ \isacommand{using}\isamarkupfalse%
\ {\isachardoublequoteopen}{\isadigit{3}}{\isachardoublequoteclose}\isanewline
\ \ \ \ \ \ \isacommand{by}\isamarkupfalse%
\ auto\isanewline
\ \ \isacommand{qed}\isamarkupfalse%
\isanewline
\ \ \isacommand{hence}\isamarkupfalse%
\isanewline
\ \ \ \ {\isachardoublequoteopen}m\ A\ p\ {\isacharequal}{\kern0pt}\isanewline
\ \ \ \ \ \ {\isacharparenleft}{\kern0pt}{\isacharbraceleft}{\kern0pt}{\isacharbraceright}{\kern0pt}{\isacharcomma}{\kern0pt}\isanewline
\ \ \ \ \ \ \ \ A\ {\isacharminus}{\kern0pt}\ defer\ m\ A\ p{\isacharcomma}{\kern0pt}\isanewline
\ \ \ \ \ \ \ \ {\isacharbraceleft}{\kern0pt}d\ {\isasymin}\ A{\isachardot}{\kern0pt}\ {\isasymforall}x{\isasymin}A\ {\isacharminus}{\kern0pt}\ {\isacharbraceleft}{\kern0pt}d{\isacharbraceright}{\kern0pt}{\isachardot}{\kern0pt}\ wins\ d\ p\ x{\isacharbraceright}{\kern0pt}{\isacharparenright}{\kern0pt}{\isachardoublequoteclose}\isanewline
\ \ \ \ \isacommand{using}\isamarkupfalse%
\ finiteness\ prof{\isacharunderscore}{\kern0pt}A\ winner\ Collect{\isacharunderscore}{\kern0pt}cong\isanewline
\ \ \ \ \isacommand{by}\isamarkupfalse%
\ auto\isanewline
\ \ \isacommand{hence}\isamarkupfalse%
\isanewline
\ \ \ \ {\isachardoublequoteopen}m\ A\ p\ {\isacharequal}{\kern0pt}\isanewline
\ \ \ \ \ \ \ \ {\isacharparenleft}{\kern0pt}{\isacharbraceleft}{\kern0pt}{\isacharbraceright}{\kern0pt}{\isacharcomma}{\kern0pt}\isanewline
\ \ \ \ \ \ \ \ \ \ A\ {\isacharminus}{\kern0pt}\ defer\ m\ A\ p{\isacharcomma}{\kern0pt}\isanewline
\ \ \ \ \ \ \ \ \ \ {\isacharbraceleft}{\kern0pt}d\ {\isasymin}\ A{\isachardot}{\kern0pt}\ {\isasymforall}x{\isasymin}A\ {\isacharminus}{\kern0pt}\ {\isacharbraceleft}{\kern0pt}d{\isacharbraceright}{\kern0pt}{\isachardot}{\kern0pt}\isanewline
\ \ \ \ \ \ \ \ \ \ \ \ prefer{\isacharunderscore}{\kern0pt}count\ p\ x\ d\ {\isacharless}{\kern0pt}\ prefer{\isacharunderscore}{\kern0pt}count\ p\ d\ x{\isacharbraceright}{\kern0pt}{\isacharparenright}{\kern0pt}{\isachardoublequoteclose}\isanewline
\ \ \ \ \isacommand{by}\isamarkupfalse%
\ simp\isanewline
\ \ \isacommand{hence}\isamarkupfalse%
\isanewline
\ \ \ \ {\isachardoublequoteopen}m\ A\ p\ {\isacharequal}{\kern0pt}\isanewline
\ \ \ \ \ \ \ \ {\isacharparenleft}{\kern0pt}{\isacharbraceleft}{\kern0pt}{\isacharbraceright}{\kern0pt}{\isacharcomma}{\kern0pt}\isanewline
\ \ \ \ \ \ \ \ \ \ A\ {\isacharminus}{\kern0pt}\ defer\ m\ A\ p{\isacharcomma}{\kern0pt}\isanewline
\ \ \ \ \ \ \ \ \ \ {\isacharbraceleft}{\kern0pt}d\ {\isasymin}\ A{\isachardot}{\kern0pt}\ {\isasymforall}x{\isasymin}A\ {\isacharminus}{\kern0pt}\ {\isacharbraceleft}{\kern0pt}d{\isacharbraceright}{\kern0pt}{\isachardot}{\kern0pt}\isanewline
\ \ \ \ \ \ \ \ \ \ \ \ card\ {\isacharbraceleft}{\kern0pt}i{\isachardot}{\kern0pt}\ i\ {\isacharless}{\kern0pt}\ length\ p\ {\isasymand}\ {\isacharparenleft}{\kern0pt}let\ r\ {\isacharequal}{\kern0pt}\ {\isacharparenleft}{\kern0pt}p{\isacharbang}{\kern0pt}i{\isacharparenright}{\kern0pt}\ in\ {\isacharparenleft}{\kern0pt}d\ {\isasympreceq}\isactrlsub r\ x{\isacharparenright}{\kern0pt}{\isacharparenright}{\kern0pt}{\isacharbraceright}{\kern0pt}\ {\isacharless}{\kern0pt}\isanewline
\ \ \ \ \ \ \ \ \ \ \ \ \ \ \ \ card\ {\isacharbraceleft}{\kern0pt}i{\isachardot}{\kern0pt}\ i\ {\isacharless}{\kern0pt}\ length\ p\ {\isasymand}\ {\isacharparenleft}{\kern0pt}let\ r\ {\isacharequal}{\kern0pt}\ {\isacharparenleft}{\kern0pt}p{\isacharbang}{\kern0pt}i{\isacharparenright}{\kern0pt}\ in\ {\isacharparenleft}{\kern0pt}x\ {\isasympreceq}\isactrlsub r\ d{\isacharparenright}{\kern0pt}{\isacharparenright}{\kern0pt}{\isacharbraceright}{\kern0pt}{\isacharbraceright}{\kern0pt}{\isacharparenright}{\kern0pt}{\isachardoublequoteclose}\isanewline
\ \ \ \ \isacommand{by}\isamarkupfalse%
\ simp\isanewline
\ \ \isacommand{thus}\isamarkupfalse%
\isanewline
\ \ \ \ {\isachardoublequoteopen}m\ A\ p\ {\isacharequal}{\kern0pt}\isanewline
\ \ \ \ \ \ \ \ {\isacharparenleft}{\kern0pt}{\isacharbraceleft}{\kern0pt}{\isacharbraceright}{\kern0pt}{\isacharcomma}{\kern0pt}\isanewline
\ \ \ \ \ \ \ \ \ \ A\ {\isacharminus}{\kern0pt}\ defer\ m\ A\ p{\isacharcomma}{\kern0pt}\isanewline
\ \ \ \ \ \ \ \ \ \ {\isacharbraceleft}{\kern0pt}d\ {\isasymin}\ A{\isachardot}{\kern0pt}\ {\isasymforall}x{\isasymin}A\ {\isacharminus}{\kern0pt}\ {\isacharbraceleft}{\kern0pt}d{\isacharbraceright}{\kern0pt}{\isachardot}{\kern0pt}\isanewline
\ \ \ \ \ \ \ \ \ \ \ \ card\ {\isacharbraceleft}{\kern0pt}i{\isachardot}{\kern0pt}\ i\ {\isacharless}{\kern0pt}\ length\ p\ {\isasymand}\ {\isacharparenleft}{\kern0pt}d{\isacharcomma}{\kern0pt}\ x{\isacharparenright}{\kern0pt}\ {\isasymin}\ {\isacharparenleft}{\kern0pt}p{\isacharbang}{\kern0pt}i{\isacharparenright}{\kern0pt}{\isacharbraceright}{\kern0pt}\ {\isacharless}{\kern0pt}\isanewline
\ \ \ \ \ \ \ \ \ \ \ \ \ \ card\ {\isacharbraceleft}{\kern0pt}i{\isachardot}{\kern0pt}\ i\ {\isacharless}{\kern0pt}\ length\ p\ {\isasymand}\ {\isacharparenleft}{\kern0pt}x{\isacharcomma}{\kern0pt}\ d{\isacharparenright}{\kern0pt}\ {\isasymin}\ {\isacharparenleft}{\kern0pt}p{\isacharbang}{\kern0pt}i{\isacharparenright}{\kern0pt}{\isacharbraceright}{\kern0pt}{\isacharbraceright}{\kern0pt}{\isacharparenright}{\kern0pt}{\isachardoublequoteclose}\isanewline
\ \ \ \ \isacommand{by}\isamarkupfalse%
\ simp\isanewline
\isacommand{qed}\isamarkupfalse%
%
\endisatagproof
{\isafoldproof}%
%
\isadelimproof
\isanewline
%
\endisadelimproof
\isanewline
\isacommand{lemma}\isamarkupfalse%
\ cr{\isacharunderscore}{\kern0pt}eval{\isacharunderscore}{\kern0pt}imp{\isacharunderscore}{\kern0pt}dcc{\isacharunderscore}{\kern0pt}max{\isacharunderscore}{\kern0pt}elim{\isacharunderscore}{\kern0pt}helper{\isadigit{1}}{\isacharcolon}{\kern0pt}\isanewline
\ \ \isakeyword{assumes}\isanewline
\ \ \ \ f{\isacharunderscore}{\kern0pt}prof{\isacharcolon}{\kern0pt}\ {\isachardoublequoteopen}finite{\isacharunderscore}{\kern0pt}profile\ A\ p{\isachardoublequoteclose}\ \isakeyword{and}\isanewline
\ \ \ \ rating{\isacharcolon}{\kern0pt}\ {\isachardoublequoteopen}condorcet{\isacharunderscore}{\kern0pt}rating\ e{\isachardoublequoteclose}\ \isakeyword{and}\isanewline
\ \ \ \ winner{\isacharcolon}{\kern0pt}\ {\isachardoublequoteopen}condorcet{\isacharunderscore}{\kern0pt}winner\ A\ p\ w{\isachardoublequoteclose}\isanewline
\ \ \isakeyword{shows}\ {\isachardoublequoteopen}elimination{\isacharunderscore}{\kern0pt}set\ e\ {\isacharparenleft}{\kern0pt}Max\ {\isacharbraceleft}{\kern0pt}e\ x\ A\ p\ {\isacharbar}{\kern0pt}\ x{\isachardot}{\kern0pt}\ x\ {\isasymin}\ A{\isacharbraceright}{\kern0pt}{\isacharparenright}{\kern0pt}\ {\isacharparenleft}{\kern0pt}{\isacharless}{\kern0pt}{\isacharparenright}{\kern0pt}\ A\ p\ {\isacharequal}{\kern0pt}\ A\ {\isacharminus}{\kern0pt}\ {\isacharbraceleft}{\kern0pt}w{\isacharbraceright}{\kern0pt}{\isachardoublequoteclose}\isanewline
%
\isadelimproof
%
\endisadelimproof
%
\isatagproof
\isacommand{proof}\isamarkupfalse%
\ {\isacharparenleft}{\kern0pt}safe{\isacharcomma}{\kern0pt}\ simp{\isacharunderscore}{\kern0pt}all{\isacharcomma}{\kern0pt}\ safe{\isacharparenright}{\kern0pt}\isanewline
\ \ \isacommand{assume}\isamarkupfalse%
\isanewline
\ \ \ \ w{\isacharunderscore}{\kern0pt}in{\isacharunderscore}{\kern0pt}A{\isacharcolon}{\kern0pt}\ {\isachardoublequoteopen}w\ {\isasymin}\ A{\isachardoublequoteclose}\ \isakeyword{and}\isanewline
\ \ \ \ max{\isacharcolon}{\kern0pt}\ {\isachardoublequoteopen}e\ w\ A\ p\ {\isacharless}{\kern0pt}\ Max\ {\isacharbraceleft}{\kern0pt}e\ x\ A\ p\ {\isacharbar}{\kern0pt}x{\isachardot}{\kern0pt}\ x\ {\isasymin}\ A{\isacharbraceright}{\kern0pt}{\isachardoublequoteclose}\isanewline
\ \ \isacommand{show}\isamarkupfalse%
\ {\isachardoublequoteopen}False{\isachardoublequoteclose}\isanewline
\ \ \ \ \isacommand{using}\isamarkupfalse%
\ cond{\isacharunderscore}{\kern0pt}winner{\isacharunderscore}{\kern0pt}imp{\isacharunderscore}{\kern0pt}max{\isacharunderscore}{\kern0pt}eval{\isacharunderscore}{\kern0pt}val\isanewline
\ \ \ \ \ \ \ \ \ \ rating\ winner\ f{\isacharunderscore}{\kern0pt}prof\ max\isanewline
\ \ \ \ \isacommand{by}\isamarkupfalse%
\ fastforce\isanewline
\isacommand{next}\isamarkupfalse%
\isanewline
\ \ \isacommand{fix}\isamarkupfalse%
\isanewline
\ \ \ \ x\ {\isacharcolon}{\kern0pt}{\isacharcolon}{\kern0pt}\ {\isachardoublequoteopen}{\isacharprime}{\kern0pt}a{\isachardoublequoteclose}\isanewline
\ \ \isacommand{assume}\isamarkupfalse%
\isanewline
\ \ \ \ x{\isacharunderscore}{\kern0pt}in{\isacharunderscore}{\kern0pt}A{\isacharcolon}{\kern0pt}\ {\isachardoublequoteopen}x\ {\isasymin}\ A{\isachardoublequoteclose}\ \isakeyword{and}\isanewline
\ \ \ \ not{\isacharunderscore}{\kern0pt}max{\isacharcolon}{\kern0pt}\ {\isachardoublequoteopen}{\isasymnot}\ e\ x\ A\ p\ {\isacharless}{\kern0pt}\ Max\ {\isacharbraceleft}{\kern0pt}e\ y\ A\ p\ {\isacharbar}{\kern0pt}y{\isachardot}{\kern0pt}\ y\ {\isasymin}\ A{\isacharbraceright}{\kern0pt}{\isachardoublequoteclose}\isanewline
\ \ \isacommand{show}\isamarkupfalse%
\ {\isachardoublequoteopen}x\ {\isacharequal}{\kern0pt}\ w{\isachardoublequoteclose}\isanewline
\ \ \ \ \isacommand{using}\isamarkupfalse%
\ non{\isacharunderscore}{\kern0pt}cond{\isacharunderscore}{\kern0pt}winner{\isacharunderscore}{\kern0pt}not{\isacharunderscore}{\kern0pt}max{\isacharunderscore}{\kern0pt}eval\ x{\isacharunderscore}{\kern0pt}in{\isacharunderscore}{\kern0pt}A\isanewline
\ \ \ \ \ \ \ \ \ \ rating\ winner\ f{\isacharunderscore}{\kern0pt}prof\ not{\isacharunderscore}{\kern0pt}max\isanewline
\ \ \ \ \isacommand{by}\isamarkupfalse%
\ {\isacharparenleft}{\kern0pt}metis\ {\isacharparenleft}{\kern0pt}mono{\isacharunderscore}{\kern0pt}tags{\isacharcomma}{\kern0pt}\ lifting{\isacharparenright}{\kern0pt}{\isacharparenright}{\kern0pt}\isanewline
\isacommand{qed}\isamarkupfalse%
%
\endisatagproof
{\isafoldproof}%
%
\isadelimproof
\isanewline
%
\endisadelimproof
\isanewline
\isanewline
\isacommand{theorem}\isamarkupfalse%
\ cr{\isacharunderscore}{\kern0pt}eval{\isacharunderscore}{\kern0pt}imp{\isacharunderscore}{\kern0pt}dcc{\isacharunderscore}{\kern0pt}max{\isacharunderscore}{\kern0pt}elim{\isacharbrackleft}{\kern0pt}simp{\isacharbrackright}{\kern0pt}{\isacharcolon}{\kern0pt}\isanewline
\ \ \isakeyword{assumes}\ rating{\isacharcolon}{\kern0pt}\ {\isachardoublequoteopen}condorcet{\isacharunderscore}{\kern0pt}rating\ e{\isachardoublequoteclose}\isanewline
\ \ \isakeyword{shows}\ {\isachardoublequoteopen}defer{\isacharunderscore}{\kern0pt}condorcet{\isacharunderscore}{\kern0pt}consistency\ {\isacharparenleft}{\kern0pt}max{\isacharunderscore}{\kern0pt}eliminator\ e{\isacharparenright}{\kern0pt}{\isachardoublequoteclose}\isanewline
%
\isadelimproof
\ \ %
\endisadelimproof
%
\isatagproof
\isacommand{unfolding}\isamarkupfalse%
\ defer{\isacharunderscore}{\kern0pt}condorcet{\isacharunderscore}{\kern0pt}consistency{\isacharunderscore}{\kern0pt}def\isanewline
\isacommand{proof}\isamarkupfalse%
\ {\isacharparenleft}{\kern0pt}safe{\isacharcomma}{\kern0pt}\ simp{\isacharparenright}{\kern0pt}\isanewline
\ \ \isacommand{fix}\isamarkupfalse%
\isanewline
\ \ \ \ A\ {\isacharcolon}{\kern0pt}{\isacharcolon}{\kern0pt}\ {\isachardoublequoteopen}{\isacharprime}{\kern0pt}a\ set{\isachardoublequoteclose}\ \isakeyword{and}\isanewline
\ \ \ \ p\ {\isacharcolon}{\kern0pt}{\isacharcolon}{\kern0pt}\ {\isachardoublequoteopen}{\isacharprime}{\kern0pt}a\ Profile{\isachardoublequoteclose}\ \isakeyword{and}\isanewline
\ \ \ \ w\ {\isacharcolon}{\kern0pt}{\isacharcolon}{\kern0pt}\ {\isachardoublequoteopen}{\isacharprime}{\kern0pt}a{\isachardoublequoteclose}\isanewline
\ \ \isacommand{assume}\isamarkupfalse%
\isanewline
\ \ \ \ winner{\isacharcolon}{\kern0pt}\ {\isachardoublequoteopen}condorcet{\isacharunderscore}{\kern0pt}winner\ A\ p\ w{\isachardoublequoteclose}\ \isakeyword{and}\isanewline
\ \ \ \ finite{\isacharcolon}{\kern0pt}\ {\isachardoublequoteopen}finite\ A{\isachardoublequoteclose}\isanewline
\ \ \isacommand{let}\isamarkupfalse%
\ {\isacharquery}{\kern0pt}trsh\ {\isacharequal}{\kern0pt}\ {\isachardoublequoteopen}{\isacharparenleft}{\kern0pt}Max\ {\isacharbraceleft}{\kern0pt}e\ y\ A\ p\ {\isacharbar}{\kern0pt}\ y{\isachardot}{\kern0pt}\ y\ {\isasymin}\ A{\isacharbraceright}{\kern0pt}{\isacharparenright}{\kern0pt}{\isachardoublequoteclose}\isanewline
\ \ \isacommand{show}\isamarkupfalse%
\isanewline
\ \ \ \ {\isachardoublequoteopen}max{\isacharunderscore}{\kern0pt}eliminator\ e\ A\ p\ {\isacharequal}{\kern0pt}\isanewline
\ \ \ \ \ \ {\isacharparenleft}{\kern0pt}{\isacharbraceleft}{\kern0pt}{\isacharbraceright}{\kern0pt}{\isacharcomma}{\kern0pt}\isanewline
\ \ \ \ \ \ \ \ A\ {\isacharminus}{\kern0pt}\ defer\ {\isacharparenleft}{\kern0pt}max{\isacharunderscore}{\kern0pt}eliminator\ e{\isacharparenright}{\kern0pt}\ A\ p{\isacharcomma}{\kern0pt}\isanewline
\ \ \ \ \ \ \ \ {\isacharbraceleft}{\kern0pt}a\ {\isasymin}\ A{\isachardot}{\kern0pt}\ condorcet{\isacharunderscore}{\kern0pt}winner\ A\ p\ a{\isacharbraceright}{\kern0pt}{\isacharparenright}{\kern0pt}{\isachardoublequoteclose}\isanewline
\ \ \isacommand{proof}\isamarkupfalse%
\ {\isacharparenleft}{\kern0pt}cases\ {\isachardoublequoteopen}elimination{\isacharunderscore}{\kern0pt}set\ e\ {\isacharparenleft}{\kern0pt}{\isacharquery}{\kern0pt}trsh{\isacharparenright}{\kern0pt}\ {\isacharparenleft}{\kern0pt}{\isacharless}{\kern0pt}{\isacharparenright}{\kern0pt}\ A\ p\ {\isasymnoteq}\ A{\isachardoublequoteclose}{\isacharparenright}{\kern0pt}\isanewline
\ \ \ \ \isacommand{case}\isamarkupfalse%
\ True\isanewline
\ \ \ \ \isacommand{have}\isamarkupfalse%
\ profile{\isacharcolon}{\kern0pt}\ {\isachardoublequoteopen}finite{\isacharunderscore}{\kern0pt}profile\ A\ p{\isachardoublequoteclose}\isanewline
\ \ \ \ \ \ \isacommand{using}\isamarkupfalse%
\ winner\isanewline
\ \ \ \ \ \ \isacommand{by}\isamarkupfalse%
\ simp\isanewline
\ \ \ \ \isacommand{with}\isamarkupfalse%
\ rating\ winner\ \isacommand{have}\isamarkupfalse%
\ {\isadigit{0}}{\isacharcolon}{\kern0pt}\isanewline
\ \ \ \ \ \ {\isachardoublequoteopen}{\isacharparenleft}{\kern0pt}elimination{\isacharunderscore}{\kern0pt}set\ e\ {\isacharquery}{\kern0pt}trsh\ {\isacharparenleft}{\kern0pt}{\isacharless}{\kern0pt}{\isacharparenright}{\kern0pt}\ A\ p{\isacharparenright}{\kern0pt}\ {\isacharequal}{\kern0pt}\ A\ {\isacharminus}{\kern0pt}\ {\isacharbraceleft}{\kern0pt}w{\isacharbraceright}{\kern0pt}{\isachardoublequoteclose}\isanewline
\ \ \ \ \ \ \isacommand{using}\isamarkupfalse%
\ cr{\isacharunderscore}{\kern0pt}eval{\isacharunderscore}{\kern0pt}imp{\isacharunderscore}{\kern0pt}dcc{\isacharunderscore}{\kern0pt}max{\isacharunderscore}{\kern0pt}elim{\isacharunderscore}{\kern0pt}helper{\isadigit{1}}\isanewline
\ \ \ \ \ \ \isacommand{by}\isamarkupfalse%
\ {\isacharparenleft}{\kern0pt}metis\ {\isacharparenleft}{\kern0pt}mono{\isacharunderscore}{\kern0pt}tags{\isacharcomma}{\kern0pt}\ lifting{\isacharparenright}{\kern0pt}{\isacharparenright}{\kern0pt}\isanewline
\ \ \ \ \isacommand{have}\isamarkupfalse%
\isanewline
\ \ \ \ \ \ {\isachardoublequoteopen}max{\isacharunderscore}{\kern0pt}eliminator\ e\ A\ p\ {\isacharequal}{\kern0pt}\isanewline
\ \ \ \ \ \ \ \ {\isacharparenleft}{\kern0pt}{\isacharbraceleft}{\kern0pt}{\isacharbraceright}{\kern0pt}{\isacharcomma}{\kern0pt}\isanewline
\ \ \ \ \ \ \ \ \ \ {\isacharparenleft}{\kern0pt}elimination{\isacharunderscore}{\kern0pt}set\ e\ {\isacharquery}{\kern0pt}trsh\ {\isacharparenleft}{\kern0pt}{\isacharless}{\kern0pt}{\isacharparenright}{\kern0pt}\ A\ p{\isacharparenright}{\kern0pt}{\isacharcomma}{\kern0pt}\isanewline
\ \ \ \ \ \ \ \ \ \ A\ {\isacharminus}{\kern0pt}\ {\isacharparenleft}{\kern0pt}elimination{\isacharunderscore}{\kern0pt}set\ e\ {\isacharquery}{\kern0pt}trsh\ {\isacharparenleft}{\kern0pt}{\isacharless}{\kern0pt}{\isacharparenright}{\kern0pt}\ A\ p{\isacharparenright}{\kern0pt}{\isacharparenright}{\kern0pt}{\isachardoublequoteclose}\isanewline
\ \ \ \ \ \ \isacommand{using}\isamarkupfalse%
\ True\isanewline
\ \ \ \ \ \ \isacommand{by}\isamarkupfalse%
\ simp\isanewline
\ \ \ \ \isacommand{also}\isamarkupfalse%
\ \isacommand{have}\isamarkupfalse%
\ {\isachardoublequoteopen}{\isachardot}{\kern0pt}{\isachardot}{\kern0pt}{\isachardot}{\kern0pt}\ {\isacharequal}{\kern0pt}\ {\isacharparenleft}{\kern0pt}{\isacharbraceleft}{\kern0pt}{\isacharbraceright}{\kern0pt}{\isacharcomma}{\kern0pt}\ A\ {\isacharminus}{\kern0pt}\ {\isacharbraceleft}{\kern0pt}w{\isacharbraceright}{\kern0pt}{\isacharcomma}{\kern0pt}\ A\ {\isacharminus}{\kern0pt}\ {\isacharparenleft}{\kern0pt}A\ {\isacharminus}{\kern0pt}\ {\isacharbraceleft}{\kern0pt}w{\isacharbraceright}{\kern0pt}{\isacharparenright}{\kern0pt}{\isacharparenright}{\kern0pt}{\isachardoublequoteclose}\isanewline
\ \ \ \ \ \ \isacommand{using}\isamarkupfalse%
\ {\isachardoublequoteopen}{\isadigit{0}}{\isachardoublequoteclose}\isanewline
\ \ \ \ \ \ \isacommand{by}\isamarkupfalse%
\ presburger\isanewline
\ \ \ \ \isacommand{also}\isamarkupfalse%
\ \isacommand{have}\isamarkupfalse%
\ {\isachardoublequoteopen}{\isachardot}{\kern0pt}{\isachardot}{\kern0pt}{\isachardot}{\kern0pt}\ {\isacharequal}{\kern0pt}\ {\isacharparenleft}{\kern0pt}{\isacharbraceleft}{\kern0pt}{\isacharbraceright}{\kern0pt}{\isacharcomma}{\kern0pt}\ A\ {\isacharminus}{\kern0pt}\ {\isacharbraceleft}{\kern0pt}w{\isacharbraceright}{\kern0pt}{\isacharcomma}{\kern0pt}\ {\isacharbraceleft}{\kern0pt}w{\isacharbraceright}{\kern0pt}{\isacharparenright}{\kern0pt}{\isachardoublequoteclose}\isanewline
\ \ \ \ \ \ \isacommand{using}\isamarkupfalse%
\ winner\isanewline
\ \ \ \ \ \ \isacommand{by}\isamarkupfalse%
\ auto\isanewline
\ \ \ \ \isacommand{also}\isamarkupfalse%
\ \isacommand{have}\isamarkupfalse%
\ {\isachardoublequoteopen}{\isachardot}{\kern0pt}{\isachardot}{\kern0pt}{\isachardot}{\kern0pt}\ {\isacharequal}{\kern0pt}\ {\isacharparenleft}{\kern0pt}{\isacharbraceleft}{\kern0pt}{\isacharbraceright}{\kern0pt}{\isacharcomma}{\kern0pt}A\ {\isacharminus}{\kern0pt}\ defer\ {\isacharparenleft}{\kern0pt}max{\isacharunderscore}{\kern0pt}eliminator\ e{\isacharparenright}{\kern0pt}\ A\ p{\isacharcomma}{\kern0pt}\ {\isacharbraceleft}{\kern0pt}w{\isacharbraceright}{\kern0pt}{\isacharparenright}{\kern0pt}{\isachardoublequoteclose}\isanewline
\ \ \ \ \ \ \isacommand{using}\isamarkupfalse%
\ calculation\isanewline
\ \ \ \ \ \ \isacommand{by}\isamarkupfalse%
\ auto\isanewline
\ \ \ \ \isacommand{also}\isamarkupfalse%
\ \isacommand{have}\isamarkupfalse%
\isanewline
\ \ \ \ \ \ {\isachardoublequoteopen}{\isachardot}{\kern0pt}{\isachardot}{\kern0pt}{\isachardot}{\kern0pt}\ {\isacharequal}{\kern0pt}\isanewline
\ \ \ \ \ \ \ \ {\isacharparenleft}{\kern0pt}{\isacharbraceleft}{\kern0pt}{\isacharbraceright}{\kern0pt}{\isacharcomma}{\kern0pt}\isanewline
\ \ \ \ \ \ \ \ \ \ A\ {\isacharminus}{\kern0pt}\ defer\ {\isacharparenleft}{\kern0pt}max{\isacharunderscore}{\kern0pt}eliminator\ e{\isacharparenright}{\kern0pt}\ A\ p{\isacharcomma}{\kern0pt}\isanewline
\ \ \ \ \ \ \ \ \ \ {\isacharbraceleft}{\kern0pt}d\ {\isasymin}\ A{\isachardot}{\kern0pt}\ condorcet{\isacharunderscore}{\kern0pt}winner\ A\ p\ d{\isacharbraceright}{\kern0pt}{\isacharparenright}{\kern0pt}{\isachardoublequoteclose}\isanewline
\ \ \ \ \ \ \isacommand{using}\isamarkupfalse%
\ cond{\isacharunderscore}{\kern0pt}winner{\isacharunderscore}{\kern0pt}unique{\isadigit{3}}\ winner\ Collect{\isacharunderscore}{\kern0pt}cong\isanewline
\ \ \ \ \ \ \isacommand{by}\isamarkupfalse%
\ {\isacharparenleft}{\kern0pt}metis\ {\isacharparenleft}{\kern0pt}no{\isacharunderscore}{\kern0pt}types{\isacharcomma}{\kern0pt}\ lifting{\isacharparenright}{\kern0pt}{\isacharparenright}{\kern0pt}\isanewline
\ \ \ \ \isacommand{finally}\isamarkupfalse%
\ \isacommand{show}\isamarkupfalse%
\ {\isacharquery}{\kern0pt}thesis\isanewline
\ \ \ \ \ \ \isacommand{using}\isamarkupfalse%
\ finite\ winner\isanewline
\ \ \ \ \ \ \isacommand{by}\isamarkupfalse%
\ metis\isanewline
\ \ \isacommand{next}\isamarkupfalse%
\isanewline
\ \ \ \ \isacommand{case}\isamarkupfalse%
\ False\isanewline
\ \ \ \ \isacommand{thus}\isamarkupfalse%
\ {\isacharquery}{\kern0pt}thesis\isanewline
\ \ \ \ \isacommand{proof}\isamarkupfalse%
\ {\isacharminus}{\kern0pt}\isanewline
\ \ \ \ \ \ \isacommand{have}\isamarkupfalse%
\ f{\isadigit{1}}{\isacharcolon}{\kern0pt}\isanewline
\ \ \ \ \ \ \ \ {\isachardoublequoteopen}finite\ A\ {\isasymand}\ profile\ A\ p\ {\isasymand}\ w\ {\isasymin}\ A\ {\isasymand}\ {\isacharparenleft}{\kern0pt}{\isasymforall}a{\isachardot}{\kern0pt}\ a\ {\isasymnotin}\ A\ {\isacharminus}{\kern0pt}\ {\isacharbraceleft}{\kern0pt}w{\isacharbraceright}{\kern0pt}\ {\isasymor}\ wins\ w\ p\ a{\isacharparenright}{\kern0pt}{\isachardoublequoteclose}\isanewline
\ \ \ \ \ \ \ \ \isacommand{using}\isamarkupfalse%
\ winner\isanewline
\ \ \ \ \ \ \ \ \isacommand{by}\isamarkupfalse%
\ auto\isanewline
\ \ \ \ \ \ \isacommand{hence}\isamarkupfalse%
\isanewline
\ \ \ \ \ \ \ \ {\isachardoublequoteopen}{\isacharquery}{\kern0pt}trsh\ {\isacharequal}{\kern0pt}\ e\ w\ A\ p{\isachardoublequoteclose}\isanewline
\ \ \ \ \ \ \ \ \isacommand{using}\isamarkupfalse%
\ rating\ winner\isanewline
\ \ \ \ \ \ \ \ \isacommand{by}\isamarkupfalse%
\ {\isacharparenleft}{\kern0pt}simp\ add{\isacharcolon}{\kern0pt}\ cond{\isacharunderscore}{\kern0pt}winner{\isacharunderscore}{\kern0pt}imp{\isacharunderscore}{\kern0pt}max{\isacharunderscore}{\kern0pt}eval{\isacharunderscore}{\kern0pt}val{\isacharparenright}{\kern0pt}\isanewline
\ \ \ \ \ \ \isacommand{hence}\isamarkupfalse%
\ False\isanewline
\ \ \ \ \ \ \ \ \isacommand{using}\isamarkupfalse%
\ f{\isadigit{1}}\ False\isanewline
\ \ \ \ \ \ \ \ \isacommand{by}\isamarkupfalse%
\ auto\isanewline
\ \ \ \ \ \ \isacommand{thus}\isamarkupfalse%
\ {\isacharquery}{\kern0pt}thesis\isanewline
\ \ \ \ \ \ \ \ \isacommand{by}\isamarkupfalse%
\ simp\isanewline
\ \ \ \ \isacommand{qed}\isamarkupfalse%
\isanewline
\ \ \isacommand{qed}\isamarkupfalse%
\isanewline
\isacommand{qed}\isamarkupfalse%
%
\endisatagproof
{\isafoldproof}%
%
\isadelimproof
\isanewline
%
\endisadelimproof
\isanewline
\isacommand{lemma}\isamarkupfalse%
\ condorcet{\isacharunderscore}{\kern0pt}consistency{\isadigit{2}}{\isacharcolon}{\kern0pt}\isanewline
\ \ {\isachardoublequoteopen}condorcet{\isacharunderscore}{\kern0pt}consistency\ m\ {\isasymlongleftrightarrow}\isanewline
\ \ \ \ \ \ electoral{\isacharunderscore}{\kern0pt}module\ m\ {\isasymand}\isanewline
\ \ \ \ \ \ \ \ {\isacharparenleft}{\kern0pt}{\isasymforall}\ A\ p\ w{\isachardot}{\kern0pt}\ condorcet{\isacharunderscore}{\kern0pt}winner\ A\ p\ w\ {\isasymlongrightarrow}\isanewline
\ \ \ \ \ \ \ \ \ \ \ \ {\isacharparenleft}{\kern0pt}m\ A\ p\ {\isacharequal}{\kern0pt}\isanewline
\ \ \ \ \ \ \ \ \ \ \ \ \ \ {\isacharparenleft}{\kern0pt}{\isacharbraceleft}{\kern0pt}w{\isacharbraceright}{\kern0pt}{\isacharcomma}{\kern0pt}\ A\ {\isacharminus}{\kern0pt}\ {\isacharparenleft}{\kern0pt}elect\ m\ A\ p{\isacharparenright}{\kern0pt}{\isacharcomma}{\kern0pt}\ {\isacharbraceleft}{\kern0pt}{\isacharbraceright}{\kern0pt}{\isacharparenright}{\kern0pt}{\isacharparenright}{\kern0pt}{\isacharparenright}{\kern0pt}{\isachardoublequoteclose}\isanewline
%
\isadelimproof
%
\endisadelimproof
%
\isatagproof
\isacommand{proof}\isamarkupfalse%
\ {\isacharparenleft}{\kern0pt}auto{\isacharparenright}{\kern0pt}\isanewline
\ \ \isacommand{show}\isamarkupfalse%
\ {\isachardoublequoteopen}condorcet{\isacharunderscore}{\kern0pt}consistency\ m\ {\isasymLongrightarrow}\ electoral{\isacharunderscore}{\kern0pt}module\ m{\isachardoublequoteclose}\isanewline
\ \ \ \ \isacommand{using}\isamarkupfalse%
\ condorcet{\isacharunderscore}{\kern0pt}consistency{\isacharunderscore}{\kern0pt}def\isanewline
\ \ \ \ \isacommand{by}\isamarkupfalse%
\ metis\isanewline
\isacommand{next}\isamarkupfalse%
\isanewline
\ \ \isacommand{fix}\isamarkupfalse%
\isanewline
\ \ \ \ A\ {\isacharcolon}{\kern0pt}{\isacharcolon}{\kern0pt}\ {\isachardoublequoteopen}{\isacharprime}{\kern0pt}a\ set{\isachardoublequoteclose}\ \isakeyword{and}\isanewline
\ \ \ \ p\ {\isacharcolon}{\kern0pt}{\isacharcolon}{\kern0pt}\ {\isachardoublequoteopen}{\isacharprime}{\kern0pt}a\ Profile{\isachardoublequoteclose}\ \isakeyword{and}\isanewline
\ \ \ \ w\ {\isacharcolon}{\kern0pt}{\isacharcolon}{\kern0pt}\ {\isachardoublequoteopen}{\isacharprime}{\kern0pt}a{\isachardoublequoteclose}\isanewline
\ \ \isacommand{assume}\isamarkupfalse%
\isanewline
\ \ \ \ cc{\isacharcolon}{\kern0pt}\ {\isachardoublequoteopen}condorcet{\isacharunderscore}{\kern0pt}consistency\ m{\isachardoublequoteclose}\isanewline
\ \ \isacommand{have}\isamarkupfalse%
\ assm{\isadigit{0}}{\isacharcolon}{\kern0pt}\isanewline
\ \ \ \ {\isachardoublequoteopen}condorcet{\isacharunderscore}{\kern0pt}winner\ A\ p\ w\ {\isasymLongrightarrow}\ m\ A\ p\ {\isacharequal}{\kern0pt}\ {\isacharparenleft}{\kern0pt}{\isacharbraceleft}{\kern0pt}w{\isacharbraceright}{\kern0pt}{\isacharcomma}{\kern0pt}\ A\ {\isacharminus}{\kern0pt}\ elect\ m\ A\ p{\isacharcomma}{\kern0pt}\ {\isacharbraceleft}{\kern0pt}{\isacharbraceright}{\kern0pt}{\isacharparenright}{\kern0pt}{\isachardoublequoteclose}\isanewline
\ \ \ \ \isacommand{using}\isamarkupfalse%
\ cond{\isacharunderscore}{\kern0pt}winner{\isacharunderscore}{\kern0pt}unique{\isadigit{3}}\ condorcet{\isacharunderscore}{\kern0pt}consistency{\isacharunderscore}{\kern0pt}def\ cc\isanewline
\ \ \ \ \isacommand{by}\isamarkupfalse%
\ {\isacharparenleft}{\kern0pt}metis\ {\isacharparenleft}{\kern0pt}mono{\isacharunderscore}{\kern0pt}tags{\isacharcomma}{\kern0pt}\ lifting{\isacharparenright}{\kern0pt}{\isacharparenright}{\kern0pt}\isanewline
\ \ \isacommand{assume}\isamarkupfalse%
\isanewline
\ \ \ \ finite{\isacharunderscore}{\kern0pt}A{\isacharcolon}{\kern0pt}\ {\isachardoublequoteopen}finite\ A{\isachardoublequoteclose}\ \isakeyword{and}\isanewline
\ \ \ \ prof{\isacharunderscore}{\kern0pt}A{\isacharcolon}{\kern0pt}\ {\isachardoublequoteopen}profile\ A\ p{\isachardoublequoteclose}\ \isakeyword{and}\isanewline
\ \ \ \ w{\isacharunderscore}{\kern0pt}in{\isacharunderscore}{\kern0pt}A{\isacharcolon}{\kern0pt}\ {\isachardoublequoteopen}w\ {\isasymin}\ A{\isachardoublequoteclose}\isanewline
\ \ \isacommand{also}\isamarkupfalse%
\ \isacommand{have}\isamarkupfalse%
\isanewline
\ \ \ \ {\isachardoublequoteopen}{\isasymforall}x{\isasymin}A\ {\isacharminus}{\kern0pt}\ {\isacharbraceleft}{\kern0pt}w{\isacharbraceright}{\kern0pt}{\isachardot}{\kern0pt}\isanewline
\ \ \ \ \ \ prefer{\isacharunderscore}{\kern0pt}count\ p\ w\ x\ {\isachargreater}{\kern0pt}\ prefer{\isacharunderscore}{\kern0pt}count\ p\ x\ w\ {\isasymLongrightarrow}\isanewline
\ \ \ \ \ \ \ \ condorcet{\isacharunderscore}{\kern0pt}winner\ A\ p\ w{\isachardoublequoteclose}\isanewline
\ \ \ \ \isacommand{using}\isamarkupfalse%
\ finite{\isacharunderscore}{\kern0pt}A\ prof{\isacharunderscore}{\kern0pt}A\ w{\isacharunderscore}{\kern0pt}in{\isacharunderscore}{\kern0pt}A\ wins{\isachardot}{\kern0pt}elims\isanewline
\ \ \ \ \isacommand{by}\isamarkupfalse%
\ simp\isanewline
\ \ \isacommand{ultimately}\isamarkupfalse%
\ \isacommand{show}\isamarkupfalse%
\isanewline
\ \ \ \ {\isachardoublequoteopen}{\isasymforall}x{\isasymin}A\ {\isacharminus}{\kern0pt}\ {\isacharbraceleft}{\kern0pt}w{\isacharbraceright}{\kern0pt}{\isachardot}{\kern0pt}\isanewline
\ \ \ \ \ \ \ \ card\ {\isacharbraceleft}{\kern0pt}i{\isachardot}{\kern0pt}\ i\ {\isacharless}{\kern0pt}\ length\ p\ {\isasymand}\ {\isacharparenleft}{\kern0pt}w{\isacharcomma}{\kern0pt}\ x{\isacharparenright}{\kern0pt}\ {\isasymin}\ {\isacharparenleft}{\kern0pt}p{\isacharbang}{\kern0pt}i{\isacharparenright}{\kern0pt}{\isacharbraceright}{\kern0pt}\ {\isacharless}{\kern0pt}\isanewline
\ \ \ \ \ \ \ \ \ \ \ \ card\ {\isacharbraceleft}{\kern0pt}i{\isachardot}{\kern0pt}\ i\ {\isacharless}{\kern0pt}\ length\ p\ {\isasymand}\ {\isacharparenleft}{\kern0pt}x{\isacharcomma}{\kern0pt}\ w{\isacharparenright}{\kern0pt}\ {\isasymin}\ {\isacharparenleft}{\kern0pt}p{\isacharbang}{\kern0pt}i{\isacharparenright}{\kern0pt}{\isacharbraceright}{\kern0pt}\ {\isasymLongrightarrow}\isanewline
\ \ \ \ \ \ \ \ \ \ \ \ \ \ \ \ m\ A\ p\ {\isacharequal}{\kern0pt}\ {\isacharparenleft}{\kern0pt}{\isacharbraceleft}{\kern0pt}w{\isacharbraceright}{\kern0pt}{\isacharcomma}{\kern0pt}\ A\ {\isacharminus}{\kern0pt}\ elect\ m\ A\ p{\isacharcomma}{\kern0pt}\ {\isacharbraceleft}{\kern0pt}{\isacharbraceright}{\kern0pt}{\isacharparenright}{\kern0pt}{\isachardoublequoteclose}\isanewline
\ \ \ \ \isacommand{using}\isamarkupfalse%
\ assm{\isadigit{0}}\isanewline
\ \ \ \ \isacommand{by}\isamarkupfalse%
\ auto\isanewline
\isacommand{next}\isamarkupfalse%
\isanewline
\ \ \isacommand{have}\isamarkupfalse%
\ assm{\isadigit{0}}{\isacharcolon}{\kern0pt}\isanewline
\ \ \ \ {\isachardoublequoteopen}electoral{\isacharunderscore}{\kern0pt}module\ m\ {\isasymLongrightarrow}\isanewline
\ \ \ \ \ \ {\isasymforall}A\ p\ w{\isachardot}{\kern0pt}\ condorcet{\isacharunderscore}{\kern0pt}winner\ A\ p\ w\ {\isasymlongrightarrow}\isanewline
\ \ \ \ \ \ \ \ \ \ m\ A\ p\ {\isacharequal}{\kern0pt}\ {\isacharparenleft}{\kern0pt}{\isacharbraceleft}{\kern0pt}w{\isacharbraceright}{\kern0pt}{\isacharcomma}{\kern0pt}\ A\ {\isacharminus}{\kern0pt}\ elect\ m\ A\ p{\isacharcomma}{\kern0pt}\ {\isacharbraceleft}{\kern0pt}{\isacharbraceright}{\kern0pt}{\isacharparenright}{\kern0pt}\ {\isasymLongrightarrow}\isanewline
\ \ \ \ \ \ \ \ \ \ \ \ condorcet{\isacharunderscore}{\kern0pt}consistency\ m{\isachardoublequoteclose}\isanewline
\ \ \ \ \isacommand{using}\isamarkupfalse%
\ condorcet{\isacharunderscore}{\kern0pt}consistency{\isacharunderscore}{\kern0pt}def\ cond{\isacharunderscore}{\kern0pt}winner{\isacharunderscore}{\kern0pt}unique{\isadigit{3}}\isanewline
\ \ \ \ \isacommand{by}\isamarkupfalse%
\ {\isacharparenleft}{\kern0pt}smt\ {\isacharparenleft}{\kern0pt}verit{\isacharcomma}{\kern0pt}\ del{\isacharunderscore}{\kern0pt}insts{\isacharparenright}{\kern0pt}{\isacharparenright}{\kern0pt}\isanewline
\ \ \isacommand{assume}\isamarkupfalse%
\ e{\isacharunderscore}{\kern0pt}mod{\isacharcolon}{\kern0pt}\isanewline
\ \ \ \ {\isachardoublequoteopen}electoral{\isacharunderscore}{\kern0pt}module\ m{\isachardoublequoteclose}\isanewline
\ \ \isacommand{thus}\isamarkupfalse%
\isanewline
\ \ \ \ {\isachardoublequoteopen}{\isasymforall}A\ p\ w{\isachardot}{\kern0pt}\ finite\ A\ {\isasymand}\ profile\ A\ p\ {\isasymand}\ w\ {\isasymin}\ A\ {\isasymand}\isanewline
\ \ \ \ \ \ \ {\isacharparenleft}{\kern0pt}{\isasymforall}x{\isasymin}A\ {\isacharminus}{\kern0pt}\ {\isacharbraceleft}{\kern0pt}w{\isacharbraceright}{\kern0pt}{\isachardot}{\kern0pt}\isanewline
\ \ \ \ \ \ \ \ \ \ card\ {\isacharbraceleft}{\kern0pt}i{\isachardot}{\kern0pt}\ i\ {\isacharless}{\kern0pt}\ length\ p\ {\isasymand}\ {\isacharparenleft}{\kern0pt}w{\isacharcomma}{\kern0pt}\ x{\isacharparenright}{\kern0pt}\ {\isasymin}\ {\isacharparenleft}{\kern0pt}p{\isacharbang}{\kern0pt}i{\isacharparenright}{\kern0pt}{\isacharbraceright}{\kern0pt}\ {\isacharless}{\kern0pt}\isanewline
\ \ \ \ \ \ \ \ \ \ \ \ card\ {\isacharbraceleft}{\kern0pt}i{\isachardot}{\kern0pt}\ i\ {\isacharless}{\kern0pt}\ length\ p\ {\isasymand}\ {\isacharparenleft}{\kern0pt}x{\isacharcomma}{\kern0pt}\ w{\isacharparenright}{\kern0pt}\ {\isasymin}\ {\isacharparenleft}{\kern0pt}p{\isacharbang}{\kern0pt}i{\isacharparenright}{\kern0pt}{\isacharbraceright}{\kern0pt}{\isacharparenright}{\kern0pt}\ {\isasymlongrightarrow}\isanewline
\ \ \ \ \ \ \ m\ A\ p\ {\isacharequal}{\kern0pt}\ {\isacharparenleft}{\kern0pt}{\isacharbraceleft}{\kern0pt}w{\isacharbraceright}{\kern0pt}{\isacharcomma}{\kern0pt}\ A\ {\isacharminus}{\kern0pt}\ elect\ m\ A\ p{\isacharcomma}{\kern0pt}\ {\isacharbraceleft}{\kern0pt}{\isacharbraceright}{\kern0pt}{\isacharparenright}{\kern0pt}\ {\isasymLongrightarrow}\isanewline
\ \ \ \ \ \ \ \ \ \ condorcet{\isacharunderscore}{\kern0pt}consistency\ m{\isachardoublequoteclose}\isanewline
\ \ \ \ \isacommand{using}\isamarkupfalse%
\ assm{\isadigit{0}}\ e{\isacharunderscore}{\kern0pt}mod\isanewline
\ \ \ \ \isacommand{by}\isamarkupfalse%
\ simp\isanewline
\isacommand{qed}\isamarkupfalse%
%
\endisatagproof
{\isafoldproof}%
%
\isadelimproof
\isanewline
%
\endisadelimproof
%
\isadelimtheory
\isanewline
%
\endisadelimtheory
%
\isatagtheory
\isacommand{end}\isamarkupfalse%
%
\endisatagtheory
{\isafoldtheory}%
%
\isadelimtheory
%
\endisadelimtheory
%
\end{isabellebody}%
\endinput
%:%file=~/Documents/Studies/VotingRuleGenerator/virage/src/test/resources/verifiedVotingRuleConstruction/theories/Compositional_Framework/Composition_Rules/Condorcet_Rules.thy%:%
%:%10=1%:%
%:%11=1%:%
%:%12=2%:%
%:%13=3%:%
%:%14=4%:%
%:%15=5%:%
%:%16=6%:%
%:%17=7%:%
%:%18=8%:%
%:%23=8%:%
%:%26=9%:%
%:%27=13%:%
%:%28=14%:%
%:%29=14%:%
%:%30=15%:%
%:%31=16%:%
%:%32=17%:%
%:%33=18%:%
%:%34=19%:%
%:%41=20%:%
%:%42=20%:%
%:%43=21%:%
%:%43=24%:%
%:%44=25%:%
%:%45=25%:%
%:%46=26%:%
%:%47=27%:%
%:%48=28%:%
%:%49=29%:%
%:%50=29%:%
%:%51=29%:%
%:%52=30%:%
%:%53=30%:%
%:%54=31%:%
%:%55=32%:%
%:%56=32%:%
%:%57=33%:%
%:%58=33%:%
%:%59=34%:%
%:%60=34%:%
%:%61=35%:%
%:%62=36%:%
%:%63=36%:%
%:%64=37%:%
%:%65=37%:%
%:%66=38%:%
%:%67=38%:%
%:%68=39%:%
%:%69=40%:%
%:%70=40%:%
%:%71=41%:%
%:%72=41%:%
%:%73=42%:%
%:%74=43%:%
%:%75=43%:%
%:%76=44%:%
%:%77=45%:%
%:%78=45%:%
%:%79=45%:%
%:%80=46%:%
%:%81=47%:%
%:%82=47%:%
%:%83=48%:%
%:%84=48%:%
%:%85=48%:%
%:%86=49%:%
%:%87=49%:%
%:%88=50%:%
%:%94=50%:%
%:%97=51%:%
%:%98=57%:%
%:%99=58%:%
%:%100=58%:%
%:%101=59%:%
%:%102=60%:%
%:%103=61%:%
%:%104=62%:%
%:%105=63%:%
%:%106=64%:%
%:%107=65%:%
%:%114=66%:%
%:%115=66%:%
%:%116=67%:%
%:%117=67%:%
%:%118=68%:%
%:%119=68%:%
%:%120=69%:%
%:%121=69%:%
%:%122=70%:%
%:%123=70%:%
%:%124=70%:%
%:%125=71%:%
%:%126=71%:%
%:%127=72%:%
%:%128=72%:%
%:%129=73%:%
%:%130=73%:%
%:%131=73%:%
%:%132=74%:%
%:%133=74%:%
%:%134=75%:%
%:%140=75%:%
%:%143=76%:%
%:%144=78%:%
%:%145=79%:%
%:%146=79%:%
%:%147=80%:%
%:%148=81%:%
%:%149=82%:%
%:%150=83%:%
%:%151=84%:%
%:%154=85%:%
%:%158=85%:%
%:%159=85%:%
%:%160=86%:%
%:%161=86%:%
%:%162=87%:%
%:%163=87%:%
%:%164=88%:%
%:%166=90%:%
%:%167=91%:%
%:%168=91%:%
%:%169=92%:%
%:%170=92%:%
%:%171=93%:%
%:%172=93%:%
%:%173=94%:%
%:%179=100%:%
%:%180=101%:%
%:%181=101%:%
%:%182=102%:%
%:%188=102%:%
%:%191=103%:%
%:%192=104%:%
%:%193=105%:%
%:%194=105%:%
%:%195=106%:%
%:%196=107%:%
%:%203=108%:%
%:%204=108%:%
%:%205=109%:%
%:%206=110%:%
%:%207=110%:%
%:%208=111%:%
%:%209=111%:%
%:%210=112%:%
%:%211=113%:%
%:%212=113%:%
%:%213=114%:%
%:%214=114%:%
%:%215=115%:%
%:%216=115%:%
%:%217=116%:%
%:%221=120%:%
%:%222=121%:%
%:%223=121%:%
%:%224=122%:%
%:%225=122%:%
%:%226=123%:%
%:%227=124%:%
%:%228=125%:%
%:%229=126%:%
%:%230=127%:%
%:%231=127%:%
%:%232=128%:%
%:%233=129%:%
%:%234=130%:%
%:%235=130%:%
%:%236=131%:%
%:%239=134%:%
%:%240=135%:%
%:%241=135%:%
%:%242=136%:%
%:%243=137%:%
%:%244=137%:%
%:%245=138%:%
%:%246=138%:%
%:%247=139%:%
%:%248=139%:%
%:%249=140%:%
%:%250=140%:%
%:%251=141%:%
%:%252=142%:%
%:%253=143%:%
%:%254=144%:%
%:%255=145%:%
%:%256=146%:%
%:%257=146%:%
%:%258=147%:%
%:%259=148%:%
%:%260=149%:%
%:%261=150%:%
%:%262=151%:%
%:%264=153%:%
%:%265=154%:%
%:%266=154%:%
%:%267=155%:%
%:%268=155%:%
%:%269=156%:%
%:%270=156%:%
%:%271=157%:%
%:%272=157%:%
%:%273=158%:%
%:%274=158%:%
%:%275=159%:%
%:%276=160%:%
%:%277=160%:%
%:%278=161%:%
%:%279=161%:%
%:%280=162%:%
%:%281=162%:%
%:%282=163%:%
%:%283=164%:%
%:%284=165%:%
%:%285=166%:%
%:%286=167%:%
%:%287=167%:%
%:%288=168%:%
%:%289=169%:%
%:%290=170%:%
%:%291=171%:%
%:%293=173%:%
%:%294=174%:%
%:%295=175%:%
%:%296=175%:%
%:%297=176%:%
%:%298=176%:%
%:%299=177%:%
%:%300=177%:%
%:%301=178%:%
%:%302=178%:%
%:%303=179%:%
%:%304=179%:%
%:%305=180%:%
%:%306=181%:%
%:%307=182%:%
%:%308=182%:%
%:%309=183%:%
%:%310=183%:%
%:%311=184%:%
%:%312=184%:%
%:%313=185%:%
%:%314=186%:%
%:%315=187%:%
%:%316=188%:%
%:%317=189%:%
%:%318=190%:%
%:%319=190%:%
%:%320=191%:%
%:%321=192%:%
%:%322=193%:%
%:%323=194%:%
%:%324=195%:%
%:%325=196%:%
%:%327=198%:%
%:%328=199%:%
%:%329=200%:%
%:%330=201%:%
%:%331=201%:%
%:%332=202%:%
%:%333=202%:%
%:%334=203%:%
%:%335=203%:%
%:%336=204%:%
%:%337=204%:%
%:%338=205%:%
%:%339=205%:%
%:%340=206%:%
%:%341=207%:%
%:%342=208%:%
%:%343=208%:%
%:%344=209%:%
%:%345=209%:%
%:%346=210%:%
%:%347=210%:%
%:%348=211%:%
%:%349=212%:%
%:%350=213%:%
%:%351=214%:%
%:%352=215%:%
%:%353=215%:%
%:%354=216%:%
%:%355=217%:%
%:%356=218%:%
%:%357=219%:%
%:%358=220%:%
%:%359=221%:%
%:%361=223%:%
%:%362=224%:%
%:%364=226%:%
%:%365=227%:%
%:%366=227%:%
%:%367=228%:%
%:%368=228%:%
%:%369=229%:%
%:%370=229%:%
%:%371=230%:%
%:%372=230%:%
%:%373=230%:%
%:%374=231%:%
%:%375=231%:%
%:%376=232%:%
%:%377=232%:%
%:%378=233%:%
%:%379=233%:%
%:%380=233%:%
%:%381=234%:%
%:%382=234%:%
%:%383=235%:%
%:%384=236%:%
%:%385=237%:%
%:%386=237%:%
%:%387=238%:%
%:%388=238%:%
%:%389=239%:%
%:%390=239%:%
%:%391=240%:%
%:%392=241%:%
%:%393=242%:%
%:%394=243%:%
%:%395=244%:%
%:%396=244%:%
%:%397=245%:%
%:%398=246%:%
%:%399=247%:%
%:%400=248%:%
%:%401=249%:%
%:%403=251%:%
%:%404=252%:%
%:%405=252%:%
%:%406=253%:%
%:%407=253%:%
%:%408=254%:%
%:%409=254%:%
%:%410=255%:%
%:%411=255%:%
%:%412=256%:%
%:%413=256%:%
%:%414=257%:%
%:%415=258%:%
%:%416=258%:%
%:%417=259%:%
%:%418=259%:%
%:%419=260%:%
%:%420=260%:%
%:%421=261%:%
%:%422=262%:%
%:%423=263%:%
%:%424=264%:%
%:%425=265%:%
%:%426=265%:%
%:%427=266%:%
%:%428=267%:%
%:%429=268%:%
%:%430=269%:%
%:%431=270%:%
%:%432=271%:%
%:%434=273%:%
%:%435=274%:%
%:%436=274%:%
%:%437=275%:%
%:%438=275%:%
%:%439=276%:%
%:%440=276%:%
%:%441=277%:%
%:%442=277%:%
%:%443=278%:%
%:%444=278%:%
%:%445=279%:%
%:%446=280%:%
%:%447=280%:%
%:%448=281%:%
%:%449=281%:%
%:%450=282%:%
%:%451=282%:%
%:%452=283%:%
%:%453=284%:%
%:%454=285%:%
%:%455=286%:%
%:%456=287%:%
%:%457=287%:%
%:%458=288%:%
%:%459=289%:%
%:%460=290%:%
%:%461=291%:%
%:%462=292%:%
%:%463=293%:%
%:%465=295%:%
%:%466=296%:%
%:%467=296%:%
%:%468=297%:%
%:%469=297%:%
%:%470=298%:%
%:%471=298%:%
%:%472=299%:%
%:%473=299%:%
%:%474=300%:%
%:%475=300%:%
%:%476=301%:%
%:%477=302%:%
%:%478=302%:%
%:%479=303%:%
%:%480=303%:%
%:%481=304%:%
%:%482=304%:%
%:%483=305%:%
%:%484=306%:%
%:%485=307%:%
%:%486=308%:%
%:%487=309%:%
%:%488=309%:%
%:%489=310%:%
%:%490=311%:%
%:%491=312%:%
%:%492=313%:%
%:%493=314%:%
%:%494=315%:%
%:%495=316%:%
%:%497=318%:%
%:%498=319%:%
%:%499=319%:%
%:%500=320%:%
%:%501=320%:%
%:%502=321%:%
%:%503=321%:%
%:%504=322%:%
%:%505=322%:%
%:%506=323%:%
%:%507=323%:%
%:%508=324%:%
%:%509=325%:%
%:%510=325%:%
%:%511=326%:%
%:%517=326%:%
%:%520=327%:%
%:%521=328%:%
%:%522=328%:%
%:%523=329%:%
%:%524=330%:%
%:%525=331%:%
%:%526=332%:%
%:%533=333%:%
%:%534=333%:%
%:%535=334%:%
%:%536=334%:%
%:%537=335%:%
%:%538=335%:%
%:%539=335%:%
%:%540=336%:%
%:%541=337%:%
%:%542=337%:%
%:%543=338%:%
%:%544=338%:%
%:%545=339%:%
%:%546=339%:%
%:%547=340%:%
%:%548=341%:%
%:%549=341%:%
%:%550=342%:%
%:%551=342%:%
%:%552=343%:%
%:%553=343%:%
%:%554=344%:%
%:%555=344%:%
%:%556=345%:%
%:%557=346%:%
%:%558=346%:%
%:%559=347%:%
%:%560=347%:%
%:%561=348%:%
%:%562=348%:%
%:%563=349%:%
%:%564=350%:%
%:%565=350%:%
%:%566=351%:%
%:%567=351%:%
%:%568=351%:%
%:%569=352%:%
%:%570=352%:%
%:%571=353%:%
%:%572=353%:%
%:%573=354%:%
%:%574=354%:%
%:%575=355%:%
%:%576=355%:%
%:%577=356%:%
%:%578=356%:%
%:%579=357%:%
%:%580=357%:%
%:%581=358%:%
%:%582=358%:%
%:%583=359%:%
%:%584=359%:%
%:%585=360%:%
%:%586=360%:%
%:%587=361%:%
%:%588=361%:%
%:%589=361%:%
%:%590=362%:%
%:%591=363%:%
%:%592=363%:%
%:%593=364%:%
%:%594=365%:%
%:%595=365%:%
%:%596=366%:%
%:%597=366%:%
%:%598=366%:%
%:%599=367%:%
%:%600=367%:%
%:%601=368%:%
%:%602=368%:%
%:%603=369%:%
%:%604=369%:%
%:%605=370%:%
%:%606=370%:%
%:%607=371%:%
%:%613=371%:%
%:%616=372%:%
%:%617=373%:%
%:%618=373%:%
%:%619=374%:%
%:%620=375%:%
%:%621=376%:%
%:%628=377%:%
%:%629=377%:%
%:%630=378%:%
%:%631=378%:%
%:%632=379%:%
%:%633=379%:%
%:%634=380%:%
%:%635=380%:%
%:%636=381%:%
%:%637=381%:%
%:%638=382%:%
%:%639=382%:%
%:%640=383%:%
%:%641=384%:%
%:%642=385%:%
%:%643=386%:%
%:%644=386%:%
%:%645=387%:%
%:%646=388%:%
%:%647=389%:%
%:%648=390%:%
%:%650=392%:%
%:%651=393%:%
%:%652=393%:%
%:%653=394%:%
%:%654=394%:%
%:%655=395%:%
%:%656=395%:%
%:%657=396%:%
%:%658=396%:%
%:%659=397%:%
%:%662=400%:%
%:%663=401%:%
%:%664=401%:%
%:%665=402%:%
%:%666=403%:%
%:%667=403%:%
%:%668=403%:%
%:%669=404%:%
%:%670=405%:%
%:%671=405%:%
%:%672=406%:%
%:%673=407%:%
%:%674=407%:%
%:%675=408%:%
%:%676=409%:%
%:%677=409%:%
%:%678=409%:%
%:%679=410%:%
%:%680=410%:%
%:%681=411%:%
%:%682=411%:%
%:%683=412%:%
%:%684=413%:%
%:%685=413%:%
%:%686=413%:%
%:%687=414%:%
%:%688=414%:%
%:%689=415%:%
%:%690=416%:%
%:%691=416%:%
%:%692=417%:%
%:%693=417%:%
%:%694=417%:%
%:%695=418%:%
%:%696=418%:%
%:%697=419%:%
%:%698=419%:%
%:%699=420%:%
%:%700=420%:%
%:%701=421%:%
%:%702=421%:%
%:%703=422%:%
%:%704=422%:%
%:%705=423%:%
%:%706=423%:%
%:%707=424%:%
%:%708=424%:%
%:%709=425%:%
%:%710=425%:%
%:%711=426%:%
%:%712=426%:%
%:%713=427%:%
%:%714=427%:%
%:%715=428%:%
%:%718=431%:%
%:%719=432%:%
%:%720=432%:%
%:%721=433%:%
%:%722=433%:%
%:%723=434%:%
%:%724=434%:%
%:%725=435%:%
%:%729=439%:%
%:%730=440%:%
%:%731=440%:%
%:%732=441%:%
%:%733=441%:%
%:%734=442%:%
%:%739=447%:%
%:%740=448%:%
%:%741=448%:%
%:%742=449%:%
%:%743=449%:%
%:%744=450%:%
%:%749=455%:%
%:%750=456%:%
%:%751=456%:%
%:%752=457%:%
%:%758=457%:%
%:%761=458%:%
%:%762=459%:%
%:%763=459%:%
%:%764=460%:%
%:%765=461%:%
%:%766=462%:%
%:%767=463%:%
%:%768=464%:%
%:%775=465%:%
%:%776=465%:%
%:%777=466%:%
%:%778=466%:%
%:%779=467%:%
%:%780=468%:%
%:%781=469%:%
%:%782=469%:%
%:%783=470%:%
%:%784=470%:%
%:%785=471%:%
%:%786=472%:%
%:%787=472%:%
%:%788=473%:%
%:%789=473%:%
%:%790=474%:%
%:%791=474%:%
%:%792=475%:%
%:%793=476%:%
%:%794=476%:%
%:%795=477%:%
%:%796=478%:%
%:%797=479%:%
%:%798=479%:%
%:%799=480%:%
%:%800=480%:%
%:%801=481%:%
%:%802=482%:%
%:%803=482%:%
%:%804=483%:%
%:%810=483%:%
%:%813=484%:%
%:%814=488%:%
%:%815=489%:%
%:%816=489%:%
%:%817=490%:%
%:%818=491%:%
%:%821=492%:%
%:%825=492%:%
%:%826=492%:%
%:%827=493%:%
%:%828=493%:%
%:%829=494%:%
%:%830=494%:%
%:%831=495%:%
%:%832=496%:%
%:%833=497%:%
%:%834=498%:%
%:%835=498%:%
%:%836=499%:%
%:%837=500%:%
%:%838=501%:%
%:%839=501%:%
%:%840=502%:%
%:%841=502%:%
%:%842=503%:%
%:%845=506%:%
%:%846=507%:%
%:%847=507%:%
%:%848=508%:%
%:%849=508%:%
%:%850=509%:%
%:%851=509%:%
%:%852=510%:%
%:%853=510%:%
%:%854=511%:%
%:%855=511%:%
%:%856=512%:%
%:%857=512%:%
%:%858=512%:%
%:%859=513%:%
%:%860=514%:%
%:%861=514%:%
%:%862=515%:%
%:%863=515%:%
%:%864=516%:%
%:%865=516%:%
%:%866=517%:%
%:%869=520%:%
%:%870=521%:%
%:%871=521%:%
%:%872=522%:%
%:%873=522%:%
%:%874=523%:%
%:%875=523%:%
%:%876=523%:%
%:%877=524%:%
%:%878=524%:%
%:%879=525%:%
%:%880=525%:%
%:%881=526%:%
%:%882=526%:%
%:%883=526%:%
%:%884=527%:%
%:%885=527%:%
%:%886=528%:%
%:%887=528%:%
%:%888=529%:%
%:%889=529%:%
%:%890=529%:%
%:%891=530%:%
%:%892=530%:%
%:%893=531%:%
%:%894=531%:%
%:%895=532%:%
%:%896=532%:%
%:%897=532%:%
%:%898=533%:%
%:%901=536%:%
%:%902=537%:%
%:%903=537%:%
%:%904=538%:%
%:%905=538%:%
%:%906=539%:%
%:%907=539%:%
%:%908=539%:%
%:%909=540%:%
%:%910=540%:%
%:%911=541%:%
%:%912=541%:%
%:%913=542%:%
%:%914=542%:%
%:%915=543%:%
%:%916=543%:%
%:%917=544%:%
%:%918=544%:%
%:%919=545%:%
%:%920=545%:%
%:%921=546%:%
%:%922=546%:%
%:%923=547%:%
%:%924=548%:%
%:%925=548%:%
%:%926=549%:%
%:%927=549%:%
%:%928=550%:%
%:%929=550%:%
%:%930=551%:%
%:%931=552%:%
%:%932=552%:%
%:%933=553%:%
%:%934=553%:%
%:%935=554%:%
%:%936=554%:%
%:%937=555%:%
%:%938=555%:%
%:%939=556%:%
%:%940=556%:%
%:%941=557%:%
%:%942=557%:%
%:%943=558%:%
%:%944=558%:%
%:%945=559%:%
%:%946=559%:%
%:%947=560%:%
%:%948=560%:%
%:%949=561%:%
%:%955=561%:%
%:%958=562%:%
%:%959=563%:%
%:%960=563%:%
%:%961=564%:%
%:%965=568%:%
%:%972=569%:%
%:%973=569%:%
%:%974=570%:%
%:%975=570%:%
%:%976=571%:%
%:%977=571%:%
%:%978=572%:%
%:%979=572%:%
%:%980=573%:%
%:%981=573%:%
%:%982=574%:%
%:%983=574%:%
%:%984=575%:%
%:%985=576%:%
%:%986=577%:%
%:%987=578%:%
%:%988=578%:%
%:%989=579%:%
%:%990=580%:%
%:%991=580%:%
%:%992=581%:%
%:%993=582%:%
%:%994=582%:%
%:%995=583%:%
%:%996=583%:%
%:%997=584%:%
%:%998=584%:%
%:%999=585%:%
%:%1000=586%:%
%:%1001=587%:%
%:%1002=588%:%
%:%1003=588%:%
%:%1004=588%:%
%:%1005=589%:%
%:%1007=591%:%
%:%1008=592%:%
%:%1009=592%:%
%:%1010=593%:%
%:%1011=593%:%
%:%1012=594%:%
%:%1013=594%:%
%:%1014=594%:%
%:%1015=595%:%
%:%1018=598%:%
%:%1019=599%:%
%:%1020=599%:%
%:%1021=600%:%
%:%1022=600%:%
%:%1023=601%:%
%:%1024=601%:%
%:%1025=602%:%
%:%1026=602%:%
%:%1027=603%:%
%:%1030=606%:%
%:%1031=607%:%
%:%1032=607%:%
%:%1033=608%:%
%:%1034=608%:%
%:%1035=609%:%
%:%1036=609%:%
%:%1037=610%:%
%:%1038=611%:%
%:%1039=611%:%
%:%1040=612%:%
%:%1045=617%:%
%:%1046=618%:%
%:%1047=618%:%
%:%1048=619%:%
%:%1049=619%:%
%:%1050=620%:%
%:%1056=620%:%
%:%1061=621%:%
%:%1066=622%:%
%
\begin{isabellebody}%
\setisabellecontext{Condorcet{\isacharunderscore}{\kern0pt}Facts}%
%
\isadelimtheory
%
\endisadelimtheory
%
\isatagtheory
\isacommand{theory}\isamarkupfalse%
\ Condorcet{\isacharunderscore}{\kern0pt}Facts\isanewline
\ \ \isakeyword{imports}\ {\isachardoublequoteopen}{\isachardot}{\kern0pt}{\isachardot}{\kern0pt}{\isacharslash}{\kern0pt}Properties{\isacharslash}{\kern0pt}Condorcet{\isacharunderscore}{\kern0pt}Properties{\isachardoublequoteclose}\isanewline
\ \ \ \ \ \ \ \ \ \ {\isachardoublequoteopen}{\isachardot}{\kern0pt}{\isachardot}{\kern0pt}{\isacharslash}{\kern0pt}Components{\isacharslash}{\kern0pt}Composites{\isacharslash}{\kern0pt}Composite{\isacharunderscore}{\kern0pt}Elimination{\isacharunderscore}{\kern0pt}Modules{\isachardoublequoteclose}\isanewline
\ \ \ \ \ \ \ \ \ \ {\isachardoublequoteopen}{\isachardot}{\kern0pt}{\isachardot}{\kern0pt}{\isacharslash}{\kern0pt}{\isachardot}{\kern0pt}{\isachardot}{\kern0pt}{\isacharslash}{\kern0pt}Social{\isacharunderscore}{\kern0pt}Choice{\isacharunderscore}{\kern0pt}Properties{\isacharslash}{\kern0pt}Condorcet{\isacharunderscore}{\kern0pt}Consistency{\isachardoublequoteclose}\isanewline
\ \ \ \ \ \ \ \ \ \ Condorcet{\isacharunderscore}{\kern0pt}Rules\isanewline
\isanewline
\isakeyword{begin}%
\endisatagtheory
{\isafoldtheory}%
%
\isadelimtheory
\isanewline
%
\endisadelimtheory
\isanewline
\isanewline
\isacommand{theorem}\isamarkupfalse%
\ condorcet{\isacharunderscore}{\kern0pt}score{\isacharunderscore}{\kern0pt}is{\isacharunderscore}{\kern0pt}condorcet{\isacharunderscore}{\kern0pt}rating{\isacharcolon}{\kern0pt}\ {\isachardoublequoteopen}condorcet{\isacharunderscore}{\kern0pt}rating\ condorcet{\isacharunderscore}{\kern0pt}score{\isachardoublequoteclose}\isanewline
%
\isadelimproof
%
\endisadelimproof
%
\isatagproof
\isacommand{proof}\isamarkupfalse%
\ {\isacharminus}{\kern0pt}\isanewline
\ \ \isacommand{have}\isamarkupfalse%
\isanewline
\ \ \ \ {\isachardoublequoteopen}{\isasymforall}f{\isachardot}{\kern0pt}\isanewline
\ \ \ \ \ \ {\isacharparenleft}{\kern0pt}{\isasymnot}\ condorcet{\isacharunderscore}{\kern0pt}rating\ f\ {\isasymlongrightarrow}\isanewline
\ \ \ \ \ \ \ \ \ \ {\isacharparenleft}{\kern0pt}{\isasymexists}A\ rs\ a{\isachardot}{\kern0pt}\isanewline
\ \ \ \ \ \ \ \ \ \ \ \ condorcet{\isacharunderscore}{\kern0pt}winner\ A\ rs\ a\ {\isasymand}\isanewline
\ \ \ \ \ \ \ \ \ \ \ \ \ \ {\isacharparenleft}{\kern0pt}{\isasymexists}aa{\isachardot}{\kern0pt}\ {\isasymnot}\ f\ {\isacharparenleft}{\kern0pt}aa{\isacharcolon}{\kern0pt}{\isacharcolon}{\kern0pt}{\isacharprime}{\kern0pt}a{\isacharparenright}{\kern0pt}\ A\ rs\ {\isacharless}{\kern0pt}\ f\ a\ A\ rs\ {\isasymand}\ a\ {\isasymnoteq}\ aa\ {\isasymand}\ aa\ {\isasymin}\ A{\isacharparenright}{\kern0pt}{\isacharparenright}{\kern0pt}{\isacharparenright}{\kern0pt}\ {\isasymand}\isanewline
\ \ \ \ \ \ \ \ {\isacharparenleft}{\kern0pt}condorcet{\isacharunderscore}{\kern0pt}rating\ f\ {\isasymlongrightarrow}\isanewline
\ \ \ \ \ \ \ \ \ \ {\isacharparenleft}{\kern0pt}{\isasymforall}A\ rs\ a{\isachardot}{\kern0pt}\ condorcet{\isacharunderscore}{\kern0pt}winner\ A\ rs\ a\ {\isasymlongrightarrow}\isanewline
\ \ \ \ \ \ \ \ \ \ \ \ {\isacharparenleft}{\kern0pt}{\isasymforall}aa{\isachardot}{\kern0pt}\ f\ aa\ A\ rs\ {\isacharless}{\kern0pt}\ f\ a\ A\ rs\ {\isasymor}\ a\ {\isacharequal}{\kern0pt}\ aa\ {\isasymor}\ aa\ {\isasymnotin}\ A{\isacharparenright}{\kern0pt}{\isacharparenright}{\kern0pt}{\isacharparenright}{\kern0pt}{\isachardoublequoteclose}\isanewline
\ \ \ \ \isacommand{unfolding}\isamarkupfalse%
\ condorcet{\isacharunderscore}{\kern0pt}rating{\isacharunderscore}{\kern0pt}def\isanewline
\ \ \ \ \isacommand{by}\isamarkupfalse%
\ {\isacharparenleft}{\kern0pt}metis\ {\isacharparenleft}{\kern0pt}mono{\isacharunderscore}{\kern0pt}tags{\isacharcomma}{\kern0pt}\ hide{\isacharunderscore}{\kern0pt}lams{\isacharparenright}{\kern0pt}{\isacharparenright}{\kern0pt}\isanewline
\ \ \isacommand{thus}\isamarkupfalse%
\ {\isacharquery}{\kern0pt}thesis\isanewline
\ \ \ \ \isacommand{using}\isamarkupfalse%
\ cond{\isacharunderscore}{\kern0pt}winner{\isacharunderscore}{\kern0pt}unique\ condorcet{\isacharunderscore}{\kern0pt}score{\isachardot}{\kern0pt}simps\ zero{\isacharunderscore}{\kern0pt}less{\isacharunderscore}{\kern0pt}one\isanewline
\ \ \ \ \isacommand{by}\isamarkupfalse%
\ {\isacharparenleft}{\kern0pt}metis\ {\isacharparenleft}{\kern0pt}no{\isacharunderscore}{\kern0pt}types{\isacharparenright}{\kern0pt}{\isacharparenright}{\kern0pt}\isanewline
\isacommand{qed}\isamarkupfalse%
%
\endisatagproof
{\isafoldproof}%
%
\isadelimproof
\isanewline
%
\endisadelimproof
\isanewline
\isanewline
\isacommand{theorem}\isamarkupfalse%
\ copeland{\isacharunderscore}{\kern0pt}score{\isacharunderscore}{\kern0pt}is{\isacharunderscore}{\kern0pt}cr{\isacharcolon}{\kern0pt}\ {\isachardoublequoteopen}condorcet{\isacharunderscore}{\kern0pt}rating\ copeland{\isacharunderscore}{\kern0pt}score{\isachardoublequoteclose}\isanewline
%
\isadelimproof
\ \ %
\endisadelimproof
%
\isatagproof
\isacommand{unfolding}\isamarkupfalse%
\ condorcet{\isacharunderscore}{\kern0pt}rating{\isacharunderscore}{\kern0pt}def\isanewline
\isacommand{proof}\isamarkupfalse%
\ {\isacharparenleft}{\kern0pt}unfold\ copeland{\isacharunderscore}{\kern0pt}score{\isachardot}{\kern0pt}simps{\isacharcomma}{\kern0pt}\ safe{\isacharparenright}{\kern0pt}\isanewline
\ \ \isacommand{fix}\isamarkupfalse%
\isanewline
\ \ \ \ A\ {\isacharcolon}{\kern0pt}{\isacharcolon}{\kern0pt}\ {\isachardoublequoteopen}{\isacharprime}{\kern0pt}a\ set{\isachardoublequoteclose}\ \isakeyword{and}\isanewline
\ \ \ \ p\ {\isacharcolon}{\kern0pt}{\isacharcolon}{\kern0pt}\ {\isachardoublequoteopen}{\isacharprime}{\kern0pt}a\ Profile{\isachardoublequoteclose}\ \isakeyword{and}\isanewline
\ \ \ \ w\ {\isacharcolon}{\kern0pt}{\isacharcolon}{\kern0pt}\ {\isachardoublequoteopen}{\isacharprime}{\kern0pt}a{\isachardoublequoteclose}\ \isakeyword{and}\isanewline
\ \ \ \ l\ {\isacharcolon}{\kern0pt}{\isacharcolon}{\kern0pt}\ {\isachardoublequoteopen}{\isacharprime}{\kern0pt}a{\isachardoublequoteclose}\isanewline
\ \ \isacommand{assume}\isamarkupfalse%
\isanewline
\ \ \ \ winner{\isacharcolon}{\kern0pt}\ {\isachardoublequoteopen}condorcet{\isacharunderscore}{\kern0pt}winner\ A\ p\ w{\isachardoublequoteclose}\ \isakeyword{and}\isanewline
\ \ \ \ l{\isacharunderscore}{\kern0pt}in{\isacharunderscore}{\kern0pt}A{\isacharcolon}{\kern0pt}\ {\isachardoublequoteopen}l\ {\isasymin}\ A{\isachardoublequoteclose}\ \isakeyword{and}\isanewline
\ \ \ \ l{\isacharunderscore}{\kern0pt}neq{\isacharunderscore}{\kern0pt}w{\isacharcolon}{\kern0pt}\ {\isachardoublequoteopen}l\ {\isasymnoteq}\ w{\isachardoublequoteclose}\isanewline
\ \ \isacommand{show}\isamarkupfalse%
\isanewline
\ \ \ \ {\isachardoublequoteopen}card\ {\isacharbraceleft}{\kern0pt}y\ {\isasymin}\ A{\isachardot}{\kern0pt}\ wins\ l\ p\ y{\isacharbraceright}{\kern0pt}\ {\isacharminus}{\kern0pt}\ card\ {\isacharbraceleft}{\kern0pt}y\ {\isasymin}\ A{\isachardot}{\kern0pt}\ wins\ y\ p\ l{\isacharbraceright}{\kern0pt}\isanewline
\ \ \ \ \ \ \ \ {\isacharless}{\kern0pt}\ card\ {\isacharbraceleft}{\kern0pt}y\ {\isasymin}\ A{\isachardot}{\kern0pt}\ wins\ w\ p\ y{\isacharbraceright}{\kern0pt}\ {\isacharminus}{\kern0pt}\ card\ {\isacharbraceleft}{\kern0pt}y\ {\isasymin}\ A{\isachardot}{\kern0pt}\ wins\ y\ p\ w{\isacharbraceright}{\kern0pt}{\isachardoublequoteclose}\isanewline
\ \ \isacommand{proof}\isamarkupfalse%
\ {\isacharminus}{\kern0pt}\isanewline
\ \ \ \ \isacommand{from}\isamarkupfalse%
\ winner\ \isacommand{have}\isamarkupfalse%
\ {\isadigit{0}}{\isacharcolon}{\kern0pt}\isanewline
\ \ \ \ \ \ {\isachardoublequoteopen}card\ {\isacharbraceleft}{\kern0pt}y\ {\isasymin}\ A{\isachardot}{\kern0pt}\ wins\ w\ p\ y{\isacharbraceright}{\kern0pt}\ {\isacharminus}{\kern0pt}\ card\ {\isacharbraceleft}{\kern0pt}y\ {\isasymin}\ A{\isachardot}{\kern0pt}\ wins\ y\ p\ w{\isacharbraceright}{\kern0pt}\ {\isacharequal}{\kern0pt}\isanewline
\ \ \ \ \ \ \ \ card\ A\ {\isacharminus}{\kern0pt}{\isadigit{1}}{\isachardoublequoteclose}\isanewline
\ \ \ \ \ \ \isacommand{using}\isamarkupfalse%
\ cond{\isacharunderscore}{\kern0pt}winner{\isacharunderscore}{\kern0pt}imp{\isacharunderscore}{\kern0pt}copeland{\isacharunderscore}{\kern0pt}score\isanewline
\ \ \ \ \ \ \isacommand{by}\isamarkupfalse%
\ fastforce\isanewline
\ \ \ \ \isacommand{from}\isamarkupfalse%
\ winner\ l{\isacharunderscore}{\kern0pt}neq{\isacharunderscore}{\kern0pt}w\ l{\isacharunderscore}{\kern0pt}in{\isacharunderscore}{\kern0pt}A\ \isacommand{have}\isamarkupfalse%
\ {\isadigit{1}}{\isacharcolon}{\kern0pt}\isanewline
\ \ \ \ \ \ {\isachardoublequoteopen}card\ {\isacharbraceleft}{\kern0pt}y\ {\isasymin}\ A{\isachardot}{\kern0pt}\ wins\ l\ p\ y{\isacharbraceright}{\kern0pt}\ {\isacharminus}{\kern0pt}\ card\ {\isacharbraceleft}{\kern0pt}y\ {\isasymin}\ A{\isachardot}{\kern0pt}\ wins\ y\ p\ l{\isacharbraceright}{\kern0pt}\ {\isasymle}\isanewline
\ \ \ \ \ \ \ \ \ \ card\ A\ {\isacharminus}{\kern0pt}{\isadigit{2}}{\isachardoublequoteclose}\isanewline
\ \ \ \ \ \ \isacommand{using}\isamarkupfalse%
\ non{\isacharunderscore}{\kern0pt}cond{\isacharunderscore}{\kern0pt}winner{\isacharunderscore}{\kern0pt}imp{\isacharunderscore}{\kern0pt}win{\isacharunderscore}{\kern0pt}count\isanewline
\ \ \ \ \ \ \isacommand{by}\isamarkupfalse%
\ fastforce\isanewline
\ \ \ \ \isacommand{have}\isamarkupfalse%
\ {\isadigit{2}}{\isacharcolon}{\kern0pt}\ {\isachardoublequoteopen}card\ A\ {\isacharminus}{\kern0pt}{\isadigit{2}}\ {\isacharless}{\kern0pt}\ card\ A\ {\isacharminus}{\kern0pt}{\isadigit{1}}{\isachardoublequoteclose}\isanewline
\ \ \ \ \ \ \isacommand{using}\isamarkupfalse%
\ card{\isacharunderscore}{\kern0pt}{\isadigit{0}}{\isacharunderscore}{\kern0pt}eq\ card{\isacharunderscore}{\kern0pt}Diff{\isacharunderscore}{\kern0pt}singleton\isanewline
\ \ \ \ \ \ \ \ \ \ \ \ condorcet{\isacharunderscore}{\kern0pt}winner{\isachardot}{\kern0pt}simps\ diff{\isacharunderscore}{\kern0pt}less{\isacharunderscore}{\kern0pt}mono{\isadigit{2}}\isanewline
\ \ \ \ \ \ \ \ \ \ \ \ empty{\isacharunderscore}{\kern0pt}iff\ finite{\isacharunderscore}{\kern0pt}Diff\ insertE\ insert{\isacharunderscore}{\kern0pt}Diff\isanewline
\ \ \ \ \ \ \ \ \ \ \ \ l{\isacharunderscore}{\kern0pt}in{\isacharunderscore}{\kern0pt}A\ l{\isacharunderscore}{\kern0pt}neq{\isacharunderscore}{\kern0pt}w\ neq{\isadigit{0}}{\isacharunderscore}{\kern0pt}conv\ one{\isacharunderscore}{\kern0pt}less{\isacharunderscore}{\kern0pt}numeral{\isacharunderscore}{\kern0pt}iff\isanewline
\ \ \ \ \ \ \ \ \ \ \ \ semiring{\isacharunderscore}{\kern0pt}norm{\isacharparenleft}{\kern0pt}{\isadigit{7}}{\isadigit{6}}{\isacharparenright}{\kern0pt}\ winner\ zero{\isacharunderscore}{\kern0pt}less{\isacharunderscore}{\kern0pt}diff\isanewline
\ \ \ \ \ \ \isacommand{by}\isamarkupfalse%
\ metis\isanewline
\ \ \ \ \isacommand{hence}\isamarkupfalse%
\isanewline
\ \ \ \ \ \ {\isachardoublequoteopen}card\ {\isacharbraceleft}{\kern0pt}y\ {\isasymin}\ A{\isachardot}{\kern0pt}\ wins\ l\ p\ y{\isacharbraceright}{\kern0pt}\ {\isacharminus}{\kern0pt}\ card\ {\isacharbraceleft}{\kern0pt}y\ {\isasymin}\ A{\isachardot}{\kern0pt}\ wins\ y\ p\ l{\isacharbraceright}{\kern0pt}\ {\isacharless}{\kern0pt}\isanewline
\ \ \ \ \ \ \ \ card\ A\ {\isacharminus}{\kern0pt}{\isadigit{1}}{\isachardoublequoteclose}\isanewline
\ \ \ \ \ \ \isacommand{using}\isamarkupfalse%
\ {\isachardoublequoteopen}{\isadigit{1}}{\isachardoublequoteclose}\ le{\isacharunderscore}{\kern0pt}less{\isacharunderscore}{\kern0pt}trans\isanewline
\ \ \ \ \ \ \isacommand{by}\isamarkupfalse%
\ blast\isanewline
\ \ \ \ \isacommand{with}\isamarkupfalse%
\ {\isadigit{0}}\isanewline
\ \ \ \ \isacommand{show}\isamarkupfalse%
\ {\isacharquery}{\kern0pt}thesis\isanewline
\ \ \ \ \ \ \isacommand{by}\isamarkupfalse%
\ linarith\isanewline
\ \ \isacommand{qed}\isamarkupfalse%
\isanewline
\isacommand{qed}\isamarkupfalse%
%
\endisatagproof
{\isafoldproof}%
%
\isadelimproof
\isanewline
%
\endisadelimproof
\isanewline
\isacommand{theorem}\isamarkupfalse%
\ condorcet{\isacharunderscore}{\kern0pt}is{\isacharunderscore}{\kern0pt}dcc{\isacharcolon}{\kern0pt}\ {\isachardoublequoteopen}defer{\isacharunderscore}{\kern0pt}condorcet{\isacharunderscore}{\kern0pt}consistency\ condorcet{\isachardoublequoteclose}\isanewline
%
\isadelimproof
%
\endisadelimproof
%
\isatagproof
\isacommand{proof}\isamarkupfalse%
\ {\isacharminus}{\kern0pt}\isanewline
\ \ \isacommand{have}\isamarkupfalse%
\ max{\isacharunderscore}{\kern0pt}cscore{\isacharunderscore}{\kern0pt}dcc{\isacharcolon}{\kern0pt}\isanewline
\ \ \ \ {\isachardoublequoteopen}defer{\isacharunderscore}{\kern0pt}condorcet{\isacharunderscore}{\kern0pt}consistency\ {\isacharparenleft}{\kern0pt}max{\isacharunderscore}{\kern0pt}eliminator\ condorcet{\isacharunderscore}{\kern0pt}score{\isacharparenright}{\kern0pt}{\isachardoublequoteclose}\isanewline
\ \ \ \ \isacommand{using}\isamarkupfalse%
\ cr{\isacharunderscore}{\kern0pt}eval{\isacharunderscore}{\kern0pt}imp{\isacharunderscore}{\kern0pt}dcc{\isacharunderscore}{\kern0pt}max{\isacharunderscore}{\kern0pt}elim\isanewline
\ \ \ \ \isacommand{by}\isamarkupfalse%
\ {\isacharparenleft}{\kern0pt}simp\ add{\isacharcolon}{\kern0pt}\ condorcet{\isacharunderscore}{\kern0pt}score{\isacharunderscore}{\kern0pt}is{\isacharunderscore}{\kern0pt}condorcet{\isacharunderscore}{\kern0pt}rating{\isacharparenright}{\kern0pt}\isanewline
\ \ \isacommand{have}\isamarkupfalse%
\ cond{\isacharunderscore}{\kern0pt}eq{\isacharunderscore}{\kern0pt}max{\isacharunderscore}{\kern0pt}cond{\isacharcolon}{\kern0pt}\isanewline
\ \ \ \ {\isachardoublequoteopen}{\isasymAnd}A\ p{\isachardot}{\kern0pt}\ {\isacharparenleft}{\kern0pt}condorcet\ A\ p\ {\isasymequiv}\ max{\isacharunderscore}{\kern0pt}eliminator\ condorcet{\isacharunderscore}{\kern0pt}score\ A\ p{\isacharparenright}{\kern0pt}{\isachardoublequoteclose}\isanewline
\ \ \ \ \isacommand{by}\isamarkupfalse%
\ simp\isanewline
\ \ \isacommand{from}\isamarkupfalse%
\ max{\isacharunderscore}{\kern0pt}cscore{\isacharunderscore}{\kern0pt}dcc\ cond{\isacharunderscore}{\kern0pt}eq{\isacharunderscore}{\kern0pt}max{\isacharunderscore}{\kern0pt}cond\ \isacommand{show}\isamarkupfalse%
\ {\isacharquery}{\kern0pt}thesis\isanewline
\ \ \ \ \isacommand{unfolding}\isamarkupfalse%
\ defer{\isacharunderscore}{\kern0pt}condorcet{\isacharunderscore}{\kern0pt}consistency{\isacharunderscore}{\kern0pt}def\ electoral{\isacharunderscore}{\kern0pt}module{\isacharunderscore}{\kern0pt}def\isanewline
\ \ \ \ \isacommand{by}\isamarkupfalse%
\ {\isacharparenleft}{\kern0pt}smt\ {\isacharparenleft}{\kern0pt}verit{\isacharcomma}{\kern0pt}\ ccfv{\isacharunderscore}{\kern0pt}threshold{\isacharparenright}{\kern0pt}{\isacharparenright}{\kern0pt}\isanewline
\isacommand{qed}\isamarkupfalse%
%
\endisatagproof
{\isafoldproof}%
%
\isadelimproof
\isanewline
%
\endisadelimproof
\isanewline
\isacommand{theorem}\isamarkupfalse%
\ copeland{\isacharunderscore}{\kern0pt}is{\isacharunderscore}{\kern0pt}dcc{\isacharcolon}{\kern0pt}\ {\isachardoublequoteopen}defer{\isacharunderscore}{\kern0pt}condorcet{\isacharunderscore}{\kern0pt}consistency\ copeland{\isachardoublequoteclose}\isanewline
%
\isadelimproof
%
\endisadelimproof
%
\isatagproof
\isacommand{proof}\isamarkupfalse%
\ {\isacharminus}{\kern0pt}\isanewline
\ \ \isacommand{have}\isamarkupfalse%
\ max{\isacharunderscore}{\kern0pt}cplscore{\isacharunderscore}{\kern0pt}dcc{\isacharcolon}{\kern0pt}\isanewline
\ \ \ \ {\isachardoublequoteopen}defer{\isacharunderscore}{\kern0pt}condorcet{\isacharunderscore}{\kern0pt}consistency\ {\isacharparenleft}{\kern0pt}max{\isacharunderscore}{\kern0pt}eliminator\ copeland{\isacharunderscore}{\kern0pt}score{\isacharparenright}{\kern0pt}{\isachardoublequoteclose}\isanewline
\ \ \ \ \isacommand{using}\isamarkupfalse%
\ cr{\isacharunderscore}{\kern0pt}eval{\isacharunderscore}{\kern0pt}imp{\isacharunderscore}{\kern0pt}dcc{\isacharunderscore}{\kern0pt}max{\isacharunderscore}{\kern0pt}elim\isanewline
\ \ \ \ \isacommand{by}\isamarkupfalse%
\ {\isacharparenleft}{\kern0pt}simp\ add{\isacharcolon}{\kern0pt}\ copeland{\isacharunderscore}{\kern0pt}score{\isacharunderscore}{\kern0pt}is{\isacharunderscore}{\kern0pt}cr{\isacharparenright}{\kern0pt}\isanewline
\ \ \isacommand{have}\isamarkupfalse%
\ copel{\isacharunderscore}{\kern0pt}eq{\isacharunderscore}{\kern0pt}max{\isacharunderscore}{\kern0pt}copel{\isacharcolon}{\kern0pt}\isanewline
\ \ \ \ {\isachardoublequoteopen}{\isasymAnd}A\ p{\isachardot}{\kern0pt}\ {\isacharparenleft}{\kern0pt}copeland\ A\ p\ {\isasymequiv}\ max{\isacharunderscore}{\kern0pt}eliminator\ copeland{\isacharunderscore}{\kern0pt}score\ A\ p{\isacharparenright}{\kern0pt}{\isachardoublequoteclose}\isanewline
\ \ \ \ \isacommand{by}\isamarkupfalse%
\ simp\isanewline
\ \ \isacommand{from}\isamarkupfalse%
\ max{\isacharunderscore}{\kern0pt}cplscore{\isacharunderscore}{\kern0pt}dcc\ copel{\isacharunderscore}{\kern0pt}eq{\isacharunderscore}{\kern0pt}max{\isacharunderscore}{\kern0pt}copel\isanewline
\ \ \isacommand{show}\isamarkupfalse%
\ {\isacharquery}{\kern0pt}thesis\isanewline
\ \ \ \ \isacommand{unfolding}\isamarkupfalse%
\ defer{\isacharunderscore}{\kern0pt}condorcet{\isacharunderscore}{\kern0pt}consistency{\isacharunderscore}{\kern0pt}def\ electoral{\isacharunderscore}{\kern0pt}module{\isacharunderscore}{\kern0pt}def\isanewline
\ \ \ \ \isacommand{by}\isamarkupfalse%
\ {\isacharparenleft}{\kern0pt}smt\ {\isacharparenleft}{\kern0pt}verit{\isacharcomma}{\kern0pt}\ ccfv{\isacharunderscore}{\kern0pt}threshold{\isacharparenright}{\kern0pt}{\isacharparenright}{\kern0pt}\isanewline
\isacommand{qed}\isamarkupfalse%
%
\endisatagproof
{\isafoldproof}%
%
\isadelimproof
\isanewline
%
\endisadelimproof
\isanewline
\isacommand{theorem}\isamarkupfalse%
\ minimax{\isacharunderscore}{\kern0pt}score{\isacharunderscore}{\kern0pt}cond{\isacharunderscore}{\kern0pt}rating{\isacharcolon}{\kern0pt}\ {\isachardoublequoteopen}condorcet{\isacharunderscore}{\kern0pt}rating\ minimax{\isacharunderscore}{\kern0pt}score{\isachardoublequoteclose}\isanewline
%
\isadelimproof
%
\endisadelimproof
%
\isatagproof
\isacommand{proof}\isamarkupfalse%
\ {\isacharparenleft}{\kern0pt}unfold\ condorcet{\isacharunderscore}{\kern0pt}rating{\isacharunderscore}{\kern0pt}def\ minimax{\isacharunderscore}{\kern0pt}score{\isachardot}{\kern0pt}simps\ prefer{\isacharunderscore}{\kern0pt}count{\isachardot}{\kern0pt}simps{\isacharcomma}{\kern0pt}\ safe{\isacharparenright}{\kern0pt}\isanewline
\ \ \isacommand{fix}\isamarkupfalse%
\isanewline
\ \ \ \ A\ {\isacharcolon}{\kern0pt}{\isacharcolon}{\kern0pt}\ {\isachardoublequoteopen}{\isacharprime}{\kern0pt}a\ set{\isachardoublequoteclose}\ \isakeyword{and}\isanewline
\ \ \ \ p\ {\isacharcolon}{\kern0pt}{\isacharcolon}{\kern0pt}\ {\isachardoublequoteopen}{\isacharprime}{\kern0pt}a\ Profile{\isachardoublequoteclose}\ \isakeyword{and}\isanewline
\ \ \ \ w\ {\isacharcolon}{\kern0pt}{\isacharcolon}{\kern0pt}\ {\isachardoublequoteopen}{\isacharprime}{\kern0pt}a{\isachardoublequoteclose}\ \isakeyword{and}\isanewline
\ \ \ \ l\ {\isacharcolon}{\kern0pt}{\isacharcolon}{\kern0pt}\ {\isachardoublequoteopen}{\isacharprime}{\kern0pt}a{\isachardoublequoteclose}\isanewline
\ \ \isacommand{assume}\isamarkupfalse%
\isanewline
\ \ \ \ winner{\isacharcolon}{\kern0pt}\ {\isachardoublequoteopen}condorcet{\isacharunderscore}{\kern0pt}winner\ A\ p\ w{\isachardoublequoteclose}\ \isakeyword{and}\isanewline
\ \ \ \ l{\isacharunderscore}{\kern0pt}in{\isacharunderscore}{\kern0pt}A{\isacharcolon}{\kern0pt}\ {\isachardoublequoteopen}l\ {\isasymin}\ A{\isachardoublequoteclose}\ \isakeyword{and}\isanewline
\ \ \ \ l{\isacharunderscore}{\kern0pt}neq{\isacharunderscore}{\kern0pt}w{\isacharcolon}{\kern0pt}{\isachardoublequoteopen}l\ {\isasymnoteq}\ w{\isachardoublequoteclose}\isanewline
\ \ \isacommand{show}\isamarkupfalse%
\isanewline
\ \ \ \ {\isachardoublequoteopen}Min\ {\isacharbraceleft}{\kern0pt}card\ {\isacharbraceleft}{\kern0pt}i{\isachardot}{\kern0pt}\ i\ {\isacharless}{\kern0pt}\ length\ p\ {\isasymand}\ {\isacharparenleft}{\kern0pt}let\ r\ {\isacharequal}{\kern0pt}\ {\isacharparenleft}{\kern0pt}p{\isacharbang}{\kern0pt}i{\isacharparenright}{\kern0pt}\ in\ {\isacharparenleft}{\kern0pt}y\ {\isasympreceq}\isactrlsub r\ l{\isacharparenright}{\kern0pt}{\isacharparenright}{\kern0pt}{\isacharbraceright}{\kern0pt}\ {\isacharbar}{\kern0pt}\isanewline
\ \ \ \ \ \ \ \ y{\isachardot}{\kern0pt}\ y\ {\isasymin}\ A\ {\isacharminus}{\kern0pt}\ {\isacharbraceleft}{\kern0pt}l{\isacharbraceright}{\kern0pt}{\isacharbraceright}{\kern0pt}\ {\isacharless}{\kern0pt}\isanewline
\ \ \ \ \ \ Min\ {\isacharbraceleft}{\kern0pt}card\ {\isacharbraceleft}{\kern0pt}i{\isachardot}{\kern0pt}\ i\ {\isacharless}{\kern0pt}\ length\ p\ {\isasymand}\ {\isacharparenleft}{\kern0pt}let\ r\ {\isacharequal}{\kern0pt}\ {\isacharparenleft}{\kern0pt}p{\isacharbang}{\kern0pt}i{\isacharparenright}{\kern0pt}\ in\ {\isacharparenleft}{\kern0pt}y\ {\isasympreceq}\isactrlsub r\ w{\isacharparenright}{\kern0pt}{\isacharparenright}{\kern0pt}{\isacharbraceright}{\kern0pt}\ {\isacharbar}{\kern0pt}\isanewline
\ \ \ \ \ \ \ \ \ \ y{\isachardot}{\kern0pt}\ y\ {\isasymin}\ A\ {\isacharminus}{\kern0pt}\ {\isacharbraceleft}{\kern0pt}w{\isacharbraceright}{\kern0pt}{\isacharbraceright}{\kern0pt}{\isachardoublequoteclose}\isanewline
\ \ \isacommand{proof}\isamarkupfalse%
\ {\isacharparenleft}{\kern0pt}rule\ ccontr{\isacharparenright}{\kern0pt}\isanewline
\ \ \ \ \isacommand{assume}\isamarkupfalse%
\isanewline
\ \ \ \ \ \ {\isachardoublequoteopen}{\isasymnot}\ Min\ {\isacharbraceleft}{\kern0pt}card\ {\isacharbraceleft}{\kern0pt}i{\isachardot}{\kern0pt}\ i\ {\isacharless}{\kern0pt}\ length\ p\ {\isasymand}\ {\isacharparenleft}{\kern0pt}let\ r\ {\isacharequal}{\kern0pt}\ {\isacharparenleft}{\kern0pt}p{\isacharbang}{\kern0pt}i{\isacharparenright}{\kern0pt}\ in\ {\isacharparenleft}{\kern0pt}y\ {\isasympreceq}\isactrlsub r\ l{\isacharparenright}{\kern0pt}{\isacharparenright}{\kern0pt}{\isacharbraceright}{\kern0pt}\ {\isacharbar}{\kern0pt}\isanewline
\ \ \ \ \ \ \ \ \ \ y{\isachardot}{\kern0pt}\ y\ {\isasymin}\ A\ {\isacharminus}{\kern0pt}\ {\isacharbraceleft}{\kern0pt}l{\isacharbraceright}{\kern0pt}{\isacharbraceright}{\kern0pt}\ {\isacharless}{\kern0pt}\isanewline
\ \ \ \ \ \ \ \ Min\ {\isacharbraceleft}{\kern0pt}card\ {\isacharbraceleft}{\kern0pt}i{\isachardot}{\kern0pt}\ i\ {\isacharless}{\kern0pt}\ length\ p\ {\isasymand}\ {\isacharparenleft}{\kern0pt}let\ r\ {\isacharequal}{\kern0pt}\ {\isacharparenleft}{\kern0pt}p{\isacharbang}{\kern0pt}i{\isacharparenright}{\kern0pt}\ in\ {\isacharparenleft}{\kern0pt}y\ {\isasympreceq}\isactrlsub r\ w{\isacharparenright}{\kern0pt}{\isacharparenright}{\kern0pt}{\isacharbraceright}{\kern0pt}\ {\isacharbar}{\kern0pt}\isanewline
\ \ \ \ \ \ \ \ \ \ \ \ y{\isachardot}{\kern0pt}\ y\ {\isasymin}\ A\ {\isacharminus}{\kern0pt}\ {\isacharbraceleft}{\kern0pt}w{\isacharbraceright}{\kern0pt}{\isacharbraceright}{\kern0pt}{\isachardoublequoteclose}\isanewline
\ \ \ \ \isacommand{hence}\isamarkupfalse%
\isanewline
\ \ \ \ \ \ {\isachardoublequoteopen}Min\ {\isacharbraceleft}{\kern0pt}card\ {\isacharbraceleft}{\kern0pt}i{\isachardot}{\kern0pt}\ i\ {\isacharless}{\kern0pt}\ length\ p\ {\isasymand}\ {\isacharparenleft}{\kern0pt}let\ r\ {\isacharequal}{\kern0pt}\ {\isacharparenleft}{\kern0pt}p{\isacharbang}{\kern0pt}i{\isacharparenright}{\kern0pt}\ in\ {\isacharparenleft}{\kern0pt}y\ {\isasympreceq}\isactrlsub r\ l{\isacharparenright}{\kern0pt}{\isacharparenright}{\kern0pt}{\isacharbraceright}{\kern0pt}\ {\isacharbar}{\kern0pt}\isanewline
\ \ \ \ \ \ \ \ \ \ y{\isachardot}{\kern0pt}\ y\ {\isasymin}\ A\ {\isacharminus}{\kern0pt}\ {\isacharbraceleft}{\kern0pt}l{\isacharbraceright}{\kern0pt}{\isacharbraceright}{\kern0pt}\ {\isasymge}\isanewline
\ \ \ \ \ \ \ \ Min\ {\isacharbraceleft}{\kern0pt}card\ {\isacharbraceleft}{\kern0pt}i{\isachardot}{\kern0pt}\ i\ {\isacharless}{\kern0pt}\ length\ p\ {\isasymand}\ {\isacharparenleft}{\kern0pt}let\ r\ {\isacharequal}{\kern0pt}\ {\isacharparenleft}{\kern0pt}p{\isacharbang}{\kern0pt}i{\isacharparenright}{\kern0pt}\ in\ {\isacharparenleft}{\kern0pt}y\ {\isasympreceq}\isactrlsub r\ w{\isacharparenright}{\kern0pt}{\isacharparenright}{\kern0pt}{\isacharbraceright}{\kern0pt}\ {\isacharbar}{\kern0pt}\isanewline
\ \ \ \ \ \ \ \ \ \ \ \ y{\isachardot}{\kern0pt}\ y\ {\isasymin}\ A\ {\isacharminus}{\kern0pt}\ {\isacharbraceleft}{\kern0pt}w{\isacharbraceright}{\kern0pt}{\isacharbraceright}{\kern0pt}{\isachardoublequoteclose}\isanewline
\ \ \ \ \ \ \isacommand{by}\isamarkupfalse%
\ linarith\isanewline
\ \ \ \ \isacommand{hence}\isamarkupfalse%
\ {\isadigit{0}}{\isadigit{0}}{\isadigit{0}}{\isacharcolon}{\kern0pt}\isanewline
\ \ \ \ \ \ {\isachardoublequoteopen}Min\ {\isacharbraceleft}{\kern0pt}prefer{\isacharunderscore}{\kern0pt}count\ p\ l\ y\ {\isacharbar}{\kern0pt}y\ {\isachardot}{\kern0pt}\ y\ {\isasymin}\ A{\isacharminus}{\kern0pt}{\isacharbraceleft}{\kern0pt}l{\isacharbraceright}{\kern0pt}{\isacharbraceright}{\kern0pt}\ {\isasymge}\isanewline
\ \ \ \ \ \ \ \ Min\ {\isacharbraceleft}{\kern0pt}prefer{\isacharunderscore}{\kern0pt}count\ p\ w\ y\ {\isacharbar}{\kern0pt}y\ {\isachardot}{\kern0pt}\ y\ {\isasymin}\ A{\isacharminus}{\kern0pt}{\isacharbraceleft}{\kern0pt}w{\isacharbraceright}{\kern0pt}{\isacharbraceright}{\kern0pt}{\isachardoublequoteclose}\isanewline
\ \ \ \ \ \ \isacommand{by}\isamarkupfalse%
\ auto\isanewline
\ \ \ \ \isacommand{have}\isamarkupfalse%
\ prof{\isacharcolon}{\kern0pt}\ {\isachardoublequoteopen}profile\ A\ p{\isachardoublequoteclose}\isanewline
\ \ \ \ \ \ \isacommand{using}\isamarkupfalse%
\ condorcet{\isacharunderscore}{\kern0pt}winner{\isachardot}{\kern0pt}simps\ winner\isanewline
\ \ \ \ \ \ \isacommand{by}\isamarkupfalse%
\ metis\isanewline
\ \ \ \ \isacommand{from}\isamarkupfalse%
\ prof\ winner\ l{\isacharunderscore}{\kern0pt}in{\isacharunderscore}{\kern0pt}A\ l{\isacharunderscore}{\kern0pt}neq{\isacharunderscore}{\kern0pt}w\isanewline
\ \ \ \ \isacommand{have}\isamarkupfalse%
\ {\isadigit{1}}{\isadigit{0}}{\isadigit{0}}{\isacharcolon}{\kern0pt}\isanewline
\ \ \ \ \ \ {\isachardoublequoteopen}prefer{\isacharunderscore}{\kern0pt}count\ p\ l\ w\ \ {\isasymge}\ Min\ {\isacharbraceleft}{\kern0pt}prefer{\isacharunderscore}{\kern0pt}count\ p\ l\ y\ {\isacharbar}{\kern0pt}y\ {\isachardot}{\kern0pt}\ y\ {\isasymin}\ A{\isacharminus}{\kern0pt}{\isacharbraceleft}{\kern0pt}l{\isacharbraceright}{\kern0pt}{\isacharbraceright}{\kern0pt}{\isachardoublequoteclose}\isanewline
\ \ \ \ \ \ \isacommand{using}\isamarkupfalse%
\ non{\isacharunderscore}{\kern0pt}cond{\isacharunderscore}{\kern0pt}winner{\isacharunderscore}{\kern0pt}minimax{\isacharunderscore}{\kern0pt}score\ minimax{\isacharunderscore}{\kern0pt}score{\isachardot}{\kern0pt}simps\isanewline
\ \ \ \ \ \ \isacommand{by}\isamarkupfalse%
\ metis\isanewline
\isanewline
\ \ \ \ \isacommand{from}\isamarkupfalse%
\ l{\isacharunderscore}{\kern0pt}in{\isacharunderscore}{\kern0pt}A\isanewline
\ \ \ \ \isacommand{have}\isamarkupfalse%
\ l{\isacharunderscore}{\kern0pt}in{\isacharunderscore}{\kern0pt}A{\isacharunderscore}{\kern0pt}without{\isacharunderscore}{\kern0pt}w{\isacharcolon}{\kern0pt}\ {\isachardoublequoteopen}l\ {\isasymin}\ A{\isacharminus}{\kern0pt}{\isacharbraceleft}{\kern0pt}w{\isacharbraceright}{\kern0pt}{\isachardoublequoteclose}\isanewline
\ \ \ \ \ \ \isacommand{by}\isamarkupfalse%
\ {\isacharparenleft}{\kern0pt}simp\ add{\isacharcolon}{\kern0pt}\ l{\isacharunderscore}{\kern0pt}neq{\isacharunderscore}{\kern0pt}w{\isacharparenright}{\kern0pt}\isanewline
\ \ \ \ \isacommand{hence}\isamarkupfalse%
\ {\isadigit{2}}{\isacharcolon}{\kern0pt}\ {\isachardoublequoteopen}{\isacharbraceleft}{\kern0pt}prefer{\isacharunderscore}{\kern0pt}count\ p\ w\ y\ {\isacharbar}{\kern0pt}y\ {\isachardot}{\kern0pt}\ y\ {\isasymin}\ A{\isacharminus}{\kern0pt}{\isacharbraceleft}{\kern0pt}w{\isacharbraceright}{\kern0pt}{\isacharbraceright}{\kern0pt}\ {\isasymnoteq}\ {\isacharbraceleft}{\kern0pt}{\isacharbraceright}{\kern0pt}{\isachardoublequoteclose}\isanewline
\ \ \ \ \ \ \isacommand{by}\isamarkupfalse%
\ blast\isanewline
\ \ \ \ \isacommand{have}\isamarkupfalse%
\ {\isachardoublequoteopen}finite\ {\isacharparenleft}{\kern0pt}A{\isacharminus}{\kern0pt}{\isacharbraceleft}{\kern0pt}w{\isacharbraceright}{\kern0pt}{\isacharparenright}{\kern0pt}{\isachardoublequoteclose}\isanewline
\ \ \ \ \ \ \isacommand{using}\isamarkupfalse%
\ prof\ condorcet{\isacharunderscore}{\kern0pt}winner{\isachardot}{\kern0pt}simps\ winner\ finite{\isacharunderscore}{\kern0pt}Diff\isanewline
\ \ \ \ \ \ \isacommand{by}\isamarkupfalse%
\ metis\isanewline
\ \ \ \ \isacommand{hence}\isamarkupfalse%
\ {\isadigit{3}}{\isacharcolon}{\kern0pt}\ {\isachardoublequoteopen}finite\ {\isacharbraceleft}{\kern0pt}prefer{\isacharunderscore}{\kern0pt}count\ p\ w\ y\ {\isacharbar}{\kern0pt}y\ {\isachardot}{\kern0pt}\ y\ {\isasymin}\ A{\isacharminus}{\kern0pt}{\isacharbraceleft}{\kern0pt}w{\isacharbraceright}{\kern0pt}{\isacharbraceright}{\kern0pt}{\isachardoublequoteclose}\isanewline
\ \ \ \ \ \ \isacommand{by}\isamarkupfalse%
\ simp\isanewline
\ \ \ \ \isacommand{from}\isamarkupfalse%
\ {\isadigit{2}}\ {\isadigit{3}}\isanewline
\ \ \ \ \isacommand{have}\isamarkupfalse%
\ {\isadigit{4}}{\isacharcolon}{\kern0pt}\isanewline
\ \ \ \ \ \ {\isachardoublequoteopen}{\isasymexists}\ n\ {\isasymin}\ A{\isacharminus}{\kern0pt}{\isacharbraceleft}{\kern0pt}w{\isacharbraceright}{\kern0pt}\ {\isachardot}{\kern0pt}\ prefer{\isacharunderscore}{\kern0pt}count\ p\ w\ n\ {\isacharequal}{\kern0pt}\isanewline
\ \ \ \ \ \ \ \ Min\ {\isacharbraceleft}{\kern0pt}prefer{\isacharunderscore}{\kern0pt}count\ p\ w\ y\ {\isacharbar}{\kern0pt}y\ {\isachardot}{\kern0pt}\ y\ {\isasymin}\ A{\isacharminus}{\kern0pt}{\isacharbraceleft}{\kern0pt}w{\isacharbraceright}{\kern0pt}{\isacharbraceright}{\kern0pt}{\isachardoublequoteclose}\isanewline
\ \ \ \ \ \ \isacommand{using}\isamarkupfalse%
\ Min{\isacharunderscore}{\kern0pt}in\isanewline
\ \ \ \ \ \ \isacommand{by}\isamarkupfalse%
\ fastforce\isanewline
\ \ \ \ \isacommand{then}\isamarkupfalse%
\ \isacommand{obtain}\isamarkupfalse%
\ n\ \isakeyword{where}\ {\isadigit{2}}{\isadigit{0}}{\isadigit{0}}{\isacharcolon}{\kern0pt}\isanewline
\ \ \ \ \ \ {\isachardoublequoteopen}prefer{\isacharunderscore}{\kern0pt}count\ p\ w\ n\ {\isacharequal}{\kern0pt}\isanewline
\ \ \ \ \ \ \ \ Min\ {\isacharbraceleft}{\kern0pt}prefer{\isacharunderscore}{\kern0pt}count\ p\ w\ y\ {\isacharbar}{\kern0pt}y\ {\isachardot}{\kern0pt}\ y\ {\isasymin}\ A{\isacharminus}{\kern0pt}{\isacharbraceleft}{\kern0pt}w{\isacharbraceright}{\kern0pt}{\isacharbraceright}{\kern0pt}{\isachardoublequoteclose}\ \isakeyword{and}\isanewline
\ \ \ \ \ \ {\isadigit{6}}{\isacharcolon}{\kern0pt}\ {\isachardoublequoteopen}n\ {\isasymin}\ A{\isacharminus}{\kern0pt}{\isacharbraceleft}{\kern0pt}w{\isacharbraceright}{\kern0pt}{\isachardoublequoteclose}\isanewline
\ \ \ \ \ \ \isacommand{by}\isamarkupfalse%
\ metis\isanewline
\ \ \ \ \isacommand{hence}\isamarkupfalse%
\ n{\isacharunderscore}{\kern0pt}in{\isacharunderscore}{\kern0pt}A{\isacharcolon}{\kern0pt}\ {\isachardoublequoteopen}n\ {\isasymin}\ A{\isachardoublequoteclose}\isanewline
\ \ \ \ \ \ \isacommand{using}\isamarkupfalse%
\ DiffE\isanewline
\ \ \ \ \ \ \isacommand{by}\isamarkupfalse%
\ metis\isanewline
\ \ \ \ \isacommand{from}\isamarkupfalse%
\ {\isadigit{6}}\isanewline
\ \ \ \ \isacommand{have}\isamarkupfalse%
\ n{\isacharunderscore}{\kern0pt}neq{\isacharunderscore}{\kern0pt}w{\isacharcolon}{\kern0pt}\ {\isachardoublequoteopen}n\ {\isasymnoteq}\ w{\isachardoublequoteclose}\isanewline
\ \ \ \ \ \ \isacommand{by}\isamarkupfalse%
\ auto\isanewline
\ \ \ \ \isacommand{from}\isamarkupfalse%
\ winner\isanewline
\ \ \ \ \isacommand{have}\isamarkupfalse%
\ w{\isacharunderscore}{\kern0pt}in{\isacharunderscore}{\kern0pt}A{\isacharcolon}{\kern0pt}\ {\isachardoublequoteopen}w\ {\isasymin}\ A{\isachardoublequoteclose}\isanewline
\ \ \ \ \ \ \isacommand{by}\isamarkupfalse%
\ simp\isanewline
\ \ \ \ \isacommand{from}\isamarkupfalse%
\ {\isadigit{6}}\ prof\ winner\isanewline
\ \ \ \ \isacommand{have}\isamarkupfalse%
\ {\isadigit{3}}{\isadigit{0}}{\isadigit{0}}{\isacharcolon}{\kern0pt}\ {\isachardoublequoteopen}prefer{\isacharunderscore}{\kern0pt}count\ p\ w\ n\ {\isachargreater}{\kern0pt}\ prefer{\isacharunderscore}{\kern0pt}count\ p\ n\ w{\isachardoublequoteclose}\isanewline
\ \ \ \ \ \ \isacommand{by}\isamarkupfalse%
\ simp\isanewline
\ \ \ \ \isacommand{from}\isamarkupfalse%
\ {\isadigit{1}}{\isadigit{0}}{\isadigit{0}}\ {\isadigit{0}}{\isadigit{0}}{\isadigit{0}}\ {\isadigit{2}}{\isadigit{0}}{\isadigit{0}}\isanewline
\ \ \ \ \isacommand{have}\isamarkupfalse%
\ {\isadigit{4}}{\isadigit{0}}{\isadigit{0}}{\isacharcolon}{\kern0pt}\ {\isachardoublequoteopen}prefer{\isacharunderscore}{\kern0pt}count\ p\ l\ w\ {\isasymge}\ prefer{\isacharunderscore}{\kern0pt}count\ p\ w\ n{\isachardoublequoteclose}\isanewline
\ \ \ \ \ \ \isacommand{by}\isamarkupfalse%
\ linarith\isanewline
\ \ \ \ \isacommand{with}\isamarkupfalse%
\ prof\ n{\isacharunderscore}{\kern0pt}in{\isacharunderscore}{\kern0pt}A\ w{\isacharunderscore}{\kern0pt}in{\isacharunderscore}{\kern0pt}A\ l{\isacharunderscore}{\kern0pt}in{\isacharunderscore}{\kern0pt}A\ n{\isacharunderscore}{\kern0pt}neq{\isacharunderscore}{\kern0pt}w\isanewline
\ \ \ \ \ \ \ \ \ l{\isacharunderscore}{\kern0pt}neq{\isacharunderscore}{\kern0pt}w\ pref{\isacharunderscore}{\kern0pt}count{\isacharunderscore}{\kern0pt}sym\isanewline
\ \ \ \ \isacommand{have}\isamarkupfalse%
\ {\isadigit{7}}{\isadigit{0}}{\isadigit{0}}{\isacharcolon}{\kern0pt}\ {\isachardoublequoteopen}prefer{\isacharunderscore}{\kern0pt}count\ p\ n\ w\ {\isasymge}\ prefer{\isacharunderscore}{\kern0pt}count\ p\ w\ l{\isachardoublequoteclose}\isanewline
\ \ \ \ \ \ \isacommand{by}\isamarkupfalse%
\ metis\isanewline
\ \ \ \ \isacommand{have}\isamarkupfalse%
\ {\isachardoublequoteopen}prefer{\isacharunderscore}{\kern0pt}count\ p\ l\ w\ {\isachargreater}{\kern0pt}\ prefer{\isacharunderscore}{\kern0pt}count\ p\ w\ l{\isachardoublequoteclose}\isanewline
\ \ \ \ \ \ \isacommand{using}\isamarkupfalse%
\ {\isachardoublequoteopen}{\isadigit{3}}{\isadigit{0}}{\isadigit{0}}{\isachardoublequoteclose}\ {\isachardoublequoteopen}{\isadigit{4}}{\isadigit{0}}{\isadigit{0}}{\isachardoublequoteclose}\ {\isachardoublequoteopen}{\isadigit{7}}{\isadigit{0}}{\isadigit{0}}{\isachardoublequoteclose}\isanewline
\ \ \ \ \ \ \isacommand{by}\isamarkupfalse%
\ linarith\isanewline
\ \ \ \ \isacommand{hence}\isamarkupfalse%
\ {\isachardoublequoteopen}wins\ l\ p\ w{\isachardoublequoteclose}\isanewline
\ \ \ \ \ \ \isacommand{by}\isamarkupfalse%
\ simp\isanewline
\ \ \ \ \isacommand{thus}\isamarkupfalse%
\ False\isanewline
\ \ \ \ \ \ \isacommand{using}\isamarkupfalse%
\ condorcet{\isacharunderscore}{\kern0pt}winner{\isachardot}{\kern0pt}simps\ l{\isacharunderscore}{\kern0pt}in{\isacharunderscore}{\kern0pt}A{\isacharunderscore}{\kern0pt}without{\isacharunderscore}{\kern0pt}w\isanewline
\ \ \ \ \ \ \ \ \ \ \ \ wins{\isacharunderscore}{\kern0pt}antisym\ winner\isanewline
\ \ \ \ \ \ \isacommand{by}\isamarkupfalse%
\ metis\isanewline
\ \ \isacommand{qed}\isamarkupfalse%
\isanewline
\isacommand{qed}\isamarkupfalse%
%
\endisatagproof
{\isafoldproof}%
%
\isadelimproof
\isanewline
%
\endisadelimproof
\isanewline
\isacommand{theorem}\isamarkupfalse%
\ minimax{\isacharunderscore}{\kern0pt}is{\isacharunderscore}{\kern0pt}dcc{\isacharcolon}{\kern0pt}\ {\isachardoublequoteopen}defer{\isacharunderscore}{\kern0pt}condorcet{\isacharunderscore}{\kern0pt}consistency\ minimax{\isachardoublequoteclose}\isanewline
%
\isadelimproof
%
\endisadelimproof
%
\isatagproof
\isacommand{proof}\isamarkupfalse%
\ {\isacharminus}{\kern0pt}\isanewline
\ \ \isacommand{have}\isamarkupfalse%
\ max{\isacharunderscore}{\kern0pt}mmaxscore{\isacharunderscore}{\kern0pt}dcc{\isacharcolon}{\kern0pt}\isanewline
\ \ \ \ {\isachardoublequoteopen}defer{\isacharunderscore}{\kern0pt}condorcet{\isacharunderscore}{\kern0pt}consistency\ {\isacharparenleft}{\kern0pt}max{\isacharunderscore}{\kern0pt}eliminator\ minimax{\isacharunderscore}{\kern0pt}score{\isacharparenright}{\kern0pt}{\isachardoublequoteclose}\isanewline
\ \ \ \ \isacommand{using}\isamarkupfalse%
\ cr{\isacharunderscore}{\kern0pt}eval{\isacharunderscore}{\kern0pt}imp{\isacharunderscore}{\kern0pt}dcc{\isacharunderscore}{\kern0pt}max{\isacharunderscore}{\kern0pt}elim\isanewline
\ \ \ \ \isacommand{by}\isamarkupfalse%
\ {\isacharparenleft}{\kern0pt}simp\ add{\isacharcolon}{\kern0pt}\ minimax{\isacharunderscore}{\kern0pt}score{\isacharunderscore}{\kern0pt}cond{\isacharunderscore}{\kern0pt}rating{\isacharparenright}{\kern0pt}\isanewline
\ \ \isacommand{have}\isamarkupfalse%
\ mmax{\isacharunderscore}{\kern0pt}eq{\isacharunderscore}{\kern0pt}max{\isacharunderscore}{\kern0pt}mmax{\isacharcolon}{\kern0pt}\isanewline
\ \ \ \ {\isachardoublequoteopen}{\isasymAnd}A\ p{\isachardot}{\kern0pt}\ {\isacharparenleft}{\kern0pt}minimax\ A\ p\ {\isasymequiv}\ max{\isacharunderscore}{\kern0pt}eliminator\ minimax{\isacharunderscore}{\kern0pt}score\ A\ p{\isacharparenright}{\kern0pt}{\isachardoublequoteclose}\isanewline
\ \ \ \ \isacommand{by}\isamarkupfalse%
\ simp\isanewline
\ \ \isacommand{from}\isamarkupfalse%
\ max{\isacharunderscore}{\kern0pt}mmaxscore{\isacharunderscore}{\kern0pt}dcc\ mmax{\isacharunderscore}{\kern0pt}eq{\isacharunderscore}{\kern0pt}max{\isacharunderscore}{\kern0pt}mmax\isanewline
\ \ \isacommand{show}\isamarkupfalse%
\ {\isacharquery}{\kern0pt}thesis\isanewline
\ \ \ \ \isacommand{unfolding}\isamarkupfalse%
\ defer{\isacharunderscore}{\kern0pt}condorcet{\isacharunderscore}{\kern0pt}consistency{\isacharunderscore}{\kern0pt}def\ electoral{\isacharunderscore}{\kern0pt}module{\isacharunderscore}{\kern0pt}def\isanewline
\ \ \ \ \isacommand{by}\isamarkupfalse%
\ {\isacharparenleft}{\kern0pt}smt\ {\isacharparenleft}{\kern0pt}verit{\isacharcomma}{\kern0pt}\ ccfv{\isacharunderscore}{\kern0pt}threshold{\isacharparenright}{\kern0pt}{\isacharparenright}{\kern0pt}\isanewline
\isacommand{qed}\isamarkupfalse%
%
\endisatagproof
{\isafoldproof}%
%
\isadelimproof
\isanewline
%
\endisadelimproof
\isanewline
\isanewline
%
\isadelimtheory
\isanewline
%
\endisadelimtheory
%
\isatagtheory
\isacommand{end}\isamarkupfalse%
%
\endisatagtheory
{\isafoldtheory}%
%
\isadelimtheory
%
\endisadelimtheory
%
\end{isabellebody}%
\endinput
%:%file=~/Documents/Studies/VotingRuleGenerator/virage/src/test/resources/verifiedVotingRuleConstruction/theories/Compositional_Framework/Composition_Rules/Condorcet_Facts.thy%:%
%:%10=1%:%
%:%11=1%:%
%:%12=2%:%
%:%13=3%:%
%:%14=4%:%
%:%15=5%:%
%:%16=6%:%
%:%17=7%:%
%:%22=7%:%
%:%25=8%:%
%:%26=9%:%
%:%27=10%:%
%:%28=10%:%
%:%35=11%:%
%:%36=11%:%
%:%37=12%:%
%:%38=12%:%
%:%39=13%:%
%:%46=20%:%
%:%47=21%:%
%:%48=21%:%
%:%49=22%:%
%:%50=22%:%
%:%51=23%:%
%:%52=23%:%
%:%53=24%:%
%:%54=24%:%
%:%55=25%:%
%:%56=25%:%
%:%57=26%:%
%:%63=26%:%
%:%66=27%:%
%:%67=28%:%
%:%68=29%:%
%:%69=29%:%
%:%72=30%:%
%:%76=30%:%
%:%77=30%:%
%:%78=31%:%
%:%79=31%:%
%:%80=32%:%
%:%81=32%:%
%:%82=33%:%
%:%83=34%:%
%:%84=35%:%
%:%85=36%:%
%:%86=37%:%
%:%87=37%:%
%:%88=38%:%
%:%89=39%:%
%:%90=40%:%
%:%91=41%:%
%:%92=41%:%
%:%93=42%:%
%:%94=43%:%
%:%95=44%:%
%:%96=44%:%
%:%97=45%:%
%:%98=45%:%
%:%99=45%:%
%:%100=46%:%
%:%101=47%:%
%:%102=48%:%
%:%103=48%:%
%:%104=49%:%
%:%105=49%:%
%:%106=50%:%
%:%107=50%:%
%:%108=50%:%
%:%109=51%:%
%:%110=52%:%
%:%111=53%:%
%:%112=53%:%
%:%113=54%:%
%:%114=54%:%
%:%115=55%:%
%:%116=55%:%
%:%117=56%:%
%:%118=56%:%
%:%119=57%:%
%:%120=58%:%
%:%121=59%:%
%:%122=60%:%
%:%123=61%:%
%:%124=61%:%
%:%125=62%:%
%:%126=62%:%
%:%127=63%:%
%:%128=64%:%
%:%129=65%:%
%:%130=65%:%
%:%131=66%:%
%:%132=66%:%
%:%133=67%:%
%:%134=67%:%
%:%135=68%:%
%:%136=68%:%
%:%137=69%:%
%:%138=69%:%
%:%139=70%:%
%:%140=70%:%
%:%141=71%:%
%:%147=71%:%
%:%150=72%:%
%:%151=73%:%
%:%152=73%:%
%:%159=74%:%
%:%160=74%:%
%:%161=75%:%
%:%162=75%:%
%:%163=76%:%
%:%164=77%:%
%:%165=77%:%
%:%166=78%:%
%:%167=78%:%
%:%168=79%:%
%:%169=79%:%
%:%170=80%:%
%:%171=81%:%
%:%172=81%:%
%:%173=82%:%
%:%174=82%:%
%:%175=82%:%
%:%176=83%:%
%:%177=83%:%
%:%178=84%:%
%:%179=84%:%
%:%180=85%:%
%:%186=85%:%
%:%189=86%:%
%:%190=87%:%
%:%191=87%:%
%:%198=88%:%
%:%199=88%:%
%:%200=89%:%
%:%201=89%:%
%:%202=90%:%
%:%203=91%:%
%:%204=91%:%
%:%205=92%:%
%:%206=92%:%
%:%207=93%:%
%:%208=93%:%
%:%209=94%:%
%:%210=95%:%
%:%211=95%:%
%:%212=96%:%
%:%213=96%:%
%:%214=97%:%
%:%215=97%:%
%:%216=98%:%
%:%217=98%:%
%:%218=99%:%
%:%219=99%:%
%:%220=100%:%
%:%226=100%:%
%:%229=101%:%
%:%230=102%:%
%:%231=102%:%
%:%238=103%:%
%:%239=103%:%
%:%240=104%:%
%:%241=104%:%
%:%242=105%:%
%:%243=106%:%
%:%244=107%:%
%:%245=108%:%
%:%246=109%:%
%:%247=109%:%
%:%248=110%:%
%:%249=111%:%
%:%250=112%:%
%:%251=113%:%
%:%252=113%:%
%:%253=114%:%
%:%256=117%:%
%:%257=118%:%
%:%258=118%:%
%:%259=119%:%
%:%260=119%:%
%:%261=120%:%
%:%264=123%:%
%:%265=124%:%
%:%266=124%:%
%:%267=125%:%
%:%270=128%:%
%:%271=129%:%
%:%272=129%:%
%:%273=130%:%
%:%274=130%:%
%:%275=131%:%
%:%276=132%:%
%:%277=133%:%
%:%278=133%:%
%:%279=134%:%
%:%280=134%:%
%:%281=135%:%
%:%282=135%:%
%:%283=136%:%
%:%284=136%:%
%:%285=137%:%
%:%286=137%:%
%:%287=138%:%
%:%288=138%:%
%:%289=139%:%
%:%290=140%:%
%:%291=140%:%
%:%292=141%:%
%:%293=141%:%
%:%294=159%:%
%:%295=160%:%
%:%296=160%:%
%:%297=161%:%
%:%298=161%:%
%:%299=162%:%
%:%300=162%:%
%:%301=163%:%
%:%302=163%:%
%:%303=164%:%
%:%304=164%:%
%:%305=165%:%
%:%306=165%:%
%:%307=166%:%
%:%308=166%:%
%:%309=167%:%
%:%310=167%:%
%:%311=168%:%
%:%312=168%:%
%:%313=169%:%
%:%314=169%:%
%:%315=170%:%
%:%316=170%:%
%:%317=171%:%
%:%318=171%:%
%:%319=172%:%
%:%320=173%:%
%:%321=174%:%
%:%322=174%:%
%:%323=175%:%
%:%324=175%:%
%:%325=176%:%
%:%326=176%:%
%:%327=176%:%
%:%328=177%:%
%:%329=178%:%
%:%330=179%:%
%:%331=180%:%
%:%332=180%:%
%:%333=181%:%
%:%334=181%:%
%:%335=182%:%
%:%336=182%:%
%:%337=183%:%
%:%338=183%:%
%:%339=184%:%
%:%340=184%:%
%:%341=185%:%
%:%342=185%:%
%:%343=186%:%
%:%344=186%:%
%:%345=187%:%
%:%346=187%:%
%:%347=188%:%
%:%348=188%:%
%:%349=189%:%
%:%350=189%:%
%:%351=190%:%
%:%352=190%:%
%:%353=191%:%
%:%354=191%:%
%:%355=192%:%
%:%356=192%:%
%:%357=193%:%
%:%358=193%:%
%:%359=194%:%
%:%360=194%:%
%:%361=195%:%
%:%362=195%:%
%:%363=196%:%
%:%364=196%:%
%:%365=197%:%
%:%366=198%:%
%:%367=198%:%
%:%368=199%:%
%:%369=199%:%
%:%370=200%:%
%:%371=200%:%
%:%372=201%:%
%:%373=201%:%
%:%374=202%:%
%:%375=202%:%
%:%376=203%:%
%:%377=203%:%
%:%378=204%:%
%:%379=204%:%
%:%380=205%:%
%:%381=205%:%
%:%382=206%:%
%:%383=206%:%
%:%384=207%:%
%:%385=208%:%
%:%386=208%:%
%:%387=209%:%
%:%388=209%:%
%:%389=210%:%
%:%395=210%:%
%:%398=211%:%
%:%399=212%:%
%:%400=212%:%
%:%407=213%:%
%:%408=213%:%
%:%409=214%:%
%:%410=214%:%
%:%411=215%:%
%:%412=216%:%
%:%413=216%:%
%:%414=217%:%
%:%415=217%:%
%:%416=218%:%
%:%417=218%:%
%:%418=219%:%
%:%419=220%:%
%:%420=220%:%
%:%421=221%:%
%:%422=221%:%
%:%423=222%:%
%:%424=222%:%
%:%425=223%:%
%:%426=223%:%
%:%427=224%:%
%:%428=224%:%
%:%429=225%:%
%:%435=225%:%
%:%438=226%:%
%:%439=227%:%
%:%442=228%:%
%:%447=229%:%
%
\begin{isabellebody}%
\setisabellecontext{Pairwise{\isacharunderscore}{\kern0pt}Majority{\isacharunderscore}{\kern0pt}Rule}%
%
\isadelimdocument
\isanewline
%
\endisadelimdocument
%
\isatagdocument
\isanewline
%
\isamarkupsection{Pairwise Majority Rule%
}
\isamarkuptrue%
%
\endisatagdocument
{\isafolddocument}%
%
\isadelimdocument
%
\endisadelimdocument
%
\isadelimtheory
%
\endisadelimtheory
%
\isatagtheory
\isacommand{theory}\isamarkupfalse%
\ Pairwise{\isacharunderscore}{\kern0pt}Majority{\isacharunderscore}{\kern0pt}Rule\isanewline
\ \ \isakeyword{imports}\ {\isachardoublequoteopen}Compositional{\isacharunderscore}{\kern0pt}Structures{\isacharslash}{\kern0pt}Basic{\isacharunderscore}{\kern0pt}Modules{\isacharslash}{\kern0pt}Condorcet{\isacharunderscore}{\kern0pt}Module{\isachardoublequoteclose}\isanewline
\ \ \ \ \ \ \ \ \ \ {\isachardoublequoteopen}Compositional{\isacharunderscore}{\kern0pt}Structures{\isacharslash}{\kern0pt}Defer{\isacharunderscore}{\kern0pt}One{\isacharunderscore}{\kern0pt}Loop{\isacharunderscore}{\kern0pt}Composition{\isachardoublequoteclose}\isanewline
\isakeyword{begin}%
\endisatagtheory
{\isafoldtheory}%
%
\isadelimtheory
%
\endisadelimtheory
%
\begin{isamarkuptext}%
This is the pairwise majority rule, a voting rule that implements the
Condorcet criterion, i.e., it elects the Condorcet winner if it exists,
otherwise a tie remains between all alternatives.%
\end{isamarkuptext}\isamarkuptrue%
%
\isadelimdocument
%
\endisadelimdocument
%
\isatagdocument
%
\isamarkupsubsection{Definition%
}
\isamarkuptrue%
%
\endisatagdocument
{\isafolddocument}%
%
\isadelimdocument
%
\endisadelimdocument
\isacommand{fun}\isamarkupfalse%
\ pairwise{\isacharunderscore}{\kern0pt}majority{\isacharunderscore}{\kern0pt}rule\ {\isacharcolon}{\kern0pt}{\isacharcolon}{\kern0pt}\ {\isachardoublequoteopen}{\isacharprime}{\kern0pt}a\ Electoral{\isacharunderscore}{\kern0pt}Module{\isachardoublequoteclose}\ \isakeyword{where}\isanewline
\ \ {\isachardoublequoteopen}pairwise{\isacharunderscore}{\kern0pt}majority{\isacharunderscore}{\kern0pt}rule\ A\ p\ {\isacharequal}{\kern0pt}\ elector\ condorcet\ A\ p{\isachardoublequoteclose}\isanewline
\isanewline
\isacommand{fun}\isamarkupfalse%
\ pairwise{\isacharunderscore}{\kern0pt}majority{\isacharunderscore}{\kern0pt}rule{\isacharunderscore}{\kern0pt}code\ {\isacharcolon}{\kern0pt}{\isacharcolon}{\kern0pt}\ {\isachardoublequoteopen}{\isacharprime}{\kern0pt}a\ Electoral{\isacharunderscore}{\kern0pt}Module{\isachardoublequoteclose}\ \isakeyword{where}\isanewline
\ \ {\isachardoublequoteopen}pairwise{\isacharunderscore}{\kern0pt}majority{\isacharunderscore}{\kern0pt}rule{\isacharunderscore}{\kern0pt}code\ A\ p\ {\isacharequal}{\kern0pt}\ elector\ condorcet{\isacharunderscore}{\kern0pt}code\ A\ p{\isachardoublequoteclose}\isanewline
\isanewline
\isacommand{fun}\isamarkupfalse%
\ condorcet{\isacharprime}{\kern0pt}\ {\isacharcolon}{\kern0pt}{\isacharcolon}{\kern0pt}\ {\isachardoublequoteopen}{\isacharprime}{\kern0pt}a\ Electoral{\isacharunderscore}{\kern0pt}Module{\isachardoublequoteclose}\ \isakeyword{where}\isanewline
{\isachardoublequoteopen}condorcet{\isacharprime}{\kern0pt}\ A\ p\ {\isacharequal}{\kern0pt}\isanewline
\ \ {\isacharparenleft}{\kern0pt}{\isacharparenleft}{\kern0pt}min{\isacharunderscore}{\kern0pt}eliminator\ condorcet{\isacharunderscore}{\kern0pt}score{\isacharparenright}{\kern0pt}\ {\isasymcirclearrowleft}\isactrlsub {\isasymexists}\isactrlsub {\isacharbang}{\kern0pt}\isactrlsub d{\isacharparenright}{\kern0pt}\ A\ p{\isachardoublequoteclose}\isanewline
\isanewline
\isacommand{fun}\isamarkupfalse%
\ pairwise{\isacharunderscore}{\kern0pt}majority{\isacharunderscore}{\kern0pt}rule{\isacharprime}{\kern0pt}\ {\isacharcolon}{\kern0pt}{\isacharcolon}{\kern0pt}\ {\isachardoublequoteopen}{\isacharprime}{\kern0pt}a\ Electoral{\isacharunderscore}{\kern0pt}Module{\isachardoublequoteclose}\ \isakeyword{where}\isanewline
{\isachardoublequoteopen}pairwise{\isacharunderscore}{\kern0pt}majority{\isacharunderscore}{\kern0pt}rule{\isacharprime}{\kern0pt}\ A\ p\ {\isacharequal}{\kern0pt}\ iterelect\ condorcet{\isacharprime}{\kern0pt}\ A\ p{\isachardoublequoteclose}%
\isadelimdocument
%
\endisadelimdocument
%
\isatagdocument
%
\isamarkupsubsection{Condorcet Consistency Property%
}
\isamarkuptrue%
%
\endisatagdocument
{\isafolddocument}%
%
\isadelimdocument
%
\endisadelimdocument
\isacommand{theorem}\isamarkupfalse%
\ condorcet{\isacharunderscore}{\kern0pt}condorcet{\isacharcolon}{\kern0pt}\ {\isachardoublequoteopen}condorcet{\isacharunderscore}{\kern0pt}consistency\ pairwise{\isacharunderscore}{\kern0pt}majority{\isacharunderscore}{\kern0pt}rule{\isachardoublequoteclose}\isanewline
%
\isadelimproof
%
\endisadelimproof
%
\isatagproof
\isacommand{proof}\isamarkupfalse%
\ {\isacharminus}{\kern0pt}\isanewline
\ \ \isacommand{have}\isamarkupfalse%
\isanewline
\ \ \ \ {\isachardoublequoteopen}condorcet{\isacharunderscore}{\kern0pt}consistency\ {\isacharparenleft}{\kern0pt}elector\ condorcet{\isacharparenright}{\kern0pt}{\isachardoublequoteclose}\isanewline
\ \ \ \ \isacommand{using}\isamarkupfalse%
\ condorcet{\isacharunderscore}{\kern0pt}is{\isacharunderscore}{\kern0pt}dcc\ dcc{\isacharunderscore}{\kern0pt}imp{\isacharunderscore}{\kern0pt}cc{\isacharunderscore}{\kern0pt}elector\isanewline
\ \ \ \ \isacommand{by}\isamarkupfalse%
\ metis\isanewline
\ \ \isacommand{thus}\isamarkupfalse%
\ {\isacharquery}{\kern0pt}thesis\isanewline
\ \ \ \ \isacommand{using}\isamarkupfalse%
\ condorcet{\isacharunderscore}{\kern0pt}consistency{\isadigit{2}}\ electoral{\isacharunderscore}{\kern0pt}module{\isacharunderscore}{\kern0pt}def\isanewline
\ \ \ \ \ \ \ \ \ \ pairwise{\isacharunderscore}{\kern0pt}majority{\isacharunderscore}{\kern0pt}rule{\isachardot}{\kern0pt}simps\isanewline
\ \ \ \ \isacommand{by}\isamarkupfalse%
\ metis\isanewline
\isacommand{qed}\isamarkupfalse%
%
\endisatagproof
{\isafoldproof}%
%
\isadelimproof
\isanewline
%
\endisadelimproof
%
\isadelimtheory
\isanewline
%
\endisadelimtheory
%
\isatagtheory
\isacommand{end}\isamarkupfalse%
%
\endisatagtheory
{\isafoldtheory}%
%
\isadelimtheory
%
\endisadelimtheory
%
\end{isabellebody}%
\endinput
%:%file=~/Documents/Studies/VotingRuleGenerator/virage/src/test/resources/old_theories/Pairwise_Majority_Rule.thy%:%
%:%6=3%:%
%:%11=4%:%
%:%13=7%:%
%:%29=9%:%
%:%30=9%:%
%:%31=10%:%
%:%32=11%:%
%:%33=12%:%
%:%42=15%:%
%:%43=16%:%
%:%44=17%:%
%:%53=19%:%
%:%63=21%:%
%:%64=21%:%
%:%65=22%:%
%:%66=23%:%
%:%67=24%:%
%:%68=24%:%
%:%69=25%:%
%:%70=26%:%
%:%71=27%:%
%:%72=27%:%
%:%73=28%:%
%:%74=29%:%
%:%75=30%:%
%:%76=31%:%
%:%77=31%:%
%:%78=32%:%
%:%85=34%:%
%:%95=36%:%
%:%96=36%:%
%:%103=37%:%
%:%104=37%:%
%:%105=38%:%
%:%106=38%:%
%:%107=39%:%
%:%108=40%:%
%:%109=40%:%
%:%110=41%:%
%:%111=41%:%
%:%112=42%:%
%:%113=42%:%
%:%114=43%:%
%:%115=43%:%
%:%116=44%:%
%:%117=45%:%
%:%118=45%:%
%:%119=46%:%
%:%125=46%:%
%:%130=47%:%
%:%135=48%:%
%
\begin{isabellebody}%
\setisabellecontext{Copeland{\isacharunderscore}{\kern0pt}Rule}%
%
\isadelimdocument
\isanewline
%
\endisadelimdocument
%
\isatagdocument
\isanewline
%
\isamarkupsection{Copeland Rule%
}
\isamarkuptrue%
%
\endisatagdocument
{\isafolddocument}%
%
\isadelimdocument
%
\endisadelimdocument
%
\isadelimtheory
%
\endisadelimtheory
%
\isatagtheory
\isacommand{theory}\isamarkupfalse%
\ Copeland{\isacharunderscore}{\kern0pt}Rule\isanewline
\ \ \isakeyword{imports}\ {\isachardoublequoteopen}{\isachardot}{\kern0pt}{\isachardot}{\kern0pt}{\isacharslash}{\kern0pt}Compositional{\isacharunderscore}{\kern0pt}Framework{\isacharslash}{\kern0pt}Components{\isacharslash}{\kern0pt}Composites{\isacharslash}{\kern0pt}Composite{\isacharunderscore}{\kern0pt}Elimination{\isacharunderscore}{\kern0pt}Modules{\isachardoublequoteclose}\isanewline
\ \ \ \ \ \ \ \ \ \ {\isachardoublequoteopen}{\isachardot}{\kern0pt}{\isachardot}{\kern0pt}{\isacharslash}{\kern0pt}Compositional{\isacharunderscore}{\kern0pt}Framework{\isacharslash}{\kern0pt}Components{\isacharslash}{\kern0pt}Composites{\isacharslash}{\kern0pt}Composite{\isacharunderscore}{\kern0pt}Structures{\isachardoublequoteclose}\isanewline
\ \ \ \ \ \ \ \ \ \ {\isachardoublequoteopen}{\isachardot}{\kern0pt}{\isachardot}{\kern0pt}{\isacharslash}{\kern0pt}Compositional{\isacharunderscore}{\kern0pt}Framework{\isacharslash}{\kern0pt}Composition{\isacharunderscore}{\kern0pt}Rules{\isacharslash}{\kern0pt}Condorcet{\isacharunderscore}{\kern0pt}Facts{\isachardoublequoteclose}\isanewline
\isanewline
\isakeyword{begin}%
\endisatagtheory
{\isafoldtheory}%
%
\isadelimtheory
%
\endisadelimtheory
%
\begin{isamarkuptext}%
This is the Copeland voting rule. The idea is to elect the alternatives with
the highest difference between the amount of simple-majority wins and the
amount of simple-majority losses.%
\end{isamarkuptext}\isamarkuptrue%
%
\isadelimdocument
%
\endisadelimdocument
%
\isatagdocument
%
\isamarkupsubsection{Definition%
}
\isamarkuptrue%
%
\endisatagdocument
{\isafolddocument}%
%
\isadelimdocument
%
\endisadelimdocument
\isacommand{fun}\isamarkupfalse%
\ copeland{\isacharunderscore}{\kern0pt}rule\ {\isacharcolon}{\kern0pt}{\isacharcolon}{\kern0pt}\ {\isachardoublequoteopen}{\isacharprime}{\kern0pt}a\ Electoral{\isacharunderscore}{\kern0pt}Module{\isachardoublequoteclose}\ \isakeyword{where}\isanewline
\ \ {\isachardoublequoteopen}copeland{\isacharunderscore}{\kern0pt}rule\ A\ p\ {\isacharequal}{\kern0pt}\ elector\ copeland\ A\ p{\isachardoublequoteclose}\isanewline
\isanewline
\isanewline
\isacommand{theorem}\isamarkupfalse%
\ copeland{\isacharunderscore}{\kern0pt}condorcet{\isacharcolon}{\kern0pt}\ {\isachardoublequoteopen}condorcet{\isacharunderscore}{\kern0pt}consistency\ copeland{\isacharunderscore}{\kern0pt}rule{\isachardoublequoteclose}\isanewline
%
\isadelimproof
%
\endisadelimproof
%
\isatagproof
\isacommand{proof}\isamarkupfalse%
\ {\isacharminus}{\kern0pt}\isanewline
\ \ \isacommand{have}\isamarkupfalse%
\isanewline
\ \ \ \ {\isachardoublequoteopen}condorcet{\isacharunderscore}{\kern0pt}consistency\ {\isacharparenleft}{\kern0pt}elector\ copeland{\isacharparenright}{\kern0pt}{\isachardoublequoteclose}\isanewline
\ \ \ \ \isacommand{using}\isamarkupfalse%
\ copeland{\isacharunderscore}{\kern0pt}is{\isacharunderscore}{\kern0pt}dcc\ dcc{\isacharunderscore}{\kern0pt}imp{\isacharunderscore}{\kern0pt}cc{\isacharunderscore}{\kern0pt}elector\isanewline
\ \ \ \ \isacommand{by}\isamarkupfalse%
\ metis\isanewline
\ \ \isacommand{thus}\isamarkupfalse%
\ {\isacharquery}{\kern0pt}thesis\isanewline
\ \ \ \ \isacommand{using}\isamarkupfalse%
\ condorcet{\isacharunderscore}{\kern0pt}consistency{\isadigit{2}}\ electoral{\isacharunderscore}{\kern0pt}module{\isacharunderscore}{\kern0pt}def\isanewline
\ \ \ \ \ \ \ \ \ \ copeland{\isacharunderscore}{\kern0pt}rule{\isachardot}{\kern0pt}simps\isanewline
\ \ \ \ \isacommand{by}\isamarkupfalse%
\ metis\isanewline
\isacommand{qed}\isamarkupfalse%
%
\endisatagproof
{\isafoldproof}%
%
\isadelimproof
\isanewline
%
\endisadelimproof
%
\isadelimtheory
\isanewline
%
\endisadelimtheory
%
\isatagtheory
\isacommand{end}\isamarkupfalse%
%
\endisatagtheory
{\isafoldtheory}%
%
\isadelimtheory
%
\endisadelimtheory
%
\end{isabellebody}%
\endinput
%:%file=~/Documents/Studies/VotingRuleGenerator/virage/src/test/resources/verifiedVotingRuleConstruction/theories/Voting_Rules/Copeland_Rule.thy%:%
%:%6=3%:%
%:%11=4%:%
%:%13=7%:%
%:%29=9%:%
%:%30=9%:%
%:%31=10%:%
%:%32=11%:%
%:%33=12%:%
%:%34=13%:%
%:%35=14%:%
%:%44=17%:%
%:%45=18%:%
%:%46=19%:%
%:%55=21%:%
%:%65=23%:%
%:%66=23%:%
%:%67=24%:%
%:%68=25%:%
%:%69=26%:%
%:%70=27%:%
%:%71=27%:%
%:%78=28%:%
%:%79=28%:%
%:%80=29%:%
%:%81=29%:%
%:%82=30%:%
%:%83=31%:%
%:%84=31%:%
%:%85=32%:%
%:%86=32%:%
%:%87=33%:%
%:%88=33%:%
%:%89=34%:%
%:%90=34%:%
%:%91=35%:%
%:%92=36%:%
%:%93=36%:%
%:%94=37%:%
%:%100=37%:%
%:%105=38%:%
%:%110=39%:%
%
\begin{isabellebody}%
\setisabellecontext{Minimax{\isacharunderscore}{\kern0pt}Rule}%
%
\isadelimdocument
\isanewline
%
\endisadelimdocument
%
\isatagdocument
\isanewline
%
\isamarkupsection{Minimax Rule%
}
\isamarkuptrue%
%
\endisatagdocument
{\isafolddocument}%
%
\isadelimdocument
%
\endisadelimdocument
%
\isadelimtheory
%
\endisadelimtheory
%
\isatagtheory
\isacommand{theory}\isamarkupfalse%
\ Minimax{\isacharunderscore}{\kern0pt}Rule\isanewline
\ \ \isakeyword{imports}\ {\isachardoublequoteopen}{\isachardot}{\kern0pt}{\isachardot}{\kern0pt}{\isacharslash}{\kern0pt}Compositional{\isacharunderscore}{\kern0pt}Framework{\isacharslash}{\kern0pt}Components{\isacharslash}{\kern0pt}Composites{\isacharslash}{\kern0pt}Composite{\isacharunderscore}{\kern0pt}Elimination{\isacharunderscore}{\kern0pt}Modules{\isachardoublequoteclose}\isanewline
\ \ \ \ \ \ \ \ \ \ {\isachardoublequoteopen}{\isachardot}{\kern0pt}{\isachardot}{\kern0pt}{\isacharslash}{\kern0pt}Compositional{\isacharunderscore}{\kern0pt}Framework{\isacharslash}{\kern0pt}Components{\isacharslash}{\kern0pt}Composites{\isacharslash}{\kern0pt}Composite{\isacharunderscore}{\kern0pt}Structures{\isachardoublequoteclose}\isanewline
\ \ \ \ \ \ \ \ \ \ {\isachardoublequoteopen}{\isachardot}{\kern0pt}{\isachardot}{\kern0pt}{\isacharslash}{\kern0pt}Compositional{\isacharunderscore}{\kern0pt}Framework{\isacharslash}{\kern0pt}Composition{\isacharunderscore}{\kern0pt}Rules{\isacharslash}{\kern0pt}Condorcet{\isacharunderscore}{\kern0pt}Facts{\isachardoublequoteclose}\isanewline
\isanewline
\isakeyword{begin}%
\endisatagtheory
{\isafoldtheory}%
%
\isadelimtheory
%
\endisadelimtheory
%
\begin{isamarkuptext}%
This is the Minimax voting rule. It elects the alternatives with the highest
Minimax score.%
\end{isamarkuptext}\isamarkuptrue%
%
\isadelimdocument
%
\endisadelimdocument
%
\isatagdocument
%
\isamarkupsubsection{Definition%
}
\isamarkuptrue%
%
\endisatagdocument
{\isafolddocument}%
%
\isadelimdocument
%
\endisadelimdocument
\isacommand{fun}\isamarkupfalse%
\ minimax{\isacharunderscore}{\kern0pt}rule\ {\isacharcolon}{\kern0pt}{\isacharcolon}{\kern0pt}\ {\isachardoublequoteopen}{\isacharprime}{\kern0pt}a\ Electoral{\isacharunderscore}{\kern0pt}Module{\isachardoublequoteclose}\ \isakeyword{where}\isanewline
\ \ {\isachardoublequoteopen}minimax{\isacharunderscore}{\kern0pt}rule\ A\ p\ {\isacharequal}{\kern0pt}\ elector\ minimax\ A\ p{\isachardoublequoteclose}\isanewline
\isanewline
\isanewline
\isacommand{theorem}\isamarkupfalse%
\ minimax{\isacharunderscore}{\kern0pt}condorcet{\isacharcolon}{\kern0pt}\ {\isachardoublequoteopen}condorcet{\isacharunderscore}{\kern0pt}consistency\ minimax{\isacharunderscore}{\kern0pt}rule{\isachardoublequoteclose}\isanewline
%
\isadelimproof
%
\endisadelimproof
%
\isatagproof
\isacommand{proof}\isamarkupfalse%
\ {\isacharminus}{\kern0pt}\isanewline
\ \ \isacommand{have}\isamarkupfalse%
\isanewline
\ \ \ \ {\isachardoublequoteopen}condorcet{\isacharunderscore}{\kern0pt}consistency\ {\isacharparenleft}{\kern0pt}elector\ minimax{\isacharparenright}{\kern0pt}{\isachardoublequoteclose}\isanewline
\ \ \ \ \isacommand{using}\isamarkupfalse%
\ minimax{\isacharunderscore}{\kern0pt}is{\isacharunderscore}{\kern0pt}dcc\ dcc{\isacharunderscore}{\kern0pt}imp{\isacharunderscore}{\kern0pt}cc{\isacharunderscore}{\kern0pt}elector\isanewline
\ \ \ \ \isacommand{by}\isamarkupfalse%
\ metis\isanewline
\ \ \isacommand{thus}\isamarkupfalse%
\ {\isacharquery}{\kern0pt}thesis\isanewline
\ \ \ \ \isacommand{using}\isamarkupfalse%
\ condorcet{\isacharunderscore}{\kern0pt}consistency{\isadigit{2}}\ electoral{\isacharunderscore}{\kern0pt}module{\isacharunderscore}{\kern0pt}def\isanewline
\ \ \ \ \ \ \ \ \ \ minimax{\isacharunderscore}{\kern0pt}rule{\isachardot}{\kern0pt}simps\isanewline
\ \ \ \ \isacommand{by}\isamarkupfalse%
\ metis\isanewline
\isacommand{qed}\isamarkupfalse%
%
\endisatagproof
{\isafoldproof}%
%
\isadelimproof
\isanewline
%
\endisadelimproof
%
\isadelimtheory
\isanewline
%
\endisadelimtheory
%
\isatagtheory
\isacommand{end}\isamarkupfalse%
%
\endisatagtheory
{\isafoldtheory}%
%
\isadelimtheory
%
\endisadelimtheory
%
\end{isabellebody}%
\endinput
%:%file=~/Documents/Studies/VotingRuleGenerator/virage/src/test/resources/verifiedVotingRuleConstruction/theories/Voting_Rules/Minimax_Rule.thy%:%
%:%6=3%:%
%:%11=4%:%
%:%13=7%:%
%:%29=9%:%
%:%30=9%:%
%:%31=10%:%
%:%32=11%:%
%:%33=12%:%
%:%34=13%:%
%:%35=14%:%
%:%44=17%:%
%:%45=18%:%
%:%54=20%:%
%:%64=22%:%
%:%65=22%:%
%:%66=23%:%
%:%67=24%:%
%:%68=25%:%
%:%69=26%:%
%:%70=26%:%
%:%77=27%:%
%:%78=27%:%
%:%79=28%:%
%:%80=28%:%
%:%81=29%:%
%:%82=30%:%
%:%83=30%:%
%:%84=31%:%
%:%85=31%:%
%:%86=32%:%
%:%87=32%:%
%:%88=33%:%
%:%89=33%:%
%:%90=34%:%
%:%91=35%:%
%:%92=35%:%
%:%93=36%:%
%:%99=36%:%
%:%104=37%:%
%:%109=38%:%
%
\begin{isabellebody}%
\setisabellecontext{Blacks{\isacharunderscore}{\kern0pt}Rule}%
%
\isadelimdocument
\isanewline
%
\endisadelimdocument
%
\isatagdocument
\isanewline
%
\isamarkupsection{Black's Rule%
}
\isamarkuptrue%
%
\endisatagdocument
{\isafolddocument}%
%
\isadelimdocument
%
\endisadelimdocument
%
\isadelimtheory
%
\endisadelimtheory
%
\isatagtheory
\isacommand{theory}\isamarkupfalse%
\ Blacks{\isacharunderscore}{\kern0pt}Rule\isanewline
\ \ \isakeyword{imports}\ Pairwise{\isacharunderscore}{\kern0pt}Majority{\isacharunderscore}{\kern0pt}Rule\isanewline
\ \ \ \ \ \ \ \ \ \ Borda{\isacharunderscore}{\kern0pt}Rule\isanewline
\isakeyword{begin}%
\endisatagtheory
{\isafoldtheory}%
%
\isadelimtheory
%
\endisadelimtheory
%
\begin{isamarkuptext}%
This is Black's voting rule. It is composed of a function that determines
the Condorcet winner, i.e., the Pairwise Majority rule, and the Borda rule.
Whenever there exists no Condorcet winner, it elects the choice made by the
Borda rule, otherwise the Condorcet winner is elected.%
\end{isamarkuptext}\isamarkuptrue%
%
\isadelimdocument
%
\endisadelimdocument
%
\isatagdocument
%
\isamarkupsubsection{Definition%
}
\isamarkuptrue%
%
\endisatagdocument
{\isafolddocument}%
%
\isadelimdocument
%
\endisadelimdocument
\isacommand{fun}\isamarkupfalse%
\ blacks{\isacharunderscore}{\kern0pt}rule\ {\isacharcolon}{\kern0pt}{\isacharcolon}{\kern0pt}\ {\isachardoublequoteopen}{\isacharprime}{\kern0pt}a\ Electoral{\isacharunderscore}{\kern0pt}Module{\isachardoublequoteclose}\ \isakeyword{where}\isanewline
\ \ {\isachardoublequoteopen}blacks{\isacharunderscore}{\kern0pt}rule\ A\ p\ {\isacharequal}{\kern0pt}\ {\isacharparenleft}{\kern0pt}pairwise{\isacharunderscore}{\kern0pt}majority{\isacharunderscore}{\kern0pt}rule\ {\isasymtriangleright}\ borda{\isacharunderscore}{\kern0pt}rule{\isacharparenright}{\kern0pt}\ A\ p{\isachardoublequoteclose}\isanewline
\isanewline
\isacommand{fun}\isamarkupfalse%
\ blacks{\isacharunderscore}{\kern0pt}rule{\isacharunderscore}{\kern0pt}code\ {\isacharcolon}{\kern0pt}{\isacharcolon}{\kern0pt}\ {\isachardoublequoteopen}{\isacharprime}{\kern0pt}a\ Electoral{\isacharunderscore}{\kern0pt}Module{\isachardoublequoteclose}\ \isakeyword{where}\isanewline
\ \ {\isachardoublequoteopen}blacks{\isacharunderscore}{\kern0pt}rule{\isacharunderscore}{\kern0pt}code\ A\ p\ {\isacharequal}{\kern0pt}\ {\isacharparenleft}{\kern0pt}pairwise{\isacharunderscore}{\kern0pt}majority{\isacharunderscore}{\kern0pt}rule{\isacharunderscore}{\kern0pt}code\ {\isasymtriangleright}\ borda{\isacharunderscore}{\kern0pt}rule{\isacharunderscore}{\kern0pt}code{\isacharparenright}{\kern0pt}\ A\ p{\isachardoublequoteclose}\isanewline
%
\isadelimtheory
\isanewline
%
\endisadelimtheory
%
\isatagtheory
\isacommand{end}\isamarkupfalse%
%
\endisatagtheory
{\isafoldtheory}%
%
\isadelimtheory
%
\endisadelimtheory
%
\end{isabellebody}%
\endinput
%:%file=~/Documents/Studies/VotingRuleGenerator/virage/src/test/resources/old_theories/Blacks_Rule.thy%:%
%:%6=3%:%
%:%11=4%:%
%:%13=7%:%
%:%29=9%:%
%:%30=9%:%
%:%31=10%:%
%:%32=11%:%
%:%33=12%:%
%:%42=15%:%
%:%43=16%:%
%:%44=17%:%
%:%45=18%:%
%:%54=20%:%
%:%64=22%:%
%:%65=22%:%
%:%66=23%:%
%:%67=24%:%
%:%68=25%:%
%:%69=25%:%
%:%70=26%:%
%:%73=27%:%
%:%78=28%:%
%
\begin{isabellebody}%
\setisabellecontext{Nanson{\isacharunderscore}{\kern0pt}Baldwin{\isacharunderscore}{\kern0pt}Rule}%
%
\isadelimdocument
\isanewline
%
\endisadelimdocument
%
\isatagdocument
\isanewline
%
\isamarkupsection{Nanson-Baldwin Rule%
}
\isamarkuptrue%
%
\endisatagdocument
{\isafolddocument}%
%
\isadelimdocument
%
\endisadelimdocument
%
\isadelimtheory
%
\endisadelimtheory
%
\isatagtheory
\isacommand{theory}\isamarkupfalse%
\ Nanson{\isacharunderscore}{\kern0pt}Baldwin{\isacharunderscore}{\kern0pt}Rule\isanewline
\ \ \isakeyword{imports}\ {\isachardoublequoteopen}{\isachardot}{\kern0pt}{\isachardot}{\kern0pt}{\isacharslash}{\kern0pt}Compositional{\isacharunderscore}{\kern0pt}Framework{\isacharslash}{\kern0pt}Components{\isacharslash}{\kern0pt}Composites{\isacharslash}{\kern0pt}Composite{\isacharunderscore}{\kern0pt}Elimination{\isacharunderscore}{\kern0pt}Modules{\isachardoublequoteclose}\isanewline
\ \ \ \ \ \ \ \ \ \ {\isachardoublequoteopen}{\isachardot}{\kern0pt}{\isachardot}{\kern0pt}{\isacharslash}{\kern0pt}Compositional{\isacharunderscore}{\kern0pt}Framework{\isacharslash}{\kern0pt}Components{\isacharslash}{\kern0pt}Composites{\isacharslash}{\kern0pt}Composite{\isacharunderscore}{\kern0pt}Structures{\isachardoublequoteclose}\isanewline
\isakeyword{begin}%
\endisatagtheory
{\isafoldtheory}%
%
\isadelimtheory
%
\endisadelimtheory
%
\begin{isamarkuptext}%
This is the Nanson-Baldwin voting rule. It excludes alternatives with the
lowest Borda score from the set of possible winners and then adjusts the
Borda score to the new (remaining) set of still eligible alternatives.%
\end{isamarkuptext}\isamarkuptrue%
%
\isadelimdocument
%
\endisadelimdocument
%
\isatagdocument
%
\isamarkupsubsection{Definition%
}
\isamarkuptrue%
%
\endisatagdocument
{\isafolddocument}%
%
\isadelimdocument
%
\endisadelimdocument
\isacommand{fun}\isamarkupfalse%
\ nanson{\isacharunderscore}{\kern0pt}baldwin{\isacharunderscore}{\kern0pt}rule\ {\isacharcolon}{\kern0pt}{\isacharcolon}{\kern0pt}\ {\isachardoublequoteopen}{\isacharprime}{\kern0pt}a\ Electoral{\isacharunderscore}{\kern0pt}Module{\isachardoublequoteclose}\ \isakeyword{where}\isanewline
\ \ {\isachardoublequoteopen}nanson{\isacharunderscore}{\kern0pt}baldwin{\isacharunderscore}{\kern0pt}rule\ A\ p\ {\isacharequal}{\kern0pt}\isanewline
\ \ \ \ {\isacharparenleft}{\kern0pt}{\isacharparenleft}{\kern0pt}min{\isacharunderscore}{\kern0pt}eliminator\ borda{\isacharunderscore}{\kern0pt}score{\isacharparenright}{\kern0pt}\ {\isasymcirclearrowleft}\isactrlsub {\isasymexists}\isactrlsub {\isacharbang}{\kern0pt}\isactrlsub d{\isacharparenright}{\kern0pt}\ A\ p{\isachardoublequoteclose}\isanewline
%
\isadelimtheory
\isanewline
%
\endisadelimtheory
%
\isatagtheory
\isacommand{end}\isamarkupfalse%
%
\endisatagtheory
{\isafoldtheory}%
%
\isadelimtheory
%
\endisadelimtheory
%
\end{isabellebody}%
\endinput
%:%file=~/Documents/Studies/VotingRuleGenerator/virage/src/test/resources/verifiedVotingRuleConstruction/theories/Voting_Rules/Nanson_Baldwin_Rule.thy%:%
%:%6=3%:%
%:%11=4%:%
%:%13=7%:%
%:%29=9%:%
%:%30=9%:%
%:%31=10%:%
%:%32=11%:%
%:%33=12%:%
%:%42=15%:%
%:%43=16%:%
%:%44=17%:%
%:%53=19%:%
%:%63=21%:%
%:%64=21%:%
%:%65=22%:%
%:%66=23%:%
%:%69=24%:%
%:%74=25%:%
%
\begin{isabellebody}%
\setisabellecontext{Classic{\isacharunderscore}{\kern0pt}Nanson{\isacharunderscore}{\kern0pt}Rule}%
%
\isadelimdocument
\isanewline
%
\endisadelimdocument
%
\isatagdocument
\isanewline
%
\isamarkupsection{Classic Nanson Rule%
}
\isamarkuptrue%
%
\endisatagdocument
{\isafolddocument}%
%
\isadelimdocument
%
\endisadelimdocument
%
\isadelimtheory
%
\endisadelimtheory
%
\isatagtheory
\isacommand{theory}\isamarkupfalse%
\ Classic{\isacharunderscore}{\kern0pt}Nanson{\isacharunderscore}{\kern0pt}Rule\isanewline
\ \ \isakeyword{imports}\ {\isachardoublequoteopen}{\isachardot}{\kern0pt}{\isachardot}{\kern0pt}{\isacharslash}{\kern0pt}Compositional{\isacharunderscore}{\kern0pt}Framework{\isacharslash}{\kern0pt}Components{\isacharslash}{\kern0pt}Composites{\isacharslash}{\kern0pt}Composite{\isacharunderscore}{\kern0pt}Elimination{\isacharunderscore}{\kern0pt}Modules{\isachardoublequoteclose}\isanewline
\ \ \ \ \ \ \ \ \ \ {\isachardoublequoteopen}{\isachardot}{\kern0pt}{\isachardot}{\kern0pt}{\isacharslash}{\kern0pt}Compositional{\isacharunderscore}{\kern0pt}Framework{\isacharslash}{\kern0pt}Components{\isacharslash}{\kern0pt}Composites{\isacharslash}{\kern0pt}Composite{\isacharunderscore}{\kern0pt}Structures{\isachardoublequoteclose}\isanewline
\isakeyword{begin}%
\endisatagtheory
{\isafoldtheory}%
%
\isadelimtheory
%
\endisadelimtheory
%
\begin{isamarkuptext}%
This is the classic Nanson's voting rule, i.e., the rule that was originally
invented by Nanson, but not the Nanson-Baldwin rule. The idea is similar,
however, as alternatives with a Borda score less or equal than the average
Borda score are excluded. The Borda scores of the remaining alternatives
are hence adjusted to the new set of (still) eligible alternatives.%
\end{isamarkuptext}\isamarkuptrue%
%
\isadelimdocument
%
\endisadelimdocument
%
\isatagdocument
%
\isamarkupsubsection{Definition%
}
\isamarkuptrue%
%
\endisatagdocument
{\isafolddocument}%
%
\isadelimdocument
%
\endisadelimdocument
\isacommand{fun}\isamarkupfalse%
\ classic{\isacharunderscore}{\kern0pt}nanson{\isacharunderscore}{\kern0pt}rule\ {\isacharcolon}{\kern0pt}{\isacharcolon}{\kern0pt}\ {\isachardoublequoteopen}{\isacharprime}{\kern0pt}a\ Electoral{\isacharunderscore}{\kern0pt}Module{\isachardoublequoteclose}\ \isakeyword{where}\isanewline
\ \ {\isachardoublequoteopen}classic{\isacharunderscore}{\kern0pt}nanson{\isacharunderscore}{\kern0pt}rule\ A\ p\ {\isacharequal}{\kern0pt}\isanewline
\ \ \ \ {\isacharparenleft}{\kern0pt}{\isacharparenleft}{\kern0pt}leq{\isacharunderscore}{\kern0pt}average{\isacharunderscore}{\kern0pt}eliminator\ borda{\isacharunderscore}{\kern0pt}score{\isacharparenright}{\kern0pt}\ {\isasymcirclearrowleft}\isactrlsub {\isasymexists}\isactrlsub {\isacharbang}{\kern0pt}\isactrlsub d{\isacharparenright}{\kern0pt}\ A\ p{\isachardoublequoteclose}\isanewline
%
\isadelimtheory
\isanewline
%
\endisadelimtheory
%
\isatagtheory
\isacommand{end}\isamarkupfalse%
%
\endisatagtheory
{\isafoldtheory}%
%
\isadelimtheory
%
\endisadelimtheory
%
\end{isabellebody}%
\endinput
%:%file=~/Documents/Studies/VotingRuleGenerator/virage/src/test/resources/verifiedVotingRuleConstruction/theories/Voting_Rules/Classic_Nanson_Rule.thy%:%
%:%6=3%:%
%:%11=4%:%
%:%13=7%:%
%:%29=9%:%
%:%30=9%:%
%:%31=10%:%
%:%32=11%:%
%:%33=12%:%
%:%42=15%:%
%:%43=16%:%
%:%44=17%:%
%:%45=18%:%
%:%46=19%:%
%:%55=21%:%
%:%65=23%:%
%:%66=23%:%
%:%67=24%:%
%:%68=25%:%
%:%71=26%:%
%:%76=27%:%
%
\begin{isabellebody}%
\setisabellecontext{Schwartz{\isacharunderscore}{\kern0pt}Rule}%
%
\isadelimdocument
\isanewline
%
\endisadelimdocument
%
\isatagdocument
\isanewline
%
\isamarkupsection{Schwartz Rule%
}
\isamarkuptrue%
%
\endisatagdocument
{\isafolddocument}%
%
\isadelimdocument
%
\endisadelimdocument
%
\isadelimtheory
%
\endisadelimtheory
%
\isatagtheory
\isacommand{theory}\isamarkupfalse%
\ Schwartz{\isacharunderscore}{\kern0pt}Rule\isanewline
\ \ \isakeyword{imports}\ {\isachardoublequoteopen}{\isachardot}{\kern0pt}{\isachardot}{\kern0pt}{\isacharslash}{\kern0pt}Compositional{\isacharunderscore}{\kern0pt}Framework{\isacharslash}{\kern0pt}Components{\isacharslash}{\kern0pt}Composites{\isacharslash}{\kern0pt}Composite{\isacharunderscore}{\kern0pt}Elimination{\isacharunderscore}{\kern0pt}Modules{\isachardoublequoteclose}\isanewline
\ \ \ \ \ \ \ \ \ \ {\isachardoublequoteopen}{\isachardot}{\kern0pt}{\isachardot}{\kern0pt}{\isacharslash}{\kern0pt}Compositional{\isacharunderscore}{\kern0pt}Framework{\isacharslash}{\kern0pt}Components{\isacharslash}{\kern0pt}Composites{\isacharslash}{\kern0pt}Composite{\isacharunderscore}{\kern0pt}Structures{\isachardoublequoteclose}\isanewline
\isanewline
\isakeyword{begin}%
\endisatagtheory
{\isafoldtheory}%
%
\isadelimtheory
%
\endisadelimtheory
%
\begin{isamarkuptext}%
This is the Schwartz voting rule. Confusingly, it is sometimes also referred as
Nanson's rule. The Schwartz rule proceeds as in the classic Nanson's rule, but
excludes alternatives with a Borda score that is strictly less than the
average Borda score.%
\end{isamarkuptext}\isamarkuptrue%
%
\isadelimdocument
%
\endisadelimdocument
%
\isatagdocument
%
\isamarkupsubsection{Definition%
}
\isamarkuptrue%
%
\endisatagdocument
{\isafolddocument}%
%
\isadelimdocument
%
\endisadelimdocument
\isacommand{fun}\isamarkupfalse%
\ schwartz{\isacharunderscore}{\kern0pt}rule\ {\isacharcolon}{\kern0pt}{\isacharcolon}{\kern0pt}\ {\isachardoublequoteopen}{\isacharprime}{\kern0pt}a\ Electoral{\isacharunderscore}{\kern0pt}Module{\isachardoublequoteclose}\ \isakeyword{where}\isanewline
\ \ {\isachardoublequoteopen}schwartz{\isacharunderscore}{\kern0pt}rule\ A\ p\ {\isacharequal}{\kern0pt}\isanewline
\ \ \ \ {\isacharparenleft}{\kern0pt}{\isacharparenleft}{\kern0pt}less{\isacharunderscore}{\kern0pt}average{\isacharunderscore}{\kern0pt}eliminator\ borda{\isacharunderscore}{\kern0pt}score{\isacharparenright}{\kern0pt}\ {\isasymcirclearrowleft}\isactrlsub {\isasymexists}\isactrlsub {\isacharbang}{\kern0pt}\isactrlsub d{\isacharparenright}{\kern0pt}\ A\ p{\isachardoublequoteclose}\isanewline
%
\isadelimtheory
\isanewline
%
\endisadelimtheory
%
\isatagtheory
\isacommand{end}\isamarkupfalse%
%
\endisatagtheory
{\isafoldtheory}%
%
\isadelimtheory
%
\endisadelimtheory
%
\end{isabellebody}%
\endinput
%:%file=~/Documents/Studies/VotingRuleGenerator/virage/src/test/resources/verifiedVotingRuleConstruction/theories/Voting_Rules/Schwartz_Rule.thy%:%
%:%6=3%:%
%:%11=4%:%
%:%13=7%:%
%:%29=9%:%
%:%30=9%:%
%:%31=10%:%
%:%32=11%:%
%:%33=12%:%
%:%34=13%:%
%:%43=16%:%
%:%44=17%:%
%:%45=18%:%
%:%46=19%:%
%:%55=21%:%
%:%65=23%:%
%:%66=23%:%
%:%67=24%:%
%:%68=25%:%
%:%71=26%:%
%:%76=27%:%
%
\begin{isabellebody}%
\setisabellecontext{Monotonicity{\isacharunderscore}{\kern0pt}Properties}%
%
\isadelimtheory
%
\endisadelimtheory
%
\isatagtheory
\isacommand{theory}\isamarkupfalse%
\ Monotonicity{\isacharunderscore}{\kern0pt}Properties\isanewline
\ \ \isakeyword{imports}\ {\isachardoublequoteopen}{\isachardot}{\kern0pt}{\isachardot}{\kern0pt}{\isacharslash}{\kern0pt}Components{\isacharslash}{\kern0pt}Electoral{\isacharunderscore}{\kern0pt}Module{\isachardoublequoteclose}\isanewline
\ \ \ \ \ \ \ \ \ \ Result{\isacharunderscore}{\kern0pt}Properties\isanewline
\isanewline
\isakeyword{begin}%
\endisatagtheory
{\isafoldtheory}%
%
\isadelimtheory
\isanewline
%
\endisadelimtheory
\isanewline
\isanewline
\isacommand{definition}\isamarkupfalse%
\ defer{\isacharunderscore}{\kern0pt}lift{\isacharunderscore}{\kern0pt}invariance\ {\isacharcolon}{\kern0pt}{\isacharcolon}{\kern0pt}\ {\isachardoublequoteopen}{\isacharprime}{\kern0pt}a\ Electoral{\isacharunderscore}{\kern0pt}Module\ {\isasymRightarrow}\ bool{\isachardoublequoteclose}\ \isakeyword{where}\isanewline
\ \ {\isachardoublequoteopen}defer{\isacharunderscore}{\kern0pt}lift{\isacharunderscore}{\kern0pt}invariance\ m\ {\isasymequiv}\isanewline
\ \ \ \ electoral{\isacharunderscore}{\kern0pt}module\ m\ {\isasymand}\isanewline
\ \ \ \ \ \ {\isacharparenleft}{\kern0pt}{\isasymforall}A\ p\ q\ a{\isachardot}{\kern0pt}\isanewline
\ \ \ \ \ \ \ \ \ \ {\isacharparenleft}{\kern0pt}a\ {\isasymin}\ {\isacharparenleft}{\kern0pt}defer\ m\ A\ p{\isacharparenright}{\kern0pt}\ {\isasymand}\ lifted\ A\ p\ q\ a{\isacharparenright}{\kern0pt}\ {\isasymlongrightarrow}\ m\ A\ p\ {\isacharequal}{\kern0pt}\ m\ A\ q{\isacharparenright}{\kern0pt}{\isachardoublequoteclose}\isanewline
\isanewline
\isanewline
\isacommand{definition}\isamarkupfalse%
\ invariant{\isacharunderscore}{\kern0pt}monotonicity\ {\isacharcolon}{\kern0pt}{\isacharcolon}{\kern0pt}\ {\isachardoublequoteopen}{\isacharprime}{\kern0pt}a\ Electoral{\isacharunderscore}{\kern0pt}Module\ {\isasymRightarrow}\ bool{\isachardoublequoteclose}\ \isakeyword{where}\isanewline
\ \ {\isachardoublequoteopen}invariant{\isacharunderscore}{\kern0pt}monotonicity\ m\ {\isasymequiv}\isanewline
\ \ \ \ electoral{\isacharunderscore}{\kern0pt}module\ m\ {\isasymand}\isanewline
\ \ \ \ \ \ \ \ {\isacharparenleft}{\kern0pt}{\isasymforall}A\ p\ q\ a{\isachardot}{\kern0pt}\ {\isacharparenleft}{\kern0pt}a\ {\isasymin}\ elect\ m\ A\ p\ {\isasymand}\ lifted\ A\ p\ q\ a{\isacharparenright}{\kern0pt}\ {\isasymlongrightarrow}\isanewline
\ \ \ \ \ \ \ \ \ \ {\isacharparenleft}{\kern0pt}elect\ m\ A\ q\ {\isacharequal}{\kern0pt}\ elect\ m\ A\ p\ {\isasymor}\ elect\ m\ A\ q\ {\isacharequal}{\kern0pt}\ {\isacharbraceleft}{\kern0pt}a{\isacharbraceright}{\kern0pt}{\isacharparenright}{\kern0pt}{\isacharparenright}{\kern0pt}{\isachardoublequoteclose}\isanewline
\isanewline
\isanewline
\isacommand{definition}\isamarkupfalse%
\ defer{\isacharunderscore}{\kern0pt}invariant{\isacharunderscore}{\kern0pt}monotonicity\ {\isacharcolon}{\kern0pt}{\isacharcolon}{\kern0pt}\ {\isachardoublequoteopen}{\isacharprime}{\kern0pt}a\ Electoral{\isacharunderscore}{\kern0pt}Module\ {\isasymRightarrow}\ bool{\isachardoublequoteclose}\ \isakeyword{where}\isanewline
\ \ {\isachardoublequoteopen}defer{\isacharunderscore}{\kern0pt}invariant{\isacharunderscore}{\kern0pt}monotonicity\ m\ {\isasymequiv}\isanewline
\ \ \ \ electoral{\isacharunderscore}{\kern0pt}module\ m\ {\isasymand}\ non{\isacharunderscore}{\kern0pt}electing\ m\ {\isasymand}\isanewline
\ \ \ \ \ \ \ \ {\isacharparenleft}{\kern0pt}{\isasymforall}A\ p\ q\ a{\isachardot}{\kern0pt}\ {\isacharparenleft}{\kern0pt}a\ {\isasymin}\ defer\ m\ A\ p\ {\isasymand}\ lifted\ A\ p\ q\ a{\isacharparenright}{\kern0pt}\ {\isasymlongrightarrow}\isanewline
\ \ \ \ \ \ \ \ \ \ {\isacharparenleft}{\kern0pt}defer\ m\ A\ q\ {\isacharequal}{\kern0pt}\ defer\ m\ A\ p\ {\isasymor}\ defer\ m\ A\ q\ {\isacharequal}{\kern0pt}\ {\isacharbraceleft}{\kern0pt}a{\isacharbraceright}{\kern0pt}{\isacharparenright}{\kern0pt}{\isacharparenright}{\kern0pt}{\isachardoublequoteclose}\isanewline
\isanewline
\isanewline
\isacommand{definition}\isamarkupfalse%
\ defer{\isacharunderscore}{\kern0pt}monotonicity\ {\isacharcolon}{\kern0pt}{\isacharcolon}{\kern0pt}\ {\isachardoublequoteopen}{\isacharprime}{\kern0pt}a\ Electoral{\isacharunderscore}{\kern0pt}Module\ {\isasymRightarrow}\ bool{\isachardoublequoteclose}\ \isakeyword{where}\isanewline
\ \ {\isachardoublequoteopen}defer{\isacharunderscore}{\kern0pt}monotonicity\ m\ {\isasymequiv}\isanewline
\ \ \ \ electoral{\isacharunderscore}{\kern0pt}module\ m\ {\isasymand}\isanewline
\ \ \ \ \ \ {\isacharparenleft}{\kern0pt}{\isasymforall}A\ p\ q\ w{\isachardot}{\kern0pt}\isanewline
\ \ \ \ \ \ \ \ \ \ {\isacharparenleft}{\kern0pt}finite\ A\ {\isasymand}\ w\ {\isasymin}\ defer\ m\ A\ p\ {\isasymand}\ lifted\ A\ p\ q\ w{\isacharparenright}{\kern0pt}\ {\isasymlongrightarrow}\ w\ {\isasymin}\ defer\ m\ A\ q{\isacharparenright}{\kern0pt}{\isachardoublequoteclose}\isanewline
\isanewline
%
\isadelimtheory
\isanewline
%
\endisadelimtheory
%
\isatagtheory
\isacommand{end}\isamarkupfalse%
%
\endisatagtheory
{\isafoldtheory}%
%
\isadelimtheory
%
\endisadelimtheory
%
\end{isabellebody}%
\endinput
%:%file=~/Documents/Studies/VotingRuleGenerator/virage/src/test/resources/verifiedVotingRuleConstruction/theories/Compositional_Framework/Properties/Monotonicity_Properties.thy%:%
%:%10=1%:%
%:%11=1%:%
%:%12=2%:%
%:%13=3%:%
%:%14=4%:%
%:%15=5%:%
%:%20=5%:%
%:%23=6%:%
%:%24=10%:%
%:%25=11%:%
%:%26=11%:%
%:%27=12%:%
%:%30=15%:%
%:%31=16%:%
%:%32=21%:%
%:%33=22%:%
%:%34=22%:%
%:%35=23%:%
%:%38=26%:%
%:%39=27%:%
%:%40=32%:%
%:%41=33%:%
%:%42=33%:%
%:%43=34%:%
%:%46=37%:%
%:%47=38%:%
%:%48=42%:%
%:%49=43%:%
%:%50=43%:%
%:%51=44%:%
%:%54=47%:%
%:%55=48%:%
%:%58=49%:%
%:%63=50%:%
%
\begin{isabellebody}%
\setisabellecontext{Weak{\isacharunderscore}{\kern0pt}Monotonicity}%
%
\isadelimtheory
%
\endisadelimtheory
%
\isatagtheory
\isacommand{theory}\isamarkupfalse%
\ Weak{\isacharunderscore}{\kern0pt}Monotonicity\isanewline
\ \ \isakeyword{imports}\ {\isachardoublequoteopen}{\isachardot}{\kern0pt}{\isachardot}{\kern0pt}{\isacharslash}{\kern0pt}Compositional{\isacharunderscore}{\kern0pt}Framework{\isacharslash}{\kern0pt}Components{\isacharslash}{\kern0pt}Electoral{\isacharunderscore}{\kern0pt}Module{\isachardoublequoteclose}\isanewline
\isanewline
\isakeyword{begin}%
\endisatagtheory
{\isafoldtheory}%
%
\isadelimtheory
\isanewline
%
\endisadelimtheory
\isanewline
\isanewline
\isacommand{definition}\isamarkupfalse%
\ monotonicity\ {\isacharcolon}{\kern0pt}{\isacharcolon}{\kern0pt}\ {\isachardoublequoteopen}{\isacharprime}{\kern0pt}a\ Electoral{\isacharunderscore}{\kern0pt}Module\ {\isasymRightarrow}\ bool{\isachardoublequoteclose}\ \isakeyword{where}\isanewline
\ \ {\isachardoublequoteopen}monotonicity\ m\ {\isasymequiv}\isanewline
\ \ \ \ electoral{\isacharunderscore}{\kern0pt}module\ m\ {\isasymand}\isanewline
\ \ \ \ \ \ {\isacharparenleft}{\kern0pt}{\isasymforall}A\ p\ q\ w{\isachardot}{\kern0pt}\isanewline
\ \ \ \ \ \ \ \ \ \ {\isacharparenleft}{\kern0pt}finite\ A\ {\isasymand}\ w\ {\isasymin}\ elect\ m\ A\ p\ {\isasymand}\ lifted\ A\ p\ q\ w{\isacharparenright}{\kern0pt}\ {\isasymlongrightarrow}\ w\ {\isasymin}\ elect\ m\ A\ q{\isacharparenright}{\kern0pt}{\isachardoublequoteclose}\isanewline
%
\isadelimtheory
\isanewline
%
\endisadelimtheory
%
\isatagtheory
\isacommand{end}\isamarkupfalse%
%
\endisatagtheory
{\isafoldtheory}%
%
\isadelimtheory
%
\endisadelimtheory
%
\end{isabellebody}%
\endinput
%:%file=~/Documents/Studies/VotingRuleGenerator/virage/src/test/resources/verifiedVotingRuleConstruction/theories/Social_Choice_Properties/Weak_Monotonicity.thy%:%
%:%10=1%:%
%:%11=1%:%
%:%12=2%:%
%:%13=3%:%
%:%14=4%:%
%:%19=4%:%
%:%22=5%:%
%:%23=9%:%
%:%24=10%:%
%:%25=10%:%
%:%26=11%:%
%:%29=14%:%
%:%32=15%:%
%:%37=16%:%
%
\begin{isabellebody}%
\setisabellecontext{Result{\isacharunderscore}{\kern0pt}Facts}%
%
\isadelimtheory
%
\endisadelimtheory
%
\isatagtheory
\isacommand{theory}\isamarkupfalse%
\ Result{\isacharunderscore}{\kern0pt}Facts\isanewline
\ \ \isakeyword{imports}\ {\isachardoublequoteopen}{\isachardot}{\kern0pt}{\isachardot}{\kern0pt}{\isacharslash}{\kern0pt}Properties{\isacharslash}{\kern0pt}Result{\isacharunderscore}{\kern0pt}Properties{\isachardoublequoteclose}\isanewline
\ \ \ \ \ \ \ \ \ \ {\isachardoublequoteopen}{\isachardot}{\kern0pt}{\isachardot}{\kern0pt}{\isacharslash}{\kern0pt}Components{\isacharslash}{\kern0pt}Basic{\isacharunderscore}{\kern0pt}Modules{\isacharslash}{\kern0pt}Elect{\isacharunderscore}{\kern0pt}Module{\isachardoublequoteclose}\isanewline
\ \ \ \ \ \ \ \ \ \ {\isachardoublequoteopen}{\isachardot}{\kern0pt}{\isachardot}{\kern0pt}{\isacharslash}{\kern0pt}Components{\isacharslash}{\kern0pt}Basic{\isacharunderscore}{\kern0pt}Modules{\isacharslash}{\kern0pt}Plurality{\isacharunderscore}{\kern0pt}Module{\isachardoublequoteclose}\isanewline
\ \ \ \ \ \ \ \ \ \ {\isachardoublequoteopen}{\isachardot}{\kern0pt}{\isachardot}{\kern0pt}{\isacharslash}{\kern0pt}Components{\isacharslash}{\kern0pt}Basic{\isacharunderscore}{\kern0pt}Modules{\isacharslash}{\kern0pt}Defer{\isacharunderscore}{\kern0pt}Module{\isachardoublequoteclose}\isanewline
\ \ \ \ \ \ \ \ \ \ {\isachardoublequoteopen}{\isachardot}{\kern0pt}{\isachardot}{\kern0pt}{\isacharslash}{\kern0pt}Components{\isacharslash}{\kern0pt}Basic{\isacharunderscore}{\kern0pt}Modules{\isacharslash}{\kern0pt}Drop{\isacharunderscore}{\kern0pt}Module{\isachardoublequoteclose}\isanewline
\ \ \ \ \ \ \ \ \ \ {\isachardoublequoteopen}{\isachardot}{\kern0pt}{\isachardot}{\kern0pt}{\isacharslash}{\kern0pt}Components{\isacharslash}{\kern0pt}Basic{\isacharunderscore}{\kern0pt}Modules{\isacharslash}{\kern0pt}Pass{\isacharunderscore}{\kern0pt}Module{\isachardoublequoteclose}\isanewline
\ \ \ \ \ \ \ \ \ \ {\isachardoublequoteopen}{\isachardot}{\kern0pt}{\isachardot}{\kern0pt}{\isacharslash}{\kern0pt}Components{\isacharslash}{\kern0pt}Compositional{\isacharunderscore}{\kern0pt}Structures{\isacharslash}{\kern0pt}Revision{\isacharunderscore}{\kern0pt}Composition{\isachardoublequoteclose}\isanewline
\ \ \ \ \ \ \ \ \ \ {\isachardoublequoteopen}{\isachardot}{\kern0pt}{\isachardot}{\kern0pt}{\isacharslash}{\kern0pt}Components{\isacharslash}{\kern0pt}Composites{\isacharslash}{\kern0pt}Composite{\isacharunderscore}{\kern0pt}Elimination{\isacharunderscore}{\kern0pt}Modules{\isachardoublequoteclose}\isanewline
\isanewline
\isakeyword{begin}%
\endisatagtheory
{\isafoldtheory}%
%
\isadelimtheory
\isanewline
%
\endisadelimtheory
\isanewline
\isacommand{theorem}\isamarkupfalse%
\ elect{\isacharunderscore}{\kern0pt}mod{\isacharunderscore}{\kern0pt}electing{\isacharbrackleft}{\kern0pt}simp{\isacharbrackright}{\kern0pt}{\isacharcolon}{\kern0pt}\ {\isachardoublequoteopen}electing\ elect{\isacharunderscore}{\kern0pt}module{\isachardoublequoteclose}\isanewline
%
\isadelimproof
\ \ %
\endisadelimproof
%
\isatagproof
\isacommand{unfolding}\isamarkupfalse%
\ electing{\isacharunderscore}{\kern0pt}def\isanewline
\ \ \isacommand{by}\isamarkupfalse%
\ simp%
\endisatagproof
{\isafoldproof}%
%
\isadelimproof
\isanewline
%
\endisadelimproof
\isanewline
\isacommand{lemma}\isamarkupfalse%
\ plurality{\isacharunderscore}{\kern0pt}electing{\isadigit{2}}{\isacharcolon}{\kern0pt}\ {\isachardoublequoteopen}{\isasymforall}A\ p{\isachardot}{\kern0pt}\isanewline
\ \ \ \ \ \ \ \ \ \ \ \ \ \ \ \ \ \ \ \ \ \ \ \ \ \ \ \ \ \ {\isacharparenleft}{\kern0pt}A\ {\isasymnoteq}\ {\isacharbraceleft}{\kern0pt}{\isacharbraceright}{\kern0pt}\ {\isasymand}\ finite{\isacharunderscore}{\kern0pt}profile\ A\ p{\isacharparenright}{\kern0pt}\ {\isasymlongrightarrow}\isanewline
\ \ \ \ \ \ \ \ \ \ \ \ \ \ \ \ \ \ \ \ \ \ \ \ \ \ \ \ \ \ \ \ elect\ plurality\ A\ p\ {\isasymnoteq}\ {\isacharbraceleft}{\kern0pt}{\isacharbraceright}{\kern0pt}{\isachardoublequoteclose}\isanewline
%
\isadelimproof
%
\endisadelimproof
%
\isatagproof
\isacommand{proof}\isamarkupfalse%
\ {\isacharparenleft}{\kern0pt}intro\ allI\ impI\ conjI{\isacharparenright}{\kern0pt}\isanewline
\ \ \isacommand{fix}\isamarkupfalse%
\isanewline
\ \ \ \ A\ {\isacharcolon}{\kern0pt}{\isacharcolon}{\kern0pt}\ {\isachardoublequoteopen}{\isacharprime}{\kern0pt}a\ set{\isachardoublequoteclose}\ \isakeyword{and}\isanewline
\ \ \ \ p\ {\isacharcolon}{\kern0pt}{\isacharcolon}{\kern0pt}\ {\isachardoublequoteopen}{\isacharprime}{\kern0pt}a\ Profile{\isachardoublequoteclose}\isanewline
\ \ \isacommand{assume}\isamarkupfalse%
\isanewline
\ \ \ \ assm{\isadigit{0}}{\isacharcolon}{\kern0pt}\ {\isachardoublequoteopen}A\ {\isasymnoteq}\ {\isacharbraceleft}{\kern0pt}{\isacharbraceright}{\kern0pt}\ {\isasymand}\ finite{\isacharunderscore}{\kern0pt}profile\ A\ p{\isachardoublequoteclose}\isanewline
\ \ \isacommand{show}\isamarkupfalse%
\isanewline
\ \ \ \ {\isachardoublequoteopen}elect\ plurality\ A\ p\ {\isasymnoteq}\ {\isacharbraceleft}{\kern0pt}{\isacharbraceright}{\kern0pt}{\isachardoublequoteclose}\isanewline
\ \ \isacommand{proof}\isamarkupfalse%
\isanewline
\ \ \ \ \isacommand{obtain}\isamarkupfalse%
\ max\ \isakeyword{where}\isanewline
\ \ \ \ \ \ max{\isacharcolon}{\kern0pt}\ {\isachardoublequoteopen}max\ {\isacharequal}{\kern0pt}\ Max{\isacharparenleft}{\kern0pt}win{\isacharunderscore}{\kern0pt}count\ p\ {\isacharbackquote}{\kern0pt}\ A{\isacharparenright}{\kern0pt}{\isachardoublequoteclose}\isanewline
\ \ \ \ \ \ \isacommand{by}\isamarkupfalse%
\ simp\isanewline
\ \ \ \ \isacommand{then}\isamarkupfalse%
\ \isacommand{obtain}\isamarkupfalse%
\ a\ \isakeyword{where}\isanewline
\ \ \ \ \ \ a{\isacharcolon}{\kern0pt}\ {\isachardoublequoteopen}win{\isacharunderscore}{\kern0pt}count\ p\ a\ {\isacharequal}{\kern0pt}\ max\ {\isasymand}\ a\ {\isasymin}\ A{\isachardoublequoteclose}\isanewline
\ \ \ \ \ \ \isacommand{using}\isamarkupfalse%
\ Max{\isacharunderscore}{\kern0pt}in\ assm{\isadigit{0}}\ empty{\isacharunderscore}{\kern0pt}is{\isacharunderscore}{\kern0pt}image\isanewline
\ \ \ \ \ \ \ \ \ \ \ \ finite{\isacharunderscore}{\kern0pt}imageI\ imageE\isanewline
\ \ \ \ \ \ \isacommand{by}\isamarkupfalse%
\ {\isacharparenleft}{\kern0pt}metis\ {\isacharparenleft}{\kern0pt}no{\isacharunderscore}{\kern0pt}types{\isacharcomma}{\kern0pt}\ lifting{\isacharparenright}{\kern0pt}{\isacharparenright}{\kern0pt}\isanewline
\ \ \ \ \isacommand{hence}\isamarkupfalse%
\isanewline
\ \ \ \ \ \ {\isachardoublequoteopen}{\isasymforall}x\ {\isasymin}\ A{\isachardot}{\kern0pt}\ win{\isacharunderscore}{\kern0pt}count\ p\ x\ {\isasymle}\ win{\isacharunderscore}{\kern0pt}count\ p\ a{\isachardoublequoteclose}\isanewline
\ \ \ \ \ \ \isacommand{by}\isamarkupfalse%
\ {\isacharparenleft}{\kern0pt}simp\ add{\isacharcolon}{\kern0pt}\ max\ assm{\isadigit{0}}{\isacharparenright}{\kern0pt}\isanewline
\ \ \ \ \isacommand{moreover}\isamarkupfalse%
\ \isacommand{have}\isamarkupfalse%
\isanewline
\ \ \ \ \ \ {\isachardoublequoteopen}a\ {\isasymin}\ A{\isachardoublequoteclose}\isanewline
\ \ \ \ \ \ \isacommand{using}\isamarkupfalse%
\ a\isanewline
\ \ \ \ \ \ \isacommand{by}\isamarkupfalse%
\ simp\isanewline
\ \ \ \ \isacommand{ultimately}\isamarkupfalse%
\ \isacommand{have}\isamarkupfalse%
\isanewline
\ \ \ \ \ \ {\isachardoublequoteopen}a\ {\isasymin}\ {\isacharbraceleft}{\kern0pt}a\ {\isasymin}\ A{\isachardot}{\kern0pt}\ {\isasymforall}x\ {\isasymin}\ A{\isachardot}{\kern0pt}\ win{\isacharunderscore}{\kern0pt}count\ p\ x\ {\isasymle}\ win{\isacharunderscore}{\kern0pt}count\ p\ a{\isacharbraceright}{\kern0pt}{\isachardoublequoteclose}\isanewline
\ \ \ \ \ \ \isacommand{by}\isamarkupfalse%
\ blast\isanewline
\ \ \ \ \isacommand{hence}\isamarkupfalse%
\ a{\isacharunderscore}{\kern0pt}elem{\isacharcolon}{\kern0pt}\isanewline
\ \ \ \ \ \ {\isachardoublequoteopen}a\ {\isasymin}\ elect\ plurality\ A\ p{\isachardoublequoteclose}\isanewline
\ \ \ \ \ \ \isacommand{by}\isamarkupfalse%
\ simp\isanewline
\ \ \ \ \isacommand{assume}\isamarkupfalse%
\isanewline
\ \ \ \ \ \ assm{\isadigit{1}}{\isacharcolon}{\kern0pt}\ {\isachardoublequoteopen}elect\ plurality\ A\ p\ {\isacharequal}{\kern0pt}\ {\isacharbraceleft}{\kern0pt}{\isacharbraceright}{\kern0pt}{\isachardoublequoteclose}\isanewline
\ \ \ \ \isacommand{thus}\isamarkupfalse%
\ {\isachardoublequoteopen}False{\isachardoublequoteclose}\isanewline
\ \ \ \ \ \ \isacommand{using}\isamarkupfalse%
\ a{\isacharunderscore}{\kern0pt}elem\ assm{\isadigit{1}}\ all{\isacharunderscore}{\kern0pt}not{\isacharunderscore}{\kern0pt}in{\isacharunderscore}{\kern0pt}conv\isanewline
\ \ \ \ \ \ \isacommand{by}\isamarkupfalse%
\ metis\isanewline
\ \ \isacommand{qed}\isamarkupfalse%
\isanewline
\isacommand{qed}\isamarkupfalse%
%
\endisatagproof
{\isafoldproof}%
%
\isadelimproof
\isanewline
%
\endisadelimproof
\isanewline
\isanewline
\isacommand{theorem}\isamarkupfalse%
\ plurality{\isacharunderscore}{\kern0pt}electing{\isacharbrackleft}{\kern0pt}simp{\isacharbrackright}{\kern0pt}{\isacharcolon}{\kern0pt}\ {\isachardoublequoteopen}electing\ plurality{\isachardoublequoteclose}\isanewline
%
\isadelimproof
%
\endisadelimproof
%
\isatagproof
\isacommand{proof}\isamarkupfalse%
\ {\isacharminus}{\kern0pt}\isanewline
\ \ \isacommand{have}\isamarkupfalse%
\ {\isachardoublequoteopen}electoral{\isacharunderscore}{\kern0pt}module\ plurality\ {\isasymand}\isanewline
\ \ \ \ \ \ {\isacharparenleft}{\kern0pt}{\isasymforall}A\ p{\isachardot}{\kern0pt}\ {\isacharparenleft}{\kern0pt}A\ {\isasymnoteq}\ {\isacharbraceleft}{\kern0pt}{\isacharbraceright}{\kern0pt}\ {\isasymand}\ finite{\isacharunderscore}{\kern0pt}profile\ A\ p{\isacharparenright}{\kern0pt}\ {\isasymlongrightarrow}\ elect\ plurality\ A\ p\ {\isasymnoteq}\ {\isacharbraceleft}{\kern0pt}{\isacharbraceright}{\kern0pt}{\isacharparenright}{\kern0pt}{\isachardoublequoteclose}\isanewline
\ \ \isacommand{proof}\isamarkupfalse%
\isanewline
\ \ \ \ \isacommand{show}\isamarkupfalse%
\ {\isachardoublequoteopen}electoral{\isacharunderscore}{\kern0pt}module\ plurality{\isachardoublequoteclose}\isanewline
\ \ \ \ \ \ \isacommand{by}\isamarkupfalse%
\ simp\isanewline
\ \ \isacommand{next}\isamarkupfalse%
\isanewline
\ \ \ \ \isacommand{show}\isamarkupfalse%
\ {\isachardoublequoteopen}{\isacharparenleft}{\kern0pt}{\isasymforall}A\ p{\isachardot}{\kern0pt}\ {\isacharparenleft}{\kern0pt}A\ {\isasymnoteq}\ {\isacharbraceleft}{\kern0pt}{\isacharbraceright}{\kern0pt}\ {\isasymand}\ finite{\isacharunderscore}{\kern0pt}profile\ A\ p{\isacharparenright}{\kern0pt}\ {\isasymlongrightarrow}\ elect\ plurality\ A\ p\ {\isasymnoteq}\ {\isacharbraceleft}{\kern0pt}{\isacharbraceright}{\kern0pt}{\isacharparenright}{\kern0pt}{\isachardoublequoteclose}\isanewline
\ \ \ \ \ \ \isacommand{using}\isamarkupfalse%
\ plurality{\isacharunderscore}{\kern0pt}electing{\isadigit{2}}\isanewline
\ \ \ \ \ \ \isacommand{by}\isamarkupfalse%
\ metis\isanewline
\ \ \isacommand{qed}\isamarkupfalse%
\isanewline
\ \ \isacommand{thus}\isamarkupfalse%
\ {\isacharquery}{\kern0pt}thesis\isanewline
\ \ \ \ \ \ \isacommand{by}\isamarkupfalse%
\ {\isacharparenleft}{\kern0pt}simp\ add{\isacharcolon}{\kern0pt}\ electing{\isacharunderscore}{\kern0pt}def{\isacharparenright}{\kern0pt}\isanewline
\ \ \isacommand{qed}\isamarkupfalse%
%
\endisatagproof
{\isafoldproof}%
%
\isadelimproof
\isanewline
%
\endisadelimproof
\isanewline
\isacommand{theorem}\isamarkupfalse%
\ def{\isacharunderscore}{\kern0pt}mod{\isacharunderscore}{\kern0pt}non{\isacharunderscore}{\kern0pt}electing{\isacharcolon}{\kern0pt}\ {\isachardoublequoteopen}non{\isacharunderscore}{\kern0pt}electing\ defer{\isacharunderscore}{\kern0pt}module{\isachardoublequoteclose}\isanewline
%
\isadelimproof
\ \ %
\endisadelimproof
%
\isatagproof
\isacommand{unfolding}\isamarkupfalse%
\ non{\isacharunderscore}{\kern0pt}electing{\isacharunderscore}{\kern0pt}def\isanewline
\ \ \isacommand{by}\isamarkupfalse%
\ simp%
\endisatagproof
{\isafoldproof}%
%
\isadelimproof
\isanewline
%
\endisadelimproof
\isanewline
\isanewline
\isanewline
\isacommand{theorem}\isamarkupfalse%
\ drop{\isacharunderscore}{\kern0pt}mod{\isacharunderscore}{\kern0pt}non{\isacharunderscore}{\kern0pt}electing{\isacharbrackleft}{\kern0pt}simp{\isacharbrackright}{\kern0pt}{\isacharcolon}{\kern0pt}\isanewline
\ \ \isakeyword{assumes}\ order{\isacharcolon}{\kern0pt}\ {\isachardoublequoteopen}linear{\isacharunderscore}{\kern0pt}order\ r{\isachardoublequoteclose}\isanewline
\ \ \isakeyword{shows}\ {\isachardoublequoteopen}non{\isacharunderscore}{\kern0pt}electing\ {\isacharparenleft}{\kern0pt}drop{\isacharunderscore}{\kern0pt}module\ n\ r{\isacharparenright}{\kern0pt}{\isachardoublequoteclose}\isanewline
%
\isadelimproof
\ \ %
\endisadelimproof
%
\isatagproof
\isacommand{by}\isamarkupfalse%
\ {\isacharparenleft}{\kern0pt}simp\ add{\isacharcolon}{\kern0pt}\ non{\isacharunderscore}{\kern0pt}electing{\isacharunderscore}{\kern0pt}def\ order{\isacharparenright}{\kern0pt}%
\endisatagproof
{\isafoldproof}%
%
\isadelimproof
\isanewline
%
\endisadelimproof
\isanewline
\isacommand{lemma}\isamarkupfalse%
\ elim{\isacharunderscore}{\kern0pt}mod{\isacharunderscore}{\kern0pt}non{\isacharunderscore}{\kern0pt}electing{\isacharcolon}{\kern0pt}\isanewline
\ \ \isakeyword{assumes}\ profile{\isacharcolon}{\kern0pt}\ {\isachardoublequoteopen}finite{\isacharunderscore}{\kern0pt}profile\ A\ p{\isachardoublequoteclose}\isanewline
\ \ \isakeyword{shows}\ {\isachardoublequoteopen}non{\isacharunderscore}{\kern0pt}electing\ {\isacharparenleft}{\kern0pt}elimination{\isacharunderscore}{\kern0pt}module\ e\ t\ r\ {\isacharparenright}{\kern0pt}{\isachardoublequoteclose}\isanewline
%
\isadelimproof
\ \ %
\endisadelimproof
%
\isatagproof
\isacommand{by}\isamarkupfalse%
\ {\isacharparenleft}{\kern0pt}simp\ add{\isacharcolon}{\kern0pt}\ non{\isacharunderscore}{\kern0pt}electing{\isacharunderscore}{\kern0pt}def{\isacharparenright}{\kern0pt}%
\endisatagproof
{\isafoldproof}%
%
\isadelimproof
\isanewline
%
\endisadelimproof
\isanewline
\isacommand{lemma}\isamarkupfalse%
\ less{\isacharunderscore}{\kern0pt}elim{\isacharunderscore}{\kern0pt}non{\isacharunderscore}{\kern0pt}electing{\isacharcolon}{\kern0pt}\isanewline
\ \ \isakeyword{assumes}\ profile{\isacharcolon}{\kern0pt}\ {\isachardoublequoteopen}finite{\isacharunderscore}{\kern0pt}profile\ A\ p{\isachardoublequoteclose}\isanewline
\ \ \isakeyword{shows}\ {\isachardoublequoteopen}non{\isacharunderscore}{\kern0pt}electing\ {\isacharparenleft}{\kern0pt}less{\isacharunderscore}{\kern0pt}eliminator\ e\ t{\isacharparenright}{\kern0pt}{\isachardoublequoteclose}\isanewline
%
\isadelimproof
\ \ %
\endisadelimproof
%
\isatagproof
\isacommand{using}\isamarkupfalse%
\ elim{\isacharunderscore}{\kern0pt}mod{\isacharunderscore}{\kern0pt}non{\isacharunderscore}{\kern0pt}electing\ profile\ less{\isacharunderscore}{\kern0pt}elim{\isacharunderscore}{\kern0pt}sound\isanewline
\ \ \isacommand{by}\isamarkupfalse%
\ {\isacharparenleft}{\kern0pt}simp\ add{\isacharcolon}{\kern0pt}\ non{\isacharunderscore}{\kern0pt}electing{\isacharunderscore}{\kern0pt}def{\isacharparenright}{\kern0pt}%
\endisatagproof
{\isafoldproof}%
%
\isadelimproof
\isanewline
%
\endisadelimproof
\isanewline
\isacommand{lemma}\isamarkupfalse%
\ leq{\isacharunderscore}{\kern0pt}elim{\isacharunderscore}{\kern0pt}non{\isacharunderscore}{\kern0pt}electing{\isacharcolon}{\kern0pt}\isanewline
\ \ \isakeyword{assumes}\ profile{\isacharcolon}{\kern0pt}\ {\isachardoublequoteopen}finite{\isacharunderscore}{\kern0pt}profile\ A\ p{\isachardoublequoteclose}\isanewline
\ \ \isakeyword{shows}\ {\isachardoublequoteopen}non{\isacharunderscore}{\kern0pt}electing\ {\isacharparenleft}{\kern0pt}leq{\isacharunderscore}{\kern0pt}eliminator\ e\ t{\isacharparenright}{\kern0pt}{\isachardoublequoteclose}\isanewline
%
\isadelimproof
%
\endisadelimproof
%
\isatagproof
\isacommand{proof}\isamarkupfalse%
\ {\isacharminus}{\kern0pt}\isanewline
\ \ \isacommand{have}\isamarkupfalse%
\ {\isachardoublequoteopen}non{\isacharunderscore}{\kern0pt}electing\ {\isacharparenleft}{\kern0pt}elimination{\isacharunderscore}{\kern0pt}module\ e\ t\ {\isacharparenleft}{\kern0pt}{\isasymle}{\isacharparenright}{\kern0pt}{\isacharparenright}{\kern0pt}{\isachardoublequoteclose}\isanewline
\ \ \ \ \isacommand{by}\isamarkupfalse%
\ {\isacharparenleft}{\kern0pt}simp\ add{\isacharcolon}{\kern0pt}\ non{\isacharunderscore}{\kern0pt}electing{\isacharunderscore}{\kern0pt}def{\isacharparenright}{\kern0pt}\isanewline
\ \ \isacommand{thus}\isamarkupfalse%
\ {\isacharquery}{\kern0pt}thesis\isanewline
\ \ \ \ \isacommand{by}\isamarkupfalse%
\ {\isacharparenleft}{\kern0pt}simp\ add{\isacharcolon}{\kern0pt}\ non{\isacharunderscore}{\kern0pt}electing{\isacharunderscore}{\kern0pt}def{\isacharparenright}{\kern0pt}\isanewline
\isacommand{qed}\isamarkupfalse%
%
\endisatagproof
{\isafoldproof}%
%
\isadelimproof
\isanewline
%
\endisadelimproof
\isanewline
\isacommand{lemma}\isamarkupfalse%
\ max{\isacharunderscore}{\kern0pt}elim{\isacharunderscore}{\kern0pt}non{\isacharunderscore}{\kern0pt}electing{\isacharcolon}{\kern0pt}\isanewline
\ \ \isakeyword{assumes}\ profile{\isacharcolon}{\kern0pt}\ {\isachardoublequoteopen}finite{\isacharunderscore}{\kern0pt}profile\ A\ p{\isachardoublequoteclose}\isanewline
\ \ \isakeyword{shows}\ {\isachardoublequoteopen}non{\isacharunderscore}{\kern0pt}electing\ {\isacharparenleft}{\kern0pt}max{\isacharunderscore}{\kern0pt}eliminator\ e{\isacharparenright}{\kern0pt}{\isachardoublequoteclose}\isanewline
%
\isadelimproof
%
\endisadelimproof
%
\isatagproof
\isacommand{proof}\isamarkupfalse%
\ {\isacharminus}{\kern0pt}\isanewline
\ \ \isacommand{have}\isamarkupfalse%
\ {\isachardoublequoteopen}non{\isacharunderscore}{\kern0pt}electing\ {\isacharparenleft}{\kern0pt}elimination{\isacharunderscore}{\kern0pt}module\ e\ t\ {\isacharparenleft}{\kern0pt}{\isacharless}{\kern0pt}{\isacharparenright}{\kern0pt}{\isacharparenright}{\kern0pt}{\isachardoublequoteclose}\isanewline
\ \ \ \ \isacommand{by}\isamarkupfalse%
\ {\isacharparenleft}{\kern0pt}simp\ add{\isacharcolon}{\kern0pt}\ non{\isacharunderscore}{\kern0pt}electing{\isacharunderscore}{\kern0pt}def{\isacharparenright}{\kern0pt}\isanewline
\ \ \isacommand{thus}\isamarkupfalse%
\ {\isacharquery}{\kern0pt}thesis\isanewline
\ \ \ \ \isacommand{by}\isamarkupfalse%
\ {\isacharparenleft}{\kern0pt}simp\ add{\isacharcolon}{\kern0pt}\ non{\isacharunderscore}{\kern0pt}electing{\isacharunderscore}{\kern0pt}def{\isacharparenright}{\kern0pt}\isanewline
\isacommand{qed}\isamarkupfalse%
%
\endisatagproof
{\isafoldproof}%
%
\isadelimproof
\isanewline
%
\endisadelimproof
\isanewline
\isacommand{lemma}\isamarkupfalse%
\ min{\isacharunderscore}{\kern0pt}elim{\isacharunderscore}{\kern0pt}non{\isacharunderscore}{\kern0pt}electing{\isacharcolon}{\kern0pt}\isanewline
\ \ \isakeyword{assumes}\ profile{\isacharcolon}{\kern0pt}\ {\isachardoublequoteopen}finite{\isacharunderscore}{\kern0pt}profile\ A\ p{\isachardoublequoteclose}\isanewline
\ \ \isakeyword{shows}\ {\isachardoublequoteopen}non{\isacharunderscore}{\kern0pt}electing\ {\isacharparenleft}{\kern0pt}min{\isacharunderscore}{\kern0pt}eliminator\ e{\isacharparenright}{\kern0pt}{\isachardoublequoteclose}\isanewline
%
\isadelimproof
%
\endisadelimproof
%
\isatagproof
\isacommand{proof}\isamarkupfalse%
\ {\isacharminus}{\kern0pt}\isanewline
\ \ \isacommand{have}\isamarkupfalse%
\ {\isachardoublequoteopen}non{\isacharunderscore}{\kern0pt}electing\ {\isacharparenleft}{\kern0pt}elimination{\isacharunderscore}{\kern0pt}module\ e\ t\ {\isacharparenleft}{\kern0pt}{\isacharless}{\kern0pt}{\isacharparenright}{\kern0pt}{\isacharparenright}{\kern0pt}{\isachardoublequoteclose}\isanewline
\ \ \ \ \isacommand{by}\isamarkupfalse%
\ {\isacharparenleft}{\kern0pt}simp\ add{\isacharcolon}{\kern0pt}\ non{\isacharunderscore}{\kern0pt}electing{\isacharunderscore}{\kern0pt}def{\isacharparenright}{\kern0pt}\isanewline
\ \ \isacommand{thus}\isamarkupfalse%
\ {\isacharquery}{\kern0pt}thesis\isanewline
\ \ \ \ \isacommand{by}\isamarkupfalse%
\ {\isacharparenleft}{\kern0pt}simp\ add{\isacharcolon}{\kern0pt}\ non{\isacharunderscore}{\kern0pt}electing{\isacharunderscore}{\kern0pt}def{\isacharparenright}{\kern0pt}\isanewline
\isacommand{qed}\isamarkupfalse%
%
\endisatagproof
{\isafoldproof}%
%
\isadelimproof
\isanewline
%
\endisadelimproof
\isanewline
\isacommand{lemma}\isamarkupfalse%
\ less{\isacharunderscore}{\kern0pt}avg{\isacharunderscore}{\kern0pt}elim{\isacharunderscore}{\kern0pt}non{\isacharunderscore}{\kern0pt}electing{\isacharcolon}{\kern0pt}\isanewline
\ \ \isakeyword{assumes}\ profile{\isacharcolon}{\kern0pt}\ {\isachardoublequoteopen}finite{\isacharunderscore}{\kern0pt}profile\ A\ p{\isachardoublequoteclose}\isanewline
\ \ \isakeyword{shows}\ {\isachardoublequoteopen}non{\isacharunderscore}{\kern0pt}electing\ {\isacharparenleft}{\kern0pt}less{\isacharunderscore}{\kern0pt}average{\isacharunderscore}{\kern0pt}eliminator\ e{\isacharparenright}{\kern0pt}{\isachardoublequoteclose}\isanewline
%
\isadelimproof
%
\endisadelimproof
%
\isatagproof
\isacommand{proof}\isamarkupfalse%
\ {\isacharminus}{\kern0pt}\isanewline
\ \ \isacommand{have}\isamarkupfalse%
\ {\isachardoublequoteopen}non{\isacharunderscore}{\kern0pt}electing\ {\isacharparenleft}{\kern0pt}elimination{\isacharunderscore}{\kern0pt}module\ e\ t\ {\isacharparenleft}{\kern0pt}{\isacharless}{\kern0pt}{\isacharparenright}{\kern0pt}{\isacharparenright}{\kern0pt}{\isachardoublequoteclose}\isanewline
\ \ \ \ \isacommand{by}\isamarkupfalse%
\ {\isacharparenleft}{\kern0pt}simp\ add{\isacharcolon}{\kern0pt}\ non{\isacharunderscore}{\kern0pt}electing{\isacharunderscore}{\kern0pt}def{\isacharparenright}{\kern0pt}\isanewline
\ \ \isacommand{thus}\isamarkupfalse%
\ {\isacharquery}{\kern0pt}thesis\isanewline
\ \ \ \ \isacommand{by}\isamarkupfalse%
\ {\isacharparenleft}{\kern0pt}simp\ add{\isacharcolon}{\kern0pt}\ non{\isacharunderscore}{\kern0pt}electing{\isacharunderscore}{\kern0pt}def{\isacharparenright}{\kern0pt}\isanewline
\isacommand{qed}\isamarkupfalse%
%
\endisatagproof
{\isafoldproof}%
%
\isadelimproof
\isanewline
%
\endisadelimproof
\isanewline
\isacommand{lemma}\isamarkupfalse%
\ leq{\isacharunderscore}{\kern0pt}avg{\isacharunderscore}{\kern0pt}elim{\isacharunderscore}{\kern0pt}non{\isacharunderscore}{\kern0pt}electing{\isacharcolon}{\kern0pt}\isanewline
\ \ \isakeyword{assumes}\ profile{\isacharcolon}{\kern0pt}\ {\isachardoublequoteopen}finite{\isacharunderscore}{\kern0pt}profile\ A\ p{\isachardoublequoteclose}\isanewline
\ \ \isakeyword{shows}\ {\isachardoublequoteopen}non{\isacharunderscore}{\kern0pt}electing\ {\isacharparenleft}{\kern0pt}leq{\isacharunderscore}{\kern0pt}average{\isacharunderscore}{\kern0pt}eliminator\ e{\isacharparenright}{\kern0pt}{\isachardoublequoteclose}\isanewline
%
\isadelimproof
%
\endisadelimproof
%
\isatagproof
\isacommand{proof}\isamarkupfalse%
\ {\isacharminus}{\kern0pt}\isanewline
\ \ \isacommand{have}\isamarkupfalse%
\ {\isachardoublequoteopen}non{\isacharunderscore}{\kern0pt}electing\ {\isacharparenleft}{\kern0pt}elimination{\isacharunderscore}{\kern0pt}module\ e\ t\ {\isacharparenleft}{\kern0pt}{\isasymle}{\isacharparenright}{\kern0pt}{\isacharparenright}{\kern0pt}{\isachardoublequoteclose}\isanewline
\ \ \ \ \isacommand{by}\isamarkupfalse%
\ {\isacharparenleft}{\kern0pt}simp\ add{\isacharcolon}{\kern0pt}\ non{\isacharunderscore}{\kern0pt}electing{\isacharunderscore}{\kern0pt}def{\isacharparenright}{\kern0pt}\isanewline
\ \ \isacommand{thus}\isamarkupfalse%
\ {\isacharquery}{\kern0pt}thesis\isanewline
\ \ \ \ \isacommand{by}\isamarkupfalse%
\ {\isacharparenleft}{\kern0pt}simp\ add{\isacharcolon}{\kern0pt}\ non{\isacharunderscore}{\kern0pt}electing{\isacharunderscore}{\kern0pt}def{\isacharparenright}{\kern0pt}\isanewline
\isacommand{qed}\isamarkupfalse%
%
\endisatagproof
{\isafoldproof}%
%
\isadelimproof
\isanewline
%
\endisadelimproof
\isanewline
\isanewline
\isacommand{theorem}\isamarkupfalse%
\ pass{\isacharunderscore}{\kern0pt}mod{\isacharunderscore}{\kern0pt}non{\isacharunderscore}{\kern0pt}electing{\isacharbrackleft}{\kern0pt}simp{\isacharbrackright}{\kern0pt}{\isacharcolon}{\kern0pt}\isanewline
\ \ \isakeyword{assumes}\ order{\isacharcolon}{\kern0pt}\ {\isachardoublequoteopen}linear{\isacharunderscore}{\kern0pt}order\ r{\isachardoublequoteclose}\isanewline
\ \ \isakeyword{shows}\ {\isachardoublequoteopen}non{\isacharunderscore}{\kern0pt}electing\ {\isacharparenleft}{\kern0pt}pass{\isacharunderscore}{\kern0pt}module\ n\ r{\isacharparenright}{\kern0pt}{\isachardoublequoteclose}\isanewline
%
\isadelimproof
\ \ %
\endisadelimproof
%
\isatagproof
\isacommand{by}\isamarkupfalse%
\ {\isacharparenleft}{\kern0pt}simp\ add{\isacharcolon}{\kern0pt}\ non{\isacharunderscore}{\kern0pt}electing{\isacharunderscore}{\kern0pt}def\ order{\isacharparenright}{\kern0pt}%
\endisatagproof
{\isafoldproof}%
%
\isadelimproof
\isanewline
%
\endisadelimproof
\isanewline
\isanewline
\isacommand{theorem}\isamarkupfalse%
\ rev{\isacharunderscore}{\kern0pt}comp{\isacharunderscore}{\kern0pt}non{\isacharunderscore}{\kern0pt}electing{\isacharbrackleft}{\kern0pt}simp{\isacharbrackright}{\kern0pt}{\isacharcolon}{\kern0pt}\isanewline
\ \ \isakeyword{assumes}\ {\isachardoublequoteopen}electoral{\isacharunderscore}{\kern0pt}module\ m{\isachardoublequoteclose}\isanewline
\ \ \isakeyword{shows}\ {\isachardoublequoteopen}non{\isacharunderscore}{\kern0pt}electing\ {\isacharparenleft}{\kern0pt}m{\isasymdown}{\isacharparenright}{\kern0pt}{\isachardoublequoteclose}\isanewline
%
\isadelimproof
\ \ %
\endisadelimproof
%
\isatagproof
\isacommand{by}\isamarkupfalse%
\ {\isacharparenleft}{\kern0pt}simp\ add{\isacharcolon}{\kern0pt}\ assms\ non{\isacharunderscore}{\kern0pt}electing{\isacharunderscore}{\kern0pt}def{\isacharparenright}{\kern0pt}%
\endisatagproof
{\isafoldproof}%
%
\isadelimproof
\isanewline
%
\endisadelimproof
\isanewline
\isanewline
\isacommand{theorem}\isamarkupfalse%
\ pass{\isacharunderscore}{\kern0pt}mod{\isacharunderscore}{\kern0pt}non{\isacharunderscore}{\kern0pt}blocking{\isacharbrackleft}{\kern0pt}simp{\isacharbrackright}{\kern0pt}{\isacharcolon}{\kern0pt}\isanewline
\ \ \isakeyword{assumes}\ order{\isacharcolon}{\kern0pt}\ {\isachardoublequoteopen}linear{\isacharunderscore}{\kern0pt}order\ r{\isachardoublequoteclose}\ \isakeyword{and}\isanewline
\ \ \ \ \ \ \ \ \ \ g{\isadigit{0}}{\isacharunderscore}{\kern0pt}n{\isacharcolon}{\kern0pt}\ \ {\isachardoublequoteopen}n\ {\isachargreater}{\kern0pt}\ {\isadigit{0}}{\isachardoublequoteclose}\isanewline
\ \ \ \ \ \ \ \ \isakeyword{shows}\ {\isachardoublequoteopen}non{\isacharunderscore}{\kern0pt}blocking\ {\isacharparenleft}{\kern0pt}pass{\isacharunderscore}{\kern0pt}module\ n\ r{\isacharparenright}{\kern0pt}{\isachardoublequoteclose}\isanewline
%
\isadelimproof
\ \ %
\endisadelimproof
%
\isatagproof
\isacommand{unfolding}\isamarkupfalse%
\ non{\isacharunderscore}{\kern0pt}blocking{\isacharunderscore}{\kern0pt}def\isanewline
\isacommand{proof}\isamarkupfalse%
\ {\isacharparenleft}{\kern0pt}safe{\isacharcomma}{\kern0pt}\ simp{\isacharunderscore}{\kern0pt}all{\isacharparenright}{\kern0pt}\isanewline
\ \ \isacommand{show}\isamarkupfalse%
\ {\isachardoublequoteopen}electoral{\isacharunderscore}{\kern0pt}module\ {\isacharparenleft}{\kern0pt}pass{\isacharunderscore}{\kern0pt}module\ n\ r{\isacharparenright}{\kern0pt}{\isachardoublequoteclose}\isanewline
\ \ \ \ \isacommand{using}\isamarkupfalse%
\ pass{\isacharunderscore}{\kern0pt}mod{\isacharunderscore}{\kern0pt}sound\ order\isanewline
\ \ \ \ \isacommand{by}\isamarkupfalse%
\ simp\isanewline
\isacommand{next}\isamarkupfalse%
\isanewline
\ \ \isacommand{fix}\isamarkupfalse%
\isanewline
\ \ \ \ A\ {\isacharcolon}{\kern0pt}{\isacharcolon}{\kern0pt}\ {\isachardoublequoteopen}{\isacharprime}{\kern0pt}a\ set{\isachardoublequoteclose}\ \isakeyword{and}\isanewline
\ \ \ \ p\ {\isacharcolon}{\kern0pt}{\isacharcolon}{\kern0pt}\ {\isachardoublequoteopen}{\isacharprime}{\kern0pt}a\ Profile{\isachardoublequoteclose}\ \isakeyword{and}\isanewline
\ \ \ \ x\ {\isacharcolon}{\kern0pt}{\isacharcolon}{\kern0pt}\ {\isachardoublequoteopen}{\isacharprime}{\kern0pt}a{\isachardoublequoteclose}\isanewline
\ \ \isacommand{assume}\isamarkupfalse%
\isanewline
\ \ \ \ fin{\isacharunderscore}{\kern0pt}A{\isacharcolon}{\kern0pt}\ {\isachardoublequoteopen}finite\ A{\isachardoublequoteclose}\ \isakeyword{and}\isanewline
\ \ \ \ prof{\isacharunderscore}{\kern0pt}A{\isacharcolon}{\kern0pt}\ {\isachardoublequoteopen}profile\ A\ p{\isachardoublequoteclose}\ \isakeyword{and}\isanewline
\ \ \ \ card{\isacharunderscore}{\kern0pt}A{\isacharcolon}{\kern0pt}\isanewline
\ \ \ \ {\isachardoublequoteopen}{\isacharbraceleft}{\kern0pt}a\ {\isasymin}\ A{\isachardot}{\kern0pt}\ n\ {\isacharless}{\kern0pt}\isanewline
\ \ \ \ \ \ card\ {\isacharparenleft}{\kern0pt}above\isanewline
\ \ \ \ \ \ \ \ {\isacharbraceleft}{\kern0pt}{\isacharparenleft}{\kern0pt}a{\isacharcomma}{\kern0pt}\ b{\isacharparenright}{\kern0pt}{\isachardot}{\kern0pt}\ {\isacharparenleft}{\kern0pt}a{\isacharcomma}{\kern0pt}\ b{\isacharparenright}{\kern0pt}\ {\isasymin}\ r\ {\isasymand}\isanewline
\ \ \ \ \ \ \ \ \ \ a\ {\isasymin}\ A\ {\isasymand}\ b\ {\isasymin}\ A{\isacharbraceright}{\kern0pt}\ a{\isacharparenright}{\kern0pt}{\isacharbraceright}{\kern0pt}\ {\isacharequal}{\kern0pt}\ A{\isachardoublequoteclose}\ \isakeyword{and}\isanewline
\ \ \ \ x{\isacharunderscore}{\kern0pt}in{\isacharunderscore}{\kern0pt}A{\isacharcolon}{\kern0pt}\ {\isachardoublequoteopen}x\ {\isasymin}\ A{\isachardoublequoteclose}\isanewline
\ \ \isacommand{have}\isamarkupfalse%
\ lin{\isacharunderscore}{\kern0pt}ord{\isacharunderscore}{\kern0pt}A{\isacharcolon}{\kern0pt}\isanewline
\ \ \ \ {\isachardoublequoteopen}linear{\isacharunderscore}{\kern0pt}order{\isacharunderscore}{\kern0pt}on\ A\ {\isacharparenleft}{\kern0pt}limit\ A\ r{\isacharparenright}{\kern0pt}{\isachardoublequoteclose}\isanewline
\ \ \ \ \isacommand{using}\isamarkupfalse%
\ limit{\isacharunderscore}{\kern0pt}presv{\isacharunderscore}{\kern0pt}lin{\isacharunderscore}{\kern0pt}ord\ order\ top{\isacharunderscore}{\kern0pt}greatest\isanewline
\ \ \ \ \isacommand{by}\isamarkupfalse%
\ metis\isanewline
\ \ \isacommand{have}\isamarkupfalse%
\isanewline
\ \ \ \ {\isachardoublequoteopen}{\isasymexists}a{\isasymin}A{\isachardot}{\kern0pt}\ above\ {\isacharparenleft}{\kern0pt}limit\ A\ r{\isacharparenright}{\kern0pt}\ a\ {\isacharequal}{\kern0pt}\ {\isacharbraceleft}{\kern0pt}a{\isacharbraceright}{\kern0pt}\ {\isasymand}\isanewline
\ \ \ \ \ \ {\isacharparenleft}{\kern0pt}{\isasymforall}x{\isasymin}A{\isachardot}{\kern0pt}\ above\ {\isacharparenleft}{\kern0pt}limit\ A\ r{\isacharparenright}{\kern0pt}\ x\ {\isacharequal}{\kern0pt}\ {\isacharbraceleft}{\kern0pt}x{\isacharbraceright}{\kern0pt}\ {\isasymlongrightarrow}\ x\ {\isacharequal}{\kern0pt}\ a{\isacharparenright}{\kern0pt}{\isachardoublequoteclose}\isanewline
\ \ \ \ \isacommand{using}\isamarkupfalse%
\ above{\isacharunderscore}{\kern0pt}one\ fin{\isacharunderscore}{\kern0pt}A\ lin{\isacharunderscore}{\kern0pt}ord{\isacharunderscore}{\kern0pt}A\ x{\isacharunderscore}{\kern0pt}in{\isacharunderscore}{\kern0pt}A\isanewline
\ \ \ \ \isacommand{by}\isamarkupfalse%
\ blast\isanewline
\ \ \isacommand{hence}\isamarkupfalse%
\ not{\isacharunderscore}{\kern0pt}all{\isacharcolon}{\kern0pt}\isanewline
\ \ \ \ {\isachardoublequoteopen}{\isacharbraceleft}{\kern0pt}a\ {\isasymin}\ A{\isachardot}{\kern0pt}\ card{\isacharparenleft}{\kern0pt}above\ {\isacharparenleft}{\kern0pt}limit\ A\ r{\isacharparenright}{\kern0pt}\ a{\isacharparenright}{\kern0pt}\ {\isachargreater}{\kern0pt}\ n{\isacharbraceright}{\kern0pt}\ {\isasymnoteq}\ A{\isachardoublequoteclose}\isanewline
\ \ \ \ \isacommand{using}\isamarkupfalse%
\ One{\isacharunderscore}{\kern0pt}nat{\isacharunderscore}{\kern0pt}def\ Suc{\isacharunderscore}{\kern0pt}leI\ assms{\isacharparenleft}{\kern0pt}{\isadigit{2}}{\isacharparenright}{\kern0pt}\ is{\isacharunderscore}{\kern0pt}singletonI\isanewline
\ \ \ \ \ \ \ \ \ \ is{\isacharunderscore}{\kern0pt}singleton{\isacharunderscore}{\kern0pt}altdef\ leD\ mem{\isacharunderscore}{\kern0pt}Collect{\isacharunderscore}{\kern0pt}eq\isanewline
\ \ \ \ \isacommand{by}\isamarkupfalse%
\ {\isacharparenleft}{\kern0pt}metis\ {\isacharparenleft}{\kern0pt}no{\isacharunderscore}{\kern0pt}types{\isacharcomma}{\kern0pt}\ lifting{\isacharparenright}{\kern0pt}{\isacharparenright}{\kern0pt}\isanewline
\ \ \isacommand{hence}\isamarkupfalse%
\ {\isachardoublequoteopen}reject\ {\isacharparenleft}{\kern0pt}pass{\isacharunderscore}{\kern0pt}module\ n\ r{\isacharparenright}{\kern0pt}\ A\ p\ {\isasymnoteq}\ A{\isachardoublequoteclose}\isanewline
\ \ \ \ \isacommand{by}\isamarkupfalse%
\ simp\isanewline
\ \ \isacommand{thus}\isamarkupfalse%
\ {\isachardoublequoteopen}False{\isachardoublequoteclose}\isanewline
\ \ \ \ \isacommand{using}\isamarkupfalse%
\ order\ card{\isacharunderscore}{\kern0pt}A\isanewline
\ \ \ \ \isacommand{by}\isamarkupfalse%
\ simp\isanewline
\isacommand{qed}\isamarkupfalse%
%
\endisatagproof
{\isafoldproof}%
%
\isadelimproof
\isanewline
%
\endisadelimproof
\isanewline
\isacommand{theorem}\isamarkupfalse%
\ pass{\isacharunderscore}{\kern0pt}zero{\isacharunderscore}{\kern0pt}mod{\isacharunderscore}{\kern0pt}def{\isacharunderscore}{\kern0pt}zero{\isacharbrackleft}{\kern0pt}simp{\isacharbrackright}{\kern0pt}{\isacharcolon}{\kern0pt}\isanewline
\ \ \isakeyword{assumes}\ order{\isacharcolon}{\kern0pt}\ {\isachardoublequoteopen}linear{\isacharunderscore}{\kern0pt}order\ r{\isachardoublequoteclose}\isanewline
\ \ \isakeyword{shows}\ {\isachardoublequoteopen}defers\ {\isadigit{0}}\ {\isacharparenleft}{\kern0pt}pass{\isacharunderscore}{\kern0pt}module\ {\isadigit{0}}\ r{\isacharparenright}{\kern0pt}{\isachardoublequoteclose}\isanewline
%
\isadelimproof
\ \ %
\endisadelimproof
%
\isatagproof
\isacommand{unfolding}\isamarkupfalse%
\ defers{\isacharunderscore}{\kern0pt}def\isanewline
\isacommand{proof}\isamarkupfalse%
\ {\isacharparenleft}{\kern0pt}safe{\isacharparenright}{\kern0pt}\isanewline
\ \ \isacommand{show}\isamarkupfalse%
\ {\isachardoublequoteopen}electoral{\isacharunderscore}{\kern0pt}module\ {\isacharparenleft}{\kern0pt}pass{\isacharunderscore}{\kern0pt}module\ {\isadigit{0}}\ r{\isacharparenright}{\kern0pt}{\isachardoublequoteclose}\isanewline
\ \ \ \ \isacommand{using}\isamarkupfalse%
\ pass{\isacharunderscore}{\kern0pt}mod{\isacharunderscore}{\kern0pt}sound\ order\isanewline
\ \ \ \ \isacommand{by}\isamarkupfalse%
\ simp\isanewline
\isacommand{next}\isamarkupfalse%
\isanewline
\ \ \isacommand{fix}\isamarkupfalse%
\isanewline
\ \ \ \ A\ {\isacharcolon}{\kern0pt}{\isacharcolon}{\kern0pt}\ {\isachardoublequoteopen}{\isacharprime}{\kern0pt}a\ set{\isachardoublequoteclose}\ \isakeyword{and}\isanewline
\ \ \ \ p\ {\isacharcolon}{\kern0pt}{\isacharcolon}{\kern0pt}\ {\isachardoublequoteopen}{\isacharprime}{\kern0pt}a\ Profile{\isachardoublequoteclose}\isanewline
\ \ \isacommand{assume}\isamarkupfalse%
\isanewline
\ \ \ \ card{\isacharunderscore}{\kern0pt}pos{\isacharcolon}{\kern0pt}\ {\isachardoublequoteopen}{\isadigit{0}}\ {\isasymle}\ card\ A{\isachardoublequoteclose}\ \isakeyword{and}\isanewline
\ \ \ \ finite{\isacharunderscore}{\kern0pt}A{\isacharcolon}{\kern0pt}\ {\isachardoublequoteopen}finite\ A{\isachardoublequoteclose}\ \isakeyword{and}\isanewline
\ \ \ \ prof{\isacharunderscore}{\kern0pt}A{\isacharcolon}{\kern0pt}\ {\isachardoublequoteopen}profile\ A\ p{\isachardoublequoteclose}\isanewline
\ \ \isacommand{show}\isamarkupfalse%
\isanewline
\ \ \ \ {\isachardoublequoteopen}card\ {\isacharparenleft}{\kern0pt}defer\ {\isacharparenleft}{\kern0pt}pass{\isacharunderscore}{\kern0pt}module\ {\isadigit{0}}\ r{\isacharparenright}{\kern0pt}\ A\ p{\isacharparenright}{\kern0pt}\ {\isacharequal}{\kern0pt}\ {\isadigit{0}}{\isachardoublequoteclose}\isanewline
\ \ \isacommand{proof}\isamarkupfalse%
\ {\isacharminus}{\kern0pt}\isanewline
\ \ \ \ \isacommand{have}\isamarkupfalse%
\ lin{\isacharunderscore}{\kern0pt}ord{\isacharunderscore}{\kern0pt}on{\isacharunderscore}{\kern0pt}A{\isacharcolon}{\kern0pt}\isanewline
\ \ \ \ \ \ {\isachardoublequoteopen}linear{\isacharunderscore}{\kern0pt}order{\isacharunderscore}{\kern0pt}on\ A\ {\isacharparenleft}{\kern0pt}limit\ A\ r{\isacharparenright}{\kern0pt}{\isachardoublequoteclose}\isanewline
\ \ \ \ \ \ \isacommand{using}\isamarkupfalse%
\ order\ limit{\isacharunderscore}{\kern0pt}presv{\isacharunderscore}{\kern0pt}lin{\isacharunderscore}{\kern0pt}ord\isanewline
\ \ \ \ \ \ \isacommand{by}\isamarkupfalse%
\ blast\isanewline
\ \ \ \ \isacommand{have}\isamarkupfalse%
\ f{\isadigit{1}}{\isacharcolon}{\kern0pt}\ {\isachardoublequoteopen}connex\ A\ {\isacharparenleft}{\kern0pt}limit\ A\ r{\isacharparenright}{\kern0pt}{\isachardoublequoteclose}\isanewline
\ \ \ \ \ \ \isacommand{using}\isamarkupfalse%
\ lin{\isacharunderscore}{\kern0pt}ord{\isacharunderscore}{\kern0pt}imp{\isacharunderscore}{\kern0pt}connex\ lin{\isacharunderscore}{\kern0pt}ord{\isacharunderscore}{\kern0pt}on{\isacharunderscore}{\kern0pt}A\isanewline
\ \ \ \ \ \ \isacommand{by}\isamarkupfalse%
\ simp\isanewline
\ \ \ \ \isacommand{obtain}\isamarkupfalse%
\ aa\ {\isacharcolon}{\kern0pt}{\isacharcolon}{\kern0pt}\ {\isachardoublequoteopen}{\isacharparenleft}{\kern0pt}{\isacharprime}{\kern0pt}a\ {\isasymRightarrow}\ bool{\isacharparenright}{\kern0pt}\ {\isasymRightarrow}\ {\isacharprime}{\kern0pt}a{\isachardoublequoteclose}\ \isakeyword{where}\isanewline
\ \ \ \ \ \ f{\isadigit{2}}{\isacharcolon}{\kern0pt}\isanewline
\ \ \ \ \ \ {\isachardoublequoteopen}{\isasymforall}p{\isachardot}{\kern0pt}\ {\isacharparenleft}{\kern0pt}Collect\ p\ {\isacharequal}{\kern0pt}\ {\isacharbraceleft}{\kern0pt}{\isacharbraceright}{\kern0pt}\ {\isasymlongrightarrow}\ {\isacharparenleft}{\kern0pt}{\isasymforall}a{\isachardot}{\kern0pt}\ {\isasymnot}\ p\ a{\isacharparenright}{\kern0pt}{\isacharparenright}{\kern0pt}\ {\isasymand}\isanewline
\ \ \ \ \ \ \ \ \ \ \ \ {\isacharparenleft}{\kern0pt}Collect\ p\ {\isasymnoteq}\ {\isacharbraceleft}{\kern0pt}{\isacharbraceright}{\kern0pt}\ {\isasymlongrightarrow}\ p\ {\isacharparenleft}{\kern0pt}aa\ p{\isacharparenright}{\kern0pt}{\isacharparenright}{\kern0pt}{\isachardoublequoteclose}\isanewline
\ \ \ \ \ \ \isacommand{by}\isamarkupfalse%
\ moura\isanewline
\ \ \ \ \isacommand{have}\isamarkupfalse%
\ {\isachardoublequoteopen}{\isasymforall}n{\isachardot}{\kern0pt}\ {\isasymnot}\ {\isacharparenleft}{\kern0pt}n{\isacharcolon}{\kern0pt}{\isacharcolon}{\kern0pt}nat{\isacharparenright}{\kern0pt}\ {\isasymle}\ {\isadigit{0}}\ {\isasymor}\ n\ {\isacharequal}{\kern0pt}\ {\isadigit{0}}{\isachardoublequoteclose}\isanewline
\ \ \ \ \ \ \isacommand{by}\isamarkupfalse%
\ blast\isanewline
\ \ \ \ \isacommand{hence}\isamarkupfalse%
\isanewline
\ \ \ \ \ \ {\isachardoublequoteopen}{\isasymforall}a\ Aa{\isachardot}{\kern0pt}\ {\isasymnot}\ connex\ Aa\ {\isacharparenleft}{\kern0pt}limit\ A\ r{\isacharparenright}{\kern0pt}\ {\isasymor}\ a\ {\isasymnotin}\ Aa\ {\isasymor}\ a\ {\isasymnotin}\ A\ {\isasymor}\isanewline
\ \ \ \ \ \ \ \ \ \ \ \ \ \ \ \ \ \ {\isasymnot}\ card\ {\isacharparenleft}{\kern0pt}above\ {\isacharparenleft}{\kern0pt}limit\ A\ r{\isacharparenright}{\kern0pt}\ a{\isacharparenright}{\kern0pt}\ {\isasymle}\ {\isadigit{0}}{\isachardoublequoteclose}\isanewline
\ \ \ \ \ \ \isacommand{using}\isamarkupfalse%
\ above{\isacharunderscore}{\kern0pt}connex\ above{\isacharunderscore}{\kern0pt}presv{\isacharunderscore}{\kern0pt}limit\ card{\isacharunderscore}{\kern0pt}eq{\isacharunderscore}{\kern0pt}{\isadigit{0}}{\isacharunderscore}{\kern0pt}iff\isanewline
\ \ \ \ \ \ \ \ \ \ \ \ equals{\isadigit{0}}D\ finite{\isacharunderscore}{\kern0pt}A\ order\ rev{\isacharunderscore}{\kern0pt}finite{\isacharunderscore}{\kern0pt}subset\isanewline
\ \ \ \ \ \ \isacommand{by}\isamarkupfalse%
\ {\isacharparenleft}{\kern0pt}metis\ {\isacharparenleft}{\kern0pt}no{\isacharunderscore}{\kern0pt}types{\isacharparenright}{\kern0pt}{\isacharparenright}{\kern0pt}\isanewline
\ \ \ \ \isacommand{hence}\isamarkupfalse%
\ {\isachardoublequoteopen}{\isacharbraceleft}{\kern0pt}a\ {\isasymin}\ A{\isachardot}{\kern0pt}\ card{\isacharparenleft}{\kern0pt}above\ {\isacharparenleft}{\kern0pt}limit\ A\ r{\isacharparenright}{\kern0pt}\ a{\isacharparenright}{\kern0pt}\ {\isasymle}\ {\isadigit{0}}{\isacharbraceright}{\kern0pt}\ {\isacharequal}{\kern0pt}\ {\isacharbraceleft}{\kern0pt}{\isacharbraceright}{\kern0pt}{\isachardoublequoteclose}\isanewline
\ \ \ \ \ \ \isacommand{using}\isamarkupfalse%
\ f{\isadigit{1}}\isanewline
\ \ \ \ \ \ \isacommand{by}\isamarkupfalse%
\ auto\isanewline
\ \ \ \ \isacommand{hence}\isamarkupfalse%
\ {\isachardoublequoteopen}card\ {\isacharbraceleft}{\kern0pt}a\ {\isasymin}\ A{\isachardot}{\kern0pt}\ card{\isacharparenleft}{\kern0pt}above\ {\isacharparenleft}{\kern0pt}limit\ A\ r{\isacharparenright}{\kern0pt}\ a{\isacharparenright}{\kern0pt}\ {\isasymle}\ {\isadigit{0}}{\isacharbraceright}{\kern0pt}\ {\isacharequal}{\kern0pt}\ {\isadigit{0}}{\isachardoublequoteclose}\isanewline
\ \ \ \ \ \ \isacommand{using}\isamarkupfalse%
\ card{\isachardot}{\kern0pt}empty\isanewline
\ \ \ \ \ \ \isacommand{by}\isamarkupfalse%
\ metis\isanewline
\ \ \ \ \isacommand{thus}\isamarkupfalse%
\ {\isachardoublequoteopen}card\ {\isacharparenleft}{\kern0pt}defer\ {\isacharparenleft}{\kern0pt}pass{\isacharunderscore}{\kern0pt}module\ {\isadigit{0}}\ r{\isacharparenright}{\kern0pt}\ A\ p{\isacharparenright}{\kern0pt}\ {\isacharequal}{\kern0pt}\ {\isadigit{0}}{\isachardoublequoteclose}\isanewline
\ \ \ \ \ \ \isacommand{by}\isamarkupfalse%
\ simp\isanewline
\ \ \isacommand{qed}\isamarkupfalse%
\isanewline
\isacommand{qed}\isamarkupfalse%
%
\endisatagproof
{\isafoldproof}%
%
\isadelimproof
\isanewline
%
\endisadelimproof
\isanewline
\isanewline
\isacommand{theorem}\isamarkupfalse%
\ pass{\isacharunderscore}{\kern0pt}one{\isacharunderscore}{\kern0pt}mod{\isacharunderscore}{\kern0pt}def{\isacharunderscore}{\kern0pt}one{\isacharbrackleft}{\kern0pt}simp{\isacharbrackright}{\kern0pt}{\isacharcolon}{\kern0pt}\isanewline
\ \ \isakeyword{assumes}\ order{\isacharcolon}{\kern0pt}\ {\isachardoublequoteopen}linear{\isacharunderscore}{\kern0pt}order\ r{\isachardoublequoteclose}\isanewline
\ \ \isakeyword{shows}\ {\isachardoublequoteopen}defers\ {\isadigit{1}}\ {\isacharparenleft}{\kern0pt}pass{\isacharunderscore}{\kern0pt}module\ {\isadigit{1}}\ r{\isacharparenright}{\kern0pt}{\isachardoublequoteclose}\isanewline
%
\isadelimproof
\ \ %
\endisadelimproof
%
\isatagproof
\isacommand{unfolding}\isamarkupfalse%
\ defers{\isacharunderscore}{\kern0pt}def\isanewline
\isacommand{proof}\isamarkupfalse%
\ {\isacharparenleft}{\kern0pt}safe{\isacharparenright}{\kern0pt}\isanewline
\ \ \isacommand{show}\isamarkupfalse%
\ {\isachardoublequoteopen}electoral{\isacharunderscore}{\kern0pt}module\ {\isacharparenleft}{\kern0pt}pass{\isacharunderscore}{\kern0pt}module\ {\isadigit{1}}\ r{\isacharparenright}{\kern0pt}{\isachardoublequoteclose}\isanewline
\ \ \ \ \isacommand{using}\isamarkupfalse%
\ pass{\isacharunderscore}{\kern0pt}mod{\isacharunderscore}{\kern0pt}sound\ order\isanewline
\ \ \ \ \isacommand{by}\isamarkupfalse%
\ simp\isanewline
\isacommand{next}\isamarkupfalse%
\isanewline
\ \ \isacommand{fix}\isamarkupfalse%
\isanewline
\ \ \ \ A\ {\isacharcolon}{\kern0pt}{\isacharcolon}{\kern0pt}\ {\isachardoublequoteopen}{\isacharprime}{\kern0pt}a\ set{\isachardoublequoteclose}\ \isakeyword{and}\isanewline
\ \ \ \ p\ {\isacharcolon}{\kern0pt}{\isacharcolon}{\kern0pt}\ {\isachardoublequoteopen}{\isacharprime}{\kern0pt}a\ Profile{\isachardoublequoteclose}\isanewline
\ \ \isacommand{assume}\isamarkupfalse%
\isanewline
\ \ \ \ card{\isacharunderscore}{\kern0pt}pos{\isacharcolon}{\kern0pt}\ {\isachardoublequoteopen}{\isadigit{1}}\ {\isasymle}\ card\ A{\isachardoublequoteclose}\ \isakeyword{and}\isanewline
\ \ \ \ finite{\isacharunderscore}{\kern0pt}A{\isacharcolon}{\kern0pt}\ {\isachardoublequoteopen}finite\ A{\isachardoublequoteclose}\ \isakeyword{and}\isanewline
\ \ \ \ prof{\isacharunderscore}{\kern0pt}A{\isacharcolon}{\kern0pt}\ {\isachardoublequoteopen}profile\ A\ p{\isachardoublequoteclose}\isanewline
\ \ \isacommand{show}\isamarkupfalse%
\isanewline
\ \ \ \ {\isachardoublequoteopen}card\ {\isacharparenleft}{\kern0pt}defer\ {\isacharparenleft}{\kern0pt}pass{\isacharunderscore}{\kern0pt}module\ {\isadigit{1}}\ r{\isacharparenright}{\kern0pt}\ A\ p{\isacharparenright}{\kern0pt}\ {\isacharequal}{\kern0pt}\ {\isadigit{1}}{\isachardoublequoteclose}\isanewline
\ \ \isacommand{proof}\isamarkupfalse%
\ {\isacharminus}{\kern0pt}\isanewline
\ \ \ \ \isacommand{have}\isamarkupfalse%
\ {\isachardoublequoteopen}A\ {\isasymnoteq}\ {\isacharbraceleft}{\kern0pt}{\isacharbraceright}{\kern0pt}{\isachardoublequoteclose}\isanewline
\ \ \ \ \ \ \isacommand{using}\isamarkupfalse%
\ card{\isacharunderscore}{\kern0pt}pos\isanewline
\ \ \ \ \ \ \isacommand{by}\isamarkupfalse%
\ auto\isanewline
\ \ \ \ \isacommand{moreover}\isamarkupfalse%
\ \isacommand{have}\isamarkupfalse%
\ lin{\isacharunderscore}{\kern0pt}ord{\isacharunderscore}{\kern0pt}on{\isacharunderscore}{\kern0pt}A{\isacharcolon}{\kern0pt}\isanewline
\ \ \ \ \ \ {\isachardoublequoteopen}linear{\isacharunderscore}{\kern0pt}order{\isacharunderscore}{\kern0pt}on\ A\ {\isacharparenleft}{\kern0pt}limit\ A\ r{\isacharparenright}{\kern0pt}{\isachardoublequoteclose}\isanewline
\ \ \ \ \ \ \isacommand{using}\isamarkupfalse%
\ order\ limit{\isacharunderscore}{\kern0pt}presv{\isacharunderscore}{\kern0pt}lin{\isacharunderscore}{\kern0pt}ord\isanewline
\ \ \ \ \ \ \isacommand{by}\isamarkupfalse%
\ blast\isanewline
\ \ \ \ \isacommand{ultimately}\isamarkupfalse%
\ \isacommand{have}\isamarkupfalse%
\ winner{\isacharunderscore}{\kern0pt}exists{\isacharcolon}{\kern0pt}\isanewline
\ \ \ \ \ \ {\isachardoublequoteopen}{\isasymexists}a{\isasymin}A{\isachardot}{\kern0pt}\ above\ {\isacharparenleft}{\kern0pt}limit\ A\ r{\isacharparenright}{\kern0pt}\ a\ {\isacharequal}{\kern0pt}\ {\isacharbraceleft}{\kern0pt}a{\isacharbraceright}{\kern0pt}\ {\isasymand}\isanewline
\ \ \ \ \ \ \ \ {\isacharparenleft}{\kern0pt}{\isasymforall}x{\isasymin}A{\isachardot}{\kern0pt}\ above\ {\isacharparenleft}{\kern0pt}limit\ A\ r{\isacharparenright}{\kern0pt}\ x\ {\isacharequal}{\kern0pt}\ {\isacharbraceleft}{\kern0pt}x{\isacharbraceright}{\kern0pt}\ {\isasymlongrightarrow}\ x\ {\isacharequal}{\kern0pt}\ a{\isacharparenright}{\kern0pt}{\isachardoublequoteclose}\isanewline
\ \ \ \ \ \ \isacommand{using}\isamarkupfalse%
\ finite{\isacharunderscore}{\kern0pt}A\isanewline
\ \ \ \ \ \ \isacommand{by}\isamarkupfalse%
\ {\isacharparenleft}{\kern0pt}simp\ add{\isacharcolon}{\kern0pt}\ above{\isacharunderscore}{\kern0pt}one{\isacharparenright}{\kern0pt}\isanewline
\ \ \ \ \isacommand{then}\isamarkupfalse%
\ \isacommand{obtain}\isamarkupfalse%
\ w\ \isakeyword{where}\ w{\isacharunderscore}{\kern0pt}unique{\isacharunderscore}{\kern0pt}top{\isacharcolon}{\kern0pt}\isanewline
\ \ \ \ \ \ {\isachardoublequoteopen}above\ {\isacharparenleft}{\kern0pt}limit\ A\ r{\isacharparenright}{\kern0pt}\ w\ {\isacharequal}{\kern0pt}\ {\isacharbraceleft}{\kern0pt}w{\isacharbraceright}{\kern0pt}\ {\isasymand}\isanewline
\ \ \ \ \ \ \ \ {\isacharparenleft}{\kern0pt}{\isasymforall}x{\isasymin}A{\isachardot}{\kern0pt}\ above\ {\isacharparenleft}{\kern0pt}limit\ A\ r{\isacharparenright}{\kern0pt}\ x\ {\isacharequal}{\kern0pt}\ {\isacharbraceleft}{\kern0pt}x{\isacharbraceright}{\kern0pt}\ {\isasymlongrightarrow}\ x\ {\isacharequal}{\kern0pt}\ w{\isacharparenright}{\kern0pt}{\isachardoublequoteclose}\isanewline
\ \ \ \ \ \ \isacommand{using}\isamarkupfalse%
\ above{\isacharunderscore}{\kern0pt}one\isanewline
\ \ \ \ \ \ \isacommand{by}\isamarkupfalse%
\ auto\isanewline
\ \ \ \ \isacommand{hence}\isamarkupfalse%
\ {\isachardoublequoteopen}{\isacharbraceleft}{\kern0pt}a\ {\isasymin}\ A{\isachardot}{\kern0pt}\ card{\isacharparenleft}{\kern0pt}above\ {\isacharparenleft}{\kern0pt}limit\ A\ r{\isacharparenright}{\kern0pt}\ a{\isacharparenright}{\kern0pt}\ {\isasymle}\ {\isadigit{1}}{\isacharbraceright}{\kern0pt}\ {\isacharequal}{\kern0pt}\ {\isacharbraceleft}{\kern0pt}w{\isacharbraceright}{\kern0pt}{\isachardoublequoteclose}\isanewline
\ \ \ \ \isacommand{proof}\isamarkupfalse%
\isanewline
\ \ \ \ \ \ \isacommand{assume}\isamarkupfalse%
\isanewline
\ \ \ \ \ \ \ \ w{\isacharunderscore}{\kern0pt}top{\isacharcolon}{\kern0pt}\ {\isachardoublequoteopen}above\ {\isacharparenleft}{\kern0pt}limit\ A\ r{\isacharparenright}{\kern0pt}\ w\ {\isacharequal}{\kern0pt}\ {\isacharbraceleft}{\kern0pt}w{\isacharbraceright}{\kern0pt}{\isachardoublequoteclose}\ \isakeyword{and}\isanewline
\ \ \ \ \ \ \ \ w{\isacharunderscore}{\kern0pt}unique{\isacharcolon}{\kern0pt}\ {\isachardoublequoteopen}{\isasymforall}x{\isasymin}A{\isachardot}{\kern0pt}\ above\ {\isacharparenleft}{\kern0pt}limit\ A\ r{\isacharparenright}{\kern0pt}\ x\ {\isacharequal}{\kern0pt}\ {\isacharbraceleft}{\kern0pt}x{\isacharbraceright}{\kern0pt}\ {\isasymlongrightarrow}\ x\ {\isacharequal}{\kern0pt}\ w{\isachardoublequoteclose}\isanewline
\ \ \ \ \ \ \isacommand{have}\isamarkupfalse%
\ {\isachardoublequoteopen}card\ {\isacharparenleft}{\kern0pt}above\ {\isacharparenleft}{\kern0pt}limit\ A\ r{\isacharparenright}{\kern0pt}\ w{\isacharparenright}{\kern0pt}\ {\isasymle}\ {\isadigit{1}}{\isachardoublequoteclose}\isanewline
\ \ \ \ \ \ \ \ \isacommand{using}\isamarkupfalse%
\ w{\isacharunderscore}{\kern0pt}top\isanewline
\ \ \ \ \ \ \ \ \isacommand{by}\isamarkupfalse%
\ auto\isanewline
\ \ \ \ \ \ \isacommand{hence}\isamarkupfalse%
\ {\isachardoublequoteopen}{\isacharbraceleft}{\kern0pt}w{\isacharbraceright}{\kern0pt}\ {\isasymsubseteq}\ {\isacharbraceleft}{\kern0pt}a\ {\isasymin}\ A{\isachardot}{\kern0pt}\ card{\isacharparenleft}{\kern0pt}above\ {\isacharparenleft}{\kern0pt}limit\ A\ r{\isacharparenright}{\kern0pt}\ a{\isacharparenright}{\kern0pt}\ {\isasymle}\ {\isadigit{1}}{\isacharbraceright}{\kern0pt}{\isachardoublequoteclose}\isanewline
\ \ \ \ \ \ \ \ \isacommand{using}\isamarkupfalse%
\ winner{\isacharunderscore}{\kern0pt}exists\ w{\isacharunderscore}{\kern0pt}unique{\isacharunderscore}{\kern0pt}top\isanewline
\ \ \ \ \ \ \ \ \isacommand{by}\isamarkupfalse%
\ blast\isanewline
\ \ \ \ \ \ \isacommand{moreover}\isamarkupfalse%
\ \isacommand{have}\isamarkupfalse%
\isanewline
\ \ \ \ \ \ \ \ {\isachardoublequoteopen}{\isacharbraceleft}{\kern0pt}a\ {\isasymin}\ A{\isachardot}{\kern0pt}\ card{\isacharparenleft}{\kern0pt}above\ {\isacharparenleft}{\kern0pt}limit\ A\ r{\isacharparenright}{\kern0pt}\ a{\isacharparenright}{\kern0pt}\ {\isasymle}\ {\isadigit{1}}{\isacharbraceright}{\kern0pt}\ {\isasymsubseteq}\ {\isacharbraceleft}{\kern0pt}w{\isacharbraceright}{\kern0pt}{\isachardoublequoteclose}\isanewline
\ \ \ \ \ \ \isacommand{proof}\isamarkupfalse%
\isanewline
\ \ \ \ \ \ \ \ \isacommand{fix}\isamarkupfalse%
\isanewline
\ \ \ \ \ \ \ \ \ \ x\ {\isacharcolon}{\kern0pt}{\isacharcolon}{\kern0pt}\ {\isachardoublequoteopen}{\isacharprime}{\kern0pt}a{\isachardoublequoteclose}\isanewline
\ \ \ \ \ \ \ \ \isacommand{assume}\isamarkupfalse%
\ x{\isacharunderscore}{\kern0pt}in{\isacharunderscore}{\kern0pt}winner{\isacharunderscore}{\kern0pt}set{\isacharcolon}{\kern0pt}\isanewline
\ \ \ \ \ \ \ \ \ \ {\isachardoublequoteopen}x\ {\isasymin}\ {\isacharbraceleft}{\kern0pt}a\ {\isasymin}\ A{\isachardot}{\kern0pt}\ card\ {\isacharparenleft}{\kern0pt}above\ {\isacharparenleft}{\kern0pt}limit\ A\ r{\isacharparenright}{\kern0pt}\ a{\isacharparenright}{\kern0pt}\ {\isasymle}\ {\isadigit{1}}{\isacharbraceright}{\kern0pt}{\isachardoublequoteclose}\isanewline
\ \ \ \ \ \ \ \ \isacommand{hence}\isamarkupfalse%
\ x{\isacharunderscore}{\kern0pt}in{\isacharunderscore}{\kern0pt}A{\isacharcolon}{\kern0pt}\ {\isachardoublequoteopen}x\ {\isasymin}\ A{\isachardoublequoteclose}\isanewline
\ \ \ \ \ \ \ \ \ \ \isacommand{by}\isamarkupfalse%
\ auto\isanewline
\ \ \ \ \ \ \ \ \isacommand{hence}\isamarkupfalse%
\ connex{\isacharunderscore}{\kern0pt}limit{\isacharcolon}{\kern0pt}\isanewline
\ \ \ \ \ \ \ \ \ \ {\isachardoublequoteopen}connex\ A\ {\isacharparenleft}{\kern0pt}limit\ A\ r{\isacharparenright}{\kern0pt}{\isachardoublequoteclose}\isanewline
\ \ \ \ \ \ \ \ \ \ \isacommand{using}\isamarkupfalse%
\ lin{\isacharunderscore}{\kern0pt}ord{\isacharunderscore}{\kern0pt}imp{\isacharunderscore}{\kern0pt}connex\ lin{\isacharunderscore}{\kern0pt}ord{\isacharunderscore}{\kern0pt}on{\isacharunderscore}{\kern0pt}A\isanewline
\ \ \ \ \ \ \ \ \ \ \isacommand{by}\isamarkupfalse%
\ simp\isanewline
\ \ \ \ \ \ \ \ \isacommand{hence}\isamarkupfalse%
\ {\isachardoublequoteopen}let\ q\ {\isacharequal}{\kern0pt}\ limit\ A\ r\ in\ x\ {\isasympreceq}\isactrlsub q\ x{\isachardoublequoteclose}\isanewline
\ \ \ \ \ \ \ \ \ \ \isacommand{using}\isamarkupfalse%
\ connex{\isacharunderscore}{\kern0pt}limit\ above{\isacharunderscore}{\kern0pt}connex\isanewline
\ \ \ \ \ \ \ \ \ \ \ \ \ \ \ \ pref{\isacharunderscore}{\kern0pt}imp{\isacharunderscore}{\kern0pt}in{\isacharunderscore}{\kern0pt}above\ x{\isacharunderscore}{\kern0pt}in{\isacharunderscore}{\kern0pt}A\isanewline
\ \ \ \ \ \ \ \ \ \ \isacommand{by}\isamarkupfalse%
\ metis\isanewline
\ \ \ \ \ \ \ \ \isacommand{hence}\isamarkupfalse%
\ {\isachardoublequoteopen}{\isacharparenleft}{\kern0pt}x{\isacharcomma}{\kern0pt}x{\isacharparenright}{\kern0pt}\ {\isasymin}\ limit\ A\ r{\isachardoublequoteclose}\isanewline
\ \ \ \ \ \ \ \ \ \ \isacommand{by}\isamarkupfalse%
\ simp\isanewline
\ \ \ \ \ \ \ \ \isacommand{hence}\isamarkupfalse%
\ x{\isacharunderscore}{\kern0pt}above{\isacharunderscore}{\kern0pt}x{\isacharcolon}{\kern0pt}\ {\isachardoublequoteopen}x\ {\isasymin}\ above\ {\isacharparenleft}{\kern0pt}limit\ A\ r{\isacharparenright}{\kern0pt}\ x{\isachardoublequoteclose}\isanewline
\ \ \ \ \ \ \ \ \ \ \isacommand{by}\isamarkupfalse%
\ {\isacharparenleft}{\kern0pt}simp\ add{\isacharcolon}{\kern0pt}\ above{\isacharunderscore}{\kern0pt}def{\isacharparenright}{\kern0pt}\isanewline
\ \ \ \ \ \ \ \ \isacommand{have}\isamarkupfalse%
\ {\isachardoublequoteopen}above\ {\isacharparenleft}{\kern0pt}limit\ A\ r{\isacharparenright}{\kern0pt}\ x\ {\isasymsubseteq}\ A{\isachardoublequoteclose}\isanewline
\ \ \ \ \ \ \ \ \ \ \isacommand{using}\isamarkupfalse%
\ above{\isacharunderscore}{\kern0pt}presv{\isacharunderscore}{\kern0pt}limit\ order\isanewline
\ \ \ \ \ \ \ \ \ \ \isacommand{by}\isamarkupfalse%
\ fastforce\isanewline
\ \ \ \ \ \ \ \ \isacommand{hence}\isamarkupfalse%
\ above{\isacharunderscore}{\kern0pt}finite{\isacharcolon}{\kern0pt}\ {\isachardoublequoteopen}finite\ {\isacharparenleft}{\kern0pt}above\ {\isacharparenleft}{\kern0pt}limit\ A\ r{\isacharparenright}{\kern0pt}\ x{\isacharparenright}{\kern0pt}{\isachardoublequoteclose}\isanewline
\ \ \ \ \ \ \ \ \ \ \isacommand{by}\isamarkupfalse%
\ {\isacharparenleft}{\kern0pt}simp\ add{\isacharcolon}{\kern0pt}\ finite{\isacharunderscore}{\kern0pt}A\ finite{\isacharunderscore}{\kern0pt}subset{\isacharparenright}{\kern0pt}\isanewline
\ \ \ \ \ \ \ \ \isacommand{have}\isamarkupfalse%
\ {\isachardoublequoteopen}card\ {\isacharparenleft}{\kern0pt}above\ {\isacharparenleft}{\kern0pt}limit\ A\ r{\isacharparenright}{\kern0pt}\ x{\isacharparenright}{\kern0pt}\ {\isasymle}\ {\isadigit{1}}{\isachardoublequoteclose}\isanewline
\ \ \ \ \ \ \ \ \ \ \isacommand{using}\isamarkupfalse%
\ x{\isacharunderscore}{\kern0pt}in{\isacharunderscore}{\kern0pt}winner{\isacharunderscore}{\kern0pt}set\isanewline
\ \ \ \ \ \ \ \ \ \ \isacommand{by}\isamarkupfalse%
\ simp\isanewline
\ \ \ \ \ \ \ \ \isacommand{moreover}\isamarkupfalse%
\ \isacommand{have}\isamarkupfalse%
\isanewline
\ \ \ \ \ \ \ \ \ \ {\isachardoublequoteopen}card\ {\isacharparenleft}{\kern0pt}above\ {\isacharparenleft}{\kern0pt}limit\ A\ r{\isacharparenright}{\kern0pt}\ x{\isacharparenright}{\kern0pt}\ {\isasymge}\ {\isadigit{1}}{\isachardoublequoteclose}\isanewline
\ \ \ \ \ \ \ \ \ \ \isacommand{using}\isamarkupfalse%
\ One{\isacharunderscore}{\kern0pt}nat{\isacharunderscore}{\kern0pt}def\ Suc{\isacharunderscore}{\kern0pt}leI\ above{\isacharunderscore}{\kern0pt}finite\ card{\isacharunderscore}{\kern0pt}eq{\isacharunderscore}{\kern0pt}{\isadigit{0}}{\isacharunderscore}{\kern0pt}iff\isanewline
\ \ \ \ \ \ \ \ \ \ \ \ \ \ \ \ equals{\isadigit{0}}D\ neq{\isadigit{0}}{\isacharunderscore}{\kern0pt}conv\ x{\isacharunderscore}{\kern0pt}above{\isacharunderscore}{\kern0pt}x\isanewline
\ \ \ \ \ \ \ \ \ \ \isacommand{by}\isamarkupfalse%
\ metis\isanewline
\ \ \ \ \ \ \ \ \isacommand{ultimately}\isamarkupfalse%
\ \isacommand{have}\isamarkupfalse%
\isanewline
\ \ \ \ \ \ \ \ \ \ {\isachardoublequoteopen}card\ {\isacharparenleft}{\kern0pt}above\ {\isacharparenleft}{\kern0pt}limit\ A\ r{\isacharparenright}{\kern0pt}\ x{\isacharparenright}{\kern0pt}\ {\isacharequal}{\kern0pt}\ {\isadigit{1}}{\isachardoublequoteclose}\isanewline
\ \ \ \ \ \ \ \ \ \ \isacommand{by}\isamarkupfalse%
\ simp\isanewline
\ \ \ \ \ \ \ \ \isacommand{hence}\isamarkupfalse%
\ {\isachardoublequoteopen}{\isacharbraceleft}{\kern0pt}x{\isacharbraceright}{\kern0pt}\ {\isacharequal}{\kern0pt}\ above\ {\isacharparenleft}{\kern0pt}limit\ A\ r{\isacharparenright}{\kern0pt}\ x{\isachardoublequoteclose}\isanewline
\ \ \ \ \ \ \ \ \ \ \isacommand{using}\isamarkupfalse%
\ is{\isacharunderscore}{\kern0pt}singletonE\ is{\isacharunderscore}{\kern0pt}singleton{\isacharunderscore}{\kern0pt}altdef\ singletonD\ x{\isacharunderscore}{\kern0pt}above{\isacharunderscore}{\kern0pt}x\isanewline
\ \ \ \ \ \ \ \ \ \ \isacommand{by}\isamarkupfalse%
\ metis\isanewline
\ \ \ \ \ \ \ \ \isacommand{hence}\isamarkupfalse%
\ {\isachardoublequoteopen}x\ {\isacharequal}{\kern0pt}\ w{\isachardoublequoteclose}\isanewline
\ \ \ \ \ \ \ \ \ \ \isacommand{using}\isamarkupfalse%
\ w{\isacharunderscore}{\kern0pt}unique\isanewline
\ \ \ \ \ \ \ \ \ \ \isacommand{by}\isamarkupfalse%
\ {\isacharparenleft}{\kern0pt}simp\ add{\isacharcolon}{\kern0pt}\ x{\isacharunderscore}{\kern0pt}in{\isacharunderscore}{\kern0pt}A{\isacharparenright}{\kern0pt}\isanewline
\ \ \ \ \ \ \ \ \isacommand{thus}\isamarkupfalse%
\ {\isachardoublequoteopen}x\ {\isasymin}\ {\isacharbraceleft}{\kern0pt}w{\isacharbraceright}{\kern0pt}{\isachardoublequoteclose}\isanewline
\ \ \ \ \ \ \ \ \ \ \isacommand{by}\isamarkupfalse%
\ simp\isanewline
\ \ \ \ \ \ \isacommand{qed}\isamarkupfalse%
\isanewline
\ \ \ \ \ \ \isacommand{ultimately}\isamarkupfalse%
\ \isacommand{have}\isamarkupfalse%
\isanewline
\ \ \ \ \ \ \ \ {\isachardoublequoteopen}{\isacharbraceleft}{\kern0pt}w{\isacharbraceright}{\kern0pt}\ {\isacharequal}{\kern0pt}\ {\isacharbraceleft}{\kern0pt}a\ {\isasymin}\ A{\isachardot}{\kern0pt}\ card\ {\isacharparenleft}{\kern0pt}above\ {\isacharparenleft}{\kern0pt}limit\ A\ r{\isacharparenright}{\kern0pt}\ a{\isacharparenright}{\kern0pt}\ {\isasymle}\ {\isadigit{1}}{\isacharbraceright}{\kern0pt}{\isachardoublequoteclose}\isanewline
\ \ \ \ \ \ \ \ \isacommand{by}\isamarkupfalse%
\ auto\isanewline
\ \ \ \ \ \ \isacommand{thus}\isamarkupfalse%
\ {\isacharquery}{\kern0pt}thesis\isanewline
\ \ \ \ \ \ \ \ \isacommand{by}\isamarkupfalse%
\ simp\isanewline
\ \ \ \ \isacommand{qed}\isamarkupfalse%
\isanewline
\ \ \ \ \isacommand{hence}\isamarkupfalse%
\ {\isachardoublequoteopen}defer\ {\isacharparenleft}{\kern0pt}pass{\isacharunderscore}{\kern0pt}module\ {\isadigit{1}}\ r{\isacharparenright}{\kern0pt}\ A\ p\ {\isacharequal}{\kern0pt}\ {\isacharbraceleft}{\kern0pt}w{\isacharbraceright}{\kern0pt}{\isachardoublequoteclose}\isanewline
\ \ \ \ \ \ \isacommand{by}\isamarkupfalse%
\ simp\isanewline
\ \ \ \ \isacommand{thus}\isamarkupfalse%
\ {\isachardoublequoteopen}card\ {\isacharparenleft}{\kern0pt}defer\ {\isacharparenleft}{\kern0pt}pass{\isacharunderscore}{\kern0pt}module\ {\isadigit{1}}\ r{\isacharparenright}{\kern0pt}\ A\ p{\isacharparenright}{\kern0pt}\ {\isacharequal}{\kern0pt}\ {\isadigit{1}}{\isachardoublequoteclose}\isanewline
\ \ \ \ \ \ \isacommand{by}\isamarkupfalse%
\ simp\isanewline
\ \ \isacommand{qed}\isamarkupfalse%
\isanewline
\isacommand{qed}\isamarkupfalse%
%
\endisatagproof
{\isafoldproof}%
%
\isadelimproof
\isanewline
%
\endisadelimproof
\isanewline
\isacommand{theorem}\isamarkupfalse%
\ pass{\isacharunderscore}{\kern0pt}two{\isacharunderscore}{\kern0pt}mod{\isacharunderscore}{\kern0pt}def{\isacharunderscore}{\kern0pt}two{\isacharcolon}{\kern0pt}\isanewline
\ \ \isakeyword{assumes}\ order{\isacharcolon}{\kern0pt}\ {\isachardoublequoteopen}linear{\isacharunderscore}{\kern0pt}order\ r{\isachardoublequoteclose}\isanewline
\ \ \isakeyword{shows}\ {\isachardoublequoteopen}defers\ {\isadigit{2}}\ {\isacharparenleft}{\kern0pt}pass{\isacharunderscore}{\kern0pt}module\ {\isadigit{2}}\ r{\isacharparenright}{\kern0pt}{\isachardoublequoteclose}\isanewline
%
\isadelimproof
\ \ %
\endisadelimproof
%
\isatagproof
\isacommand{unfolding}\isamarkupfalse%
\ defers{\isacharunderscore}{\kern0pt}def\isanewline
\isacommand{proof}\isamarkupfalse%
\ {\isacharparenleft}{\kern0pt}safe{\isacharparenright}{\kern0pt}\isanewline
\ \ \isacommand{show}\isamarkupfalse%
\ {\isachardoublequoteopen}electoral{\isacharunderscore}{\kern0pt}module\ {\isacharparenleft}{\kern0pt}pass{\isacharunderscore}{\kern0pt}module\ {\isadigit{2}}\ r{\isacharparenright}{\kern0pt}{\isachardoublequoteclose}\isanewline
\ \ \ \ \isacommand{using}\isamarkupfalse%
\ order\isanewline
\ \ \ \ \isacommand{by}\isamarkupfalse%
\ simp\isanewline
\isacommand{next}\isamarkupfalse%
\isanewline
\ \ \isacommand{fix}\isamarkupfalse%
\isanewline
\ \ \ \ A\ {\isacharcolon}{\kern0pt}{\isacharcolon}{\kern0pt}\ {\isachardoublequoteopen}{\isacharprime}{\kern0pt}a\ set{\isachardoublequoteclose}\ \isakeyword{and}\isanewline
\ \ \ \ p\ {\isacharcolon}{\kern0pt}{\isacharcolon}{\kern0pt}\ {\isachardoublequoteopen}{\isacharprime}{\kern0pt}a\ Profile{\isachardoublequoteclose}\isanewline
\ \ \isacommand{assume}\isamarkupfalse%
\isanewline
\ \ \ \ min{\isacharunderscore}{\kern0pt}{\isadigit{2}}{\isacharunderscore}{\kern0pt}card{\isacharcolon}{\kern0pt}\ {\isachardoublequoteopen}{\isadigit{2}}\ {\isasymle}\ card\ A{\isachardoublequoteclose}\ \isakeyword{and}\isanewline
\ \ \ \ finA{\isacharcolon}{\kern0pt}\ {\isachardoublequoteopen}finite\ A{\isachardoublequoteclose}\ \isakeyword{and}\isanewline
\ \ \ \ profA{\isacharcolon}{\kern0pt}\ {\isachardoublequoteopen}profile\ A\ p{\isachardoublequoteclose}\isanewline
\ \ \isacommand{from}\isamarkupfalse%
\ min{\isacharunderscore}{\kern0pt}{\isadigit{2}}{\isacharunderscore}{\kern0pt}card\isanewline
\ \ \isacommand{have}\isamarkupfalse%
\ not{\isacharunderscore}{\kern0pt}empty{\isacharunderscore}{\kern0pt}A{\isacharcolon}{\kern0pt}\ {\isachardoublequoteopen}A\ {\isasymnoteq}\ {\isacharbraceleft}{\kern0pt}{\isacharbraceright}{\kern0pt}{\isachardoublequoteclose}\isanewline
\ \ \ \ \isacommand{by}\isamarkupfalse%
\ auto\isanewline
\ \ \isacommand{moreover}\isamarkupfalse%
\ \isacommand{have}\isamarkupfalse%
\ limitA{\isacharunderscore}{\kern0pt}order{\isacharcolon}{\kern0pt}\isanewline
\ \ \ \ {\isachardoublequoteopen}linear{\isacharunderscore}{\kern0pt}order{\isacharunderscore}{\kern0pt}on\ A\ {\isacharparenleft}{\kern0pt}limit\ A\ r{\isacharparenright}{\kern0pt}{\isachardoublequoteclose}\isanewline
\ \ \ \ \isacommand{using}\isamarkupfalse%
\ limit{\isacharunderscore}{\kern0pt}presv{\isacharunderscore}{\kern0pt}lin{\isacharunderscore}{\kern0pt}ord\ order\isanewline
\ \ \ \ \isacommand{by}\isamarkupfalse%
\ auto\isanewline
\ \ \isacommand{ultimately}\isamarkupfalse%
\ \isacommand{obtain}\isamarkupfalse%
\ a\ \isakeyword{where}\isanewline
\ \ \ \ a{\isacharcolon}{\kern0pt}\ {\isachardoublequoteopen}above\ {\isacharparenleft}{\kern0pt}limit\ A\ r{\isacharparenright}{\kern0pt}\ a\ {\isacharequal}{\kern0pt}\ {\isacharbraceleft}{\kern0pt}a{\isacharbraceright}{\kern0pt}{\isachardoublequoteclose}\isanewline
\ \ \ \ \isacommand{using}\isamarkupfalse%
\ above{\isacharunderscore}{\kern0pt}one\ min{\isacharunderscore}{\kern0pt}{\isadigit{2}}{\isacharunderscore}{\kern0pt}card\ finA\ profA\isanewline
\ \ \ \ \isacommand{by}\isamarkupfalse%
\ blast\isanewline
\ \ \isacommand{hence}\isamarkupfalse%
\ {\isachardoublequoteopen}{\isasymforall}b\ {\isasymin}\ A{\isachardot}{\kern0pt}\ let\ q\ {\isacharequal}{\kern0pt}\ limit\ A\ r\ in\ {\isacharparenleft}{\kern0pt}b\ {\isasympreceq}\isactrlsub q\ a{\isacharparenright}{\kern0pt}{\isachardoublequoteclose}\isanewline
\ \ \ \ \isacommand{using}\isamarkupfalse%
\ limitA{\isacharunderscore}{\kern0pt}order\ pref{\isacharunderscore}{\kern0pt}imp{\isacharunderscore}{\kern0pt}in{\isacharunderscore}{\kern0pt}above\ empty{\isacharunderscore}{\kern0pt}iff\isanewline
\ \ \ \ \ \ \ \ \ \ insert{\isacharunderscore}{\kern0pt}iff\ insert{\isacharunderscore}{\kern0pt}subset\ above{\isacharunderscore}{\kern0pt}presv{\isacharunderscore}{\kern0pt}limit\isanewline
\ \ \ \ \ \ \ \ \ \ order\ connex{\isacharunderscore}{\kern0pt}def\ lin{\isacharunderscore}{\kern0pt}ord{\isacharunderscore}{\kern0pt}imp{\isacharunderscore}{\kern0pt}connex\isanewline
\ \ \ \ \isacommand{by}\isamarkupfalse%
\ metis\isanewline
\ \ \isacommand{hence}\isamarkupfalse%
\ a{\isacharunderscore}{\kern0pt}best{\isacharcolon}{\kern0pt}\ {\isachardoublequoteopen}{\isasymforall}b\ {\isasymin}\ A{\isachardot}{\kern0pt}\ {\isacharparenleft}{\kern0pt}b{\isacharcomma}{\kern0pt}\ a{\isacharparenright}{\kern0pt}\ {\isasymin}\ limit\ A\ r{\isachardoublequoteclose}\isanewline
\ \ \ \ \isacommand{by}\isamarkupfalse%
\ simp\isanewline
\ \ \isacommand{hence}\isamarkupfalse%
\ a{\isacharunderscore}{\kern0pt}above{\isacharcolon}{\kern0pt}\ {\isachardoublequoteopen}{\isasymforall}b\ {\isasymin}\ A{\isachardot}{\kern0pt}\ a\ {\isasymin}\ above\ {\isacharparenleft}{\kern0pt}limit\ A\ r{\isacharparenright}{\kern0pt}\ b{\isachardoublequoteclose}\isanewline
\ \ \ \ \isacommand{by}\isamarkupfalse%
\ {\isacharparenleft}{\kern0pt}simp\ add{\isacharcolon}{\kern0pt}\ above{\isacharunderscore}{\kern0pt}def{\isacharparenright}{\kern0pt}\isanewline
\ \ \isacommand{from}\isamarkupfalse%
\ a\ \isacommand{have}\isamarkupfalse%
\ {\isachardoublequoteopen}a\ {\isasymin}\ {\isacharbraceleft}{\kern0pt}a\ {\isasymin}\ A{\isachardot}{\kern0pt}\ card{\isacharparenleft}{\kern0pt}above\ {\isacharparenleft}{\kern0pt}limit\ A\ r{\isacharparenright}{\kern0pt}\ a{\isacharparenright}{\kern0pt}\ {\isasymle}\ {\isadigit{2}}{\isacharbraceright}{\kern0pt}{\isachardoublequoteclose}\isanewline
\ \ \ \ \isacommand{using}\isamarkupfalse%
\ CollectI\ Suc{\isacharunderscore}{\kern0pt}leI\ not{\isacharunderscore}{\kern0pt}empty{\isacharunderscore}{\kern0pt}A\ a{\isacharunderscore}{\kern0pt}above\ card{\isacharunderscore}{\kern0pt}UNIV{\isacharunderscore}{\kern0pt}bool\isanewline
\ \ \ \ \ \ \ \ \ \ card{\isacharunderscore}{\kern0pt}eq{\isacharunderscore}{\kern0pt}{\isadigit{0}}{\isacharunderscore}{\kern0pt}iff\ card{\isacharunderscore}{\kern0pt}insert{\isacharunderscore}{\kern0pt}disjoint\ empty{\isacharunderscore}{\kern0pt}iff\ finA\isanewline
\ \ \ \ \ \ \ \ \ \ finite{\isachardot}{\kern0pt}emptyI\ insert{\isacharunderscore}{\kern0pt}iff\ limitA{\isacharunderscore}{\kern0pt}order\ above{\isacharunderscore}{\kern0pt}one\isanewline
\ \ \ \ \ \ \ \ \ \ UNIV{\isacharunderscore}{\kern0pt}bool\ nat{\isachardot}{\kern0pt}simps{\isacharparenleft}{\kern0pt}{\isadigit{3}}{\isacharparenright}{\kern0pt}\ zero{\isacharunderscore}{\kern0pt}less{\isacharunderscore}{\kern0pt}Suc\isanewline
\ \ \ \ \isacommand{by}\isamarkupfalse%
\ {\isacharparenleft}{\kern0pt}metis\ {\isacharparenleft}{\kern0pt}no{\isacharunderscore}{\kern0pt}types{\isacharcomma}{\kern0pt}\ lifting{\isacharparenright}{\kern0pt}{\isacharparenright}{\kern0pt}\isanewline
\ \ \isacommand{hence}\isamarkupfalse%
\ a{\isacharunderscore}{\kern0pt}in{\isacharunderscore}{\kern0pt}defer{\isacharcolon}{\kern0pt}\ {\isachardoublequoteopen}a\ {\isasymin}\ defer\ {\isacharparenleft}{\kern0pt}pass{\isacharunderscore}{\kern0pt}module\ {\isadigit{2}}\ r{\isacharparenright}{\kern0pt}\ A\ p{\isachardoublequoteclose}\isanewline
\ \ \ \ \isacommand{by}\isamarkupfalse%
\ simp\isanewline
\ \ \isacommand{have}\isamarkupfalse%
\ {\isachardoublequoteopen}finite\ {\isacharparenleft}{\kern0pt}A{\isacharminus}{\kern0pt}{\isacharbraceleft}{\kern0pt}a{\isacharbraceright}{\kern0pt}{\isacharparenright}{\kern0pt}{\isachardoublequoteclose}\isanewline
\ \ \ \ \isacommand{by}\isamarkupfalse%
\ {\isacharparenleft}{\kern0pt}simp\ add{\isacharcolon}{\kern0pt}\ finA{\isacharparenright}{\kern0pt}\isanewline
\ \ \isacommand{moreover}\isamarkupfalse%
\ \isacommand{have}\isamarkupfalse%
\ A{\isacharunderscore}{\kern0pt}not{\isacharunderscore}{\kern0pt}only{\isacharunderscore}{\kern0pt}a{\isacharcolon}{\kern0pt}\ {\isachardoublequoteopen}A{\isacharminus}{\kern0pt}{\isacharbraceleft}{\kern0pt}a{\isacharbraceright}{\kern0pt}\ {\isasymnoteq}\ {\isacharbraceleft}{\kern0pt}{\isacharbraceright}{\kern0pt}{\isachardoublequoteclose}\isanewline
\ \ \ \ \isacommand{using}\isamarkupfalse%
\ min{\isacharunderscore}{\kern0pt}{\isadigit{2}}{\isacharunderscore}{\kern0pt}card\ Diff{\isacharunderscore}{\kern0pt}empty\ Diff{\isacharunderscore}{\kern0pt}idemp\ Diff{\isacharunderscore}{\kern0pt}insert{\isadigit{0}}\isanewline
\ \ \ \ \ \ \ \ \ \ One{\isacharunderscore}{\kern0pt}nat{\isacharunderscore}{\kern0pt}def\ not{\isacharunderscore}{\kern0pt}empty{\isacharunderscore}{\kern0pt}A\ card{\isachardot}{\kern0pt}insert{\isacharunderscore}{\kern0pt}remove\isanewline
\ \ \ \ \ \ \ \ \ \ card{\isacharunderscore}{\kern0pt}eq{\isacharunderscore}{\kern0pt}{\isadigit{0}}{\isacharunderscore}{\kern0pt}iff\ finite{\isachardot}{\kern0pt}emptyI\ insert{\isacharunderscore}{\kern0pt}Diff\isanewline
\ \ \ \ \ \ \ \ \ \ numeral{\isacharunderscore}{\kern0pt}le{\isacharunderscore}{\kern0pt}one{\isacharunderscore}{\kern0pt}iff\ semiring{\isacharunderscore}{\kern0pt}norm{\isacharparenleft}{\kern0pt}{\isadigit{6}}{\isadigit{9}}{\isacharparenright}{\kern0pt}\ card{\isachardot}{\kern0pt}empty\isanewline
\ \ \ \ \isacommand{by}\isamarkupfalse%
\ metis\isanewline
\ \ \isacommand{moreover}\isamarkupfalse%
\ \isacommand{have}\isamarkupfalse%
\ limitAa{\isacharunderscore}{\kern0pt}order{\isacharcolon}{\kern0pt}\isanewline
\ \ \ \ {\isachardoublequoteopen}linear{\isacharunderscore}{\kern0pt}order{\isacharunderscore}{\kern0pt}on\ {\isacharparenleft}{\kern0pt}A{\isacharminus}{\kern0pt}{\isacharbraceleft}{\kern0pt}a{\isacharbraceright}{\kern0pt}{\isacharparenright}{\kern0pt}\ {\isacharparenleft}{\kern0pt}limit\ {\isacharparenleft}{\kern0pt}A{\isacharminus}{\kern0pt}{\isacharbraceleft}{\kern0pt}a{\isacharbraceright}{\kern0pt}{\isacharparenright}{\kern0pt}\ r{\isacharparenright}{\kern0pt}{\isachardoublequoteclose}\isanewline
\ \ \ \ \isacommand{using}\isamarkupfalse%
\ limit{\isacharunderscore}{\kern0pt}presv{\isacharunderscore}{\kern0pt}lin{\isacharunderscore}{\kern0pt}ord\ order\ top{\isacharunderscore}{\kern0pt}greatest\isanewline
\ \ \ \ \isacommand{by}\isamarkupfalse%
\ blast\isanewline
\ \ \isacommand{ultimately}\isamarkupfalse%
\ \isacommand{obtain}\isamarkupfalse%
\ b\ \isakeyword{where}\ b{\isacharcolon}{\kern0pt}\ {\isachardoublequoteopen}above\ {\isacharparenleft}{\kern0pt}limit\ {\isacharparenleft}{\kern0pt}A{\isacharminus}{\kern0pt}{\isacharbraceleft}{\kern0pt}a{\isacharbraceright}{\kern0pt}{\isacharparenright}{\kern0pt}\ r{\isacharparenright}{\kern0pt}\ b\ {\isacharequal}{\kern0pt}\ {\isacharbraceleft}{\kern0pt}b{\isacharbraceright}{\kern0pt}{\isachardoublequoteclose}\isanewline
\ \ \ \ \isacommand{using}\isamarkupfalse%
\ above{\isacharunderscore}{\kern0pt}one\isanewline
\ \ \ \ \isacommand{by}\isamarkupfalse%
\ metis\isanewline
\ \ \isacommand{hence}\isamarkupfalse%
\ {\isachardoublequoteopen}{\isasymforall}c\ {\isasymin}\ A{\isacharminus}{\kern0pt}{\isacharbraceleft}{\kern0pt}a{\isacharbraceright}{\kern0pt}{\isachardot}{\kern0pt}\ let\ q\ {\isacharequal}{\kern0pt}\ limit\ {\isacharparenleft}{\kern0pt}A{\isacharminus}{\kern0pt}{\isacharbraceleft}{\kern0pt}a{\isacharbraceright}{\kern0pt}{\isacharparenright}{\kern0pt}\ r\ in\ {\isacharparenleft}{\kern0pt}c\ {\isasympreceq}\isactrlsub q\ b{\isacharparenright}{\kern0pt}{\isachardoublequoteclose}\isanewline
\ \ \ \ \isacommand{using}\isamarkupfalse%
\ limitAa{\isacharunderscore}{\kern0pt}order\ pref{\isacharunderscore}{\kern0pt}imp{\isacharunderscore}{\kern0pt}in{\isacharunderscore}{\kern0pt}above\ empty{\isacharunderscore}{\kern0pt}iff\ insert{\isacharunderscore}{\kern0pt}iff\isanewline
\ \ \ \ \ \ \ \ \ \ insert{\isacharunderscore}{\kern0pt}subset\ above{\isacharunderscore}{\kern0pt}presv{\isacharunderscore}{\kern0pt}limit\ order\ connex{\isacharunderscore}{\kern0pt}def\isanewline
\ \ \ \ \ \ \ \ \ \ lin{\isacharunderscore}{\kern0pt}ord{\isacharunderscore}{\kern0pt}imp{\isacharunderscore}{\kern0pt}connex\isanewline
\ \ \ \ \isacommand{by}\isamarkupfalse%
\ metis\isanewline
\ \ \isacommand{hence}\isamarkupfalse%
\ b{\isacharunderscore}{\kern0pt}in{\isacharunderscore}{\kern0pt}limit{\isacharcolon}{\kern0pt}\ {\isachardoublequoteopen}{\isasymforall}c\ {\isasymin}\ A{\isacharminus}{\kern0pt}{\isacharbraceleft}{\kern0pt}a{\isacharbraceright}{\kern0pt}{\isachardot}{\kern0pt}\ {\isacharparenleft}{\kern0pt}c{\isacharcomma}{\kern0pt}\ b{\isacharparenright}{\kern0pt}\ {\isasymin}\ limit\ {\isacharparenleft}{\kern0pt}A{\isacharminus}{\kern0pt}{\isacharbraceleft}{\kern0pt}a{\isacharbraceright}{\kern0pt}{\isacharparenright}{\kern0pt}\ r{\isachardoublequoteclose}\isanewline
\ \ \ \ \isacommand{by}\isamarkupfalse%
\ simp\isanewline
\ \ \isacommand{hence}\isamarkupfalse%
\ b{\isacharunderscore}{\kern0pt}best{\isacharcolon}{\kern0pt}\ {\isachardoublequoteopen}{\isasymforall}c\ {\isasymin}\ A{\isacharminus}{\kern0pt}{\isacharbraceleft}{\kern0pt}a{\isacharbraceright}{\kern0pt}{\isachardot}{\kern0pt}\ {\isacharparenleft}{\kern0pt}c{\isacharcomma}{\kern0pt}\ b{\isacharparenright}{\kern0pt}\ {\isasymin}\ limit\ A\ r{\isachardoublequoteclose}\isanewline
\ \ \ \ \isacommand{by}\isamarkupfalse%
\ auto\isanewline
\ \ \isacommand{hence}\isamarkupfalse%
\ c{\isacharunderscore}{\kern0pt}not{\isacharunderscore}{\kern0pt}above{\isacharunderscore}{\kern0pt}b{\isacharcolon}{\kern0pt}\ {\isachardoublequoteopen}{\isasymforall}c\ {\isasymin}\ A{\isacharminus}{\kern0pt}{\isacharbraceleft}{\kern0pt}a{\isacharcomma}{\kern0pt}\ b{\isacharbraceright}{\kern0pt}{\isachardot}{\kern0pt}\ c\ {\isasymnotin}\ above\ {\isacharparenleft}{\kern0pt}limit\ A\ r{\isacharparenright}{\kern0pt}\ b{\isachardoublequoteclose}\isanewline
\ \ \ \ \isacommand{using}\isamarkupfalse%
\ b\ Diff{\isacharunderscore}{\kern0pt}iff\ Diff{\isacharunderscore}{\kern0pt}insert{\isadigit{2}}\ subset{\isacharunderscore}{\kern0pt}UNIV\ above{\isacharunderscore}{\kern0pt}presv{\isacharunderscore}{\kern0pt}limit\isanewline
\ \ \ \ \ \ \ \ \ \ insert{\isacharunderscore}{\kern0pt}subset\ order\ limit{\isacharunderscore}{\kern0pt}presv{\isacharunderscore}{\kern0pt}above\ limit{\isacharunderscore}{\kern0pt}presv{\isacharunderscore}{\kern0pt}above{\isadigit{2}}\isanewline
\ \ \ \ \isacommand{by}\isamarkupfalse%
\ metis\isanewline
\ \ \isacommand{moreover}\isamarkupfalse%
\ \isacommand{have}\isamarkupfalse%
\ above{\isacharunderscore}{\kern0pt}subset{\isacharcolon}{\kern0pt}\ {\isachardoublequoteopen}above\ {\isacharparenleft}{\kern0pt}limit\ A\ r{\isacharparenright}{\kern0pt}\ b\ {\isasymsubseteq}\ A{\isachardoublequoteclose}\isanewline
\ \ \ \ \isacommand{using}\isamarkupfalse%
\ above{\isacharunderscore}{\kern0pt}presv{\isacharunderscore}{\kern0pt}limit\ order\isanewline
\ \ \ \ \isacommand{by}\isamarkupfalse%
\ metis\isanewline
\ \ \isacommand{moreover}\isamarkupfalse%
\ \isacommand{have}\isamarkupfalse%
\ b{\isacharunderscore}{\kern0pt}above{\isacharunderscore}{\kern0pt}b{\isacharcolon}{\kern0pt}\ {\isachardoublequoteopen}b\ {\isasymin}\ above\ {\isacharparenleft}{\kern0pt}limit\ A\ r{\isacharparenright}{\kern0pt}\ b{\isachardoublequoteclose}\isanewline
\ \ \ \ \isacommand{using}\isamarkupfalse%
\ above{\isacharunderscore}{\kern0pt}def\ b\ b{\isacharunderscore}{\kern0pt}best\ above{\isacharunderscore}{\kern0pt}presv{\isacharunderscore}{\kern0pt}limit\isanewline
\ \ \ \ \ \ \ \ \ \ mem{\isacharunderscore}{\kern0pt}Collect{\isacharunderscore}{\kern0pt}eq\ order\ insert{\isacharunderscore}{\kern0pt}subset\isanewline
\ \ \ \ \isacommand{by}\isamarkupfalse%
\ metis\isanewline
\ \ \isacommand{ultimately}\isamarkupfalse%
\ \isacommand{have}\isamarkupfalse%
\ above{\isacharunderscore}{\kern0pt}b{\isacharunderscore}{\kern0pt}eq{\isacharunderscore}{\kern0pt}ab{\isacharcolon}{\kern0pt}\ {\isachardoublequoteopen}above\ {\isacharparenleft}{\kern0pt}limit\ A\ r{\isacharparenright}{\kern0pt}\ b\ {\isacharequal}{\kern0pt}\ {\isacharbraceleft}{\kern0pt}a{\isacharcomma}{\kern0pt}\ b{\isacharbraceright}{\kern0pt}{\isachardoublequoteclose}\isanewline
\ \ \ \ \isacommand{using}\isamarkupfalse%
\ a{\isacharunderscore}{\kern0pt}above\isanewline
\ \ \ \ \isacommand{by}\isamarkupfalse%
\ auto\isanewline
\ \ \isacommand{hence}\isamarkupfalse%
\ card{\isacharunderscore}{\kern0pt}above{\isacharunderscore}{\kern0pt}b{\isacharunderscore}{\kern0pt}eq{\isacharunderscore}{\kern0pt}{\isadigit{2}}{\isacharcolon}{\kern0pt}\ {\isachardoublequoteopen}card\ {\isacharparenleft}{\kern0pt}above\ {\isacharparenleft}{\kern0pt}limit\ A\ r{\isacharparenright}{\kern0pt}\ b{\isacharparenright}{\kern0pt}\ {\isacharequal}{\kern0pt}\ {\isadigit{2}}{\isachardoublequoteclose}\isanewline
\ \ \ \ \isacommand{using}\isamarkupfalse%
\ A{\isacharunderscore}{\kern0pt}not{\isacharunderscore}{\kern0pt}only{\isacharunderscore}{\kern0pt}a\ b{\isacharunderscore}{\kern0pt}in{\isacharunderscore}{\kern0pt}limit\isanewline
\ \ \ \ \isacommand{by}\isamarkupfalse%
\ auto\isanewline
\ \ \isacommand{hence}\isamarkupfalse%
\ b{\isacharunderscore}{\kern0pt}in{\isacharunderscore}{\kern0pt}defer{\isacharcolon}{\kern0pt}\ {\isachardoublequoteopen}b\ {\isasymin}\ defer\ {\isacharparenleft}{\kern0pt}pass{\isacharunderscore}{\kern0pt}module\ {\isadigit{2}}\ r{\isacharparenright}{\kern0pt}\ A\ p{\isachardoublequoteclose}\isanewline
\ \ \ \ \isacommand{using}\isamarkupfalse%
\ b{\isacharunderscore}{\kern0pt}above{\isacharunderscore}{\kern0pt}b\ above{\isacharunderscore}{\kern0pt}subset\isanewline
\ \ \ \ \isacommand{by}\isamarkupfalse%
\ auto\isanewline
\ \ \isacommand{from}\isamarkupfalse%
\ b{\isacharunderscore}{\kern0pt}best\ \isacommand{have}\isamarkupfalse%
\ b{\isacharunderscore}{\kern0pt}above{\isacharcolon}{\kern0pt}\isanewline
\ \ \ \ {\isachardoublequoteopen}{\isasymforall}c\ {\isasymin}\ A{\isacharminus}{\kern0pt}{\isacharbraceleft}{\kern0pt}a{\isacharbraceright}{\kern0pt}{\isachardot}{\kern0pt}\ b\ {\isasymin}\ above\ {\isacharparenleft}{\kern0pt}limit\ A\ r{\isacharparenright}{\kern0pt}\ c{\isachardoublequoteclose}\isanewline
\ \ \ \ \isacommand{using}\isamarkupfalse%
\ above{\isacharunderscore}{\kern0pt}def\ mem{\isacharunderscore}{\kern0pt}Collect{\isacharunderscore}{\kern0pt}eq\isanewline
\ \ \ \ \isacommand{by}\isamarkupfalse%
\ metis\isanewline
\ \ \isacommand{have}\isamarkupfalse%
\ {\isachardoublequoteopen}connex\ A\ {\isacharparenleft}{\kern0pt}limit\ A\ r{\isacharparenright}{\kern0pt}{\isachardoublequoteclose}\isanewline
\ \ \ \ \isacommand{using}\isamarkupfalse%
\ limitA{\isacharunderscore}{\kern0pt}order\ lin{\isacharunderscore}{\kern0pt}ord{\isacharunderscore}{\kern0pt}imp{\isacharunderscore}{\kern0pt}connex\isanewline
\ \ \ \ \isacommand{by}\isamarkupfalse%
\ auto\isanewline
\ \ \isacommand{hence}\isamarkupfalse%
\ {\isachardoublequoteopen}{\isasymforall}c\ {\isasymin}\ A{\isachardot}{\kern0pt}\ c\ {\isasymin}\ above\ {\isacharparenleft}{\kern0pt}limit\ A\ r{\isacharparenright}{\kern0pt}\ c{\isachardoublequoteclose}\isanewline
\ \ \ \ \isacommand{by}\isamarkupfalse%
\ {\isacharparenleft}{\kern0pt}simp\ add{\isacharcolon}{\kern0pt}\ above{\isacharunderscore}{\kern0pt}connex{\isacharparenright}{\kern0pt}\isanewline
\ \ \isacommand{hence}\isamarkupfalse%
\ {\isachardoublequoteopen}{\isasymforall}c\ {\isasymin}\ A{\isacharminus}{\kern0pt}{\isacharbraceleft}{\kern0pt}a{\isacharcomma}{\kern0pt}\ b{\isacharbraceright}{\kern0pt}{\isachardot}{\kern0pt}\ {\isacharbraceleft}{\kern0pt}a{\isacharcomma}{\kern0pt}\ b{\isacharcomma}{\kern0pt}\ c{\isacharbraceright}{\kern0pt}\ {\isasymsubseteq}\ above\ {\isacharparenleft}{\kern0pt}limit\ A\ r{\isacharparenright}{\kern0pt}\ c{\isachardoublequoteclose}\isanewline
\ \ \ \ \isacommand{using}\isamarkupfalse%
\ a{\isacharunderscore}{\kern0pt}above\ b{\isacharunderscore}{\kern0pt}above\isanewline
\ \ \ \ \isacommand{by}\isamarkupfalse%
\ auto\isanewline
\ \ \isacommand{moreover}\isamarkupfalse%
\ \isacommand{have}\isamarkupfalse%
\ {\isachardoublequoteopen}{\isasymforall}c\ {\isasymin}\ A{\isacharminus}{\kern0pt}{\isacharbraceleft}{\kern0pt}a{\isacharcomma}{\kern0pt}\ b{\isacharbraceright}{\kern0pt}{\isachardot}{\kern0pt}\ card{\isacharbraceleft}{\kern0pt}a{\isacharcomma}{\kern0pt}\ b{\isacharcomma}{\kern0pt}\ c{\isacharbraceright}{\kern0pt}\ {\isacharequal}{\kern0pt}\ {\isadigit{3}}{\isachardoublequoteclose}\isanewline
\ \ \ \ \isacommand{using}\isamarkupfalse%
\ DiffE\ One{\isacharunderscore}{\kern0pt}nat{\isacharunderscore}{\kern0pt}def\ Suc{\isacharunderscore}{\kern0pt}{\isadigit{1}}\ above{\isacharunderscore}{\kern0pt}b{\isacharunderscore}{\kern0pt}eq{\isacharunderscore}{\kern0pt}ab\ card{\isacharunderscore}{\kern0pt}above{\isacharunderscore}{\kern0pt}b{\isacharunderscore}{\kern0pt}eq{\isacharunderscore}{\kern0pt}{\isadigit{2}}\isanewline
\ \ \ \ \ \ \ \ \ \ above{\isacharunderscore}{\kern0pt}subset\ card{\isacharunderscore}{\kern0pt}insert{\isacharunderscore}{\kern0pt}disjoint\ finA\ finite{\isacharunderscore}{\kern0pt}subset\isanewline
\ \ \ \ \ \ \ \ \ \ insert{\isacharunderscore}{\kern0pt}commute\ numeral{\isacharunderscore}{\kern0pt}{\isadigit{3}}{\isacharunderscore}{\kern0pt}eq{\isacharunderscore}{\kern0pt}{\isadigit{3}}\isanewline
\ \ \ \ \isacommand{by}\isamarkupfalse%
\ metis\isanewline
\ \ \isacommand{ultimately}\isamarkupfalse%
\ \isacommand{have}\isamarkupfalse%
\isanewline
\ \ \ \ {\isachardoublequoteopen}{\isasymforall}c\ {\isasymin}\ A{\isacharminus}{\kern0pt}{\isacharbraceleft}{\kern0pt}a{\isacharcomma}{\kern0pt}\ b{\isacharbraceright}{\kern0pt}{\isachardot}{\kern0pt}\ card\ {\isacharparenleft}{\kern0pt}above\ {\isacharparenleft}{\kern0pt}limit\ A\ r{\isacharparenright}{\kern0pt}\ c{\isacharparenright}{\kern0pt}\ {\isasymge}\ {\isadigit{3}}{\isachardoublequoteclose}\isanewline
\ \ \ \ \isacommand{using}\isamarkupfalse%
\ card{\isacharunderscore}{\kern0pt}mono\ finA\ finite{\isacharunderscore}{\kern0pt}subset\ above{\isacharunderscore}{\kern0pt}presv{\isacharunderscore}{\kern0pt}limit\ order\isanewline
\ \ \ \ \isacommand{by}\isamarkupfalse%
\ metis\isanewline
\ \ \isacommand{hence}\isamarkupfalse%
\ {\isachardoublequoteopen}{\isasymforall}c\ {\isasymin}\ A{\isacharminus}{\kern0pt}{\isacharbraceleft}{\kern0pt}a{\isacharcomma}{\kern0pt}\ b{\isacharbraceright}{\kern0pt}{\isachardot}{\kern0pt}\ card\ {\isacharparenleft}{\kern0pt}above\ {\isacharparenleft}{\kern0pt}limit\ A\ r{\isacharparenright}{\kern0pt}\ c{\isacharparenright}{\kern0pt}\ {\isachargreater}{\kern0pt}\ {\isadigit{2}}{\isachardoublequoteclose}\isanewline
\ \ \ \ \isacommand{using}\isamarkupfalse%
\ less{\isacharunderscore}{\kern0pt}le{\isacharunderscore}{\kern0pt}trans\ numeral{\isacharunderscore}{\kern0pt}less{\isacharunderscore}{\kern0pt}iff\ order{\isacharunderscore}{\kern0pt}refl\ semiring{\isacharunderscore}{\kern0pt}norm{\isacharparenleft}{\kern0pt}{\isadigit{7}}{\isadigit{9}}{\isacharparenright}{\kern0pt}\isanewline
\ \ \ \ \isacommand{by}\isamarkupfalse%
\ metis\isanewline
\ \ \isacommand{hence}\isamarkupfalse%
\ {\isachardoublequoteopen}{\isasymforall}c\ {\isasymin}\ A{\isacharminus}{\kern0pt}{\isacharbraceleft}{\kern0pt}a{\isacharcomma}{\kern0pt}\ b{\isacharbraceright}{\kern0pt}{\isachardot}{\kern0pt}\ c\ {\isasymnotin}\ defer\ {\isacharparenleft}{\kern0pt}pass{\isacharunderscore}{\kern0pt}module\ {\isadigit{2}}\ r{\isacharparenright}{\kern0pt}\ A\ p{\isachardoublequoteclose}\isanewline
\ \ \ \ \isacommand{by}\isamarkupfalse%
\ {\isacharparenleft}{\kern0pt}simp\ add{\isacharcolon}{\kern0pt}\ not{\isacharunderscore}{\kern0pt}le{\isacharparenright}{\kern0pt}\isanewline
\ \ \isacommand{moreover}\isamarkupfalse%
\ \isacommand{have}\isamarkupfalse%
\ {\isachardoublequoteopen}defer\ {\isacharparenleft}{\kern0pt}pass{\isacharunderscore}{\kern0pt}module\ {\isadigit{2}}\ r{\isacharparenright}{\kern0pt}\ A\ p\ {\isasymsubseteq}\ A{\isachardoublequoteclose}\isanewline
\ \ \ \ \isacommand{by}\isamarkupfalse%
\ auto\isanewline
\ \ \isacommand{ultimately}\isamarkupfalse%
\ \isacommand{have}\isamarkupfalse%
\ {\isachardoublequoteopen}defer\ {\isacharparenleft}{\kern0pt}pass{\isacharunderscore}{\kern0pt}module\ {\isadigit{2}}\ r{\isacharparenright}{\kern0pt}\ A\ p\ {\isasymsubseteq}\ {\isacharbraceleft}{\kern0pt}a{\isacharcomma}{\kern0pt}\ b{\isacharbraceright}{\kern0pt}{\isachardoublequoteclose}\isanewline
\ \ \ \ \isacommand{by}\isamarkupfalse%
\ blast\isanewline
\ \ \isacommand{with}\isamarkupfalse%
\ a{\isacharunderscore}{\kern0pt}in{\isacharunderscore}{\kern0pt}defer\ b{\isacharunderscore}{\kern0pt}in{\isacharunderscore}{\kern0pt}defer\ \isacommand{have}\isamarkupfalse%
\isanewline
\ \ \ \ {\isachardoublequoteopen}defer\ {\isacharparenleft}{\kern0pt}pass{\isacharunderscore}{\kern0pt}module\ {\isadigit{2}}\ r{\isacharparenright}{\kern0pt}\ A\ p\ {\isacharequal}{\kern0pt}\ {\isacharbraceleft}{\kern0pt}a{\isacharcomma}{\kern0pt}\ b{\isacharbraceright}{\kern0pt}{\isachardoublequoteclose}\isanewline
\ \ \ \ \isacommand{by}\isamarkupfalse%
\ fastforce\isanewline
\ \ \isacommand{thus}\isamarkupfalse%
\ {\isachardoublequoteopen}card\ {\isacharparenleft}{\kern0pt}defer\ {\isacharparenleft}{\kern0pt}pass{\isacharunderscore}{\kern0pt}module\ {\isadigit{2}}\ r{\isacharparenright}{\kern0pt}\ A\ p{\isacharparenright}{\kern0pt}\ {\isacharequal}{\kern0pt}\ {\isadigit{2}}{\isachardoublequoteclose}\isanewline
\ \ \ \ \isacommand{using}\isamarkupfalse%
\ above{\isacharunderscore}{\kern0pt}b{\isacharunderscore}{\kern0pt}eq{\isacharunderscore}{\kern0pt}ab\ card{\isacharunderscore}{\kern0pt}above{\isacharunderscore}{\kern0pt}b{\isacharunderscore}{\kern0pt}eq{\isacharunderscore}{\kern0pt}{\isadigit{2}}\isanewline
\ \ \ \ \isacommand{by}\isamarkupfalse%
\ presburger\isanewline
\isacommand{qed}\isamarkupfalse%
%
\endisatagproof
{\isafoldproof}%
%
\isadelimproof
\isanewline
%
\endisadelimproof
\isanewline
\isacommand{theorem}\isamarkupfalse%
\ drop{\isacharunderscore}{\kern0pt}zero{\isacharunderscore}{\kern0pt}mod{\isacharunderscore}{\kern0pt}rej{\isacharunderscore}{\kern0pt}zero{\isacharbrackleft}{\kern0pt}simp{\isacharbrackright}{\kern0pt}{\isacharcolon}{\kern0pt}\isanewline
\ \ \isakeyword{assumes}\ order{\isacharcolon}{\kern0pt}\ {\isachardoublequoteopen}linear{\isacharunderscore}{\kern0pt}order\ r{\isachardoublequoteclose}\isanewline
\ \ \isakeyword{shows}\ {\isachardoublequoteopen}rejects\ {\isadigit{0}}\ {\isacharparenleft}{\kern0pt}drop{\isacharunderscore}{\kern0pt}module\ {\isadigit{0}}\ r{\isacharparenright}{\kern0pt}{\isachardoublequoteclose}\isanewline
%
\isadelimproof
\ \ %
\endisadelimproof
%
\isatagproof
\isacommand{unfolding}\isamarkupfalse%
\ rejects{\isacharunderscore}{\kern0pt}def\isanewline
\isacommand{proof}\isamarkupfalse%
\ {\isacharparenleft}{\kern0pt}safe{\isacharparenright}{\kern0pt}\isanewline
\ \ \isacommand{show}\isamarkupfalse%
\ {\isachardoublequoteopen}electoral{\isacharunderscore}{\kern0pt}module\ {\isacharparenleft}{\kern0pt}drop{\isacharunderscore}{\kern0pt}module\ {\isadigit{0}}\ r{\isacharparenright}{\kern0pt}{\isachardoublequoteclose}\isanewline
\ \ \ \ \isacommand{using}\isamarkupfalse%
\ order\isanewline
\ \ \ \ \isacommand{by}\isamarkupfalse%
\ simp\isanewline
\isacommand{next}\isamarkupfalse%
\isanewline
\ \ \isacommand{fix}\isamarkupfalse%
\isanewline
\ \ \ \ A\ {\isacharcolon}{\kern0pt}{\isacharcolon}{\kern0pt}\ {\isachardoublequoteopen}{\isacharprime}{\kern0pt}a\ set{\isachardoublequoteclose}\ \isakeyword{and}\isanewline
\ \ \ \ p\ {\isacharcolon}{\kern0pt}{\isacharcolon}{\kern0pt}\ {\isachardoublequoteopen}{\isacharprime}{\kern0pt}a\ Profile{\isachardoublequoteclose}\isanewline
\ \ \isacommand{assume}\isamarkupfalse%
\isanewline
\ \ \ \ card{\isacharunderscore}{\kern0pt}pos{\isacharcolon}{\kern0pt}\ {\isachardoublequoteopen}{\isadigit{0}}\ {\isasymle}\ card\ A{\isachardoublequoteclose}\ \isakeyword{and}\isanewline
\ \ \ \ finite{\isacharunderscore}{\kern0pt}A{\isacharcolon}{\kern0pt}\ {\isachardoublequoteopen}finite\ A{\isachardoublequoteclose}\ \isakeyword{and}\isanewline
\ \ \ \ prof{\isacharunderscore}{\kern0pt}A{\isacharcolon}{\kern0pt}\ {\isachardoublequoteopen}profile\ A\ p{\isachardoublequoteclose}\isanewline
\ \ \isacommand{have}\isamarkupfalse%
\ f{\isadigit{1}}{\isacharcolon}{\kern0pt}\ {\isachardoublequoteopen}connex\ UNIV\ r{\isachardoublequoteclose}\isanewline
\ \ \ \ \isacommand{using}\isamarkupfalse%
\ assms\ lin{\isacharunderscore}{\kern0pt}ord{\isacharunderscore}{\kern0pt}imp{\isacharunderscore}{\kern0pt}connex\isanewline
\ \ \ \ \isacommand{by}\isamarkupfalse%
\ auto\isanewline
\ \ \isacommand{obtain}\isamarkupfalse%
\ aa\ {\isacharcolon}{\kern0pt}{\isacharcolon}{\kern0pt}\ {\isachardoublequoteopen}{\isacharparenleft}{\kern0pt}{\isacharprime}{\kern0pt}a\ {\isasymRightarrow}\ bool{\isacharparenright}{\kern0pt}\ {\isasymRightarrow}\ {\isacharprime}{\kern0pt}a{\isachardoublequoteclose}\ \isakeyword{where}\isanewline
\ \ \ \ f{\isadigit{2}}{\isacharcolon}{\kern0pt}\isanewline
\ \ \ \ {\isachardoublequoteopen}{\isasymforall}p{\isachardot}{\kern0pt}\ {\isacharparenleft}{\kern0pt}Collect\ p\ {\isacharequal}{\kern0pt}\ {\isacharbraceleft}{\kern0pt}{\isacharbraceright}{\kern0pt}\ {\isasymlongrightarrow}\ {\isacharparenleft}{\kern0pt}{\isasymforall}a{\isachardot}{\kern0pt}\ {\isasymnot}\ p\ a{\isacharparenright}{\kern0pt}{\isacharparenright}{\kern0pt}\ {\isasymand}\isanewline
\ \ \ \ \ \ \ \ \ \ {\isacharparenleft}{\kern0pt}Collect\ p\ {\isasymnoteq}\ {\isacharbraceleft}{\kern0pt}{\isacharbraceright}{\kern0pt}\ {\isasymlongrightarrow}\ p\ {\isacharparenleft}{\kern0pt}aa\ p{\isacharparenright}{\kern0pt}{\isacharparenright}{\kern0pt}{\isachardoublequoteclose}\isanewline
\ \ \ \ \isacommand{by}\isamarkupfalse%
\ moura\isanewline
\ \ \isacommand{have}\isamarkupfalse%
\ f{\isadigit{3}}{\isacharcolon}{\kern0pt}\ {\isachardoublequoteopen}{\isasymforall}a{\isachardot}{\kern0pt}\ {\isacharparenleft}{\kern0pt}a{\isacharcolon}{\kern0pt}{\isacharcolon}{\kern0pt}{\isacharprime}{\kern0pt}a{\isacharparenright}{\kern0pt}\ {\isasymnotin}\ {\isacharbraceleft}{\kern0pt}{\isacharbraceright}{\kern0pt}{\isachardoublequoteclose}\isanewline
\ \ \ \ \isacommand{using}\isamarkupfalse%
\ empty{\isacharunderscore}{\kern0pt}iff\isanewline
\ \ \ \ \isacommand{by}\isamarkupfalse%
\ simp\isanewline
\ \ \isacommand{have}\isamarkupfalse%
\ connex{\isacharcolon}{\kern0pt}\isanewline
\ \ \ \ {\isachardoublequoteopen}connex\ A\ {\isacharparenleft}{\kern0pt}limit\ A\ r{\isacharparenright}{\kern0pt}{\isachardoublequoteclose}\isanewline
\ \ \ \ \isacommand{using}\isamarkupfalse%
\ f{\isadigit{1}}\ limit{\isacharunderscore}{\kern0pt}presv{\isacharunderscore}{\kern0pt}connex\ subset{\isacharunderscore}{\kern0pt}UNIV\isanewline
\ \ \ \ \isacommand{by}\isamarkupfalse%
\ metis\isanewline
\ \ \isacommand{have}\isamarkupfalse%
\ rej{\isacharunderscore}{\kern0pt}drop{\isacharunderscore}{\kern0pt}eq{\isacharunderscore}{\kern0pt}def{\isacharunderscore}{\kern0pt}pass{\isacharcolon}{\kern0pt}\isanewline
\ \ \ \ {\isachardoublequoteopen}reject\ {\isacharparenleft}{\kern0pt}drop{\isacharunderscore}{\kern0pt}module\ {\isadigit{0}}\ r{\isacharparenright}{\kern0pt}\ {\isacharequal}{\kern0pt}\ defer\ {\isacharparenleft}{\kern0pt}pass{\isacharunderscore}{\kern0pt}module\ {\isadigit{0}}\ r{\isacharparenright}{\kern0pt}{\isachardoublequoteclose}\isanewline
\ \ \ \ \isacommand{by}\isamarkupfalse%
\ simp\isanewline
\ \ \isacommand{have}\isamarkupfalse%
\ f{\isadigit{4}}{\isacharcolon}{\kern0pt}\isanewline
\ \ \ \ {\isachardoublequoteopen}{\isasymforall}a\ Aa{\isachardot}{\kern0pt}\isanewline
\ \ \ \ \ \ {\isasymnot}\ connex\ Aa\ {\isacharparenleft}{\kern0pt}limit\ A\ r{\isacharparenright}{\kern0pt}\ {\isasymor}\ a\ {\isasymnotin}\ Aa\ {\isasymor}\ a\ {\isasymnotin}\ A\ {\isasymor}\isanewline
\ \ \ \ \ \ \ \ {\isasymnot}\ card\ {\isacharparenleft}{\kern0pt}above\ {\isacharparenleft}{\kern0pt}limit\ A\ r{\isacharparenright}{\kern0pt}\ a{\isacharparenright}{\kern0pt}\ {\isasymle}\ {\isadigit{0}}{\isachardoublequoteclose}\isanewline
\ \ \ \ \isacommand{using}\isamarkupfalse%
\ above{\isacharunderscore}{\kern0pt}connex\ above{\isacharunderscore}{\kern0pt}presv{\isacharunderscore}{\kern0pt}limit\ bot{\isacharunderscore}{\kern0pt}nat{\isacharunderscore}{\kern0pt}{\isadigit{0}}{\isachardot}{\kern0pt}extremum{\isacharunderscore}{\kern0pt}uniqueI\isanewline
\ \ \ \ \ \ \ \ \ \ card{\isacharunderscore}{\kern0pt}{\isadigit{0}}{\isacharunderscore}{\kern0pt}eq\ emptyE\ finite{\isacharunderscore}{\kern0pt}A\ order\ rev{\isacharunderscore}{\kern0pt}finite{\isacharunderscore}{\kern0pt}subset\isanewline
\ \ \ \ \isacommand{by}\isamarkupfalse%
\ {\isacharparenleft}{\kern0pt}metis\ {\isacharparenleft}{\kern0pt}lifting{\isacharparenright}{\kern0pt}{\isacharparenright}{\kern0pt}\isanewline
\ \ \isacommand{have}\isamarkupfalse%
\ {\isachardoublequoteopen}{\isacharbraceleft}{\kern0pt}a\ {\isasymin}\ A{\isachardot}{\kern0pt}\ card{\isacharparenleft}{\kern0pt}above\ {\isacharparenleft}{\kern0pt}limit\ A\ r{\isacharparenright}{\kern0pt}\ a{\isacharparenright}{\kern0pt}\ {\isasymle}\ {\isadigit{0}}{\isacharbraceright}{\kern0pt}\ {\isacharequal}{\kern0pt}\ {\isacharbraceleft}{\kern0pt}{\isacharbraceright}{\kern0pt}{\isachardoublequoteclose}\isanewline
\ \ \ \ \isacommand{using}\isamarkupfalse%
\ connex\ f{\isadigit{4}}\isanewline
\ \ \ \ \isacommand{by}\isamarkupfalse%
\ auto\isanewline
\ \ \isacommand{hence}\isamarkupfalse%
\ {\isachardoublequoteopen}card\ {\isacharbraceleft}{\kern0pt}a\ {\isasymin}\ A{\isachardot}{\kern0pt}\ card{\isacharparenleft}{\kern0pt}above\ {\isacharparenleft}{\kern0pt}limit\ A\ r{\isacharparenright}{\kern0pt}\ a{\isacharparenright}{\kern0pt}\ {\isasymle}\ {\isadigit{0}}{\isacharbraceright}{\kern0pt}\ {\isacharequal}{\kern0pt}\ {\isadigit{0}}{\isachardoublequoteclose}\isanewline
\ \ \ \ \isacommand{using}\isamarkupfalse%
\ card{\isachardot}{\kern0pt}empty\isanewline
\ \ \ \ \isacommand{by}\isamarkupfalse%
\ {\isacharparenleft}{\kern0pt}metis\ {\isacharparenleft}{\kern0pt}full{\isacharunderscore}{\kern0pt}types{\isacharparenright}{\kern0pt}{\isacharparenright}{\kern0pt}\isanewline
\ \ \isacommand{thus}\isamarkupfalse%
\ {\isachardoublequoteopen}card\ {\isacharparenleft}{\kern0pt}reject\ {\isacharparenleft}{\kern0pt}drop{\isacharunderscore}{\kern0pt}module\ {\isadigit{0}}\ r{\isacharparenright}{\kern0pt}\ A\ p{\isacharparenright}{\kern0pt}\ {\isacharequal}{\kern0pt}\ {\isadigit{0}}{\isachardoublequoteclose}\isanewline
\ \ \ \ \isacommand{by}\isamarkupfalse%
\ simp\isanewline
\isacommand{qed}\isamarkupfalse%
%
\endisatagproof
{\isafoldproof}%
%
\isadelimproof
\isanewline
%
\endisadelimproof
\isanewline
\isanewline
\isacommand{theorem}\isamarkupfalse%
\ drop{\isacharunderscore}{\kern0pt}two{\isacharunderscore}{\kern0pt}mod{\isacharunderscore}{\kern0pt}rej{\isacharunderscore}{\kern0pt}two{\isacharbrackleft}{\kern0pt}simp{\isacharbrackright}{\kern0pt}{\isacharcolon}{\kern0pt}\isanewline
\ \ \isakeyword{assumes}\ order{\isacharcolon}{\kern0pt}\ {\isachardoublequoteopen}linear{\isacharunderscore}{\kern0pt}order\ r{\isachardoublequoteclose}\isanewline
\ \ \isakeyword{shows}\ {\isachardoublequoteopen}rejects\ {\isadigit{2}}\ {\isacharparenleft}{\kern0pt}drop{\isacharunderscore}{\kern0pt}module\ {\isadigit{2}}\ r{\isacharparenright}{\kern0pt}{\isachardoublequoteclose}\isanewline
%
\isadelimproof
%
\endisadelimproof
%
\isatagproof
\isacommand{proof}\isamarkupfalse%
\ {\isacharminus}{\kern0pt}\isanewline
\ \ \isacommand{have}\isamarkupfalse%
\ rej{\isacharunderscore}{\kern0pt}drop{\isacharunderscore}{\kern0pt}eq{\isacharunderscore}{\kern0pt}def{\isacharunderscore}{\kern0pt}pass{\isacharcolon}{\kern0pt}\isanewline
\ \ \ \ {\isachardoublequoteopen}reject\ {\isacharparenleft}{\kern0pt}drop{\isacharunderscore}{\kern0pt}module\ {\isadigit{2}}\ r{\isacharparenright}{\kern0pt}\ {\isacharequal}{\kern0pt}\ defer\ {\isacharparenleft}{\kern0pt}pass{\isacharunderscore}{\kern0pt}module\ {\isadigit{2}}\ r{\isacharparenright}{\kern0pt}{\isachardoublequoteclose}\isanewline
\ \ \ \ \isacommand{by}\isamarkupfalse%
\ simp\isanewline
\ \ \isacommand{thus}\isamarkupfalse%
\ {\isacharquery}{\kern0pt}thesis\isanewline
\ \ \isacommand{proof}\isamarkupfalse%
\ {\isacharminus}{\kern0pt}\isanewline
\ \ \ \ \isacommand{obtain}\isamarkupfalse%
\isanewline
\ \ \ \ \ \ AA\ {\isacharcolon}{\kern0pt}{\isacharcolon}{\kern0pt}\ {\isachardoublequoteopen}{\isacharparenleft}{\kern0pt}{\isacharprime}{\kern0pt}a\ Electoral{\isacharunderscore}{\kern0pt}Module{\isacharparenright}{\kern0pt}\ {\isasymRightarrow}\ nat\ {\isasymRightarrow}\ {\isacharprime}{\kern0pt}a\ set{\isachardoublequoteclose}\ \isakeyword{and}\isanewline
\ \ \ \ \ \ rrs\ {\isacharcolon}{\kern0pt}{\isacharcolon}{\kern0pt}\ {\isachardoublequoteopen}{\isacharparenleft}{\kern0pt}{\isacharprime}{\kern0pt}a\ Electoral{\isacharunderscore}{\kern0pt}Module{\isacharparenright}{\kern0pt}\ {\isasymRightarrow}\ nat\ {\isasymRightarrow}\ {\isacharprime}{\kern0pt}a\ Profile{\isachardoublequoteclose}\ \isakeyword{where}\isanewline
\ \ \ \ \ \ {\isachardoublequoteopen}{\isasymforall}x{\isadigit{0}}\ x{\isadigit{1}}{\isachardot}{\kern0pt}\ {\isacharparenleft}{\kern0pt}{\isasymexists}v{\isadigit{2}}\ v{\isadigit{3}}{\isachardot}{\kern0pt}\ {\isacharparenleft}{\kern0pt}x{\isadigit{1}}\ {\isasymle}\ card\ v{\isadigit{2}}\ {\isasymand}\ finite{\isacharunderscore}{\kern0pt}profile\ v{\isadigit{2}}\ v{\isadigit{3}}{\isacharparenright}{\kern0pt}\ {\isasymand}\isanewline
\ \ \ \ \ \ \ \ \ \ card\ {\isacharparenleft}{\kern0pt}reject\ x{\isadigit{0}}\ v{\isadigit{2}}\ v{\isadigit{3}}{\isacharparenright}{\kern0pt}\ {\isasymnoteq}\ x{\isadigit{1}}{\isacharparenright}{\kern0pt}\ {\isacharequal}{\kern0pt}\isanewline
\ \ \ \ \ \ \ \ \ \ \ \ \ \ {\isacharparenleft}{\kern0pt}{\isacharparenleft}{\kern0pt}x{\isadigit{1}}\ {\isasymle}\ card\ {\isacharparenleft}{\kern0pt}AA\ x{\isadigit{0}}\ x{\isadigit{1}}{\isacharparenright}{\kern0pt}\ {\isasymand}\isanewline
\ \ \ \ \ \ \ \ \ \ \ \ \ \ \ \ finite{\isacharunderscore}{\kern0pt}profile\ {\isacharparenleft}{\kern0pt}AA\ x{\isadigit{0}}\ x{\isadigit{1}}{\isacharparenright}{\kern0pt}\ {\isacharparenleft}{\kern0pt}rrs\ x{\isadigit{0}}\ x{\isadigit{1}}{\isacharparenright}{\kern0pt}{\isacharparenright}{\kern0pt}\ {\isasymand}\isanewline
\ \ \ \ \ \ \ \ \ \ \ \ \ \ \ \ card\ {\isacharparenleft}{\kern0pt}reject\ x{\isadigit{0}}\ {\isacharparenleft}{\kern0pt}AA\ x{\isadigit{0}}\ x{\isadigit{1}}{\isacharparenright}{\kern0pt}\ {\isacharparenleft}{\kern0pt}rrs\ x{\isadigit{0}}\ x{\isadigit{1}}{\isacharparenright}{\kern0pt}{\isacharparenright}{\kern0pt}\ {\isasymnoteq}\ x{\isadigit{1}}{\isacharparenright}{\kern0pt}{\isachardoublequoteclose}\isanewline
\ \ \ \ \ \ \isacommand{by}\isamarkupfalse%
\ moura\isanewline
\ \ \ \ \isacommand{hence}\isamarkupfalse%
\isanewline
\ \ \ \ \ \ {\isachardoublequoteopen}{\isasymforall}n\ f{\isachardot}{\kern0pt}\ {\isacharparenleft}{\kern0pt}{\isasymnot}\ rejects\ n\ f\ {\isasymor}\ electoral{\isacharunderscore}{\kern0pt}module\ f\ {\isasymand}\isanewline
\ \ \ \ \ \ \ \ \ \ {\isacharparenleft}{\kern0pt}{\isasymforall}A\ rs{\isachardot}{\kern0pt}\ {\isacharparenleft}{\kern0pt}{\isasymnot}\ n\ {\isasymle}\ card\ A\ {\isasymor}\ infinite\ A\ {\isasymor}\ {\isasymnot}\ profile\ A\ rs{\isacharparenright}{\kern0pt}\ {\isasymor}\isanewline
\ \ \ \ \ \ \ \ \ \ \ \ \ \ card\ {\isacharparenleft}{\kern0pt}reject\ f\ A\ rs{\isacharparenright}{\kern0pt}\ {\isacharequal}{\kern0pt}\ n{\isacharparenright}{\kern0pt}{\isacharparenright}{\kern0pt}\ {\isasymand}\isanewline
\ \ \ \ \ \ \ \ \ \ {\isacharparenleft}{\kern0pt}rejects\ n\ f\ {\isasymor}\ {\isasymnot}\ electoral{\isacharunderscore}{\kern0pt}module\ f\ {\isasymor}\ {\isacharparenleft}{\kern0pt}n\ {\isasymle}\ card\ {\isacharparenleft}{\kern0pt}AA\ f\ n{\isacharparenright}{\kern0pt}\ {\isasymand}\isanewline
\ \ \ \ \ \ \ \ \ \ \ \ \ \ finite{\isacharunderscore}{\kern0pt}profile\ {\isacharparenleft}{\kern0pt}AA\ f\ n{\isacharparenright}{\kern0pt}\ {\isacharparenleft}{\kern0pt}rrs\ f\ n{\isacharparenright}{\kern0pt}{\isacharparenright}{\kern0pt}\ {\isasymand}\isanewline
\ \ \ \ \ \ \ \ \ \ \ \ \ \ card\ {\isacharparenleft}{\kern0pt}reject\ f\ {\isacharparenleft}{\kern0pt}AA\ f\ n{\isacharparenright}{\kern0pt}\ {\isacharparenleft}{\kern0pt}rrs\ f\ n{\isacharparenright}{\kern0pt}{\isacharparenright}{\kern0pt}\ {\isasymnoteq}\ n{\isacharparenright}{\kern0pt}{\isachardoublequoteclose}\isanewline
\ \ \ \ \ \ \isacommand{using}\isamarkupfalse%
\ rejects{\isacharunderscore}{\kern0pt}def\isanewline
\ \ \ \ \ \ \isacommand{by}\isamarkupfalse%
\ force\isanewline
\ \ \ \ \isacommand{hence}\isamarkupfalse%
\ f{\isadigit{1}}{\isacharcolon}{\kern0pt}\isanewline
\ \ \ \ \ \ {\isachardoublequoteopen}{\isasymforall}n\ f{\isachardot}{\kern0pt}\ {\isacharparenleft}{\kern0pt}{\isasymnot}\ rejects\ n\ f\ {\isasymor}\ electoral{\isacharunderscore}{\kern0pt}module\ f\ {\isasymand}\isanewline
\ \ \ \ \ \ \ \ {\isacharparenleft}{\kern0pt}{\isasymforall}A\ rs{\isachardot}{\kern0pt}\ {\isasymnot}\ n\ {\isasymle}\ card\ A\ {\isasymor}\ infinite\ A\ {\isasymor}\ {\isasymnot}\ profile\ A\ rs\ {\isasymor}\isanewline
\ \ \ \ \ \ \ \ \ \ \ \ card\ {\isacharparenleft}{\kern0pt}reject\ f\ A\ rs{\isacharparenright}{\kern0pt}\ {\isacharequal}{\kern0pt}\ n{\isacharparenright}{\kern0pt}{\isacharparenright}{\kern0pt}\ {\isasymand}\isanewline
\ \ \ \ \ \ \ \ {\isacharparenleft}{\kern0pt}rejects\ n\ f\ {\isasymor}\ {\isasymnot}\ electoral{\isacharunderscore}{\kern0pt}module\ f\ {\isasymor}\ n\ {\isasymle}\ card\ {\isacharparenleft}{\kern0pt}AA\ f\ n{\isacharparenright}{\kern0pt}\ {\isasymand}\isanewline
\ \ \ \ \ \ \ \ \ \ \ \ finite\ {\isacharparenleft}{\kern0pt}AA\ f\ n{\isacharparenright}{\kern0pt}\ {\isasymand}\ profile\ {\isacharparenleft}{\kern0pt}AA\ f\ n{\isacharparenright}{\kern0pt}\ {\isacharparenleft}{\kern0pt}rrs\ f\ n{\isacharparenright}{\kern0pt}\ {\isasymand}\isanewline
\ \ \ \ \ \ \ \ \ \ \ \ card\ {\isacharparenleft}{\kern0pt}reject\ f\ {\isacharparenleft}{\kern0pt}AA\ f\ n{\isacharparenright}{\kern0pt}\ {\isacharparenleft}{\kern0pt}rrs\ f\ n{\isacharparenright}{\kern0pt}{\isacharparenright}{\kern0pt}\ {\isasymnoteq}\ n{\isacharparenright}{\kern0pt}{\isachardoublequoteclose}\isanewline
\ \ \ \ \ \ \isacommand{by}\isamarkupfalse%
\ presburger\isanewline
\ \ \ \ \isacommand{have}\isamarkupfalse%
\isanewline
\ \ \ \ \ \ {\isachardoublequoteopen}{\isasymnot}\ {\isadigit{2}}\ {\isasymle}\ card\ {\isacharparenleft}{\kern0pt}AA\ {\isacharparenleft}{\kern0pt}drop{\isacharunderscore}{\kern0pt}module\ {\isadigit{2}}\ r{\isacharparenright}{\kern0pt}\ {\isadigit{2}}{\isacharparenright}{\kern0pt}\ {\isasymor}\isanewline
\ \ \ \ \ \ \ \ \ \ infinite\ {\isacharparenleft}{\kern0pt}AA\ {\isacharparenleft}{\kern0pt}drop{\isacharunderscore}{\kern0pt}module\ {\isadigit{2}}\ r{\isacharparenright}{\kern0pt}\ {\isadigit{2}}{\isacharparenright}{\kern0pt}\ {\isasymor}\isanewline
\ \ \ \ \ \ \ \ \ \ {\isasymnot}\ profile\ {\isacharparenleft}{\kern0pt}AA\ {\isacharparenleft}{\kern0pt}drop{\isacharunderscore}{\kern0pt}module\ {\isadigit{2}}\ r{\isacharparenright}{\kern0pt}\ {\isadigit{2}}{\isacharparenright}{\kern0pt}\ {\isacharparenleft}{\kern0pt}rrs\ {\isacharparenleft}{\kern0pt}drop{\isacharunderscore}{\kern0pt}module\ {\isadigit{2}}\ r{\isacharparenright}{\kern0pt}\ {\isadigit{2}}{\isacharparenright}{\kern0pt}\ {\isasymor}\isanewline
\ \ \ \ \ \ \ \ \ \ card\ {\isacharparenleft}{\kern0pt}reject\ {\isacharparenleft}{\kern0pt}drop{\isacharunderscore}{\kern0pt}module\ {\isadigit{2}}\ r{\isacharparenright}{\kern0pt}\ {\isacharparenleft}{\kern0pt}AA\ {\isacharparenleft}{\kern0pt}drop{\isacharunderscore}{\kern0pt}module\ {\isadigit{2}}\ r{\isacharparenright}{\kern0pt}\ {\isadigit{2}}{\isacharparenright}{\kern0pt}\isanewline
\ \ \ \ \ \ \ \ \ \ \ \ \ \ {\isacharparenleft}{\kern0pt}rrs\ {\isacharparenleft}{\kern0pt}drop{\isacharunderscore}{\kern0pt}module\ {\isadigit{2}}\ r{\isacharparenright}{\kern0pt}\ {\isadigit{2}}{\isacharparenright}{\kern0pt}{\isacharparenright}{\kern0pt}\ {\isacharequal}{\kern0pt}\ {\isadigit{2}}{\isachardoublequoteclose}\isanewline
\ \ \ \ \ \ \isacommand{using}\isamarkupfalse%
\ rej{\isacharunderscore}{\kern0pt}drop{\isacharunderscore}{\kern0pt}eq{\isacharunderscore}{\kern0pt}def{\isacharunderscore}{\kern0pt}pass\ defers{\isacharunderscore}{\kern0pt}def\ order\isanewline
\ \ \ \ \ \ \ \ \ \ \ \ pass{\isacharunderscore}{\kern0pt}two{\isacharunderscore}{\kern0pt}mod{\isacharunderscore}{\kern0pt}def{\isacharunderscore}{\kern0pt}two\isanewline
\ \ \ \ \ \ \isacommand{by}\isamarkupfalse%
\ {\isacharparenleft}{\kern0pt}metis\ {\isacharparenleft}{\kern0pt}no{\isacharunderscore}{\kern0pt}types{\isacharparenright}{\kern0pt}{\isacharparenright}{\kern0pt}\isanewline
\ \ \ \ \isacommand{thus}\isamarkupfalse%
\ {\isacharquery}{\kern0pt}thesis\isanewline
\ \ \ \ \ \ \isacommand{using}\isamarkupfalse%
\ f{\isadigit{1}}\ drop{\isacharunderscore}{\kern0pt}mod{\isacharunderscore}{\kern0pt}sound\ order\isanewline
\ \ \ \ \ \ \isacommand{by}\isamarkupfalse%
\ blast\isanewline
\ \ \isacommand{qed}\isamarkupfalse%
\isanewline
\isacommand{qed}\isamarkupfalse%
%
\endisatagproof
{\isafoldproof}%
%
\isadelimproof
\isanewline
%
\endisadelimproof
%
\isadelimtheory
\isanewline
%
\endisadelimtheory
%
\isatagtheory
\isacommand{end}\isamarkupfalse%
%
\endisatagtheory
{\isafoldtheory}%
%
\isadelimtheory
%
\endisadelimtheory
%
\end{isabellebody}%
\endinput
%:%file=~/Documents/Studies/VotingRuleGenerator/virage/src/test/resources/verifiedVotingRuleConstruction/theories/Compositional_Framework/Composition_Rules/Result_Facts.thy%:%
%:%10=1%:%
%:%11=1%:%
%:%12=2%:%
%:%13=3%:%
%:%14=4%:%
%:%15=5%:%
%:%16=6%:%
%:%17=7%:%
%:%18=8%:%
%:%19=9%:%
%:%20=10%:%
%:%21=11%:%
%:%26=11%:%
%:%29=12%:%
%:%30=13%:%
%:%31=13%:%
%:%34=14%:%
%:%38=14%:%
%:%39=14%:%
%:%40=15%:%
%:%41=15%:%
%:%46=15%:%
%:%49=16%:%
%:%50=17%:%
%:%51=17%:%
%:%53=19%:%
%:%60=20%:%
%:%61=20%:%
%:%62=21%:%
%:%63=21%:%
%:%64=22%:%
%:%65=23%:%
%:%66=24%:%
%:%67=24%:%
%:%68=25%:%
%:%69=26%:%
%:%70=26%:%
%:%71=27%:%
%:%72=28%:%
%:%73=28%:%
%:%74=29%:%
%:%75=29%:%
%:%76=30%:%
%:%77=31%:%
%:%78=31%:%
%:%79=32%:%
%:%80=32%:%
%:%81=32%:%
%:%82=33%:%
%:%83=34%:%
%:%84=34%:%
%:%85=35%:%
%:%86=36%:%
%:%87=36%:%
%:%88=37%:%
%:%89=37%:%
%:%90=38%:%
%:%91=39%:%
%:%92=39%:%
%:%93=40%:%
%:%94=40%:%
%:%95=40%:%
%:%96=41%:%
%:%97=42%:%
%:%98=42%:%
%:%99=43%:%
%:%100=43%:%
%:%101=44%:%
%:%102=44%:%
%:%103=44%:%
%:%104=45%:%
%:%105=46%:%
%:%106=46%:%
%:%107=47%:%
%:%108=47%:%
%:%109=48%:%
%:%110=49%:%
%:%111=49%:%
%:%112=50%:%
%:%113=50%:%
%:%114=51%:%
%:%115=52%:%
%:%116=52%:%
%:%117=53%:%
%:%118=53%:%
%:%119=54%:%
%:%120=54%:%
%:%121=55%:%
%:%122=55%:%
%:%123=56%:%
%:%129=56%:%
%:%132=57%:%
%:%133=58%:%
%:%134=59%:%
%:%135=59%:%
%:%142=60%:%
%:%143=60%:%
%:%144=61%:%
%:%145=61%:%
%:%146=62%:%
%:%147=63%:%
%:%148=63%:%
%:%149=64%:%
%:%150=64%:%
%:%151=65%:%
%:%152=65%:%
%:%153=66%:%
%:%154=66%:%
%:%155=67%:%
%:%156=67%:%
%:%157=68%:%
%:%158=68%:%
%:%159=69%:%
%:%160=69%:%
%:%161=70%:%
%:%162=70%:%
%:%163=71%:%
%:%164=71%:%
%:%165=72%:%
%:%166=72%:%
%:%167=73%:%
%:%173=73%:%
%:%176=74%:%
%:%177=75%:%
%:%178=75%:%
%:%181=76%:%
%:%185=76%:%
%:%186=76%:%
%:%187=77%:%
%:%188=77%:%
%:%193=77%:%
%:%196=78%:%
%:%197=79%:%
%:%198=80%:%
%:%199=81%:%
%:%200=81%:%
%:%201=82%:%
%:%202=83%:%
%:%205=84%:%
%:%209=84%:%
%:%210=84%:%
%:%215=84%:%
%:%218=85%:%
%:%219=86%:%
%:%220=86%:%
%:%221=87%:%
%:%222=88%:%
%:%225=89%:%
%:%229=89%:%
%:%230=89%:%
%:%235=89%:%
%:%238=90%:%
%:%239=91%:%
%:%240=91%:%
%:%241=92%:%
%:%242=93%:%
%:%245=94%:%
%:%249=94%:%
%:%250=94%:%
%:%251=95%:%
%:%252=95%:%
%:%257=95%:%
%:%260=96%:%
%:%261=97%:%
%:%262=97%:%
%:%263=98%:%
%:%264=99%:%
%:%271=100%:%
%:%272=100%:%
%:%273=101%:%
%:%274=101%:%
%:%275=102%:%
%:%276=102%:%
%:%277=103%:%
%:%278=103%:%
%:%279=104%:%
%:%280=104%:%
%:%281=105%:%
%:%287=105%:%
%:%290=106%:%
%:%291=107%:%
%:%292=107%:%
%:%293=108%:%
%:%294=109%:%
%:%301=110%:%
%:%302=110%:%
%:%303=111%:%
%:%304=111%:%
%:%305=112%:%
%:%306=112%:%
%:%307=113%:%
%:%308=113%:%
%:%309=114%:%
%:%310=114%:%
%:%311=115%:%
%:%317=115%:%
%:%320=116%:%
%:%321=117%:%
%:%322=117%:%
%:%323=118%:%
%:%324=119%:%
%:%331=120%:%
%:%332=120%:%
%:%333=121%:%
%:%334=121%:%
%:%335=122%:%
%:%336=122%:%
%:%337=123%:%
%:%338=123%:%
%:%339=124%:%
%:%340=124%:%
%:%341=125%:%
%:%347=125%:%
%:%350=126%:%
%:%351=127%:%
%:%352=127%:%
%:%353=128%:%
%:%354=129%:%
%:%361=130%:%
%:%362=130%:%
%:%363=131%:%
%:%364=131%:%
%:%365=132%:%
%:%366=132%:%
%:%367=133%:%
%:%368=133%:%
%:%369=134%:%
%:%370=134%:%
%:%371=135%:%
%:%377=135%:%
%:%380=136%:%
%:%381=137%:%
%:%382=137%:%
%:%383=138%:%
%:%384=139%:%
%:%391=140%:%
%:%392=140%:%
%:%393=141%:%
%:%394=141%:%
%:%395=142%:%
%:%396=142%:%
%:%397=143%:%
%:%398=143%:%
%:%399=144%:%
%:%400=144%:%
%:%401=145%:%
%:%407=145%:%
%:%410=146%:%
%:%411=147%:%
%:%412=148%:%
%:%413=148%:%
%:%414=149%:%
%:%415=150%:%
%:%418=151%:%
%:%422=151%:%
%:%423=151%:%
%:%428=151%:%
%:%431=152%:%
%:%432=153%:%
%:%433=154%:%
%:%434=154%:%
%:%435=155%:%
%:%436=156%:%
%:%439=157%:%
%:%443=157%:%
%:%444=157%:%
%:%449=157%:%
%:%452=158%:%
%:%453=159%:%
%:%454=160%:%
%:%455=160%:%
%:%456=161%:%
%:%457=162%:%
%:%458=163%:%
%:%461=164%:%
%:%465=164%:%
%:%466=164%:%
%:%467=165%:%
%:%468=165%:%
%:%469=166%:%
%:%470=166%:%
%:%471=167%:%
%:%472=167%:%
%:%473=168%:%
%:%474=168%:%
%:%475=169%:%
%:%476=169%:%
%:%477=170%:%
%:%478=170%:%
%:%479=171%:%
%:%480=172%:%
%:%481=173%:%
%:%482=174%:%
%:%483=174%:%
%:%484=175%:%
%:%485=176%:%
%:%486=177%:%
%:%487=178%:%
%:%490=181%:%
%:%491=182%:%
%:%492=183%:%
%:%493=183%:%
%:%494=184%:%
%:%495=185%:%
%:%496=185%:%
%:%497=186%:%
%:%498=186%:%
%:%499=187%:%
%:%500=187%:%
%:%501=188%:%
%:%502=189%:%
%:%503=190%:%
%:%504=190%:%
%:%505=191%:%
%:%506=191%:%
%:%507=192%:%
%:%508=192%:%
%:%509=193%:%
%:%510=194%:%
%:%511=194%:%
%:%512=195%:%
%:%513=196%:%
%:%514=196%:%
%:%515=197%:%
%:%516=197%:%
%:%517=198%:%
%:%518=198%:%
%:%519=199%:%
%:%520=199%:%
%:%521=200%:%
%:%522=200%:%
%:%523=201%:%
%:%524=201%:%
%:%525=202%:%
%:%531=202%:%
%:%534=203%:%
%:%535=204%:%
%:%536=204%:%
%:%537=205%:%
%:%538=206%:%
%:%541=207%:%
%:%545=207%:%
%:%546=207%:%
%:%547=208%:%
%:%548=208%:%
%:%549=209%:%
%:%550=209%:%
%:%551=210%:%
%:%552=210%:%
%:%553=211%:%
%:%554=211%:%
%:%555=212%:%
%:%556=212%:%
%:%557=213%:%
%:%558=213%:%
%:%559=214%:%
%:%560=215%:%
%:%561=216%:%
%:%562=216%:%
%:%563=217%:%
%:%564=218%:%
%:%565=219%:%
%:%566=220%:%
%:%567=220%:%
%:%568=221%:%
%:%569=222%:%
%:%570=222%:%
%:%571=223%:%
%:%572=223%:%
%:%573=224%:%
%:%574=225%:%
%:%575=225%:%
%:%576=226%:%
%:%577=226%:%
%:%578=227%:%
%:%579=227%:%
%:%580=228%:%
%:%581=228%:%
%:%582=229%:%
%:%583=229%:%
%:%584=230%:%
%:%585=230%:%
%:%586=231%:%
%:%587=232%:%
%:%588=233%:%
%:%589=234%:%
%:%590=234%:%
%:%591=235%:%
%:%592=235%:%
%:%593=236%:%
%:%594=236%:%
%:%595=237%:%
%:%596=237%:%
%:%597=238%:%
%:%598=239%:%
%:%599=240%:%
%:%600=240%:%
%:%601=241%:%
%:%602=242%:%
%:%603=242%:%
%:%604=243%:%
%:%605=243%:%
%:%606=244%:%
%:%607=244%:%
%:%608=245%:%
%:%609=245%:%
%:%610=246%:%
%:%611=246%:%
%:%612=247%:%
%:%613=247%:%
%:%614=248%:%
%:%615=248%:%
%:%616=249%:%
%:%617=249%:%
%:%618=250%:%
%:%619=250%:%
%:%620=251%:%
%:%621=251%:%
%:%622=252%:%
%:%628=252%:%
%:%631=253%:%
%:%632=259%:%
%:%633=260%:%
%:%634=260%:%
%:%635=261%:%
%:%636=262%:%
%:%639=263%:%
%:%643=263%:%
%:%644=263%:%
%:%645=264%:%
%:%646=264%:%
%:%647=265%:%
%:%648=265%:%
%:%649=266%:%
%:%650=266%:%
%:%651=267%:%
%:%652=267%:%
%:%653=268%:%
%:%654=268%:%
%:%655=269%:%
%:%656=269%:%
%:%657=270%:%
%:%658=271%:%
%:%659=272%:%
%:%660=272%:%
%:%661=273%:%
%:%662=274%:%
%:%663=275%:%
%:%664=276%:%
%:%665=276%:%
%:%666=277%:%
%:%667=278%:%
%:%668=278%:%
%:%669=279%:%
%:%670=279%:%
%:%671=280%:%
%:%672=280%:%
%:%673=281%:%
%:%674=281%:%
%:%675=282%:%
%:%676=282%:%
%:%677=282%:%
%:%678=283%:%
%:%679=284%:%
%:%680=284%:%
%:%681=285%:%
%:%682=285%:%
%:%683=286%:%
%:%684=286%:%
%:%685=286%:%
%:%686=287%:%
%:%687=288%:%
%:%688=289%:%
%:%689=289%:%
%:%690=290%:%
%:%691=290%:%
%:%692=291%:%
%:%693=291%:%
%:%694=291%:%
%:%695=292%:%
%:%696=293%:%
%:%697=294%:%
%:%698=294%:%
%:%699=295%:%
%:%700=295%:%
%:%701=296%:%
%:%702=296%:%
%:%703=297%:%
%:%704=297%:%
%:%705=298%:%
%:%706=298%:%
%:%707=299%:%
%:%708=300%:%
%:%709=301%:%
%:%710=301%:%
%:%711=302%:%
%:%712=302%:%
%:%713=303%:%
%:%714=303%:%
%:%715=304%:%
%:%716=304%:%
%:%717=305%:%
%:%718=305%:%
%:%719=306%:%
%:%720=306%:%
%:%721=307%:%
%:%722=307%:%
%:%723=307%:%
%:%724=308%:%
%:%725=309%:%
%:%726=309%:%
%:%727=310%:%
%:%728=310%:%
%:%729=311%:%
%:%730=312%:%
%:%731=312%:%
%:%732=313%:%
%:%733=314%:%
%:%734=314%:%
%:%735=315%:%
%:%736=315%:%
%:%737=316%:%
%:%738=316%:%
%:%739=317%:%
%:%740=318%:%
%:%741=318%:%
%:%742=319%:%
%:%743=319%:%
%:%744=320%:%
%:%745=320%:%
%:%746=321%:%
%:%747=321%:%
%:%748=322%:%
%:%749=323%:%
%:%750=323%:%
%:%751=324%:%
%:%752=324%:%
%:%753=325%:%
%:%754=325%:%
%:%755=326%:%
%:%756=326%:%
%:%757=327%:%
%:%758=327%:%
%:%759=328%:%
%:%760=328%:%
%:%761=329%:%
%:%762=329%:%
%:%763=330%:%
%:%764=330%:%
%:%765=331%:%
%:%766=331%:%
%:%767=332%:%
%:%768=332%:%
%:%769=333%:%
%:%770=333%:%
%:%771=334%:%
%:%772=334%:%
%:%773=335%:%
%:%774=335%:%
%:%775=336%:%
%:%776=336%:%
%:%777=336%:%
%:%778=337%:%
%:%779=338%:%
%:%780=338%:%
%:%781=339%:%
%:%782=340%:%
%:%783=340%:%
%:%784=341%:%
%:%785=341%:%
%:%786=341%:%
%:%787=342%:%
%:%788=343%:%
%:%789=343%:%
%:%790=344%:%
%:%791=344%:%
%:%792=345%:%
%:%793=345%:%
%:%794=346%:%
%:%795=346%:%
%:%796=347%:%
%:%797=347%:%
%:%798=348%:%
%:%799=348%:%
%:%800=349%:%
%:%801=349%:%
%:%802=350%:%
%:%803=350%:%
%:%804=351%:%
%:%805=351%:%
%:%806=352%:%
%:%807=352%:%
%:%808=353%:%
%:%809=353%:%
%:%810=353%:%
%:%811=354%:%
%:%812=355%:%
%:%813=355%:%
%:%814=356%:%
%:%815=356%:%
%:%816=357%:%
%:%817=357%:%
%:%818=358%:%
%:%819=358%:%
%:%820=359%:%
%:%821=359%:%
%:%822=360%:%
%:%823=360%:%
%:%824=361%:%
%:%825=361%:%
%:%826=362%:%
%:%827=362%:%
%:%828=363%:%
%:%829=363%:%
%:%830=364%:%
%:%836=364%:%
%:%839=365%:%
%:%840=366%:%
%:%841=366%:%
%:%842=367%:%
%:%843=368%:%
%:%846=369%:%
%:%850=369%:%
%:%851=369%:%
%:%852=370%:%
%:%853=370%:%
%:%854=371%:%
%:%855=371%:%
%:%856=372%:%
%:%857=372%:%
%:%858=373%:%
%:%859=373%:%
%:%860=374%:%
%:%861=374%:%
%:%862=375%:%
%:%863=375%:%
%:%864=376%:%
%:%865=377%:%
%:%866=378%:%
%:%867=378%:%
%:%868=379%:%
%:%869=380%:%
%:%870=381%:%
%:%871=382%:%
%:%872=382%:%
%:%873=383%:%
%:%874=383%:%
%:%875=384%:%
%:%876=384%:%
%:%877=385%:%
%:%878=385%:%
%:%879=385%:%
%:%880=386%:%
%:%881=387%:%
%:%882=387%:%
%:%883=388%:%
%:%884=388%:%
%:%885=389%:%
%:%886=389%:%
%:%887=389%:%
%:%888=390%:%
%:%889=391%:%
%:%890=391%:%
%:%891=392%:%
%:%892=392%:%
%:%893=393%:%
%:%894=393%:%
%:%895=394%:%
%:%896=394%:%
%:%897=395%:%
%:%898=396%:%
%:%899=397%:%
%:%900=397%:%
%:%901=398%:%
%:%902=398%:%
%:%903=399%:%
%:%904=399%:%
%:%905=400%:%
%:%906=400%:%
%:%907=401%:%
%:%908=401%:%
%:%909=402%:%
%:%910=402%:%
%:%911=402%:%
%:%912=403%:%
%:%913=403%:%
%:%914=404%:%
%:%915=405%:%
%:%916=406%:%
%:%917=407%:%
%:%918=407%:%
%:%919=408%:%
%:%920=408%:%
%:%921=409%:%
%:%922=409%:%
%:%923=410%:%
%:%924=410%:%
%:%925=411%:%
%:%926=411%:%
%:%927=412%:%
%:%928=412%:%
%:%929=412%:%
%:%930=413%:%
%:%931=413%:%
%:%932=414%:%
%:%933=415%:%
%:%934=416%:%
%:%935=417%:%
%:%936=417%:%
%:%937=418%:%
%:%938=418%:%
%:%939=418%:%
%:%940=419%:%
%:%941=420%:%
%:%942=420%:%
%:%943=421%:%
%:%944=421%:%
%:%945=422%:%
%:%946=422%:%
%:%947=422%:%
%:%948=423%:%
%:%949=423%:%
%:%950=424%:%
%:%951=424%:%
%:%952=425%:%
%:%953=425%:%
%:%954=426%:%
%:%955=426%:%
%:%956=427%:%
%:%957=428%:%
%:%958=429%:%
%:%959=429%:%
%:%960=430%:%
%:%961=430%:%
%:%962=431%:%
%:%963=431%:%
%:%964=432%:%
%:%965=432%:%
%:%966=433%:%
%:%967=433%:%
%:%968=434%:%
%:%969=434%:%
%:%970=435%:%
%:%971=435%:%
%:%972=436%:%
%:%973=437%:%
%:%974=437%:%
%:%975=438%:%
%:%976=438%:%
%:%977=438%:%
%:%978=439%:%
%:%979=439%:%
%:%980=440%:%
%:%981=440%:%
%:%982=441%:%
%:%983=441%:%
%:%984=441%:%
%:%985=442%:%
%:%986=442%:%
%:%987=443%:%
%:%988=444%:%
%:%989=444%:%
%:%990=445%:%
%:%991=445%:%
%:%992=445%:%
%:%993=446%:%
%:%994=446%:%
%:%995=447%:%
%:%996=447%:%
%:%997=448%:%
%:%998=448%:%
%:%999=449%:%
%:%1000=449%:%
%:%1001=450%:%
%:%1002=450%:%
%:%1003=451%:%
%:%1004=451%:%
%:%1005=452%:%
%:%1006=452%:%
%:%1007=453%:%
%:%1008=453%:%
%:%1009=454%:%
%:%1010=454%:%
%:%1011=454%:%
%:%1012=455%:%
%:%1013=456%:%
%:%1014=456%:%
%:%1015=457%:%
%:%1016=457%:%
%:%1017=458%:%
%:%1018=458%:%
%:%1019=459%:%
%:%1020=459%:%
%:%1021=460%:%
%:%1022=460%:%
%:%1023=461%:%
%:%1024=461%:%
%:%1025=462%:%
%:%1026=462%:%
%:%1027=463%:%
%:%1028=463%:%
%:%1029=464%:%
%:%1030=464%:%
%:%1031=465%:%
%:%1032=465%:%
%:%1033=466%:%
%:%1034=466%:%
%:%1035=466%:%
%:%1036=467%:%
%:%1037=467%:%
%:%1038=468%:%
%:%1039=469%:%
%:%1040=470%:%
%:%1041=470%:%
%:%1042=471%:%
%:%1043=471%:%
%:%1044=471%:%
%:%1045=472%:%
%:%1046=473%:%
%:%1047=473%:%
%:%1048=474%:%
%:%1049=474%:%
%:%1050=475%:%
%:%1051=475%:%
%:%1052=476%:%
%:%1053=476%:%
%:%1054=477%:%
%:%1055=477%:%
%:%1056=478%:%
%:%1057=478%:%
%:%1058=479%:%
%:%1059=479%:%
%:%1060=480%:%
%:%1061=480%:%
%:%1062=480%:%
%:%1063=481%:%
%:%1064=481%:%
%:%1065=482%:%
%:%1066=482%:%
%:%1067=482%:%
%:%1068=483%:%
%:%1069=483%:%
%:%1070=484%:%
%:%1071=484%:%
%:%1072=484%:%
%:%1073=485%:%
%:%1074=486%:%
%:%1075=486%:%
%:%1076=487%:%
%:%1077=487%:%
%:%1078=488%:%
%:%1079=488%:%
%:%1080=489%:%
%:%1081=489%:%
%:%1082=490%:%
%:%1088=490%:%
%:%1091=491%:%
%:%1092=492%:%
%:%1093=492%:%
%:%1094=493%:%
%:%1095=494%:%
%:%1098=495%:%
%:%1102=495%:%
%:%1103=495%:%
%:%1104=496%:%
%:%1105=496%:%
%:%1106=497%:%
%:%1107=497%:%
%:%1108=498%:%
%:%1109=498%:%
%:%1110=499%:%
%:%1111=499%:%
%:%1112=500%:%
%:%1113=500%:%
%:%1114=501%:%
%:%1115=501%:%
%:%1116=502%:%
%:%1117=503%:%
%:%1118=504%:%
%:%1119=504%:%
%:%1120=505%:%
%:%1121=506%:%
%:%1122=507%:%
%:%1123=508%:%
%:%1124=508%:%
%:%1125=509%:%
%:%1126=509%:%
%:%1127=510%:%
%:%1128=510%:%
%:%1129=511%:%
%:%1130=511%:%
%:%1131=512%:%
%:%1132=513%:%
%:%1133=514%:%
%:%1134=515%:%
%:%1135=515%:%
%:%1136=516%:%
%:%1137=516%:%
%:%1138=517%:%
%:%1139=517%:%
%:%1140=518%:%
%:%1141=518%:%
%:%1142=519%:%
%:%1143=519%:%
%:%1144=520%:%
%:%1145=521%:%
%:%1146=521%:%
%:%1147=522%:%
%:%1148=522%:%
%:%1149=523%:%
%:%1150=523%:%
%:%1151=524%:%
%:%1152=525%:%
%:%1153=525%:%
%:%1154=526%:%
%:%1155=526%:%
%:%1156=527%:%
%:%1158=529%:%
%:%1159=530%:%
%:%1160=530%:%
%:%1161=531%:%
%:%1162=532%:%
%:%1163=532%:%
%:%1164=533%:%
%:%1165=533%:%
%:%1166=534%:%
%:%1167=534%:%
%:%1168=535%:%
%:%1169=535%:%
%:%1170=536%:%
%:%1171=536%:%
%:%1172=537%:%
%:%1173=537%:%
%:%1174=538%:%
%:%1175=538%:%
%:%1176=539%:%
%:%1177=539%:%
%:%1178=540%:%
%:%1179=540%:%
%:%1180=541%:%
%:%1186=541%:%
%:%1189=542%:%
%:%1190=546%:%
%:%1191=547%:%
%:%1192=547%:%
%:%1193=548%:%
%:%1194=549%:%
%:%1201=550%:%
%:%1202=550%:%
%:%1203=551%:%
%:%1204=551%:%
%:%1205=552%:%
%:%1206=553%:%
%:%1207=553%:%
%:%1208=554%:%
%:%1209=554%:%
%:%1210=555%:%
%:%1211=555%:%
%:%1212=556%:%
%:%1213=556%:%
%:%1214=557%:%
%:%1215=558%:%
%:%1216=559%:%
%:%1220=563%:%
%:%1221=564%:%
%:%1222=564%:%
%:%1223=565%:%
%:%1224=565%:%
%:%1225=566%:%
%:%1230=571%:%
%:%1231=572%:%
%:%1232=572%:%
%:%1233=573%:%
%:%1234=573%:%
%:%1235=574%:%
%:%1236=574%:%
%:%1237=575%:%
%:%1242=580%:%
%:%1243=581%:%
%:%1244=581%:%
%:%1245=582%:%
%:%1246=582%:%
%:%1247=583%:%
%:%1251=587%:%
%:%1252=588%:%
%:%1253=588%:%
%:%1254=589%:%
%:%1255=590%:%
%:%1256=590%:%
%:%1257=591%:%
%:%1258=591%:%
%:%1259=592%:%
%:%1260=592%:%
%:%1261=593%:%
%:%1262=593%:%
%:%1263=594%:%
%:%1264=594%:%
%:%1265=595%:%
%:%1271=595%:%
%:%1276=596%:%
%:%1281=597%:%
%
\begin{isabellebody}%
\setisabellecontext{Result{\isacharunderscore}{\kern0pt}Rules}%
%
\isadelimtheory
%
\endisadelimtheory
%
\isatagtheory
\isacommand{theory}\isamarkupfalse%
\ Result{\isacharunderscore}{\kern0pt}Rules\isanewline
\ \ \isakeyword{imports}\ {\isachardoublequoteopen}{\isachardot}{\kern0pt}{\isachardot}{\kern0pt}{\isacharslash}{\kern0pt}Properties{\isacharslash}{\kern0pt}Result{\isacharunderscore}{\kern0pt}Properties{\isachardoublequoteclose}\isanewline
\ \ \ \ \ \ \ \ \ \ {\isachardoublequoteopen}{\isachardot}{\kern0pt}{\isachardot}{\kern0pt}{\isacharslash}{\kern0pt}Components{\isacharslash}{\kern0pt}Basic{\isacharunderscore}{\kern0pt}Modules{\isacharslash}{\kern0pt}Elect{\isacharunderscore}{\kern0pt}Module{\isachardoublequoteclose}\isanewline
\ \ \ \ \ \ \ \ \ \ {\isachardoublequoteopen}{\isachardot}{\kern0pt}{\isachardot}{\kern0pt}{\isacharslash}{\kern0pt}Components{\isacharslash}{\kern0pt}Composites{\isacharslash}{\kern0pt}Composite{\isacharunderscore}{\kern0pt}Structures{\isachardoublequoteclose}\isanewline
\ \ \ \ \ \ \ \ \ \ Result{\isacharunderscore}{\kern0pt}Facts\isanewline
\isanewline
\isakeyword{begin}%
\endisatagtheory
{\isafoldtheory}%
%
\isadelimtheory
\isanewline
%
\endisadelimtheory
\isanewline
\isacommand{theorem}\isamarkupfalse%
\ electing{\isacharunderscore}{\kern0pt}imp{\isacharunderscore}{\kern0pt}non{\isacharunderscore}{\kern0pt}blocking{\isacharcolon}{\kern0pt}\isanewline
\ \ \isakeyword{assumes}\ electing{\isacharcolon}{\kern0pt}\ {\isachardoublequoteopen}electing\ m{\isachardoublequoteclose}\isanewline
\ \ \isakeyword{shows}\ {\isachardoublequoteopen}non{\isacharunderscore}{\kern0pt}blocking\ m{\isachardoublequoteclose}\isanewline
%
\isadelimproof
\ \ %
\endisadelimproof
%
\isatagproof
\isacommand{using}\isamarkupfalse%
\ Diff{\isacharunderscore}{\kern0pt}disjoint\ Diff{\isacharunderscore}{\kern0pt}empty\ Int{\isacharunderscore}{\kern0pt}absorb{\isadigit{2}}\ electing\isanewline
\ \ \ \ \ \ \ \ defer{\isacharunderscore}{\kern0pt}in{\isacharunderscore}{\kern0pt}alts\ elect{\isacharunderscore}{\kern0pt}in{\isacharunderscore}{\kern0pt}alts\ electing{\isacharunderscore}{\kern0pt}def\isanewline
\ \ \ \ \ \ \ \ non{\isacharunderscore}{\kern0pt}blocking{\isacharunderscore}{\kern0pt}def\ reject{\isacharunderscore}{\kern0pt}not{\isacharunderscore}{\kern0pt}elec{\isacharunderscore}{\kern0pt}or{\isacharunderscore}{\kern0pt}def\isanewline
\ \ \isacommand{by}\isamarkupfalse%
\ {\isacharparenleft}{\kern0pt}smt\ {\isacharparenleft}{\kern0pt}verit{\isacharcomma}{\kern0pt}\ ccfv{\isacharunderscore}{\kern0pt}SIG{\isacharparenright}{\kern0pt}{\isacharparenright}{\kern0pt}%
\endisatagproof
{\isafoldproof}%
%
\isadelimproof
\isanewline
%
\endisadelimproof
\isanewline
\isanewline
\isacommand{theorem}\isamarkupfalse%
\ seq{\isacharunderscore}{\kern0pt}comp{\isacharunderscore}{\kern0pt}presv{\isacharunderscore}{\kern0pt}non{\isacharunderscore}{\kern0pt}blocking{\isacharbrackleft}{\kern0pt}simp{\isacharbrackright}{\kern0pt}{\isacharcolon}{\kern0pt}\isanewline
\ \ \isakeyword{assumes}\isanewline
\ \ \ \ non{\isacharunderscore}{\kern0pt}blocking{\isacharunderscore}{\kern0pt}m{\isacharcolon}{\kern0pt}\ {\isachardoublequoteopen}non{\isacharunderscore}{\kern0pt}blocking\ m{\isachardoublequoteclose}\ \isakeyword{and}\isanewline
\ \ \ \ non{\isacharunderscore}{\kern0pt}blocking{\isacharunderscore}{\kern0pt}n{\isacharcolon}{\kern0pt}\ {\isachardoublequoteopen}non{\isacharunderscore}{\kern0pt}blocking\ n{\isachardoublequoteclose}\isanewline
\ \ \isakeyword{shows}\ {\isachardoublequoteopen}non{\isacharunderscore}{\kern0pt}blocking\ {\isacharparenleft}{\kern0pt}m\ {\isasymtriangleright}\ n{\isacharparenright}{\kern0pt}{\isachardoublequoteclose}\isanewline
%
\isadelimproof
%
\endisadelimproof
%
\isatagproof
\isacommand{proof}\isamarkupfalse%
\ {\isacharminus}{\kern0pt}\isanewline
\ \ \isacommand{fix}\isamarkupfalse%
\isanewline
\ \ \ \ A\ {\isacharcolon}{\kern0pt}{\isacharcolon}{\kern0pt}\ {\isachardoublequoteopen}{\isacharprime}{\kern0pt}a\ set{\isachardoublequoteclose}\ \isakeyword{and}\isanewline
\ \ \ \ p\ {\isacharcolon}{\kern0pt}{\isacharcolon}{\kern0pt}\ {\isachardoublequoteopen}{\isacharprime}{\kern0pt}a\ Profile{\isachardoublequoteclose}\isanewline
\ \ \isacommand{let}\isamarkupfalse%
\ {\isacharquery}{\kern0pt}input{\isacharunderscore}{\kern0pt}sound\ {\isacharequal}{\kern0pt}\ {\isachardoublequoteopen}{\isacharparenleft}{\kern0pt}{\isacharparenleft}{\kern0pt}A{\isacharcolon}{\kern0pt}{\isacharcolon}{\kern0pt}{\isacharprime}{\kern0pt}a\ set{\isacharparenright}{\kern0pt}\ {\isasymnoteq}\ {\isacharbraceleft}{\kern0pt}{\isacharbraceright}{\kern0pt}\ {\isasymand}\ finite{\isacharunderscore}{\kern0pt}profile\ A\ p{\isacharparenright}{\kern0pt}{\isachardoublequoteclose}\isanewline
\ \ \isacommand{from}\isamarkupfalse%
\ non{\isacharunderscore}{\kern0pt}blocking{\isacharunderscore}{\kern0pt}m\ \isacommand{have}\isamarkupfalse%
\isanewline
\ \ \ \ {\isachardoublequoteopen}{\isacharquery}{\kern0pt}input{\isacharunderscore}{\kern0pt}sound\ {\isasymlongrightarrow}\ reject\ m\ A\ p\ {\isasymnoteq}\ A{\isachardoublequoteclose}\isanewline
\ \ \ \ \isacommand{by}\isamarkupfalse%
\ {\isacharparenleft}{\kern0pt}simp\ add{\isacharcolon}{\kern0pt}\ non{\isacharunderscore}{\kern0pt}blocking{\isacharunderscore}{\kern0pt}def{\isacharparenright}{\kern0pt}\isanewline
\ \ \isacommand{with}\isamarkupfalse%
\ non{\isacharunderscore}{\kern0pt}blocking{\isacharunderscore}{\kern0pt}m\ \isacommand{have}\isamarkupfalse%
\ {\isadigit{0}}{\isacharcolon}{\kern0pt}\isanewline
\ \ \ \ {\isachardoublequoteopen}{\isacharquery}{\kern0pt}input{\isacharunderscore}{\kern0pt}sound\ {\isasymlongrightarrow}\ A\ {\isacharminus}{\kern0pt}\ reject\ m\ A\ p\ {\isasymnoteq}\ {\isacharbraceleft}{\kern0pt}{\isacharbraceright}{\kern0pt}{\isachardoublequoteclose}\isanewline
\ \ \ \ \isacommand{using}\isamarkupfalse%
\ Diff{\isacharunderscore}{\kern0pt}eq{\isacharunderscore}{\kern0pt}empty{\isacharunderscore}{\kern0pt}iff\ non{\isacharunderscore}{\kern0pt}blocking{\isacharunderscore}{\kern0pt}def\isanewline
\ \ \ \ \ \ \ \ \ \ reject{\isacharunderscore}{\kern0pt}in{\isacharunderscore}{\kern0pt}alts\ subset{\isacharunderscore}{\kern0pt}antisym\isanewline
\ \ \ \ \isacommand{by}\isamarkupfalse%
\ metis\isanewline
\ \ \isacommand{from}\isamarkupfalse%
\ non{\isacharunderscore}{\kern0pt}blocking{\isacharunderscore}{\kern0pt}m\ \isacommand{have}\isamarkupfalse%
\isanewline
\ \ \ \ {\isachardoublequoteopen}{\isacharquery}{\kern0pt}input{\isacharunderscore}{\kern0pt}sound\ {\isasymlongrightarrow}\ well{\isacharunderscore}{\kern0pt}formed\ A\ {\isacharparenleft}{\kern0pt}m\ A\ p{\isacharparenright}{\kern0pt}{\isachardoublequoteclose}\isanewline
\ \ \ \ \isacommand{by}\isamarkupfalse%
\ {\isacharparenleft}{\kern0pt}simp\ add{\isacharcolon}{\kern0pt}\ electoral{\isacharunderscore}{\kern0pt}module{\isacharunderscore}{\kern0pt}def\ non{\isacharunderscore}{\kern0pt}blocking{\isacharunderscore}{\kern0pt}def{\isacharparenright}{\kern0pt}\isanewline
\ \ \isacommand{hence}\isamarkupfalse%
\isanewline
\ \ \ \ {\isachardoublequoteopen}{\isacharquery}{\kern0pt}input{\isacharunderscore}{\kern0pt}sound\ {\isasymlongrightarrow}\isanewline
\ \ \ \ \ \ \ \ elect\ m\ A\ p\ {\isasymunion}\ defer\ m\ A\ p\ {\isacharequal}{\kern0pt}\ A\ {\isacharminus}{\kern0pt}\ reject\ m\ A\ p{\isachardoublequoteclose}\isanewline
\ \ \ \ \isacommand{using}\isamarkupfalse%
\ non{\isacharunderscore}{\kern0pt}blocking{\isacharunderscore}{\kern0pt}def\ non{\isacharunderscore}{\kern0pt}blocking{\isacharunderscore}{\kern0pt}m\ elec{\isacharunderscore}{\kern0pt}and{\isacharunderscore}{\kern0pt}def{\isacharunderscore}{\kern0pt}not{\isacharunderscore}{\kern0pt}rej\isanewline
\ \ \ \ \isacommand{by}\isamarkupfalse%
\ metis\isanewline
\ \ \isacommand{with}\isamarkupfalse%
\ {\isadigit{0}}\ \isacommand{have}\isamarkupfalse%
\isanewline
\ \ \ \ {\isachardoublequoteopen}{\isacharquery}{\kern0pt}input{\isacharunderscore}{\kern0pt}sound\ {\isasymlongrightarrow}\ elect\ m\ A\ p\ {\isasymunion}\ defer\ m\ A\ p\ {\isasymnoteq}\ {\isacharbraceleft}{\kern0pt}{\isacharbraceright}{\kern0pt}{\isachardoublequoteclose}\isanewline
\ \ \ \ \isacommand{by}\isamarkupfalse%
\ auto\isanewline
\ \ \isacommand{hence}\isamarkupfalse%
\ {\isachardoublequoteopen}{\isacharquery}{\kern0pt}input{\isacharunderscore}{\kern0pt}sound\ {\isasymlongrightarrow}\ {\isacharparenleft}{\kern0pt}elect\ m\ A\ p\ {\isasymnoteq}\ {\isacharbraceleft}{\kern0pt}{\isacharbraceright}{\kern0pt}\ {\isasymor}\ defer\ m\ A\ p\ {\isasymnoteq}\ {\isacharbraceleft}{\kern0pt}{\isacharbraceright}{\kern0pt}{\isacharparenright}{\kern0pt}{\isachardoublequoteclose}\isanewline
\ \ \ \ \isacommand{by}\isamarkupfalse%
\ simp\isanewline
\ \ \isacommand{with}\isamarkupfalse%
\ non{\isacharunderscore}{\kern0pt}blocking{\isacharunderscore}{\kern0pt}m\ non{\isacharunderscore}{\kern0pt}blocking{\isacharunderscore}{\kern0pt}n\isanewline
\ \ \isacommand{show}\isamarkupfalse%
\ {\isacharquery}{\kern0pt}thesis\isanewline
\ \ \ \ \isacommand{using}\isamarkupfalse%
\ Diff{\isacharunderscore}{\kern0pt}empty\ Diff{\isacharunderscore}{\kern0pt}subset{\isacharunderscore}{\kern0pt}conv\ Un{\isacharunderscore}{\kern0pt}empty\ fst{\isacharunderscore}{\kern0pt}conv\ snd{\isacharunderscore}{\kern0pt}conv\isanewline
\ \ \ \ \ \ \ \ \ \ defer{\isacharunderscore}{\kern0pt}not{\isacharunderscore}{\kern0pt}elec{\isacharunderscore}{\kern0pt}or{\isacharunderscore}{\kern0pt}rej\ elect{\isacharunderscore}{\kern0pt}in{\isacharunderscore}{\kern0pt}alts\ inf{\isachardot}{\kern0pt}absorb{\isadigit{1}}\ sup{\isacharunderscore}{\kern0pt}bot{\isacharunderscore}{\kern0pt}right\isanewline
\ \ \ \ \ \ \ \ \ \ non{\isacharunderscore}{\kern0pt}blocking{\isacharunderscore}{\kern0pt}def\ reject{\isacharunderscore}{\kern0pt}in{\isacharunderscore}{\kern0pt}alts\ sequential{\isacharunderscore}{\kern0pt}composition{\isachardot}{\kern0pt}simps\isanewline
\ \ \ \ \ \ \ \ \ \ seq{\isacharunderscore}{\kern0pt}comp{\isacharunderscore}{\kern0pt}sound\ def{\isacharunderscore}{\kern0pt}presv{\isacharunderscore}{\kern0pt}fin{\isacharunderscore}{\kern0pt}prof\ result{\isacharunderscore}{\kern0pt}disj\ subset{\isacharunderscore}{\kern0pt}antisym\isanewline
\ \ \ \ \isacommand{by}\isamarkupfalse%
\ {\isacharparenleft}{\kern0pt}smt\ {\isacharparenleft}{\kern0pt}verit{\isacharparenright}{\kern0pt}{\isacharparenright}{\kern0pt}\isanewline
\isacommand{qed}\isamarkupfalse%
%
\endisatagproof
{\isafoldproof}%
%
\isadelimproof
\isanewline
%
\endisadelimproof
\isanewline
\isacommand{theorem}\isamarkupfalse%
\ elector{\isacharunderscore}{\kern0pt}electing{\isacharbrackleft}{\kern0pt}simp{\isacharbrackright}{\kern0pt}{\isacharcolon}{\kern0pt}\isanewline
\ \ \isakeyword{assumes}\isanewline
\ \ \ \ module{\isacharunderscore}{\kern0pt}m{\isacharcolon}{\kern0pt}\ {\isachardoublequoteopen}electoral{\isacharunderscore}{\kern0pt}module\ m{\isachardoublequoteclose}\ \isakeyword{and}\isanewline
\ \ \ \ non{\isacharunderscore}{\kern0pt}block{\isacharunderscore}{\kern0pt}m{\isacharcolon}{\kern0pt}\ {\isachardoublequoteopen}non{\isacharunderscore}{\kern0pt}blocking\ m{\isachardoublequoteclose}\isanewline
\ \ \isakeyword{shows}\ {\isachardoublequoteopen}electing\ {\isacharparenleft}{\kern0pt}elector\ m{\isacharparenright}{\kern0pt}{\isachardoublequoteclose}\isanewline
%
\isadelimproof
%
\endisadelimproof
%
\isatagproof
\isacommand{proof}\isamarkupfalse%
\ {\isacharminus}{\kern0pt}\isanewline
\ \ \isacommand{obtain}\isamarkupfalse%
\isanewline
\ \ \ \ AA\ {\isacharcolon}{\kern0pt}{\isacharcolon}{\kern0pt}\ {\isachardoublequoteopen}{\isacharprime}{\kern0pt}a\ Electoral{\isacharunderscore}{\kern0pt}Module\ {\isasymRightarrow}\ {\isacharprime}{\kern0pt}a\ set{\isachardoublequoteclose}\ \isakeyword{and}\isanewline
\ \ \ \ rrs\ {\isacharcolon}{\kern0pt}{\isacharcolon}{\kern0pt}\ {\isachardoublequoteopen}{\isacharprime}{\kern0pt}a\ Electoral{\isacharunderscore}{\kern0pt}Module\ {\isasymRightarrow}\ {\isacharprime}{\kern0pt}a\ Profile{\isachardoublequoteclose}\ \isakeyword{where}\isanewline
\ \ \ \ f{\isadigit{1}}{\isacharcolon}{\kern0pt}\isanewline
\ \ \ \ {\isachardoublequoteopen}{\isasymforall}f{\isachardot}{\kern0pt}\isanewline
\ \ \ \ \ \ {\isacharparenleft}{\kern0pt}electing\ f\ {\isasymor}\isanewline
\ \ \ \ \ \ \ \ {\isacharbraceleft}{\kern0pt}{\isacharbraceright}{\kern0pt}\ {\isacharequal}{\kern0pt}\ elect\ f\ {\isacharparenleft}{\kern0pt}AA\ f{\isacharparenright}{\kern0pt}\ {\isacharparenleft}{\kern0pt}rrs\ f{\isacharparenright}{\kern0pt}\ {\isasymand}\ profile\ {\isacharparenleft}{\kern0pt}AA\ f{\isacharparenright}{\kern0pt}\ {\isacharparenleft}{\kern0pt}rrs\ f{\isacharparenright}{\kern0pt}\ {\isasymand}\isanewline
\ \ \ \ \ \ \ \ \ \ \ \ finite\ {\isacharparenleft}{\kern0pt}AA\ f{\isacharparenright}{\kern0pt}\ {\isasymand}\ {\isacharbraceleft}{\kern0pt}{\isacharbraceright}{\kern0pt}\ {\isasymnoteq}\ AA\ f\ {\isasymor}\isanewline
\ \ \ \ \ \ \ \ {\isasymnot}\ electoral{\isacharunderscore}{\kern0pt}module\ f{\isacharparenright}{\kern0pt}\ {\isasymand}\isanewline
\ \ \ \ \ \ \ \ \ \ \ \ {\isacharparenleft}{\kern0pt}{\isacharparenleft}{\kern0pt}{\isasymforall}A\ rs{\isachardot}{\kern0pt}\ {\isacharbraceleft}{\kern0pt}{\isacharbraceright}{\kern0pt}\ {\isasymnoteq}\ elect\ f\ A\ rs\ {\isasymor}\ {\isasymnot}\ profile\ A\ rs\ {\isasymor}\isanewline
\ \ \ \ \ \ \ \ \ \ \ \ \ \ \ \ infinite\ A\ {\isasymor}\ {\isacharbraceleft}{\kern0pt}{\isacharbraceright}{\kern0pt}\ {\isacharequal}{\kern0pt}\ A{\isacharparenright}{\kern0pt}\ {\isasymand}\isanewline
\ \ \ \ \ \ \ \ \ \ \ \ electoral{\isacharunderscore}{\kern0pt}module\ f\ {\isasymor}\isanewline
\ \ \ \ \ \ \ \ {\isasymnot}\ electing\ f{\isacharparenright}{\kern0pt}{\isachardoublequoteclose}\isanewline
\ \ \ \ \isacommand{using}\isamarkupfalse%
\ electing{\isacharunderscore}{\kern0pt}def\isanewline
\ \ \ \ \isacommand{by}\isamarkupfalse%
\ metis\isanewline
\ \ \isacommand{have}\isamarkupfalse%
\ non{\isacharunderscore}{\kern0pt}block{\isacharcolon}{\kern0pt}\isanewline
\ \ \ \ {\isachardoublequoteopen}non{\isacharunderscore}{\kern0pt}blocking\isanewline
\ \ \ \ \ \ {\isacharparenleft}{\kern0pt}elect{\isacharunderscore}{\kern0pt}module{\isacharcolon}{\kern0pt}{\isacharcolon}{\kern0pt}{\isacharprime}{\kern0pt}a\ set\ {\isasymRightarrow}\ {\isacharunderscore}{\kern0pt}\ Profile\ {\isasymRightarrow}\ {\isacharunderscore}{\kern0pt}\ Result{\isacharparenright}{\kern0pt}{\isachardoublequoteclose}\isanewline
\ \ \ \ \isacommand{by}\isamarkupfalse%
\ {\isacharparenleft}{\kern0pt}simp\ add{\isacharcolon}{\kern0pt}\ electing{\isacharunderscore}{\kern0pt}imp{\isacharunderscore}{\kern0pt}non{\isacharunderscore}{\kern0pt}blocking{\isacharparenright}{\kern0pt}\isanewline
\ \ \isacommand{thus}\isamarkupfalse%
\ {\isacharquery}{\kern0pt}thesis\isanewline
\ \ \ \ \isanewline
\ \ \isacommand{proof}\isamarkupfalse%
\ {\isacharminus}{\kern0pt}\isanewline
\ \ \ \ \isacommand{obtain}\isamarkupfalse%
\isanewline
\ \ \ \ \ \ AAa\ {\isacharcolon}{\kern0pt}{\isacharcolon}{\kern0pt}\ {\isachardoublequoteopen}{\isacharprime}{\kern0pt}a\ Electoral{\isacharunderscore}{\kern0pt}Module\ {\isasymRightarrow}\ {\isacharprime}{\kern0pt}a\ set{\isachardoublequoteclose}\ \isakeyword{and}\isanewline
\ \ \ \ \ \ rrsa\ {\isacharcolon}{\kern0pt}{\isacharcolon}{\kern0pt}\ {\isachardoublequoteopen}{\isacharprime}{\kern0pt}a\ Electoral{\isacharunderscore}{\kern0pt}Module\ {\isasymRightarrow}\ {\isacharprime}{\kern0pt}a\ Profile{\isachardoublequoteclose}\ \isakeyword{where}\isanewline
\ \ \ \ \ \ f{\isadigit{1}}{\isacharcolon}{\kern0pt}\isanewline
\ \ \ \ \ \ {\isachardoublequoteopen}{\isasymforall}f{\isachardot}{\kern0pt}\isanewline
\ \ \ \ \ \ \ \ {\isacharparenleft}{\kern0pt}electing\ f\ {\isasymor}\isanewline
\ \ \ \ \ \ \ \ \ \ {\isacharbraceleft}{\kern0pt}{\isacharbraceright}{\kern0pt}\ {\isacharequal}{\kern0pt}\ elect\ f\ {\isacharparenleft}{\kern0pt}AAa\ f{\isacharparenright}{\kern0pt}\ {\isacharparenleft}{\kern0pt}rrsa\ f{\isacharparenright}{\kern0pt}\ {\isasymand}\ profile\ {\isacharparenleft}{\kern0pt}AAa\ f{\isacharparenright}{\kern0pt}\ {\isacharparenleft}{\kern0pt}rrsa\ f{\isacharparenright}{\kern0pt}\ {\isasymand}\isanewline
\ \ \ \ \ \ \ \ \ \ \ \ \ \ finite\ {\isacharparenleft}{\kern0pt}AAa\ f{\isacharparenright}{\kern0pt}\ {\isasymand}\ {\isacharbraceleft}{\kern0pt}{\isacharbraceright}{\kern0pt}\ {\isasymnoteq}\ AAa\ f\ {\isasymor}\isanewline
\ \ \ \ \ \ \ \ {\isasymnot}\ electoral{\isacharunderscore}{\kern0pt}module\ f{\isacharparenright}{\kern0pt}\ {\isasymand}\ {\isacharparenleft}{\kern0pt}{\isacharparenleft}{\kern0pt}{\isasymforall}A\ rs{\isachardot}{\kern0pt}\ {\isacharbraceleft}{\kern0pt}{\isacharbraceright}{\kern0pt}\ {\isasymnoteq}\ elect\ f\ A\ rs\ {\isasymor}\isanewline
\ \ \ \ \ \ \ \ {\isasymnot}\ profile\ A\ rs\ {\isasymor}\ infinite\ A\ {\isasymor}\ {\isacharbraceleft}{\kern0pt}{\isacharbraceright}{\kern0pt}\ {\isacharequal}{\kern0pt}\ A{\isacharparenright}{\kern0pt}\ {\isasymand}\ electoral{\isacharunderscore}{\kern0pt}module\ f\ {\isasymor}\isanewline
\ \ \ \ \ \ \ \ {\isasymnot}\ electing\ f{\isacharparenright}{\kern0pt}{\isachardoublequoteclose}\isanewline
\ \ \ \ \ \ \isacommand{using}\isamarkupfalse%
\ electing{\isacharunderscore}{\kern0pt}def\isanewline
\ \ \ \ \ \ \isacommand{by}\isamarkupfalse%
\ metis\isanewline
\ \ \ \ \isacommand{obtain}\isamarkupfalse%
\isanewline
\ \ \ \ \ \ AAb\ {\isacharcolon}{\kern0pt}{\isacharcolon}{\kern0pt}\ {\isachardoublequoteopen}{\isacharprime}{\kern0pt}a\ Result\ {\isasymRightarrow}\ {\isacharprime}{\kern0pt}a\ set{\isachardoublequoteclose}\ \isakeyword{and}\isanewline
\ \ \ \ \ \ AAc\ {\isacharcolon}{\kern0pt}{\isacharcolon}{\kern0pt}\ {\isachardoublequoteopen}{\isacharprime}{\kern0pt}a\ Result\ {\isasymRightarrow}\ {\isacharprime}{\kern0pt}a\ set{\isachardoublequoteclose}\ \isakeyword{and}\isanewline
\ \ \ \ \ \ AAd\ {\isacharcolon}{\kern0pt}{\isacharcolon}{\kern0pt}\ {\isachardoublequoteopen}{\isacharprime}{\kern0pt}a\ Result\ {\isasymRightarrow}\ {\isacharprime}{\kern0pt}a\ set{\isachardoublequoteclose}\ \isakeyword{where}\isanewline
\ \ \ \ \ \ f{\isadigit{2}}{\isacharcolon}{\kern0pt}\isanewline
\ \ \ \ \ \ {\isachardoublequoteopen}{\isasymforall}p{\isachardot}{\kern0pt}\ {\isacharparenleft}{\kern0pt}AAb\ p{\isacharcomma}{\kern0pt}\ AAc\ p{\isacharcomma}{\kern0pt}\ AAd\ p{\isacharparenright}{\kern0pt}\ {\isacharequal}{\kern0pt}\ p{\isachardoublequoteclose}\isanewline
\ \ \ \ \ \ \isacommand{using}\isamarkupfalse%
\ disjoint{\isadigit{3}}{\isachardot}{\kern0pt}cases\isanewline
\ \ \ \ \ \ \isacommand{by}\isamarkupfalse%
\ {\isacharparenleft}{\kern0pt}metis\ {\isacharparenleft}{\kern0pt}no{\isacharunderscore}{\kern0pt}types{\isacharparenright}{\kern0pt}{\isacharparenright}{\kern0pt}\isanewline
\ \ \ \ \isacommand{have}\isamarkupfalse%
\ f{\isadigit{3}}{\isacharcolon}{\kern0pt}\isanewline
\ \ \ \ \ \ {\isachardoublequoteopen}electoral{\isacharunderscore}{\kern0pt}module\ {\isacharparenleft}{\kern0pt}elector\ m{\isacharparenright}{\kern0pt}{\isachardoublequoteclose}\isanewline
\ \ \ \ \ \ \isacommand{using}\isamarkupfalse%
\ elector{\isacharunderscore}{\kern0pt}sound\ module{\isacharunderscore}{\kern0pt}m\isanewline
\ \ \ \ \ \ \isacommand{by}\isamarkupfalse%
\ simp\isanewline
\ \ \ \ \isacommand{have}\isamarkupfalse%
\ f{\isadigit{4}}{\isacharcolon}{\kern0pt}\isanewline
\ \ \ \ \ \ {\isachardoublequoteopen}{\isasymforall}p{\isachardot}{\kern0pt}\ {\isacharparenleft}{\kern0pt}elect{\isacharunderscore}{\kern0pt}r\ p{\isacharcomma}{\kern0pt}\ AAc\ p{\isacharcomma}{\kern0pt}\ AAd\ p{\isacharparenright}{\kern0pt}\ {\isacharequal}{\kern0pt}\ p{\isachardoublequoteclose}\isanewline
\ \ \ \ \ \ \isacommand{using}\isamarkupfalse%
\ f{\isadigit{2}}\isanewline
\ \ \ \ \ \ \isacommand{by}\isamarkupfalse%
\ simp\isanewline
\ \ \ \ \isacommand{have}\isamarkupfalse%
\isanewline
\ \ \ \ \ \ {\isachardoublequoteopen}finite\ {\isacharparenleft}{\kern0pt}AAa\ {\isacharparenleft}{\kern0pt}elector\ m{\isacharparenright}{\kern0pt}{\isacharparenright}{\kern0pt}\ {\isasymand}\isanewline
\ \ \ \ \ \ \ \ profile\ {\isacharparenleft}{\kern0pt}AAa\ {\isacharparenleft}{\kern0pt}elector\ m{\isacharparenright}{\kern0pt}{\isacharparenright}{\kern0pt}\ {\isacharparenleft}{\kern0pt}rrsa\ {\isacharparenleft}{\kern0pt}elector\ m{\isacharparenright}{\kern0pt}{\isacharparenright}{\kern0pt}\ {\isasymand}\isanewline
\ \ \ \ \ \ \ \ {\isacharbraceleft}{\kern0pt}{\isacharbraceright}{\kern0pt}\ {\isacharequal}{\kern0pt}\ elect\ {\isacharparenleft}{\kern0pt}elector\ m{\isacharparenright}{\kern0pt}\ {\isacharparenleft}{\kern0pt}AAa\ {\isacharparenleft}{\kern0pt}elector\ m{\isacharparenright}{\kern0pt}{\isacharparenright}{\kern0pt}\ {\isacharparenleft}{\kern0pt}rrsa\ {\isacharparenleft}{\kern0pt}elector\ m{\isacharparenright}{\kern0pt}{\isacharparenright}{\kern0pt}\ {\isasymand}\isanewline
\ \ \ \ \ \ \ \ {\isacharbraceleft}{\kern0pt}{\isacharbraceright}{\kern0pt}\ {\isacharequal}{\kern0pt}\ AAd\ {\isacharparenleft}{\kern0pt}elector\ m\ {\isacharparenleft}{\kern0pt}AAa\ {\isacharparenleft}{\kern0pt}elector\ m{\isacharparenright}{\kern0pt}{\isacharparenright}{\kern0pt}\ {\isacharparenleft}{\kern0pt}rrsa\ {\isacharparenleft}{\kern0pt}elector\ m{\isacharparenright}{\kern0pt}{\isacharparenright}{\kern0pt}{\isacharparenright}{\kern0pt}\ {\isasymand}\isanewline
\ \ \ \ \ \ \ \ reject\ {\isacharparenleft}{\kern0pt}elector\ m{\isacharparenright}{\kern0pt}\ {\isacharparenleft}{\kern0pt}AAa\ {\isacharparenleft}{\kern0pt}elector\ m{\isacharparenright}{\kern0pt}{\isacharparenright}{\kern0pt}\ {\isacharparenleft}{\kern0pt}rrsa\ {\isacharparenleft}{\kern0pt}elector\ m{\isacharparenright}{\kern0pt}{\isacharparenright}{\kern0pt}\ {\isacharequal}{\kern0pt}\isanewline
\ \ \ \ \ \ \ \ \ \ AAc\ {\isacharparenleft}{\kern0pt}elector\ m\ {\isacharparenleft}{\kern0pt}AAa\ {\isacharparenleft}{\kern0pt}elector\ m{\isacharparenright}{\kern0pt}{\isacharparenright}{\kern0pt}\ {\isacharparenleft}{\kern0pt}rrsa\ {\isacharparenleft}{\kern0pt}elector\ m{\isacharparenright}{\kern0pt}{\isacharparenright}{\kern0pt}{\isacharparenright}{\kern0pt}\ {\isasymlongrightarrow}\isanewline
\ \ \ \ \ \ \ \ \ \ \ \ \ \ electing\ {\isacharparenleft}{\kern0pt}elector\ m{\isacharparenright}{\kern0pt}{\isachardoublequoteclose}\isanewline
\ \ \ \ \ \ \isacommand{using}\isamarkupfalse%
\ f{\isadigit{2}}\ f{\isadigit{1}}\ Diff{\isacharunderscore}{\kern0pt}empty\ elector{\isachardot}{\kern0pt}simps\ non{\isacharunderscore}{\kern0pt}block{\isacharunderscore}{\kern0pt}m\ snd{\isacharunderscore}{\kern0pt}conv\isanewline
\ \ \ \ \ \ \ \ \ \ \ \ non{\isacharunderscore}{\kern0pt}blocking{\isacharunderscore}{\kern0pt}def\ reject{\isacharunderscore}{\kern0pt}not{\isacharunderscore}{\kern0pt}elec{\isacharunderscore}{\kern0pt}or{\isacharunderscore}{\kern0pt}def\ non{\isacharunderscore}{\kern0pt}block\isanewline
\ \ \ \ \ \ \ \ \ \ \ \ seq{\isacharunderscore}{\kern0pt}comp{\isacharunderscore}{\kern0pt}presv{\isacharunderscore}{\kern0pt}non{\isacharunderscore}{\kern0pt}blocking\isanewline
\ \ \ \ \ \ \isacommand{by}\isamarkupfalse%
\ metis\isanewline
\ \ \ \ \isacommand{moreover}\isamarkupfalse%
\isanewline
\ \ \ \ \isacommand{{\isacharbraceleft}{\kern0pt}}\isamarkupfalse%
\isanewline
\ \ \ \ \ \ \isacommand{assume}\isamarkupfalse%
\isanewline
\ \ \ \ \ \ \ \ {\isachardoublequoteopen}{\isacharbraceleft}{\kern0pt}{\isacharbraceright}{\kern0pt}\ {\isasymnoteq}\ AAd\ {\isacharparenleft}{\kern0pt}elector\ m\ {\isacharparenleft}{\kern0pt}AAa\ {\isacharparenleft}{\kern0pt}elector\ m{\isacharparenright}{\kern0pt}{\isacharparenright}{\kern0pt}\ {\isacharparenleft}{\kern0pt}rrsa\ {\isacharparenleft}{\kern0pt}elector\ m{\isacharparenright}{\kern0pt}{\isacharparenright}{\kern0pt}{\isacharparenright}{\kern0pt}{\isachardoublequoteclose}\isanewline
\ \ \ \ \ \ \isacommand{hence}\isamarkupfalse%
\isanewline
\ \ \ \ \ \ \ \ {\isachardoublequoteopen}{\isasymnot}\ profile\ {\isacharparenleft}{\kern0pt}AAa\ {\isacharparenleft}{\kern0pt}elector\ m{\isacharparenright}{\kern0pt}{\isacharparenright}{\kern0pt}\ {\isacharparenleft}{\kern0pt}rrsa\ {\isacharparenleft}{\kern0pt}elector\ m{\isacharparenright}{\kern0pt}{\isacharparenright}{\kern0pt}\ {\isasymor}\isanewline
\ \ \ \ \ \ \ \ \ \ infinite\ {\isacharparenleft}{\kern0pt}AAa\ {\isacharparenleft}{\kern0pt}elector\ m{\isacharparenright}{\kern0pt}{\isacharparenright}{\kern0pt}{\isachardoublequoteclose}\isanewline
\ \ \ \ \ \ \ \ \isacommand{using}\isamarkupfalse%
\ f{\isadigit{4}}\isanewline
\ \ \ \ \ \ \ \ \isacommand{by}\isamarkupfalse%
\ simp\isanewline
\ \ \ \ \isacommand{{\isacharbraceright}{\kern0pt}}\isamarkupfalse%
\isanewline
\ \ \ \ \isacommand{ultimately}\isamarkupfalse%
\ \isacommand{show}\isamarkupfalse%
\ {\isacharquery}{\kern0pt}thesis\isanewline
\ \ \ \ \ \ \isacommand{using}\isamarkupfalse%
\ f{\isadigit{4}}\ f{\isadigit{3}}\ f{\isadigit{1}}\ fst{\isacharunderscore}{\kern0pt}conv\ snd{\isacharunderscore}{\kern0pt}conv\isanewline
\ \ \ \ \ \ \isacommand{by}\isamarkupfalse%
\ metis\isanewline
\ \ \isacommand{qed}\isamarkupfalse%
\isanewline
\isacommand{qed}\isamarkupfalse%
%
\endisatagproof
{\isafoldproof}%
%
\isadelimproof
\isanewline
%
\endisadelimproof
\isanewline
\isanewline
\isacommand{theorem}\isamarkupfalse%
\ seq{\isacharunderscore}{\kern0pt}comp{\isacharunderscore}{\kern0pt}electing{\isacharbrackleft}{\kern0pt}simp{\isacharbrackright}{\kern0pt}{\isacharcolon}{\kern0pt}\isanewline
\ \ \isakeyword{assumes}\ def{\isacharunderscore}{\kern0pt}one{\isacharunderscore}{\kern0pt}m{\isadigit{1}}{\isacharcolon}{\kern0pt}\ \ {\isachardoublequoteopen}defers\ {\isadigit{1}}\ m{\isadigit{1}}{\isachardoublequoteclose}\ \isakeyword{and}\isanewline
\ \ \ \ \ \ \ \ \ \ electing{\isacharunderscore}{\kern0pt}m{\isadigit{2}}{\isacharcolon}{\kern0pt}\ {\isachardoublequoteopen}electing\ m{\isadigit{2}}{\isachardoublequoteclose}\isanewline
\ \ \isakeyword{shows}\ {\isachardoublequoteopen}electing\ {\isacharparenleft}{\kern0pt}m{\isadigit{1}}\ {\isasymtriangleright}\ m{\isadigit{2}}{\isacharparenright}{\kern0pt}{\isachardoublequoteclose}\isanewline
%
\isadelimproof
%
\endisadelimproof
%
\isatagproof
\isacommand{proof}\isamarkupfalse%
\ {\isacharminus}{\kern0pt}\isanewline
\ \ \isacommand{have}\isamarkupfalse%
\isanewline
\ \ \ \ {\isachardoublequoteopen}{\isasymforall}A\ p{\isachardot}{\kern0pt}\ {\isacharparenleft}{\kern0pt}card\ A\ {\isasymge}\ {\isadigit{1}}\ {\isasymand}\ finite{\isacharunderscore}{\kern0pt}profile\ A\ p{\isacharparenright}{\kern0pt}\ {\isasymlongrightarrow}\isanewline
\ \ \ \ \ \ \ \ card\ {\isacharparenleft}{\kern0pt}defer\ m{\isadigit{1}}\ A\ p{\isacharparenright}{\kern0pt}\ {\isacharequal}{\kern0pt}\ {\isadigit{1}}{\isachardoublequoteclose}\isanewline
\ \ \ \ \isacommand{using}\isamarkupfalse%
\ def{\isacharunderscore}{\kern0pt}one{\isacharunderscore}{\kern0pt}m{\isadigit{1}}\ defers{\isacharunderscore}{\kern0pt}def\isanewline
\ \ \ \ \isacommand{by}\isamarkupfalse%
\ blast\isanewline
\ \ \isacommand{hence}\isamarkupfalse%
\isanewline
\ \ \ \ {\isachardoublequoteopen}{\isasymforall}A\ p{\isachardot}{\kern0pt}\ {\isacharparenleft}{\kern0pt}A\ {\isasymnoteq}\ {\isacharbraceleft}{\kern0pt}{\isacharbraceright}{\kern0pt}\ {\isasymand}\ finite{\isacharunderscore}{\kern0pt}profile\ A\ p{\isacharparenright}{\kern0pt}\ {\isasymlongrightarrow}\isanewline
\ \ \ \ \ \ \ \ defer\ m{\isadigit{1}}\ A\ p\ {\isasymnoteq}\ {\isacharbraceleft}{\kern0pt}{\isacharbraceright}{\kern0pt}{\isachardoublequoteclose}\isanewline
\ \ \ \ \isacommand{using}\isamarkupfalse%
\ One{\isacharunderscore}{\kern0pt}nat{\isacharunderscore}{\kern0pt}def\ Suc{\isacharunderscore}{\kern0pt}leI\ card{\isacharunderscore}{\kern0pt}eq{\isacharunderscore}{\kern0pt}{\isadigit{0}}{\isacharunderscore}{\kern0pt}iff\isanewline
\ \ \ \ \ \ \ \ \ \ card{\isacharunderscore}{\kern0pt}gt{\isacharunderscore}{\kern0pt}{\isadigit{0}}{\isacharunderscore}{\kern0pt}iff\ zero{\isacharunderscore}{\kern0pt}neq{\isacharunderscore}{\kern0pt}one\isanewline
\ \ \ \ \isacommand{by}\isamarkupfalse%
\ metis\isanewline
\ \ \isacommand{thus}\isamarkupfalse%
\ {\isacharquery}{\kern0pt}thesis\isanewline
\ \ \ \ \isacommand{using}\isamarkupfalse%
\ Un{\isacharunderscore}{\kern0pt}empty\ def{\isacharunderscore}{\kern0pt}one{\isacharunderscore}{\kern0pt}m{\isadigit{1}}\ defers{\isacharunderscore}{\kern0pt}def\ electing{\isacharunderscore}{\kern0pt}def\isanewline
\ \ \ \ \ \ \ \ \ \ electing{\isacharunderscore}{\kern0pt}m{\isadigit{2}}\ seq{\isacharunderscore}{\kern0pt}comp{\isacharunderscore}{\kern0pt}def{\isacharunderscore}{\kern0pt}then{\isacharunderscore}{\kern0pt}elect{\isacharunderscore}{\kern0pt}elec{\isacharunderscore}{\kern0pt}set\isanewline
\ \ \ \ \ \ \ \ \ \ seq{\isacharunderscore}{\kern0pt}comp{\isacharunderscore}{\kern0pt}sound\ def{\isacharunderscore}{\kern0pt}presv{\isacharunderscore}{\kern0pt}fin{\isacharunderscore}{\kern0pt}prof\isanewline
\ \ \ \ \isacommand{by}\isamarkupfalse%
\ {\isacharparenleft}{\kern0pt}smt\ {\isacharparenleft}{\kern0pt}verit{\isacharcomma}{\kern0pt}\ ccfv{\isacharunderscore}{\kern0pt}threshold{\isacharparenright}{\kern0pt}{\isacharparenright}{\kern0pt}\isanewline
\isacommand{qed}\isamarkupfalse%
%
\endisatagproof
{\isafoldproof}%
%
\isadelimproof
\isanewline
%
\endisadelimproof
\isanewline
\isanewline
\isacommand{theorem}\isamarkupfalse%
\ conserv{\isacharunderscore}{\kern0pt}agg{\isacharunderscore}{\kern0pt}presv{\isacharunderscore}{\kern0pt}non{\isacharunderscore}{\kern0pt}electing{\isacharbrackleft}{\kern0pt}simp{\isacharbrackright}{\kern0pt}{\isacharcolon}{\kern0pt}\isanewline
\ \ \isakeyword{assumes}\isanewline
\ \ \ \ non{\isacharunderscore}{\kern0pt}electing{\isacharunderscore}{\kern0pt}m{\isacharcolon}{\kern0pt}\ {\isachardoublequoteopen}non{\isacharunderscore}{\kern0pt}electing\ m{\isachardoublequoteclose}\ \isakeyword{and}\isanewline
\ \ \ \ non{\isacharunderscore}{\kern0pt}electing{\isacharunderscore}{\kern0pt}n{\isacharcolon}{\kern0pt}\ {\isachardoublequoteopen}non{\isacharunderscore}{\kern0pt}electing\ n{\isachardoublequoteclose}\ \isakeyword{and}\isanewline
\ \ \ \ conservative{\isacharcolon}{\kern0pt}\ {\isachardoublequoteopen}agg{\isacharunderscore}{\kern0pt}conservative\ a{\isachardoublequoteclose}\isanewline
\ \ \isakeyword{shows}\ {\isachardoublequoteopen}non{\isacharunderscore}{\kern0pt}electing\ {\isacharparenleft}{\kern0pt}m\ {\isasymparallel}\isactrlsub a\ n{\isacharparenright}{\kern0pt}{\isachardoublequoteclose}\isanewline
%
\isadelimproof
\ \ %
\endisadelimproof
%
\isatagproof
\isacommand{unfolding}\isamarkupfalse%
\ non{\isacharunderscore}{\kern0pt}electing{\isacharunderscore}{\kern0pt}def\isanewline
\isacommand{proof}\isamarkupfalse%
\ {\isacharparenleft}{\kern0pt}safe{\isacharparenright}{\kern0pt}\isanewline
\ \ \isacommand{have}\isamarkupfalse%
\ emod{\isacharunderscore}{\kern0pt}m{\isacharcolon}{\kern0pt}\ {\isachardoublequoteopen}electoral{\isacharunderscore}{\kern0pt}module\ m{\isachardoublequoteclose}\isanewline
\ \ \ \ \isacommand{using}\isamarkupfalse%
\ non{\isacharunderscore}{\kern0pt}electing{\isacharunderscore}{\kern0pt}m\isanewline
\ \ \ \ \isacommand{by}\isamarkupfalse%
\ {\isacharparenleft}{\kern0pt}simp\ add{\isacharcolon}{\kern0pt}\ non{\isacharunderscore}{\kern0pt}electing{\isacharunderscore}{\kern0pt}def{\isacharparenright}{\kern0pt}\isanewline
\ \ \isacommand{have}\isamarkupfalse%
\ emod{\isacharunderscore}{\kern0pt}n{\isacharcolon}{\kern0pt}\ {\isachardoublequoteopen}electoral{\isacharunderscore}{\kern0pt}module\ n{\isachardoublequoteclose}\isanewline
\ \ \ \ \isacommand{using}\isamarkupfalse%
\ non{\isacharunderscore}{\kern0pt}electing{\isacharunderscore}{\kern0pt}n\isanewline
\ \ \ \ \isacommand{by}\isamarkupfalse%
\ {\isacharparenleft}{\kern0pt}simp\ add{\isacharcolon}{\kern0pt}\ non{\isacharunderscore}{\kern0pt}electing{\isacharunderscore}{\kern0pt}def{\isacharparenright}{\kern0pt}\isanewline
\ \ \isacommand{have}\isamarkupfalse%
\ agg{\isacharunderscore}{\kern0pt}a{\isacharcolon}{\kern0pt}\ {\isachardoublequoteopen}aggregator\ a{\isachardoublequoteclose}\isanewline
\ \ \ \ \isacommand{using}\isamarkupfalse%
\ conservative\isanewline
\ \ \ \ \isacommand{by}\isamarkupfalse%
\ {\isacharparenleft}{\kern0pt}simp\ add{\isacharcolon}{\kern0pt}\ agg{\isacharunderscore}{\kern0pt}conservative{\isacharunderscore}{\kern0pt}def{\isacharparenright}{\kern0pt}\isanewline
\ \ \isacommand{thus}\isamarkupfalse%
\ {\isachardoublequoteopen}electoral{\isacharunderscore}{\kern0pt}module\ {\isacharparenleft}{\kern0pt}m\ {\isasymparallel}\isactrlsub a\ n{\isacharparenright}{\kern0pt}{\isachardoublequoteclose}\isanewline
\ \ \ \ \isacommand{using}\isamarkupfalse%
\ emod{\isacharunderscore}{\kern0pt}m\ emod{\isacharunderscore}{\kern0pt}n\ agg{\isacharunderscore}{\kern0pt}a\ par{\isacharunderscore}{\kern0pt}comp{\isacharunderscore}{\kern0pt}sound\isanewline
\ \ \ \ \isacommand{by}\isamarkupfalse%
\ simp\isanewline
\isacommand{next}\isamarkupfalse%
\isanewline
\ \ \isacommand{fix}\isamarkupfalse%
\isanewline
\ \ \ \ A\ {\isacharcolon}{\kern0pt}{\isacharcolon}{\kern0pt}\ {\isachardoublequoteopen}{\isacharprime}{\kern0pt}a\ set{\isachardoublequoteclose}\ \isakeyword{and}\isanewline
\ \ \ \ p\ {\isacharcolon}{\kern0pt}{\isacharcolon}{\kern0pt}\ {\isachardoublequoteopen}{\isacharprime}{\kern0pt}a\ Profile{\isachardoublequoteclose}\ \isakeyword{and}\isanewline
\ \ \ \ x\ {\isacharcolon}{\kern0pt}{\isacharcolon}{\kern0pt}\ {\isachardoublequoteopen}{\isacharprime}{\kern0pt}a{\isachardoublequoteclose}\isanewline
\ \ \isacommand{assume}\isamarkupfalse%
\isanewline
\ \ \ \ fin{\isacharunderscore}{\kern0pt}A{\isacharcolon}{\kern0pt}\ {\isachardoublequoteopen}finite\ A{\isachardoublequoteclose}\ \isakeyword{and}\isanewline
\ \ \ \ prof{\isacharunderscore}{\kern0pt}A{\isacharcolon}{\kern0pt}\ {\isachardoublequoteopen}profile\ A\ p{\isachardoublequoteclose}\ \isakeyword{and}\isanewline
\ \ \ \ x{\isacharunderscore}{\kern0pt}wins{\isacharcolon}{\kern0pt}\ {\isachardoublequoteopen}x\ {\isasymin}\ elect\ {\isacharparenleft}{\kern0pt}m\ {\isasymparallel}\isactrlsub a\ n{\isacharparenright}{\kern0pt}\ A\ p{\isachardoublequoteclose}\isanewline
\ \ \isacommand{have}\isamarkupfalse%
\ emod{\isacharunderscore}{\kern0pt}m{\isacharcolon}{\kern0pt}\ {\isachardoublequoteopen}electoral{\isacharunderscore}{\kern0pt}module\ m{\isachardoublequoteclose}\isanewline
\ \ \ \ \isacommand{using}\isamarkupfalse%
\ non{\isacharunderscore}{\kern0pt}electing{\isacharunderscore}{\kern0pt}m\isanewline
\ \ \ \ \isacommand{by}\isamarkupfalse%
\ {\isacharparenleft}{\kern0pt}simp\ add{\isacharcolon}{\kern0pt}\ non{\isacharunderscore}{\kern0pt}electing{\isacharunderscore}{\kern0pt}def{\isacharparenright}{\kern0pt}\isanewline
\ \ \isacommand{have}\isamarkupfalse%
\ emod{\isacharunderscore}{\kern0pt}n{\isacharcolon}{\kern0pt}\ {\isachardoublequoteopen}electoral{\isacharunderscore}{\kern0pt}module\ n{\isachardoublequoteclose}\isanewline
\ \ \ \ \isacommand{using}\isamarkupfalse%
\ non{\isacharunderscore}{\kern0pt}electing{\isacharunderscore}{\kern0pt}n\isanewline
\ \ \ \ \isacommand{by}\isamarkupfalse%
\ {\isacharparenleft}{\kern0pt}simp\ add{\isacharcolon}{\kern0pt}\ non{\isacharunderscore}{\kern0pt}electing{\isacharunderscore}{\kern0pt}def{\isacharparenright}{\kern0pt}\isanewline
\ \ \isacommand{have}\isamarkupfalse%
\isanewline
\ \ \ \ {\isachardoublequoteopen}let\ c\ {\isacharequal}{\kern0pt}\ {\isacharparenleft}{\kern0pt}a\ A\ {\isacharparenleft}{\kern0pt}m\ A\ p{\isacharparenright}{\kern0pt}\ {\isacharparenleft}{\kern0pt}n\ A\ p{\isacharparenright}{\kern0pt}{\isacharparenright}{\kern0pt}\ in\isanewline
\ \ \ \ \ \ {\isacharparenleft}{\kern0pt}elect{\isacharunderscore}{\kern0pt}r\ c\ {\isasymsubseteq}\ {\isacharparenleft}{\kern0pt}{\isacharparenleft}{\kern0pt}elect\ m\ A\ p{\isacharparenright}{\kern0pt}\ {\isasymunion}\ {\isacharparenleft}{\kern0pt}elect\ n\ A\ p{\isacharparenright}{\kern0pt}{\isacharparenright}{\kern0pt}{\isacharparenright}{\kern0pt}{\isachardoublequoteclose}\isanewline
\ \ \ \ \isacommand{using}\isamarkupfalse%
\ conservative\ agg{\isacharunderscore}{\kern0pt}conservative{\isacharunderscore}{\kern0pt}def\isanewline
\ \ \ \ \ \ \ \ \ \ emod{\isacharunderscore}{\kern0pt}m\ emod{\isacharunderscore}{\kern0pt}n\ par{\isacharunderscore}{\kern0pt}comp{\isacharunderscore}{\kern0pt}result{\isacharunderscore}{\kern0pt}sound\isanewline
\ \ \ \ \ \ \ \ \ \ combine{\isacharunderscore}{\kern0pt}ele{\isacharunderscore}{\kern0pt}rej{\isacharunderscore}{\kern0pt}def\ fin{\isacharunderscore}{\kern0pt}A\ prof{\isacharunderscore}{\kern0pt}A\isanewline
\ \ \ \ \isacommand{by}\isamarkupfalse%
\ {\isacharparenleft}{\kern0pt}smt\ {\isacharparenleft}{\kern0pt}verit{\isacharcomma}{\kern0pt}\ ccfv{\isacharunderscore}{\kern0pt}SIG{\isacharparenright}{\kern0pt}{\isacharparenright}{\kern0pt}\isanewline
\ \ \isacommand{hence}\isamarkupfalse%
\ {\isachardoublequoteopen}x\ {\isasymin}\ {\isacharparenleft}{\kern0pt}{\isacharparenleft}{\kern0pt}elect\ m\ A\ p{\isacharparenright}{\kern0pt}\ {\isasymunion}\ {\isacharparenleft}{\kern0pt}elect\ n\ A\ p{\isacharparenright}{\kern0pt}{\isacharparenright}{\kern0pt}{\isachardoublequoteclose}\isanewline
\ \ \ \ \isacommand{using}\isamarkupfalse%
\ x{\isacharunderscore}{\kern0pt}wins\isanewline
\ \ \ \ \isacommand{by}\isamarkupfalse%
\ auto\isanewline
\ \ \isacommand{thus}\isamarkupfalse%
\ {\isachardoublequoteopen}x\ {\isasymin}\ {\isacharbraceleft}{\kern0pt}{\isacharbraceright}{\kern0pt}{\isachardoublequoteclose}\isanewline
\ \ \ \ \isacommand{using}\isamarkupfalse%
\ sup{\isacharunderscore}{\kern0pt}bot{\isacharunderscore}{\kern0pt}right\ non{\isacharunderscore}{\kern0pt}electing{\isacharunderscore}{\kern0pt}def\ fin{\isacharunderscore}{\kern0pt}A\isanewline
\ \ \ \ \ \ \ \ \ \ non{\isacharunderscore}{\kern0pt}electing{\isacharunderscore}{\kern0pt}m\ non{\isacharunderscore}{\kern0pt}electing{\isacharunderscore}{\kern0pt}n\ prof{\isacharunderscore}{\kern0pt}A\isanewline
\ \ \ \ \isacommand{by}\isamarkupfalse%
\ {\isacharparenleft}{\kern0pt}metis\ {\isacharparenleft}{\kern0pt}no{\isacharunderscore}{\kern0pt}types{\isacharcomma}{\kern0pt}\ lifting{\isacharparenright}{\kern0pt}{\isacharparenright}{\kern0pt}\isanewline
\isacommand{qed}\isamarkupfalse%
%
\endisatagproof
{\isafoldproof}%
%
\isadelimproof
\isanewline
%
\endisadelimproof
\isanewline
\isanewline
\isacommand{theorem}\isamarkupfalse%
\ conserv{\isacharunderscore}{\kern0pt}max{\isacharunderscore}{\kern0pt}agg{\isacharunderscore}{\kern0pt}presv{\isacharunderscore}{\kern0pt}non{\isacharunderscore}{\kern0pt}electing{\isacharbrackleft}{\kern0pt}simp{\isacharbrackright}{\kern0pt}{\isacharcolon}{\kern0pt}\isanewline
\ \ \isakeyword{assumes}\isanewline
\ \ \ \ non{\isacharunderscore}{\kern0pt}electing{\isacharunderscore}{\kern0pt}m{\isacharcolon}{\kern0pt}\ {\isachardoublequoteopen}non{\isacharunderscore}{\kern0pt}electing\ m{\isachardoublequoteclose}\ \isakeyword{and}\isanewline
\ \ \ \ non{\isacharunderscore}{\kern0pt}electing{\isacharunderscore}{\kern0pt}n{\isacharcolon}{\kern0pt}\ {\isachardoublequoteopen}non{\isacharunderscore}{\kern0pt}electing\ n{\isachardoublequoteclose}\isanewline
\ \ \isakeyword{shows}\ {\isachardoublequoteopen}non{\isacharunderscore}{\kern0pt}electing\ {\isacharparenleft}{\kern0pt}m\ {\isasymparallel}\isactrlsub {\isasymup}\ n{\isacharparenright}{\kern0pt}{\isachardoublequoteclose}\isanewline
%
\isadelimproof
\ \ %
\endisadelimproof
%
\isatagproof
\isacommand{using}\isamarkupfalse%
\ non{\isacharunderscore}{\kern0pt}electing{\isacharunderscore}{\kern0pt}m\ non{\isacharunderscore}{\kern0pt}electing{\isacharunderscore}{\kern0pt}n\isanewline
\ \ \isacommand{by}\isamarkupfalse%
\ simp%
\endisatagproof
{\isafoldproof}%
%
\isadelimproof
\isanewline
%
\endisadelimproof
\isanewline
\isanewline
\isacommand{theorem}\isamarkupfalse%
\ seq{\isacharunderscore}{\kern0pt}comp{\isacharunderscore}{\kern0pt}presv{\isacharunderscore}{\kern0pt}non{\isacharunderscore}{\kern0pt}electing{\isacharbrackleft}{\kern0pt}simp{\isacharbrackright}{\kern0pt}{\isacharcolon}{\kern0pt}\isanewline
\ \ \isakeyword{assumes}\isanewline
\ \ \ \ {\isachardoublequoteopen}non{\isacharunderscore}{\kern0pt}electing\ m{\isachardoublequoteclose}\ \isakeyword{and}\isanewline
\ \ \ \ {\isachardoublequoteopen}non{\isacharunderscore}{\kern0pt}electing\ n{\isachardoublequoteclose}\isanewline
\ \ \isakeyword{shows}\ {\isachardoublequoteopen}non{\isacharunderscore}{\kern0pt}electing\ {\isacharparenleft}{\kern0pt}m\ {\isasymtriangleright}\ n{\isacharparenright}{\kern0pt}{\isachardoublequoteclose}\isanewline
%
\isadelimproof
\ \ %
\endisadelimproof
%
\isatagproof
\isacommand{using}\isamarkupfalse%
\ Un{\isacharunderscore}{\kern0pt}empty\ assms\ non{\isacharunderscore}{\kern0pt}electing{\isacharunderscore}{\kern0pt}def\ prod{\isachardot}{\kern0pt}sel\ seq{\isacharunderscore}{\kern0pt}comp{\isacharunderscore}{\kern0pt}sound\isanewline
\ \ \ \ \ \ \ \ sequential{\isacharunderscore}{\kern0pt}composition{\isachardot}{\kern0pt}simps\ def{\isacharunderscore}{\kern0pt}presv{\isacharunderscore}{\kern0pt}fin{\isacharunderscore}{\kern0pt}prof\isanewline
\ \ \isacommand{by}\isamarkupfalse%
\ {\isacharparenleft}{\kern0pt}smt\ {\isacharparenleft}{\kern0pt}verit{\isacharcomma}{\kern0pt}\ del{\isacharunderscore}{\kern0pt}insts{\isacharparenright}{\kern0pt}{\isacharparenright}{\kern0pt}%
\endisatagproof
{\isafoldproof}%
%
\isadelimproof
\isanewline
%
\endisadelimproof
\isanewline
\isacommand{lemma}\isamarkupfalse%
\ loop{\isacharunderscore}{\kern0pt}comp{\isacharunderscore}{\kern0pt}presv{\isacharunderscore}{\kern0pt}non{\isacharunderscore}{\kern0pt}electing{\isacharunderscore}{\kern0pt}helper{\isacharcolon}{\kern0pt}\isanewline
\ \ \isakeyword{assumes}\isanewline
\ \ \ \ non{\isacharunderscore}{\kern0pt}electing{\isacharunderscore}{\kern0pt}m{\isacharcolon}{\kern0pt}\ {\isachardoublequoteopen}non{\isacharunderscore}{\kern0pt}electing\ m{\isachardoublequoteclose}\ \isakeyword{and}\isanewline
\ \ \ \ f{\isacharunderscore}{\kern0pt}prof{\isacharcolon}{\kern0pt}\ {\isachardoublequoteopen}finite{\isacharunderscore}{\kern0pt}profile\ A\ p{\isachardoublequoteclose}\isanewline
\ \ \isakeyword{shows}\isanewline
\ \ \ \ {\isachardoublequoteopen}{\isacharparenleft}{\kern0pt}n\ {\isacharequal}{\kern0pt}\ card\ {\isacharparenleft}{\kern0pt}defer\ acc\ A\ p{\isacharparenright}{\kern0pt}\ {\isasymand}\ non{\isacharunderscore}{\kern0pt}electing\ acc{\isacharparenright}{\kern0pt}\ {\isasymLongrightarrow}\isanewline
\ \ \ \ \ \ \ \ elect\ {\isacharparenleft}{\kern0pt}loop{\isacharunderscore}{\kern0pt}comp{\isacharunderscore}{\kern0pt}helper\ acc\ m\ t{\isacharparenright}{\kern0pt}\ A\ p\ {\isacharequal}{\kern0pt}\ {\isacharbraceleft}{\kern0pt}{\isacharbraceright}{\kern0pt}{\isachardoublequoteclose}\isanewline
%
\isadelimproof
%
\endisadelimproof
%
\isatagproof
\isacommand{proof}\isamarkupfalse%
\ {\isacharparenleft}{\kern0pt}induct\ n\ arbitrary{\isacharcolon}{\kern0pt}\ acc\ rule{\isacharcolon}{\kern0pt}\ less{\isacharunderscore}{\kern0pt}induct{\isacharparenright}{\kern0pt}\isanewline
\ \ \isacommand{case}\isamarkupfalse%
{\isacharparenleft}{\kern0pt}less\ n{\isacharparenright}{\kern0pt}\isanewline
\ \ \isacommand{thus}\isamarkupfalse%
\ {\isacharquery}{\kern0pt}case\isanewline
\ \ \ \ \isacommand{using}\isamarkupfalse%
\ loop{\isacharunderscore}{\kern0pt}comp{\isacharunderscore}{\kern0pt}helper{\isachardot}{\kern0pt}simps{\isacharparenleft}{\kern0pt}{\isadigit{1}}{\isacharparenright}{\kern0pt}\ loop{\isacharunderscore}{\kern0pt}comp{\isacharunderscore}{\kern0pt}helper{\isachardot}{\kern0pt}simps{\isacharparenleft}{\kern0pt}{\isadigit{2}}{\isacharparenright}{\kern0pt}\isanewline
\ \ \ \ \ \ \ \ \ \ non{\isacharunderscore}{\kern0pt}electing{\isacharunderscore}{\kern0pt}def\ non{\isacharunderscore}{\kern0pt}electing{\isacharunderscore}{\kern0pt}m\ f{\isacharunderscore}{\kern0pt}prof\ psubset{\isacharunderscore}{\kern0pt}card{\isacharunderscore}{\kern0pt}mono\isanewline
\ \ \ \ \ \ \ \ \ \ seq{\isacharunderscore}{\kern0pt}comp{\isacharunderscore}{\kern0pt}presv{\isacharunderscore}{\kern0pt}non{\isacharunderscore}{\kern0pt}electing\isanewline
\ \ \ \ \isacommand{by}\isamarkupfalse%
\ {\isacharparenleft}{\kern0pt}smt\ {\isacharparenleft}{\kern0pt}verit{\isacharcomma}{\kern0pt}\ ccfv{\isacharunderscore}{\kern0pt}threshold{\isacharparenright}{\kern0pt}{\isacharparenright}{\kern0pt}\isanewline
\isacommand{qed}\isamarkupfalse%
%
\endisatagproof
{\isafoldproof}%
%
\isadelimproof
\isanewline
%
\endisadelimproof
\isanewline
\isanewline
\isacommand{theorem}\isamarkupfalse%
\ loop{\isacharunderscore}{\kern0pt}comp{\isacharunderscore}{\kern0pt}presv{\isacharunderscore}{\kern0pt}non{\isacharunderscore}{\kern0pt}electing{\isacharbrackleft}{\kern0pt}simp{\isacharbrackright}{\kern0pt}{\isacharcolon}{\kern0pt}\isanewline
\ \ \isakeyword{assumes}\ non{\isacharunderscore}{\kern0pt}electing{\isacharunderscore}{\kern0pt}m{\isacharcolon}{\kern0pt}\ {\isachardoublequoteopen}non{\isacharunderscore}{\kern0pt}electing\ m{\isachardoublequoteclose}\isanewline
\ \ \isakeyword{shows}\ {\isachardoublequoteopen}non{\isacharunderscore}{\kern0pt}electing\ {\isacharparenleft}{\kern0pt}m\ {\isasymcirclearrowleft}\isactrlsub t{\isacharparenright}{\kern0pt}{\isachardoublequoteclose}\isanewline
%
\isadelimproof
\ \ %
\endisadelimproof
%
\isatagproof
\isacommand{unfolding}\isamarkupfalse%
\ non{\isacharunderscore}{\kern0pt}electing{\isacharunderscore}{\kern0pt}def\isanewline
\isacommand{proof}\isamarkupfalse%
\ {\isacharparenleft}{\kern0pt}safe{\isacharcomma}{\kern0pt}\ simp{\isacharunderscore}{\kern0pt}all{\isacharparenright}{\kern0pt}\isanewline
\ \ \isacommand{show}\isamarkupfalse%
\ {\isachardoublequoteopen}electoral{\isacharunderscore}{\kern0pt}module\ {\isacharparenleft}{\kern0pt}m\ {\isasymcirclearrowleft}\isactrlsub t{\isacharparenright}{\kern0pt}{\isachardoublequoteclose}\isanewline
\ \ \ \ \isacommand{using}\isamarkupfalse%
\ loop{\isacharunderscore}{\kern0pt}comp{\isacharunderscore}{\kern0pt}sound\ non{\isacharunderscore}{\kern0pt}electing{\isacharunderscore}{\kern0pt}def\ non{\isacharunderscore}{\kern0pt}electing{\isacharunderscore}{\kern0pt}m\isanewline
\ \ \ \ \isacommand{by}\isamarkupfalse%
\ metis\isanewline
\isacommand{next}\isamarkupfalse%
\isanewline
\ \ \ \ \isacommand{fix}\isamarkupfalse%
\isanewline
\ \ \ \ \ \ A\ {\isacharcolon}{\kern0pt}{\isacharcolon}{\kern0pt}\ {\isachardoublequoteopen}{\isacharprime}{\kern0pt}a\ set{\isachardoublequoteclose}\ \isakeyword{and}\isanewline
\ \ \ \ \ \ p\ {\isacharcolon}{\kern0pt}{\isacharcolon}{\kern0pt}\ {\isachardoublequoteopen}{\isacharprime}{\kern0pt}a\ Profile{\isachardoublequoteclose}\ \isakeyword{and}\isanewline
\ \ \ \ \ \ x\ {\isacharcolon}{\kern0pt}{\isacharcolon}{\kern0pt}\ {\isachardoublequoteopen}{\isacharprime}{\kern0pt}a{\isachardoublequoteclose}\isanewline
\ \ \ \ \isacommand{assume}\isamarkupfalse%
\isanewline
\ \ \ \ \ \ fin{\isacharunderscore}{\kern0pt}A{\isacharcolon}{\kern0pt}\ {\isachardoublequoteopen}finite\ A{\isachardoublequoteclose}\ \isakeyword{and}\isanewline
\ \ \ \ \ \ prof{\isacharunderscore}{\kern0pt}A{\isacharcolon}{\kern0pt}\ {\isachardoublequoteopen}profile\ A\ p{\isachardoublequoteclose}\ \isakeyword{and}\isanewline
\ \ \ \ \ \ x{\isacharunderscore}{\kern0pt}elect{\isacharcolon}{\kern0pt}\ {\isachardoublequoteopen}x\ {\isasymin}\ elect\ {\isacharparenleft}{\kern0pt}m\ {\isasymcirclearrowleft}\isactrlsub t{\isacharparenright}{\kern0pt}\ A\ p{\isachardoublequoteclose}\isanewline
\ \ \ \ \isacommand{show}\isamarkupfalse%
\ {\isachardoublequoteopen}False{\isachardoublequoteclose}\isanewline
\ \ \isacommand{using}\isamarkupfalse%
\ def{\isacharunderscore}{\kern0pt}mod{\isacharunderscore}{\kern0pt}non{\isacharunderscore}{\kern0pt}electing\ loop{\isacharunderscore}{\kern0pt}comp{\isacharunderscore}{\kern0pt}presv{\isacharunderscore}{\kern0pt}non{\isacharunderscore}{\kern0pt}electing{\isacharunderscore}{\kern0pt}helper\isanewline
\ \ \ \ \ \ \ \ non{\isacharunderscore}{\kern0pt}electing{\isacharunderscore}{\kern0pt}m\ empty{\isacharunderscore}{\kern0pt}iff\ fin{\isacharunderscore}{\kern0pt}A\ loop{\isacharunderscore}{\kern0pt}comp{\isacharunderscore}{\kern0pt}code\isanewline
\ \ \ \ \ \ \ \ non{\isacharunderscore}{\kern0pt}electing{\isacharunderscore}{\kern0pt}def\ prof{\isacharunderscore}{\kern0pt}A\ x{\isacharunderscore}{\kern0pt}elect\isanewline
\ \ \isacommand{by}\isamarkupfalse%
\ metis\isanewline
\isacommand{qed}\isamarkupfalse%
%
\endisatagproof
{\isafoldproof}%
%
\isadelimproof
\isanewline
%
\endisadelimproof
\isanewline
\isanewline
\isacommand{theorem}\isamarkupfalse%
\ rev{\isacharunderscore}{\kern0pt}comp{\isacharunderscore}{\kern0pt}non{\isacharunderscore}{\kern0pt}blocking{\isacharbrackleft}{\kern0pt}simp{\isacharbrackright}{\kern0pt}{\isacharcolon}{\kern0pt}\isanewline
\ \ \isakeyword{assumes}\ {\isachardoublequoteopen}electing\ m{\isachardoublequoteclose}\isanewline
\ \ \isakeyword{shows}\ {\isachardoublequoteopen}non{\isacharunderscore}{\kern0pt}blocking\ {\isacharparenleft}{\kern0pt}m{\isasymdown}{\isacharparenright}{\kern0pt}{\isachardoublequoteclose}\isanewline
%
\isadelimproof
\ \ %
\endisadelimproof
%
\isatagproof
\isacommand{unfolding}\isamarkupfalse%
\ non{\isacharunderscore}{\kern0pt}blocking{\isacharunderscore}{\kern0pt}def\isanewline
\isacommand{proof}\isamarkupfalse%
\ {\isacharparenleft}{\kern0pt}safe{\isacharcomma}{\kern0pt}\ simp{\isacharunderscore}{\kern0pt}all{\isacharparenright}{\kern0pt}\isanewline
\ \ \isacommand{show}\isamarkupfalse%
\ {\isachardoublequoteopen}electoral{\isacharunderscore}{\kern0pt}module\ {\isacharparenleft}{\kern0pt}m{\isasymdown}{\isacharparenright}{\kern0pt}{\isachardoublequoteclose}\isanewline
\ \ \ \ \isacommand{using}\isamarkupfalse%
\ assms\ electing{\isacharunderscore}{\kern0pt}def\ rev{\isacharunderscore}{\kern0pt}comp{\isacharunderscore}{\kern0pt}sound\isanewline
\ \ \ \ \isacommand{by}\isamarkupfalse%
\ {\isacharparenleft}{\kern0pt}metis\ {\isacharparenleft}{\kern0pt}no{\isacharunderscore}{\kern0pt}types{\isacharcomma}{\kern0pt}\ lifting{\isacharparenright}{\kern0pt}{\isacharparenright}{\kern0pt}\isanewline
\isacommand{next}\isamarkupfalse%
\isanewline
\ \ \isacommand{fix}\isamarkupfalse%
\isanewline
\ \ \ \ A\ {\isacharcolon}{\kern0pt}{\isacharcolon}{\kern0pt}\ {\isachardoublequoteopen}{\isacharprime}{\kern0pt}a\ set{\isachardoublequoteclose}\ \isakeyword{and}\isanewline
\ \ \ \ p\ {\isacharcolon}{\kern0pt}{\isacharcolon}{\kern0pt}\ {\isachardoublequoteopen}{\isacharprime}{\kern0pt}a\ Profile{\isachardoublequoteclose}\ \isakeyword{and}\isanewline
\ \ \ \ x\ {\isacharcolon}{\kern0pt}{\isacharcolon}{\kern0pt}\ {\isachardoublequoteopen}{\isacharprime}{\kern0pt}a{\isachardoublequoteclose}\isanewline
\ \ \isacommand{assume}\isamarkupfalse%
\isanewline
\ \ \ \ fin{\isacharunderscore}{\kern0pt}A{\isacharcolon}{\kern0pt}\ {\isachardoublequoteopen}finite\ A{\isachardoublequoteclose}\ \isakeyword{and}\isanewline
\ \ \ \ prof{\isacharunderscore}{\kern0pt}A{\isacharcolon}{\kern0pt}\ {\isachardoublequoteopen}profile\ A\ p{\isachardoublequoteclose}\ \isakeyword{and}\isanewline
\ \ \ \ no{\isacharunderscore}{\kern0pt}elect{\isacharcolon}{\kern0pt}\ {\isachardoublequoteopen}A\ {\isacharminus}{\kern0pt}\ elect\ m\ A\ p\ {\isacharequal}{\kern0pt}\ A{\isachardoublequoteclose}\ \isakeyword{and}\isanewline
\ \ \ \ x{\isacharunderscore}{\kern0pt}in{\isacharunderscore}{\kern0pt}A{\isacharcolon}{\kern0pt}\ {\isachardoublequoteopen}x\ {\isasymin}\ A{\isachardoublequoteclose}\isanewline
\ \ \isacommand{from}\isamarkupfalse%
\ no{\isacharunderscore}{\kern0pt}elect\ \isacommand{have}\isamarkupfalse%
\ non{\isacharunderscore}{\kern0pt}elect{\isacharcolon}{\kern0pt}\isanewline
\ \ \ \ {\isachardoublequoteopen}non{\isacharunderscore}{\kern0pt}electing\ m{\isachardoublequoteclose}\isanewline
\ \ \ \ \isacommand{using}\isamarkupfalse%
\ assms\ prof{\isacharunderscore}{\kern0pt}A\ x{\isacharunderscore}{\kern0pt}in{\isacharunderscore}{\kern0pt}A\ fin{\isacharunderscore}{\kern0pt}A\ electing{\isacharunderscore}{\kern0pt}def\ empty{\isacharunderscore}{\kern0pt}iff\isanewline
\ \ \ \ \ \ \ \ \ \ Diff{\isacharunderscore}{\kern0pt}disjoint\ Int{\isacharunderscore}{\kern0pt}absorb{\isadigit{2}}\ elect{\isacharunderscore}{\kern0pt}in{\isacharunderscore}{\kern0pt}alts\isanewline
\ \ \ \ \isacommand{by}\isamarkupfalse%
\ {\isacharparenleft}{\kern0pt}metis\ {\isacharparenleft}{\kern0pt}no{\isacharunderscore}{\kern0pt}types{\isacharcomma}{\kern0pt}\ lifting{\isacharparenright}{\kern0pt}{\isacharparenright}{\kern0pt}\isanewline
\ \ \isacommand{show}\isamarkupfalse%
\ {\isachardoublequoteopen}False{\isachardoublequoteclose}\isanewline
\ \ \ \ \isacommand{using}\isamarkupfalse%
\ non{\isacharunderscore}{\kern0pt}elect\ assms\ electing{\isacharunderscore}{\kern0pt}def\ empty{\isacharunderscore}{\kern0pt}iff\ fin{\isacharunderscore}{\kern0pt}A\isanewline
\ \ \ \ \ \ \ \ \ \ non{\isacharunderscore}{\kern0pt}electing{\isacharunderscore}{\kern0pt}def\ prof{\isacharunderscore}{\kern0pt}A\ x{\isacharunderscore}{\kern0pt}in{\isacharunderscore}{\kern0pt}A\isanewline
\ \ \ \ \isacommand{by}\isamarkupfalse%
\ {\isacharparenleft}{\kern0pt}metis\ {\isacharparenleft}{\kern0pt}no{\isacharunderscore}{\kern0pt}types{\isacharcomma}{\kern0pt}\ lifting{\isacharparenright}{\kern0pt}{\isacharparenright}{\kern0pt}\isanewline
\isacommand{qed}\isamarkupfalse%
%
\endisatagproof
{\isafoldproof}%
%
\isadelimproof
\isanewline
%
\endisadelimproof
\isanewline
\isanewline
\isacommand{theorem}\isamarkupfalse%
\ seq{\isacharunderscore}{\kern0pt}comp{\isacharunderscore}{\kern0pt}def{\isacharunderscore}{\kern0pt}one{\isacharbrackleft}{\kern0pt}simp{\isacharbrackright}{\kern0pt}{\isacharcolon}{\kern0pt}\isanewline
\ \ \isakeyword{assumes}\isanewline
\ \ \ \ non{\isacharunderscore}{\kern0pt}blocking{\isacharunderscore}{\kern0pt}m{\isacharcolon}{\kern0pt}\ {\isachardoublequoteopen}non{\isacharunderscore}{\kern0pt}blocking\ m{\isachardoublequoteclose}\ \isakeyword{and}\isanewline
\ \ \ \ non{\isacharunderscore}{\kern0pt}electing{\isacharunderscore}{\kern0pt}m{\isacharcolon}{\kern0pt}\ {\isachardoublequoteopen}non{\isacharunderscore}{\kern0pt}electing\ m{\isachardoublequoteclose}\ \isakeyword{and}\isanewline
\ \ \ \ def{\isacharunderscore}{\kern0pt}{\isadigit{1}}{\isacharunderscore}{\kern0pt}n{\isacharcolon}{\kern0pt}\ {\isachardoublequoteopen}defers\ {\isadigit{1}}\ n{\isachardoublequoteclose}\isanewline
\ \ \isakeyword{shows}\ {\isachardoublequoteopen}defers\ {\isadigit{1}}\ {\isacharparenleft}{\kern0pt}m\ {\isasymtriangleright}\ n{\isacharparenright}{\kern0pt}{\isachardoublequoteclose}\isanewline
%
\isadelimproof
\ \ %
\endisadelimproof
%
\isatagproof
\isacommand{unfolding}\isamarkupfalse%
\ defers{\isacharunderscore}{\kern0pt}def\isanewline
\isacommand{proof}\isamarkupfalse%
\ {\isacharparenleft}{\kern0pt}safe{\isacharparenright}{\kern0pt}\isanewline
\ \ \isacommand{have}\isamarkupfalse%
\ electoral{\isacharunderscore}{\kern0pt}mod{\isacharunderscore}{\kern0pt}m{\isacharcolon}{\kern0pt}\ {\isachardoublequoteopen}electoral{\isacharunderscore}{\kern0pt}module\ m{\isachardoublequoteclose}\isanewline
\ \ \ \ \isacommand{using}\isamarkupfalse%
\ non{\isacharunderscore}{\kern0pt}electing{\isacharunderscore}{\kern0pt}m\isanewline
\ \ \ \ \isacommand{by}\isamarkupfalse%
\ {\isacharparenleft}{\kern0pt}simp\ add{\isacharcolon}{\kern0pt}\ non{\isacharunderscore}{\kern0pt}electing{\isacharunderscore}{\kern0pt}def{\isacharparenright}{\kern0pt}\isanewline
\ \ \isacommand{have}\isamarkupfalse%
\ electoral{\isacharunderscore}{\kern0pt}mod{\isacharunderscore}{\kern0pt}n{\isacharcolon}{\kern0pt}\ {\isachardoublequoteopen}electoral{\isacharunderscore}{\kern0pt}module\ n{\isachardoublequoteclose}\isanewline
\ \ \ \ \isacommand{using}\isamarkupfalse%
\ def{\isacharunderscore}{\kern0pt}{\isadigit{1}}{\isacharunderscore}{\kern0pt}n\isanewline
\ \ \ \ \isacommand{by}\isamarkupfalse%
\ {\isacharparenleft}{\kern0pt}simp\ add{\isacharcolon}{\kern0pt}\ defers{\isacharunderscore}{\kern0pt}def{\isacharparenright}{\kern0pt}\isanewline
\ \ \isacommand{show}\isamarkupfalse%
\ {\isachardoublequoteopen}electoral{\isacharunderscore}{\kern0pt}module\ {\isacharparenleft}{\kern0pt}m\ {\isasymtriangleright}\ n{\isacharparenright}{\kern0pt}{\isachardoublequoteclose}\isanewline
\ \ \ \ \isacommand{using}\isamarkupfalse%
\ electoral{\isacharunderscore}{\kern0pt}mod{\isacharunderscore}{\kern0pt}m\ electoral{\isacharunderscore}{\kern0pt}mod{\isacharunderscore}{\kern0pt}n\isanewline
\ \ \ \ \isacommand{by}\isamarkupfalse%
\ simp\isanewline
\isacommand{next}\isamarkupfalse%
\isanewline
\ \ \isacommand{fix}\isamarkupfalse%
\isanewline
\ \ \ \ A\ {\isacharcolon}{\kern0pt}{\isacharcolon}{\kern0pt}\ {\isachardoublequoteopen}{\isacharprime}{\kern0pt}a\ set{\isachardoublequoteclose}\ \isakeyword{and}\isanewline
\ \ \ \ p\ {\isacharcolon}{\kern0pt}{\isacharcolon}{\kern0pt}\ {\isachardoublequoteopen}{\isacharprime}{\kern0pt}a\ Profile{\isachardoublequoteclose}\isanewline
\ \ \isacommand{assume}\isamarkupfalse%
\isanewline
\ \ \ \ pos{\isacharunderscore}{\kern0pt}card{\isacharcolon}{\kern0pt}\ {\isachardoublequoteopen}{\isadigit{1}}\ {\isasymle}\ card\ A{\isachardoublequoteclose}\ \isakeyword{and}\isanewline
\ \ \ \ fin{\isacharunderscore}{\kern0pt}A{\isacharcolon}{\kern0pt}\ {\isachardoublequoteopen}finite\ A{\isachardoublequoteclose}\ \isakeyword{and}\isanewline
\ \ \ \ prof{\isacharunderscore}{\kern0pt}A{\isacharcolon}{\kern0pt}\ {\isachardoublequoteopen}profile\ A\ p{\isachardoublequoteclose}\isanewline
\ \ \isacommand{from}\isamarkupfalse%
\ pos{\isacharunderscore}{\kern0pt}card\ \isacommand{have}\isamarkupfalse%
\isanewline
\ \ \ \ {\isachardoublequoteopen}A\ {\isasymnoteq}\ {\isacharbraceleft}{\kern0pt}{\isacharbraceright}{\kern0pt}{\isachardoublequoteclose}\isanewline
\ \ \ \ \isacommand{by}\isamarkupfalse%
\ auto\isanewline
\ \ \isacommand{with}\isamarkupfalse%
\ fin{\isacharunderscore}{\kern0pt}A\ prof{\isacharunderscore}{\kern0pt}A\ \isacommand{have}\isamarkupfalse%
\ m{\isacharunderscore}{\kern0pt}non{\isacharunderscore}{\kern0pt}blocking{\isacharcolon}{\kern0pt}\isanewline
\ \ \ \ {\isachardoublequoteopen}reject\ m\ A\ p\ {\isasymnoteq}\ A{\isachardoublequoteclose}\isanewline
\ \ \ \ \isacommand{using}\isamarkupfalse%
\ non{\isacharunderscore}{\kern0pt}blocking{\isacharunderscore}{\kern0pt}m\ non{\isacharunderscore}{\kern0pt}blocking{\isacharunderscore}{\kern0pt}def\isanewline
\ \ \ \ \isacommand{by}\isamarkupfalse%
\ metis\isanewline
\ \ \isacommand{hence}\isamarkupfalse%
\isanewline
\ \ \ \ {\isachardoublequoteopen}{\isasymexists}a{\isachardot}{\kern0pt}\ a\ {\isasymin}\ A\ {\isasymand}\ a\ {\isasymnotin}\ reject\ m\ A\ p{\isachardoublequoteclose}\isanewline
\ \ \ \ \isacommand{using}\isamarkupfalse%
\ pos{\isacharunderscore}{\kern0pt}card\ non{\isacharunderscore}{\kern0pt}electing{\isacharunderscore}{\kern0pt}def\ non{\isacharunderscore}{\kern0pt}electing{\isacharunderscore}{\kern0pt}m\isanewline
\ \ \ \ \ \ \ \ \ \ reject{\isacharunderscore}{\kern0pt}in{\isacharunderscore}{\kern0pt}alts\ subset{\isacharunderscore}{\kern0pt}antisym\ subset{\isacharunderscore}{\kern0pt}iff\isanewline
\ \ \ \ \ \ \ \ \ \ fin{\isacharunderscore}{\kern0pt}A\ prof{\isacharunderscore}{\kern0pt}A\ subsetI\isanewline
\ \ \ \ \isacommand{by}\isamarkupfalse%
\ metis\isanewline
\ \ \isacommand{hence}\isamarkupfalse%
\ {\isachardoublequoteopen}defer\ m\ A\ p\ {\isasymnoteq}\ {\isacharbraceleft}{\kern0pt}{\isacharbraceright}{\kern0pt}{\isachardoublequoteclose}\isanewline
\ \ \ \ \isacommand{using}\isamarkupfalse%
\ electoral{\isacharunderscore}{\kern0pt}mod{\isacharunderscore}{\kern0pt}defer{\isacharunderscore}{\kern0pt}elem\ empty{\isacharunderscore}{\kern0pt}iff\ pos{\isacharunderscore}{\kern0pt}card\isanewline
\ \ \ \ \ \ \ \ \ \ non{\isacharunderscore}{\kern0pt}electing{\isacharunderscore}{\kern0pt}def\ non{\isacharunderscore}{\kern0pt}electing{\isacharunderscore}{\kern0pt}m\ fin{\isacharunderscore}{\kern0pt}A\ prof{\isacharunderscore}{\kern0pt}A\isanewline
\ \ \ \ \isacommand{by}\isamarkupfalse%
\ {\isacharparenleft}{\kern0pt}metis\ {\isacharparenleft}{\kern0pt}no{\isacharunderscore}{\kern0pt}types{\isacharparenright}{\kern0pt}{\isacharparenright}{\kern0pt}\isanewline
\ \ \isacommand{hence}\isamarkupfalse%
\ defer{\isacharunderscore}{\kern0pt}non{\isacharunderscore}{\kern0pt}empty{\isacharcolon}{\kern0pt}\isanewline
\ \ \ \ {\isachardoublequoteopen}card\ {\isacharparenleft}{\kern0pt}defer\ m\ A\ p{\isacharparenright}{\kern0pt}\ {\isasymge}\ {\isadigit{1}}{\isachardoublequoteclose}\isanewline
\ \ \ \ \isacommand{using}\isamarkupfalse%
\ One{\isacharunderscore}{\kern0pt}nat{\isacharunderscore}{\kern0pt}def\ Suc{\isacharunderscore}{\kern0pt}leI\ card{\isacharunderscore}{\kern0pt}gt{\isacharunderscore}{\kern0pt}{\isadigit{0}}{\isacharunderscore}{\kern0pt}iff\ pos{\isacharunderscore}{\kern0pt}card\ fin{\isacharunderscore}{\kern0pt}A\ prof{\isacharunderscore}{\kern0pt}A\isanewline
\ \ \ \ \ \ \ \ \ \ non{\isacharunderscore}{\kern0pt}blocking{\isacharunderscore}{\kern0pt}def\ non{\isacharunderscore}{\kern0pt}blocking{\isacharunderscore}{\kern0pt}m\ def{\isacharunderscore}{\kern0pt}presv{\isacharunderscore}{\kern0pt}fin{\isacharunderscore}{\kern0pt}prof\isanewline
\ \ \ \ \isacommand{by}\isamarkupfalse%
\ metis\isanewline
\ \ \isacommand{have}\isamarkupfalse%
\ defer{\isacharunderscore}{\kern0pt}fun{\isacharcolon}{\kern0pt}\isanewline
\ \ \ \ {\isachardoublequoteopen}defer\ {\isacharparenleft}{\kern0pt}m\ {\isasymtriangleright}\ n{\isacharparenright}{\kern0pt}\ A\ p\ {\isacharequal}{\kern0pt}\isanewline
\ \ \ \ \ \ defer\ n\ {\isacharparenleft}{\kern0pt}defer\ m\ A\ p{\isacharparenright}{\kern0pt}\ {\isacharparenleft}{\kern0pt}limit{\isacharunderscore}{\kern0pt}profile\ {\isacharparenleft}{\kern0pt}defer\ m\ A\ p{\isacharparenright}{\kern0pt}\ p{\isacharparenright}{\kern0pt}{\isachardoublequoteclose}\isanewline
\ \ \ \ \isacommand{using}\isamarkupfalse%
\ def{\isacharunderscore}{\kern0pt}{\isadigit{1}}{\isacharunderscore}{\kern0pt}n\ defers{\isacharunderscore}{\kern0pt}def\ fin{\isacharunderscore}{\kern0pt}A\ non{\isacharunderscore}{\kern0pt}blocking{\isacharunderscore}{\kern0pt}def\ non{\isacharunderscore}{\kern0pt}blocking{\isacharunderscore}{\kern0pt}m\isanewline
\ \ \ \ \ \ \ \ \ \ prof{\isacharunderscore}{\kern0pt}A\ seq{\isacharunderscore}{\kern0pt}comp{\isacharunderscore}{\kern0pt}defers{\isacharunderscore}{\kern0pt}def{\isacharunderscore}{\kern0pt}set\isanewline
\ \ \ \ \isacommand{by}\isamarkupfalse%
\ {\isacharparenleft}{\kern0pt}metis\ {\isacharparenleft}{\kern0pt}no{\isacharunderscore}{\kern0pt}types{\isacharcomma}{\kern0pt}\ hide{\isacharunderscore}{\kern0pt}lams{\isacharparenright}{\kern0pt}{\isacharparenright}{\kern0pt}\isanewline
\ \ \isacommand{have}\isamarkupfalse%
\isanewline
\ \ \ \ {\isachardoublequoteopen}{\isasymforall}n\ f{\isachardot}{\kern0pt}\ defers\ n\ f\ {\isacharequal}{\kern0pt}\isanewline
\ \ \ \ \ \ {\isacharparenleft}{\kern0pt}electoral{\isacharunderscore}{\kern0pt}module\ f\ {\isasymand}\isanewline
\ \ \ \ \ \ \ \ {\isacharparenleft}{\kern0pt}{\isasymforall}A\ rs{\isachardot}{\kern0pt}\isanewline
\ \ \ \ \ \ \ \ \ \ {\isacharparenleft}{\kern0pt}{\isasymnot}\ n\ {\isasymle}\ card\ {\isacharparenleft}{\kern0pt}A{\isacharcolon}{\kern0pt}{\isacharcolon}{\kern0pt}{\isacharprime}{\kern0pt}a\ set{\isacharparenright}{\kern0pt}\ {\isasymor}\ infinite\ A\ {\isasymor}\isanewline
\ \ \ \ \ \ \ \ \ \ \ \ {\isasymnot}\ profile\ A\ rs{\isacharparenright}{\kern0pt}\ {\isasymor}\isanewline
\ \ \ \ \ \ \ \ \ \ card\ {\isacharparenleft}{\kern0pt}defer\ f\ A\ rs{\isacharparenright}{\kern0pt}\ {\isacharequal}{\kern0pt}\ n{\isacharparenright}{\kern0pt}{\isacharparenright}{\kern0pt}{\isachardoublequoteclose}\isanewline
\ \ \ \ \isacommand{using}\isamarkupfalse%
\ defers{\isacharunderscore}{\kern0pt}def\isanewline
\ \ \ \ \isacommand{by}\isamarkupfalse%
\ blast\isanewline
\ \ \isacommand{hence}\isamarkupfalse%
\isanewline
\ \ \ \ {\isachardoublequoteopen}card\ {\isacharparenleft}{\kern0pt}defer\ n\ {\isacharparenleft}{\kern0pt}defer\ m\ A\ p{\isacharparenright}{\kern0pt}\isanewline
\ \ \ \ \ \ {\isacharparenleft}{\kern0pt}limit{\isacharunderscore}{\kern0pt}profile\ {\isacharparenleft}{\kern0pt}defer\ m\ A\ p{\isacharparenright}{\kern0pt}\ p{\isacharparenright}{\kern0pt}{\isacharparenright}{\kern0pt}\ {\isacharequal}{\kern0pt}\ {\isadigit{1}}{\isachardoublequoteclose}\isanewline
\ \ \ \ \isacommand{using}\isamarkupfalse%
\ defer{\isacharunderscore}{\kern0pt}non{\isacharunderscore}{\kern0pt}empty\ def{\isacharunderscore}{\kern0pt}{\isadigit{1}}{\isacharunderscore}{\kern0pt}n\isanewline
\ \ \ \ \ \ \ \ \ \ fin{\isacharunderscore}{\kern0pt}A\ prof{\isacharunderscore}{\kern0pt}A\ non{\isacharunderscore}{\kern0pt}blocking{\isacharunderscore}{\kern0pt}def\isanewline
\ \ \ \ \ \ \ \ \ \ non{\isacharunderscore}{\kern0pt}blocking{\isacharunderscore}{\kern0pt}m\ def{\isacharunderscore}{\kern0pt}presv{\isacharunderscore}{\kern0pt}fin{\isacharunderscore}{\kern0pt}prof\isanewline
\ \ \ \ \isacommand{by}\isamarkupfalse%
\ metis\isanewline
\ \ \isacommand{thus}\isamarkupfalse%
\ {\isachardoublequoteopen}card\ {\isacharparenleft}{\kern0pt}defer\ {\isacharparenleft}{\kern0pt}m\ {\isasymtriangleright}\ n{\isacharparenright}{\kern0pt}\ A\ p{\isacharparenright}{\kern0pt}\ {\isacharequal}{\kern0pt}\ {\isadigit{1}}{\isachardoublequoteclose}\isanewline
\ \ \ \ \isacommand{using}\isamarkupfalse%
\ defer{\isacharunderscore}{\kern0pt}fun\isanewline
\ \ \ \ \isacommand{by}\isamarkupfalse%
\ auto\isanewline
\isacommand{qed}\isamarkupfalse%
%
\endisatagproof
{\isafoldproof}%
%
\isadelimproof
\isanewline
%
\endisadelimproof
\isanewline
\isacommand{lemma}\isamarkupfalse%
\ loop{\isacharunderscore}{\kern0pt}comp{\isacharunderscore}{\kern0pt}helper{\isacharunderscore}{\kern0pt}iter{\isacharunderscore}{\kern0pt}elim{\isacharunderscore}{\kern0pt}def{\isacharunderscore}{\kern0pt}n{\isacharunderscore}{\kern0pt}helper{\isacharcolon}{\kern0pt}\isanewline
\ \ \isakeyword{assumes}\isanewline
\ \ \ \ non{\isacharunderscore}{\kern0pt}electing{\isacharunderscore}{\kern0pt}m{\isacharcolon}{\kern0pt}\ {\isachardoublequoteopen}non{\isacharunderscore}{\kern0pt}electing\ m{\isachardoublequoteclose}\ \isakeyword{and}\isanewline
\ \ \ \ single{\isacharunderscore}{\kern0pt}elimination{\isacharcolon}{\kern0pt}\ {\isachardoublequoteopen}eliminates\ {\isadigit{1}}\ m{\isachardoublequoteclose}\ \isakeyword{and}\isanewline
\ \ \ \ terminate{\isacharunderscore}{\kern0pt}if{\isacharunderscore}{\kern0pt}n{\isacharunderscore}{\kern0pt}left{\isacharcolon}{\kern0pt}\ {\isachardoublequoteopen}{\isasymforall}\ r{\isachardot}{\kern0pt}\ {\isacharparenleft}{\kern0pt}{\isacharparenleft}{\kern0pt}t\ r{\isacharparenright}{\kern0pt}\ {\isasymlongleftrightarrow}\ {\isacharparenleft}{\kern0pt}card\ {\isacharparenleft}{\kern0pt}defer{\isacharunderscore}{\kern0pt}r\ r{\isacharparenright}{\kern0pt}\ {\isacharequal}{\kern0pt}\ x{\isacharparenright}{\kern0pt}{\isacharparenright}{\kern0pt}{\isachardoublequoteclose}\ \isakeyword{and}\isanewline
\ \ \ \ x{\isacharunderscore}{\kern0pt}greater{\isacharunderscore}{\kern0pt}zero{\isacharcolon}{\kern0pt}\ {\isachardoublequoteopen}x\ {\isachargreater}{\kern0pt}\ {\isadigit{0}}{\isachardoublequoteclose}\ \isakeyword{and}\isanewline
\ \ \ \ f{\isacharunderscore}{\kern0pt}prof{\isacharcolon}{\kern0pt}\ {\isachardoublequoteopen}finite{\isacharunderscore}{\kern0pt}profile\ A\ p{\isachardoublequoteclose}\isanewline
\ \ \isakeyword{shows}\isanewline
\ \ \ \ {\isachardoublequoteopen}{\isacharparenleft}{\kern0pt}n\ {\isacharequal}{\kern0pt}\ card\ {\isacharparenleft}{\kern0pt}defer\ acc\ A\ p{\isacharparenright}{\kern0pt}\ {\isasymand}\ n\ {\isasymge}\ x\ {\isasymand}\ card\ {\isacharparenleft}{\kern0pt}defer\ acc\ A\ p{\isacharparenright}{\kern0pt}\ {\isachargreater}{\kern0pt}\ {\isadigit{1}}\ {\isasymand}\isanewline
\ \ \ \ \ \ non{\isacharunderscore}{\kern0pt}electing\ acc{\isacharparenright}{\kern0pt}\ {\isasymlongrightarrow}\isanewline
\ \ \ \ \ \ \ \ \ \ card\ {\isacharparenleft}{\kern0pt}defer\ {\isacharparenleft}{\kern0pt}loop{\isacharunderscore}{\kern0pt}comp{\isacharunderscore}{\kern0pt}helper\ acc\ m\ t{\isacharparenright}{\kern0pt}\ A\ p{\isacharparenright}{\kern0pt}\ {\isacharequal}{\kern0pt}\ x{\isachardoublequoteclose}\isanewline
%
\isadelimproof
%
\endisadelimproof
%
\isatagproof
\isacommand{proof}\isamarkupfalse%
\ {\isacharparenleft}{\kern0pt}induct\ n\ arbitrary{\isacharcolon}{\kern0pt}\ acc\ rule{\isacharcolon}{\kern0pt}\ less{\isacharunderscore}{\kern0pt}induct{\isacharparenright}{\kern0pt}\isanewline
\ \ \isacommand{case}\isamarkupfalse%
{\isacharparenleft}{\kern0pt}less\ n{\isacharparenright}{\kern0pt}\isanewline
\ \ \isacommand{have}\isamarkupfalse%
\ subset{\isacharcolon}{\kern0pt}\isanewline
\ \ \ \ {\isachardoublequoteopen}{\isacharparenleft}{\kern0pt}card\ {\isacharparenleft}{\kern0pt}defer\ acc\ A\ p{\isacharparenright}{\kern0pt}\ {\isachargreater}{\kern0pt}\ {\isadigit{1}}\ {\isasymand}\ finite{\isacharunderscore}{\kern0pt}profile\ A\ p\ {\isasymand}\ electoral{\isacharunderscore}{\kern0pt}module\ acc{\isacharparenright}{\kern0pt}\ {\isasymlongrightarrow}\isanewline
\ \ \ \ \ \ \ \ defer\ {\isacharparenleft}{\kern0pt}acc\ {\isasymtriangleright}\ m{\isacharparenright}{\kern0pt}\ A\ p\ {\isasymsubset}\ defer\ acc\ A\ p{\isachardoublequoteclose}\isanewline
\ \ \ \ \isacommand{using}\isamarkupfalse%
\ seq{\isacharunderscore}{\kern0pt}comp{\isacharunderscore}{\kern0pt}elim{\isacharunderscore}{\kern0pt}one{\isacharunderscore}{\kern0pt}red{\isacharunderscore}{\kern0pt}def{\isacharunderscore}{\kern0pt}set\ single{\isacharunderscore}{\kern0pt}elimination\isanewline
\ \ \ \ \isacommand{by}\isamarkupfalse%
\ blast\isanewline
\ \ \isacommand{hence}\isamarkupfalse%
\ step{\isacharunderscore}{\kern0pt}reduces{\isacharunderscore}{\kern0pt}defer{\isacharunderscore}{\kern0pt}set{\isacharcolon}{\kern0pt}\isanewline
\ \ \ \ {\isachardoublequoteopen}{\isacharparenleft}{\kern0pt}card\ {\isacharparenleft}{\kern0pt}defer\ acc\ A\ p{\isacharparenright}{\kern0pt}\ {\isachargreater}{\kern0pt}\ {\isadigit{1}}\ {\isasymand}\ finite{\isacharunderscore}{\kern0pt}profile\ A\ p\ {\isasymand}\ non{\isacharunderscore}{\kern0pt}electing\ acc{\isacharparenright}{\kern0pt}\ {\isasymlongrightarrow}\isanewline
\ \ \ \ \ \ \ \ defer\ {\isacharparenleft}{\kern0pt}acc\ {\isasymtriangleright}\ m{\isacharparenright}{\kern0pt}\ A\ p\ {\isasymsubset}\ defer\ acc\ A\ p{\isachardoublequoteclose}\isanewline
\ \ \ \ \isacommand{using}\isamarkupfalse%
\ non{\isacharunderscore}{\kern0pt}electing{\isacharunderscore}{\kern0pt}def\isanewline
\ \ \ \ \isacommand{by}\isamarkupfalse%
\ auto\isanewline
\ \ \isacommand{thus}\isamarkupfalse%
\ {\isacharquery}{\kern0pt}case\isanewline
\ \ \isacommand{proof}\isamarkupfalse%
\ cases\isanewline
\ \ \ \ \isacommand{assume}\isamarkupfalse%
\ term{\isacharunderscore}{\kern0pt}satisfied{\isacharcolon}{\kern0pt}\ {\isachardoublequoteopen}t\ {\isacharparenleft}{\kern0pt}acc\ A\ p{\isacharparenright}{\kern0pt}{\isachardoublequoteclose}\isanewline
\ \ \ \ \isacommand{have}\isamarkupfalse%
\ {\isachardoublequoteopen}card\ {\isacharparenleft}{\kern0pt}defer{\isacharunderscore}{\kern0pt}r\ {\isacharparenleft}{\kern0pt}loop{\isacharunderscore}{\kern0pt}comp{\isacharunderscore}{\kern0pt}helper\ acc\ m\ t\ A\ p{\isacharparenright}{\kern0pt}{\isacharparenright}{\kern0pt}\ {\isacharequal}{\kern0pt}\ x{\isachardoublequoteclose}\isanewline
\ \ \ \ \ \ \isacommand{using}\isamarkupfalse%
\ loop{\isacharunderscore}{\kern0pt}comp{\isacharunderscore}{\kern0pt}helper{\isachardot}{\kern0pt}simps{\isacharparenleft}{\kern0pt}{\isadigit{1}}{\isacharparenright}{\kern0pt}\ term{\isacharunderscore}{\kern0pt}satisfied\ terminate{\isacharunderscore}{\kern0pt}if{\isacharunderscore}{\kern0pt}n{\isacharunderscore}{\kern0pt}left\isanewline
\ \ \ \ \ \ \isacommand{by}\isamarkupfalse%
\ metis\isanewline
\ \ \ \ \isacommand{thus}\isamarkupfalse%
\ {\isacharquery}{\kern0pt}case\isanewline
\ \ \ \ \ \ \isacommand{by}\isamarkupfalse%
\ blast\isanewline
\ \ \isacommand{next}\isamarkupfalse%
\isanewline
\ \ \ \ \isacommand{assume}\isamarkupfalse%
\ term{\isacharunderscore}{\kern0pt}not{\isacharunderscore}{\kern0pt}satisfied{\isacharcolon}{\kern0pt}\ {\isachardoublequoteopen}{\isasymnot}{\isacharparenleft}{\kern0pt}t\ {\isacharparenleft}{\kern0pt}acc\ A\ p{\isacharparenright}{\kern0pt}{\isacharparenright}{\kern0pt}{\isachardoublequoteclose}\isanewline
\ \ \ \ \isacommand{hence}\isamarkupfalse%
\ card{\isacharunderscore}{\kern0pt}not{\isacharunderscore}{\kern0pt}eq{\isacharunderscore}{\kern0pt}x{\isacharcolon}{\kern0pt}\ {\isachardoublequoteopen}card\ {\isacharparenleft}{\kern0pt}defer\ acc\ A\ p{\isacharparenright}{\kern0pt}\ {\isasymnoteq}\ x{\isachardoublequoteclose}\isanewline
\ \ \ \ \ \ \isacommand{by}\isamarkupfalse%
\ {\isacharparenleft}{\kern0pt}simp\ add{\isacharcolon}{\kern0pt}\ terminate{\isacharunderscore}{\kern0pt}if{\isacharunderscore}{\kern0pt}n{\isacharunderscore}{\kern0pt}left{\isacharparenright}{\kern0pt}\isanewline
\ \ \ \ \isacommand{have}\isamarkupfalse%
\ rec{\isacharunderscore}{\kern0pt}step{\isacharcolon}{\kern0pt}\isanewline
\ \ \ \ \ \ {\isachardoublequoteopen}{\isacharparenleft}{\kern0pt}card\ {\isacharparenleft}{\kern0pt}defer\ acc\ A\ p{\isacharparenright}{\kern0pt}\ {\isachargreater}{\kern0pt}\ {\isadigit{1}}\ {\isasymand}\ finite{\isacharunderscore}{\kern0pt}profile\ A\ p\ {\isasymand}\ non{\isacharunderscore}{\kern0pt}electing\ acc{\isacharparenright}{\kern0pt}\ {\isasymlongrightarrow}\isanewline
\ \ \ \ \ \ \ \ \ \ loop{\isacharunderscore}{\kern0pt}comp{\isacharunderscore}{\kern0pt}helper\ acc\ m\ t\ A\ p\ {\isacharequal}{\kern0pt}\isanewline
\ \ \ \ \ \ \ \ \ \ \ \ \ \ loop{\isacharunderscore}{\kern0pt}comp{\isacharunderscore}{\kern0pt}helper\ {\isacharparenleft}{\kern0pt}acc\ {\isasymtriangleright}\ m{\isacharparenright}{\kern0pt}\ m\ t\ A\ p{\isachardoublequoteclose}\ \isanewline
\ \ \ \ \ \ \isacommand{using}\isamarkupfalse%
\ loop{\isacharunderscore}{\kern0pt}comp{\isacharunderscore}{\kern0pt}helper{\isachardot}{\kern0pt}simps{\isacharparenleft}{\kern0pt}{\isadigit{2}}{\isacharparenright}{\kern0pt}\ non{\isacharunderscore}{\kern0pt}electing{\isacharunderscore}{\kern0pt}def\ def{\isacharunderscore}{\kern0pt}presv{\isacharunderscore}{\kern0pt}fin{\isacharunderscore}{\kern0pt}prof\isanewline
\ \ \ \ \ \ \ \ \ \ \ \ step{\isacharunderscore}{\kern0pt}reduces{\isacharunderscore}{\kern0pt}defer{\isacharunderscore}{\kern0pt}set\ term{\isacharunderscore}{\kern0pt}not{\isacharunderscore}{\kern0pt}satisfied\isanewline
\ \ \ \ \ \ \isacommand{by}\isamarkupfalse%
\ metis\isanewline
\ \ \ \ \isacommand{thus}\isamarkupfalse%
\ {\isacharquery}{\kern0pt}case\isanewline
\ \ \ \ \isacommand{proof}\isamarkupfalse%
\ cases\isanewline
\ \ \ \ \ \ \isacommand{assume}\isamarkupfalse%
\ card{\isacharunderscore}{\kern0pt}too{\isacharunderscore}{\kern0pt}small{\isacharcolon}{\kern0pt}\ {\isachardoublequoteopen}card\ {\isacharparenleft}{\kern0pt}defer\ acc\ A\ p{\isacharparenright}{\kern0pt}\ {\isacharless}{\kern0pt}\ x{\isachardoublequoteclose}\isanewline
\ \ \ \ \ \ \isacommand{thus}\isamarkupfalse%
\ {\isacharquery}{\kern0pt}thesis\isanewline
\ \ \ \ \ \ \ \ \isacommand{using}\isamarkupfalse%
\ not{\isacharunderscore}{\kern0pt}le\isanewline
\ \ \ \ \ \ \ \ \isacommand{by}\isamarkupfalse%
\ blast\isanewline
\ \ \ \ \isacommand{next}\isamarkupfalse%
\isanewline
\ \ \ \ \ \ \isacommand{assume}\isamarkupfalse%
\ old{\isacharunderscore}{\kern0pt}card{\isacharunderscore}{\kern0pt}at{\isacharunderscore}{\kern0pt}least{\isacharunderscore}{\kern0pt}x{\isacharcolon}{\kern0pt}\ {\isachardoublequoteopen}{\isasymnot}{\isacharparenleft}{\kern0pt}card\ {\isacharparenleft}{\kern0pt}defer\ acc\ A\ p{\isacharparenright}{\kern0pt}\ {\isacharless}{\kern0pt}\ x{\isacharparenright}{\kern0pt}{\isachardoublequoteclose}\isanewline
\ \ \ \ \ \ \isacommand{obtain}\isamarkupfalse%
\ i\ \isakeyword{where}\ i{\isacharunderscore}{\kern0pt}is{\isacharunderscore}{\kern0pt}new{\isacharunderscore}{\kern0pt}card{\isacharcolon}{\kern0pt}\ {\isachardoublequoteopen}i\ {\isacharequal}{\kern0pt}\ card\ {\isacharparenleft}{\kern0pt}defer\ {\isacharparenleft}{\kern0pt}acc\ {\isasymtriangleright}\ m{\isacharparenright}{\kern0pt}\ A\ p{\isacharparenright}{\kern0pt}{\isachardoublequoteclose}\isanewline
\ \ \ \ \ \ \ \ \isacommand{by}\isamarkupfalse%
\ blast\isanewline
\ \ \ \ \ \ \isacommand{with}\isamarkupfalse%
\ card{\isacharunderscore}{\kern0pt}not{\isacharunderscore}{\kern0pt}eq{\isacharunderscore}{\kern0pt}x\ \isacommand{have}\isamarkupfalse%
\ card{\isacharunderscore}{\kern0pt}too{\isacharunderscore}{\kern0pt}big{\isacharcolon}{\kern0pt}\isanewline
\ \ \ \ \ \ \ \ {\isachardoublequoteopen}card\ {\isacharparenleft}{\kern0pt}defer\ acc\ A\ p{\isacharparenright}{\kern0pt}\ {\isachargreater}{\kern0pt}\ x{\isachardoublequoteclose}\isanewline
\ \ \ \ \ \ \ \ \isacommand{using}\isamarkupfalse%
\ nat{\isacharunderscore}{\kern0pt}neq{\isacharunderscore}{\kern0pt}iff\ old{\isacharunderscore}{\kern0pt}card{\isacharunderscore}{\kern0pt}at{\isacharunderscore}{\kern0pt}least{\isacharunderscore}{\kern0pt}x\isanewline
\ \ \ \ \ \ \ \ \isacommand{by}\isamarkupfalse%
\ blast\isanewline
\ \ \ \ \ \ \isacommand{hence}\isamarkupfalse%
\ enough{\isacharunderscore}{\kern0pt}leftover{\isacharcolon}{\kern0pt}\ {\isachardoublequoteopen}card\ {\isacharparenleft}{\kern0pt}defer\ acc\ A\ p{\isacharparenright}{\kern0pt}\ {\isachargreater}{\kern0pt}\ {\isadigit{1}}{\isachardoublequoteclose}\isanewline
\ \ \ \ \ \ \ \ \isacommand{using}\isamarkupfalse%
\ x{\isacharunderscore}{\kern0pt}greater{\isacharunderscore}{\kern0pt}zero\isanewline
\ \ \ \ \ \ \ \ \isacommand{by}\isamarkupfalse%
\ auto\isanewline
\ \ \ \ \ \ \isacommand{have}\isamarkupfalse%
\ {\isachardoublequoteopen}electoral{\isacharunderscore}{\kern0pt}module\ acc\ {\isasymlongrightarrow}\ {\isacharparenleft}{\kern0pt}defer\ acc\ A\ p{\isacharparenright}{\kern0pt}\ {\isasymsubseteq}\ A{\isachardoublequoteclose}\isanewline
\ \ \ \ \ \ \ \ \isacommand{by}\isamarkupfalse%
\ {\isacharparenleft}{\kern0pt}simp\ add{\isacharcolon}{\kern0pt}\ defer{\isacharunderscore}{\kern0pt}in{\isacharunderscore}{\kern0pt}alts\ f{\isacharunderscore}{\kern0pt}prof{\isacharparenright}{\kern0pt}\isanewline
\ \ \ \ \ \ \isacommand{hence}\isamarkupfalse%
\ step{\isacharunderscore}{\kern0pt}profile{\isacharcolon}{\kern0pt}\isanewline
\ \ \ \ \ \ \ \ {\isachardoublequoteopen}electoral{\isacharunderscore}{\kern0pt}module\ acc\ {\isasymlongrightarrow}\isanewline
\ \ \ \ \ \ \ \ \ \ \ \ finite{\isacharunderscore}{\kern0pt}profile\ {\isacharparenleft}{\kern0pt}defer\ acc\ A\ p{\isacharparenright}{\kern0pt}\isanewline
\ \ \ \ \ \ \ \ \ \ \ \ \ \ {\isacharparenleft}{\kern0pt}limit{\isacharunderscore}{\kern0pt}profile\ {\isacharparenleft}{\kern0pt}defer\ acc\ A\ p{\isacharparenright}{\kern0pt}\ p{\isacharparenright}{\kern0pt}{\isachardoublequoteclose}\isanewline
\ \ \ \ \ \ \ \ \isacommand{using}\isamarkupfalse%
\ f{\isacharunderscore}{\kern0pt}prof\ limit{\isacharunderscore}{\kern0pt}profile{\isacharunderscore}{\kern0pt}sound\isanewline
\ \ \ \ \ \ \ \ \isacommand{by}\isamarkupfalse%
\ auto\isanewline
\ \ \ \ \ \ \isacommand{hence}\isamarkupfalse%
\isanewline
\ \ \ \ \ \ \ \ {\isachardoublequoteopen}electoral{\isacharunderscore}{\kern0pt}module\ acc\ {\isasymlongrightarrow}\isanewline
\ \ \ \ \ \ \ \ \ \ \ \ card\ {\isacharparenleft}{\kern0pt}defer\ m\ {\isacharparenleft}{\kern0pt}defer\ acc\ A\ p{\isacharparenright}{\kern0pt}\isanewline
\ \ \ \ \ \ \ \ \ \ \ \ \ \ {\isacharparenleft}{\kern0pt}limit{\isacharunderscore}{\kern0pt}profile\ {\isacharparenleft}{\kern0pt}defer\ acc\ A\ p{\isacharparenright}{\kern0pt}\ p{\isacharparenright}{\kern0pt}{\isacharparenright}{\kern0pt}\ {\isacharequal}{\kern0pt}\isanewline
\ \ \ \ \ \ \ \ \ \ \ \ \ \ \ \ card\ {\isacharparenleft}{\kern0pt}defer\ acc\ A\ p{\isacharparenright}{\kern0pt}\ {\isacharminus}{\kern0pt}\ {\isadigit{1}}{\isachardoublequoteclose}\isanewline
\ \ \ \ \ \ \ \ \isacommand{using}\isamarkupfalse%
\ non{\isacharunderscore}{\kern0pt}electing{\isacharunderscore}{\kern0pt}m\ single{\isacharunderscore}{\kern0pt}elimination\isanewline
\ \ \ \ \ \ \ \ \ \ \ \ \ \ single{\isacharunderscore}{\kern0pt}elim{\isacharunderscore}{\kern0pt}decr{\isacharunderscore}{\kern0pt}def{\isacharunderscore}{\kern0pt}card{\isadigit{2}}\ enough{\isacharunderscore}{\kern0pt}leftover\isanewline
\ \ \ \ \ \ \ \ \isacommand{by}\isamarkupfalse%
\ blast\isanewline
\ \ \ \ \ \ \isacommand{hence}\isamarkupfalse%
\ {\isachardoublequoteopen}electoral{\isacharunderscore}{\kern0pt}module\ acc\ {\isasymlongrightarrow}\ i\ {\isacharequal}{\kern0pt}\ card\ {\isacharparenleft}{\kern0pt}defer\ acc\ A\ p{\isacharparenright}{\kern0pt}\ {\isacharminus}{\kern0pt}\ {\isadigit{1}}{\isachardoublequoteclose}\isanewline
\ \ \ \ \ \ \ \ \isacommand{using}\isamarkupfalse%
\ sequential{\isacharunderscore}{\kern0pt}composition{\isachardot}{\kern0pt}simps\ snd{\isacharunderscore}{\kern0pt}conv\ i{\isacharunderscore}{\kern0pt}is{\isacharunderscore}{\kern0pt}new{\isacharunderscore}{\kern0pt}card\isanewline
\ \ \ \ \ \ \ \ \isacommand{by}\isamarkupfalse%
\ metis\isanewline
\ \ \ \ \ \ \isacommand{hence}\isamarkupfalse%
\ {\isachardoublequoteopen}electoral{\isacharunderscore}{\kern0pt}module\ acc\ {\isasymlongrightarrow}\ i\ {\isasymge}\ x{\isachardoublequoteclose}\isanewline
\ \ \ \ \ \ \ \ \isacommand{using}\isamarkupfalse%
\ card{\isacharunderscore}{\kern0pt}too{\isacharunderscore}{\kern0pt}big\isanewline
\ \ \ \ \ \ \ \ \isacommand{by}\isamarkupfalse%
\ linarith\isanewline
\ \ \ \ \ \ \isacommand{hence}\isamarkupfalse%
\ new{\isacharunderscore}{\kern0pt}card{\isacharunderscore}{\kern0pt}still{\isacharunderscore}{\kern0pt}big{\isacharunderscore}{\kern0pt}enough{\isacharcolon}{\kern0pt}\ {\isachardoublequoteopen}electoral{\isacharunderscore}{\kern0pt}module\ acc\ {\isasymlongrightarrow}\ x\ {\isasymle}\ i{\isachardoublequoteclose}\isanewline
\ \ \ \ \ \ \ \ \isacommand{by}\isamarkupfalse%
\ blast\isanewline
\ \ \ \ \ \ \isacommand{have}\isamarkupfalse%
\isanewline
\ \ \ \ \ \ \ \ {\isachardoublequoteopen}electoral{\isacharunderscore}{\kern0pt}module\ acc\ {\isasymand}\ electoral{\isacharunderscore}{\kern0pt}module\ m\ {\isasymlongrightarrow}\isanewline
\ \ \ \ \ \ \ \ \ \ \ \ defer\ {\isacharparenleft}{\kern0pt}acc\ {\isasymtriangleright}\ m{\isacharparenright}{\kern0pt}\ A\ p\ {\isasymsubseteq}\ defer\ acc\ A\ p{\isachardoublequoteclose}\isanewline
\ \ \ \ \ \ \ \ \isacommand{using}\isamarkupfalse%
\ enough{\isacharunderscore}{\kern0pt}leftover\ f{\isacharunderscore}{\kern0pt}prof\ subset\isanewline
\ \ \ \ \ \ \ \ \isacommand{by}\isamarkupfalse%
\ blast\isanewline
\ \ \ \ \ \ \isacommand{hence}\isamarkupfalse%
\isanewline
\ \ \ \ \ \ \ \ {\isachardoublequoteopen}electoral{\isacharunderscore}{\kern0pt}module\ acc\ {\isasymand}\ electoral{\isacharunderscore}{\kern0pt}module\ m\ {\isasymlongrightarrow}\isanewline
\ \ \ \ \ \ \ \ \ \ \ \ i\ {\isasymle}\ card\ {\isacharparenleft}{\kern0pt}defer\ acc\ A\ p{\isacharparenright}{\kern0pt}{\isachardoublequoteclose}\isanewline
\ \ \ \ \ \ \ \ \isacommand{using}\isamarkupfalse%
\ card{\isacharunderscore}{\kern0pt}mono\ i{\isacharunderscore}{\kern0pt}is{\isacharunderscore}{\kern0pt}new{\isacharunderscore}{\kern0pt}card\ step{\isacharunderscore}{\kern0pt}profile\isanewline
\ \ \ \ \ \ \ \ \isacommand{by}\isamarkupfalse%
\ blast\isanewline
\ \ \ \ \ \ \isacommand{hence}\isamarkupfalse%
\ i{\isacharunderscore}{\kern0pt}geq{\isacharunderscore}{\kern0pt}x{\isacharcolon}{\kern0pt}\isanewline
\ \ \ \ \ \ \ \ {\isachardoublequoteopen}electoral{\isacharunderscore}{\kern0pt}module\ acc\ {\isasymand}\ electoral{\isacharunderscore}{\kern0pt}module\ m\ {\isasymlongrightarrow}\ {\isacharparenleft}{\kern0pt}i\ {\isacharequal}{\kern0pt}\ x\ {\isasymor}\ i\ {\isachargreater}{\kern0pt}\ x{\isacharparenright}{\kern0pt}{\isachardoublequoteclose}\isanewline
\ \ \ \ \ \ \ \ \isacommand{using}\isamarkupfalse%
\ nat{\isacharunderscore}{\kern0pt}less{\isacharunderscore}{\kern0pt}le\ new{\isacharunderscore}{\kern0pt}card{\isacharunderscore}{\kern0pt}still{\isacharunderscore}{\kern0pt}big{\isacharunderscore}{\kern0pt}enough\isanewline
\ \ \ \ \ \ \ \ \isacommand{by}\isamarkupfalse%
\ blast\isanewline
\ \ \ \ \ \ \isacommand{thus}\isamarkupfalse%
\ {\isacharquery}{\kern0pt}thesis\isanewline
\ \ \ \ \ \ \isacommand{proof}\isamarkupfalse%
\ cases\isanewline
\ \ \ \ \ \ \ \ \isacommand{assume}\isamarkupfalse%
\ new{\isacharunderscore}{\kern0pt}card{\isacharunderscore}{\kern0pt}greater{\isacharunderscore}{\kern0pt}x{\isacharcolon}{\kern0pt}\ {\isachardoublequoteopen}electoral{\isacharunderscore}{\kern0pt}module\ acc\ {\isasymlongrightarrow}\ i\ {\isachargreater}{\kern0pt}\ x{\isachardoublequoteclose}\isanewline
\ \ \ \ \ \ \ \ \isacommand{hence}\isamarkupfalse%
\ {\isachardoublequoteopen}electoral{\isacharunderscore}{\kern0pt}module\ acc\ {\isasymlongrightarrow}\ {\isadigit{1}}\ {\isacharless}{\kern0pt}\ card\ {\isacharparenleft}{\kern0pt}defer\ {\isacharparenleft}{\kern0pt}acc\ {\isasymtriangleright}\ m{\isacharparenright}{\kern0pt}\ A\ p{\isacharparenright}{\kern0pt}{\isachardoublequoteclose}\isanewline
\ \ \ \ \ \ \ \ \ \ \isacommand{using}\isamarkupfalse%
\ x{\isacharunderscore}{\kern0pt}greater{\isacharunderscore}{\kern0pt}zero\ i{\isacharunderscore}{\kern0pt}is{\isacharunderscore}{\kern0pt}new{\isacharunderscore}{\kern0pt}card\isanewline
\ \ \ \ \ \ \ \ \ \ \isacommand{by}\isamarkupfalse%
\ linarith\isanewline
\ \ \ \ \ \ \ \ \isacommand{moreover}\isamarkupfalse%
\ \isacommand{have}\isamarkupfalse%
\ new{\isacharunderscore}{\kern0pt}card{\isacharunderscore}{\kern0pt}still{\isacharunderscore}{\kern0pt}big{\isacharunderscore}{\kern0pt}enough{\isadigit{2}}{\isacharcolon}{\kern0pt}\isanewline
\ \ \ \ \ \ \ \ \ \ {\isachardoublequoteopen}electoral{\isacharunderscore}{\kern0pt}module\ acc\ {\isasymlongrightarrow}\ x\ {\isasymle}\ i{\isachardoublequoteclose}\ \isanewline
\ \ \ \ \ \ \ \ \ \ \isacommand{using}\isamarkupfalse%
\ i{\isacharunderscore}{\kern0pt}is{\isacharunderscore}{\kern0pt}new{\isacharunderscore}{\kern0pt}card\ new{\isacharunderscore}{\kern0pt}card{\isacharunderscore}{\kern0pt}still{\isacharunderscore}{\kern0pt}big{\isacharunderscore}{\kern0pt}enough\isanewline
\ \ \ \ \ \ \ \ \ \ \isacommand{by}\isamarkupfalse%
\ blast\isanewline
\ \ \ \ \ \ \ \ \isacommand{moreover}\isamarkupfalse%
\ \isacommand{have}\isamarkupfalse%
\isanewline
\ \ \ \ \ \ \ \ \ \ {\isachardoublequoteopen}n\ {\isacharequal}{\kern0pt}\ card\ {\isacharparenleft}{\kern0pt}defer\ acc\ A\ p{\isacharparenright}{\kern0pt}\ {\isasymlongrightarrow}\isanewline
\ \ \ \ \ \ \ \ \ \ \ \ \ \ {\isacharparenleft}{\kern0pt}electoral{\isacharunderscore}{\kern0pt}module\ acc\ {\isasymlongrightarrow}\ i\ {\isacharless}{\kern0pt}\ n{\isacharparenright}{\kern0pt}{\isachardoublequoteclose}\ \isanewline
\ \ \ \ \ \ \ \ \ \ \isacommand{using}\isamarkupfalse%
\ subset\ step{\isacharunderscore}{\kern0pt}profile\ enough{\isacharunderscore}{\kern0pt}leftover\ f{\isacharunderscore}{\kern0pt}prof\ psubset{\isacharunderscore}{\kern0pt}card{\isacharunderscore}{\kern0pt}mono\isanewline
\ \ \ \ \ \ \ \ \ \ \ \ \ \ \ \ i{\isacharunderscore}{\kern0pt}is{\isacharunderscore}{\kern0pt}new{\isacharunderscore}{\kern0pt}card\isanewline
\ \ \ \ \ \ \ \ \ \ \isacommand{by}\isamarkupfalse%
\ blast\isanewline
\ \ \ \ \ \ \ \ \isacommand{moreover}\isamarkupfalse%
\ \isacommand{have}\isamarkupfalse%
\isanewline
\ \ \ \ \ \ \ \ \ \ {\isachardoublequoteopen}electoral{\isacharunderscore}{\kern0pt}module\ acc\ {\isasymlongrightarrow}\isanewline
\ \ \ \ \ \ \ \ \ \ \ \ \ \ electoral{\isacharunderscore}{\kern0pt}module\ {\isacharparenleft}{\kern0pt}acc\ {\isasymtriangleright}\ m{\isacharparenright}{\kern0pt}{\isachardoublequoteclose}\ \isanewline
\ \ \ \ \ \ \ \ \ \ \isacommand{using}\isamarkupfalse%
\ non{\isacharunderscore}{\kern0pt}electing{\isacharunderscore}{\kern0pt}def\ non{\isacharunderscore}{\kern0pt}electing{\isacharunderscore}{\kern0pt}m\ seq{\isacharunderscore}{\kern0pt}comp{\isacharunderscore}{\kern0pt}sound\isanewline
\ \ \ \ \ \ \ \ \ \ \isacommand{by}\isamarkupfalse%
\ blast\isanewline
\ \ \ \ \ \ \ \ \isacommand{moreover}\isamarkupfalse%
\ \isacommand{have}\isamarkupfalse%
\ non{\isacharunderscore}{\kern0pt}electing{\isacharunderscore}{\kern0pt}new{\isacharcolon}{\kern0pt}\isanewline
\ \ \ \ \ \ \ \ \ \ {\isachardoublequoteopen}non{\isacharunderscore}{\kern0pt}electing\ acc\ {\isasymlongrightarrow}\ non{\isacharunderscore}{\kern0pt}electing\ {\isacharparenleft}{\kern0pt}acc\ {\isasymtriangleright}\ m{\isacharparenright}{\kern0pt}{\isachardoublequoteclose}\isanewline
\ \ \ \ \ \ \ \ \ \ \isacommand{by}\isamarkupfalse%
\ {\isacharparenleft}{\kern0pt}simp\ add{\isacharcolon}{\kern0pt}\ non{\isacharunderscore}{\kern0pt}electing{\isacharunderscore}{\kern0pt}m{\isacharparenright}{\kern0pt}\isanewline
\ \ \ \ \ \ \ \ \isacommand{ultimately}\isamarkupfalse%
\ \isacommand{have}\isamarkupfalse%
\isanewline
\ \ \ \ \ \ \ \ \ \ {\isachardoublequoteopen}{\isacharparenleft}{\kern0pt}n\ {\isacharequal}{\kern0pt}\ card\ {\isacharparenleft}{\kern0pt}defer\ acc\ A\ p{\isacharparenright}{\kern0pt}\ {\isasymand}\ non{\isacharunderscore}{\kern0pt}electing\ acc\ {\isasymand}\isanewline
\ \ \ \ \ \ \ \ \ \ \ \ \ \ electoral{\isacharunderscore}{\kern0pt}module\ acc{\isacharparenright}{\kern0pt}\ {\isasymlongrightarrow}\isanewline
\ \ \ \ \ \ \ \ \ \ \ \ \ \ \ \ \ \ card\ {\isacharparenleft}{\kern0pt}defer\ {\isacharparenleft}{\kern0pt}loop{\isacharunderscore}{\kern0pt}comp{\isacharunderscore}{\kern0pt}helper\ {\isacharparenleft}{\kern0pt}acc\ {\isasymtriangleright}\ m{\isacharparenright}{\kern0pt}\ m\ t{\isacharparenright}{\kern0pt}\ A\ p{\isacharparenright}{\kern0pt}\ {\isacharequal}{\kern0pt}\ x{\isachardoublequoteclose}\isanewline
\ \ \ \ \ \ \ \ \ \ \isacommand{using}\isamarkupfalse%
\ less{\isachardot}{\kern0pt}hyps\ i{\isacharunderscore}{\kern0pt}is{\isacharunderscore}{\kern0pt}new{\isacharunderscore}{\kern0pt}card\ new{\isacharunderscore}{\kern0pt}card{\isacharunderscore}{\kern0pt}greater{\isacharunderscore}{\kern0pt}x\isanewline
\ \ \ \ \ \ \ \ \ \ \isacommand{by}\isamarkupfalse%
\ blast\isanewline
\ \ \ \ \ \ \ \ \isacommand{thus}\isamarkupfalse%
\ {\isacharquery}{\kern0pt}thesis\isanewline
\ \ \ \ \ \ \ \ \ \ \isacommand{using}\isamarkupfalse%
\ f{\isacharunderscore}{\kern0pt}prof\ rec{\isacharunderscore}{\kern0pt}step\ non{\isacharunderscore}{\kern0pt}electing{\isacharunderscore}{\kern0pt}def\isanewline
\ \ \ \ \ \ \ \ \ \ \isacommand{by}\isamarkupfalse%
\ metis\isanewline
\ \ \ \ \ \ \isacommand{next}\isamarkupfalse%
\isanewline
\ \ \ \ \ \ \ \ \isacommand{assume}\isamarkupfalse%
\ i{\isacharunderscore}{\kern0pt}not{\isacharunderscore}{\kern0pt}gt{\isacharunderscore}{\kern0pt}x{\isacharcolon}{\kern0pt}\ {\isachardoublequoteopen}{\isasymnot}{\isacharparenleft}{\kern0pt}electoral{\isacharunderscore}{\kern0pt}module\ acc\ {\isasymlongrightarrow}\ i\ {\isachargreater}{\kern0pt}\ x{\isacharparenright}{\kern0pt}{\isachardoublequoteclose}\isanewline
\ \ \ \ \ \ \ \ \isacommand{hence}\isamarkupfalse%
\ {\isachardoublequoteopen}electoral{\isacharunderscore}{\kern0pt}module\ acc\ {\isasymand}\ electoral{\isacharunderscore}{\kern0pt}module\ m\ {\isasymlongrightarrow}\ i\ {\isacharequal}{\kern0pt}\ x{\isachardoublequoteclose}\isanewline
\ \ \ \ \ \ \ \ \ \ \isacommand{using}\isamarkupfalse%
\ i{\isacharunderscore}{\kern0pt}geq{\isacharunderscore}{\kern0pt}x\isanewline
\ \ \ \ \ \ \ \ \ \ \isacommand{by}\isamarkupfalse%
\ blast\isanewline
\ \ \ \ \ \ \ \ \isacommand{hence}\isamarkupfalse%
\ {\isachardoublequoteopen}electoral{\isacharunderscore}{\kern0pt}module\ acc\ {\isasymand}\ electoral{\isacharunderscore}{\kern0pt}module\ m\ {\isasymlongrightarrow}\ t\ {\isacharparenleft}{\kern0pt}{\isacharparenleft}{\kern0pt}acc\ {\isasymtriangleright}\ m{\isacharparenright}{\kern0pt}\ A\ p{\isacharparenright}{\kern0pt}{\isachardoublequoteclose}\isanewline
\ \ \ \ \ \ \ \ \ \ \isacommand{using}\isamarkupfalse%
\ i{\isacharunderscore}{\kern0pt}is{\isacharunderscore}{\kern0pt}new{\isacharunderscore}{\kern0pt}card\ terminate{\isacharunderscore}{\kern0pt}if{\isacharunderscore}{\kern0pt}n{\isacharunderscore}{\kern0pt}left\isanewline
\ \ \ \ \ \ \ \ \ \ \isacommand{by}\isamarkupfalse%
\ blast\isanewline
\ \ \ \ \ \ \ \ \isacommand{hence}\isamarkupfalse%
\isanewline
\ \ \ \ \ \ \ \ \ \ {\isachardoublequoteopen}electoral{\isacharunderscore}{\kern0pt}module\ acc\ {\isasymand}\ electoral{\isacharunderscore}{\kern0pt}module\ m\ {\isasymlongrightarrow}\isanewline
\ \ \ \ \ \ \ \ \ \ \ \ \ \ card\ {\isacharparenleft}{\kern0pt}defer{\isacharunderscore}{\kern0pt}r\ {\isacharparenleft}{\kern0pt}loop{\isacharunderscore}{\kern0pt}comp{\isacharunderscore}{\kern0pt}helper\ {\isacharparenleft}{\kern0pt}acc\ {\isasymtriangleright}\ m{\isacharparenright}{\kern0pt}\ m\ t\ A\ p{\isacharparenright}{\kern0pt}{\isacharparenright}{\kern0pt}\ {\isacharequal}{\kern0pt}\ x{\isachardoublequoteclose}\isanewline
\ \ \ \ \ \ \ \ \ \ \isacommand{using}\isamarkupfalse%
\ loop{\isacharunderscore}{\kern0pt}comp{\isacharunderscore}{\kern0pt}helper{\isachardot}{\kern0pt}simps{\isacharparenleft}{\kern0pt}{\isadigit{1}}{\isacharparenright}{\kern0pt}\ terminate{\isacharunderscore}{\kern0pt}if{\isacharunderscore}{\kern0pt}n{\isacharunderscore}{\kern0pt}left\isanewline
\ \ \ \ \ \ \ \ \ \ \isacommand{by}\isamarkupfalse%
\ metis\isanewline
\ \ \ \ \ \ \ \ \isacommand{thus}\isamarkupfalse%
\ {\isacharquery}{\kern0pt}thesis\isanewline
\ \ \ \ \ \ \ \ \ \ \isacommand{using}\isamarkupfalse%
\ i{\isacharunderscore}{\kern0pt}not{\isacharunderscore}{\kern0pt}gt{\isacharunderscore}{\kern0pt}x\ dual{\isacharunderscore}{\kern0pt}order{\isachardot}{\kern0pt}strict{\isacharunderscore}{\kern0pt}iff{\isacharunderscore}{\kern0pt}order\ i{\isacharunderscore}{\kern0pt}is{\isacharunderscore}{\kern0pt}new{\isacharunderscore}{\kern0pt}card\isanewline
\ \ \ \ \ \ \ \ \ \ \ \ \ \ \ \ loop{\isacharunderscore}{\kern0pt}comp{\isacharunderscore}{\kern0pt}helper{\isachardot}{\kern0pt}simps{\isacharparenleft}{\kern0pt}{\isadigit{1}}{\isacharparenright}{\kern0pt}\ new{\isacharunderscore}{\kern0pt}card{\isacharunderscore}{\kern0pt}still{\isacharunderscore}{\kern0pt}big{\isacharunderscore}{\kern0pt}enough\isanewline
\ \ \ \ \ \ \ \ \ \ \ \ \ \ \ \ f{\isacharunderscore}{\kern0pt}prof\ rec{\isacharunderscore}{\kern0pt}step\ terminate{\isacharunderscore}{\kern0pt}if{\isacharunderscore}{\kern0pt}n{\isacharunderscore}{\kern0pt}left\isanewline
\ \ \ \ \ \ \ \ \ \ \isacommand{by}\isamarkupfalse%
\ metis\isanewline
\ \ \ \ \ \ \isacommand{qed}\isamarkupfalse%
\isanewline
\ \ \ \ \isacommand{qed}\isamarkupfalse%
\isanewline
\ \ \isacommand{qed}\isamarkupfalse%
\isanewline
\isacommand{qed}\isamarkupfalse%
%
\endisatagproof
{\isafoldproof}%
%
\isadelimproof
\isanewline
%
\endisadelimproof
\isanewline
\isacommand{lemma}\isamarkupfalse%
\ loop{\isacharunderscore}{\kern0pt}comp{\isacharunderscore}{\kern0pt}helper{\isacharunderscore}{\kern0pt}iter{\isacharunderscore}{\kern0pt}elim{\isacharunderscore}{\kern0pt}def{\isacharunderscore}{\kern0pt}n{\isacharcolon}{\kern0pt}\isanewline
\ \ \isakeyword{assumes}\isanewline
\ \ \ \ non{\isacharunderscore}{\kern0pt}electing{\isacharunderscore}{\kern0pt}m{\isacharcolon}{\kern0pt}\ {\isachardoublequoteopen}non{\isacharunderscore}{\kern0pt}electing\ m{\isachardoublequoteclose}\ \isakeyword{and}\isanewline
\ \ \ \ single{\isacharunderscore}{\kern0pt}elimination{\isacharcolon}{\kern0pt}\ {\isachardoublequoteopen}eliminates\ {\isadigit{1}}\ m{\isachardoublequoteclose}\ \isakeyword{and}\isanewline
\ \ \ \ terminate{\isacharunderscore}{\kern0pt}if{\isacharunderscore}{\kern0pt}n{\isacharunderscore}{\kern0pt}left{\isacharcolon}{\kern0pt}\ {\isachardoublequoteopen}{\isasymforall}\ r{\isachardot}{\kern0pt}\ {\isacharparenleft}{\kern0pt}{\isacharparenleft}{\kern0pt}t\ r{\isacharparenright}{\kern0pt}\ {\isasymlongleftrightarrow}\ {\isacharparenleft}{\kern0pt}card\ {\isacharparenleft}{\kern0pt}defer{\isacharunderscore}{\kern0pt}r\ r{\isacharparenright}{\kern0pt}\ {\isacharequal}{\kern0pt}\ x{\isacharparenright}{\kern0pt}{\isacharparenright}{\kern0pt}{\isachardoublequoteclose}\ \isakeyword{and}\isanewline
\ \ \ \ x{\isacharunderscore}{\kern0pt}greater{\isacharunderscore}{\kern0pt}zero{\isacharcolon}{\kern0pt}\ {\isachardoublequoteopen}x\ {\isachargreater}{\kern0pt}\ {\isadigit{0}}{\isachardoublequoteclose}\ \isakeyword{and}\isanewline
\ \ \ \ f{\isacharunderscore}{\kern0pt}prof{\isacharcolon}{\kern0pt}\ {\isachardoublequoteopen}finite{\isacharunderscore}{\kern0pt}profile\ A\ p{\isachardoublequoteclose}\ \isakeyword{and}\isanewline
\ \ \ \ acc{\isacharunderscore}{\kern0pt}defers{\isacharunderscore}{\kern0pt}enough{\isacharcolon}{\kern0pt}\ {\isachardoublequoteopen}card\ {\isacharparenleft}{\kern0pt}defer\ acc\ A\ p{\isacharparenright}{\kern0pt}\ {\isasymge}\ x{\isachardoublequoteclose}\ \isakeyword{and}\isanewline
\ \ \ \ non{\isacharunderscore}{\kern0pt}electing{\isacharunderscore}{\kern0pt}acc{\isacharcolon}{\kern0pt}\ {\isachardoublequoteopen}non{\isacharunderscore}{\kern0pt}electing\ acc{\isachardoublequoteclose}\isanewline
\ \ \isakeyword{shows}\ {\isachardoublequoteopen}card\ {\isacharparenleft}{\kern0pt}defer\ {\isacharparenleft}{\kern0pt}loop{\isacharunderscore}{\kern0pt}comp{\isacharunderscore}{\kern0pt}helper\ acc\ m\ t{\isacharparenright}{\kern0pt}\ A\ p{\isacharparenright}{\kern0pt}\ {\isacharequal}{\kern0pt}\ x{\isachardoublequoteclose}\isanewline
%
\isadelimproof
\ \ %
\endisadelimproof
%
\isatagproof
\isacommand{using}\isamarkupfalse%
\ acc{\isacharunderscore}{\kern0pt}defers{\isacharunderscore}{\kern0pt}enough\ gr{\isacharunderscore}{\kern0pt}implies{\isacharunderscore}{\kern0pt}not{\isadigit{0}}\ le{\isacharunderscore}{\kern0pt}neq{\isacharunderscore}{\kern0pt}implies{\isacharunderscore}{\kern0pt}less\isanewline
\ \ \ \ \ \ \ \ less{\isacharunderscore}{\kern0pt}one\ linorder{\isacharunderscore}{\kern0pt}neqE{\isacharunderscore}{\kern0pt}nat\ loop{\isacharunderscore}{\kern0pt}comp{\isacharunderscore}{\kern0pt}helper{\isachardot}{\kern0pt}simps{\isacharparenleft}{\kern0pt}{\isadigit{1}}{\isacharparenright}{\kern0pt}\isanewline
\ \ \ \ \ \ \ \ loop{\isacharunderscore}{\kern0pt}comp{\isacharunderscore}{\kern0pt}helper{\isacharunderscore}{\kern0pt}iter{\isacharunderscore}{\kern0pt}elim{\isacharunderscore}{\kern0pt}def{\isacharunderscore}{\kern0pt}n{\isacharunderscore}{\kern0pt}helper\ non{\isacharunderscore}{\kern0pt}electing{\isacharunderscore}{\kern0pt}acc\isanewline
\ \ \ \ \ \ \ \ non{\isacharunderscore}{\kern0pt}electing{\isacharunderscore}{\kern0pt}m\ f{\isacharunderscore}{\kern0pt}prof\ single{\isacharunderscore}{\kern0pt}elimination\ nat{\isacharunderscore}{\kern0pt}neq{\isacharunderscore}{\kern0pt}iff\isanewline
\ \ \ \ \ \ \ \ terminate{\isacharunderscore}{\kern0pt}if{\isacharunderscore}{\kern0pt}n{\isacharunderscore}{\kern0pt}left\ x{\isacharunderscore}{\kern0pt}greater{\isacharunderscore}{\kern0pt}zero\ less{\isacharunderscore}{\kern0pt}le\isanewline
\ \ \isacommand{by}\isamarkupfalse%
\ {\isacharparenleft}{\kern0pt}metis\ {\isacharparenleft}{\kern0pt}no{\isacharunderscore}{\kern0pt}types{\isacharcomma}{\kern0pt}\ lifting{\isacharparenright}{\kern0pt}{\isacharparenright}{\kern0pt}%
\endisatagproof
{\isafoldproof}%
%
\isadelimproof
\isanewline
%
\endisadelimproof
\isanewline
\isacommand{lemma}\isamarkupfalse%
\ iter{\isacharunderscore}{\kern0pt}elim{\isacharunderscore}{\kern0pt}def{\isacharunderscore}{\kern0pt}n{\isacharunderscore}{\kern0pt}helper{\isacharcolon}{\kern0pt}\isanewline
\ \ \isakeyword{assumes}\isanewline
\ \ \ \ non{\isacharunderscore}{\kern0pt}electing{\isacharunderscore}{\kern0pt}m{\isacharcolon}{\kern0pt}\ {\isachardoublequoteopen}non{\isacharunderscore}{\kern0pt}electing\ m{\isachardoublequoteclose}\ \isakeyword{and}\isanewline
\ \ \ \ single{\isacharunderscore}{\kern0pt}elimination{\isacharcolon}{\kern0pt}\ {\isachardoublequoteopen}eliminates\ {\isadigit{1}}\ m{\isachardoublequoteclose}\ \isakeyword{and}\isanewline
\ \ \ \ terminate{\isacharunderscore}{\kern0pt}if{\isacharunderscore}{\kern0pt}n{\isacharunderscore}{\kern0pt}left{\isacharcolon}{\kern0pt}\ {\isachardoublequoteopen}{\isasymforall}\ r{\isachardot}{\kern0pt}\ {\isacharparenleft}{\kern0pt}{\isacharparenleft}{\kern0pt}t\ r{\isacharparenright}{\kern0pt}\ {\isasymlongleftrightarrow}\ {\isacharparenleft}{\kern0pt}card\ {\isacharparenleft}{\kern0pt}defer{\isacharunderscore}{\kern0pt}r\ r{\isacharparenright}{\kern0pt}\ {\isacharequal}{\kern0pt}\ x{\isacharparenright}{\kern0pt}{\isacharparenright}{\kern0pt}{\isachardoublequoteclose}\ \isakeyword{and}\isanewline
\ \ \ \ x{\isacharunderscore}{\kern0pt}greater{\isacharunderscore}{\kern0pt}zero{\isacharcolon}{\kern0pt}\ {\isachardoublequoteopen}x\ {\isachargreater}{\kern0pt}\ {\isadigit{0}}{\isachardoublequoteclose}\ \isakeyword{and}\isanewline
\ \ \ \ f{\isacharunderscore}{\kern0pt}prof{\isacharcolon}{\kern0pt}\ {\isachardoublequoteopen}finite{\isacharunderscore}{\kern0pt}profile\ A\ p{\isachardoublequoteclose}\ \isakeyword{and}\isanewline
\ \ \ \ enough{\isacharunderscore}{\kern0pt}alternatives{\isacharcolon}{\kern0pt}\ {\isachardoublequoteopen}card\ A\ {\isasymge}\ x{\isachardoublequoteclose}\isanewline
\ \ \isakeyword{shows}\ {\isachardoublequoteopen}card\ {\isacharparenleft}{\kern0pt}defer\ {\isacharparenleft}{\kern0pt}m\ {\isasymcirclearrowleft}\isactrlsub t{\isacharparenright}{\kern0pt}\ A\ p{\isacharparenright}{\kern0pt}\ {\isacharequal}{\kern0pt}\ x{\isachardoublequoteclose}\isanewline
%
\isadelimproof
%
\endisadelimproof
%
\isatagproof
\isacommand{proof}\isamarkupfalse%
\ cases\isanewline
\ \ \isacommand{assume}\isamarkupfalse%
\ {\isachardoublequoteopen}card\ A\ {\isacharequal}{\kern0pt}\ x{\isachardoublequoteclose}\isanewline
\ \ \isacommand{thus}\isamarkupfalse%
\ {\isacharquery}{\kern0pt}thesis\isanewline
\ \ \ \ \isacommand{by}\isamarkupfalse%
\ {\isacharparenleft}{\kern0pt}simp\ add{\isacharcolon}{\kern0pt}\ terminate{\isacharunderscore}{\kern0pt}if{\isacharunderscore}{\kern0pt}n{\isacharunderscore}{\kern0pt}left{\isacharparenright}{\kern0pt}\isanewline
\isacommand{next}\isamarkupfalse%
\isanewline
\ \ \isacommand{assume}\isamarkupfalse%
\ card{\isacharunderscore}{\kern0pt}not{\isacharunderscore}{\kern0pt}x{\isacharcolon}{\kern0pt}\ {\isachardoublequoteopen}{\isasymnot}\ card\ A\ {\isacharequal}{\kern0pt}\ x{\isachardoublequoteclose}\isanewline
\ \ \isacommand{thus}\isamarkupfalse%
\ {\isacharquery}{\kern0pt}thesis\isanewline
\ \ \isacommand{proof}\isamarkupfalse%
\ cases\isanewline
\ \ \ \ \isacommand{assume}\isamarkupfalse%
\ {\isachardoublequoteopen}card\ A\ {\isacharless}{\kern0pt}\ x{\isachardoublequoteclose}\isanewline
\ \ \ \ \isacommand{thus}\isamarkupfalse%
\ {\isacharquery}{\kern0pt}thesis\isanewline
\ \ \ \ \ \ \isacommand{using}\isamarkupfalse%
\ enough{\isacharunderscore}{\kern0pt}alternatives\ not{\isacharunderscore}{\kern0pt}le\isanewline
\ \ \ \ \ \ \isacommand{by}\isamarkupfalse%
\ blast\isanewline
\ \ \isacommand{next}\isamarkupfalse%
\isanewline
\ \ \ \ \isacommand{assume}\isamarkupfalse%
\ {\isachardoublequoteopen}{\isasymnot}card\ A\ {\isacharless}{\kern0pt}\ x{\isachardoublequoteclose}\isanewline
\ \ \ \ \isacommand{hence}\isamarkupfalse%
\ card{\isacharunderscore}{\kern0pt}big{\isacharunderscore}{\kern0pt}enough{\isacharunderscore}{\kern0pt}A{\isacharcolon}{\kern0pt}\ {\isachardoublequoteopen}card\ A\ {\isachargreater}{\kern0pt}\ x{\isachardoublequoteclose}\isanewline
\ \ \ \ \ \ \isacommand{using}\isamarkupfalse%
\ card{\isacharunderscore}{\kern0pt}not{\isacharunderscore}{\kern0pt}x\isanewline
\ \ \ \ \ \ \isacommand{by}\isamarkupfalse%
\ linarith\isanewline
\ \ \ \ \isacommand{hence}\isamarkupfalse%
\ card{\isacharunderscore}{\kern0pt}m{\isacharcolon}{\kern0pt}\ {\isachardoublequoteopen}card\ {\isacharparenleft}{\kern0pt}defer\ m\ A\ p{\isacharparenright}{\kern0pt}\ {\isacharequal}{\kern0pt}\ card\ A\ {\isacharminus}{\kern0pt}\ {\isadigit{1}}{\isachardoublequoteclose}\isanewline
\ \ \ \ \ \ \isacommand{using}\isamarkupfalse%
\ non{\isacharunderscore}{\kern0pt}electing{\isacharunderscore}{\kern0pt}m\ f{\isacharunderscore}{\kern0pt}prof\ single{\isacharunderscore}{\kern0pt}elimination\isanewline
\ \ \ \ \ \ \ \ \ \ \ \ single{\isacharunderscore}{\kern0pt}elim{\isacharunderscore}{\kern0pt}decr{\isacharunderscore}{\kern0pt}def{\isacharunderscore}{\kern0pt}card{\isadigit{2}}\ x{\isacharunderscore}{\kern0pt}greater{\isacharunderscore}{\kern0pt}zero\isanewline
\ \ \ \ \ \ \isacommand{by}\isamarkupfalse%
\ fastforce\isanewline
\ \ \ \ \isacommand{hence}\isamarkupfalse%
\ card{\isacharunderscore}{\kern0pt}big{\isacharunderscore}{\kern0pt}enough{\isacharunderscore}{\kern0pt}m{\isacharcolon}{\kern0pt}\ {\isachardoublequoteopen}card\ {\isacharparenleft}{\kern0pt}defer\ m\ A\ p{\isacharparenright}{\kern0pt}\ {\isasymge}\ x{\isachardoublequoteclose}\isanewline
\ \ \ \ \ \ \isacommand{using}\isamarkupfalse%
\ card{\isacharunderscore}{\kern0pt}big{\isacharunderscore}{\kern0pt}enough{\isacharunderscore}{\kern0pt}A\isanewline
\ \ \ \ \ \ \isacommand{by}\isamarkupfalse%
\ linarith\isanewline
\ \ \ \ \isacommand{hence}\isamarkupfalse%
\ {\isachardoublequoteopen}{\isacharparenleft}{\kern0pt}m\ {\isasymcirclearrowleft}\isactrlsub t{\isacharparenright}{\kern0pt}\ A\ p\ {\isacharequal}{\kern0pt}\ {\isacharparenleft}{\kern0pt}loop{\isacharunderscore}{\kern0pt}comp{\isacharunderscore}{\kern0pt}helper\ m\ m\ t{\isacharparenright}{\kern0pt}\ A\ p{\isachardoublequoteclose}\isanewline
\ \ \ \ \ \ \isacommand{by}\isamarkupfalse%
\ {\isacharparenleft}{\kern0pt}simp\ add{\isacharcolon}{\kern0pt}\ card{\isacharunderscore}{\kern0pt}not{\isacharunderscore}{\kern0pt}x\ terminate{\isacharunderscore}{\kern0pt}if{\isacharunderscore}{\kern0pt}n{\isacharunderscore}{\kern0pt}left{\isacharparenright}{\kern0pt}\isanewline
\ \ \ \ \isacommand{thus}\isamarkupfalse%
\ {\isacharquery}{\kern0pt}thesis\isanewline
\ \ \ \ \ \ \isacommand{using}\isamarkupfalse%
\ card{\isacharunderscore}{\kern0pt}big{\isacharunderscore}{\kern0pt}enough{\isacharunderscore}{\kern0pt}m\ non{\isacharunderscore}{\kern0pt}electing{\isacharunderscore}{\kern0pt}m\ f{\isacharunderscore}{\kern0pt}prof\ single{\isacharunderscore}{\kern0pt}elimination\isanewline
\ \ \ \ \ \ \ \ \ \ \ \ terminate{\isacharunderscore}{\kern0pt}if{\isacharunderscore}{\kern0pt}n{\isacharunderscore}{\kern0pt}left\ x{\isacharunderscore}{\kern0pt}greater{\isacharunderscore}{\kern0pt}zero\ loop{\isacharunderscore}{\kern0pt}comp{\isacharunderscore}{\kern0pt}helper{\isacharunderscore}{\kern0pt}iter{\isacharunderscore}{\kern0pt}elim{\isacharunderscore}{\kern0pt}def{\isacharunderscore}{\kern0pt}n\isanewline
\ \ \ \ \ \ \isacommand{by}\isamarkupfalse%
\ metis\isanewline
\ \ \isacommand{qed}\isamarkupfalse%
\isanewline
\isacommand{qed}\isamarkupfalse%
%
\endisatagproof
{\isafoldproof}%
%
\isadelimproof
\isanewline
%
\endisadelimproof
\isanewline
\isacommand{theorem}\isamarkupfalse%
\ iter{\isacharunderscore}{\kern0pt}elim{\isacharunderscore}{\kern0pt}def{\isacharunderscore}{\kern0pt}n{\isacharbrackleft}{\kern0pt}simp{\isacharbrackright}{\kern0pt}{\isacharcolon}{\kern0pt}\isanewline
\ \ \isakeyword{assumes}\isanewline
\ \ \ \ non{\isacharunderscore}{\kern0pt}electing{\isacharunderscore}{\kern0pt}m{\isacharcolon}{\kern0pt}\ {\isachardoublequoteopen}non{\isacharunderscore}{\kern0pt}electing\ m{\isachardoublequoteclose}\ \isakeyword{and}\isanewline
\ \ \ \ single{\isacharunderscore}{\kern0pt}elimination{\isacharcolon}{\kern0pt}\ {\isachardoublequoteopen}eliminates\ {\isadigit{1}}\ m{\isachardoublequoteclose}\ \isakeyword{and}\isanewline
\ \ \ \ terminate{\isacharunderscore}{\kern0pt}if{\isacharunderscore}{\kern0pt}n{\isacharunderscore}{\kern0pt}left{\isacharcolon}{\kern0pt}\ {\isachardoublequoteopen}{\isasymforall}\ r{\isachardot}{\kern0pt}\ {\isacharparenleft}{\kern0pt}{\isacharparenleft}{\kern0pt}t\ r{\isacharparenright}{\kern0pt}\ {\isasymlongleftrightarrow}\ {\isacharparenleft}{\kern0pt}card\ {\isacharparenleft}{\kern0pt}defer{\isacharunderscore}{\kern0pt}r\ r{\isacharparenright}{\kern0pt}\ {\isacharequal}{\kern0pt}\ n{\isacharparenright}{\kern0pt}{\isacharparenright}{\kern0pt}{\isachardoublequoteclose}\ \isakeyword{and}\isanewline
\ \ \ \ x{\isacharunderscore}{\kern0pt}greater{\isacharunderscore}{\kern0pt}zero{\isacharcolon}{\kern0pt}\ {\isachardoublequoteopen}n\ {\isachargreater}{\kern0pt}\ {\isadigit{0}}{\isachardoublequoteclose}\isanewline
\ \ \isakeyword{shows}\ {\isachardoublequoteopen}defers\ n\ {\isacharparenleft}{\kern0pt}m\ {\isasymcirclearrowleft}\isactrlsub t{\isacharparenright}{\kern0pt}{\isachardoublequoteclose}\isanewline
%
\isadelimproof
%
\endisadelimproof
%
\isatagproof
\isacommand{proof}\isamarkupfalse%
\ {\isacharminus}{\kern0pt}\isanewline
\ \ \isacommand{have}\isamarkupfalse%
\isanewline
\ \ \ \ {\isachardoublequoteopen}{\isasymforall}\ A\ p{\isachardot}{\kern0pt}\ finite{\isacharunderscore}{\kern0pt}profile\ A\ p\ {\isasymand}\ card\ A\ {\isasymge}\ n\ {\isasymlongrightarrow}\isanewline
\ \ \ \ \ \ \ \ card\ {\isacharparenleft}{\kern0pt}defer\ {\isacharparenleft}{\kern0pt}m\ {\isasymcirclearrowleft}\isactrlsub t{\isacharparenright}{\kern0pt}\ A\ p{\isacharparenright}{\kern0pt}\ {\isacharequal}{\kern0pt}\ n{\isachardoublequoteclose}\isanewline
\ \ \ \ \isacommand{using}\isamarkupfalse%
\ iter{\isacharunderscore}{\kern0pt}elim{\isacharunderscore}{\kern0pt}def{\isacharunderscore}{\kern0pt}n{\isacharunderscore}{\kern0pt}helper\ non{\isacharunderscore}{\kern0pt}electing{\isacharunderscore}{\kern0pt}m\ single{\isacharunderscore}{\kern0pt}elimination\isanewline
\ \ \ \ \ \ \ \ \ \ terminate{\isacharunderscore}{\kern0pt}if{\isacharunderscore}{\kern0pt}n{\isacharunderscore}{\kern0pt}left\ x{\isacharunderscore}{\kern0pt}greater{\isacharunderscore}{\kern0pt}zero\isanewline
\ \ \ \ \isacommand{by}\isamarkupfalse%
\ blast\isanewline
\ \ \isacommand{moreover}\isamarkupfalse%
\ \isacommand{have}\isamarkupfalse%
\ {\isachardoublequoteopen}electoral{\isacharunderscore}{\kern0pt}module\ {\isacharparenleft}{\kern0pt}m\ {\isasymcirclearrowleft}\isactrlsub t{\isacharparenright}{\kern0pt}{\isachardoublequoteclose}\isanewline
\ \ \ \ \isacommand{using}\isamarkupfalse%
\ loop{\isacharunderscore}{\kern0pt}comp{\isacharunderscore}{\kern0pt}sound\ eliminates{\isacharunderscore}{\kern0pt}def\ single{\isacharunderscore}{\kern0pt}elimination\isanewline
\ \ \ \ \isacommand{by}\isamarkupfalse%
\ blast\isanewline
\ \ \isacommand{thus}\isamarkupfalse%
\ {\isacharquery}{\kern0pt}thesis\isanewline
\ \ \ \ \isacommand{by}\isamarkupfalse%
\ {\isacharparenleft}{\kern0pt}simp\ add{\isacharcolon}{\kern0pt}\ calculation\ defers{\isacharunderscore}{\kern0pt}def{\isacharparenright}{\kern0pt}\isanewline
\isacommand{qed}\isamarkupfalse%
%
\endisatagproof
{\isafoldproof}%
%
\isadelimproof
\isanewline
%
\endisadelimproof
\isanewline
\isanewline
\isacommand{theorem}\isamarkupfalse%
\ par{\isacharunderscore}{\kern0pt}comp{\isacharunderscore}{\kern0pt}elim{\isacharunderscore}{\kern0pt}one{\isacharbrackleft}{\kern0pt}simp{\isacharbrackright}{\kern0pt}{\isacharcolon}{\kern0pt}\isanewline
\ \ \isakeyword{assumes}\isanewline
\ \ \ \ defers{\isacharunderscore}{\kern0pt}m{\isacharunderscore}{\kern0pt}{\isadigit{1}}{\isacharcolon}{\kern0pt}\ {\isachardoublequoteopen}defers\ {\isadigit{1}}\ m{\isachardoublequoteclose}\ \isakeyword{and}\isanewline
\ \ \ \ non{\isacharunderscore}{\kern0pt}elec{\isacharunderscore}{\kern0pt}m{\isacharcolon}{\kern0pt}\ {\isachardoublequoteopen}non{\isacharunderscore}{\kern0pt}electing\ m{\isachardoublequoteclose}\ \isakeyword{and}\isanewline
\ \ \ \ rejec{\isacharunderscore}{\kern0pt}n{\isacharunderscore}{\kern0pt}{\isadigit{2}}{\isacharcolon}{\kern0pt}\ {\isachardoublequoteopen}rejects\ {\isadigit{2}}\ n{\isachardoublequoteclose}\ \isakeyword{and}\isanewline
\ \ \ \ disj{\isacharunderscore}{\kern0pt}comp{\isacharcolon}{\kern0pt}\ {\isachardoublequoteopen}disjoint{\isacharunderscore}{\kern0pt}compatibility\ m\ n{\isachardoublequoteclose}\isanewline
\ \ \isakeyword{shows}\ {\isachardoublequoteopen}eliminates\ {\isadigit{1}}\ {\isacharparenleft}{\kern0pt}m\ {\isasymparallel}\isactrlsub {\isasymup}\ n{\isacharparenright}{\kern0pt}{\isachardoublequoteclose}\isanewline
%
\isadelimproof
\ \ %
\endisadelimproof
%
\isatagproof
\isacommand{unfolding}\isamarkupfalse%
\ eliminates{\isacharunderscore}{\kern0pt}def\isanewline
\isacommand{proof}\isamarkupfalse%
\ {\isacharparenleft}{\kern0pt}safe{\isacharparenright}{\kern0pt}\isanewline
\ \ \isacommand{have}\isamarkupfalse%
\ electoral{\isacharunderscore}{\kern0pt}mod{\isacharunderscore}{\kern0pt}m{\isacharcolon}{\kern0pt}\ {\isachardoublequoteopen}electoral{\isacharunderscore}{\kern0pt}module\ m{\isachardoublequoteclose}\isanewline
\ \ \ \ \isacommand{using}\isamarkupfalse%
\ non{\isacharunderscore}{\kern0pt}elec{\isacharunderscore}{\kern0pt}m\isanewline
\ \ \ \ \isacommand{by}\isamarkupfalse%
\ {\isacharparenleft}{\kern0pt}simp\ add{\isacharcolon}{\kern0pt}\ non{\isacharunderscore}{\kern0pt}electing{\isacharunderscore}{\kern0pt}def{\isacharparenright}{\kern0pt}\isanewline
\ \ \isacommand{have}\isamarkupfalse%
\ electoral{\isacharunderscore}{\kern0pt}mod{\isacharunderscore}{\kern0pt}n{\isacharcolon}{\kern0pt}\ {\isachardoublequoteopen}electoral{\isacharunderscore}{\kern0pt}module\ n{\isachardoublequoteclose}\isanewline
\ \ \ \ \isacommand{using}\isamarkupfalse%
\ rejec{\isacharunderscore}{\kern0pt}n{\isacharunderscore}{\kern0pt}{\isadigit{2}}\isanewline
\ \ \ \ \isacommand{by}\isamarkupfalse%
\ {\isacharparenleft}{\kern0pt}simp\ add{\isacharcolon}{\kern0pt}\ rejects{\isacharunderscore}{\kern0pt}def{\isacharparenright}{\kern0pt}\isanewline
\ \ \isacommand{show}\isamarkupfalse%
\ {\isachardoublequoteopen}electoral{\isacharunderscore}{\kern0pt}module\ {\isacharparenleft}{\kern0pt}m\ {\isasymparallel}\isactrlsub {\isasymup}\ n{\isacharparenright}{\kern0pt}{\isachardoublequoteclose}\isanewline
\ \ \ \ \isacommand{using}\isamarkupfalse%
\ electoral{\isacharunderscore}{\kern0pt}mod{\isacharunderscore}{\kern0pt}m\ electoral{\isacharunderscore}{\kern0pt}mod{\isacharunderscore}{\kern0pt}n\isanewline
\ \ \ \ \isacommand{by}\isamarkupfalse%
\ simp\isanewline
\isacommand{next}\isamarkupfalse%
\isanewline
\ \ \isacommand{fix}\isamarkupfalse%
\isanewline
\ \ \ \ A\ {\isacharcolon}{\kern0pt}{\isacharcolon}{\kern0pt}\ {\isachardoublequoteopen}{\isacharprime}{\kern0pt}a\ set{\isachardoublequoteclose}\ \isakeyword{and}\isanewline
\ \ \ \ p\ {\isacharcolon}{\kern0pt}{\isacharcolon}{\kern0pt}\ {\isachardoublequoteopen}{\isacharprime}{\kern0pt}a\ Profile{\isachardoublequoteclose}\isanewline
\ \ \isacommand{assume}\isamarkupfalse%
\isanewline
\ \ \ \ min{\isacharunderscore}{\kern0pt}{\isadigit{2}}{\isacharunderscore}{\kern0pt}card{\isacharcolon}{\kern0pt}\ {\isachardoublequoteopen}{\isadigit{1}}\ {\isacharless}{\kern0pt}\ card\ A{\isachardoublequoteclose}\ \isakeyword{and}\isanewline
\ \ \ \ fin{\isacharunderscore}{\kern0pt}A{\isacharcolon}{\kern0pt}\ {\isachardoublequoteopen}finite\ A{\isachardoublequoteclose}\ \isakeyword{and}\isanewline
\ \ \ \ prof{\isacharunderscore}{\kern0pt}A{\isacharcolon}{\kern0pt}\ {\isachardoublequoteopen}profile\ A\ p{\isachardoublequoteclose}\isanewline
\ \ \isacommand{have}\isamarkupfalse%
\ card{\isacharunderscore}{\kern0pt}geq{\isacharunderscore}{\kern0pt}{\isadigit{1}}{\isacharcolon}{\kern0pt}\ {\isachardoublequoteopen}card\ A\ {\isasymge}\ {\isadigit{1}}{\isachardoublequoteclose}\isanewline
\ \ \ \ \isacommand{using}\isamarkupfalse%
\ min{\isacharunderscore}{\kern0pt}{\isadigit{2}}{\isacharunderscore}{\kern0pt}card\ dual{\isacharunderscore}{\kern0pt}order{\isachardot}{\kern0pt}strict{\isacharunderscore}{\kern0pt}trans{\isadigit{2}}\ less{\isacharunderscore}{\kern0pt}imp{\isacharunderscore}{\kern0pt}le{\isacharunderscore}{\kern0pt}nat\isanewline
\ \ \ \ \isacommand{by}\isamarkupfalse%
\ blast\isanewline
\ \ \isacommand{have}\isamarkupfalse%
\ module{\isacharcolon}{\kern0pt}\ {\isachardoublequoteopen}electoral{\isacharunderscore}{\kern0pt}module\ m{\isachardoublequoteclose}\isanewline
\ \ \ \ \isacommand{using}\isamarkupfalse%
\ non{\isacharunderscore}{\kern0pt}elec{\isacharunderscore}{\kern0pt}m\ non{\isacharunderscore}{\kern0pt}electing{\isacharunderscore}{\kern0pt}def\isanewline
\ \ \ \ \isacommand{by}\isamarkupfalse%
\ auto\isanewline
\ \ \isacommand{have}\isamarkupfalse%
\ elec{\isacharunderscore}{\kern0pt}card{\isacharunderscore}{\kern0pt}{\isadigit{0}}{\isacharcolon}{\kern0pt}\ {\isachardoublequoteopen}card\ {\isacharparenleft}{\kern0pt}elect\ m\ A\ p{\isacharparenright}{\kern0pt}\ {\isacharequal}{\kern0pt}\ {\isadigit{0}}{\isachardoublequoteclose}\isanewline
\ \ \ \ \isacommand{using}\isamarkupfalse%
\ fin{\isacharunderscore}{\kern0pt}A\ prof{\isacharunderscore}{\kern0pt}A\ non{\isacharunderscore}{\kern0pt}elec{\isacharunderscore}{\kern0pt}m\ card{\isacharunderscore}{\kern0pt}eq{\isacharunderscore}{\kern0pt}{\isadigit{0}}{\isacharunderscore}{\kern0pt}iff\ non{\isacharunderscore}{\kern0pt}electing{\isacharunderscore}{\kern0pt}def\isanewline
\ \ \ \ \isacommand{by}\isamarkupfalse%
\ metis\isanewline
\ \ \isacommand{moreover}\isamarkupfalse%
\isanewline
\ \ \isacommand{from}\isamarkupfalse%
\ card{\isacharunderscore}{\kern0pt}geq{\isacharunderscore}{\kern0pt}{\isadigit{1}}\ \isacommand{have}\isamarkupfalse%
\ def{\isacharunderscore}{\kern0pt}card{\isacharunderscore}{\kern0pt}{\isadigit{1}}{\isacharcolon}{\kern0pt}\isanewline
\ \ \ \ {\isachardoublequoteopen}card\ {\isacharparenleft}{\kern0pt}defer\ m\ A\ p{\isacharparenright}{\kern0pt}\ {\isacharequal}{\kern0pt}\ {\isadigit{1}}{\isachardoublequoteclose}\isanewline
\ \ \ \ \isacommand{using}\isamarkupfalse%
\ defers{\isacharunderscore}{\kern0pt}m{\isacharunderscore}{\kern0pt}{\isadigit{1}}\ module\ fin{\isacharunderscore}{\kern0pt}A\ prof{\isacharunderscore}{\kern0pt}A\isanewline
\ \ \ \ \isacommand{by}\isamarkupfalse%
\ {\isacharparenleft}{\kern0pt}simp\ add{\isacharcolon}{\kern0pt}\ defers{\isacharunderscore}{\kern0pt}def{\isacharparenright}{\kern0pt}\isanewline
\ \ \isacommand{ultimately}\isamarkupfalse%
\ \isacommand{have}\isamarkupfalse%
\ card{\isacharunderscore}{\kern0pt}reject{\isacharunderscore}{\kern0pt}m{\isacharcolon}{\kern0pt}\isanewline
\ \ \ \ {\isachardoublequoteopen}card\ {\isacharparenleft}{\kern0pt}reject\ m\ A\ p{\isacharparenright}{\kern0pt}\ {\isacharequal}{\kern0pt}\ card\ A\ {\isacharminus}{\kern0pt}\ {\isadigit{1}}{\isachardoublequoteclose}\isanewline
\ \ \isacommand{proof}\isamarkupfalse%
\ {\isacharminus}{\kern0pt}\isanewline
\ \ \ \ \isacommand{have}\isamarkupfalse%
\ {\isachardoublequoteopen}finite\ A{\isachardoublequoteclose}\isanewline
\ \ \ \ \ \ \isacommand{by}\isamarkupfalse%
\ {\isacharparenleft}{\kern0pt}simp\ add{\isacharcolon}{\kern0pt}\ fin{\isacharunderscore}{\kern0pt}A{\isacharparenright}{\kern0pt}\isanewline
\ \ \ \ \isacommand{moreover}\isamarkupfalse%
\ \isacommand{have}\isamarkupfalse%
\isanewline
\ \ \ \ \ \ {\isachardoublequoteopen}well{\isacharunderscore}{\kern0pt}formed\ A\isanewline
\ \ \ \ \ \ \ \ {\isacharparenleft}{\kern0pt}elect\ m\ A\ p{\isacharcomma}{\kern0pt}\ reject\ m\ A\ p{\isacharcomma}{\kern0pt}\ defer\ m\ A\ p{\isacharparenright}{\kern0pt}{\isachardoublequoteclose}\isanewline
\ \ \ \ \ \ \isacommand{using}\isamarkupfalse%
\ fin{\isacharunderscore}{\kern0pt}A\ prof{\isacharunderscore}{\kern0pt}A\ electoral{\isacharunderscore}{\kern0pt}module{\isacharunderscore}{\kern0pt}def\ module\isanewline
\ \ \ \ \ \ \isacommand{by}\isamarkupfalse%
\ auto\isanewline
\ \ \ \ \isacommand{ultimately}\isamarkupfalse%
\ \isacommand{have}\isamarkupfalse%
\isanewline
\ \ \ \ \ \ {\isachardoublequoteopen}card\ A\ {\isacharequal}{\kern0pt}\isanewline
\ \ \ \ \ \ \ \ card\ {\isacharparenleft}{\kern0pt}elect\ m\ A\ p{\isacharparenright}{\kern0pt}\ {\isacharplus}{\kern0pt}\ card\ {\isacharparenleft}{\kern0pt}reject\ m\ A\ p{\isacharparenright}{\kern0pt}\ {\isacharplus}{\kern0pt}\isanewline
\ \ \ \ \ \ \ \ \ \ card\ {\isacharparenleft}{\kern0pt}defer\ m\ A\ p{\isacharparenright}{\kern0pt}{\isachardoublequoteclose}\isanewline
\ \ \ \ \ \ \isacommand{using}\isamarkupfalse%
\ result{\isacharunderscore}{\kern0pt}count\isanewline
\ \ \ \ \ \ \isacommand{by}\isamarkupfalse%
\ blast\isanewline
\ \ \ \ \isacommand{thus}\isamarkupfalse%
\ {\isacharquery}{\kern0pt}thesis\isanewline
\ \ \ \ \ \ \isacommand{using}\isamarkupfalse%
\ def{\isacharunderscore}{\kern0pt}card{\isacharunderscore}{\kern0pt}{\isadigit{1}}\ elec{\isacharunderscore}{\kern0pt}card{\isacharunderscore}{\kern0pt}{\isadigit{0}}\isanewline
\ \ \ \ \ \ \isacommand{by}\isamarkupfalse%
\ simp\isanewline
\ \ \isacommand{qed}\isamarkupfalse%
\isanewline
\ \ \isacommand{have}\isamarkupfalse%
\ case{\isadigit{1}}{\isacharcolon}{\kern0pt}\ {\isachardoublequoteopen}card\ A\ {\isasymge}\ {\isadigit{2}}{\isachardoublequoteclose}\isanewline
\ \ \ \ \isacommand{using}\isamarkupfalse%
\ min{\isacharunderscore}{\kern0pt}{\isadigit{2}}{\isacharunderscore}{\kern0pt}card\isanewline
\ \ \ \ \isacommand{by}\isamarkupfalse%
\ auto\isanewline
\ \ \isacommand{from}\isamarkupfalse%
\ case{\isadigit{1}}\ \isacommand{have}\isamarkupfalse%
\ card{\isacharunderscore}{\kern0pt}reject{\isacharunderscore}{\kern0pt}n{\isacharcolon}{\kern0pt}\isanewline
\ \ \ \ {\isachardoublequoteopen}card\ {\isacharparenleft}{\kern0pt}reject\ n\ A\ p{\isacharparenright}{\kern0pt}\ {\isacharequal}{\kern0pt}\ {\isadigit{2}}{\isachardoublequoteclose}\isanewline
\ \ \ \ \isacommand{using}\isamarkupfalse%
\ fin{\isacharunderscore}{\kern0pt}A\ prof{\isacharunderscore}{\kern0pt}A\ rejec{\isacharunderscore}{\kern0pt}n{\isacharunderscore}{\kern0pt}{\isadigit{2}}\ rejects{\isacharunderscore}{\kern0pt}def\isanewline
\ \ \ \ \isacommand{by}\isamarkupfalse%
\ blast\isanewline
\ \ \isacommand{from}\isamarkupfalse%
\ card{\isacharunderscore}{\kern0pt}reject{\isacharunderscore}{\kern0pt}m\ card{\isacharunderscore}{\kern0pt}reject{\isacharunderscore}{\kern0pt}n\isanewline
\ \ \isacommand{have}\isamarkupfalse%
\isanewline
\ \ \ \ {\isachardoublequoteopen}card\ {\isacharparenleft}{\kern0pt}reject\ m\ A\ p{\isacharparenright}{\kern0pt}\ {\isacharplus}{\kern0pt}\ card\ {\isacharparenleft}{\kern0pt}reject\ n\ A\ p{\isacharparenright}{\kern0pt}\ {\isacharequal}{\kern0pt}\isanewline
\ \ \ \ \ \ card\ A\ {\isacharplus}{\kern0pt}\ {\isadigit{1}}{\isachardoublequoteclose}\isanewline
\ \ \ \ \isacommand{using}\isamarkupfalse%
\ card{\isacharunderscore}{\kern0pt}geq{\isacharunderscore}{\kern0pt}{\isadigit{1}}\isanewline
\ \ \ \ \isacommand{by}\isamarkupfalse%
\ linarith\isanewline
\ \ \isacommand{with}\isamarkupfalse%
\ disj{\isacharunderscore}{\kern0pt}comp\ prof{\isacharunderscore}{\kern0pt}A\ fin{\isacharunderscore}{\kern0pt}A\ card{\isacharunderscore}{\kern0pt}reject{\isacharunderscore}{\kern0pt}m\ card{\isacharunderscore}{\kern0pt}reject{\isacharunderscore}{\kern0pt}n\isanewline
\ \ \isacommand{show}\isamarkupfalse%
\isanewline
\ \ \ \ {\isachardoublequoteopen}card\ {\isacharparenleft}{\kern0pt}reject\ {\isacharparenleft}{\kern0pt}m\ {\isasymparallel}\isactrlsub {\isasymup}\ n{\isacharparenright}{\kern0pt}\ A\ p{\isacharparenright}{\kern0pt}\ {\isacharequal}{\kern0pt}\ {\isadigit{1}}{\isachardoublequoteclose}\isanewline
\ \ \ \ \isacommand{using}\isamarkupfalse%
\ par{\isacharunderscore}{\kern0pt}comp{\isacharunderscore}{\kern0pt}rej{\isacharunderscore}{\kern0pt}card\isanewline
\ \ \ \ \isacommand{by}\isamarkupfalse%
\ blast\isanewline
\isacommand{qed}\isamarkupfalse%
%
\endisatagproof
{\isafoldproof}%
%
\isadelimproof
\isanewline
%
\endisadelimproof
%
\isadelimtheory
\isanewline
%
\endisadelimtheory
%
\isatagtheory
\isacommand{end}\isamarkupfalse%
%
\endisatagtheory
{\isafoldtheory}%
%
\isadelimtheory
%
\endisadelimtheory
%
\end{isabellebody}%
\endinput
%:%file=~/Documents/Studies/VotingRuleGenerator/virage/src/test/resources/verifiedVotingRuleConstruction/theories/Compositional_Framework/Composition_Rules/Result_Rules.thy%:%
%:%10=1%:%
%:%11=1%:%
%:%12=2%:%
%:%13=3%:%
%:%14=4%:%
%:%15=5%:%
%:%16=6%:%
%:%17=7%:%
%:%22=7%:%
%:%25=8%:%
%:%26=9%:%
%:%27=9%:%
%:%28=10%:%
%:%29=11%:%
%:%32=12%:%
%:%36=12%:%
%:%37=12%:%
%:%38=13%:%
%:%39=14%:%
%:%40=15%:%
%:%41=15%:%
%:%46=15%:%
%:%49=16%:%
%:%50=17%:%
%:%51=18%:%
%:%52=18%:%
%:%53=19%:%
%:%54=20%:%
%:%55=21%:%
%:%56=22%:%
%:%63=23%:%
%:%64=23%:%
%:%65=24%:%
%:%66=24%:%
%:%67=25%:%
%:%68=26%:%
%:%69=27%:%
%:%70=27%:%
%:%71=28%:%
%:%72=28%:%
%:%73=28%:%
%:%74=29%:%
%:%75=30%:%
%:%76=30%:%
%:%77=31%:%
%:%78=31%:%
%:%79=31%:%
%:%80=32%:%
%:%81=33%:%
%:%82=33%:%
%:%83=34%:%
%:%84=35%:%
%:%85=35%:%
%:%86=36%:%
%:%87=36%:%
%:%88=36%:%
%:%89=37%:%
%:%90=38%:%
%:%91=38%:%
%:%92=39%:%
%:%93=39%:%
%:%94=40%:%
%:%95=41%:%
%:%96=42%:%
%:%97=42%:%
%:%98=43%:%
%:%99=43%:%
%:%100=44%:%
%:%101=44%:%
%:%102=44%:%
%:%103=45%:%
%:%104=46%:%
%:%105=46%:%
%:%106=47%:%
%:%107=47%:%
%:%108=48%:%
%:%109=48%:%
%:%110=49%:%
%:%111=49%:%
%:%112=50%:%
%:%113=50%:%
%:%114=51%:%
%:%115=51%:%
%:%116=52%:%
%:%117=53%:%
%:%118=54%:%
%:%119=55%:%
%:%120=55%:%
%:%121=56%:%
%:%127=56%:%
%:%130=57%:%
%:%131=58%:%
%:%132=58%:%
%:%133=59%:%
%:%134=60%:%
%:%135=61%:%
%:%136=62%:%
%:%143=63%:%
%:%144=63%:%
%:%145=64%:%
%:%146=64%:%
%:%147=65%:%
%:%148=66%:%
%:%149=67%:%
%:%150=68%:%
%:%158=76%:%
%:%159=77%:%
%:%160=77%:%
%:%161=78%:%
%:%162=78%:%
%:%163=79%:%
%:%164=79%:%
%:%165=80%:%
%:%166=81%:%
%:%167=82%:%
%:%168=82%:%
%:%169=83%:%
%:%170=83%:%
%:%171=84%:%
%:%171=88%:%
%:%172=89%:%
%:%173=89%:%
%:%174=90%:%
%:%175=90%:%
%:%176=91%:%
%:%177=92%:%
%:%178=93%:%
%:%179=94%:%
%:%185=100%:%
%:%186=101%:%
%:%187=101%:%
%:%188=102%:%
%:%189=102%:%
%:%190=103%:%
%:%191=103%:%
%:%192=104%:%
%:%193=105%:%
%:%194=106%:%
%:%195=107%:%
%:%196=108%:%
%:%197=109%:%
%:%198=109%:%
%:%199=110%:%
%:%200=110%:%
%:%201=111%:%
%:%202=111%:%
%:%203=112%:%
%:%204=113%:%
%:%205=113%:%
%:%206=114%:%
%:%207=114%:%
%:%208=115%:%
%:%209=115%:%
%:%210=116%:%
%:%211=117%:%
%:%212=117%:%
%:%213=118%:%
%:%214=118%:%
%:%215=119%:%
%:%216=119%:%
%:%217=120%:%
%:%223=126%:%
%:%224=127%:%
%:%225=127%:%
%:%226=128%:%
%:%227=129%:%
%:%228=130%:%
%:%229=130%:%
%:%230=131%:%
%:%231=131%:%
%:%232=132%:%
%:%233=132%:%
%:%234=133%:%
%:%235=133%:%
%:%236=134%:%
%:%237=135%:%
%:%238=135%:%
%:%239=136%:%
%:%240=137%:%
%:%241=138%:%
%:%242=138%:%
%:%243=139%:%
%:%244=139%:%
%:%245=140%:%
%:%246=140%:%
%:%247=141%:%
%:%248=141%:%
%:%249=141%:%
%:%250=142%:%
%:%251=142%:%
%:%252=143%:%
%:%253=143%:%
%:%254=144%:%
%:%255=144%:%
%:%256=145%:%
%:%262=145%:%
%:%265=146%:%
%:%266=151%:%
%:%267=152%:%
%:%268=152%:%
%:%269=153%:%
%:%270=154%:%
%:%271=155%:%
%:%278=156%:%
%:%279=156%:%
%:%280=157%:%
%:%281=157%:%
%:%282=158%:%
%:%283=159%:%
%:%284=160%:%
%:%285=160%:%
%:%286=161%:%
%:%287=161%:%
%:%288=162%:%
%:%289=162%:%
%:%290=163%:%
%:%291=164%:%
%:%292=165%:%
%:%293=165%:%
%:%294=166%:%
%:%295=167%:%
%:%296=167%:%
%:%297=168%:%
%:%298=168%:%
%:%299=169%:%
%:%300=169%:%
%:%301=170%:%
%:%302=171%:%
%:%303=172%:%
%:%304=172%:%
%:%305=173%:%
%:%311=173%:%
%:%314=174%:%
%:%315=178%:%
%:%316=179%:%
%:%317=179%:%
%:%318=180%:%
%:%319=181%:%
%:%320=182%:%
%:%321=183%:%
%:%322=184%:%
%:%325=185%:%
%:%329=185%:%
%:%330=185%:%
%:%331=186%:%
%:%332=186%:%
%:%333=187%:%
%:%334=187%:%
%:%335=188%:%
%:%336=188%:%
%:%337=189%:%
%:%338=189%:%
%:%339=190%:%
%:%340=190%:%
%:%341=191%:%
%:%342=191%:%
%:%343=192%:%
%:%344=192%:%
%:%345=193%:%
%:%346=193%:%
%:%347=194%:%
%:%348=194%:%
%:%349=195%:%
%:%350=195%:%
%:%351=196%:%
%:%352=196%:%
%:%353=197%:%
%:%354=197%:%
%:%355=198%:%
%:%356=198%:%
%:%357=199%:%
%:%358=199%:%
%:%359=200%:%
%:%360=200%:%
%:%361=201%:%
%:%362=202%:%
%:%363=203%:%
%:%364=204%:%
%:%365=204%:%
%:%366=205%:%
%:%367=206%:%
%:%368=207%:%
%:%369=208%:%
%:%370=208%:%
%:%371=209%:%
%:%372=209%:%
%:%373=210%:%
%:%374=210%:%
%:%375=211%:%
%:%376=211%:%
%:%377=212%:%
%:%378=212%:%
%:%379=213%:%
%:%380=213%:%
%:%381=214%:%
%:%382=214%:%
%:%383=215%:%
%:%384=216%:%
%:%385=217%:%
%:%386=217%:%
%:%387=218%:%
%:%388=219%:%
%:%389=220%:%
%:%390=220%:%
%:%391=221%:%
%:%392=221%:%
%:%393=222%:%
%:%394=222%:%
%:%395=223%:%
%:%396=223%:%
%:%397=224%:%
%:%398=224%:%
%:%399=225%:%
%:%400=225%:%
%:%401=226%:%
%:%402=227%:%
%:%403=227%:%
%:%404=228%:%
%:%410=228%:%
%:%413=229%:%
%:%414=233%:%
%:%415=234%:%
%:%416=234%:%
%:%417=235%:%
%:%418=236%:%
%:%419=237%:%
%:%420=238%:%
%:%423=239%:%
%:%427=239%:%
%:%428=239%:%
%:%429=240%:%
%:%430=240%:%
%:%435=240%:%
%:%438=241%:%
%:%439=242%:%
%:%440=243%:%
%:%441=243%:%
%:%442=244%:%
%:%443=245%:%
%:%444=246%:%
%:%445=247%:%
%:%448=248%:%
%:%452=248%:%
%:%453=248%:%
%:%454=249%:%
%:%455=250%:%
%:%456=250%:%
%:%461=250%:%
%:%464=251%:%
%:%465=252%:%
%:%466=252%:%
%:%467=253%:%
%:%468=254%:%
%:%469=255%:%
%:%470=256%:%
%:%471=257%:%
%:%472=258%:%
%:%479=259%:%
%:%480=259%:%
%:%481=260%:%
%:%482=260%:%
%:%483=261%:%
%:%484=261%:%
%:%485=262%:%
%:%486=262%:%
%:%487=263%:%
%:%488=264%:%
%:%489=265%:%
%:%490=265%:%
%:%491=266%:%
%:%497=266%:%
%:%500=267%:%
%:%501=268%:%
%:%502=269%:%
%:%503=269%:%
%:%504=270%:%
%:%505=271%:%
%:%508=272%:%
%:%512=272%:%
%:%513=272%:%
%:%514=273%:%
%:%515=273%:%
%:%516=274%:%
%:%517=274%:%
%:%518=275%:%
%:%519=275%:%
%:%520=276%:%
%:%521=276%:%
%:%522=277%:%
%:%523=277%:%
%:%524=278%:%
%:%525=278%:%
%:%526=279%:%
%:%527=280%:%
%:%528=281%:%
%:%529=282%:%
%:%530=282%:%
%:%531=283%:%
%:%532=284%:%
%:%533=285%:%
%:%534=286%:%
%:%535=286%:%
%:%536=287%:%
%:%537=287%:%
%:%538=288%:%
%:%539=289%:%
%:%540=290%:%
%:%541=290%:%
%:%542=291%:%
%:%548=291%:%
%:%551=292%:%
%:%552=296%:%
%:%553=297%:%
%:%554=297%:%
%:%555=298%:%
%:%556=299%:%
%:%559=300%:%
%:%563=300%:%
%:%564=300%:%
%:%565=301%:%
%:%566=301%:%
%:%567=302%:%
%:%568=302%:%
%:%569=303%:%
%:%570=303%:%
%:%571=304%:%
%:%572=304%:%
%:%573=305%:%
%:%574=305%:%
%:%575=306%:%
%:%576=306%:%
%:%577=307%:%
%:%578=308%:%
%:%579=309%:%
%:%580=310%:%
%:%581=310%:%
%:%582=311%:%
%:%583=312%:%
%:%584=313%:%
%:%585=314%:%
%:%586=315%:%
%:%587=315%:%
%:%588=315%:%
%:%589=316%:%
%:%590=317%:%
%:%591=317%:%
%:%592=318%:%
%:%593=319%:%
%:%594=319%:%
%:%595=320%:%
%:%596=320%:%
%:%597=321%:%
%:%598=321%:%
%:%599=322%:%
%:%600=323%:%
%:%601=323%:%
%:%602=324%:%
%:%608=324%:%
%:%611=325%:%
%:%612=331%:%
%:%613=332%:%
%:%614=332%:%
%:%615=333%:%
%:%616=334%:%
%:%617=335%:%
%:%618=336%:%
%:%619=337%:%
%:%622=338%:%
%:%626=338%:%
%:%627=338%:%
%:%628=339%:%
%:%629=339%:%
%:%630=340%:%
%:%631=340%:%
%:%632=341%:%
%:%633=341%:%
%:%634=342%:%
%:%635=342%:%
%:%636=343%:%
%:%637=343%:%
%:%638=344%:%
%:%639=344%:%
%:%640=345%:%
%:%641=345%:%
%:%642=346%:%
%:%643=346%:%
%:%644=347%:%
%:%645=347%:%
%:%646=348%:%
%:%647=348%:%
%:%648=349%:%
%:%649=349%:%
%:%650=350%:%
%:%651=350%:%
%:%652=351%:%
%:%653=352%:%
%:%654=353%:%
%:%655=353%:%
%:%656=354%:%
%:%657=355%:%
%:%658=356%:%
%:%659=357%:%
%:%660=357%:%
%:%661=357%:%
%:%662=358%:%
%:%663=359%:%
%:%664=359%:%
%:%665=360%:%
%:%666=360%:%
%:%667=360%:%
%:%668=361%:%
%:%669=362%:%
%:%670=362%:%
%:%671=363%:%
%:%672=363%:%
%:%673=364%:%
%:%674=364%:%
%:%675=365%:%
%:%676=366%:%
%:%677=366%:%
%:%678=367%:%
%:%679=368%:%
%:%680=369%:%
%:%681=369%:%
%:%682=370%:%
%:%683=370%:%
%:%684=371%:%
%:%685=371%:%
%:%686=372%:%
%:%687=373%:%
%:%688=373%:%
%:%689=374%:%
%:%690=374%:%
%:%691=375%:%
%:%692=376%:%
%:%693=376%:%
%:%694=377%:%
%:%695=378%:%
%:%696=378%:%
%:%697=379%:%
%:%698=379%:%
%:%699=380%:%
%:%700=381%:%
%:%701=382%:%
%:%702=382%:%
%:%703=383%:%
%:%704=384%:%
%:%705=384%:%
%:%706=385%:%
%:%707=385%:%
%:%708=386%:%
%:%713=391%:%
%:%714=392%:%
%:%715=392%:%
%:%716=393%:%
%:%717=393%:%
%:%718=394%:%
%:%719=394%:%
%:%720=395%:%
%:%721=396%:%
%:%722=397%:%
%:%723=397%:%
%:%724=398%:%
%:%725=399%:%
%:%726=400%:%
%:%727=400%:%
%:%728=401%:%
%:%729=401%:%
%:%730=402%:%
%:%731=402%:%
%:%732=403%:%
%:%733=403%:%
%:%734=404%:%
%:%740=404%:%
%:%743=405%:%
%:%744=406%:%
%:%745=406%:%
%:%746=407%:%
%:%747=408%:%
%:%748=409%:%
%:%749=410%:%
%:%750=411%:%
%:%751=412%:%
%:%752=413%:%
%:%753=414%:%
%:%755=416%:%
%:%762=417%:%
%:%763=417%:%
%:%764=418%:%
%:%765=418%:%
%:%766=419%:%
%:%767=419%:%
%:%768=420%:%
%:%769=421%:%
%:%770=422%:%
%:%771=422%:%
%:%772=423%:%
%:%773=423%:%
%:%774=424%:%
%:%775=424%:%
%:%776=425%:%
%:%777=426%:%
%:%778=427%:%
%:%779=427%:%
%:%780=428%:%
%:%781=428%:%
%:%782=429%:%
%:%783=429%:%
%:%784=430%:%
%:%785=430%:%
%:%786=431%:%
%:%787=431%:%
%:%788=432%:%
%:%789=432%:%
%:%790=433%:%
%:%791=433%:%
%:%792=434%:%
%:%793=434%:%
%:%794=435%:%
%:%795=435%:%
%:%796=436%:%
%:%797=436%:%
%:%798=437%:%
%:%799=437%:%
%:%800=438%:%
%:%801=438%:%
%:%802=439%:%
%:%803=439%:%
%:%804=440%:%
%:%805=440%:%
%:%806=441%:%
%:%807=441%:%
%:%808=442%:%
%:%810=444%:%
%:%811=445%:%
%:%812=445%:%
%:%813=446%:%
%:%814=447%:%
%:%815=447%:%
%:%816=448%:%
%:%817=448%:%
%:%818=449%:%
%:%819=449%:%
%:%820=450%:%
%:%821=450%:%
%:%822=451%:%
%:%823=451%:%
%:%824=452%:%
%:%825=452%:%
%:%826=453%:%
%:%827=453%:%
%:%828=454%:%
%:%829=454%:%
%:%830=455%:%
%:%831=455%:%
%:%832=456%:%
%:%833=456%:%
%:%834=457%:%
%:%835=457%:%
%:%836=458%:%
%:%837=458%:%
%:%838=458%:%
%:%839=459%:%
%:%840=460%:%
%:%841=460%:%
%:%842=461%:%
%:%843=461%:%
%:%844=462%:%
%:%845=462%:%
%:%846=463%:%
%:%847=463%:%
%:%848=464%:%
%:%849=464%:%
%:%850=465%:%
%:%851=465%:%
%:%852=466%:%
%:%853=466%:%
%:%854=467%:%
%:%855=467%:%
%:%856=468%:%
%:%858=470%:%
%:%859=471%:%
%:%860=471%:%
%:%861=472%:%
%:%862=472%:%
%:%863=473%:%
%:%864=473%:%
%:%865=474%:%
%:%868=477%:%
%:%869=478%:%
%:%870=478%:%
%:%871=479%:%
%:%872=480%:%
%:%873=480%:%
%:%874=481%:%
%:%875=481%:%
%:%876=482%:%
%:%877=482%:%
%:%878=483%:%
%:%879=483%:%
%:%880=484%:%
%:%881=484%:%
%:%882=485%:%
%:%883=485%:%
%:%884=486%:%
%:%885=486%:%
%:%886=487%:%
%:%887=487%:%
%:%888=488%:%
%:%889=488%:%
%:%890=489%:%
%:%891=489%:%
%:%892=490%:%
%:%893=491%:%
%:%894=492%:%
%:%895=492%:%
%:%896=493%:%
%:%897=493%:%
%:%898=494%:%
%:%899=494%:%
%:%900=495%:%
%:%901=496%:%
%:%902=497%:%
%:%903=497%:%
%:%904=498%:%
%:%905=498%:%
%:%906=499%:%
%:%907=499%:%
%:%908=500%:%
%:%909=501%:%
%:%910=501%:%
%:%911=502%:%
%:%912=502%:%
%:%913=503%:%
%:%914=503%:%
%:%915=504%:%
%:%916=504%:%
%:%917=505%:%
%:%918=505%:%
%:%919=506%:%
%:%920=506%:%
%:%921=507%:%
%:%922=507%:%
%:%923=508%:%
%:%924=508%:%
%:%925=509%:%
%:%926=509%:%
%:%927=509%:%
%:%928=510%:%
%:%929=511%:%
%:%930=511%:%
%:%931=512%:%
%:%932=512%:%
%:%933=513%:%
%:%934=513%:%
%:%935=513%:%
%:%936=514%:%
%:%937=515%:%
%:%938=516%:%
%:%939=516%:%
%:%940=517%:%
%:%941=518%:%
%:%942=518%:%
%:%943=519%:%
%:%944=519%:%
%:%945=519%:%
%:%946=520%:%
%:%947=521%:%
%:%948=522%:%
%:%949=522%:%
%:%950=523%:%
%:%951=523%:%
%:%952=524%:%
%:%953=524%:%
%:%954=524%:%
%:%955=525%:%
%:%956=526%:%
%:%957=526%:%
%:%958=527%:%
%:%959=527%:%
%:%960=527%:%
%:%961=528%:%
%:%963=530%:%
%:%964=531%:%
%:%965=531%:%
%:%966=532%:%
%:%967=532%:%
%:%968=533%:%
%:%969=533%:%
%:%970=534%:%
%:%971=534%:%
%:%972=535%:%
%:%973=535%:%
%:%974=536%:%
%:%975=536%:%
%:%976=537%:%
%:%977=537%:%
%:%978=538%:%
%:%979=538%:%
%:%980=539%:%
%:%981=539%:%
%:%982=540%:%
%:%983=540%:%
%:%984=541%:%
%:%985=541%:%
%:%986=542%:%
%:%987=542%:%
%:%988=543%:%
%:%989=543%:%
%:%990=544%:%
%:%991=544%:%
%:%992=545%:%
%:%993=546%:%
%:%994=547%:%
%:%995=547%:%
%:%996=548%:%
%:%997=548%:%
%:%998=549%:%
%:%999=549%:%
%:%1000=550%:%
%:%1001=550%:%
%:%1002=551%:%
%:%1003=552%:%
%:%1004=553%:%
%:%1005=553%:%
%:%1006=554%:%
%:%1007=554%:%
%:%1008=555%:%
%:%1009=555%:%
%:%1010=556%:%
%:%1011=556%:%
%:%1012=557%:%
%:%1018=557%:%
%:%1021=558%:%
%:%1022=559%:%
%:%1023=559%:%
%:%1024=560%:%
%:%1025=561%:%
%:%1026=562%:%
%:%1027=563%:%
%:%1028=564%:%
%:%1029=565%:%
%:%1030=566%:%
%:%1031=567%:%
%:%1032=568%:%
%:%1035=569%:%
%:%1039=569%:%
%:%1040=569%:%
%:%1041=570%:%
%:%1042=571%:%
%:%1043=572%:%
%:%1044=573%:%
%:%1045=574%:%
%:%1046=574%:%
%:%1051=574%:%
%:%1054=575%:%
%:%1055=576%:%
%:%1056=576%:%
%:%1057=577%:%
%:%1058=578%:%
%:%1059=579%:%
%:%1060=580%:%
%:%1061=581%:%
%:%1062=582%:%
%:%1063=583%:%
%:%1064=584%:%
%:%1071=585%:%
%:%1072=585%:%
%:%1073=586%:%
%:%1074=586%:%
%:%1075=587%:%
%:%1076=587%:%
%:%1077=588%:%
%:%1078=588%:%
%:%1079=589%:%
%:%1080=589%:%
%:%1081=590%:%
%:%1082=590%:%
%:%1083=591%:%
%:%1084=591%:%
%:%1085=592%:%
%:%1086=592%:%
%:%1087=593%:%
%:%1088=593%:%
%:%1089=594%:%
%:%1090=594%:%
%:%1091=595%:%
%:%1092=595%:%
%:%1093=596%:%
%:%1094=596%:%
%:%1095=597%:%
%:%1096=597%:%
%:%1097=598%:%
%:%1098=598%:%
%:%1099=599%:%
%:%1100=599%:%
%:%1101=600%:%
%:%1102=600%:%
%:%1103=601%:%
%:%1104=601%:%
%:%1105=602%:%
%:%1106=602%:%
%:%1107=603%:%
%:%1108=603%:%
%:%1109=604%:%
%:%1110=605%:%
%:%1111=605%:%
%:%1112=606%:%
%:%1113=606%:%
%:%1114=607%:%
%:%1115=607%:%
%:%1116=608%:%
%:%1117=608%:%
%:%1118=609%:%
%:%1119=609%:%
%:%1120=610%:%
%:%1121=610%:%
%:%1122=611%:%
%:%1123=611%:%
%:%1124=612%:%
%:%1125=612%:%
%:%1126=613%:%
%:%1127=614%:%
%:%1128=614%:%
%:%1129=615%:%
%:%1130=615%:%
%:%1131=616%:%
%:%1137=616%:%
%:%1140=617%:%
%:%1141=618%:%
%:%1142=618%:%
%:%1143=619%:%
%:%1144=620%:%
%:%1145=621%:%
%:%1146=622%:%
%:%1147=623%:%
%:%1148=624%:%
%:%1155=625%:%
%:%1156=625%:%
%:%1157=626%:%
%:%1158=626%:%
%:%1159=627%:%
%:%1160=628%:%
%:%1161=629%:%
%:%1162=629%:%
%:%1163=630%:%
%:%1164=631%:%
%:%1165=631%:%
%:%1166=632%:%
%:%1167=632%:%
%:%1168=632%:%
%:%1169=633%:%
%:%1170=633%:%
%:%1171=634%:%
%:%1172=634%:%
%:%1173=635%:%
%:%1174=635%:%
%:%1175=636%:%
%:%1176=636%:%
%:%1177=637%:%
%:%1183=637%:%
%:%1186=638%:%
%:%1187=644%:%
%:%1188=645%:%
%:%1189=645%:%
%:%1190=646%:%
%:%1191=647%:%
%:%1192=648%:%
%:%1193=649%:%
%:%1194=650%:%
%:%1195=651%:%
%:%1198=652%:%
%:%1202=652%:%
%:%1203=652%:%
%:%1204=653%:%
%:%1205=653%:%
%:%1206=654%:%
%:%1207=654%:%
%:%1208=655%:%
%:%1209=655%:%
%:%1210=656%:%
%:%1211=656%:%
%:%1212=657%:%
%:%1213=657%:%
%:%1214=658%:%
%:%1215=658%:%
%:%1216=659%:%
%:%1217=659%:%
%:%1218=660%:%
%:%1219=660%:%
%:%1220=661%:%
%:%1221=661%:%
%:%1222=662%:%
%:%1223=662%:%
%:%1224=663%:%
%:%1225=663%:%
%:%1226=664%:%
%:%1227=664%:%
%:%1228=665%:%
%:%1229=666%:%
%:%1230=667%:%
%:%1231=667%:%
%:%1232=668%:%
%:%1233=669%:%
%:%1234=670%:%
%:%1235=671%:%
%:%1236=671%:%
%:%1237=672%:%
%:%1238=672%:%
%:%1239=673%:%
%:%1240=673%:%
%:%1241=674%:%
%:%1242=674%:%
%:%1243=675%:%
%:%1244=675%:%
%:%1245=676%:%
%:%1246=676%:%
%:%1247=677%:%
%:%1248=677%:%
%:%1249=678%:%
%:%1250=678%:%
%:%1251=679%:%
%:%1252=679%:%
%:%1253=680%:%
%:%1254=680%:%
%:%1255=681%:%
%:%1256=681%:%
%:%1257=681%:%
%:%1258=682%:%
%:%1259=683%:%
%:%1260=683%:%
%:%1261=684%:%
%:%1262=684%:%
%:%1263=685%:%
%:%1264=685%:%
%:%1265=685%:%
%:%1266=686%:%
%:%1267=687%:%
%:%1268=687%:%
%:%1269=688%:%
%:%1270=688%:%
%:%1271=689%:%
%:%1272=689%:%
%:%1273=690%:%
%:%1274=690%:%
%:%1275=690%:%
%:%1276=691%:%
%:%1277=692%:%
%:%1278=693%:%
%:%1279=693%:%
%:%1280=694%:%
%:%1281=694%:%
%:%1282=695%:%
%:%1283=695%:%
%:%1284=695%:%
%:%1285=696%:%
%:%1287=698%:%
%:%1288=699%:%
%:%1289=699%:%
%:%1290=700%:%
%:%1291=700%:%
%:%1292=701%:%
%:%1293=701%:%
%:%1294=702%:%
%:%1295=702%:%
%:%1296=703%:%
%:%1297=703%:%
%:%1298=704%:%
%:%1299=704%:%
%:%1300=705%:%
%:%1301=705%:%
%:%1302=706%:%
%:%1303=706%:%
%:%1304=707%:%
%:%1305=707%:%
%:%1306=708%:%
%:%1307=708%:%
%:%1308=708%:%
%:%1309=709%:%
%:%1310=710%:%
%:%1311=710%:%
%:%1312=711%:%
%:%1313=711%:%
%:%1314=712%:%
%:%1315=712%:%
%:%1316=713%:%
%:%1317=713%:%
%:%1318=714%:%
%:%1319=715%:%
%:%1320=716%:%
%:%1321=716%:%
%:%1322=717%:%
%:%1323=717%:%
%:%1324=718%:%
%:%1325=718%:%
%:%1326=719%:%
%:%1327=719%:%
%:%1328=720%:%
%:%1329=721%:%
%:%1330=721%:%
%:%1331=722%:%
%:%1332=722%:%
%:%1333=723%:%
%:%1339=723%:%
%:%1344=724%:%
%:%1349=725%:%
%
\begin{isabellebody}%
\setisabellecontext{Monotonicity{\isacharunderscore}{\kern0pt}Facts}%
%
\isadelimtheory
%
\endisadelimtheory
%
\isatagtheory
\isacommand{theory}\isamarkupfalse%
\ Monotonicity{\isacharunderscore}{\kern0pt}Facts\isanewline
\ \ \isakeyword{imports}\ {\isachardoublequoteopen}{\isachardot}{\kern0pt}{\isachardot}{\kern0pt}{\isacharslash}{\kern0pt}Properties{\isacharslash}{\kern0pt}Monotonicity{\isacharunderscore}{\kern0pt}Properties{\isachardoublequoteclose}\isanewline
\ \ \ \ \ \ \ \ \ \ {\isachardoublequoteopen}{\isachardot}{\kern0pt}{\isachardot}{\kern0pt}{\isacharslash}{\kern0pt}Components{\isacharslash}{\kern0pt}Basic{\isacharunderscore}{\kern0pt}Modules{\isacharslash}{\kern0pt}Defer{\isacharunderscore}{\kern0pt}Module{\isachardoublequoteclose}\isanewline
\ \ \ \ \ \ \ \ \ \ {\isachardoublequoteopen}{\isachardot}{\kern0pt}{\isachardot}{\kern0pt}{\isacharslash}{\kern0pt}Components{\isacharslash}{\kern0pt}Basic{\isacharunderscore}{\kern0pt}Modules{\isacharslash}{\kern0pt}Drop{\isacharunderscore}{\kern0pt}Module{\isachardoublequoteclose}\isanewline
\ \ \ \ \ \ \ \ \ \ {\isachardoublequoteopen}{\isachardot}{\kern0pt}{\isachardot}{\kern0pt}{\isacharslash}{\kern0pt}Components{\isacharslash}{\kern0pt}Basic{\isacharunderscore}{\kern0pt}Modules{\isacharslash}{\kern0pt}Pass{\isacharunderscore}{\kern0pt}Module{\isachardoublequoteclose}\isanewline
\ \ \ \ \ \ \ \ \ \ {\isachardoublequoteopen}{\isachardot}{\kern0pt}{\isachardot}{\kern0pt}{\isacharslash}{\kern0pt}Components{\isacharslash}{\kern0pt}Basic{\isacharunderscore}{\kern0pt}Modules{\isacharslash}{\kern0pt}Plurality{\isacharunderscore}{\kern0pt}Module{\isachardoublequoteclose}\isanewline
\isanewline
\isakeyword{begin}%
\endisatagtheory
{\isafoldtheory}%
%
\isadelimtheory
\isanewline
%
\endisadelimtheory
\isanewline
\isacommand{theorem}\isamarkupfalse%
\ def{\isacharunderscore}{\kern0pt}mod{\isacharunderscore}{\kern0pt}def{\isacharunderscore}{\kern0pt}lift{\isacharunderscore}{\kern0pt}inv{\isacharcolon}{\kern0pt}\ {\isachardoublequoteopen}defer{\isacharunderscore}{\kern0pt}lift{\isacharunderscore}{\kern0pt}invariance\ defer{\isacharunderscore}{\kern0pt}module{\isachardoublequoteclose}\isanewline
%
\isadelimproof
\ \ %
\endisadelimproof
%
\isatagproof
\isacommand{unfolding}\isamarkupfalse%
\ defer{\isacharunderscore}{\kern0pt}lift{\isacharunderscore}{\kern0pt}invariance{\isacharunderscore}{\kern0pt}def\isanewline
\ \ \isacommand{by}\isamarkupfalse%
\ simp%
\endisatagproof
{\isafoldproof}%
%
\isadelimproof
\isanewline
%
\endisadelimproof
\isanewline
\isanewline
\isacommand{theorem}\isamarkupfalse%
\ drop{\isacharunderscore}{\kern0pt}mod{\isacharunderscore}{\kern0pt}def{\isacharunderscore}{\kern0pt}lift{\isacharunderscore}{\kern0pt}inv{\isacharbrackleft}{\kern0pt}simp{\isacharbrackright}{\kern0pt}{\isacharcolon}{\kern0pt}\isanewline
\ \ \isakeyword{assumes}\ order{\isacharcolon}{\kern0pt}\ {\isachardoublequoteopen}linear{\isacharunderscore}{\kern0pt}order\ r{\isachardoublequoteclose}\isanewline
\ \ \isakeyword{shows}\ {\isachardoublequoteopen}defer{\isacharunderscore}{\kern0pt}lift{\isacharunderscore}{\kern0pt}invariance\ {\isacharparenleft}{\kern0pt}drop{\isacharunderscore}{\kern0pt}module\ n\ r{\isacharparenright}{\kern0pt}{\isachardoublequoteclose}\isanewline
%
\isadelimproof
\ \ %
\endisadelimproof
%
\isatagproof
\isacommand{by}\isamarkupfalse%
\ {\isacharparenleft}{\kern0pt}simp\ add{\isacharcolon}{\kern0pt}\ order\ defer{\isacharunderscore}{\kern0pt}lift{\isacharunderscore}{\kern0pt}invariance{\isacharunderscore}{\kern0pt}def{\isacharparenright}{\kern0pt}%
\endisatagproof
{\isafoldproof}%
%
\isadelimproof
\isanewline
%
\endisadelimproof
\isanewline
\isanewline
\isacommand{theorem}\isamarkupfalse%
\ pass{\isacharunderscore}{\kern0pt}mod{\isacharunderscore}{\kern0pt}dl{\isacharunderscore}{\kern0pt}inv{\isacharbrackleft}{\kern0pt}simp{\isacharbrackright}{\kern0pt}{\isacharcolon}{\kern0pt}\isanewline
\ \ \isakeyword{assumes}\ order{\isacharcolon}{\kern0pt}\ {\isachardoublequoteopen}linear{\isacharunderscore}{\kern0pt}order\ r{\isachardoublequoteclose}\isanewline
\ \ \isakeyword{shows}\ {\isachardoublequoteopen}defer{\isacharunderscore}{\kern0pt}lift{\isacharunderscore}{\kern0pt}invariance\ {\isacharparenleft}{\kern0pt}pass{\isacharunderscore}{\kern0pt}module\ n\ r{\isacharparenright}{\kern0pt}{\isachardoublequoteclose}\isanewline
%
\isadelimproof
\ \ %
\endisadelimproof
%
\isatagproof
\isacommand{by}\isamarkupfalse%
\ {\isacharparenleft}{\kern0pt}simp\ add{\isacharcolon}{\kern0pt}\ order\ defer{\isacharunderscore}{\kern0pt}lift{\isacharunderscore}{\kern0pt}invariance{\isacharunderscore}{\kern0pt}def{\isacharparenright}{\kern0pt}%
\endisatagproof
{\isafoldproof}%
%
\isadelimproof
\isanewline
%
\endisadelimproof
\isanewline
\isacommand{lemma}\isamarkupfalse%
\ plurality{\isacharunderscore}{\kern0pt}inv{\isacharunderscore}{\kern0pt}mono{\isadigit{2}}{\isacharcolon}{\kern0pt}\ {\isachardoublequoteopen}{\isasymforall}A\ p\ q\ a{\isachardot}{\kern0pt}\isanewline
\ \ \ \ \ \ \ \ \ \ \ \ \ \ \ \ \ \ \ \ \ \ \ \ \ \ \ \ \ \ {\isacharparenleft}{\kern0pt}a\ {\isasymin}\ elect\ plurality\ A\ p\ {\isasymand}\ lifted\ A\ p\ q\ a{\isacharparenright}{\kern0pt}\ {\isasymlongrightarrow}\isanewline
\ \ \ \ \ \ \ \ \ \ \ \ \ \ \ \ \ \ \ \ \ \ \ \ \ \ \ \ \ \ \ \ {\isacharparenleft}{\kern0pt}elect\ plurality\ A\ q\ {\isacharequal}{\kern0pt}\ elect\ plurality\ A\ p\ {\isasymor}\isanewline
\ \ \ \ \ \ \ \ \ \ \ \ \ \ \ \ \ \ \ \ \ \ \ \ \ \ \ \ \ \ \ \ \ \ \ \ elect\ plurality\ A\ q\ {\isacharequal}{\kern0pt}\ {\isacharbraceleft}{\kern0pt}a{\isacharbraceright}{\kern0pt}{\isacharparenright}{\kern0pt}{\isachardoublequoteclose}\isanewline
%
\isadelimproof
%
\endisadelimproof
%
\isatagproof
\isacommand{proof}\isamarkupfalse%
\ {\isacharparenleft}{\kern0pt}intro\ allI\ impI{\isacharparenright}{\kern0pt}\isanewline
\ \ \isacommand{fix}\isamarkupfalse%
\isanewline
\ \ \ \ A\ {\isacharcolon}{\kern0pt}{\isacharcolon}{\kern0pt}\ {\isachardoublequoteopen}{\isacharprime}{\kern0pt}a\ set{\isachardoublequoteclose}\ \isakeyword{and}\isanewline
\ \ \ \ p\ {\isacharcolon}{\kern0pt}{\isacharcolon}{\kern0pt}\ {\isachardoublequoteopen}{\isacharprime}{\kern0pt}a\ Profile{\isachardoublequoteclose}\ \isakeyword{and}\isanewline
\ \ \ \ q\ {\isacharcolon}{\kern0pt}{\isacharcolon}{\kern0pt}\ {\isachardoublequoteopen}{\isacharprime}{\kern0pt}a\ Profile{\isachardoublequoteclose}\ \isakeyword{and}\isanewline
\ \ \ \ a\ {\isacharcolon}{\kern0pt}{\isacharcolon}{\kern0pt}\ {\isachardoublequoteopen}{\isacharprime}{\kern0pt}a{\isachardoublequoteclose}\isanewline
\ \ \isacommand{assume}\isamarkupfalse%
\isanewline
\ \ \ \ asm{\isadigit{1}}{\isacharcolon}{\kern0pt}\isanewline
\ \ \ \ {\isachardoublequoteopen}a\ {\isasymin}\ elect\ plurality\ A\ p\ {\isasymand}\ lifted\ A\ p\ q\ a{\isachardoublequoteclose}\isanewline
\ \ \isacommand{show}\isamarkupfalse%
\isanewline
\ \ \ \ {\isachardoublequoteopen}elect\ plurality\ A\ q\ {\isacharequal}{\kern0pt}\ elect\ plurality\ A\ p\ {\isasymor}\isanewline
\ \ \ \ \ \ \ \ elect\ plurality\ A\ q\ {\isacharequal}{\kern0pt}\ {\isacharbraceleft}{\kern0pt}a{\isacharbraceright}{\kern0pt}{\isachardoublequoteclose}\isanewline
\ \ \isacommand{proof}\isamarkupfalse%
\ {\isacharminus}{\kern0pt}\isanewline
\ \ \ \ \isacommand{have}\isamarkupfalse%
\ lifted{\isacharunderscore}{\kern0pt}winner{\isacharcolon}{\kern0pt}\isanewline
\ \ \ \ \ \ {\isachardoublequoteopen}{\isasymforall}x\ {\isasymin}\ A{\isachardot}{\kern0pt}\isanewline
\ \ \ \ \ \ \ \ \ {\isasymforall}i{\isacharcolon}{\kern0pt}{\isacharcolon}{\kern0pt}nat{\isachardot}{\kern0pt}\ i\ {\isacharless}{\kern0pt}\ length\ p\ {\isasymlongrightarrow}\isanewline
\ \ \ \ \ \ \ \ \ \ \ {\isacharparenleft}{\kern0pt}above\ {\isacharparenleft}{\kern0pt}p{\isacharbang}{\kern0pt}i{\isacharparenright}{\kern0pt}\ x\ {\isacharequal}{\kern0pt}\ {\isacharbraceleft}{\kern0pt}x{\isacharbraceright}{\kern0pt}\ {\isasymlongrightarrow}\isanewline
\ \ \ \ \ \ \ \ \ \ \ \ \ \ {\isacharparenleft}{\kern0pt}above\ {\isacharparenleft}{\kern0pt}q{\isacharbang}{\kern0pt}i{\isacharparenright}{\kern0pt}\ x\ {\isacharequal}{\kern0pt}\ {\isacharbraceleft}{\kern0pt}x{\isacharbraceright}{\kern0pt}\ {\isasymor}\ above\ {\isacharparenleft}{\kern0pt}q{\isacharbang}{\kern0pt}i{\isacharparenright}{\kern0pt}\ a\ {\isacharequal}{\kern0pt}\ {\isacharbraceleft}{\kern0pt}a{\isacharbraceright}{\kern0pt}{\isacharparenright}{\kern0pt}{\isacharparenright}{\kern0pt}{\isachardoublequoteclose}\isanewline
\ \ \ \ \ \ \isacommand{using}\isamarkupfalse%
\ asm{\isadigit{1}}\ Profile{\isachardot}{\kern0pt}lifted{\isacharunderscore}{\kern0pt}def\ lifted{\isacharunderscore}{\kern0pt}above{\isacharunderscore}{\kern0pt}winner\isanewline
\ \ \ \ \ \ \isacommand{by}\isamarkupfalse%
\ {\isacharparenleft}{\kern0pt}metis\ {\isacharparenleft}{\kern0pt}no{\isacharunderscore}{\kern0pt}types{\isacharcomma}{\kern0pt}\ lifting{\isacharparenright}{\kern0pt}{\isacharparenright}{\kern0pt}\isanewline
\ \ \ \ \isacommand{hence}\isamarkupfalse%
\isanewline
\ \ \ \ \ \ {\isachardoublequoteopen}{\isasymforall}i{\isacharcolon}{\kern0pt}{\isacharcolon}{\kern0pt}nat{\isachardot}{\kern0pt}\ i\ {\isacharless}{\kern0pt}\ length\ p\ {\isasymlongrightarrow}\isanewline
\ \ \ \ \ \ \ \ \ \ {\isacharparenleft}{\kern0pt}above\ {\isacharparenleft}{\kern0pt}p{\isacharbang}{\kern0pt}i{\isacharparenright}{\kern0pt}\ a\ {\isacharequal}{\kern0pt}\ {\isacharbraceleft}{\kern0pt}a{\isacharbraceright}{\kern0pt}\ {\isasymlongrightarrow}\ above\ {\isacharparenleft}{\kern0pt}q{\isacharbang}{\kern0pt}i{\isacharparenright}{\kern0pt}\ a\ {\isacharequal}{\kern0pt}\ {\isacharbraceleft}{\kern0pt}a{\isacharbraceright}{\kern0pt}{\isacharparenright}{\kern0pt}{\isachardoublequoteclose}\isanewline
\ \ \ \ \ \ \isacommand{using}\isamarkupfalse%
\ asm{\isadigit{1}}\isanewline
\ \ \ \ \ \ \isacommand{by}\isamarkupfalse%
\ auto\isanewline
\ \ \ \ \isacommand{hence}\isamarkupfalse%
\ a{\isacharunderscore}{\kern0pt}win{\isacharunderscore}{\kern0pt}subset{\isacharcolon}{\kern0pt}\isanewline
\ \ \ \ \ \ {\isachardoublequoteopen}{\isacharbraceleft}{\kern0pt}i{\isacharcolon}{\kern0pt}{\isacharcolon}{\kern0pt}nat{\isachardot}{\kern0pt}\ i\ {\isacharless}{\kern0pt}\ length\ p\ {\isasymand}\ above\ {\isacharparenleft}{\kern0pt}p{\isacharbang}{\kern0pt}i{\isacharparenright}{\kern0pt}\ a\ {\isacharequal}{\kern0pt}\ {\isacharbraceleft}{\kern0pt}a{\isacharbraceright}{\kern0pt}{\isacharbraceright}{\kern0pt}\ {\isasymsubseteq}\isanewline
\ \ \ \ \ \ \ \ \ \ {\isacharbraceleft}{\kern0pt}i{\isacharcolon}{\kern0pt}{\isacharcolon}{\kern0pt}nat{\isachardot}{\kern0pt}\ i\ {\isacharless}{\kern0pt}\ length\ p\ {\isasymand}\ above\ {\isacharparenleft}{\kern0pt}q{\isacharbang}{\kern0pt}i{\isacharparenright}{\kern0pt}\ a\ {\isacharequal}{\kern0pt}\ {\isacharbraceleft}{\kern0pt}a{\isacharbraceright}{\kern0pt}{\isacharbraceright}{\kern0pt}{\isachardoublequoteclose}\isanewline
\ \ \ \ \ \ \isacommand{by}\isamarkupfalse%
\ blast\isanewline
\ \ \ \ \isacommand{moreover}\isamarkupfalse%
\ \isacommand{have}\isamarkupfalse%
\ sizes{\isacharcolon}{\kern0pt}\isanewline
\ \ \ \ \ \ {\isachardoublequoteopen}length\ p\ {\isacharequal}{\kern0pt}\ length\ q{\isachardoublequoteclose}\isanewline
\ \ \ \ \ \ \isacommand{using}\isamarkupfalse%
\ asm{\isadigit{1}}\ Profile{\isachardot}{\kern0pt}lifted{\isacharunderscore}{\kern0pt}def\isanewline
\ \ \ \ \ \ \isacommand{by}\isamarkupfalse%
\ metis\isanewline
\ \ \ \ \isacommand{ultimately}\isamarkupfalse%
\ \isacommand{have}\isamarkupfalse%
\ win{\isacharunderscore}{\kern0pt}count{\isacharunderscore}{\kern0pt}a{\isacharcolon}{\kern0pt}\isanewline
\ \ \ \ \ \ {\isachardoublequoteopen}win{\isacharunderscore}{\kern0pt}count\ p\ a\ {\isasymle}\ win{\isacharunderscore}{\kern0pt}count\ q\ a{\isachardoublequoteclose}\isanewline
\ \ \ \ \ \ \isacommand{by}\isamarkupfalse%
\ {\isacharparenleft}{\kern0pt}simp\ add{\isacharcolon}{\kern0pt}\ card{\isacharunderscore}{\kern0pt}mono{\isacharparenright}{\kern0pt}\isanewline
\ \ \ \ \isacommand{have}\isamarkupfalse%
\ fin{\isacharunderscore}{\kern0pt}A{\isacharcolon}{\kern0pt}\isanewline
\ \ \ \ \ \ {\isachardoublequoteopen}finite\ A{\isachardoublequoteclose}\isanewline
\ \ \ \ \ \ \isacommand{using}\isamarkupfalse%
\ asm{\isadigit{1}}\ Profile{\isachardot}{\kern0pt}lifted{\isacharunderscore}{\kern0pt}def\isanewline
\ \ \ \ \ \ \isacommand{by}\isamarkupfalse%
\ metis\isanewline
\ \ \ \ \isacommand{hence}\isamarkupfalse%
\isanewline
\ \ \ \ \ \ {\isachardoublequoteopen}{\isasymforall}x\ {\isasymin}\ A{\isacharminus}{\kern0pt}{\isacharbraceleft}{\kern0pt}a{\isacharbraceright}{\kern0pt}{\isachardot}{\kern0pt}\isanewline
\ \ \ \ \ \ \ \ {\isasymforall}i{\isacharcolon}{\kern0pt}{\isacharcolon}{\kern0pt}nat{\isachardot}{\kern0pt}\ i\ {\isacharless}{\kern0pt}\ length\ p\ {\isasymlongrightarrow}\isanewline
\ \ \ \ \ \ \ \ \ \ {\isacharparenleft}{\kern0pt}above\ {\isacharparenleft}{\kern0pt}q{\isacharbang}{\kern0pt}i{\isacharparenright}{\kern0pt}\ a\ {\isacharequal}{\kern0pt}\ {\isacharbraceleft}{\kern0pt}a{\isacharbraceright}{\kern0pt}\ {\isasymlongrightarrow}\ above\ {\isacharparenleft}{\kern0pt}q{\isacharbang}{\kern0pt}i{\isacharparenright}{\kern0pt}\ x\ {\isasymnoteq}\ {\isacharbraceleft}{\kern0pt}x{\isacharbraceright}{\kern0pt}{\isacharparenright}{\kern0pt}{\isachardoublequoteclose}\isanewline
\ \ \ \ \ \ \isacommand{using}\isamarkupfalse%
\ DiffE\ Profile{\isachardot}{\kern0pt}lifted{\isacharunderscore}{\kern0pt}def\ above{\isacharunderscore}{\kern0pt}one{\isadigit{2}}\isanewline
\ \ \ \ \ \ \ \ \ \ \ \ asm{\isadigit{1}}\ insertCI\ insert{\isacharunderscore}{\kern0pt}absorb\ insert{\isacharunderscore}{\kern0pt}not{\isacharunderscore}{\kern0pt}empty\isanewline
\ \ \ \ \ \ \ \ \ \ \ \ profile{\isacharunderscore}{\kern0pt}def\ sizes\isanewline
\ \ \ \ \ \ \isacommand{by}\isamarkupfalse%
\ metis\isanewline
\ \ \ \ \isacommand{with}\isamarkupfalse%
\ lifted{\isacharunderscore}{\kern0pt}winner\ \isacommand{have}\isamarkupfalse%
\ above{\isacharunderscore}{\kern0pt}QtoP{\isacharcolon}{\kern0pt}\isanewline
\ \ \ \ \ \ {\isachardoublequoteopen}{\isasymforall}x\ {\isasymin}\ A{\isacharminus}{\kern0pt}{\isacharbraceleft}{\kern0pt}a{\isacharbraceright}{\kern0pt}{\isachardot}{\kern0pt}\isanewline
\ \ \ \ \ \ \ \ {\isasymforall}i{\isacharcolon}{\kern0pt}{\isacharcolon}{\kern0pt}nat{\isachardot}{\kern0pt}\ i\ {\isacharless}{\kern0pt}\ length\ p\ {\isasymlongrightarrow}\isanewline
\ \ \ \ \ \ \ \ \ \ {\isacharparenleft}{\kern0pt}above\ {\isacharparenleft}{\kern0pt}q{\isacharbang}{\kern0pt}i{\isacharparenright}{\kern0pt}\ x\ {\isacharequal}{\kern0pt}\ {\isacharbraceleft}{\kern0pt}x{\isacharbraceright}{\kern0pt}\ {\isasymlongrightarrow}\ above\ {\isacharparenleft}{\kern0pt}p{\isacharbang}{\kern0pt}i{\isacharparenright}{\kern0pt}\ x\ {\isacharequal}{\kern0pt}\ {\isacharbraceleft}{\kern0pt}x{\isacharbraceright}{\kern0pt}{\isacharparenright}{\kern0pt}{\isachardoublequoteclose}\isanewline
\ \ \ \ \ \ \isacommand{using}\isamarkupfalse%
\ lifted{\isacharunderscore}{\kern0pt}above{\isacharunderscore}{\kern0pt}winner{\isadigit{3}}\ asm{\isadigit{1}}\isanewline
\ \ \ \ \ \ \ \ \ \ \ \ Profile{\isachardot}{\kern0pt}lifted{\isacharunderscore}{\kern0pt}def\isanewline
\ \ \ \ \ \ \isacommand{by}\isamarkupfalse%
\ metis\isanewline
\ \ \ \ \isacommand{hence}\isamarkupfalse%
\isanewline
\ \ \ \ \ \ {\isachardoublequoteopen}{\isasymforall}x\ {\isasymin}\ A{\isacharminus}{\kern0pt}{\isacharbraceleft}{\kern0pt}a{\isacharbraceright}{\kern0pt}{\isachardot}{\kern0pt}\isanewline
\ \ \ \ \ \ \ \ {\isacharbraceleft}{\kern0pt}i{\isacharcolon}{\kern0pt}{\isacharcolon}{\kern0pt}nat{\isachardot}{\kern0pt}\ i\ {\isacharless}{\kern0pt}\ length\ p\ {\isasymand}\ above\ {\isacharparenleft}{\kern0pt}q{\isacharbang}{\kern0pt}i{\isacharparenright}{\kern0pt}\ x\ {\isacharequal}{\kern0pt}\ {\isacharbraceleft}{\kern0pt}x{\isacharbraceright}{\kern0pt}{\isacharbraceright}{\kern0pt}\ {\isasymsubseteq}\isanewline
\ \ \ \ \ \ \ \ \ \ {\isacharbraceleft}{\kern0pt}i{\isacharcolon}{\kern0pt}{\isacharcolon}{\kern0pt}nat{\isachardot}{\kern0pt}\ i\ {\isacharless}{\kern0pt}\ length\ p\ {\isasymand}\ above\ {\isacharparenleft}{\kern0pt}p{\isacharbang}{\kern0pt}i{\isacharparenright}{\kern0pt}\ x\ {\isacharequal}{\kern0pt}\ {\isacharbraceleft}{\kern0pt}x{\isacharbraceright}{\kern0pt}{\isacharbraceright}{\kern0pt}{\isachardoublequoteclose}\isanewline
\ \ \ \ \ \ \isacommand{by}\isamarkupfalse%
\ {\isacharparenleft}{\kern0pt}simp\ add{\isacharcolon}{\kern0pt}\ Collect{\isacharunderscore}{\kern0pt}mono{\isacharparenright}{\kern0pt}\isanewline
\ \ \ \ \isacommand{hence}\isamarkupfalse%
\ win{\isacharunderscore}{\kern0pt}count{\isacharunderscore}{\kern0pt}other{\isacharcolon}{\kern0pt}\isanewline
\ \ \ \ \ \ {\isachardoublequoteopen}{\isasymforall}x\ {\isasymin}\ A{\isacharminus}{\kern0pt}{\isacharbraceleft}{\kern0pt}a{\isacharbraceright}{\kern0pt}{\isachardot}{\kern0pt}\ win{\isacharunderscore}{\kern0pt}count\ p\ x\ {\isasymge}\ win{\isacharunderscore}{\kern0pt}count\ q\ x{\isachardoublequoteclose}\isanewline
\ \ \ \ \ \ \isacommand{by}\isamarkupfalse%
\ {\isacharparenleft}{\kern0pt}simp\ add{\isacharcolon}{\kern0pt}\ card{\isacharunderscore}{\kern0pt}mono\ sizes{\isacharparenright}{\kern0pt}\isanewline
\ \ \ \ \isacommand{show}\isamarkupfalse%
\isanewline
\ \ \ \ \ \ {\isachardoublequoteopen}elect\ plurality\ A\ q\ {\isacharequal}{\kern0pt}\ elect\ plurality\ A\ p\ {\isasymor}\isanewline
\ \ \ \ \ \ \ \ \ \ \ elect\ plurality\ A\ q\ {\isacharequal}{\kern0pt}\ {\isacharbraceleft}{\kern0pt}a{\isacharbraceright}{\kern0pt}{\isachardoublequoteclose}\isanewline
\ \ \ \ \isacommand{proof}\isamarkupfalse%
\ cases\isanewline
\ \ \ \ \ \ \isacommand{assume}\isamarkupfalse%
\ {\isachardoublequoteopen}win{\isacharunderscore}{\kern0pt}count\ p\ a\ {\isacharequal}{\kern0pt}\ win{\isacharunderscore}{\kern0pt}count\ q\ a{\isachardoublequoteclose}\isanewline
\ \ \ \ \ \ \isacommand{hence}\isamarkupfalse%
\isanewline
\ \ \ \ \ \ \ \ {\isachardoublequoteopen}card\ {\isacharbraceleft}{\kern0pt}i{\isacharcolon}{\kern0pt}{\isacharcolon}{\kern0pt}nat{\isachardot}{\kern0pt}\ i\ {\isacharless}{\kern0pt}\ length\ p\ {\isasymand}\ above\ {\isacharparenleft}{\kern0pt}p{\isacharbang}{\kern0pt}i{\isacharparenright}{\kern0pt}\ a\ {\isacharequal}{\kern0pt}\ {\isacharbraceleft}{\kern0pt}a{\isacharbraceright}{\kern0pt}{\isacharbraceright}{\kern0pt}\ {\isacharequal}{\kern0pt}\isanewline
\ \ \ \ \ \ \ \ \ \ \ \ \ \ card\ {\isacharbraceleft}{\kern0pt}i{\isacharcolon}{\kern0pt}{\isacharcolon}{\kern0pt}nat{\isachardot}{\kern0pt}\ i\ {\isacharless}{\kern0pt}\ length\ p\ {\isasymand}\ above\ {\isacharparenleft}{\kern0pt}q{\isacharbang}{\kern0pt}i{\isacharparenright}{\kern0pt}\ a\ {\isacharequal}{\kern0pt}\ {\isacharbraceleft}{\kern0pt}a{\isacharbraceright}{\kern0pt}{\isacharbraceright}{\kern0pt}{\isachardoublequoteclose}\isanewline
\ \ \ \ \ \ \ \ \isacommand{by}\isamarkupfalse%
\ {\isacharparenleft}{\kern0pt}simp\ add{\isacharcolon}{\kern0pt}\ sizes{\isacharparenright}{\kern0pt}\isanewline
\ \ \ \ \ \ \isacommand{moreover}\isamarkupfalse%
\ \isacommand{have}\isamarkupfalse%
\isanewline
\ \ \ \ \ \ \ \ {\isachardoublequoteopen}finite\ {\isacharbraceleft}{\kern0pt}i{\isacharcolon}{\kern0pt}{\isacharcolon}{\kern0pt}nat{\isachardot}{\kern0pt}\ i\ {\isacharless}{\kern0pt}\ length\ p\ {\isasymand}\ above\ {\isacharparenleft}{\kern0pt}q{\isacharbang}{\kern0pt}i{\isacharparenright}{\kern0pt}\ a\ {\isacharequal}{\kern0pt}\ {\isacharbraceleft}{\kern0pt}a{\isacharbraceright}{\kern0pt}{\isacharbraceright}{\kern0pt}{\isachardoublequoteclose}\isanewline
\ \ \ \ \ \ \ \ \isacommand{by}\isamarkupfalse%
\ simp\isanewline
\ \ \ \ \ \ \isacommand{ultimately}\isamarkupfalse%
\ \isacommand{have}\isamarkupfalse%
\isanewline
\ \ \ \ \ \ \ \ {\isachardoublequoteopen}{\isacharbraceleft}{\kern0pt}i{\isacharcolon}{\kern0pt}{\isacharcolon}{\kern0pt}nat{\isachardot}{\kern0pt}\ i\ {\isacharless}{\kern0pt}\ length\ p\ {\isasymand}\ above\ {\isacharparenleft}{\kern0pt}p{\isacharbang}{\kern0pt}i{\isacharparenright}{\kern0pt}\ a\ {\isacharequal}{\kern0pt}\ {\isacharbraceleft}{\kern0pt}a{\isacharbraceright}{\kern0pt}{\isacharbraceright}{\kern0pt}\ {\isacharequal}{\kern0pt}\isanewline
\ \ \ \ \ \ \ \ \ \ \ \ \ \ {\isacharbraceleft}{\kern0pt}i{\isacharcolon}{\kern0pt}{\isacharcolon}{\kern0pt}nat{\isachardot}{\kern0pt}\ i\ {\isacharless}{\kern0pt}\ length\ p\ {\isasymand}\ above\ {\isacharparenleft}{\kern0pt}q{\isacharbang}{\kern0pt}i{\isacharparenright}{\kern0pt}\ a\ {\isacharequal}{\kern0pt}\ {\isacharbraceleft}{\kern0pt}a{\isacharbraceright}{\kern0pt}{\isacharbraceright}{\kern0pt}{\isachardoublequoteclose}\isanewline
\ \ \ \ \ \ \ \ \isacommand{using}\isamarkupfalse%
\ a{\isacharunderscore}{\kern0pt}win{\isacharunderscore}{\kern0pt}subset\isanewline
\ \ \ \ \ \ \ \ \isacommand{by}\isamarkupfalse%
\ {\isacharparenleft}{\kern0pt}simp\ add{\isacharcolon}{\kern0pt}\ card{\isacharunderscore}{\kern0pt}subset{\isacharunderscore}{\kern0pt}eq{\isacharparenright}{\kern0pt}\isanewline
\ \ \ \ \ \ \isacommand{hence}\isamarkupfalse%
\ above{\isacharunderscore}{\kern0pt}pq{\isacharcolon}{\kern0pt}\isanewline
\ \ \ \ \ \ \ \ {\isachardoublequoteopen}{\isasymforall}i{\isacharcolon}{\kern0pt}{\isacharcolon}{\kern0pt}nat{\isachardot}{\kern0pt}\ i\ {\isacharless}{\kern0pt}\ length\ p\ {\isasymlongrightarrow}\isanewline
\ \ \ \ \ \ \ \ \ \ \ \ above\ {\isacharparenleft}{\kern0pt}p{\isacharbang}{\kern0pt}i{\isacharparenright}{\kern0pt}\ a\ {\isacharequal}{\kern0pt}\ {\isacharbraceleft}{\kern0pt}a{\isacharbraceright}{\kern0pt}\ {\isasymlongleftrightarrow}\ above\ {\isacharparenleft}{\kern0pt}q{\isacharbang}{\kern0pt}i{\isacharparenright}{\kern0pt}\ a\ {\isacharequal}{\kern0pt}\ {\isacharbraceleft}{\kern0pt}a{\isacharbraceright}{\kern0pt}{\isachardoublequoteclose}\isanewline
\ \ \ \ \ \ \ \ \isacommand{by}\isamarkupfalse%
\ blast\isanewline
\ \ \ \ \ \ \isacommand{moreover}\isamarkupfalse%
\ \isacommand{have}\isamarkupfalse%
\isanewline
\ \ \ \ \ \ \ \ {\isachardoublequoteopen}{\isasymforall}x\ {\isasymin}\ A{\isacharminus}{\kern0pt}{\isacharbraceleft}{\kern0pt}a{\isacharbraceright}{\kern0pt}{\isachardot}{\kern0pt}\isanewline
\ \ \ \ \ \ \ \ \ \ {\isasymforall}i{\isacharcolon}{\kern0pt}{\isacharcolon}{\kern0pt}nat{\isachardot}{\kern0pt}\ i\ {\isacharless}{\kern0pt}\ length\ p\ {\isasymlongrightarrow}\isanewline
\ \ \ \ \ \ \ \ \ \ \ \ {\isacharparenleft}{\kern0pt}above\ {\isacharparenleft}{\kern0pt}p{\isacharbang}{\kern0pt}i{\isacharparenright}{\kern0pt}\ x\ {\isacharequal}{\kern0pt}\ {\isacharbraceleft}{\kern0pt}x{\isacharbraceright}{\kern0pt}\ {\isasymlongrightarrow}\isanewline
\ \ \ \ \ \ \ \ \ \ \ \ \ \ {\isacharparenleft}{\kern0pt}above\ {\isacharparenleft}{\kern0pt}q{\isacharbang}{\kern0pt}i{\isacharparenright}{\kern0pt}\ x\ {\isacharequal}{\kern0pt}\ {\isacharbraceleft}{\kern0pt}x{\isacharbraceright}{\kern0pt}\ {\isasymor}\ above\ {\isacharparenleft}{\kern0pt}q{\isacharbang}{\kern0pt}i{\isacharparenright}{\kern0pt}\ a\ {\isacharequal}{\kern0pt}\ {\isacharbraceleft}{\kern0pt}a{\isacharbraceright}{\kern0pt}{\isacharparenright}{\kern0pt}{\isacharparenright}{\kern0pt}{\isachardoublequoteclose}\isanewline
\ \ \ \ \ \ \ \ \isacommand{using}\isamarkupfalse%
\ lifted{\isacharunderscore}{\kern0pt}winner\isanewline
\ \ \ \ \ \ \ \ \isacommand{by}\isamarkupfalse%
\ auto\isanewline
\ \ \ \ \ \ \isacommand{moreover}\isamarkupfalse%
\ \isacommand{have}\isamarkupfalse%
\isanewline
\ \ \ \ \ \ \ \ {\isachardoublequoteopen}{\isasymforall}x\ {\isasymin}\ A{\isacharminus}{\kern0pt}{\isacharbraceleft}{\kern0pt}a{\isacharbraceright}{\kern0pt}{\isachardot}{\kern0pt}\isanewline
\ \ \ \ \ \ \ \ \ \ {\isasymforall}i{\isacharcolon}{\kern0pt}{\isacharcolon}{\kern0pt}nat{\isachardot}{\kern0pt}\ i\ {\isacharless}{\kern0pt}\ length\ p\ {\isasymlongrightarrow}\isanewline
\ \ \ \ \ \ \ \ \ \ \ \ {\isacharparenleft}{\kern0pt}above\ {\isacharparenleft}{\kern0pt}p{\isacharbang}{\kern0pt}i{\isacharparenright}{\kern0pt}\ x\ {\isacharequal}{\kern0pt}\ {\isacharbraceleft}{\kern0pt}x{\isacharbraceright}{\kern0pt}\ {\isasymlongrightarrow}\ above\ {\isacharparenleft}{\kern0pt}p{\isacharbang}{\kern0pt}i{\isacharparenright}{\kern0pt}\ a\ {\isasymnoteq}\ {\isacharbraceleft}{\kern0pt}a{\isacharbraceright}{\kern0pt}{\isacharparenright}{\kern0pt}{\isachardoublequoteclose}\isanewline
\ \ \ \ \ \ \isacommand{proof}\isamarkupfalse%
\ {\isacharparenleft}{\kern0pt}rule\ ccontr{\isacharparenright}{\kern0pt}\isanewline
\ \ \ \ \ \ \ \ \isacommand{assume}\isamarkupfalse%
\isanewline
\ \ \ \ \ \ \ \ \ \ {\isachardoublequoteopen}{\isasymnot}{\isacharparenleft}{\kern0pt}{\isasymforall}x\ {\isasymin}\ A{\isacharminus}{\kern0pt}{\isacharbraceleft}{\kern0pt}a{\isacharbraceright}{\kern0pt}{\isachardot}{\kern0pt}\isanewline
\ \ \ \ \ \ \ \ \ \ \ \ \ \ {\isasymforall}i{\isacharcolon}{\kern0pt}{\isacharcolon}{\kern0pt}nat{\isachardot}{\kern0pt}\ i\ {\isacharless}{\kern0pt}\ length\ p\ {\isasymlongrightarrow}\isanewline
\ \ \ \ \ \ \ \ \ \ \ \ \ \ \ \ {\isacharparenleft}{\kern0pt}above\ {\isacharparenleft}{\kern0pt}p{\isacharbang}{\kern0pt}i{\isacharparenright}{\kern0pt}\ x\ {\isacharequal}{\kern0pt}\ {\isacharbraceleft}{\kern0pt}x{\isacharbraceright}{\kern0pt}\ {\isasymlongrightarrow}\ above\ {\isacharparenleft}{\kern0pt}p{\isacharbang}{\kern0pt}i{\isacharparenright}{\kern0pt}\ a\ {\isasymnoteq}\ {\isacharbraceleft}{\kern0pt}a{\isacharbraceright}{\kern0pt}{\isacharparenright}{\kern0pt}{\isacharparenright}{\kern0pt}{\isachardoublequoteclose}\isanewline
\ \ \ \ \ \ \ \ \isacommand{hence}\isamarkupfalse%
\isanewline
\ \ \ \ \ \ \ \ \ \ {\isachardoublequoteopen}{\isasymexists}x\ {\isasymin}\ A{\isacharminus}{\kern0pt}{\isacharbraceleft}{\kern0pt}a{\isacharbraceright}{\kern0pt}{\isachardot}{\kern0pt}\isanewline
\ \ \ \ \ \ \ \ \ \ \ \ {\isasymexists}i{\isacharcolon}{\kern0pt}{\isacharcolon}{\kern0pt}nat{\isachardot}{\kern0pt}\isanewline
\ \ \ \ \ \ \ \ \ \ \ \ \ \ i\ {\isacharless}{\kern0pt}\ length\ p\ {\isasymand}\ above\ {\isacharparenleft}{\kern0pt}p{\isacharbang}{\kern0pt}i{\isacharparenright}{\kern0pt}\ x\ {\isacharequal}{\kern0pt}\ {\isacharbraceleft}{\kern0pt}x{\isacharbraceright}{\kern0pt}\ {\isasymand}\ above\ {\isacharparenleft}{\kern0pt}p{\isacharbang}{\kern0pt}i{\isacharparenright}{\kern0pt}\ a\ {\isacharequal}{\kern0pt}\ {\isacharbraceleft}{\kern0pt}a{\isacharbraceright}{\kern0pt}{\isachardoublequoteclose}\isanewline
\ \ \ \ \ \ \ \ \ \ \isacommand{by}\isamarkupfalse%
\ auto\isanewline
\ \ \ \ \ \ \ \ \isacommand{moreover}\isamarkupfalse%
\ \isacommand{from}\isamarkupfalse%
\ this\ \isacommand{have}\isamarkupfalse%
\isanewline
\ \ \ \ \ \ \ \ \ \ {\isachardoublequoteopen}finite\ A\ {\isasymand}\ A\ {\isasymnoteq}\ {\isacharbraceleft}{\kern0pt}{\isacharbraceright}{\kern0pt}{\isachardoublequoteclose}\isanewline
\ \ \ \ \ \ \ \ \ \ \isacommand{using}\isamarkupfalse%
\ fin{\isacharunderscore}{\kern0pt}A\isanewline
\ \ \ \ \ \ \ \ \ \ \isacommand{by}\isamarkupfalse%
\ blast\isanewline
\ \ \ \ \ \ \ \ \isacommand{moreover}\isamarkupfalse%
\ \isacommand{from}\isamarkupfalse%
\ asm{\isadigit{1}}\ \isacommand{have}\isamarkupfalse%
\isanewline
\ \ \ \ \ \ \ \ \ \ {\isachardoublequoteopen}{\isasymforall}i{\isacharcolon}{\kern0pt}{\isacharcolon}{\kern0pt}nat{\isachardot}{\kern0pt}\ i\ {\isacharless}{\kern0pt}\ length\ p\ {\isasymlongrightarrow}\ linear{\isacharunderscore}{\kern0pt}order{\isacharunderscore}{\kern0pt}on\ A\ {\isacharparenleft}{\kern0pt}p{\isacharbang}{\kern0pt}i{\isacharparenright}{\kern0pt}{\isachardoublequoteclose}\isanewline
\ \ \ \ \ \ \ \ \ \ \isacommand{by}\isamarkupfalse%
\ {\isacharparenleft}{\kern0pt}simp\ add{\isacharcolon}{\kern0pt}\ Profile{\isachardot}{\kern0pt}lifted{\isacharunderscore}{\kern0pt}def\ profile{\isacharunderscore}{\kern0pt}def{\isacharparenright}{\kern0pt}\isanewline
\ \ \ \ \ \ \ \ \isacommand{ultimately}\isamarkupfalse%
\ \isacommand{have}\isamarkupfalse%
\isanewline
\ \ \ \ \ \ \ \ \ \ {\isachardoublequoteopen}{\isasymexists}x\ {\isasymin}\ A{\isacharminus}{\kern0pt}{\isacharbraceleft}{\kern0pt}a{\isacharbraceright}{\kern0pt}{\isachardot}{\kern0pt}\ x\ {\isacharequal}{\kern0pt}\ a{\isachardoublequoteclose}\isanewline
\ \ \ \ \ \ \ \ \ \ \isacommand{using}\isamarkupfalse%
\ above{\isacharunderscore}{\kern0pt}one{\isadigit{2}}\isanewline
\ \ \ \ \ \ \ \ \ \ \isacommand{by}\isamarkupfalse%
\ metis\isanewline
\ \ \ \ \ \ \ \ \isacommand{thus}\isamarkupfalse%
\ {\isachardoublequoteopen}False{\isachardoublequoteclose}\isanewline
\ \ \ \ \ \ \ \ \ \ \isacommand{by}\isamarkupfalse%
\ simp\isanewline
\ \ \ \ \ \ \isacommand{qed}\isamarkupfalse%
\isanewline
\ \ \ \ \ \ \isacommand{ultimately}\isamarkupfalse%
\ \isacommand{have}\isamarkupfalse%
\ above{\isacharunderscore}{\kern0pt}PtoQ{\isacharcolon}{\kern0pt}\isanewline
\ \ \ \ \ \ \ \ {\isachardoublequoteopen}{\isasymforall}x\ {\isasymin}\ A{\isacharminus}{\kern0pt}{\isacharbraceleft}{\kern0pt}a{\isacharbraceright}{\kern0pt}{\isachardot}{\kern0pt}\isanewline
\ \ \ \ \ \ \ \ \ \ {\isasymforall}i{\isacharcolon}{\kern0pt}{\isacharcolon}{\kern0pt}nat{\isachardot}{\kern0pt}\ i\ {\isacharless}{\kern0pt}\ length\ p\ {\isasymlongrightarrow}\isanewline
\ \ \ \ \ \ \ \ \ \ \ \ {\isacharparenleft}{\kern0pt}above\ {\isacharparenleft}{\kern0pt}p{\isacharbang}{\kern0pt}i{\isacharparenright}{\kern0pt}\ x\ {\isacharequal}{\kern0pt}\ {\isacharbraceleft}{\kern0pt}x{\isacharbraceright}{\kern0pt}\ {\isasymlongrightarrow}\ above\ {\isacharparenleft}{\kern0pt}q{\isacharbang}{\kern0pt}i{\isacharparenright}{\kern0pt}\ x\ {\isacharequal}{\kern0pt}\ {\isacharbraceleft}{\kern0pt}x{\isacharbraceright}{\kern0pt}{\isacharparenright}{\kern0pt}{\isachardoublequoteclose}\isanewline
\ \ \ \ \ \ \ \ \isacommand{by}\isamarkupfalse%
\ {\isacharparenleft}{\kern0pt}simp\ add{\isacharcolon}{\kern0pt}\ above{\isacharunderscore}{\kern0pt}pq{\isacharparenright}{\kern0pt}\isanewline
\ \ \ \ \ \ \isacommand{hence}\isamarkupfalse%
\isanewline
\ \ \ \ \ \ \ \ {\isachardoublequoteopen}{\isasymforall}x\ {\isasymin}\ A{\isachardot}{\kern0pt}\isanewline
\ \ \ \ \ \ \ \ \ \ card\ {\isacharbraceleft}{\kern0pt}i{\isacharcolon}{\kern0pt}{\isacharcolon}{\kern0pt}nat{\isachardot}{\kern0pt}\ i\ {\isacharless}{\kern0pt}\ length\ p\ {\isasymand}\ above\ {\isacharparenleft}{\kern0pt}p{\isacharbang}{\kern0pt}i{\isacharparenright}{\kern0pt}\ x\ {\isacharequal}{\kern0pt}\ {\isacharbraceleft}{\kern0pt}x{\isacharbraceright}{\kern0pt}{\isacharbraceright}{\kern0pt}\ {\isacharequal}{\kern0pt}\isanewline
\ \ \ \ \ \ \ \ \ \ \ \ card\ {\isacharbraceleft}{\kern0pt}i{\isacharcolon}{\kern0pt}{\isacharcolon}{\kern0pt}nat{\isachardot}{\kern0pt}\ i\ {\isacharless}{\kern0pt}\ length\ q\ {\isasymand}\ above\ {\isacharparenleft}{\kern0pt}q{\isacharbang}{\kern0pt}i{\isacharparenright}{\kern0pt}\ x\ {\isacharequal}{\kern0pt}\ {\isacharbraceleft}{\kern0pt}x{\isacharbraceright}{\kern0pt}{\isacharbraceright}{\kern0pt}{\isachardoublequoteclose}\isanewline
\ \ \ \ \ \ \ \ \isacommand{using}\isamarkupfalse%
\ Collect{\isacharunderscore}{\kern0pt}cong\ DiffI\ above{\isacharunderscore}{\kern0pt}pq\ above{\isacharunderscore}{\kern0pt}QtoP\isanewline
\ \ \ \ \ \ \ \ \ \ \ \ \ \ insert{\isacharunderscore}{\kern0pt}absorb\ insert{\isacharunderscore}{\kern0pt}iff\ insert{\isacharunderscore}{\kern0pt}not{\isacharunderscore}{\kern0pt}empty\ sizes\isanewline
\ \ \ \ \ \ \ \ \isacommand{by}\isamarkupfalse%
\ {\isacharparenleft}{\kern0pt}smt\ {\isacharparenleft}{\kern0pt}verit{\isacharcomma}{\kern0pt}\ ccfv{\isacharunderscore}{\kern0pt}threshold{\isacharparenright}{\kern0pt}{\isacharparenright}{\kern0pt}\isanewline
\ \ \ \ \ \ \isacommand{hence}\isamarkupfalse%
\ {\isachardoublequoteopen}{\isasymforall}x\ {\isasymin}\ A{\isachardot}{\kern0pt}\ win{\isacharunderscore}{\kern0pt}count\ p\ x\ {\isacharequal}{\kern0pt}\ win{\isacharunderscore}{\kern0pt}count\ q\ x{\isachardoublequoteclose}\isanewline
\ \ \ \ \ \ \ \ \isacommand{by}\isamarkupfalse%
\ simp\isanewline
\ \ \ \ \ \ \isacommand{hence}\isamarkupfalse%
\isanewline
\ \ \ \ \ \ \ \ {\isachardoublequoteopen}{\isacharbraceleft}{\kern0pt}a\ {\isasymin}\ A{\isachardot}{\kern0pt}\ {\isasymforall}x\ {\isasymin}\ A{\isachardot}{\kern0pt}\ win{\isacharunderscore}{\kern0pt}count\ p\ x\ {\isasymle}\ win{\isacharunderscore}{\kern0pt}count\ p\ a{\isacharbraceright}{\kern0pt}\ {\isacharequal}{\kern0pt}\isanewline
\ \ \ \ \ \ \ \ \ \ \ \ {\isacharbraceleft}{\kern0pt}a\ {\isasymin}\ A{\isachardot}{\kern0pt}\ {\isasymforall}x\ {\isasymin}\ A{\isachardot}{\kern0pt}\ win{\isacharunderscore}{\kern0pt}count\ q\ x\ {\isasymle}\ win{\isacharunderscore}{\kern0pt}count\ q\ a{\isacharbraceright}{\kern0pt}{\isachardoublequoteclose}\isanewline
\ \ \ \ \ \ \ \ \isacommand{by}\isamarkupfalse%
\ auto\isanewline
\ \ \ \ \ \ \isacommand{thus}\isamarkupfalse%
\ {\isacharquery}{\kern0pt}thesis\isanewline
\ \ \ \ \ \ \ \ \isacommand{by}\isamarkupfalse%
\ simp\isanewline
\ \ \ \ \isacommand{next}\isamarkupfalse%
\isanewline
\ \ \ \ \ \ \isacommand{assume}\isamarkupfalse%
\ {\isachardoublequoteopen}win{\isacharunderscore}{\kern0pt}count\ p\ a\ {\isasymnoteq}\ win{\isacharunderscore}{\kern0pt}count\ q\ a{\isachardoublequoteclose}\isanewline
\ \ \ \ \ \ \isacommand{hence}\isamarkupfalse%
\ strict{\isacharunderscore}{\kern0pt}less{\isacharcolon}{\kern0pt}\isanewline
\ \ \ \ \ \ \ \ {\isachardoublequoteopen}win{\isacharunderscore}{\kern0pt}count\ p\ a\ {\isacharless}{\kern0pt}\ win{\isacharunderscore}{\kern0pt}count\ q\ a{\isachardoublequoteclose}\isanewline
\ \ \ \ \ \ \ \ \isacommand{using}\isamarkupfalse%
\ win{\isacharunderscore}{\kern0pt}count{\isacharunderscore}{\kern0pt}a\isanewline
\ \ \ \ \ \ \ \ \isacommand{by}\isamarkupfalse%
\ auto\isanewline
\ \ \ \ \ \ \isacommand{have}\isamarkupfalse%
\ a{\isacharunderscore}{\kern0pt}in{\isacharunderscore}{\kern0pt}win{\isacharunderscore}{\kern0pt}p{\isacharcolon}{\kern0pt}\isanewline
\ \ \ \ \ \ \ \ {\isachardoublequoteopen}a\ {\isasymin}\ {\isacharbraceleft}{\kern0pt}a\ {\isasymin}\ A{\isachardot}{\kern0pt}\ {\isasymforall}x\ {\isasymin}\ A{\isachardot}{\kern0pt}\ win{\isacharunderscore}{\kern0pt}count\ p\ x\ {\isasymle}\ win{\isacharunderscore}{\kern0pt}count\ p\ a{\isacharbraceright}{\kern0pt}{\isachardoublequoteclose}\isanewline
\ \ \ \ \ \ \ \ \isacommand{using}\isamarkupfalse%
\ asm{\isadigit{1}}\isanewline
\ \ \ \ \ \ \ \ \isacommand{by}\isamarkupfalse%
\ auto\isanewline
\ \ \ \ \ \ \isacommand{hence}\isamarkupfalse%
\ {\isachardoublequoteopen}{\isasymforall}x\ {\isasymin}\ A{\isachardot}{\kern0pt}\ win{\isacharunderscore}{\kern0pt}count\ p\ x\ {\isasymle}\ win{\isacharunderscore}{\kern0pt}count\ p\ a{\isachardoublequoteclose}\isanewline
\ \ \ \ \ \ \ \ \isacommand{by}\isamarkupfalse%
\ simp\isanewline
\ \ \ \ \ \ \isacommand{with}\isamarkupfalse%
\ strict{\isacharunderscore}{\kern0pt}less\ win{\isacharunderscore}{\kern0pt}count{\isacharunderscore}{\kern0pt}other\isanewline
\ \ \ \ \ \ \isacommand{have}\isamarkupfalse%
\ less{\isacharcolon}{\kern0pt}\isanewline
\ \ \ \ \ \ \ \ {\isachardoublequoteopen}{\isasymforall}x\ {\isasymin}\ A{\isacharminus}{\kern0pt}{\isacharbraceleft}{\kern0pt}a{\isacharbraceright}{\kern0pt}{\isachardot}{\kern0pt}\ win{\isacharunderscore}{\kern0pt}count\ q\ x\ {\isacharless}{\kern0pt}\ win{\isacharunderscore}{\kern0pt}count\ q\ a{\isachardoublequoteclose}\isanewline
\ \ \ \ \ \ \ \ \isacommand{using}\isamarkupfalse%
\ DiffD{\isadigit{1}}\ antisym\ dual{\isacharunderscore}{\kern0pt}order{\isachardot}{\kern0pt}trans\isanewline
\ \ \ \ \ \ \ \ \ \ \ \ \ \ not{\isacharunderscore}{\kern0pt}le{\isacharunderscore}{\kern0pt}imp{\isacharunderscore}{\kern0pt}less\ win{\isacharunderscore}{\kern0pt}count{\isacharunderscore}{\kern0pt}a\isanewline
\ \ \ \ \ \ \ \ \isacommand{by}\isamarkupfalse%
\ metis\isanewline
\ \ \ \ \ \ \isacommand{hence}\isamarkupfalse%
\isanewline
\ \ \ \ \ \ \ \ {\isachardoublequoteopen}{\isasymforall}x\ {\isasymin}\ A{\isacharminus}{\kern0pt}{\isacharbraceleft}{\kern0pt}a{\isacharbraceright}{\kern0pt}{\isachardot}{\kern0pt}\ {\isasymnot}{\isacharparenleft}{\kern0pt}{\isasymforall}y\ {\isasymin}\ A{\isachardot}{\kern0pt}\ win{\isacharunderscore}{\kern0pt}count\ q\ y\ {\isasymle}\ win{\isacharunderscore}{\kern0pt}count\ q\ x{\isacharparenright}{\kern0pt}{\isachardoublequoteclose}\isanewline
\ \ \ \ \ \ \ \ \isacommand{using}\isamarkupfalse%
\ asm{\isadigit{1}}\ Profile{\isachardot}{\kern0pt}lifted{\isacharunderscore}{\kern0pt}def\ not{\isacharunderscore}{\kern0pt}le\isanewline
\ \ \ \ \ \ \ \ \isacommand{by}\isamarkupfalse%
\ metis\isanewline
\ \ \ \ \ \ \isacommand{hence}\isamarkupfalse%
\isanewline
\ \ \ \ \ \ \ \ {\isachardoublequoteopen}{\isasymforall}x\ {\isasymin}\ A{\isacharminus}{\kern0pt}{\isacharbraceleft}{\kern0pt}a{\isacharbraceright}{\kern0pt}{\isachardot}{\kern0pt}\isanewline
\ \ \ \ \ \ \ \ \ \ x\ {\isasymnotin}\ {\isacharbraceleft}{\kern0pt}a\ {\isasymin}\ A{\isachardot}{\kern0pt}\ {\isasymforall}x\ {\isasymin}\ A{\isachardot}{\kern0pt}\ win{\isacharunderscore}{\kern0pt}count\ q\ x\ {\isasymle}\ win{\isacharunderscore}{\kern0pt}count\ q\ a{\isacharbraceright}{\kern0pt}{\isachardoublequoteclose}\isanewline
\ \ \ \ \ \ \ \ \isacommand{by}\isamarkupfalse%
\ blast\isanewline
\ \ \ \ \ \ \isacommand{hence}\isamarkupfalse%
\isanewline
\ \ \ \ \ \ \ \ {\isachardoublequoteopen}{\isasymforall}x\ {\isasymin}\ A{\isacharminus}{\kern0pt}{\isacharbraceleft}{\kern0pt}a{\isacharbraceright}{\kern0pt}{\isachardot}{\kern0pt}\ x\ {\isasymnotin}\ elect\ plurality\ A\ q{\isachardoublequoteclose}\isanewline
\ \ \ \ \ \ \ \ \isacommand{by}\isamarkupfalse%
\ simp\isanewline
\ \ \ \ \ \ \isacommand{moreover}\isamarkupfalse%
\ \isacommand{have}\isamarkupfalse%
\isanewline
\ \ \ \ \ \ \ \ {\isachardoublequoteopen}a\ {\isasymin}\ elect\ plurality\ A\ q{\isachardoublequoteclose}\isanewline
\ \ \ \ \ \ \isacommand{proof}\isamarkupfalse%
\ {\isacharminus}{\kern0pt}\isanewline
\ \ \ \ \ \ \ \ \isacommand{from}\isamarkupfalse%
\ less\isanewline
\ \ \ \ \ \ \ \ \isacommand{have}\isamarkupfalse%
\isanewline
\ \ \ \ \ \ \ \ \ \ {\isachardoublequoteopen}{\isasymforall}x\ {\isasymin}\ A{\isacharminus}{\kern0pt}{\isacharbraceleft}{\kern0pt}a{\isacharbraceright}{\kern0pt}{\isachardot}{\kern0pt}\ win{\isacharunderscore}{\kern0pt}count\ q\ x\ {\isasymle}\ win{\isacharunderscore}{\kern0pt}count\ q\ a{\isachardoublequoteclose}\isanewline
\ \ \ \ \ \ \ \ \ \ \isacommand{using}\isamarkupfalse%
\ less{\isacharunderscore}{\kern0pt}imp{\isacharunderscore}{\kern0pt}le\isanewline
\ \ \ \ \ \ \ \ \ \ \isacommand{by}\isamarkupfalse%
\ metis\isanewline
\ \ \ \ \ \ \ \ \isacommand{moreover}\isamarkupfalse%
\ \isacommand{have}\isamarkupfalse%
\isanewline
\ \ \ \ \ \ \ \ \ \ {\isachardoublequoteopen}win{\isacharunderscore}{\kern0pt}count\ q\ a\ {\isasymle}\ win{\isacharunderscore}{\kern0pt}count\ q\ a{\isachardoublequoteclose}\isanewline
\ \ \ \ \ \ \ \ \ \ \isacommand{by}\isamarkupfalse%
\ simp\isanewline
\ \ \ \ \ \ \ \ \isacommand{ultimately}\isamarkupfalse%
\ \isacommand{have}\isamarkupfalse%
\isanewline
\ \ \ \ \ \ \ \ \ \ {\isachardoublequoteopen}{\isasymforall}x\ {\isasymin}\ A{\isachardot}{\kern0pt}\ win{\isacharunderscore}{\kern0pt}count\ q\ x\ {\isasymle}\ win{\isacharunderscore}{\kern0pt}count\ q\ a{\isachardoublequoteclose}\isanewline
\ \ \ \ \ \ \ \ \ \ \isacommand{by}\isamarkupfalse%
\ auto\isanewline
\ \ \ \ \ \ \ \ \isacommand{moreover}\isamarkupfalse%
\ \isacommand{have}\isamarkupfalse%
\isanewline
\ \ \ \ \ \ \ \ \ \ {\isachardoublequoteopen}a\ {\isasymin}\ A{\isachardoublequoteclose}\isanewline
\ \ \ \ \ \ \ \ \ \ \isacommand{using}\isamarkupfalse%
\ a{\isacharunderscore}{\kern0pt}in{\isacharunderscore}{\kern0pt}win{\isacharunderscore}{\kern0pt}p\isanewline
\ \ \ \ \ \ \ \ \ \ \isacommand{by}\isamarkupfalse%
\ auto\isanewline
\ \ \ \ \ \ \ \ \isacommand{ultimately}\isamarkupfalse%
\ \isacommand{have}\isamarkupfalse%
\isanewline
\ \ \ \ \ \ \ \ \ \ {\isachardoublequoteopen}a\ {\isasymin}\ {\isacharbraceleft}{\kern0pt}a\ {\isasymin}\ A{\isachardot}{\kern0pt}\isanewline
\ \ \ \ \ \ \ \ \ \ \ \ \ \ {\isasymforall}x\ {\isasymin}\ A{\isachardot}{\kern0pt}\ win{\isacharunderscore}{\kern0pt}count\ q\ x\ {\isasymle}\ win{\isacharunderscore}{\kern0pt}count\ q\ a{\isacharbraceright}{\kern0pt}{\isachardoublequoteclose}\isanewline
\ \ \ \ \ \ \ \ \ \ \isacommand{by}\isamarkupfalse%
\ auto\isanewline
\ \ \ \ \ \ \ \ \isacommand{thus}\isamarkupfalse%
\ {\isacharquery}{\kern0pt}thesis\isanewline
\ \ \ \ \ \ \ \ \ \ \isacommand{by}\isamarkupfalse%
\ simp\isanewline
\ \ \ \ \ \ \isacommand{qed}\isamarkupfalse%
\isanewline
\ \ \ \ \ \ \isacommand{moreover}\isamarkupfalse%
\ \isacommand{have}\isamarkupfalse%
\isanewline
\ \ \ \ \ \ \ \ {\isachardoublequoteopen}elect\ plurality\ A\ q\ {\isasymsubseteq}\ A{\isachardoublequoteclose}\isanewline
\ \ \ \ \ \ \ \ \isacommand{by}\isamarkupfalse%
\ simp\isanewline
\ \ \ \ \ \ \isacommand{ultimately}\isamarkupfalse%
\ \isacommand{show}\isamarkupfalse%
\ {\isacharquery}{\kern0pt}thesis\isanewline
\ \ \ \ \ \ \ \ \isacommand{by}\isamarkupfalse%
\ auto\isanewline
\ \ \ \ \isacommand{qed}\isamarkupfalse%
\isanewline
\ \ \isacommand{qed}\isamarkupfalse%
\isanewline
\isacommand{qed}\isamarkupfalse%
%
\endisatagproof
{\isafoldproof}%
%
\isadelimproof
\isanewline
%
\endisadelimproof
\isanewline
\isanewline
\isacommand{theorem}\isamarkupfalse%
\ plurality{\isacharunderscore}{\kern0pt}inv{\isacharunderscore}{\kern0pt}mono{\isacharbrackleft}{\kern0pt}simp{\isacharbrackright}{\kern0pt}{\isacharcolon}{\kern0pt}\ {\isachardoublequoteopen}invariant{\isacharunderscore}{\kern0pt}monotonicity\ plurality{\isachardoublequoteclose}\isanewline
%
\isadelimproof
%
\endisadelimproof
%
\isatagproof
\isacommand{proof}\isamarkupfalse%
\ {\isacharminus}{\kern0pt}\isanewline
\ \ \isacommand{have}\isamarkupfalse%
\isanewline
\ \ \ \ {\isachardoublequoteopen}electoral{\isacharunderscore}{\kern0pt}module\ plurality\ {\isasymand}\isanewline
\ \ \ \ \ \ {\isacharparenleft}{\kern0pt}{\isasymforall}A\ p\ q\ a{\isachardot}{\kern0pt}\isanewline
\ \ \ \ \ \ \ \ {\isacharparenleft}{\kern0pt}a\ {\isasymin}\ elect\ plurality\ A\ p\ {\isasymand}\ lifted\ A\ p\ q\ a{\isacharparenright}{\kern0pt}\ {\isasymlongrightarrow}\isanewline
\ \ \ \ \ \ \ \ \ \ {\isacharparenleft}{\kern0pt}elect\ plurality\ A\ q\ {\isacharequal}{\kern0pt}\ elect\ plurality\ A\ p\ {\isasymor}\isanewline
\ \ \ \ \ \ \ \ \ \ \ \ elect\ plurality\ A\ q\ {\isacharequal}{\kern0pt}\ {\isacharbraceleft}{\kern0pt}a{\isacharbraceright}{\kern0pt}{\isacharparenright}{\kern0pt}{\isacharparenright}{\kern0pt}{\isachardoublequoteclose}\isanewline
\ \ \isacommand{proof}\isamarkupfalse%
\isanewline
\ \ \ \ \isacommand{show}\isamarkupfalse%
\ {\isachardoublequoteopen}electoral{\isacharunderscore}{\kern0pt}module\ plurality{\isachardoublequoteclose}\isanewline
\ \ \ \ \ \ \isacommand{by}\isamarkupfalse%
\ simp\isanewline
\ \ \isacommand{next}\isamarkupfalse%
\isanewline
\ \ \ \ \isacommand{show}\isamarkupfalse%
\isanewline
\ \ \ \ \ \ {\isachardoublequoteopen}{\isasymforall}A\ p\ q\ a{\isachardot}{\kern0pt}\ {\isacharparenleft}{\kern0pt}a\ {\isasymin}\ elect\ plurality\ A\ p\ {\isasymand}\ lifted\ A\ p\ q\ a{\isacharparenright}{\kern0pt}\ {\isasymlongrightarrow}\isanewline
\ \ \ \ \ \ \ \ \ \ {\isacharparenleft}{\kern0pt}elect\ plurality\ A\ q\ {\isacharequal}{\kern0pt}\ elect\ plurality\ A\ p\ {\isasymor}\isanewline
\ \ \ \ \ \ \ \ \ \ \ \ elect\ plurality\ A\ q\ {\isacharequal}{\kern0pt}\ {\isacharbraceleft}{\kern0pt}a{\isacharbraceright}{\kern0pt}{\isacharparenright}{\kern0pt}{\isachardoublequoteclose}\isanewline
\ \ \ \ \ \ \isacommand{using}\isamarkupfalse%
\ plurality{\isacharunderscore}{\kern0pt}inv{\isacharunderscore}{\kern0pt}mono{\isadigit{2}}\isanewline
\ \ \ \ \ \ \isacommand{by}\isamarkupfalse%
\ metis\isanewline
\ \ \isacommand{qed}\isamarkupfalse%
\isanewline
\ \ \isacommand{thus}\isamarkupfalse%
\ {\isacharquery}{\kern0pt}thesis\isanewline
\ \ \ \ \isacommand{by}\isamarkupfalse%
\ {\isacharparenleft}{\kern0pt}simp\ add{\isacharcolon}{\kern0pt}\ invariant{\isacharunderscore}{\kern0pt}monotonicity{\isacharunderscore}{\kern0pt}def{\isacharparenright}{\kern0pt}\isanewline
\isacommand{qed}\isamarkupfalse%
%
\endisatagproof
{\isafoldproof}%
%
\isadelimproof
\isanewline
%
\endisadelimproof
%
\isadelimtheory
\isanewline
%
\endisadelimtheory
%
\isatagtheory
\isacommand{end}\isamarkupfalse%
%
\endisatagtheory
{\isafoldtheory}%
%
\isadelimtheory
%
\endisadelimtheory
%
\end{isabellebody}%
\endinput
%:%file=~/Documents/Studies/VotingRuleGenerator/virage/src/test/resources/verifiedVotingRuleConstruction/theories/Compositional_Framework/Composition_Rules/Monotonicity_Facts.thy%:%
%:%10=1%:%
%:%11=1%:%
%:%12=2%:%
%:%13=3%:%
%:%14=4%:%
%:%15=5%:%
%:%16=6%:%
%:%17=7%:%
%:%18=8%:%
%:%23=8%:%
%:%26=9%:%
%:%27=10%:%
%:%28=10%:%
%:%31=11%:%
%:%35=11%:%
%:%36=11%:%
%:%37=12%:%
%:%38=12%:%
%:%43=12%:%
%:%46=13%:%
%:%47=14%:%
%:%48=15%:%
%:%49=15%:%
%:%50=16%:%
%:%51=17%:%
%:%54=18%:%
%:%58=18%:%
%:%59=18%:%
%:%64=18%:%
%:%67=19%:%
%:%68=20%:%
%:%69=21%:%
%:%70=21%:%
%:%71=22%:%
%:%72=23%:%
%:%75=24%:%
%:%79=24%:%
%:%80=24%:%
%:%85=24%:%
%:%88=25%:%
%:%89=26%:%
%:%90=26%:%
%:%93=29%:%
%:%100=30%:%
%:%101=30%:%
%:%102=31%:%
%:%103=31%:%
%:%104=32%:%
%:%105=33%:%
%:%106=34%:%
%:%107=35%:%
%:%108=36%:%
%:%109=36%:%
%:%110=37%:%
%:%111=38%:%
%:%112=39%:%
%:%113=39%:%
%:%114=40%:%
%:%115=41%:%
%:%116=42%:%
%:%117=42%:%
%:%118=43%:%
%:%119=43%:%
%:%120=44%:%
%:%123=47%:%
%:%124=48%:%
%:%125=48%:%
%:%126=49%:%
%:%127=49%:%
%:%128=50%:%
%:%129=50%:%
%:%130=51%:%
%:%131=52%:%
%:%132=53%:%
%:%133=53%:%
%:%134=54%:%
%:%135=54%:%
%:%136=55%:%
%:%137=55%:%
%:%138=56%:%
%:%139=57%:%
%:%140=58%:%
%:%141=58%:%
%:%142=59%:%
%:%143=59%:%
%:%144=59%:%
%:%145=60%:%
%:%146=61%:%
%:%147=61%:%
%:%148=62%:%
%:%149=62%:%
%:%150=63%:%
%:%151=63%:%
%:%152=63%:%
%:%153=64%:%
%:%154=65%:%
%:%155=65%:%
%:%156=66%:%
%:%157=66%:%
%:%158=67%:%
%:%159=68%:%
%:%160=68%:%
%:%161=69%:%
%:%162=69%:%
%:%163=70%:%
%:%164=70%:%
%:%165=71%:%
%:%167=73%:%
%:%168=74%:%
%:%169=74%:%
%:%170=75%:%
%:%171=76%:%
%:%172=77%:%
%:%173=77%:%
%:%174=78%:%
%:%175=78%:%
%:%176=78%:%
%:%177=79%:%
%:%179=81%:%
%:%180=82%:%
%:%181=82%:%
%:%182=83%:%
%:%183=84%:%
%:%184=84%:%
%:%185=85%:%
%:%186=85%:%
%:%187=86%:%
%:%189=88%:%
%:%190=89%:%
%:%191=89%:%
%:%192=90%:%
%:%193=90%:%
%:%194=91%:%
%:%195=92%:%
%:%196=92%:%
%:%197=93%:%
%:%198=93%:%
%:%199=94%:%
%:%200=95%:%
%:%201=96%:%
%:%202=96%:%
%:%203=97%:%
%:%204=97%:%
%:%205=98%:%
%:%206=98%:%
%:%207=99%:%
%:%208=100%:%
%:%209=101%:%
%:%210=101%:%
%:%211=102%:%
%:%212=102%:%
%:%213=102%:%
%:%214=103%:%
%:%215=104%:%
%:%216=104%:%
%:%217=105%:%
%:%218=105%:%
%:%219=105%:%
%:%220=106%:%
%:%221=107%:%
%:%222=108%:%
%:%223=108%:%
%:%224=109%:%
%:%225=109%:%
%:%226=110%:%
%:%227=110%:%
%:%228=111%:%
%:%229=112%:%
%:%230=113%:%
%:%231=113%:%
%:%232=114%:%
%:%233=114%:%
%:%234=114%:%
%:%235=115%:%
%:%238=118%:%
%:%239=119%:%
%:%240=119%:%
%:%241=120%:%
%:%242=120%:%
%:%243=121%:%
%:%244=121%:%
%:%245=121%:%
%:%246=122%:%
%:%248=124%:%
%:%249=125%:%
%:%250=125%:%
%:%251=126%:%
%:%252=126%:%
%:%253=127%:%
%:%255=129%:%
%:%256=130%:%
%:%257=130%:%
%:%258=131%:%
%:%260=133%:%
%:%261=134%:%
%:%262=134%:%
%:%263=135%:%
%:%264=135%:%
%:%265=135%:%
%:%266=135%:%
%:%267=136%:%
%:%268=137%:%
%:%269=137%:%
%:%270=138%:%
%:%271=138%:%
%:%272=139%:%
%:%273=139%:%
%:%274=139%:%
%:%275=139%:%
%:%276=140%:%
%:%277=141%:%
%:%278=141%:%
%:%279=142%:%
%:%280=142%:%
%:%281=142%:%
%:%282=143%:%
%:%283=144%:%
%:%284=144%:%
%:%285=145%:%
%:%286=145%:%
%:%287=146%:%
%:%288=146%:%
%:%289=147%:%
%:%290=147%:%
%:%291=148%:%
%:%292=148%:%
%:%293=149%:%
%:%294=149%:%
%:%295=149%:%
%:%296=150%:%
%:%298=152%:%
%:%299=153%:%
%:%300=153%:%
%:%301=154%:%
%:%302=154%:%
%:%303=155%:%
%:%305=157%:%
%:%306=158%:%
%:%307=158%:%
%:%308=159%:%
%:%309=160%:%
%:%310=160%:%
%:%311=161%:%
%:%312=161%:%
%:%313=162%:%
%:%314=162%:%
%:%315=163%:%
%:%316=163%:%
%:%317=164%:%
%:%318=165%:%
%:%319=166%:%
%:%320=166%:%
%:%321=167%:%
%:%322=167%:%
%:%323=168%:%
%:%324=168%:%
%:%325=169%:%
%:%326=169%:%
%:%327=170%:%
%:%328=170%:%
%:%329=171%:%
%:%330=171%:%
%:%331=172%:%
%:%332=173%:%
%:%333=173%:%
%:%334=174%:%
%:%335=174%:%
%:%336=175%:%
%:%337=175%:%
%:%338=176%:%
%:%339=177%:%
%:%340=177%:%
%:%341=178%:%
%:%342=178%:%
%:%343=179%:%
%:%344=179%:%
%:%345=180%:%
%:%346=180%:%
%:%347=181%:%
%:%348=181%:%
%:%349=182%:%
%:%350=182%:%
%:%351=183%:%
%:%352=184%:%
%:%353=184%:%
%:%354=185%:%
%:%355=186%:%
%:%356=186%:%
%:%357=187%:%
%:%358=187%:%
%:%359=188%:%
%:%360=189%:%
%:%361=189%:%
%:%362=190%:%
%:%363=190%:%
%:%364=191%:%
%:%365=191%:%
%:%366=192%:%
%:%367=193%:%
%:%368=194%:%
%:%369=194%:%
%:%370=195%:%
%:%371=195%:%
%:%372=196%:%
%:%373=197%:%
%:%374=197%:%
%:%375=198%:%
%:%376=198%:%
%:%377=198%:%
%:%378=199%:%
%:%379=200%:%
%:%380=200%:%
%:%381=201%:%
%:%382=201%:%
%:%383=202%:%
%:%384=202%:%
%:%385=203%:%
%:%386=204%:%
%:%387=204%:%
%:%388=205%:%
%:%389=205%:%
%:%390=206%:%
%:%391=206%:%
%:%392=206%:%
%:%393=207%:%
%:%394=208%:%
%:%395=208%:%
%:%396=209%:%
%:%397=209%:%
%:%398=209%:%
%:%399=210%:%
%:%400=211%:%
%:%401=211%:%
%:%402=212%:%
%:%403=212%:%
%:%404=212%:%
%:%405=213%:%
%:%406=214%:%
%:%407=214%:%
%:%408=215%:%
%:%409=215%:%
%:%410=216%:%
%:%411=216%:%
%:%412=216%:%
%:%413=217%:%
%:%414=218%:%
%:%415=219%:%
%:%416=219%:%
%:%417=220%:%
%:%418=220%:%
%:%419=221%:%
%:%420=221%:%
%:%421=222%:%
%:%422=222%:%
%:%423=223%:%
%:%424=223%:%
%:%425=223%:%
%:%426=224%:%
%:%427=225%:%
%:%428=225%:%
%:%429=226%:%
%:%430=226%:%
%:%431=226%:%
%:%432=227%:%
%:%433=227%:%
%:%434=228%:%
%:%435=228%:%
%:%436=229%:%
%:%437=229%:%
%:%438=230%:%
%:%444=230%:%
%:%447=231%:%
%:%448=232%:%
%:%449=233%:%
%:%450=233%:%
%:%457=234%:%
%:%458=234%:%
%:%459=235%:%
%:%460=235%:%
%:%461=236%:%
%:%465=240%:%
%:%466=241%:%
%:%467=241%:%
%:%468=242%:%
%:%469=242%:%
%:%470=243%:%
%:%471=243%:%
%:%472=244%:%
%:%473=244%:%
%:%474=245%:%
%:%475=245%:%
%:%476=246%:%
%:%478=248%:%
%:%479=249%:%
%:%480=249%:%
%:%481=250%:%
%:%482=250%:%
%:%483=251%:%
%:%484=251%:%
%:%485=252%:%
%:%486=252%:%
%:%487=253%:%
%:%488=253%:%
%:%489=254%:%
%:%495=254%:%
%:%500=255%:%
%:%505=256%:%
%
\begin{isabellebody}%
\setisabellecontext{Monotonicity{\isacharunderscore}{\kern0pt}Rules}%
%
\isadelimtheory
%
\endisadelimtheory
%
\isatagtheory
\isacommand{theory}\isamarkupfalse%
\ Monotonicity{\isacharunderscore}{\kern0pt}Rules\isanewline
\ \ \isakeyword{imports}\ {\isachardoublequoteopen}{\isachardot}{\kern0pt}{\isachardot}{\kern0pt}{\isacharslash}{\kern0pt}Properties{\isacharslash}{\kern0pt}Monotonicity{\isacharunderscore}{\kern0pt}Properties{\isachardoublequoteclose}\isanewline
\ \ \ \ \ \ \ \ \ \ {\isachardoublequoteopen}{\isachardot}{\kern0pt}{\isachardot}{\kern0pt}{\isacharslash}{\kern0pt}Properties{\isacharslash}{\kern0pt}Disjoint{\isacharunderscore}{\kern0pt}Compatibility{\isachardoublequoteclose}\isanewline
\ \ \ \ \ \ \ \ \ \ {\isachardoublequoteopen}{\isachardot}{\kern0pt}{\isachardot}{\kern0pt}{\isacharslash}{\kern0pt}{\isachardot}{\kern0pt}{\isachardot}{\kern0pt}{\isacharslash}{\kern0pt}Social{\isacharunderscore}{\kern0pt}Choice{\isacharunderscore}{\kern0pt}Properties{\isacharslash}{\kern0pt}Weak{\isacharunderscore}{\kern0pt}Monotonicity{\isachardoublequoteclose}\isanewline
\ \ \ \ \ \ \ \ \ \ {\isachardoublequoteopen}{\isachardot}{\kern0pt}{\isachardot}{\kern0pt}{\isacharslash}{\kern0pt}Components{\isacharslash}{\kern0pt}Compositional{\isacharunderscore}{\kern0pt}Structures{\isacharslash}{\kern0pt}Parallel{\isacharunderscore}{\kern0pt}Composition{\isachardoublequoteclose}\isanewline
\ \ \ \ \ \ \ \ \ \ {\isachardoublequoteopen}{\isachardot}{\kern0pt}{\isachardot}{\kern0pt}{\isacharslash}{\kern0pt}Components{\isacharslash}{\kern0pt}Compositional{\isacharunderscore}{\kern0pt}Structures{\isacharslash}{\kern0pt}Sequential{\isacharunderscore}{\kern0pt}Composition{\isachardoublequoteclose}\isanewline
\ \ \ \ \ \ \ \ \ \ {\isachardoublequoteopen}{\isachardot}{\kern0pt}{\isachardot}{\kern0pt}{\isacharslash}{\kern0pt}Components{\isacharslash}{\kern0pt}Basic{\isacharunderscore}{\kern0pt}Modules{\isacharslash}{\kern0pt}Maximum{\isacharunderscore}{\kern0pt}Aggregator{\isachardoublequoteclose}\isanewline
\ \ \ \ \ \ \ \ \ \ Result{\isacharunderscore}{\kern0pt}Rules\isanewline
\ \ \ \ \ \ \ \ \ \ Monotonicity{\isacharunderscore}{\kern0pt}Facts\isanewline
\isanewline
\isakeyword{begin}%
\endisatagtheory
{\isafoldtheory}%
%
\isadelimtheory
\isanewline
%
\endisadelimtheory
\isanewline
\isanewline
\isacommand{theorem}\isamarkupfalse%
\ def{\isacharunderscore}{\kern0pt}inv{\isacharunderscore}{\kern0pt}mono{\isacharunderscore}{\kern0pt}imp{\isacharunderscore}{\kern0pt}def{\isacharunderscore}{\kern0pt}lift{\isacharunderscore}{\kern0pt}inv{\isacharbrackleft}{\kern0pt}simp{\isacharbrackright}{\kern0pt}{\isacharcolon}{\kern0pt}\isanewline
\ \ \isakeyword{assumes}\isanewline
\ \ \ \ strong{\isacharunderscore}{\kern0pt}def{\isacharunderscore}{\kern0pt}mon{\isacharunderscore}{\kern0pt}m{\isacharcolon}{\kern0pt}\ {\isachardoublequoteopen}defer{\isacharunderscore}{\kern0pt}invariant{\isacharunderscore}{\kern0pt}monotonicity\ m{\isachardoublequoteclose}\ \isakeyword{and}\isanewline
\ \ \ \ non{\isacharunderscore}{\kern0pt}electing{\isacharunderscore}{\kern0pt}n{\isacharcolon}{\kern0pt}\ {\isachardoublequoteopen}non{\isacharunderscore}{\kern0pt}electing\ n{\isachardoublequoteclose}\ \isakeyword{and}\isanewline
\ \ \ \ defers{\isacharunderscore}{\kern0pt}{\isadigit{1}}{\isacharcolon}{\kern0pt}\ {\isachardoublequoteopen}defers\ {\isadigit{1}}\ n{\isachardoublequoteclose}\ \isakeyword{and}\isanewline
\ \ \ \ defer{\isacharunderscore}{\kern0pt}monotone{\isacharunderscore}{\kern0pt}n{\isacharcolon}{\kern0pt}\ {\isachardoublequoteopen}defer{\isacharunderscore}{\kern0pt}monotonicity\ n{\isachardoublequoteclose}\isanewline
\ \ \isakeyword{shows}\ {\isachardoublequoteopen}defer{\isacharunderscore}{\kern0pt}lift{\isacharunderscore}{\kern0pt}invariance\ {\isacharparenleft}{\kern0pt}m\ {\isasymtriangleright}\ n{\isacharparenright}{\kern0pt}{\isachardoublequoteclose}\isanewline
%
\isadelimproof
\ \ %
\endisadelimproof
%
\isatagproof
\isacommand{unfolding}\isamarkupfalse%
\ defer{\isacharunderscore}{\kern0pt}lift{\isacharunderscore}{\kern0pt}invariance{\isacharunderscore}{\kern0pt}def\isanewline
\isacommand{proof}\isamarkupfalse%
\ {\isacharparenleft}{\kern0pt}safe{\isacharparenright}{\kern0pt}\isanewline
\ \ \isacommand{have}\isamarkupfalse%
\ electoral{\isacharunderscore}{\kern0pt}mod{\isacharunderscore}{\kern0pt}m{\isacharcolon}{\kern0pt}\ {\isachardoublequoteopen}electoral{\isacharunderscore}{\kern0pt}module\ m{\isachardoublequoteclose}\isanewline
\ \ \ \ \isacommand{using}\isamarkupfalse%
\ defer{\isacharunderscore}{\kern0pt}invariant{\isacharunderscore}{\kern0pt}monotonicity{\isacharunderscore}{\kern0pt}def\isanewline
\ \ \ \ \ \ \ \ \ \ strong{\isacharunderscore}{\kern0pt}def{\isacharunderscore}{\kern0pt}mon{\isacharunderscore}{\kern0pt}m\isanewline
\ \ \ \ \isacommand{by}\isamarkupfalse%
\ auto\isanewline
\ \ \isacommand{have}\isamarkupfalse%
\ electoral{\isacharunderscore}{\kern0pt}mod{\isacharunderscore}{\kern0pt}n{\isacharcolon}{\kern0pt}\ {\isachardoublequoteopen}electoral{\isacharunderscore}{\kern0pt}module\ n{\isachardoublequoteclose}\isanewline
\ \ \ \ \isacommand{using}\isamarkupfalse%
\ defers{\isacharunderscore}{\kern0pt}{\isadigit{1}}\ defers{\isacharunderscore}{\kern0pt}def\isanewline
\ \ \ \ \isacommand{by}\isamarkupfalse%
\ auto\isanewline
\ \ \isacommand{show}\isamarkupfalse%
\ {\isachardoublequoteopen}electoral{\isacharunderscore}{\kern0pt}module\ {\isacharparenleft}{\kern0pt}m\ {\isasymtriangleright}\ n{\isacharparenright}{\kern0pt}{\isachardoublequoteclose}\isanewline
\ \ \ \ \isacommand{using}\isamarkupfalse%
\ electoral{\isacharunderscore}{\kern0pt}mod{\isacharunderscore}{\kern0pt}m\ electoral{\isacharunderscore}{\kern0pt}mod{\isacharunderscore}{\kern0pt}n\isanewline
\ \ \ \ \isacommand{by}\isamarkupfalse%
\ simp\isanewline
\isacommand{next}\isamarkupfalse%
\isanewline
\ \ \isacommand{fix}\isamarkupfalse%
\isanewline
\ \ \ \ A\ {\isacharcolon}{\kern0pt}{\isacharcolon}{\kern0pt}\ {\isachardoublequoteopen}{\isacharprime}{\kern0pt}a\ set{\isachardoublequoteclose}\ \isakeyword{and}\isanewline
\ \ \ \ p\ {\isacharcolon}{\kern0pt}{\isacharcolon}{\kern0pt}\ {\isachardoublequoteopen}{\isacharprime}{\kern0pt}a\ Profile{\isachardoublequoteclose}\ \isakeyword{and}\isanewline
\ \ \ \ q\ {\isacharcolon}{\kern0pt}{\isacharcolon}{\kern0pt}\ {\isachardoublequoteopen}{\isacharprime}{\kern0pt}a\ Profile{\isachardoublequoteclose}\ \isakeyword{and}\isanewline
\ \ \ \ a\ {\isacharcolon}{\kern0pt}{\isacharcolon}{\kern0pt}\ {\isachardoublequoteopen}{\isacharprime}{\kern0pt}a{\isachardoublequoteclose}\isanewline
\ \ \isacommand{assume}\isamarkupfalse%
\isanewline
\ \ defer{\isacharunderscore}{\kern0pt}a{\isacharunderscore}{\kern0pt}p{\isacharcolon}{\kern0pt}\ {\isachardoublequoteopen}a\ {\isasymin}\ defer\ {\isacharparenleft}{\kern0pt}m\ {\isasymtriangleright}\ n{\isacharparenright}{\kern0pt}\ A\ p{\isachardoublequoteclose}\ \isakeyword{and}\isanewline
\ \ lifted{\isacharunderscore}{\kern0pt}a{\isacharcolon}{\kern0pt}\ {\isachardoublequoteopen}Profile{\isachardot}{\kern0pt}lifted\ A\ p\ q\ a{\isachardoublequoteclose}\isanewline
\ \ \isacommand{from}\isamarkupfalse%
\ strong{\isacharunderscore}{\kern0pt}def{\isacharunderscore}{\kern0pt}mon{\isacharunderscore}{\kern0pt}m\isanewline
\ \ \isacommand{have}\isamarkupfalse%
\ non{\isacharunderscore}{\kern0pt}electing{\isacharunderscore}{\kern0pt}m{\isacharcolon}{\kern0pt}\ {\isachardoublequoteopen}non{\isacharunderscore}{\kern0pt}electing\ m{\isachardoublequoteclose}\isanewline
\ \ \ \ \isacommand{by}\isamarkupfalse%
\ {\isacharparenleft}{\kern0pt}simp\ add{\isacharcolon}{\kern0pt}\ defer{\isacharunderscore}{\kern0pt}invariant{\isacharunderscore}{\kern0pt}monotonicity{\isacharunderscore}{\kern0pt}def{\isacharparenright}{\kern0pt}\isanewline
\ \ \isacommand{have}\isamarkupfalse%
\ electoral{\isacharunderscore}{\kern0pt}mod{\isacharunderscore}{\kern0pt}m{\isacharcolon}{\kern0pt}\ {\isachardoublequoteopen}electoral{\isacharunderscore}{\kern0pt}module\ m{\isachardoublequoteclose}\isanewline
\ \ \ \ \isacommand{using}\isamarkupfalse%
\ strong{\isacharunderscore}{\kern0pt}def{\isacharunderscore}{\kern0pt}mon{\isacharunderscore}{\kern0pt}m\ defer{\isacharunderscore}{\kern0pt}invariant{\isacharunderscore}{\kern0pt}monotonicity{\isacharunderscore}{\kern0pt}def\isanewline
\ \ \ \ \isacommand{by}\isamarkupfalse%
\ auto\isanewline
\ \ \isacommand{have}\isamarkupfalse%
\ electoral{\isacharunderscore}{\kern0pt}mod{\isacharunderscore}{\kern0pt}n{\isacharcolon}{\kern0pt}\ {\isachardoublequoteopen}electoral{\isacharunderscore}{\kern0pt}module\ n{\isachardoublequoteclose}\isanewline
\ \ \ \ \isacommand{using}\isamarkupfalse%
\ defers{\isacharunderscore}{\kern0pt}{\isadigit{1}}\ defers{\isacharunderscore}{\kern0pt}def\isanewline
\ \ \ \ \isacommand{by}\isamarkupfalse%
\ auto\isanewline
\ \ \isacommand{have}\isamarkupfalse%
\ finite{\isacharunderscore}{\kern0pt}profile{\isacharunderscore}{\kern0pt}q{\isacharcolon}{\kern0pt}\ {\isachardoublequoteopen}finite{\isacharunderscore}{\kern0pt}profile\ A\ q{\isachardoublequoteclose}\isanewline
\ \ \ \ \isacommand{using}\isamarkupfalse%
\ lifted{\isacharunderscore}{\kern0pt}a\isanewline
\ \ \ \ \isacommand{by}\isamarkupfalse%
\ {\isacharparenleft}{\kern0pt}simp\ add{\isacharcolon}{\kern0pt}\ Profile{\isachardot}{\kern0pt}lifted{\isacharunderscore}{\kern0pt}def{\isacharparenright}{\kern0pt}\isanewline
\ \ \isacommand{have}\isamarkupfalse%
\ finite{\isacharunderscore}{\kern0pt}profile{\isacharunderscore}{\kern0pt}p{\isacharcolon}{\kern0pt}\ {\isachardoublequoteopen}profile\ A\ p{\isachardoublequoteclose}\isanewline
\ \ \ \ \isacommand{using}\isamarkupfalse%
\ lifted{\isacharunderscore}{\kern0pt}a\isanewline
\ \ \ \ \isacommand{by}\isamarkupfalse%
\ {\isacharparenleft}{\kern0pt}simp\ add{\isacharcolon}{\kern0pt}\ Profile{\isachardot}{\kern0pt}lifted{\isacharunderscore}{\kern0pt}def{\isacharparenright}{\kern0pt}\isanewline
\ \ \isacommand{show}\isamarkupfalse%
\ {\isachardoublequoteopen}{\isacharparenleft}{\kern0pt}m\ {\isasymtriangleright}\ n{\isacharparenright}{\kern0pt}\ A\ p\ {\isacharequal}{\kern0pt}\ {\isacharparenleft}{\kern0pt}m\ {\isasymtriangleright}\ n{\isacharparenright}{\kern0pt}\ A\ q{\isachardoublequoteclose}\isanewline
\ \ \isacommand{proof}\isamarkupfalse%
\ cases\isanewline
\ \ \ \ \isacommand{assume}\isamarkupfalse%
\ not{\isacharunderscore}{\kern0pt}unchanged{\isacharcolon}{\kern0pt}\ {\isachardoublequoteopen}defer\ m\ A\ q\ {\isasymnoteq}\ defer\ m\ A\ p{\isachardoublequoteclose}\isanewline
\ \ \ \ \isacommand{hence}\isamarkupfalse%
\ a{\isacharunderscore}{\kern0pt}single{\isacharunderscore}{\kern0pt}defer{\isacharcolon}{\kern0pt}\ {\isachardoublequoteopen}{\isacharbraceleft}{\kern0pt}a{\isacharbraceright}{\kern0pt}\ {\isacharequal}{\kern0pt}\ defer\ m\ A\ q{\isachardoublequoteclose}\isanewline
\ \ \ \ \ \ \isacommand{using}\isamarkupfalse%
\ strong{\isacharunderscore}{\kern0pt}def{\isacharunderscore}{\kern0pt}mon{\isacharunderscore}{\kern0pt}m\ electoral{\isacharunderscore}{\kern0pt}mod{\isacharunderscore}{\kern0pt}n\ defer{\isacharunderscore}{\kern0pt}a{\isacharunderscore}{\kern0pt}p\isanewline
\ \ \ \ \ \ \ \ \ \ \ \ defer{\isacharunderscore}{\kern0pt}invariant{\isacharunderscore}{\kern0pt}monotonicity{\isacharunderscore}{\kern0pt}def\ lifted{\isacharunderscore}{\kern0pt}a\isanewline
\ \ \ \ \ \ \ \ \ \ \ \ seq{\isacharunderscore}{\kern0pt}comp{\isacharunderscore}{\kern0pt}def{\isacharunderscore}{\kern0pt}set{\isacharunderscore}{\kern0pt}trans\ finite{\isacharunderscore}{\kern0pt}profile{\isacharunderscore}{\kern0pt}p\isanewline
\ \ \ \ \ \ \ \ \ \ \ \ finite{\isacharunderscore}{\kern0pt}profile{\isacharunderscore}{\kern0pt}q\isanewline
\ \ \ \ \ \ \isacommand{by}\isamarkupfalse%
\ metis\isanewline
\ \ \ \ \isacommand{moreover}\isamarkupfalse%
\ \isacommand{have}\isamarkupfalse%
\isanewline
\ \ \ \ \ \ {\isachardoublequoteopen}{\isacharbraceleft}{\kern0pt}a{\isacharbraceright}{\kern0pt}\ {\isacharequal}{\kern0pt}\ defer\ m\ A\ q\ {\isasymlongrightarrow}\ defer\ {\isacharparenleft}{\kern0pt}m\ {\isasymtriangleright}\ n{\isacharparenright}{\kern0pt}\ A\ q\ {\isasymsubseteq}\ {\isacharbraceleft}{\kern0pt}a{\isacharbraceright}{\kern0pt}{\isachardoublequoteclose}\isanewline
\ \ \ \ \ \ \isacommand{using}\isamarkupfalse%
\ finite{\isacharunderscore}{\kern0pt}profile{\isacharunderscore}{\kern0pt}q\ electoral{\isacharunderscore}{\kern0pt}mod{\isacharunderscore}{\kern0pt}m\ electoral{\isacharunderscore}{\kern0pt}mod{\isacharunderscore}{\kern0pt}n\isanewline
\ \ \ \ \ \ \ \ \ \ \ \ seq{\isacharunderscore}{\kern0pt}comp{\isacharunderscore}{\kern0pt}def{\isacharunderscore}{\kern0pt}set{\isacharunderscore}{\kern0pt}sound\isanewline
\ \ \ \ \ \ \isacommand{by}\isamarkupfalse%
\ {\isacharparenleft}{\kern0pt}metis\ {\isacharparenleft}{\kern0pt}no{\isacharunderscore}{\kern0pt}types{\isacharcomma}{\kern0pt}\ hide{\isacharunderscore}{\kern0pt}lams{\isacharparenright}{\kern0pt}{\isacharparenright}{\kern0pt}\isanewline
\ \ \ \ \isacommand{ultimately}\isamarkupfalse%
\ \isacommand{have}\isamarkupfalse%
\isanewline
\ \ \ \ \ \ {\isachardoublequoteopen}{\isacharparenleft}{\kern0pt}a\ {\isasymin}\ defer\ m\ A\ p{\isacharparenright}{\kern0pt}\ {\isasymlongrightarrow}\ defer\ {\isacharparenleft}{\kern0pt}m\ {\isasymtriangleright}\ n{\isacharparenright}{\kern0pt}\ A\ q\ {\isasymsubseteq}\ {\isacharbraceleft}{\kern0pt}a{\isacharbraceright}{\kern0pt}{\isachardoublequoteclose}\isanewline
\ \ \ \ \ \ \isacommand{by}\isamarkupfalse%
\ blast\ \isanewline
\ \ \ \ \isacommand{moreover}\isamarkupfalse%
\ \isacommand{have}\isamarkupfalse%
\isanewline
\ \ \ \ \ \ {\isachardoublequoteopen}{\isacharparenleft}{\kern0pt}a\ {\isasymin}\ defer\ m\ A\ p{\isacharparenright}{\kern0pt}\ {\isasymlongrightarrow}\ card\ {\isacharparenleft}{\kern0pt}defer\ {\isacharparenleft}{\kern0pt}m\ {\isasymtriangleright}\ n{\isacharparenright}{\kern0pt}\ A\ q{\isacharparenright}{\kern0pt}\ {\isacharequal}{\kern0pt}\ {\isadigit{1}}{\isachardoublequoteclose}\isanewline
\ \ \ \ \ \ \isacommand{using}\isamarkupfalse%
\ One{\isacharunderscore}{\kern0pt}nat{\isacharunderscore}{\kern0pt}def\ a{\isacharunderscore}{\kern0pt}single{\isacharunderscore}{\kern0pt}defer\ card{\isacharunderscore}{\kern0pt}eq{\isacharunderscore}{\kern0pt}{\isadigit{0}}{\isacharunderscore}{\kern0pt}iff\isanewline
\ \ \ \ \ \ \ \ \ \ \ \ card{\isacharunderscore}{\kern0pt}insert{\isacharunderscore}{\kern0pt}disjoint\ defers{\isacharunderscore}{\kern0pt}{\isadigit{1}}\ defers{\isacharunderscore}{\kern0pt}def\isanewline
\ \ \ \ \ \ \ \ \ \ \ \ electoral{\isacharunderscore}{\kern0pt}mod{\isacharunderscore}{\kern0pt}m\ empty{\isacharunderscore}{\kern0pt}iff\ finite{\isachardot}{\kern0pt}emptyI\isanewline
\ \ \ \ \ \ \ \ \ \ \ \ seq{\isacharunderscore}{\kern0pt}comp{\isacharunderscore}{\kern0pt}defers{\isacharunderscore}{\kern0pt}def{\isacharunderscore}{\kern0pt}set\ order{\isacharunderscore}{\kern0pt}refl\isanewline
\ \ \ \ \ \ \ \ \ \ \ \ def{\isacharunderscore}{\kern0pt}presv{\isacharunderscore}{\kern0pt}fin{\isacharunderscore}{\kern0pt}prof\ finite{\isacharunderscore}{\kern0pt}profile{\isacharunderscore}{\kern0pt}q\isanewline
\ \ \ \ \ \ \isacommand{by}\isamarkupfalse%
\ metis\ \isanewline
\ \ \ \ \isacommand{moreover}\isamarkupfalse%
\ \isacommand{have}\isamarkupfalse%
\ defer{\isacharunderscore}{\kern0pt}a{\isacharunderscore}{\kern0pt}in{\isacharunderscore}{\kern0pt}m{\isacharunderscore}{\kern0pt}p{\isacharcolon}{\kern0pt}\isanewline
\ \ \ \ \ \ {\isachardoublequoteopen}a\ {\isasymin}\ defer\ m\ A\ p{\isachardoublequoteclose}\isanewline
\ \ \ \ \ \ \isacommand{using}\isamarkupfalse%
\ electoral{\isacharunderscore}{\kern0pt}mod{\isacharunderscore}{\kern0pt}m\ electoral{\isacharunderscore}{\kern0pt}mod{\isacharunderscore}{\kern0pt}n\ defer{\isacharunderscore}{\kern0pt}a{\isacharunderscore}{\kern0pt}p\isanewline
\ \ \ \ \ \ \ \ \ \ \ \ seq{\isacharunderscore}{\kern0pt}comp{\isacharunderscore}{\kern0pt}def{\isacharunderscore}{\kern0pt}set{\isacharunderscore}{\kern0pt}bounded\ finite{\isacharunderscore}{\kern0pt}profile{\isacharunderscore}{\kern0pt}p\isanewline
\ \ \ \ \ \ \ \ \ \ \ \ finite{\isacharunderscore}{\kern0pt}profile{\isacharunderscore}{\kern0pt}q\isanewline
\ \ \ \ \ \ \isacommand{by}\isamarkupfalse%
\ blast\isanewline
\ \ \ \ \isacommand{ultimately}\isamarkupfalse%
\ \isacommand{have}\isamarkupfalse%
\isanewline
\ \ \ \ \ \ {\isachardoublequoteopen}defer\ {\isacharparenleft}{\kern0pt}m\ {\isasymtriangleright}\ n{\isacharparenright}{\kern0pt}\ A\ q\ {\isacharequal}{\kern0pt}\ {\isacharbraceleft}{\kern0pt}a{\isacharbraceright}{\kern0pt}{\isachardoublequoteclose}\ \isanewline
\ \ \ \ \ \ \isacommand{using}\isamarkupfalse%
\ Collect{\isacharunderscore}{\kern0pt}mem{\isacharunderscore}{\kern0pt}eq\ card{\isacharunderscore}{\kern0pt}{\isadigit{1}}{\isacharunderscore}{\kern0pt}singletonE\ empty{\isacharunderscore}{\kern0pt}Collect{\isacharunderscore}{\kern0pt}eq\isanewline
\ \ \ \ \ \ \ \ \ \ \ \ insertCI\ subset{\isacharunderscore}{\kern0pt}singletonD\isanewline
\ \ \ \ \ \ \isacommand{by}\isamarkupfalse%
\ metis\isanewline
\ \ \ \ \isacommand{moreover}\isamarkupfalse%
\ \isacommand{have}\isamarkupfalse%
\isanewline
\ \ \ \ \ \ {\isachardoublequoteopen}defer\ {\isacharparenleft}{\kern0pt}m\ {\isasymtriangleright}\ n{\isacharparenright}{\kern0pt}\ A\ p\ {\isacharequal}{\kern0pt}\ {\isacharbraceleft}{\kern0pt}a{\isacharbraceright}{\kern0pt}{\isachardoublequoteclose}\ \isanewline
\ \ \ \ \ \ \isacommand{using}\isamarkupfalse%
\ card{\isacharunderscore}{\kern0pt}mono\ defers{\isacharunderscore}{\kern0pt}def\ insert{\isacharunderscore}{\kern0pt}subset\ Diff{\isacharunderscore}{\kern0pt}insert{\isacharunderscore}{\kern0pt}absorb\isanewline
\ \ \ \ \ \ \ \ \ \ \ \ seq{\isacharunderscore}{\kern0pt}comp{\isacharunderscore}{\kern0pt}def{\isacharunderscore}{\kern0pt}set{\isacharunderscore}{\kern0pt}bounded\ elect{\isacharunderscore}{\kern0pt}in{\isacharunderscore}{\kern0pt}alts\ non{\isacharunderscore}{\kern0pt}electing{\isacharunderscore}{\kern0pt}def\isanewline
\ \ \ \ \ \ \ \ \ \ \ \ non{\isacharunderscore}{\kern0pt}electing{\isacharunderscore}{\kern0pt}n\ defers{\isacharunderscore}{\kern0pt}{\isadigit{1}}\ One{\isacharunderscore}{\kern0pt}nat{\isacharunderscore}{\kern0pt}def\ card{\isacharunderscore}{\kern0pt}{\isadigit{0}}{\isacharunderscore}{\kern0pt}eq\ empty{\isacharunderscore}{\kern0pt}iff\isanewline
\ \ \ \ \ \ \ \ \ \ \ \ card{\isacharunderscore}{\kern0pt}{\isadigit{1}}{\isacharunderscore}{\kern0pt}singletonE\ card{\isacharunderscore}{\kern0pt}Diff{\isacharunderscore}{\kern0pt}singleton\ finite{\isachardot}{\kern0pt}emptyI\isanewline
\ \ \ \ \ \ \ \ \ \ \ \ card{\isacharunderscore}{\kern0pt}insert{\isacharunderscore}{\kern0pt}disjoint\ def{\isacharunderscore}{\kern0pt}presv{\isacharunderscore}{\kern0pt}fin{\isacharunderscore}{\kern0pt}prof\ defer{\isacharunderscore}{\kern0pt}a{\isacharunderscore}{\kern0pt}p\isanewline
\ \ \ \ \ \ \ \ \ \ \ \ electoral{\isacharunderscore}{\kern0pt}mod{\isacharunderscore}{\kern0pt}m\ finite{\isacharunderscore}{\kern0pt}Diff\ insertCI\ insert{\isacharunderscore}{\kern0pt}Diff\isanewline
\ \ \ \ \ \ \ \ \ \ \ \ finite{\isacharunderscore}{\kern0pt}profile{\isacharunderscore}{\kern0pt}p\ finite{\isacharunderscore}{\kern0pt}profile{\isacharunderscore}{\kern0pt}q\ seq{\isacharunderscore}{\kern0pt}comp{\isacharunderscore}{\kern0pt}defers{\isacharunderscore}{\kern0pt}def{\isacharunderscore}{\kern0pt}set\isanewline
\ \ \ \ \ \ \isacommand{by}\isamarkupfalse%
\ {\isacharparenleft}{\kern0pt}smt\ {\isacharparenleft}{\kern0pt}verit{\isacharparenright}{\kern0pt}{\isacharparenright}{\kern0pt}\isanewline
\ \ \ \ \isacommand{ultimately}\isamarkupfalse%
\ \isacommand{have}\isamarkupfalse%
\ \isanewline
\ \ \ \ \ \ {\isachardoublequoteopen}defer\ {\isacharparenleft}{\kern0pt}m\ {\isasymtriangleright}\ n{\isacharparenright}{\kern0pt}\ A\ p\ {\isacharequal}{\kern0pt}\ defer\ {\isacharparenleft}{\kern0pt}m\ {\isasymtriangleright}\ n{\isacharparenright}{\kern0pt}\ A\ q{\isachardoublequoteclose}\isanewline
\ \ \ \ \ \ \isacommand{by}\isamarkupfalse%
\ blast\isanewline
\ \ \ \ \isacommand{moreover}\isamarkupfalse%
\ \isacommand{have}\isamarkupfalse%
\ \isanewline
\ \ \ \ \ \ {\isachardoublequoteopen}elect\ {\isacharparenleft}{\kern0pt}m\ {\isasymtriangleright}\ n{\isacharparenright}{\kern0pt}\ A\ p\ {\isacharequal}{\kern0pt}\ elect\ {\isacharparenleft}{\kern0pt}m\ {\isasymtriangleright}\ n{\isacharparenright}{\kern0pt}\ A\ q{\isachardoublequoteclose}\isanewline
\ \ \ \ \ \ \isacommand{using}\isamarkupfalse%
\ finite{\isacharunderscore}{\kern0pt}profile{\isacharunderscore}{\kern0pt}p\ finite{\isacharunderscore}{\kern0pt}profile{\isacharunderscore}{\kern0pt}q\isanewline
\ \ \ \ \ \ \ \ \ \ \ \ non{\isacharunderscore}{\kern0pt}electing{\isacharunderscore}{\kern0pt}m\ non{\isacharunderscore}{\kern0pt}electing{\isacharunderscore}{\kern0pt}n\isanewline
\ \ \ \ \ \ \ \ \ \ \ \ seq{\isacharunderscore}{\kern0pt}comp{\isacharunderscore}{\kern0pt}presv{\isacharunderscore}{\kern0pt}non{\isacharunderscore}{\kern0pt}electing\isanewline
\ \ \ \ \ \ \ \ \ \ \ \ non{\isacharunderscore}{\kern0pt}electing{\isacharunderscore}{\kern0pt}def\isanewline
\ \ \ \ \ \ \isacommand{by}\isamarkupfalse%
\ metis\ \isanewline
\ \ \ \ \isacommand{thus}\isamarkupfalse%
\ {\isacharquery}{\kern0pt}thesis\isanewline
\ \ \ \ \ \ \isacommand{using}\isamarkupfalse%
\ calculation\ eq{\isacharunderscore}{\kern0pt}def{\isacharunderscore}{\kern0pt}and{\isacharunderscore}{\kern0pt}elect{\isacharunderscore}{\kern0pt}imp{\isacharunderscore}{\kern0pt}eq\isanewline
\ \ \ \ \ \ \ \ \ \ \ \ electoral{\isacharunderscore}{\kern0pt}mod{\isacharunderscore}{\kern0pt}m\ electoral{\isacharunderscore}{\kern0pt}mod{\isacharunderscore}{\kern0pt}n\isanewline
\ \ \ \ \ \ \ \ \ \ \ \ finite{\isacharunderscore}{\kern0pt}profile{\isacharunderscore}{\kern0pt}p\ seq{\isacharunderscore}{\kern0pt}comp{\isacharunderscore}{\kern0pt}sound\isanewline
\ \ \ \ \ \ \ \ \ \ \ \ finite{\isacharunderscore}{\kern0pt}profile{\isacharunderscore}{\kern0pt}q\isanewline
\ \ \ \ \ \ \isacommand{by}\isamarkupfalse%
\ metis\isanewline
\ \ \isacommand{next}\isamarkupfalse%
\isanewline
\ \ \ \ \isacommand{assume}\isamarkupfalse%
\ not{\isacharunderscore}{\kern0pt}different{\isacharunderscore}{\kern0pt}alternatives{\isacharcolon}{\kern0pt}\isanewline
\ \ \ \ \ \ {\isachardoublequoteopen}{\isasymnot}{\isacharparenleft}{\kern0pt}defer\ m\ A\ q\ {\isasymnoteq}\ defer\ m\ A\ p{\isacharparenright}{\kern0pt}{\isachardoublequoteclose}\isanewline
\ \ \ \ \isacommand{have}\isamarkupfalse%
\ {\isachardoublequoteopen}elect\ m\ A\ p\ {\isacharequal}{\kern0pt}\ {\isacharbraceleft}{\kern0pt}{\isacharbraceright}{\kern0pt}{\isachardoublequoteclose}\isanewline
\ \ \ \ \ \ \isacommand{using}\isamarkupfalse%
\ non{\isacharunderscore}{\kern0pt}electing{\isacharunderscore}{\kern0pt}m\ finite{\isacharunderscore}{\kern0pt}profile{\isacharunderscore}{\kern0pt}p\ finite{\isacharunderscore}{\kern0pt}profile{\isacharunderscore}{\kern0pt}q\isanewline
\ \ \ \ \ \ \isacommand{by}\isamarkupfalse%
\ {\isacharparenleft}{\kern0pt}simp\ add{\isacharcolon}{\kern0pt}\ non{\isacharunderscore}{\kern0pt}electing{\isacharunderscore}{\kern0pt}def{\isacharparenright}{\kern0pt}\isanewline
\ \ \ \ \isacommand{moreover}\isamarkupfalse%
\ \isacommand{have}\isamarkupfalse%
\ {\isachardoublequoteopen}elect\ m\ A\ q\ {\isacharequal}{\kern0pt}\ {\isacharbraceleft}{\kern0pt}{\isacharbraceright}{\kern0pt}{\isachardoublequoteclose}\isanewline
\ \ \ \ \ \ \isacommand{using}\isamarkupfalse%
\ non{\isacharunderscore}{\kern0pt}electing{\isacharunderscore}{\kern0pt}m\ finite{\isacharunderscore}{\kern0pt}profile{\isacharunderscore}{\kern0pt}q\isanewline
\ \ \ \ \ \ \isacommand{by}\isamarkupfalse%
\ {\isacharparenleft}{\kern0pt}simp\ add{\isacharcolon}{\kern0pt}\ non{\isacharunderscore}{\kern0pt}electing{\isacharunderscore}{\kern0pt}def{\isacharparenright}{\kern0pt}\isanewline
\ \ \ \ \isacommand{ultimately}\isamarkupfalse%
\ \isacommand{have}\isamarkupfalse%
\ elect{\isacharunderscore}{\kern0pt}m{\isacharunderscore}{\kern0pt}equal{\isacharcolon}{\kern0pt}\isanewline
\ \ \ \ \ \ {\isachardoublequoteopen}elect\ m\ A\ p\ {\isacharequal}{\kern0pt}\ elect\ m\ A\ q{\isachardoublequoteclose}\isanewline
\ \ \ \ \ \ \isacommand{by}\isamarkupfalse%
\ simp\ \isanewline
\ \ \ \ \isacommand{from}\isamarkupfalse%
\ not{\isacharunderscore}{\kern0pt}different{\isacharunderscore}{\kern0pt}alternatives\isanewline
\ \ \ \ \isacommand{have}\isamarkupfalse%
\ same{\isacharunderscore}{\kern0pt}alternatives{\isacharcolon}{\kern0pt}\ {\isachardoublequoteopen}defer\ m\ A\ q\ {\isacharequal}{\kern0pt}\ defer\ m\ A\ p{\isachardoublequoteclose}\isanewline
\ \ \ \ \ \ \isacommand{by}\isamarkupfalse%
\ simp\isanewline
\ \ \ \ \isacommand{hence}\isamarkupfalse%
\isanewline
\ \ \ \ \ \ {\isachardoublequoteopen}{\isacharparenleft}{\kern0pt}limit{\isacharunderscore}{\kern0pt}profile\ {\isacharparenleft}{\kern0pt}defer\ m\ A\ p{\isacharparenright}{\kern0pt}\ p{\isacharparenright}{\kern0pt}\ {\isacharequal}{\kern0pt}\isanewline
\ \ \ \ \ \ \ \ {\isacharparenleft}{\kern0pt}limit{\isacharunderscore}{\kern0pt}profile\ {\isacharparenleft}{\kern0pt}defer\ m\ A\ p{\isacharparenright}{\kern0pt}\ q{\isacharparenright}{\kern0pt}\ {\isasymor}\isanewline
\ \ \ \ \ \ \ \ \ \ lifted\ {\isacharparenleft}{\kern0pt}defer\ m\ A\ q{\isacharparenright}{\kern0pt}\isanewline
\ \ \ \ \ \ \ \ \ \ \ \ {\isacharparenleft}{\kern0pt}limit{\isacharunderscore}{\kern0pt}profile\ {\isacharparenleft}{\kern0pt}defer\ m\ A\ p{\isacharparenright}{\kern0pt}\ p{\isacharparenright}{\kern0pt}\isanewline
\ \ \ \ \ \ \ \ \ \ \ \ \ \ {\isacharparenleft}{\kern0pt}limit{\isacharunderscore}{\kern0pt}profile\ {\isacharparenleft}{\kern0pt}defer\ m\ A\ p{\isacharparenright}{\kern0pt}\ q{\isacharparenright}{\kern0pt}\ a{\isachardoublequoteclose}\isanewline
\ \ \ \ \ \ \isacommand{using}\isamarkupfalse%
\ defer{\isacharunderscore}{\kern0pt}in{\isacharunderscore}{\kern0pt}alts\ electoral{\isacharunderscore}{\kern0pt}mod{\isacharunderscore}{\kern0pt}m\isanewline
\ \ \ \ \ \ \ \ \ \ \ \ lifted{\isacharunderscore}{\kern0pt}a\ finite{\isacharunderscore}{\kern0pt}profile{\isacharunderscore}{\kern0pt}q\isanewline
\ \ \ \ \ \ \ \ \ \ \ \ limit{\isacharunderscore}{\kern0pt}prof{\isacharunderscore}{\kern0pt}eq{\isacharunderscore}{\kern0pt}or{\isacharunderscore}{\kern0pt}lifted\isanewline
\ \ \ \ \ \ \isacommand{by}\isamarkupfalse%
\ metis\isanewline
\ \ \ \ \isacommand{thus}\isamarkupfalse%
\ {\isacharquery}{\kern0pt}thesis\isanewline
\ \ \ \ \isacommand{proof}\isamarkupfalse%
\isanewline
\ \ \ \ \ \ \isacommand{assume}\isamarkupfalse%
\isanewline
\ \ \ \ \ \ \ \ {\isachardoublequoteopen}limit{\isacharunderscore}{\kern0pt}profile\ {\isacharparenleft}{\kern0pt}defer\ m\ A\ p{\isacharparenright}{\kern0pt}\ p\ {\isacharequal}{\kern0pt}\isanewline
\ \ \ \ \ \ \ \ \ \ limit{\isacharunderscore}{\kern0pt}profile\ {\isacharparenleft}{\kern0pt}defer\ m\ A\ p{\isacharparenright}{\kern0pt}\ q{\isachardoublequoteclose}\isanewline
\ \ \ \ \ \ \isacommand{hence}\isamarkupfalse%
\ same{\isacharunderscore}{\kern0pt}profile{\isacharcolon}{\kern0pt}\isanewline
\ \ \ \ \ \ \ \ {\isachardoublequoteopen}limit{\isacharunderscore}{\kern0pt}profile\ {\isacharparenleft}{\kern0pt}defer\ m\ A\ p{\isacharparenright}{\kern0pt}\ p\ {\isacharequal}{\kern0pt}\isanewline
\ \ \ \ \ \ \ \ \ \ limit{\isacharunderscore}{\kern0pt}profile\ {\isacharparenleft}{\kern0pt}defer\ m\ A\ q{\isacharparenright}{\kern0pt}\ q{\isachardoublequoteclose}\isanewline
\ \ \ \ \ \ \ \ \isacommand{using}\isamarkupfalse%
\ same{\isacharunderscore}{\kern0pt}alternatives\isanewline
\ \ \ \ \ \ \ \ \isacommand{by}\isamarkupfalse%
\ simp\isanewline
\ \ \ \ \ \ \isacommand{hence}\isamarkupfalse%
\ results{\isacharunderscore}{\kern0pt}equal{\isacharunderscore}{\kern0pt}n{\isacharcolon}{\kern0pt}\isanewline
\ \ \ \ \ \ \ \ {\isachardoublequoteopen}n\ {\isacharparenleft}{\kern0pt}defer\ m\ A\ q{\isacharparenright}{\kern0pt}\ {\isacharparenleft}{\kern0pt}limit{\isacharunderscore}{\kern0pt}profile\ {\isacharparenleft}{\kern0pt}defer\ m\ A\ q{\isacharparenright}{\kern0pt}\ q{\isacharparenright}{\kern0pt}\ {\isacharequal}{\kern0pt}\isanewline
\ \ \ \ \ \ \ \ \ \ n\ {\isacharparenleft}{\kern0pt}defer\ m\ A\ p{\isacharparenright}{\kern0pt}\ {\isacharparenleft}{\kern0pt}limit{\isacharunderscore}{\kern0pt}profile\ {\isacharparenleft}{\kern0pt}defer\ m\ A\ p{\isacharparenright}{\kern0pt}\ p{\isacharparenright}{\kern0pt}{\isachardoublequoteclose}\isanewline
\ \ \ \ \ \ \ \ \isacommand{by}\isamarkupfalse%
\ {\isacharparenleft}{\kern0pt}simp\ add{\isacharcolon}{\kern0pt}\ same{\isacharunderscore}{\kern0pt}alternatives{\isacharparenright}{\kern0pt}\isanewline
\ \ \ \ \ \ \isacommand{moreover}\isamarkupfalse%
\ \isacommand{have}\isamarkupfalse%
\ results{\isacharunderscore}{\kern0pt}equal{\isacharunderscore}{\kern0pt}m{\isacharcolon}{\kern0pt}\ {\isachardoublequoteopen}m\ A\ p\ {\isacharequal}{\kern0pt}\ m\ A\ q{\isachardoublequoteclose}\isanewline
\ \ \ \ \ \ \ \ \isacommand{using}\isamarkupfalse%
\ elect{\isacharunderscore}{\kern0pt}m{\isacharunderscore}{\kern0pt}equal\ same{\isacharunderscore}{\kern0pt}alternatives\isanewline
\ \ \ \ \ \ \ \ \ \ \ \ \ \ finite{\isacharunderscore}{\kern0pt}profile{\isacharunderscore}{\kern0pt}p\ finite{\isacharunderscore}{\kern0pt}profile{\isacharunderscore}{\kern0pt}q\isanewline
\ \ \ \ \ \ \ \ \isacommand{by}\isamarkupfalse%
\ {\isacharparenleft}{\kern0pt}simp\ add{\isacharcolon}{\kern0pt}\ electoral{\isacharunderscore}{\kern0pt}mod{\isacharunderscore}{\kern0pt}m\ eq{\isacharunderscore}{\kern0pt}def{\isacharunderscore}{\kern0pt}and{\isacharunderscore}{\kern0pt}elect{\isacharunderscore}{\kern0pt}imp{\isacharunderscore}{\kern0pt}eq{\isacharparenright}{\kern0pt}\isanewline
\ \ \ \ \ \ \isacommand{hence}\isamarkupfalse%
\ {\isachardoublequoteopen}{\isacharparenleft}{\kern0pt}m\ {\isasymtriangleright}\ n{\isacharparenright}{\kern0pt}\ A\ p\ {\isacharequal}{\kern0pt}\ {\isacharparenleft}{\kern0pt}m\ {\isasymtriangleright}\ n{\isacharparenright}{\kern0pt}\ A\ q{\isachardoublequoteclose}\isanewline
\ \ \ \ \ \ \ \ \isacommand{using}\isamarkupfalse%
\ same{\isacharunderscore}{\kern0pt}profile\isanewline
\ \ \ \ \ \ \ \ \isacommand{by}\isamarkupfalse%
\ auto\isanewline
\ \ \ \ \ \ \isacommand{thus}\isamarkupfalse%
\ {\isacharquery}{\kern0pt}thesis\isanewline
\ \ \ \ \ \ \ \ \isacommand{by}\isamarkupfalse%
\ blast\isanewline
\ \ \ \ \isacommand{next}\isamarkupfalse%
\isanewline
\ \ \ \ \ \ \isacommand{assume}\isamarkupfalse%
\ still{\isacharunderscore}{\kern0pt}lifted{\isacharcolon}{\kern0pt}\isanewline
\ \ \ \ \ \ \ \ {\isachardoublequoteopen}lifted\ {\isacharparenleft}{\kern0pt}defer\ m\ A\ q{\isacharparenright}{\kern0pt}\ {\isacharparenleft}{\kern0pt}limit{\isacharunderscore}{\kern0pt}profile\ {\isacharparenleft}{\kern0pt}defer\ m\ A\ p{\isacharparenright}{\kern0pt}\ p{\isacharparenright}{\kern0pt}\isanewline
\ \ \ \ \ \ \ \ \ \ {\isacharparenleft}{\kern0pt}limit{\isacharunderscore}{\kern0pt}profile\ {\isacharparenleft}{\kern0pt}defer\ m\ A\ p{\isacharparenright}{\kern0pt}\ q{\isacharparenright}{\kern0pt}\ a{\isachardoublequoteclose}\isanewline
\ \ \ \ \ \ \isacommand{hence}\isamarkupfalse%
\ a{\isacharunderscore}{\kern0pt}in{\isacharunderscore}{\kern0pt}def{\isacharunderscore}{\kern0pt}p{\isacharcolon}{\kern0pt}\isanewline
\ \ \ \ \ \ \ \ {\isachardoublequoteopen}a\ {\isasymin}\ defer\ n\ {\isacharparenleft}{\kern0pt}defer\ m\ A\ p{\isacharparenright}{\kern0pt}\isanewline
\ \ \ \ \ \ \ \ \ \ {\isacharparenleft}{\kern0pt}limit{\isacharunderscore}{\kern0pt}profile\ {\isacharparenleft}{\kern0pt}defer\ m\ A\ p{\isacharparenright}{\kern0pt}\ p{\isacharparenright}{\kern0pt}{\isachardoublequoteclose}\isanewline
\ \ \ \ \ \ \ \ \isacommand{using}\isamarkupfalse%
\ electoral{\isacharunderscore}{\kern0pt}mod{\isacharunderscore}{\kern0pt}m\ electoral{\isacharunderscore}{\kern0pt}mod{\isacharunderscore}{\kern0pt}n\isanewline
\ \ \ \ \ \ \ \ \ \ \ \ \ \ finite{\isacharunderscore}{\kern0pt}profile{\isacharunderscore}{\kern0pt}p\ defer{\isacharunderscore}{\kern0pt}a{\isacharunderscore}{\kern0pt}p\isanewline
\ \ \ \ \ \ \ \ \ \ \ \ \ \ seq{\isacharunderscore}{\kern0pt}comp{\isacharunderscore}{\kern0pt}def{\isacharunderscore}{\kern0pt}set{\isacharunderscore}{\kern0pt}trans\isanewline
\ \ \ \ \ \ \ \ \ \ \ \ \ \ finite{\isacharunderscore}{\kern0pt}profile{\isacharunderscore}{\kern0pt}q\isanewline
\ \ \ \ \ \ \ \ \isacommand{by}\isamarkupfalse%
\ metis\isanewline
\ \ \ \ \ \ \isacommand{hence}\isamarkupfalse%
\ a{\isacharunderscore}{\kern0pt}still{\isacharunderscore}{\kern0pt}deferred{\isacharunderscore}{\kern0pt}p{\isacharcolon}{\kern0pt}\isanewline
\ \ \ \ \ \ \ \ {\isachardoublequoteopen}{\isacharbraceleft}{\kern0pt}a{\isacharbraceright}{\kern0pt}\ {\isasymsubseteq}\ defer\ n\ {\isacharparenleft}{\kern0pt}defer\ m\ A\ p{\isacharparenright}{\kern0pt}\isanewline
\ \ \ \ \ \ \ \ \ \ {\isacharparenleft}{\kern0pt}limit{\isacharunderscore}{\kern0pt}profile\ {\isacharparenleft}{\kern0pt}defer\ m\ A\ p{\isacharparenright}{\kern0pt}\ p{\isacharparenright}{\kern0pt}{\isachardoublequoteclose}\isanewline
\ \ \ \ \ \ \ \ \isacommand{by}\isamarkupfalse%
\ simp\isanewline
\ \ \ \ \ \ \isacommand{have}\isamarkupfalse%
\ card{\isacharunderscore}{\kern0pt}le{\isacharunderscore}{\kern0pt}{\isadigit{1}}{\isacharunderscore}{\kern0pt}p{\isacharcolon}{\kern0pt}\ {\isachardoublequoteopen}card\ {\isacharparenleft}{\kern0pt}defer\ m\ A\ p{\isacharparenright}{\kern0pt}\ {\isasymge}\ {\isadigit{1}}{\isachardoublequoteclose}\isanewline
\ \ \ \ \ \ \ \ \isacommand{using}\isamarkupfalse%
\ One{\isacharunderscore}{\kern0pt}nat{\isacharunderscore}{\kern0pt}def\ Suc{\isacharunderscore}{\kern0pt}leI\ card{\isacharunderscore}{\kern0pt}gt{\isacharunderscore}{\kern0pt}{\isadigit{0}}{\isacharunderscore}{\kern0pt}iff\isanewline
\ \ \ \ \ \ \ \ \ \ \ \ \ \ electoral{\isacharunderscore}{\kern0pt}mod{\isacharunderscore}{\kern0pt}m\ electoral{\isacharunderscore}{\kern0pt}mod{\isacharunderscore}{\kern0pt}n\isanewline
\ \ \ \ \ \ \ \ \ \ \ \ \ \ equals{\isadigit{0}}D\ finite{\isacharunderscore}{\kern0pt}profile{\isacharunderscore}{\kern0pt}p\ defer{\isacharunderscore}{\kern0pt}a{\isacharunderscore}{\kern0pt}p\isanewline
\ \ \ \ \ \ \ \ \ \ \ \ \ \ seq{\isacharunderscore}{\kern0pt}comp{\isacharunderscore}{\kern0pt}def{\isacharunderscore}{\kern0pt}set{\isacharunderscore}{\kern0pt}trans\ def{\isacharunderscore}{\kern0pt}presv{\isacharunderscore}{\kern0pt}fin{\isacharunderscore}{\kern0pt}prof\isanewline
\ \ \ \ \ \ \ \ \ \ \ \ \ \ finite{\isacharunderscore}{\kern0pt}profile{\isacharunderscore}{\kern0pt}q\isanewline
\ \ \ \ \ \ \ \ \isacommand{by}\isamarkupfalse%
\ metis\isanewline
\ \ \ \ \ \ \isacommand{hence}\isamarkupfalse%
\isanewline
\ \ \ \ \ \ \ \ {\isachardoublequoteopen}card\ {\isacharparenleft}{\kern0pt}defer\ n\ {\isacharparenleft}{\kern0pt}defer\ m\ A\ p{\isacharparenright}{\kern0pt}\isanewline
\ \ \ \ \ \ \ \ \ \ {\isacharparenleft}{\kern0pt}limit{\isacharunderscore}{\kern0pt}profile\ {\isacharparenleft}{\kern0pt}defer\ m\ A\ p{\isacharparenright}{\kern0pt}\ p{\isacharparenright}{\kern0pt}{\isacharparenright}{\kern0pt}\ {\isacharequal}{\kern0pt}\ {\isadigit{1}}{\isachardoublequoteclose}\isanewline
\ \ \ \ \ \ \ \ \isacommand{using}\isamarkupfalse%
\ defers{\isacharunderscore}{\kern0pt}{\isadigit{1}}\ defers{\isacharunderscore}{\kern0pt}def\ electoral{\isacharunderscore}{\kern0pt}mod{\isacharunderscore}{\kern0pt}m\isanewline
\ \ \ \ \ \ \ \ \ \ \ \ \ \ finite{\isacharunderscore}{\kern0pt}profile{\isacharunderscore}{\kern0pt}p\ def{\isacharunderscore}{\kern0pt}presv{\isacharunderscore}{\kern0pt}fin{\isacharunderscore}{\kern0pt}prof\isanewline
\ \ \ \ \ \ \ \ \ \ \ \ \ \ finite{\isacharunderscore}{\kern0pt}profile{\isacharunderscore}{\kern0pt}q\isanewline
\ \ \ \ \ \ \ \ \isacommand{by}\isamarkupfalse%
\ metis\isanewline
\ \ \ \ \ \ \isacommand{hence}\isamarkupfalse%
\ def{\isacharunderscore}{\kern0pt}set{\isacharunderscore}{\kern0pt}is{\isacharunderscore}{\kern0pt}a{\isacharunderscore}{\kern0pt}p{\isacharcolon}{\kern0pt}\isanewline
\ \ \ \ \ \ \ \ {\isachardoublequoteopen}{\isacharbraceleft}{\kern0pt}a{\isacharbraceright}{\kern0pt}\ {\isacharequal}{\kern0pt}\ defer\ n\ {\isacharparenleft}{\kern0pt}defer\ m\ A\ p{\isacharparenright}{\kern0pt}\ {\isacharparenleft}{\kern0pt}limit{\isacharunderscore}{\kern0pt}profile\ {\isacharparenleft}{\kern0pt}defer\ m\ A\ p{\isacharparenright}{\kern0pt}\ p{\isacharparenright}{\kern0pt}{\isachardoublequoteclose}\isanewline
\ \ \ \ \ \ \ \ \isacommand{using}\isamarkupfalse%
\ a{\isacharunderscore}{\kern0pt}still{\isacharunderscore}{\kern0pt}deferred{\isacharunderscore}{\kern0pt}p\ card{\isacharunderscore}{\kern0pt}{\isadigit{1}}{\isacharunderscore}{\kern0pt}singletonE\isanewline
\ \ \ \ \ \ \ \ \ \ \ \ \ \ insert{\isacharunderscore}{\kern0pt}subset\ singletonD\isanewline
\ \ \ \ \ \ \ \ \isacommand{by}\isamarkupfalse%
\ metis\isanewline
\ \ \ \ \ \ \isacommand{have}\isamarkupfalse%
\ a{\isacharunderscore}{\kern0pt}still{\isacharunderscore}{\kern0pt}deferred{\isacharunderscore}{\kern0pt}q{\isacharcolon}{\kern0pt}\isanewline
\ \ \ \ \ \ \ \ {\isachardoublequoteopen}a\ {\isasymin}\ defer\ n\ {\isacharparenleft}{\kern0pt}defer\ m\ A\ q{\isacharparenright}{\kern0pt}\isanewline
\ \ \ \ \ \ \ \ \ \ {\isacharparenleft}{\kern0pt}limit{\isacharunderscore}{\kern0pt}profile\ {\isacharparenleft}{\kern0pt}defer\ m\ A\ p{\isacharparenright}{\kern0pt}\ q{\isacharparenright}{\kern0pt}{\isachardoublequoteclose}\isanewline
\ \ \ \ \ \ \ \ \isacommand{using}\isamarkupfalse%
\ still{\isacharunderscore}{\kern0pt}lifted\ a{\isacharunderscore}{\kern0pt}in{\isacharunderscore}{\kern0pt}def{\isacharunderscore}{\kern0pt}p\isanewline
\ \ \ \ \ \ \ \ \ \ \ \ \ \ defer{\isacharunderscore}{\kern0pt}monotonicity{\isacharunderscore}{\kern0pt}def\isanewline
\ \ \ \ \ \ \ \ \ \ \ \ \ \ defer{\isacharunderscore}{\kern0pt}monotone{\isacharunderscore}{\kern0pt}n\ electoral{\isacharunderscore}{\kern0pt}mod{\isacharunderscore}{\kern0pt}m\isanewline
\ \ \ \ \ \ \ \ \ \ \ \ \ \ same{\isacharunderscore}{\kern0pt}alternatives\isanewline
\ \ \ \ \ \ \ \ \ \ \ \ \ \ def{\isacharunderscore}{\kern0pt}presv{\isacharunderscore}{\kern0pt}fin{\isacharunderscore}{\kern0pt}prof\ finite{\isacharunderscore}{\kern0pt}profile{\isacharunderscore}{\kern0pt}q\isanewline
\ \ \ \ \ \ \ \ \isacommand{by}\isamarkupfalse%
\ metis\isanewline
\ \ \ \ \ \ \isacommand{have}\isamarkupfalse%
\ {\isachardoublequoteopen}card\ {\isacharparenleft}{\kern0pt}defer\ m\ A\ q{\isacharparenright}{\kern0pt}\ {\isasymge}\ {\isadigit{1}}{\isachardoublequoteclose}\isanewline
\ \ \ \ \ \ \ \ \isacommand{using}\isamarkupfalse%
\ card{\isacharunderscore}{\kern0pt}le{\isacharunderscore}{\kern0pt}{\isadigit{1}}{\isacharunderscore}{\kern0pt}p\ same{\isacharunderscore}{\kern0pt}alternatives\isanewline
\ \ \ \ \ \ \ \ \isacommand{by}\isamarkupfalse%
\ auto\isanewline
\ \ \ \ \ \ \isacommand{hence}\isamarkupfalse%
\isanewline
\ \ \ \ \ \ \ \ {\isachardoublequoteopen}card\ {\isacharparenleft}{\kern0pt}defer\ n\ {\isacharparenleft}{\kern0pt}defer\ m\ A\ q{\isacharparenright}{\kern0pt}\isanewline
\ \ \ \ \ \ \ \ \ \ {\isacharparenleft}{\kern0pt}limit{\isacharunderscore}{\kern0pt}profile\ {\isacharparenleft}{\kern0pt}defer\ m\ A\ q{\isacharparenright}{\kern0pt}\ q{\isacharparenright}{\kern0pt}{\isacharparenright}{\kern0pt}\ {\isacharequal}{\kern0pt}\ {\isadigit{1}}{\isachardoublequoteclose}\isanewline
\ \ \ \ \ \ \ \ \isacommand{using}\isamarkupfalse%
\ defers{\isacharunderscore}{\kern0pt}{\isadigit{1}}\ defers{\isacharunderscore}{\kern0pt}def\ electoral{\isacharunderscore}{\kern0pt}mod{\isacharunderscore}{\kern0pt}m\isanewline
\ \ \ \ \ \ \ \ \ \ \ \ \ \ finite{\isacharunderscore}{\kern0pt}profile{\isacharunderscore}{\kern0pt}q\ def{\isacharunderscore}{\kern0pt}presv{\isacharunderscore}{\kern0pt}fin{\isacharunderscore}{\kern0pt}prof\isanewline
\ \ \ \ \ \ \ \ \isacommand{by}\isamarkupfalse%
\ metis\isanewline
\ \ \ \ \ \ \isacommand{hence}\isamarkupfalse%
\ def{\isacharunderscore}{\kern0pt}set{\isacharunderscore}{\kern0pt}is{\isacharunderscore}{\kern0pt}a{\isacharunderscore}{\kern0pt}q{\isacharcolon}{\kern0pt}\isanewline
\ \ \ \ \ \ \ \ {\isachardoublequoteopen}{\isacharbraceleft}{\kern0pt}a{\isacharbraceright}{\kern0pt}\ {\isacharequal}{\kern0pt}\isanewline
\ \ \ \ \ \ \ \ \ \ defer\ n\ {\isacharparenleft}{\kern0pt}defer\ m\ A\ q{\isacharparenright}{\kern0pt}\isanewline
\ \ \ \ \ \ \ \ \ \ \ \ {\isacharparenleft}{\kern0pt}limit{\isacharunderscore}{\kern0pt}profile\ {\isacharparenleft}{\kern0pt}defer\ m\ A\ q{\isacharparenright}{\kern0pt}\ q{\isacharparenright}{\kern0pt}{\isachardoublequoteclose}\isanewline
\ \ \ \ \ \ \ \ \isacommand{using}\isamarkupfalse%
\ a{\isacharunderscore}{\kern0pt}still{\isacharunderscore}{\kern0pt}deferred{\isacharunderscore}{\kern0pt}q\ card{\isacharunderscore}{\kern0pt}{\isadigit{1}}{\isacharunderscore}{\kern0pt}singletonE\isanewline
\ \ \ \ \ \ \ \ \ \ \ \ \ \ same{\isacharunderscore}{\kern0pt}alternatives\ singletonD\isanewline
\ \ \ \ \ \ \ \ \isacommand{by}\isamarkupfalse%
\ metis\isanewline
\ \ \ \ \ \ \isacommand{have}\isamarkupfalse%
\isanewline
\ \ \ \ \ \ \ \ {\isachardoublequoteopen}defer\ n\ {\isacharparenleft}{\kern0pt}defer\ m\ A\ p{\isacharparenright}{\kern0pt}\isanewline
\ \ \ \ \ \ \ \ \ \ {\isacharparenleft}{\kern0pt}limit{\isacharunderscore}{\kern0pt}profile\ {\isacharparenleft}{\kern0pt}defer\ m\ A\ p{\isacharparenright}{\kern0pt}\ p{\isacharparenright}{\kern0pt}\ {\isacharequal}{\kern0pt}\isanewline
\ \ \ \ \ \ \ \ \ \ \ \ defer\ n\ {\isacharparenleft}{\kern0pt}defer\ m\ A\ q{\isacharparenright}{\kern0pt}\isanewline
\ \ \ \ \ \ \ \ \ \ \ \ \ \ {\isacharparenleft}{\kern0pt}limit{\isacharunderscore}{\kern0pt}profile\ {\isacharparenleft}{\kern0pt}defer\ m\ A\ q{\isacharparenright}{\kern0pt}\ q{\isacharparenright}{\kern0pt}{\isachardoublequoteclose}\isanewline
\ \ \ \ \ \ \ \ \isacommand{using}\isamarkupfalse%
\ def{\isacharunderscore}{\kern0pt}set{\isacharunderscore}{\kern0pt}is{\isacharunderscore}{\kern0pt}a{\isacharunderscore}{\kern0pt}q\ def{\isacharunderscore}{\kern0pt}set{\isacharunderscore}{\kern0pt}is{\isacharunderscore}{\kern0pt}a{\isacharunderscore}{\kern0pt}p\isanewline
\ \ \ \ \ \ \ \ \isacommand{by}\isamarkupfalse%
\ auto\isanewline
\ \ \ \ \ \ \isacommand{thus}\isamarkupfalse%
\ {\isacharquery}{\kern0pt}thesis\isanewline
\ \ \ \ \ \ \ \ \isacommand{using}\isamarkupfalse%
\ seq{\isacharunderscore}{\kern0pt}comp{\isacharunderscore}{\kern0pt}presv{\isacharunderscore}{\kern0pt}non{\isacharunderscore}{\kern0pt}electing\isanewline
\ \ \ \ \ \ \ \ \ \ \ \ \ \ eq{\isacharunderscore}{\kern0pt}def{\isacharunderscore}{\kern0pt}and{\isacharunderscore}{\kern0pt}elect{\isacharunderscore}{\kern0pt}imp{\isacharunderscore}{\kern0pt}eq\ non{\isacharunderscore}{\kern0pt}electing{\isacharunderscore}{\kern0pt}def\isanewline
\ \ \ \ \ \ \ \ \ \ \ \ \ \ finite{\isacharunderscore}{\kern0pt}profile{\isacharunderscore}{\kern0pt}p\ finite{\isacharunderscore}{\kern0pt}profile{\isacharunderscore}{\kern0pt}q\isanewline
\ \ \ \ \ \ \ \ \ \ \ \ \ \ non{\isacharunderscore}{\kern0pt}electing{\isacharunderscore}{\kern0pt}m\ non{\isacharunderscore}{\kern0pt}electing{\isacharunderscore}{\kern0pt}n\isanewline
\ \ \ \ \ \ \ \ \ \ \ \ \ \ seq{\isacharunderscore}{\kern0pt}comp{\isacharunderscore}{\kern0pt}defers{\isacharunderscore}{\kern0pt}def{\isacharunderscore}{\kern0pt}set\isanewline
\ \ \ \ \ \ \ \ \isacommand{by}\isamarkupfalse%
\ metis\isanewline
\ \ \ \ \isacommand{qed}\isamarkupfalse%
\isanewline
\ \ \isacommand{qed}\isamarkupfalse%
\isanewline
\isacommand{qed}\isamarkupfalse%
%
\endisatagproof
{\isafoldproof}%
%
\isadelimproof
\isanewline
%
\endisadelimproof
\isanewline
\isanewline
\isacommand{theorem}\isamarkupfalse%
\ par{\isacharunderscore}{\kern0pt}comp{\isacharunderscore}{\kern0pt}def{\isacharunderscore}{\kern0pt}lift{\isacharunderscore}{\kern0pt}inv{\isacharbrackleft}{\kern0pt}simp{\isacharbrackright}{\kern0pt}{\isacharcolon}{\kern0pt}\isanewline
\ \ \isakeyword{assumes}\isanewline
\ \ \ \ compatible{\isacharcolon}{\kern0pt}\ {\isachardoublequoteopen}disjoint{\isacharunderscore}{\kern0pt}compatibility\ m\ n{\isachardoublequoteclose}\ \isakeyword{and}\isanewline
\ \ \ \ monotone{\isacharunderscore}{\kern0pt}m{\isacharcolon}{\kern0pt}\ {\isachardoublequoteopen}defer{\isacharunderscore}{\kern0pt}lift{\isacharunderscore}{\kern0pt}invariance\ m{\isachardoublequoteclose}\ \isakeyword{and}\isanewline
\ \ \ \ monotone{\isacharunderscore}{\kern0pt}n{\isacharcolon}{\kern0pt}\ {\isachardoublequoteopen}defer{\isacharunderscore}{\kern0pt}lift{\isacharunderscore}{\kern0pt}invariance\ n{\isachardoublequoteclose}\isanewline
\ \ \isakeyword{shows}\ {\isachardoublequoteopen}defer{\isacharunderscore}{\kern0pt}lift{\isacharunderscore}{\kern0pt}invariance\ {\isacharparenleft}{\kern0pt}m\ {\isasymparallel}\isactrlsub {\isasymup}\ n{\isacharparenright}{\kern0pt}{\isachardoublequoteclose}\isanewline
%
\isadelimproof
\ \ %
\endisadelimproof
%
\isatagproof
\isacommand{unfolding}\isamarkupfalse%
\ defer{\isacharunderscore}{\kern0pt}lift{\isacharunderscore}{\kern0pt}invariance{\isacharunderscore}{\kern0pt}def\isanewline
\isacommand{proof}\isamarkupfalse%
\ {\isacharparenleft}{\kern0pt}safe{\isacharparenright}{\kern0pt}\isanewline
\ \ \isacommand{have}\isamarkupfalse%
\ electoral{\isacharunderscore}{\kern0pt}mod{\isacharunderscore}{\kern0pt}m{\isacharcolon}{\kern0pt}\ {\isachardoublequoteopen}electoral{\isacharunderscore}{\kern0pt}module\ m{\isachardoublequoteclose}\isanewline
\ \ \ \ \isacommand{using}\isamarkupfalse%
\ monotone{\isacharunderscore}{\kern0pt}m\isanewline
\ \ \ \ \isacommand{by}\isamarkupfalse%
\ {\isacharparenleft}{\kern0pt}simp\ add{\isacharcolon}{\kern0pt}\ defer{\isacharunderscore}{\kern0pt}lift{\isacharunderscore}{\kern0pt}invariance{\isacharunderscore}{\kern0pt}def{\isacharparenright}{\kern0pt}\isanewline
\ \ \isacommand{have}\isamarkupfalse%
\ electoral{\isacharunderscore}{\kern0pt}mod{\isacharunderscore}{\kern0pt}n{\isacharcolon}{\kern0pt}\ {\isachardoublequoteopen}electoral{\isacharunderscore}{\kern0pt}module\ n{\isachardoublequoteclose}\isanewline
\ \ \ \ \isacommand{using}\isamarkupfalse%
\ monotone{\isacharunderscore}{\kern0pt}n\isanewline
\ \ \ \ \isacommand{by}\isamarkupfalse%
\ {\isacharparenleft}{\kern0pt}simp\ add{\isacharcolon}{\kern0pt}\ defer{\isacharunderscore}{\kern0pt}lift{\isacharunderscore}{\kern0pt}invariance{\isacharunderscore}{\kern0pt}def{\isacharparenright}{\kern0pt}\isanewline
\ \ \isacommand{show}\isamarkupfalse%
\ {\isachardoublequoteopen}electoral{\isacharunderscore}{\kern0pt}module\ {\isacharparenleft}{\kern0pt}m\ {\isasymparallel}\isactrlsub {\isasymup}\ n{\isacharparenright}{\kern0pt}{\isachardoublequoteclose}\isanewline
\ \ \ \ \isacommand{using}\isamarkupfalse%
\ electoral{\isacharunderscore}{\kern0pt}mod{\isacharunderscore}{\kern0pt}m\ electoral{\isacharunderscore}{\kern0pt}mod{\isacharunderscore}{\kern0pt}n\isanewline
\ \ \ \ \isacommand{by}\isamarkupfalse%
\ simp\isanewline
\isacommand{next}\isamarkupfalse%
\isanewline
\ \ \isacommand{fix}\isamarkupfalse%
\isanewline
\ \ \ \ S\ {\isacharcolon}{\kern0pt}{\isacharcolon}{\kern0pt}\ {\isachardoublequoteopen}{\isacharprime}{\kern0pt}a\ set{\isachardoublequoteclose}\ \isakeyword{and}\isanewline
\ \ \ \ p\ {\isacharcolon}{\kern0pt}{\isacharcolon}{\kern0pt}\ {\isachardoublequoteopen}{\isacharprime}{\kern0pt}a\ Profile{\isachardoublequoteclose}\ \isakeyword{and}\isanewline
\ \ \ \ q\ {\isacharcolon}{\kern0pt}{\isacharcolon}{\kern0pt}\ {\isachardoublequoteopen}{\isacharprime}{\kern0pt}a\ Profile{\isachardoublequoteclose}\ \isakeyword{and}\isanewline
\ \ \ \ x\ {\isacharcolon}{\kern0pt}{\isacharcolon}{\kern0pt}\ {\isachardoublequoteopen}{\isacharprime}{\kern0pt}a{\isachardoublequoteclose}\isanewline
\ \ \isacommand{assume}\isamarkupfalse%
\isanewline
\ \ \ \ defer{\isacharunderscore}{\kern0pt}x{\isacharcolon}{\kern0pt}\ {\isachardoublequoteopen}x\ {\isasymin}\ defer\ {\isacharparenleft}{\kern0pt}m\ {\isasymparallel}\isactrlsub {\isasymup}\ n{\isacharparenright}{\kern0pt}\ S\ p{\isachardoublequoteclose}\ \isakeyword{and}\isanewline
\ \ \ \ lifted{\isacharunderscore}{\kern0pt}x{\isacharcolon}{\kern0pt}\ {\isachardoublequoteopen}Profile{\isachardot}{\kern0pt}lifted\ S\ p\ q\ x{\isachardoublequoteclose}\isanewline
\ \ \isacommand{hence}\isamarkupfalse%
\ f{\isacharunderscore}{\kern0pt}profs{\isacharcolon}{\kern0pt}\ {\isachardoublequoteopen}finite{\isacharunderscore}{\kern0pt}profile\ S\ p\ {\isasymand}\ finite{\isacharunderscore}{\kern0pt}profile\ S\ q{\isachardoublequoteclose}\isanewline
\ \ \ \ \isacommand{by}\isamarkupfalse%
\ {\isacharparenleft}{\kern0pt}simp\ add{\isacharcolon}{\kern0pt}\ lifted{\isacharunderscore}{\kern0pt}def{\isacharparenright}{\kern0pt}\isanewline
\ \ \isacommand{from}\isamarkupfalse%
\ compatible\ \isacommand{obtain}\isamarkupfalse%
\ A{\isacharcolon}{\kern0pt}{\isacharcolon}{\kern0pt}{\isachardoublequoteopen}{\isacharprime}{\kern0pt}a\ set{\isachardoublequoteclose}\ \isakeyword{where}\ A{\isacharcolon}{\kern0pt}\isanewline
\ \ \ \ {\isachardoublequoteopen}A\ {\isasymsubseteq}\ S\ {\isasymand}\ {\isacharparenleft}{\kern0pt}{\isasymforall}x\ {\isasymin}\ A{\isachardot}{\kern0pt}\ indep{\isacharunderscore}{\kern0pt}of{\isacharunderscore}{\kern0pt}alt\ m\ S\ x\ {\isasymand}\isanewline
\ \ \ \ \ \ {\isacharparenleft}{\kern0pt}{\isasymforall}p{\isachardot}{\kern0pt}\ finite{\isacharunderscore}{\kern0pt}profile\ S\ p\ {\isasymlongrightarrow}\ x\ {\isasymin}\ reject\ m\ S\ p{\isacharparenright}{\kern0pt}{\isacharparenright}{\kern0pt}\ {\isasymand}\isanewline
\ \ \ \ \ \ \ \ {\isacharparenleft}{\kern0pt}{\isasymforall}x\ {\isasymin}\ S{\isacharminus}{\kern0pt}A{\isachardot}{\kern0pt}\ indep{\isacharunderscore}{\kern0pt}of{\isacharunderscore}{\kern0pt}alt\ n\ S\ x\ {\isasymand}\isanewline
\ \ \ \ \ \ {\isacharparenleft}{\kern0pt}{\isasymforall}p{\isachardot}{\kern0pt}\ finite{\isacharunderscore}{\kern0pt}profile\ S\ p\ {\isasymlongrightarrow}\ x\ {\isasymin}\ reject\ n\ S\ p{\isacharparenright}{\kern0pt}{\isacharparenright}{\kern0pt}{\isachardoublequoteclose}\isanewline
\ \ \ \ \isacommand{using}\isamarkupfalse%
\ disjoint{\isacharunderscore}{\kern0pt}compatibility{\isacharunderscore}{\kern0pt}def\ f{\isacharunderscore}{\kern0pt}profs\isanewline
\ \ \ \ \isacommand{by}\isamarkupfalse%
\ {\isacharparenleft}{\kern0pt}metis\ {\isacharparenleft}{\kern0pt}no{\isacharunderscore}{\kern0pt}types{\isacharcomma}{\kern0pt}\ lifting{\isacharparenright}{\kern0pt}{\isacharparenright}{\kern0pt}\isanewline
\ \ \isacommand{have}\isamarkupfalse%
\isanewline
\ \ \ \ {\isachardoublequoteopen}{\isasymforall}x\ {\isasymin}\ S{\isachardot}{\kern0pt}\ prof{\isacharunderscore}{\kern0pt}contains{\isacharunderscore}{\kern0pt}result\ {\isacharparenleft}{\kern0pt}m\ {\isasymparallel}\isactrlsub {\isasymup}\ n{\isacharparenright}{\kern0pt}\ S\ p\ q\ x{\isachardoublequoteclose}\isanewline
\ \ \isacommand{proof}\isamarkupfalse%
\ cases\isanewline
\ \ \ \ \isacommand{assume}\isamarkupfalse%
\ a{\isadigit{0}}{\isacharcolon}{\kern0pt}\ {\isachardoublequoteopen}x\ {\isasymin}\ A{\isachardoublequoteclose}\isanewline
\ \ \ \ \isacommand{hence}\isamarkupfalse%
\ {\isachardoublequoteopen}x\ {\isasymin}\ reject\ m\ S\ p{\isachardoublequoteclose}\isanewline
\ \ \ \ \ \ \isacommand{using}\isamarkupfalse%
\ A\ f{\isacharunderscore}{\kern0pt}profs\isanewline
\ \ \ \ \ \ \isacommand{by}\isamarkupfalse%
\ blast\isanewline
\ \ \ \ \isacommand{with}\isamarkupfalse%
\ defer{\isacharunderscore}{\kern0pt}x\ \isacommand{have}\isamarkupfalse%
\ defer{\isacharunderscore}{\kern0pt}n{\isacharcolon}{\kern0pt}\ {\isachardoublequoteopen}x\ {\isasymin}\ defer\ n\ S\ p{\isachardoublequoteclose}\isanewline
\ \ \ \ \ \ \isacommand{using}\isamarkupfalse%
\ compatible\ disjoint{\isacharunderscore}{\kern0pt}compatibility{\isacharunderscore}{\kern0pt}def\isanewline
\ \ \ \ \ \ \ \ \ \ \ \ mod{\isacharunderscore}{\kern0pt}contains{\isacharunderscore}{\kern0pt}result{\isacharunderscore}{\kern0pt}def\ f{\isacharunderscore}{\kern0pt}profs\ max{\isacharunderscore}{\kern0pt}agg{\isacharunderscore}{\kern0pt}rej{\isadigit{4}}\isanewline
\ \ \ \ \ \ \isacommand{by}\isamarkupfalse%
\ metis\isanewline
\ \ \ \ \isacommand{have}\isamarkupfalse%
\isanewline
\ \ \ \ \ \ {\isachardoublequoteopen}{\isasymforall}x\ {\isasymin}\ A{\isachardot}{\kern0pt}\ mod{\isacharunderscore}{\kern0pt}contains{\isacharunderscore}{\kern0pt}result\ {\isacharparenleft}{\kern0pt}m\ {\isasymparallel}\isactrlsub {\isasymup}\ n{\isacharparenright}{\kern0pt}\ n\ S\ p\ x{\isachardoublequoteclose}\isanewline
\ \ \ \ \ \ \isacommand{using}\isamarkupfalse%
\ A\ compatible\ disjoint{\isacharunderscore}{\kern0pt}compatibility{\isacharunderscore}{\kern0pt}def\isanewline
\ \ \ \ \ \ \ \ \ \ \ \ max{\isacharunderscore}{\kern0pt}agg{\isacharunderscore}{\kern0pt}rej{\isadigit{4}}\ f{\isacharunderscore}{\kern0pt}profs\isanewline
\ \ \ \ \ \ \isacommand{by}\isamarkupfalse%
\ metis\isanewline
\ \ \ \ \isacommand{moreover}\isamarkupfalse%
\ \isacommand{have}\isamarkupfalse%
\ {\isachardoublequoteopen}{\isasymforall}x\ {\isasymin}\ S{\isachardot}{\kern0pt}\ prof{\isacharunderscore}{\kern0pt}contains{\isacharunderscore}{\kern0pt}result\ n\ S\ p\ q\ x{\isachardoublequoteclose}\isanewline
\ \ \ \ \ \ \isacommand{using}\isamarkupfalse%
\ defer{\isacharunderscore}{\kern0pt}n\ lifted{\isacharunderscore}{\kern0pt}x\ prof{\isacharunderscore}{\kern0pt}contains{\isacharunderscore}{\kern0pt}result{\isacharunderscore}{\kern0pt}def\ monotone{\isacharunderscore}{\kern0pt}n\ f{\isacharunderscore}{\kern0pt}profs\isanewline
\ \ \ \ \ \ \ \ \ \ \ \ defer{\isacharunderscore}{\kern0pt}lift{\isacharunderscore}{\kern0pt}invariance{\isacharunderscore}{\kern0pt}def\isanewline
\ \ \ \ \ \ \isacommand{by}\isamarkupfalse%
\ {\isacharparenleft}{\kern0pt}smt\ {\isacharparenleft}{\kern0pt}verit{\isacharcomma}{\kern0pt}\ del{\isacharunderscore}{\kern0pt}insts{\isacharparenright}{\kern0pt}{\isacharparenright}{\kern0pt}\isanewline
\ \ \ \ \isacommand{moreover}\isamarkupfalse%
\ \isacommand{have}\isamarkupfalse%
\isanewline
\ \ \ \ \ \ {\isachardoublequoteopen}{\isasymforall}x\ {\isasymin}\ A{\isachardot}{\kern0pt}\ mod{\isacharunderscore}{\kern0pt}contains{\isacharunderscore}{\kern0pt}result\ n\ {\isacharparenleft}{\kern0pt}m\ {\isasymparallel}\isactrlsub {\isasymup}\ n{\isacharparenright}{\kern0pt}\ S\ q\ x{\isachardoublequoteclose}\isanewline
\ \ \ \ \ \ \isacommand{using}\isamarkupfalse%
\ A\ compatible\ disjoint{\isacharunderscore}{\kern0pt}compatibility{\isacharunderscore}{\kern0pt}def\isanewline
\ \ \ \ \ \ \ \ \ \ \ \ max{\isacharunderscore}{\kern0pt}agg{\isacharunderscore}{\kern0pt}rej{\isadigit{3}}\ f{\isacharunderscore}{\kern0pt}profs\isanewline
\ \ \ \ \ \ \isacommand{by}\isamarkupfalse%
\ metis\isanewline
\ \ \ \ \isacommand{ultimately}\isamarkupfalse%
\ \isacommand{have}\isamarkupfalse%
\ {\isadigit{0}}{\isadigit{0}}{\isacharcolon}{\kern0pt}\isanewline
\ \ \ \ \ \ {\isachardoublequoteopen}{\isasymforall}x\ {\isasymin}\ A{\isachardot}{\kern0pt}\ prof{\isacharunderscore}{\kern0pt}contains{\isacharunderscore}{\kern0pt}result\ {\isacharparenleft}{\kern0pt}m\ {\isasymparallel}\isactrlsub {\isasymup}\ n{\isacharparenright}{\kern0pt}\ S\ p\ q\ x{\isachardoublequoteclose}\isanewline
\ \ \ \ \ \ \isacommand{by}\isamarkupfalse%
\ {\isacharparenleft}{\kern0pt}simp\ add{\isacharcolon}{\kern0pt}\ mod{\isacharunderscore}{\kern0pt}contains{\isacharunderscore}{\kern0pt}result{\isacharunderscore}{\kern0pt}def\ prof{\isacharunderscore}{\kern0pt}contains{\isacharunderscore}{\kern0pt}result{\isacharunderscore}{\kern0pt}def{\isacharparenright}{\kern0pt}\isanewline
\ \ \ \ \isacommand{have}\isamarkupfalse%
\isanewline
\ \ \ \ \ \ {\isachardoublequoteopen}{\isasymforall}x\ {\isasymin}\ S{\isacharminus}{\kern0pt}A{\isachardot}{\kern0pt}\ mod{\isacharunderscore}{\kern0pt}contains{\isacharunderscore}{\kern0pt}result\ {\isacharparenleft}{\kern0pt}m\ {\isasymparallel}\isactrlsub {\isasymup}\ n{\isacharparenright}{\kern0pt}\ m\ S\ p\ x{\isachardoublequoteclose}\isanewline
\ \ \ \ \ \ \isacommand{using}\isamarkupfalse%
\ A\ max{\isacharunderscore}{\kern0pt}agg{\isacharunderscore}{\kern0pt}rej{\isadigit{2}}\ monotone{\isacharunderscore}{\kern0pt}m\ monotone{\isacharunderscore}{\kern0pt}n\ f{\isacharunderscore}{\kern0pt}profs\isanewline
\ \ \ \ \ \ \ \ \ \ \ \ defer{\isacharunderscore}{\kern0pt}lift{\isacharunderscore}{\kern0pt}invariance{\isacharunderscore}{\kern0pt}def\isanewline
\ \ \ \ \ \ \isacommand{by}\isamarkupfalse%
\ metis\isanewline
\ \ \ \ \isacommand{moreover}\isamarkupfalse%
\ \isacommand{have}\isamarkupfalse%
\ {\isachardoublequoteopen}{\isasymforall}x\ {\isasymin}\ S{\isachardot}{\kern0pt}\ prof{\isacharunderscore}{\kern0pt}contains{\isacharunderscore}{\kern0pt}result\ m\ S\ p\ q\ x{\isachardoublequoteclose}\isanewline
\ \ \ \ \ \ \isacommand{using}\isamarkupfalse%
\ A\ lifted{\isacharunderscore}{\kern0pt}x\ a{\isadigit{0}}\ prof{\isacharunderscore}{\kern0pt}contains{\isacharunderscore}{\kern0pt}result{\isacharunderscore}{\kern0pt}def\ indep{\isacharunderscore}{\kern0pt}of{\isacharunderscore}{\kern0pt}alt{\isacharunderscore}{\kern0pt}def\isanewline
\ \ \ \ \ \ \ \ \ \ \ \ lifted{\isacharunderscore}{\kern0pt}imp{\isacharunderscore}{\kern0pt}equiv{\isacharunderscore}{\kern0pt}prof{\isacharunderscore}{\kern0pt}except{\isacharunderscore}{\kern0pt}a\ f{\isacharunderscore}{\kern0pt}profs\ IntI\isanewline
\ \ \ \ \ \ \ \ \ \ \ \ electoral{\isacharunderscore}{\kern0pt}mod{\isacharunderscore}{\kern0pt}defer{\isacharunderscore}{\kern0pt}elem\ empty{\isacharunderscore}{\kern0pt}iff\ result{\isacharunderscore}{\kern0pt}disj\isanewline
\ \ \ \ \ \ \isacommand{by}\isamarkupfalse%
\ {\isacharparenleft}{\kern0pt}smt\ {\isacharparenleft}{\kern0pt}verit{\isacharcomma}{\kern0pt}\ ccfv{\isacharunderscore}{\kern0pt}threshold{\isacharparenright}{\kern0pt}{\isacharparenright}{\kern0pt}\isanewline
\ \ \ \ \isacommand{moreover}\isamarkupfalse%
\ \isacommand{have}\isamarkupfalse%
\isanewline
\ \ \ \ \ \ {\isachardoublequoteopen}{\isasymforall}x\ {\isasymin}\ S{\isacharminus}{\kern0pt}A{\isachardot}{\kern0pt}\ mod{\isacharunderscore}{\kern0pt}contains{\isacharunderscore}{\kern0pt}result\ m\ {\isacharparenleft}{\kern0pt}m\ {\isasymparallel}\isactrlsub {\isasymup}\ n{\isacharparenright}{\kern0pt}\ S\ q\ x{\isachardoublequoteclose}\isanewline
\ \ \ \ \ \ \isacommand{using}\isamarkupfalse%
\ A\ max{\isacharunderscore}{\kern0pt}agg{\isacharunderscore}{\kern0pt}rej{\isadigit{1}}\ monotone{\isacharunderscore}{\kern0pt}m\ monotone{\isacharunderscore}{\kern0pt}n\ f{\isacharunderscore}{\kern0pt}profs\isanewline
\ \ \ \ \ \ \ \ \ \ \ \ defer{\isacharunderscore}{\kern0pt}lift{\isacharunderscore}{\kern0pt}invariance{\isacharunderscore}{\kern0pt}def\isanewline
\ \ \ \ \ \ \isacommand{by}\isamarkupfalse%
\ metis\isanewline
\ \ \ \ \isacommand{ultimately}\isamarkupfalse%
\ \isacommand{have}\isamarkupfalse%
\ {\isadigit{0}}{\isadigit{1}}{\isacharcolon}{\kern0pt}\isanewline
\ \ \ \ \ \ {\isachardoublequoteopen}{\isasymforall}x\ {\isasymin}\ S{\isacharminus}{\kern0pt}A{\isachardot}{\kern0pt}\ prof{\isacharunderscore}{\kern0pt}contains{\isacharunderscore}{\kern0pt}result\ {\isacharparenleft}{\kern0pt}m\ {\isasymparallel}\isactrlsub {\isasymup}\ n{\isacharparenright}{\kern0pt}\ S\ p\ q\ x{\isachardoublequoteclose}\isanewline
\ \ \ \ \ \ \isacommand{by}\isamarkupfalse%
\ {\isacharparenleft}{\kern0pt}simp\ add{\isacharcolon}{\kern0pt}\ mod{\isacharunderscore}{\kern0pt}contains{\isacharunderscore}{\kern0pt}result{\isacharunderscore}{\kern0pt}def\ prof{\isacharunderscore}{\kern0pt}contains{\isacharunderscore}{\kern0pt}result{\isacharunderscore}{\kern0pt}def{\isacharparenright}{\kern0pt}\isanewline
\ \ \ \ \isacommand{from}\isamarkupfalse%
\ {\isadigit{0}}{\isadigit{0}}\ {\isadigit{0}}{\isadigit{1}}\isanewline
\ \ \ \ \isacommand{show}\isamarkupfalse%
\ {\isacharquery}{\kern0pt}thesis\isanewline
\ \ \ \ \ \ \isacommand{by}\isamarkupfalse%
\ blast\isanewline
\ \ \isacommand{next}\isamarkupfalse%
\isanewline
\ \ \ \ \isacommand{assume}\isamarkupfalse%
\ {\isachardoublequoteopen}x\ {\isasymnotin}\ A{\isachardoublequoteclose}\isanewline
\ \ \ \ \isacommand{hence}\isamarkupfalse%
\ a{\isadigit{1}}{\isacharcolon}{\kern0pt}\ {\isachardoublequoteopen}x\ {\isasymin}\ S{\isacharminus}{\kern0pt}A{\isachardoublequoteclose}\isanewline
\ \ \ \ \ \ \isacommand{using}\isamarkupfalse%
\ DiffI\ lifted{\isacharunderscore}{\kern0pt}x\ compatible\ f{\isacharunderscore}{\kern0pt}profs\isanewline
\ \ \ \ \ \ \ \ \ \ \ \ Profile{\isachardot}{\kern0pt}lifted{\isacharunderscore}{\kern0pt}def\isanewline
\ \ \ \ \ \ \isacommand{by}\isamarkupfalse%
\ {\isacharparenleft}{\kern0pt}metis\ {\isacharparenleft}{\kern0pt}no{\isacharunderscore}{\kern0pt}types{\isacharcomma}{\kern0pt}\ lifting{\isacharparenright}{\kern0pt}{\isacharparenright}{\kern0pt}\isanewline
\ \ \ \ \isacommand{hence}\isamarkupfalse%
\ {\isachardoublequoteopen}x\ {\isasymin}\ reject\ n\ S\ p{\isachardoublequoteclose}\isanewline
\ \ \ \ \ \ \isacommand{using}\isamarkupfalse%
\ A\ f{\isacharunderscore}{\kern0pt}profs\isanewline
\ \ \ \ \ \ \isacommand{by}\isamarkupfalse%
\ blast\isanewline
\ \ \ \ \isacommand{with}\isamarkupfalse%
\ defer{\isacharunderscore}{\kern0pt}x\ \isacommand{have}\isamarkupfalse%
\ defer{\isacharunderscore}{\kern0pt}n{\isacharcolon}{\kern0pt}\ {\isachardoublequoteopen}x\ {\isasymin}\ defer\ m\ S\ p{\isachardoublequoteclose}\isanewline
\ \ \ \ \ \ \isacommand{using}\isamarkupfalse%
\ DiffD{\isadigit{1}}\ DiffD{\isadigit{2}}\ compatible\ dcompat{\isacharunderscore}{\kern0pt}dec{\isacharunderscore}{\kern0pt}by{\isacharunderscore}{\kern0pt}one{\isacharunderscore}{\kern0pt}mod\isanewline
\ \ \ \ \ \ \ \ \ \ \ \ defer{\isacharunderscore}{\kern0pt}not{\isacharunderscore}{\kern0pt}elec{\isacharunderscore}{\kern0pt}or{\isacharunderscore}{\kern0pt}rej\ disjoint{\isacharunderscore}{\kern0pt}compatibility{\isacharunderscore}{\kern0pt}def\isanewline
\ \ \ \ \ \ \ \ \ \ \ \ not{\isacharunderscore}{\kern0pt}rej{\isacharunderscore}{\kern0pt}imp{\isacharunderscore}{\kern0pt}elec{\isacharunderscore}{\kern0pt}or{\isacharunderscore}{\kern0pt}def\ mod{\isacharunderscore}{\kern0pt}contains{\isacharunderscore}{\kern0pt}result{\isacharunderscore}{\kern0pt}def\isanewline
\ \ \ \ \ \ \ \ \ \ \ \ max{\isacharunderscore}{\kern0pt}agg{\isacharunderscore}{\kern0pt}sound\ par{\isacharunderscore}{\kern0pt}comp{\isacharunderscore}{\kern0pt}sound\ f{\isacharunderscore}{\kern0pt}profs\isanewline
\ \ \ \ \ \ \ \ \ \ \ \ maximum{\isacharunderscore}{\kern0pt}parallel{\isacharunderscore}{\kern0pt}composition{\isachardot}{\kern0pt}simps\isanewline
\ \ \ \ \ \ \isacommand{by}\isamarkupfalse%
\ metis\isanewline
\ \ \ \ \isacommand{have}\isamarkupfalse%
\isanewline
\ \ \ \ \ \ {\isachardoublequoteopen}{\isasymforall}x\ {\isasymin}\ A{\isachardot}{\kern0pt}\ mod{\isacharunderscore}{\kern0pt}contains{\isacharunderscore}{\kern0pt}result\ {\isacharparenleft}{\kern0pt}m\ {\isasymparallel}\isactrlsub {\isasymup}\ n{\isacharparenright}{\kern0pt}\ n\ S\ p\ x{\isachardoublequoteclose}\isanewline
\ \ \ \ \ \ \isacommand{using}\isamarkupfalse%
\ A\ compatible\ disjoint{\isacharunderscore}{\kern0pt}compatibility{\isacharunderscore}{\kern0pt}def\isanewline
\ \ \ \ \ \ \ \ \ \ \ \ max{\isacharunderscore}{\kern0pt}agg{\isacharunderscore}{\kern0pt}rej{\isadigit{4}}\ f{\isacharunderscore}{\kern0pt}profs\isanewline
\ \ \ \ \ \ \isacommand{by}\isamarkupfalse%
\ metis\isanewline
\ \ \ \ \isacommand{moreover}\isamarkupfalse%
\ \isacommand{have}\isamarkupfalse%
\ {\isachardoublequoteopen}{\isasymforall}x\ {\isasymin}\ S{\isachardot}{\kern0pt}\ prof{\isacharunderscore}{\kern0pt}contains{\isacharunderscore}{\kern0pt}result\ n\ S\ p\ q\ x{\isachardoublequoteclose}\isanewline
\ \ \ \ \ \ \isacommand{using}\isamarkupfalse%
\ A\ lifted{\isacharunderscore}{\kern0pt}x\ a{\isadigit{1}}\ prof{\isacharunderscore}{\kern0pt}contains{\isacharunderscore}{\kern0pt}result{\isacharunderscore}{\kern0pt}def\ indep{\isacharunderscore}{\kern0pt}of{\isacharunderscore}{\kern0pt}alt{\isacharunderscore}{\kern0pt}def\isanewline
\ \ \ \ \ \ \ \ \ \ \ \ lifted{\isacharunderscore}{\kern0pt}imp{\isacharunderscore}{\kern0pt}equiv{\isacharunderscore}{\kern0pt}prof{\isacharunderscore}{\kern0pt}except{\isacharunderscore}{\kern0pt}a\ f{\isacharunderscore}{\kern0pt}profs\isanewline
\ \ \ \ \ \ \ \ \ \ \ \ electoral{\isacharunderscore}{\kern0pt}mod{\isacharunderscore}{\kern0pt}defer{\isacharunderscore}{\kern0pt}elem\isanewline
\ \ \ \ \ \ \isacommand{by}\isamarkupfalse%
\ {\isacharparenleft}{\kern0pt}smt\ {\isacharparenleft}{\kern0pt}verit{\isacharcomma}{\kern0pt}\ ccfv{\isacharunderscore}{\kern0pt}threshold{\isacharparenright}{\kern0pt}{\isacharparenright}{\kern0pt}\isanewline
\ \ \ \ \isacommand{moreover}\isamarkupfalse%
\ \isacommand{have}\isamarkupfalse%
\isanewline
\ \ \ \ \ \ {\isachardoublequoteopen}{\isasymforall}x\ {\isasymin}\ A{\isachardot}{\kern0pt}\ mod{\isacharunderscore}{\kern0pt}contains{\isacharunderscore}{\kern0pt}result\ n\ {\isacharparenleft}{\kern0pt}m\ {\isasymparallel}\isactrlsub {\isasymup}\ n{\isacharparenright}{\kern0pt}\ S\ q\ x{\isachardoublequoteclose}\isanewline
\ \ \ \ \ \ \isacommand{using}\isamarkupfalse%
\ A\ compatible\ disjoint{\isacharunderscore}{\kern0pt}compatibility{\isacharunderscore}{\kern0pt}def\isanewline
\ \ \ \ \ \ \ \ \ \ \ \ max{\isacharunderscore}{\kern0pt}agg{\isacharunderscore}{\kern0pt}rej{\isadigit{3}}\ f{\isacharunderscore}{\kern0pt}profs\isanewline
\ \ \ \ \ \ \isacommand{by}\isamarkupfalse%
\ metis\isanewline
\ \ \ \ \isacommand{ultimately}\isamarkupfalse%
\ \isacommand{have}\isamarkupfalse%
\ {\isadigit{1}}{\isadigit{0}}{\isacharcolon}{\kern0pt}\isanewline
\ \ \ \ \ \ {\isachardoublequoteopen}{\isasymforall}x\ {\isasymin}\ A{\isachardot}{\kern0pt}\ prof{\isacharunderscore}{\kern0pt}contains{\isacharunderscore}{\kern0pt}result\ {\isacharparenleft}{\kern0pt}m\ {\isasymparallel}\isactrlsub {\isasymup}\ n{\isacharparenright}{\kern0pt}\ S\ p\ q\ x{\isachardoublequoteclose}\isanewline
\ \ \ \ \ \ \isacommand{by}\isamarkupfalse%
\ {\isacharparenleft}{\kern0pt}simp\ add{\isacharcolon}{\kern0pt}\ mod{\isacharunderscore}{\kern0pt}contains{\isacharunderscore}{\kern0pt}result{\isacharunderscore}{\kern0pt}def\ prof{\isacharunderscore}{\kern0pt}contains{\isacharunderscore}{\kern0pt}result{\isacharunderscore}{\kern0pt}def{\isacharparenright}{\kern0pt}\isanewline
\ \ \ \ \isacommand{have}\isamarkupfalse%
\isanewline
\ \ \ \ \ \ {\isachardoublequoteopen}{\isasymforall}x\ {\isasymin}\ S{\isacharminus}{\kern0pt}A{\isachardot}{\kern0pt}\ mod{\isacharunderscore}{\kern0pt}contains{\isacharunderscore}{\kern0pt}result\ {\isacharparenleft}{\kern0pt}m\ {\isasymparallel}\isactrlsub {\isasymup}\ n{\isacharparenright}{\kern0pt}\ m\ S\ p\ x{\isachardoublequoteclose}\isanewline
\ \ \ \ \ \ \isacommand{using}\isamarkupfalse%
\ A\ max{\isacharunderscore}{\kern0pt}agg{\isacharunderscore}{\kern0pt}rej{\isadigit{2}}\ monotone{\isacharunderscore}{\kern0pt}m\ monotone{\isacharunderscore}{\kern0pt}n\isanewline
\ \ \ \ \ \ \ \ \ \ \ \ f{\isacharunderscore}{\kern0pt}profs\ defer{\isacharunderscore}{\kern0pt}lift{\isacharunderscore}{\kern0pt}invariance{\isacharunderscore}{\kern0pt}def\isanewline
\ \ \ \ \ \ \isacommand{by}\isamarkupfalse%
\ metis\isanewline
\ \ \ \ \isacommand{moreover}\isamarkupfalse%
\ \isacommand{have}\isamarkupfalse%
\ {\isachardoublequoteopen}{\isasymforall}x\ {\isasymin}\ S{\isachardot}{\kern0pt}\ prof{\isacharunderscore}{\kern0pt}contains{\isacharunderscore}{\kern0pt}result\ m\ S\ p\ q\ x{\isachardoublequoteclose}\isanewline
\ \ \ \ \ \ \isacommand{using}\isamarkupfalse%
\ lifted{\isacharunderscore}{\kern0pt}x\ defer{\isacharunderscore}{\kern0pt}n\ prof{\isacharunderscore}{\kern0pt}contains{\isacharunderscore}{\kern0pt}result{\isacharunderscore}{\kern0pt}def\ monotone{\isacharunderscore}{\kern0pt}m\isanewline
\ \ \ \ \ \ \ \ \ \ \ \ f{\isacharunderscore}{\kern0pt}profs\ defer{\isacharunderscore}{\kern0pt}lift{\isacharunderscore}{\kern0pt}invariance{\isacharunderscore}{\kern0pt}def\isanewline
\ \ \ \ \ \ \isacommand{by}\isamarkupfalse%
\ {\isacharparenleft}{\kern0pt}smt\ {\isacharparenleft}{\kern0pt}verit{\isacharcomma}{\kern0pt}\ ccfv{\isacharunderscore}{\kern0pt}threshold{\isacharparenright}{\kern0pt}{\isacharparenright}{\kern0pt}\isanewline
\ \ \ \ \isacommand{moreover}\isamarkupfalse%
\ \isacommand{have}\isamarkupfalse%
\isanewline
\ \ \ \ \ \ {\isachardoublequoteopen}{\isasymforall}x\ {\isasymin}\ S{\isacharminus}{\kern0pt}A{\isachardot}{\kern0pt}\ mod{\isacharunderscore}{\kern0pt}contains{\isacharunderscore}{\kern0pt}result\ m\ {\isacharparenleft}{\kern0pt}m\ {\isasymparallel}\isactrlsub {\isasymup}\ n{\isacharparenright}{\kern0pt}\ S\ q\ x{\isachardoublequoteclose}\isanewline
\ \ \ \ \ \ \isacommand{using}\isamarkupfalse%
\ A\ max{\isacharunderscore}{\kern0pt}agg{\isacharunderscore}{\kern0pt}rej{\isadigit{1}}\ monotone{\isacharunderscore}{\kern0pt}m\ monotone{\isacharunderscore}{\kern0pt}n\isanewline
\ \ \ \ \ \ \ \ \ \ \ \ f{\isacharunderscore}{\kern0pt}profs\ defer{\isacharunderscore}{\kern0pt}lift{\isacharunderscore}{\kern0pt}invariance{\isacharunderscore}{\kern0pt}def\isanewline
\ \ \ \ \ \ \isacommand{by}\isamarkupfalse%
\ metis\isanewline
\ \ \ \ \isacommand{ultimately}\isamarkupfalse%
\ \isacommand{have}\isamarkupfalse%
\ {\isadigit{1}}{\isadigit{1}}{\isacharcolon}{\kern0pt}\isanewline
\ \ \ \ \ \ {\isachardoublequoteopen}{\isasymforall}x\ {\isasymin}\ S{\isacharminus}{\kern0pt}A{\isachardot}{\kern0pt}\ prof{\isacharunderscore}{\kern0pt}contains{\isacharunderscore}{\kern0pt}result\ {\isacharparenleft}{\kern0pt}m\ {\isasymparallel}\isactrlsub {\isasymup}\ n{\isacharparenright}{\kern0pt}\ S\ p\ q\ x{\isachardoublequoteclose}\isanewline
\ \ \ \ \ \ \isacommand{using}\isamarkupfalse%
\ electoral{\isacharunderscore}{\kern0pt}mod{\isacharunderscore}{\kern0pt}defer{\isacharunderscore}{\kern0pt}elem\isanewline
\ \ \ \ \ \ \isacommand{by}\isamarkupfalse%
\ {\isacharparenleft}{\kern0pt}simp\ add{\isacharcolon}{\kern0pt}\ mod{\isacharunderscore}{\kern0pt}contains{\isacharunderscore}{\kern0pt}result{\isacharunderscore}{\kern0pt}def\ prof{\isacharunderscore}{\kern0pt}contains{\isacharunderscore}{\kern0pt}result{\isacharunderscore}{\kern0pt}def{\isacharparenright}{\kern0pt}\isanewline
\ \ \ \ \isacommand{from}\isamarkupfalse%
\ {\isadigit{1}}{\isadigit{0}}\ {\isadigit{1}}{\isadigit{1}}\isanewline
\ \ \ \ \isacommand{show}\isamarkupfalse%
\ {\isacharquery}{\kern0pt}thesis\isanewline
\ \ \ \ \ \ \isacommand{by}\isamarkupfalse%
\ blast\isanewline
\ \ \isacommand{qed}\isamarkupfalse%
\isanewline
\ \ \isacommand{thus}\isamarkupfalse%
\ {\isachardoublequoteopen}{\isacharparenleft}{\kern0pt}m\ {\isasymparallel}\isactrlsub {\isasymup}\ n{\isacharparenright}{\kern0pt}\ S\ p\ {\isacharequal}{\kern0pt}\ {\isacharparenleft}{\kern0pt}m\ {\isasymparallel}\isactrlsub {\isasymup}\ n{\isacharparenright}{\kern0pt}\ S\ q{\isachardoublequoteclose}\isanewline
\ \ \ \ \isacommand{using}\isamarkupfalse%
\ compatible\ disjoint{\isacharunderscore}{\kern0pt}compatibility{\isacharunderscore}{\kern0pt}def\ f{\isacharunderscore}{\kern0pt}profs\isanewline
\ \ \ \ \ \ \ \ \ \ eq{\isacharunderscore}{\kern0pt}alts{\isacharunderscore}{\kern0pt}in{\isacharunderscore}{\kern0pt}profs{\isacharunderscore}{\kern0pt}imp{\isacharunderscore}{\kern0pt}eq{\isacharunderscore}{\kern0pt}results\ max{\isacharunderscore}{\kern0pt}par{\isacharunderscore}{\kern0pt}comp{\isacharunderscore}{\kern0pt}sound\isanewline
\ \ \ \ \isacommand{by}\isamarkupfalse%
\ metis\isanewline
\isacommand{qed}\isamarkupfalse%
%
\endisatagproof
{\isafoldproof}%
%
\isadelimproof
\isanewline
%
\endisadelimproof
\isanewline
\isacommand{lemma}\isamarkupfalse%
\ def{\isacharunderscore}{\kern0pt}lift{\isacharunderscore}{\kern0pt}inv{\isacharunderscore}{\kern0pt}seq{\isacharunderscore}{\kern0pt}comp{\isacharunderscore}{\kern0pt}help{\isacharcolon}{\kern0pt}\isanewline
\ \ \isakeyword{assumes}\isanewline
\ \ \ \ monotone{\isacharunderscore}{\kern0pt}m{\isacharcolon}{\kern0pt}\ {\isachardoublequoteopen}defer{\isacharunderscore}{\kern0pt}lift{\isacharunderscore}{\kern0pt}invariance\ m{\isachardoublequoteclose}\ \isakeyword{and}\isanewline
\ \ \ \ monotone{\isacharunderscore}{\kern0pt}n{\isacharcolon}{\kern0pt}\ {\isachardoublequoteopen}defer{\isacharunderscore}{\kern0pt}lift{\isacharunderscore}{\kern0pt}invariance\ n{\isachardoublequoteclose}\ \isakeyword{and}\isanewline
\ \ \ \ def{\isacharunderscore}{\kern0pt}and{\isacharunderscore}{\kern0pt}lifted{\isacharcolon}{\kern0pt}\ {\isachardoublequoteopen}a\ {\isasymin}\ {\isacharparenleft}{\kern0pt}defer\ {\isacharparenleft}{\kern0pt}m\ {\isasymtriangleright}\ n{\isacharparenright}{\kern0pt}\ A\ p{\isacharparenright}{\kern0pt}\ {\isasymand}\ lifted\ A\ p\ q\ a{\isachardoublequoteclose}\isanewline
\ \ \isakeyword{shows}\ {\isachardoublequoteopen}{\isacharparenleft}{\kern0pt}m\ {\isasymtriangleright}\ n{\isacharparenright}{\kern0pt}\ A\ p\ {\isacharequal}{\kern0pt}\ {\isacharparenleft}{\kern0pt}m\ {\isasymtriangleright}\ n{\isacharparenright}{\kern0pt}\ A\ q{\isachardoublequoteclose}\isanewline
%
\isadelimproof
%
\endisadelimproof
%
\isatagproof
\isacommand{proof}\isamarkupfalse%
\ {\isacharminus}{\kern0pt}\isanewline
\ \ \isacommand{let}\isamarkupfalse%
\ {\isacharquery}{\kern0pt}new{\isacharunderscore}{\kern0pt}Ap\ {\isacharequal}{\kern0pt}\ {\isachardoublequoteopen}defer\ m\ A\ p{\isachardoublequoteclose}\isanewline
\ \ \isacommand{let}\isamarkupfalse%
\ {\isacharquery}{\kern0pt}new{\isacharunderscore}{\kern0pt}Aq\ {\isacharequal}{\kern0pt}\ {\isachardoublequoteopen}defer\ m\ A\ q{\isachardoublequoteclose}\isanewline
\ \ \isacommand{let}\isamarkupfalse%
\ {\isacharquery}{\kern0pt}new{\isacharunderscore}{\kern0pt}p\ {\isacharequal}{\kern0pt}\ {\isachardoublequoteopen}limit{\isacharunderscore}{\kern0pt}profile\ {\isacharquery}{\kern0pt}new{\isacharunderscore}{\kern0pt}Ap\ p{\isachardoublequoteclose}\isanewline
\ \ \isacommand{let}\isamarkupfalse%
\ {\isacharquery}{\kern0pt}new{\isacharunderscore}{\kern0pt}q\ {\isacharequal}{\kern0pt}\ {\isachardoublequoteopen}limit{\isacharunderscore}{\kern0pt}profile\ {\isacharquery}{\kern0pt}new{\isacharunderscore}{\kern0pt}Aq\ q{\isachardoublequoteclose}\isanewline
\ \ \isacommand{from}\isamarkupfalse%
\ monotone{\isacharunderscore}{\kern0pt}m\ monotone{\isacharunderscore}{\kern0pt}n\ \isacommand{have}\isamarkupfalse%
\ modules{\isacharcolon}{\kern0pt}\isanewline
\ \ \ \ {\isachardoublequoteopen}electoral{\isacharunderscore}{\kern0pt}module\ m\ {\isasymand}\ electoral{\isacharunderscore}{\kern0pt}module\ n{\isachardoublequoteclose}\isanewline
\ \ \ \ \isacommand{by}\isamarkupfalse%
\ {\isacharparenleft}{\kern0pt}simp\ add{\isacharcolon}{\kern0pt}\ defer{\isacharunderscore}{\kern0pt}lift{\isacharunderscore}{\kern0pt}invariance{\isacharunderscore}{\kern0pt}def{\isacharparenright}{\kern0pt}\isanewline
\ \ \isacommand{hence}\isamarkupfalse%
\ {\isachardoublequoteopen}finite{\isacharunderscore}{\kern0pt}profile\ A\ p\ {\isasymlongrightarrow}\ defer\ {\isacharparenleft}{\kern0pt}m\ {\isasymtriangleright}\ n{\isacharparenright}{\kern0pt}\ A\ p\ {\isasymsubseteq}\ defer\ m\ A\ p{\isachardoublequoteclose}\isanewline
\ \ \ \ \isacommand{using}\isamarkupfalse%
\ seq{\isacharunderscore}{\kern0pt}comp{\isacharunderscore}{\kern0pt}def{\isacharunderscore}{\kern0pt}set{\isacharunderscore}{\kern0pt}bounded\isanewline
\ \ \ \ \isacommand{by}\isamarkupfalse%
\ metis\isanewline
\ \ \isacommand{moreover}\isamarkupfalse%
\ \isacommand{have}\isamarkupfalse%
\ profile{\isacharunderscore}{\kern0pt}p{\isacharcolon}{\kern0pt}\ {\isachardoublequoteopen}lifted\ A\ p\ q\ a\ {\isasymlongrightarrow}\ finite{\isacharunderscore}{\kern0pt}profile\ A\ p{\isachardoublequoteclose}\isanewline
\ \ \ \ \isacommand{by}\isamarkupfalse%
\ {\isacharparenleft}{\kern0pt}simp\ add{\isacharcolon}{\kern0pt}\ lifted{\isacharunderscore}{\kern0pt}def{\isacharparenright}{\kern0pt}\isanewline
\ \ \isacommand{ultimately}\isamarkupfalse%
\ \isacommand{have}\isamarkupfalse%
\ defer{\isacharunderscore}{\kern0pt}subset{\isacharcolon}{\kern0pt}\ {\isachardoublequoteopen}defer\ {\isacharparenleft}{\kern0pt}m\ {\isasymtriangleright}\ n{\isacharparenright}{\kern0pt}\ A\ p\ {\isasymsubseteq}\ defer\ m\ A\ p{\isachardoublequoteclose}\isanewline
\ \ \ \ \isacommand{using}\isamarkupfalse%
\ def{\isacharunderscore}{\kern0pt}and{\isacharunderscore}{\kern0pt}lifted\isanewline
\ \ \ \ \isacommand{by}\isamarkupfalse%
\ blast\isanewline
\ \ \isacommand{hence}\isamarkupfalse%
\ mono{\isacharunderscore}{\kern0pt}m{\isacharcolon}{\kern0pt}\ {\isachardoublequoteopen}m\ A\ p\ {\isacharequal}{\kern0pt}\ m\ A\ q{\isachardoublequoteclose}\isanewline
\ \ \ \ \isacommand{using}\isamarkupfalse%
\ monotone{\isacharunderscore}{\kern0pt}m\ defer{\isacharunderscore}{\kern0pt}lift{\isacharunderscore}{\kern0pt}invariance{\isacharunderscore}{\kern0pt}def\ def{\isacharunderscore}{\kern0pt}and{\isacharunderscore}{\kern0pt}lifted\isanewline
\ \ \ \ \ \ \ \ \ \ modules\ profile{\isacharunderscore}{\kern0pt}p\ seq{\isacharunderscore}{\kern0pt}comp{\isacharunderscore}{\kern0pt}def{\isacharunderscore}{\kern0pt}set{\isacharunderscore}{\kern0pt}trans\isanewline
\ \ \ \ \isacommand{by}\isamarkupfalse%
\ metis\isanewline
\ \ \isacommand{hence}\isamarkupfalse%
\ new{\isacharunderscore}{\kern0pt}A{\isacharunderscore}{\kern0pt}eq{\isacharcolon}{\kern0pt}\ {\isachardoublequoteopen}{\isacharquery}{\kern0pt}new{\isacharunderscore}{\kern0pt}Ap\ {\isacharequal}{\kern0pt}\ {\isacharquery}{\kern0pt}new{\isacharunderscore}{\kern0pt}Aq{\isachardoublequoteclose}\isanewline
\ \ \ \ \isacommand{by}\isamarkupfalse%
\ presburger\isanewline
\ \ \isacommand{have}\isamarkupfalse%
\ defer{\isacharunderscore}{\kern0pt}eq{\isacharcolon}{\kern0pt}\isanewline
\ \ \ \ {\isachardoublequoteopen}defer\ {\isacharparenleft}{\kern0pt}m\ {\isasymtriangleright}\ n{\isacharparenright}{\kern0pt}\ A\ p\ {\isacharequal}{\kern0pt}\ defer\ n\ {\isacharquery}{\kern0pt}new{\isacharunderscore}{\kern0pt}Ap\ {\isacharquery}{\kern0pt}new{\isacharunderscore}{\kern0pt}p{\isachardoublequoteclose}\isanewline
\ \ \ \ \isacommand{using}\isamarkupfalse%
\ sequential{\isacharunderscore}{\kern0pt}composition{\isachardot}{\kern0pt}simps\ snd{\isacharunderscore}{\kern0pt}conv\isanewline
\ \ \ \ \isacommand{by}\isamarkupfalse%
\ metis\isanewline
\ \ \isacommand{hence}\isamarkupfalse%
\ mono{\isacharunderscore}{\kern0pt}n{\isacharcolon}{\kern0pt}\isanewline
\ \ \ \ {\isachardoublequoteopen}n\ {\isacharquery}{\kern0pt}new{\isacharunderscore}{\kern0pt}Ap\ {\isacharquery}{\kern0pt}new{\isacharunderscore}{\kern0pt}p\ {\isacharequal}{\kern0pt}\ n\ {\isacharquery}{\kern0pt}new{\isacharunderscore}{\kern0pt}Aq\ {\isacharquery}{\kern0pt}new{\isacharunderscore}{\kern0pt}q{\isachardoublequoteclose}\isanewline
\ \ \isacommand{proof}\isamarkupfalse%
\ cases\isanewline
\ \ \ \ \isacommand{assume}\isamarkupfalse%
\ {\isachardoublequoteopen}lifted\ {\isacharquery}{\kern0pt}new{\isacharunderscore}{\kern0pt}Ap\ {\isacharquery}{\kern0pt}new{\isacharunderscore}{\kern0pt}p\ {\isacharquery}{\kern0pt}new{\isacharunderscore}{\kern0pt}q\ a{\isachardoublequoteclose}\isanewline
\ \ \ \ \isacommand{thus}\isamarkupfalse%
\ {\isacharquery}{\kern0pt}thesis\isanewline
\ \ \ \ \ \ \isacommand{using}\isamarkupfalse%
\ defer{\isacharunderscore}{\kern0pt}eq\ mono{\isacharunderscore}{\kern0pt}m\ monotone{\isacharunderscore}{\kern0pt}n\isanewline
\ \ \ \ \ \ \ \ \ \ \ \ defer{\isacharunderscore}{\kern0pt}lift{\isacharunderscore}{\kern0pt}invariance{\isacharunderscore}{\kern0pt}def\ def{\isacharunderscore}{\kern0pt}and{\isacharunderscore}{\kern0pt}lifted\isanewline
\ \ \ \ \ \ \isacommand{by}\isamarkupfalse%
\ {\isacharparenleft}{\kern0pt}metis\ {\isacharparenleft}{\kern0pt}no{\isacharunderscore}{\kern0pt}types{\isacharcomma}{\kern0pt}\ lifting{\isacharparenright}{\kern0pt}{\isacharparenright}{\kern0pt}\isanewline
\ \ \isacommand{next}\isamarkupfalse%
\isanewline
\ \ \ \ \isacommand{assume}\isamarkupfalse%
\ a{\isadigit{2}}{\isacharcolon}{\kern0pt}\ {\isachardoublequoteopen}{\isasymnot}lifted\ {\isacharquery}{\kern0pt}new{\isacharunderscore}{\kern0pt}Ap\ {\isacharquery}{\kern0pt}new{\isacharunderscore}{\kern0pt}p\ {\isacharquery}{\kern0pt}new{\isacharunderscore}{\kern0pt}q\ a{\isachardoublequoteclose}\isanewline
\ \ \ \ \isacommand{from}\isamarkupfalse%
\ def{\isacharunderscore}{\kern0pt}and{\isacharunderscore}{\kern0pt}lifted\ \isacommand{have}\isamarkupfalse%
\ {\isachardoublequoteopen}finite{\isacharunderscore}{\kern0pt}profile\ A\ q{\isachardoublequoteclose}\isanewline
\ \ \ \ \ \ \isacommand{by}\isamarkupfalse%
\ {\isacharparenleft}{\kern0pt}simp\ add{\isacharcolon}{\kern0pt}\ lifted{\isacharunderscore}{\kern0pt}def{\isacharparenright}{\kern0pt}\isanewline
\ \ \ \ \isacommand{with}\isamarkupfalse%
\ modules\ new{\isacharunderscore}{\kern0pt}A{\isacharunderscore}{\kern0pt}eq\ \isacommand{have}\isamarkupfalse%
\ {\isadigit{1}}{\isacharcolon}{\kern0pt}\isanewline
\ \ \ \ \ \ {\isachardoublequoteopen}finite{\isacharunderscore}{\kern0pt}profile\ {\isacharquery}{\kern0pt}new{\isacharunderscore}{\kern0pt}Ap\ {\isacharquery}{\kern0pt}new{\isacharunderscore}{\kern0pt}q{\isachardoublequoteclose}\isanewline
\ \ \ \ \ \ \isacommand{using}\isamarkupfalse%
\ def{\isacharunderscore}{\kern0pt}presv{\isacharunderscore}{\kern0pt}fin{\isacharunderscore}{\kern0pt}prof\isanewline
\ \ \ \ \ \ \isacommand{by}\isamarkupfalse%
\ {\isacharparenleft}{\kern0pt}metis\ {\isacharparenleft}{\kern0pt}no{\isacharunderscore}{\kern0pt}types{\isacharparenright}{\kern0pt}{\isacharparenright}{\kern0pt}\isanewline
\ \ \ \ \isacommand{moreover}\isamarkupfalse%
\ \isacommand{from}\isamarkupfalse%
\ modules\ profile{\isacharunderscore}{\kern0pt}p\ def{\isacharunderscore}{\kern0pt}and{\isacharunderscore}{\kern0pt}lifted\isanewline
\ \ \ \ \isacommand{have}\isamarkupfalse%
\ {\isadigit{0}}{\isacharcolon}{\kern0pt}\isanewline
\ \ \ \ \ \ {\isachardoublequoteopen}finite{\isacharunderscore}{\kern0pt}profile\ {\isacharquery}{\kern0pt}new{\isacharunderscore}{\kern0pt}Ap\ {\isacharquery}{\kern0pt}new{\isacharunderscore}{\kern0pt}p{\isachardoublequoteclose}\isanewline
\ \ \ \ \ \ \isacommand{using}\isamarkupfalse%
\ def{\isacharunderscore}{\kern0pt}presv{\isacharunderscore}{\kern0pt}fin{\isacharunderscore}{\kern0pt}prof\isanewline
\ \ \ \ \ \ \isacommand{by}\isamarkupfalse%
\ {\isacharparenleft}{\kern0pt}metis\ {\isacharparenleft}{\kern0pt}no{\isacharunderscore}{\kern0pt}types{\isacharparenright}{\kern0pt}{\isacharparenright}{\kern0pt}\isanewline
\ \ \ \ \isacommand{moreover}\isamarkupfalse%
\ \isacommand{from}\isamarkupfalse%
\ defer{\isacharunderscore}{\kern0pt}subset\ def{\isacharunderscore}{\kern0pt}and{\isacharunderscore}{\kern0pt}lifted\isanewline
\ \ \ \ \isacommand{have}\isamarkupfalse%
\ {\isadigit{2}}{\isacharcolon}{\kern0pt}\ {\isachardoublequoteopen}a\ {\isasymin}\ {\isacharquery}{\kern0pt}new{\isacharunderscore}{\kern0pt}Ap{\isachardoublequoteclose}\isanewline
\ \ \ \ \ \ \isacommand{by}\isamarkupfalse%
\ blast\isanewline
\ \ \ \ \isacommand{moreover}\isamarkupfalse%
\ \isacommand{from}\isamarkupfalse%
\ def{\isacharunderscore}{\kern0pt}and{\isacharunderscore}{\kern0pt}lifted\ \isacommand{have}\isamarkupfalse%
\ eql{\isacharunderscore}{\kern0pt}lengths{\isacharcolon}{\kern0pt}\isanewline
\ \ \ \ \ \ {\isachardoublequoteopen}length\ {\isacharquery}{\kern0pt}new{\isacharunderscore}{\kern0pt}p\ {\isacharequal}{\kern0pt}\ length\ {\isacharquery}{\kern0pt}new{\isacharunderscore}{\kern0pt}q{\isachardoublequoteclose}\isanewline
\ \ \ \ \ \ \isacommand{by}\isamarkupfalse%
\ {\isacharparenleft}{\kern0pt}simp\ add{\isacharcolon}{\kern0pt}\ lifted{\isacharunderscore}{\kern0pt}def{\isacharparenright}{\kern0pt}\isanewline
\ \ \ \ \isacommand{ultimately}\isamarkupfalse%
\ \isacommand{have}\isamarkupfalse%
\ {\isadigit{0}}{\isacharcolon}{\kern0pt}\isanewline
\ \ \ \ \ \ {\isachardoublequoteopen}{\isacharparenleft}{\kern0pt}{\isasymforall}i{\isacharcolon}{\kern0pt}{\isacharcolon}{\kern0pt}nat{\isachardot}{\kern0pt}\ i\ {\isacharless}{\kern0pt}\ length\ {\isacharquery}{\kern0pt}new{\isacharunderscore}{\kern0pt}p\ {\isasymlongrightarrow}\isanewline
\ \ \ \ \ \ \ \ \ \ {\isasymnot}Preference{\isacharunderscore}{\kern0pt}Relation{\isachardot}{\kern0pt}lifted\ {\isacharquery}{\kern0pt}new{\isacharunderscore}{\kern0pt}Ap\ {\isacharparenleft}{\kern0pt}{\isacharquery}{\kern0pt}new{\isacharunderscore}{\kern0pt}p{\isacharbang}{\kern0pt}i{\isacharparenright}{\kern0pt}\ {\isacharparenleft}{\kern0pt}{\isacharquery}{\kern0pt}new{\isacharunderscore}{\kern0pt}q{\isacharbang}{\kern0pt}i{\isacharparenright}{\kern0pt}\ a{\isacharparenright}{\kern0pt}\ {\isasymor}\isanewline
\ \ \ \ \ \ \ {\isacharparenleft}{\kern0pt}{\isasymexists}i{\isacharcolon}{\kern0pt}{\isacharcolon}{\kern0pt}nat{\isachardot}{\kern0pt}\ i\ {\isacharless}{\kern0pt}\ length\ {\isacharquery}{\kern0pt}new{\isacharunderscore}{\kern0pt}p\ {\isasymand}\isanewline
\ \ \ \ \ \ \ \ \ \ {\isasymnot}Preference{\isacharunderscore}{\kern0pt}Relation{\isachardot}{\kern0pt}lifted\ {\isacharquery}{\kern0pt}new{\isacharunderscore}{\kern0pt}Ap\ {\isacharparenleft}{\kern0pt}{\isacharquery}{\kern0pt}new{\isacharunderscore}{\kern0pt}p{\isacharbang}{\kern0pt}i{\isacharparenright}{\kern0pt}\ {\isacharparenleft}{\kern0pt}{\isacharquery}{\kern0pt}new{\isacharunderscore}{\kern0pt}q{\isacharbang}{\kern0pt}i{\isacharparenright}{\kern0pt}\ a\ {\isasymand}\isanewline
\ \ \ \ \ \ \ \ \ \ \ \ \ \ {\isacharparenleft}{\kern0pt}{\isacharquery}{\kern0pt}new{\isacharunderscore}{\kern0pt}p{\isacharbang}{\kern0pt}i{\isacharparenright}{\kern0pt}\ {\isasymnoteq}\ {\isacharparenleft}{\kern0pt}{\isacharquery}{\kern0pt}new{\isacharunderscore}{\kern0pt}q{\isacharbang}{\kern0pt}i{\isacharparenright}{\kern0pt}{\isacharparenright}{\kern0pt}{\isachardoublequoteclose}\isanewline
\ \ \ \ \ \ \isacommand{using}\isamarkupfalse%
\ a{\isadigit{2}}\ lifted{\isacharunderscore}{\kern0pt}def\isanewline
\ \ \ \ \ \ \isacommand{by}\isamarkupfalse%
\ {\isacharparenleft}{\kern0pt}metis\ {\isacharparenleft}{\kern0pt}no{\isacharunderscore}{\kern0pt}types{\isacharcomma}{\kern0pt}\ lifting{\isacharparenright}{\kern0pt}{\isacharparenright}{\kern0pt}\isanewline
\ \ \ \ \isacommand{from}\isamarkupfalse%
\ def{\isacharunderscore}{\kern0pt}and{\isacharunderscore}{\kern0pt}lifted\ modules\ \isacommand{have}\isamarkupfalse%
\isanewline
\ \ \ \ \ \ {\isachardoublequoteopen}{\isasymforall}i{\isachardot}{\kern0pt}\ {\isacharparenleft}{\kern0pt}{\isadigit{0}}\ {\isasymle}\ i\ {\isasymand}\ i\ {\isacharless}{\kern0pt}\ length\ {\isacharquery}{\kern0pt}new{\isacharunderscore}{\kern0pt}p{\isacharparenright}{\kern0pt}\ {\isasymlongrightarrow}\isanewline
\ \ \ \ \ \ \ \ \ \ {\isacharparenleft}{\kern0pt}Preference{\isacharunderscore}{\kern0pt}Relation{\isachardot}{\kern0pt}lifted\ A\ {\isacharparenleft}{\kern0pt}p{\isacharbang}{\kern0pt}i{\isacharparenright}{\kern0pt}\ {\isacharparenleft}{\kern0pt}q{\isacharbang}{\kern0pt}i{\isacharparenright}{\kern0pt}\ a\ {\isasymor}\ {\isacharparenleft}{\kern0pt}p{\isacharbang}{\kern0pt}i{\isacharparenright}{\kern0pt}\ {\isacharequal}{\kern0pt}\ {\isacharparenleft}{\kern0pt}q{\isacharbang}{\kern0pt}i{\isacharparenright}{\kern0pt}{\isacharparenright}{\kern0pt}{\isachardoublequoteclose}\isanewline
\ \ \ \ \ \ \isacommand{using}\isamarkupfalse%
\ defer{\isacharunderscore}{\kern0pt}in{\isacharunderscore}{\kern0pt}alts\ Profile{\isachardot}{\kern0pt}lifted{\isacharunderscore}{\kern0pt}def\ limit{\isacharunderscore}{\kern0pt}prof{\isacharunderscore}{\kern0pt}presv{\isacharunderscore}{\kern0pt}size\isanewline
\ \ \ \ \ \ \isacommand{by}\isamarkupfalse%
\ metis\isanewline
\ \ \ \ \isacommand{with}\isamarkupfalse%
\ def{\isacharunderscore}{\kern0pt}and{\isacharunderscore}{\kern0pt}lifted\ modules\ mono{\isacharunderscore}{\kern0pt}m\ \isacommand{have}\isamarkupfalse%
\isanewline
\ \ \ \ \ \ {\isachardoublequoteopen}{\isasymforall}i{\isachardot}{\kern0pt}\ {\isacharparenleft}{\kern0pt}{\isadigit{0}}\ {\isasymle}\ i\ {\isasymand}\ i\ {\isacharless}{\kern0pt}\ length\ {\isacharquery}{\kern0pt}new{\isacharunderscore}{\kern0pt}p{\isacharparenright}{\kern0pt}\ {\isasymlongrightarrow}\isanewline
\ \ \ \ \ \ \ \ \ \ {\isacharparenleft}{\kern0pt}Preference{\isacharunderscore}{\kern0pt}Relation{\isachardot}{\kern0pt}lifted\ {\isacharquery}{\kern0pt}new{\isacharunderscore}{\kern0pt}Ap\ {\isacharparenleft}{\kern0pt}{\isacharquery}{\kern0pt}new{\isacharunderscore}{\kern0pt}p{\isacharbang}{\kern0pt}i{\isacharparenright}{\kern0pt}\ {\isacharparenleft}{\kern0pt}{\isacharquery}{\kern0pt}new{\isacharunderscore}{\kern0pt}q{\isacharbang}{\kern0pt}i{\isacharparenright}{\kern0pt}\ a\ {\isasymor}\isanewline
\ \ \ \ \ \ \ \ \ \ \ {\isacharparenleft}{\kern0pt}{\isacharquery}{\kern0pt}new{\isacharunderscore}{\kern0pt}p{\isacharbang}{\kern0pt}i{\isacharparenright}{\kern0pt}\ {\isacharequal}{\kern0pt}\ {\isacharparenleft}{\kern0pt}{\isacharquery}{\kern0pt}new{\isacharunderscore}{\kern0pt}q{\isacharbang}{\kern0pt}i{\isacharparenright}{\kern0pt}{\isacharparenright}{\kern0pt}{\isachardoublequoteclose}\isanewline
\ \ \ \ \ \ \isacommand{using}\isamarkupfalse%
\ limit{\isacharunderscore}{\kern0pt}lifted{\isacharunderscore}{\kern0pt}imp{\isacharunderscore}{\kern0pt}eq{\isacharunderscore}{\kern0pt}or{\isacharunderscore}{\kern0pt}lifted\ defer{\isacharunderscore}{\kern0pt}in{\isacharunderscore}{\kern0pt}alts\isanewline
\ \ \ \ \ \ \ \ \ \ \ \ Profile{\isachardot}{\kern0pt}lifted{\isacharunderscore}{\kern0pt}def\ limit{\isacharunderscore}{\kern0pt}prof{\isacharunderscore}{\kern0pt}presv{\isacharunderscore}{\kern0pt}size\isanewline
\ \ \ \ \ \ \ \ \ \ \ \ limit{\isacharunderscore}{\kern0pt}profile{\isachardot}{\kern0pt}simps\ nth{\isacharunderscore}{\kern0pt}map\isanewline
\ \ \ \ \ \ \isacommand{by}\isamarkupfalse%
\ {\isacharparenleft}{\kern0pt}metis\ {\isacharparenleft}{\kern0pt}no{\isacharunderscore}{\kern0pt}types{\isacharcomma}{\kern0pt}\ lifting{\isacharparenright}{\kern0pt}{\isacharparenright}{\kern0pt}\isanewline
\ \ \ \ \isacommand{with}\isamarkupfalse%
\ {\isadigit{0}}\ eql{\isacharunderscore}{\kern0pt}lengths\ mono{\isacharunderscore}{\kern0pt}m\isanewline
\ \ \ \ \isacommand{show}\isamarkupfalse%
\ {\isacharquery}{\kern0pt}thesis\isanewline
\ \ \ \ \ \ \isacommand{using}\isamarkupfalse%
\ leI\ not{\isacharunderscore}{\kern0pt}less{\isacharunderscore}{\kern0pt}zero\ nth{\isacharunderscore}{\kern0pt}equalityI\isanewline
\ \ \ \ \ \ \isacommand{by}\isamarkupfalse%
\ metis\isanewline
\ \ \isacommand{qed}\isamarkupfalse%
\isanewline
\ \ \isacommand{from}\isamarkupfalse%
\ mono{\isacharunderscore}{\kern0pt}m\ mono{\isacharunderscore}{\kern0pt}n\isanewline
\ \ \isacommand{show}\isamarkupfalse%
\ {\isacharquery}{\kern0pt}thesis\isanewline
\ \ \ \ \isacommand{using}\isamarkupfalse%
\ sequential{\isacharunderscore}{\kern0pt}composition{\isachardot}{\kern0pt}simps\isanewline
\ \ \ \ \isacommand{by}\isamarkupfalse%
\ {\isacharparenleft}{\kern0pt}metis\ {\isacharparenleft}{\kern0pt}full{\isacharunderscore}{\kern0pt}types{\isacharparenright}{\kern0pt}{\isacharparenright}{\kern0pt}\isanewline
\isacommand{qed}\isamarkupfalse%
%
\endisatagproof
{\isafoldproof}%
%
\isadelimproof
\isanewline
%
\endisadelimproof
\isanewline
\isanewline
\isacommand{theorem}\isamarkupfalse%
\ seq{\isacharunderscore}{\kern0pt}comp{\isacharunderscore}{\kern0pt}presv{\isacharunderscore}{\kern0pt}def{\isacharunderscore}{\kern0pt}lift{\isacharunderscore}{\kern0pt}inv{\isacharbrackleft}{\kern0pt}simp{\isacharbrackright}{\kern0pt}{\isacharcolon}{\kern0pt}\isanewline
\ \ \isakeyword{assumes}\isanewline
\ \ \ \ monotone{\isacharunderscore}{\kern0pt}m{\isacharcolon}{\kern0pt}\ {\isachardoublequoteopen}defer{\isacharunderscore}{\kern0pt}lift{\isacharunderscore}{\kern0pt}invariance\ m{\isachardoublequoteclose}\ \isakeyword{and}\isanewline
\ \ \ \ monotone{\isacharunderscore}{\kern0pt}n{\isacharcolon}{\kern0pt}\ {\isachardoublequoteopen}defer{\isacharunderscore}{\kern0pt}lift{\isacharunderscore}{\kern0pt}invariance\ n{\isachardoublequoteclose}\isanewline
\ \ \isakeyword{shows}\ {\isachardoublequoteopen}defer{\isacharunderscore}{\kern0pt}lift{\isacharunderscore}{\kern0pt}invariance\ {\isacharparenleft}{\kern0pt}m\ {\isasymtriangleright}\ n{\isacharparenright}{\kern0pt}{\isachardoublequoteclose}\isanewline
%
\isadelimproof
\ \ %
\endisadelimproof
%
\isatagproof
\isacommand{using}\isamarkupfalse%
\ monotone{\isacharunderscore}{\kern0pt}m\ monotone{\isacharunderscore}{\kern0pt}n\ def{\isacharunderscore}{\kern0pt}lift{\isacharunderscore}{\kern0pt}inv{\isacharunderscore}{\kern0pt}seq{\isacharunderscore}{\kern0pt}comp{\isacharunderscore}{\kern0pt}help\isanewline
\ \ \ \ \ \ \ \ seq{\isacharunderscore}{\kern0pt}comp{\isacharunderscore}{\kern0pt}sound\ defer{\isacharunderscore}{\kern0pt}lift{\isacharunderscore}{\kern0pt}invariance{\isacharunderscore}{\kern0pt}def\isanewline
\ \ \isacommand{by}\isamarkupfalse%
\ {\isacharparenleft}{\kern0pt}metis\ {\isacharparenleft}{\kern0pt}full{\isacharunderscore}{\kern0pt}types{\isacharparenright}{\kern0pt}{\isacharparenright}{\kern0pt}%
\endisatagproof
{\isafoldproof}%
%
\isadelimproof
\isanewline
%
\endisadelimproof
\isanewline
\isacommand{lemma}\isamarkupfalse%
\ loop{\isacharunderscore}{\kern0pt}comp{\isacharunderscore}{\kern0pt}helper{\isacharunderscore}{\kern0pt}def{\isacharunderscore}{\kern0pt}lift{\isacharunderscore}{\kern0pt}inv{\isacharunderscore}{\kern0pt}helper{\isacharcolon}{\kern0pt}\isanewline
\ \ \isakeyword{assumes}\isanewline
\ \ \ \ monotone{\isacharunderscore}{\kern0pt}m{\isacharcolon}{\kern0pt}\ {\isachardoublequoteopen}defer{\isacharunderscore}{\kern0pt}lift{\isacharunderscore}{\kern0pt}invariance\ m{\isachardoublequoteclose}\ \isakeyword{and}\isanewline
\ \ \ \ f{\isacharunderscore}{\kern0pt}prof{\isacharcolon}{\kern0pt}\ {\isachardoublequoteopen}finite{\isacharunderscore}{\kern0pt}profile\ A\ p{\isachardoublequoteclose}\isanewline
\ \ \isakeyword{shows}\isanewline
\ \ \ \ {\isachardoublequoteopen}{\isacharparenleft}{\kern0pt}defer{\isacharunderscore}{\kern0pt}lift{\isacharunderscore}{\kern0pt}invariance\ acc\ {\isasymand}\ n\ {\isacharequal}{\kern0pt}\ card\ {\isacharparenleft}{\kern0pt}defer\ acc\ A\ p{\isacharparenright}{\kern0pt}{\isacharparenright}{\kern0pt}\ {\isasymlongrightarrow}\isanewline
\ \ \ \ \ \ \ \ {\isacharparenleft}{\kern0pt}{\isasymforall}q\ a{\isachardot}{\kern0pt}\isanewline
\ \ \ \ \ \ \ \ \ \ {\isacharparenleft}{\kern0pt}a\ {\isasymin}\ {\isacharparenleft}{\kern0pt}defer\ {\isacharparenleft}{\kern0pt}loop{\isacharunderscore}{\kern0pt}comp{\isacharunderscore}{\kern0pt}helper\ acc\ m\ t{\isacharparenright}{\kern0pt}\ A\ p{\isacharparenright}{\kern0pt}\ {\isasymand}\isanewline
\ \ \ \ \ \ \ \ \ \ \ \ lifted\ A\ p\ q\ a{\isacharparenright}{\kern0pt}\ {\isasymlongrightarrow}\isanewline
\ \ \ \ \ \ \ \ \ \ \ \ \ \ \ \ {\isacharparenleft}{\kern0pt}loop{\isacharunderscore}{\kern0pt}comp{\isacharunderscore}{\kern0pt}helper\ acc\ m\ t{\isacharparenright}{\kern0pt}\ A\ p\ {\isacharequal}{\kern0pt}\isanewline
\ \ \ \ \ \ \ \ \ \ \ \ \ \ \ \ \ \ {\isacharparenleft}{\kern0pt}loop{\isacharunderscore}{\kern0pt}comp{\isacharunderscore}{\kern0pt}helper\ acc\ m\ t{\isacharparenright}{\kern0pt}\ A\ q{\isacharparenright}{\kern0pt}{\isachardoublequoteclose}\isanewline
%
\isadelimproof
%
\endisadelimproof
%
\isatagproof
\isacommand{proof}\isamarkupfalse%
\ {\isacharparenleft}{\kern0pt}induct\ n\ arbitrary{\isacharcolon}{\kern0pt}\ acc\ rule{\isacharcolon}{\kern0pt}\ less{\isacharunderscore}{\kern0pt}induct{\isacharparenright}{\kern0pt}\isanewline
\ \ \isacommand{case}\isamarkupfalse%
\ {\isacharparenleft}{\kern0pt}less\ n{\isacharparenright}{\kern0pt}\isanewline
\ \ \isacommand{have}\isamarkupfalse%
\ defer{\isacharunderscore}{\kern0pt}card{\isacharunderscore}{\kern0pt}comp{\isacharcolon}{\kern0pt}\isanewline
\ \ \ \ {\isachardoublequoteopen}defer{\isacharunderscore}{\kern0pt}lift{\isacharunderscore}{\kern0pt}invariance\ acc\ {\isasymlongrightarrow}\isanewline
\ \ \ \ \ \ \ \ {\isacharparenleft}{\kern0pt}{\isasymforall}q\ a{\isachardot}{\kern0pt}\ {\isacharparenleft}{\kern0pt}a\ {\isasymin}\ {\isacharparenleft}{\kern0pt}defer\ {\isacharparenleft}{\kern0pt}acc\ {\isasymtriangleright}\ m{\isacharparenright}{\kern0pt}\ A\ p{\isacharparenright}{\kern0pt}\ {\isasymand}\ lifted\ A\ p\ q\ a{\isacharparenright}{\kern0pt}\ {\isasymlongrightarrow}\isanewline
\ \ \ \ \ \ \ \ \ \ \ \ card\ {\isacharparenleft}{\kern0pt}defer\ {\isacharparenleft}{\kern0pt}acc\ {\isasymtriangleright}\ m{\isacharparenright}{\kern0pt}\ A\ p{\isacharparenright}{\kern0pt}\ {\isacharequal}{\kern0pt}\ card\ {\isacharparenleft}{\kern0pt}defer\ {\isacharparenleft}{\kern0pt}acc\ {\isasymtriangleright}\ m{\isacharparenright}{\kern0pt}\ A\ q{\isacharparenright}{\kern0pt}{\isacharparenright}{\kern0pt}{\isachardoublequoteclose}\isanewline
\ \ \ \ \isacommand{using}\isamarkupfalse%
\ monotone{\isacharunderscore}{\kern0pt}m\ def{\isacharunderscore}{\kern0pt}lift{\isacharunderscore}{\kern0pt}inv{\isacharunderscore}{\kern0pt}seq{\isacharunderscore}{\kern0pt}comp{\isacharunderscore}{\kern0pt}help\isanewline
\ \ \ \ \isacommand{by}\isamarkupfalse%
\ metis\isanewline
\ \ \isacommand{have}\isamarkupfalse%
\ defer{\isacharunderscore}{\kern0pt}card{\isacharunderscore}{\kern0pt}acc{\isacharcolon}{\kern0pt}\isanewline
\ \ \ \ {\isachardoublequoteopen}defer{\isacharunderscore}{\kern0pt}lift{\isacharunderscore}{\kern0pt}invariance\ acc\ {\isasymlongrightarrow}\isanewline
\ \ \ \ \ \ \ \ {\isacharparenleft}{\kern0pt}{\isasymforall}q\ a{\isachardot}{\kern0pt}\ {\isacharparenleft}{\kern0pt}a\ {\isasymin}\ {\isacharparenleft}{\kern0pt}defer\ {\isacharparenleft}{\kern0pt}acc{\isacharparenright}{\kern0pt}\ A\ p{\isacharparenright}{\kern0pt}\ {\isasymand}\ lifted\ A\ p\ q\ a{\isacharparenright}{\kern0pt}\ {\isasymlongrightarrow}\isanewline
\ \ \ \ \ \ \ \ \ \ \ \ card\ {\isacharparenleft}{\kern0pt}defer\ {\isacharparenleft}{\kern0pt}acc{\isacharparenright}{\kern0pt}\ A\ p{\isacharparenright}{\kern0pt}\ {\isacharequal}{\kern0pt}\ card\ {\isacharparenleft}{\kern0pt}defer\ {\isacharparenleft}{\kern0pt}acc{\isacharparenright}{\kern0pt}\ A\ q{\isacharparenright}{\kern0pt}{\isacharparenright}{\kern0pt}{\isachardoublequoteclose}\isanewline
\ \ \ \ \isacommand{by}\isamarkupfalse%
\ {\isacharparenleft}{\kern0pt}simp\ add{\isacharcolon}{\kern0pt}\ defer{\isacharunderscore}{\kern0pt}lift{\isacharunderscore}{\kern0pt}invariance{\isacharunderscore}{\kern0pt}def{\isacharparenright}{\kern0pt}\isanewline
\ \ \isacommand{hence}\isamarkupfalse%
\ defer{\isacharunderscore}{\kern0pt}card{\isacharunderscore}{\kern0pt}acc{\isacharunderscore}{\kern0pt}{\isadigit{2}}{\isacharcolon}{\kern0pt}\isanewline
\ \ \ \ {\isachardoublequoteopen}defer{\isacharunderscore}{\kern0pt}lift{\isacharunderscore}{\kern0pt}invariance\ acc\ {\isasymlongrightarrow}\isanewline
\ \ \ \ \ \ \ \ {\isacharparenleft}{\kern0pt}{\isasymforall}q\ a{\isachardot}{\kern0pt}\ {\isacharparenleft}{\kern0pt}a\ {\isasymin}\ {\isacharparenleft}{\kern0pt}defer\ {\isacharparenleft}{\kern0pt}acc\ {\isasymtriangleright}\ m{\isacharparenright}{\kern0pt}\ A\ p{\isacharparenright}{\kern0pt}\ {\isasymand}\ lifted\ A\ p\ q\ a{\isacharparenright}{\kern0pt}\ {\isasymlongrightarrow}\isanewline
\ \ \ \ \ \ \ \ \ \ \ \ card\ {\isacharparenleft}{\kern0pt}defer\ {\isacharparenleft}{\kern0pt}acc{\isacharparenright}{\kern0pt}\ A\ p{\isacharparenright}{\kern0pt}\ {\isacharequal}{\kern0pt}\ card\ {\isacharparenleft}{\kern0pt}defer\ {\isacharparenleft}{\kern0pt}acc{\isacharparenright}{\kern0pt}\ A\ q{\isacharparenright}{\kern0pt}{\isacharparenright}{\kern0pt}{\isachardoublequoteclose}\isanewline
\ \ \ \ \isacommand{using}\isamarkupfalse%
\ monotone{\isacharunderscore}{\kern0pt}m\ f{\isacharunderscore}{\kern0pt}prof\ defer{\isacharunderscore}{\kern0pt}lift{\isacharunderscore}{\kern0pt}invariance{\isacharunderscore}{\kern0pt}def\ seq{\isacharunderscore}{\kern0pt}comp{\isacharunderscore}{\kern0pt}def{\isacharunderscore}{\kern0pt}set{\isacharunderscore}{\kern0pt}trans\isanewline
\ \ \ \ \isacommand{by}\isamarkupfalse%
\ metis\isanewline
\ \ \isacommand{thus}\isamarkupfalse%
\ {\isacharquery}{\kern0pt}case\isanewline
\ \ \isacommand{proof}\isamarkupfalse%
\ cases\isanewline
\ \ \ \ \isacommand{assume}\isamarkupfalse%
\ card{\isacharunderscore}{\kern0pt}unchanged{\isacharcolon}{\kern0pt}\ {\isachardoublequoteopen}card\ {\isacharparenleft}{\kern0pt}defer\ {\isacharparenleft}{\kern0pt}acc\ {\isasymtriangleright}\ m{\isacharparenright}{\kern0pt}\ A\ p{\isacharparenright}{\kern0pt}\ {\isacharequal}{\kern0pt}\ card\ {\isacharparenleft}{\kern0pt}defer\ acc\ A\ p{\isacharparenright}{\kern0pt}{\isachardoublequoteclose}\isanewline
\ \ \ \ \isacommand{with}\isamarkupfalse%
\ defer{\isacharunderscore}{\kern0pt}card{\isacharunderscore}{\kern0pt}comp\ defer{\isacharunderscore}{\kern0pt}card{\isacharunderscore}{\kern0pt}acc\ monotone{\isacharunderscore}{\kern0pt}m\isanewline
\ \ \ \ \isacommand{have}\isamarkupfalse%
\isanewline
\ \ \ \ \ \ {\isachardoublequoteopen}defer{\isacharunderscore}{\kern0pt}lift{\isacharunderscore}{\kern0pt}invariance\ {\isacharparenleft}{\kern0pt}acc{\isacharparenright}{\kern0pt}\ {\isasymlongrightarrow}\isanewline
\ \ \ \ \ \ \ \ \ \ {\isacharparenleft}{\kern0pt}{\isasymforall}q\ a{\isachardot}{\kern0pt}\ {\isacharparenleft}{\kern0pt}a\ {\isasymin}\ {\isacharparenleft}{\kern0pt}defer\ {\isacharparenleft}{\kern0pt}acc{\isacharparenright}{\kern0pt}\ A\ p{\isacharparenright}{\kern0pt}\ {\isasymand}\ lifted\ A\ p\ q\ a{\isacharparenright}{\kern0pt}\ {\isasymlongrightarrow}\isanewline
\ \ \ \ \ \ \ \ \ \ \ \ \ \ {\isacharparenleft}{\kern0pt}loop{\isacharunderscore}{\kern0pt}comp{\isacharunderscore}{\kern0pt}helper\ acc\ m\ t{\isacharparenright}{\kern0pt}\ A\ q\ {\isacharequal}{\kern0pt}\ acc\ A\ q{\isacharparenright}{\kern0pt}{\isachardoublequoteclose}\isanewline
\ \ \ \ \ \ \isacommand{using}\isamarkupfalse%
\ card{\isacharunderscore}{\kern0pt}subset{\isacharunderscore}{\kern0pt}eq\ defer{\isacharunderscore}{\kern0pt}in{\isacharunderscore}{\kern0pt}alts\ less{\isacharunderscore}{\kern0pt}irrefl\isanewline
\ \ \ \ \ \ \ \ \ \ \ \ loop{\isacharunderscore}{\kern0pt}comp{\isacharunderscore}{\kern0pt}helper{\isachardot}{\kern0pt}simps{\isacharparenleft}{\kern0pt}{\isadigit{1}}{\isacharparenright}{\kern0pt}\ f{\isacharunderscore}{\kern0pt}prof\ psubset{\isacharunderscore}{\kern0pt}card{\isacharunderscore}{\kern0pt}mono\isanewline
\ \ \ \ \ \ \ \ \ \ \ \ sequential{\isacharunderscore}{\kern0pt}composition{\isachardot}{\kern0pt}simps\ def{\isacharunderscore}{\kern0pt}presv{\isacharunderscore}{\kern0pt}fin{\isacharunderscore}{\kern0pt}prof\ snd{\isacharunderscore}{\kern0pt}conv\isanewline
\ \ \ \ \ \ \ \ \ \ \ \ defer{\isacharunderscore}{\kern0pt}lift{\isacharunderscore}{\kern0pt}invariance{\isacharunderscore}{\kern0pt}def\ seq{\isacharunderscore}{\kern0pt}comp{\isacharunderscore}{\kern0pt}def{\isacharunderscore}{\kern0pt}set{\isacharunderscore}{\kern0pt}bounded\isanewline
\ \ \ \ \ \ \ \ \ \ \ \ loop{\isacharunderscore}{\kern0pt}comp{\isacharunderscore}{\kern0pt}code{\isacharunderscore}{\kern0pt}helper\isanewline
\ \ \ \ \ \ \isacommand{by}\isamarkupfalse%
\ {\isacharparenleft}{\kern0pt}smt\ {\isacharparenleft}{\kern0pt}verit{\isacharparenright}{\kern0pt}{\isacharparenright}{\kern0pt}\isanewline
\ \ \ \ \isacommand{moreover}\isamarkupfalse%
\ \isacommand{from}\isamarkupfalse%
\ card{\isacharunderscore}{\kern0pt}unchanged\ \isacommand{have}\isamarkupfalse%
\isanewline
\ \ \ \ \ \ {\isachardoublequoteopen}{\isacharparenleft}{\kern0pt}loop{\isacharunderscore}{\kern0pt}comp{\isacharunderscore}{\kern0pt}helper\ acc\ m\ t{\isacharparenright}{\kern0pt}\ A\ p\ {\isacharequal}{\kern0pt}\ acc\ A\ p{\isachardoublequoteclose}\isanewline
\ \ \ \ \ \ \isacommand{using}\isamarkupfalse%
\ loop{\isacharunderscore}{\kern0pt}comp{\isacharunderscore}{\kern0pt}helper{\isachardot}{\kern0pt}simps{\isacharparenleft}{\kern0pt}{\isadigit{1}}{\isacharparenright}{\kern0pt}\ order{\isachardot}{\kern0pt}strict{\isacharunderscore}{\kern0pt}iff{\isacharunderscore}{\kern0pt}order\isanewline
\ \ \ \ \ \ \ \ \ \ \ \ psubset{\isacharunderscore}{\kern0pt}card{\isacharunderscore}{\kern0pt}mono\isanewline
\ \ \ \ \ \ \isacommand{by}\isamarkupfalse%
\ metis\isanewline
\ \ \ \ \isacommand{ultimately}\isamarkupfalse%
\ \isacommand{have}\isamarkupfalse%
\isanewline
\ \ \ \ \ \ {\isachardoublequoteopen}{\isacharparenleft}{\kern0pt}defer{\isacharunderscore}{\kern0pt}lift{\isacharunderscore}{\kern0pt}invariance\ {\isacharparenleft}{\kern0pt}acc\ {\isasymtriangleright}\ m{\isacharparenright}{\kern0pt}\ {\isasymand}\ defer{\isacharunderscore}{\kern0pt}lift{\isacharunderscore}{\kern0pt}invariance\ acc{\isacharparenright}{\kern0pt}\ {\isasymlongrightarrow}\isanewline
\ \ \ \ \ \ \ \ \ \ {\isacharparenleft}{\kern0pt}{\isasymforall}q\ a{\isachardot}{\kern0pt}\ {\isacharparenleft}{\kern0pt}a\ {\isasymin}\ {\isacharparenleft}{\kern0pt}defer\ {\isacharparenleft}{\kern0pt}loop{\isacharunderscore}{\kern0pt}comp{\isacharunderscore}{\kern0pt}helper\ acc\ m\ t{\isacharparenright}{\kern0pt}\ A\ p{\isacharparenright}{\kern0pt}\ {\isasymand}\isanewline
\ \ \ \ \ \ \ \ \ \ \ \ \ \ lifted\ A\ p\ q\ a{\isacharparenright}{\kern0pt}\ {\isasymlongrightarrow}\isanewline
\ \ \ \ \ \ \ \ \ \ \ \ \ \ \ \ \ \ {\isacharparenleft}{\kern0pt}loop{\isacharunderscore}{\kern0pt}comp{\isacharunderscore}{\kern0pt}helper\ acc\ m\ t{\isacharparenright}{\kern0pt}\ A\ p\ {\isacharequal}{\kern0pt}\isanewline
\ \ \ \ \ \ \ \ \ \ \ \ \ \ \ \ \ \ \ \ {\isacharparenleft}{\kern0pt}loop{\isacharunderscore}{\kern0pt}comp{\isacharunderscore}{\kern0pt}helper\ acc\ m\ t{\isacharparenright}{\kern0pt}\ A\ q{\isacharparenright}{\kern0pt}{\isachardoublequoteclose}\isanewline
\ \ \ \ \ \ \isacommand{using}\isamarkupfalse%
\ defer{\isacharunderscore}{\kern0pt}lift{\isacharunderscore}{\kern0pt}invariance{\isacharunderscore}{\kern0pt}def\isanewline
\ \ \ \ \ \ \isacommand{by}\isamarkupfalse%
\ metis\isanewline
\ \ \ \ \isacommand{thus}\isamarkupfalse%
\ {\isacharquery}{\kern0pt}thesis\isanewline
\ \ \ \ \ \ \isacommand{using}\isamarkupfalse%
\ monotone{\isacharunderscore}{\kern0pt}m\ seq{\isacharunderscore}{\kern0pt}comp{\isacharunderscore}{\kern0pt}presv{\isacharunderscore}{\kern0pt}def{\isacharunderscore}{\kern0pt}lift{\isacharunderscore}{\kern0pt}inv\isanewline
\ \ \ \ \ \ \isacommand{by}\isamarkupfalse%
\ blast\isanewline
\ \ \isacommand{next}\isamarkupfalse%
\isanewline
\ \ \ \ \isacommand{assume}\isamarkupfalse%
\ card{\isacharunderscore}{\kern0pt}changed{\isacharcolon}{\kern0pt}\isanewline
\ \ \ \ \ \ {\isachardoublequoteopen}{\isasymnot}\ {\isacharparenleft}{\kern0pt}card\ {\isacharparenleft}{\kern0pt}defer\ {\isacharparenleft}{\kern0pt}acc\ {\isasymtriangleright}\ m{\isacharparenright}{\kern0pt}\ A\ p{\isacharparenright}{\kern0pt}\ {\isacharequal}{\kern0pt}\ card\ {\isacharparenleft}{\kern0pt}defer\ acc\ A\ p{\isacharparenright}{\kern0pt}{\isacharparenright}{\kern0pt}{\isachardoublequoteclose}\isanewline
\ \ \ \ \isacommand{with}\isamarkupfalse%
\ f{\isacharunderscore}{\kern0pt}prof\ seq{\isacharunderscore}{\kern0pt}comp{\isacharunderscore}{\kern0pt}def{\isacharunderscore}{\kern0pt}card{\isacharunderscore}{\kern0pt}bounded\ \isacommand{have}\isamarkupfalse%
\ card{\isacharunderscore}{\kern0pt}smaller{\isacharunderscore}{\kern0pt}for{\isacharunderscore}{\kern0pt}p{\isacharcolon}{\kern0pt}\isanewline
\ \ \ \ \ \ {\isachardoublequoteopen}electoral{\isacharunderscore}{\kern0pt}module\ {\isacharparenleft}{\kern0pt}acc{\isacharparenright}{\kern0pt}\ {\isasymlongrightarrow}\isanewline
\ \ \ \ \ \ \ \ \ \ {\isacharparenleft}{\kern0pt}card\ {\isacharparenleft}{\kern0pt}defer\ {\isacharparenleft}{\kern0pt}acc\ {\isasymtriangleright}\ m{\isacharparenright}{\kern0pt}\ A\ p{\isacharparenright}{\kern0pt}\ {\isacharless}{\kern0pt}\ card\ {\isacharparenleft}{\kern0pt}defer\ acc\ A\ p{\isacharparenright}{\kern0pt}{\isacharparenright}{\kern0pt}{\isachardoublequoteclose}\isanewline
\ \ \ \ \ \ \isacommand{using}\isamarkupfalse%
\ monotone{\isacharunderscore}{\kern0pt}m\ order{\isachardot}{\kern0pt}not{\isacharunderscore}{\kern0pt}eq{\isacharunderscore}{\kern0pt}order{\isacharunderscore}{\kern0pt}implies{\isacharunderscore}{\kern0pt}strict\isanewline
\ \ \ \ \ \ \ \ \ \ \ \ defer{\isacharunderscore}{\kern0pt}lift{\isacharunderscore}{\kern0pt}invariance{\isacharunderscore}{\kern0pt}def\isanewline
\ \ \ \ \ \ \isacommand{by}\isamarkupfalse%
\ {\isacharparenleft}{\kern0pt}metis\ {\isacharparenleft}{\kern0pt}full{\isacharunderscore}{\kern0pt}types{\isacharparenright}{\kern0pt}{\isacharparenright}{\kern0pt}\isanewline
\ \ \ \ \isacommand{with}\isamarkupfalse%
\ defer{\isacharunderscore}{\kern0pt}card{\isacharunderscore}{\kern0pt}acc{\isacharunderscore}{\kern0pt}{\isadigit{2}}\ defer{\isacharunderscore}{\kern0pt}card{\isacharunderscore}{\kern0pt}comp\ \isacommand{have}\isamarkupfalse%
\ card{\isacharunderscore}{\kern0pt}changed{\isacharunderscore}{\kern0pt}for{\isacharunderscore}{\kern0pt}q{\isacharcolon}{\kern0pt}\isanewline
\ \ \ \ \ \ {\isachardoublequoteopen}defer{\isacharunderscore}{\kern0pt}lift{\isacharunderscore}{\kern0pt}invariance\ {\isacharparenleft}{\kern0pt}acc{\isacharparenright}{\kern0pt}\ {\isasymlongrightarrow}\isanewline
\ \ \ \ \ \ \ \ \ \ {\isacharparenleft}{\kern0pt}{\isasymforall}q\ a{\isachardot}{\kern0pt}\ {\isacharparenleft}{\kern0pt}a\ {\isasymin}\ {\isacharparenleft}{\kern0pt}defer\ {\isacharparenleft}{\kern0pt}acc\ {\isasymtriangleright}\ m{\isacharparenright}{\kern0pt}\ A\ p{\isacharparenright}{\kern0pt}\ {\isasymand}\ lifted\ A\ p\ q\ a{\isacharparenright}{\kern0pt}\ {\isasymlongrightarrow}\isanewline
\ \ \ \ \ \ \ \ \ \ \ \ \ \ {\isacharparenleft}{\kern0pt}card\ {\isacharparenleft}{\kern0pt}defer\ {\isacharparenleft}{\kern0pt}acc\ {\isasymtriangleright}\ m{\isacharparenright}{\kern0pt}\ A\ q{\isacharparenright}{\kern0pt}\ {\isacharless}{\kern0pt}\ card\ {\isacharparenleft}{\kern0pt}defer\ acc\ A\ q{\isacharparenright}{\kern0pt}{\isacharparenright}{\kern0pt}{\isacharparenright}{\kern0pt}{\isachardoublequoteclose}\isanewline
\ \ \ \ \ \ \isacommand{using}\isamarkupfalse%
\ defer{\isacharunderscore}{\kern0pt}lift{\isacharunderscore}{\kern0pt}invariance{\isacharunderscore}{\kern0pt}def\isanewline
\ \ \ \ \ \ \isacommand{by}\isamarkupfalse%
\ {\isacharparenleft}{\kern0pt}metis\ {\isacharparenleft}{\kern0pt}no{\isacharunderscore}{\kern0pt}types{\isacharcomma}{\kern0pt}\ lifting{\isacharparenright}{\kern0pt}{\isacharparenright}{\kern0pt}\isanewline
\ \ \ \ \isacommand{thus}\isamarkupfalse%
\ {\isacharquery}{\kern0pt}thesis\isanewline
\ \ \ \ \isacommand{proof}\isamarkupfalse%
\ cases\isanewline
\ \ \ \ \ \ \isacommand{assume}\isamarkupfalse%
\ t{\isacharunderscore}{\kern0pt}not{\isacharunderscore}{\kern0pt}satisfied{\isacharunderscore}{\kern0pt}for{\isacharunderscore}{\kern0pt}p{\isacharcolon}{\kern0pt}\ {\isachardoublequoteopen}{\isasymnot}\ t\ {\isacharparenleft}{\kern0pt}acc\ A\ p{\isacharparenright}{\kern0pt}{\isachardoublequoteclose}\isanewline
\ \ \ \ \ \ \isacommand{hence}\isamarkupfalse%
\ t{\isacharunderscore}{\kern0pt}not{\isacharunderscore}{\kern0pt}satisfied{\isacharunderscore}{\kern0pt}for{\isacharunderscore}{\kern0pt}q{\isacharcolon}{\kern0pt}\isanewline
\ \ \ \ \ \ \ \ {\isachardoublequoteopen}defer{\isacharunderscore}{\kern0pt}lift{\isacharunderscore}{\kern0pt}invariance\ {\isacharparenleft}{\kern0pt}acc{\isacharparenright}{\kern0pt}\ {\isasymlongrightarrow}\isanewline
\ \ \ \ \ \ \ \ \ \ \ \ {\isacharparenleft}{\kern0pt}{\isasymforall}q\ a{\isachardot}{\kern0pt}\ {\isacharparenleft}{\kern0pt}a\ {\isasymin}\ {\isacharparenleft}{\kern0pt}defer\ {\isacharparenleft}{\kern0pt}acc\ {\isasymtriangleright}\ m{\isacharparenright}{\kern0pt}\ A\ p{\isacharparenright}{\kern0pt}\ {\isasymand}\ lifted\ A\ p\ q\ a{\isacharparenright}{\kern0pt}\ {\isasymlongrightarrow}\isanewline
\ \ \ \ \ \ \ \ \ \ \ \ \ \ \ \ {\isasymnot}\ t\ {\isacharparenleft}{\kern0pt}acc\ A\ q{\isacharparenright}{\kern0pt}{\isacharparenright}{\kern0pt}{\isachardoublequoteclose}\isanewline
\ \ \ \ \ \ \ \ \isacommand{using}\isamarkupfalse%
\ monotone{\isacharunderscore}{\kern0pt}m\ f{\isacharunderscore}{\kern0pt}prof\ defer{\isacharunderscore}{\kern0pt}lift{\isacharunderscore}{\kern0pt}invariance{\isacharunderscore}{\kern0pt}def\ seq{\isacharunderscore}{\kern0pt}comp{\isacharunderscore}{\kern0pt}def{\isacharunderscore}{\kern0pt}set{\isacharunderscore}{\kern0pt}trans\isanewline
\ \ \ \ \ \ \ \ \isacommand{by}\isamarkupfalse%
\ metis\isanewline
\ \ \ \ \ \ \isacommand{from}\isamarkupfalse%
\ card{\isacharunderscore}{\kern0pt}changed\ defer{\isacharunderscore}{\kern0pt}card{\isacharunderscore}{\kern0pt}comp\ defer{\isacharunderscore}{\kern0pt}card{\isacharunderscore}{\kern0pt}acc\isanewline
\ \ \ \ \ \ \isacommand{have}\isamarkupfalse%
\isanewline
\ \ \ \ \ \ \ \ {\isachardoublequoteopen}{\isacharparenleft}{\kern0pt}defer{\isacharunderscore}{\kern0pt}lift{\isacharunderscore}{\kern0pt}invariance\ {\isacharparenleft}{\kern0pt}acc\ {\isasymtriangleright}\ m{\isacharparenright}{\kern0pt}\ {\isasymand}\ defer{\isacharunderscore}{\kern0pt}lift{\isacharunderscore}{\kern0pt}invariance\ {\isacharparenleft}{\kern0pt}acc{\isacharparenright}{\kern0pt}{\isacharparenright}{\kern0pt}\ {\isasymlongrightarrow}\isanewline
\ \ \ \ \ \ \ \ \ \ \ \ {\isacharparenleft}{\kern0pt}{\isasymforall}q\ a{\isachardot}{\kern0pt}\ {\isacharparenleft}{\kern0pt}a\ {\isasymin}\ {\isacharparenleft}{\kern0pt}defer\ {\isacharparenleft}{\kern0pt}acc\ {\isasymtriangleright}\ m{\isacharparenright}{\kern0pt}\ A\ p{\isacharparenright}{\kern0pt}\ {\isasymand}\ lifted\ A\ p\ q\ a{\isacharparenright}{\kern0pt}\ {\isasymlongrightarrow}\isanewline
\ \ \ \ \ \ \ \ \ \ \ \ \ \ \ \ card\ {\isacharparenleft}{\kern0pt}defer\ {\isacharparenleft}{\kern0pt}acc\ {\isasymtriangleright}\ m{\isacharparenright}{\kern0pt}\ A\ q{\isacharparenright}{\kern0pt}\ {\isasymnoteq}\ {\isacharparenleft}{\kern0pt}card\ {\isacharparenleft}{\kern0pt}defer\ acc\ A\ q{\isacharparenright}{\kern0pt}{\isacharparenright}{\kern0pt}{\isacharparenright}{\kern0pt}{\isachardoublequoteclose}\isanewline
\ \ \ \ \ \ \isacommand{proof}\isamarkupfalse%
\ {\isacharminus}{\kern0pt}\isanewline
\ \ \ \ \ \ \ \ \isacommand{have}\isamarkupfalse%
\isanewline
\ \ \ \ \ \ \ \ \ \ {\isachardoublequoteopen}{\isasymforall}f{\isachardot}{\kern0pt}\ {\isacharparenleft}{\kern0pt}defer{\isacharunderscore}{\kern0pt}lift{\isacharunderscore}{\kern0pt}invariance\ f\ {\isasymor}\isanewline
\ \ \ \ \ \ \ \ \ \ \ \ {\isacharparenleft}{\kern0pt}{\isasymexists}A\ rs\ rsa\ a{\isachardot}{\kern0pt}\ f\ A\ rs\ {\isasymnoteq}\ f\ A\ rsa\ {\isasymand}\isanewline
\ \ \ \ \ \ \ \ \ \ \ \ \ \ Profile{\isachardot}{\kern0pt}lifted\ A\ rs\ rsa\ {\isacharparenleft}{\kern0pt}a{\isacharcolon}{\kern0pt}{\isacharcolon}{\kern0pt}{\isacharprime}{\kern0pt}a{\isacharparenright}{\kern0pt}\ {\isasymand}\isanewline
\ \ \ \ \ \ \ \ \ \ \ \ \ \ a\ {\isasymin}\ defer\ f\ A\ rs{\isacharparenright}{\kern0pt}\ {\isasymor}\ {\isasymnot}\ electoral{\isacharunderscore}{\kern0pt}module\ f{\isacharparenright}{\kern0pt}\ {\isasymand}\isanewline
\ \ \ \ \ \ \ \ \ \ \ \ \ \ {\isacharparenleft}{\kern0pt}{\isacharparenleft}{\kern0pt}{\isasymforall}A\ rs\ rsa\ a{\isachardot}{\kern0pt}\ f\ A\ rs\ {\isacharequal}{\kern0pt}\ f\ A\ rsa\ {\isasymor}\ {\isasymnot}\ Profile{\isachardot}{\kern0pt}lifted\ A\ rs\ rsa\ a\ {\isasymor}\isanewline
\ \ \ \ \ \ \ \ \ \ \ \ \ \ \ \ \ \ a\ {\isasymnotin}\ defer\ f\ A\ rs{\isacharparenright}{\kern0pt}\ {\isasymand}\ electoral{\isacharunderscore}{\kern0pt}module\ f\ {\isasymor}\ {\isasymnot}\ defer{\isacharunderscore}{\kern0pt}lift{\isacharunderscore}{\kern0pt}invariance\ f{\isacharparenright}{\kern0pt}{\isachardoublequoteclose}\isanewline
\ \ \ \ \ \ \ \ \ \ \isacommand{using}\isamarkupfalse%
\ defer{\isacharunderscore}{\kern0pt}lift{\isacharunderscore}{\kern0pt}invariance{\isacharunderscore}{\kern0pt}def\isanewline
\ \ \ \ \ \ \ \ \ \ \isacommand{by}\isamarkupfalse%
\ blast\isanewline
\ \ \ \ \ \ \ \ \isacommand{thus}\isamarkupfalse%
\ {\isacharquery}{\kern0pt}thesis\isanewline
\ \ \ \ \ \ \ \ \ \ \isacommand{using}\isamarkupfalse%
\ card{\isacharunderscore}{\kern0pt}changed\ monotone{\isacharunderscore}{\kern0pt}m\ f{\isacharunderscore}{\kern0pt}prof\ seq{\isacharunderscore}{\kern0pt}comp{\isacharunderscore}{\kern0pt}def{\isacharunderscore}{\kern0pt}set{\isacharunderscore}{\kern0pt}trans\isanewline
\ \ \ \ \ \ \ \ \ \ \isacommand{by}\isamarkupfalse%
\ {\isacharparenleft}{\kern0pt}metis\ {\isacharparenleft}{\kern0pt}no{\isacharunderscore}{\kern0pt}types{\isacharcomma}{\kern0pt}\ hide{\isacharunderscore}{\kern0pt}lams{\isacharparenright}{\kern0pt}{\isacharparenright}{\kern0pt}\isanewline
\ \ \ \ \ \ \isacommand{qed}\isamarkupfalse%
\isanewline
\ \ \ \ \ \ \isacommand{hence}\isamarkupfalse%
\isanewline
\ \ \ \ \ \ \ \ {\isachardoublequoteopen}defer{\isacharunderscore}{\kern0pt}lift{\isacharunderscore}{\kern0pt}invariance\ {\isacharparenleft}{\kern0pt}acc\ {\isasymtriangleright}\ m{\isacharparenright}{\kern0pt}\ {\isasymand}\ defer{\isacharunderscore}{\kern0pt}lift{\isacharunderscore}{\kern0pt}invariance\ {\isacharparenleft}{\kern0pt}acc{\isacharparenright}{\kern0pt}\ {\isasymlongrightarrow}\isanewline
\ \ \ \ \ \ \ \ \ \ \ \ {\isacharparenleft}{\kern0pt}{\isasymforall}q\ a{\isachardot}{\kern0pt}\ {\isacharparenleft}{\kern0pt}a\ {\isasymin}\ {\isacharparenleft}{\kern0pt}defer\ {\isacharparenleft}{\kern0pt}acc\ {\isasymtriangleright}\ m{\isacharparenright}{\kern0pt}\ A\ p{\isacharparenright}{\kern0pt}\ {\isasymand}\ lifted\ A\ p\ q\ a{\isacharparenright}{\kern0pt}\ {\isasymlongrightarrow}\isanewline
\ \ \ \ \ \ \ \ \ \ \ \ \ \ \ \ defer\ {\isacharparenleft}{\kern0pt}acc\ {\isasymtriangleright}\ m{\isacharparenright}{\kern0pt}\ A\ q\ {\isasymsubset}\ defer\ acc\ A\ q{\isacharparenright}{\kern0pt}{\isachardoublequoteclose}\isanewline
\ \ \ \ \ \ \ \ \isacommand{using}\isamarkupfalse%
\ defer{\isacharunderscore}{\kern0pt}card{\isacharunderscore}{\kern0pt}acc\ defer{\isacharunderscore}{\kern0pt}in{\isacharunderscore}{\kern0pt}alts\ monotone{\isacharunderscore}{\kern0pt}m\ prod{\isachardot}{\kern0pt}sel{\isacharparenleft}{\kern0pt}{\isadigit{2}}{\isacharparenright}{\kern0pt}\ f{\isacharunderscore}{\kern0pt}prof\isanewline
\ \ \ \ \ \ \ \ \ \ \ \ \ \ psubsetI\ sequential{\isacharunderscore}{\kern0pt}composition{\isachardot}{\kern0pt}simps\ def{\isacharunderscore}{\kern0pt}presv{\isacharunderscore}{\kern0pt}fin{\isacharunderscore}{\kern0pt}prof\isanewline
\ \ \ \ \ \ \ \ \ \ \ \ \ \ defer{\isacharunderscore}{\kern0pt}lift{\isacharunderscore}{\kern0pt}invariance{\isacharunderscore}{\kern0pt}def\ subsetCE\ Profile{\isachardot}{\kern0pt}lifted{\isacharunderscore}{\kern0pt}def\isanewline
\ \ \ \ \ \ \ \ \ \ \ \ \ \ seq{\isacharunderscore}{\kern0pt}comp{\isacharunderscore}{\kern0pt}def{\isacharunderscore}{\kern0pt}set{\isacharunderscore}{\kern0pt}bounded\isanewline
\ \ \ \ \ \ \ \ \isacommand{by}\isamarkupfalse%
\ {\isacharparenleft}{\kern0pt}smt\ {\isacharparenleft}{\kern0pt}verit{\isacharparenright}{\kern0pt}{\isacharparenright}{\kern0pt}\isanewline
\ \ \ \ \ \ \isacommand{with}\isamarkupfalse%
\ t{\isacharunderscore}{\kern0pt}not{\isacharunderscore}{\kern0pt}satisfied{\isacharunderscore}{\kern0pt}for{\isacharunderscore}{\kern0pt}p\ \isacommand{have}\isamarkupfalse%
\ rec{\isacharunderscore}{\kern0pt}step{\isacharunderscore}{\kern0pt}q{\isacharcolon}{\kern0pt}\isanewline
\ \ \ \ \ \ \ \ {\isachardoublequoteopen}{\isacharparenleft}{\kern0pt}defer{\isacharunderscore}{\kern0pt}lift{\isacharunderscore}{\kern0pt}invariance\ {\isacharparenleft}{\kern0pt}acc\ {\isasymtriangleright}\ m{\isacharparenright}{\kern0pt}\ {\isasymand}\ defer{\isacharunderscore}{\kern0pt}lift{\isacharunderscore}{\kern0pt}invariance\ {\isacharparenleft}{\kern0pt}acc{\isacharparenright}{\kern0pt}{\isacharparenright}{\kern0pt}\ {\isasymlongrightarrow}\isanewline
\ \ \ \ \ \ \ \ \ \ \ \ {\isacharparenleft}{\kern0pt}{\isasymforall}q\ a{\isachardot}{\kern0pt}\ {\isacharparenleft}{\kern0pt}a\ {\isasymin}\ {\isacharparenleft}{\kern0pt}defer\ {\isacharparenleft}{\kern0pt}acc\ {\isasymtriangleright}\ m{\isacharparenright}{\kern0pt}\ A\ p{\isacharparenright}{\kern0pt}\ {\isasymand}\ lifted\ A\ p\ q\ a{\isacharparenright}{\kern0pt}\ {\isasymlongrightarrow}\isanewline
\ \ \ \ \ \ \ \ \ \ \ \ \ \ \ \ loop{\isacharunderscore}{\kern0pt}comp{\isacharunderscore}{\kern0pt}helper\ acc\ m\ t\ A\ q\ {\isacharequal}{\kern0pt}\isanewline
\ \ \ \ \ \ \ \ \ \ \ \ \ \ \ \ \ \ loop{\isacharunderscore}{\kern0pt}comp{\isacharunderscore}{\kern0pt}helper\ {\isacharparenleft}{\kern0pt}acc\ {\isasymtriangleright}\ m{\isacharparenright}{\kern0pt}\ m\ t\ A\ q{\isacharparenright}{\kern0pt}{\isachardoublequoteclose}\isanewline
\ \ \ \ \ \ \ \ \isacommand{using}\isamarkupfalse%
\ defer{\isacharunderscore}{\kern0pt}in{\isacharunderscore}{\kern0pt}alts\ loop{\isacharunderscore}{\kern0pt}comp{\isacharunderscore}{\kern0pt}helper{\isachardot}{\kern0pt}simps{\isacharparenleft}{\kern0pt}{\isadigit{2}}{\isacharparenright}{\kern0pt}\ monotone{\isacharunderscore}{\kern0pt}m\ subsetCE\isanewline
\ \ \ \ \ \ \ \ \ \ \ \ \ \ prod{\isachardot}{\kern0pt}sel{\isacharparenleft}{\kern0pt}{\isadigit{2}}{\isacharparenright}{\kern0pt}\ f{\isacharunderscore}{\kern0pt}prof\ sequential{\isacharunderscore}{\kern0pt}composition{\isachardot}{\kern0pt}simps\ card{\isacharunderscore}{\kern0pt}eq{\isacharunderscore}{\kern0pt}{\isadigit{0}}{\isacharunderscore}{\kern0pt}iff\isanewline
\ \ \ \ \ \ \ \ \ \ \ \ \ \ def{\isacharunderscore}{\kern0pt}presv{\isacharunderscore}{\kern0pt}fin{\isacharunderscore}{\kern0pt}prof\ defer{\isacharunderscore}{\kern0pt}lift{\isacharunderscore}{\kern0pt}invariance{\isacharunderscore}{\kern0pt}def\ card{\isacharunderscore}{\kern0pt}changed{\isacharunderscore}{\kern0pt}for{\isacharunderscore}{\kern0pt}q\isanewline
\ \ \ \ \ \ \ \ \ \ \ \ \ \ gr{\isacharunderscore}{\kern0pt}implies{\isacharunderscore}{\kern0pt}not{\isadigit{0}}\ t{\isacharunderscore}{\kern0pt}not{\isacharunderscore}{\kern0pt}satisfied{\isacharunderscore}{\kern0pt}for{\isacharunderscore}{\kern0pt}q\isanewline
\ \ \ \ \ \ \ \ \isacommand{by}\isamarkupfalse%
\ {\isacharparenleft}{\kern0pt}smt\ {\isacharparenleft}{\kern0pt}verit{\isacharcomma}{\kern0pt}\ ccfv{\isacharunderscore}{\kern0pt}SIG{\isacharparenright}{\kern0pt}{\isacharparenright}{\kern0pt}\isanewline
\ \ \ \ \ \ \isacommand{have}\isamarkupfalse%
\ rec{\isacharunderscore}{\kern0pt}step{\isacharunderscore}{\kern0pt}p{\isacharcolon}{\kern0pt}\isanewline
\ \ \ \ \ \ \ \ {\isachardoublequoteopen}electoral{\isacharunderscore}{\kern0pt}module\ acc\ {\isasymlongrightarrow}\isanewline
\ \ \ \ \ \ \ \ \ \ \ \ loop{\isacharunderscore}{\kern0pt}comp{\isacharunderscore}{\kern0pt}helper\ acc\ m\ t\ A\ p\ {\isacharequal}{\kern0pt}\ loop{\isacharunderscore}{\kern0pt}comp{\isacharunderscore}{\kern0pt}helper\ {\isacharparenleft}{\kern0pt}acc\ {\isasymtriangleright}\ m{\isacharparenright}{\kern0pt}\ m\ t\ A\ p{\isachardoublequoteclose}\isanewline
\ \ \ \ \ \ \ \ \isacommand{using}\isamarkupfalse%
\ card{\isacharunderscore}{\kern0pt}changed\ defer{\isacharunderscore}{\kern0pt}in{\isacharunderscore}{\kern0pt}alts\ loop{\isacharunderscore}{\kern0pt}comp{\isacharunderscore}{\kern0pt}helper{\isachardot}{\kern0pt}simps{\isacharparenleft}{\kern0pt}{\isadigit{2}}{\isacharparenright}{\kern0pt}\isanewline
\ \ \ \ \ \ \ \ \ \ \ \ \ \ monotone{\isacharunderscore}{\kern0pt}m\ prod{\isachardot}{\kern0pt}sel{\isacharparenleft}{\kern0pt}{\isadigit{2}}{\isacharparenright}{\kern0pt}\ f{\isacharunderscore}{\kern0pt}prof\ psubsetI\ def{\isacharunderscore}{\kern0pt}presv{\isacharunderscore}{\kern0pt}fin{\isacharunderscore}{\kern0pt}prof\isanewline
\ \ \ \ \ \ \ \ \ \ \ \ \ \ sequential{\isacharunderscore}{\kern0pt}composition{\isachardot}{\kern0pt}simps\ defer{\isacharunderscore}{\kern0pt}lift{\isacharunderscore}{\kern0pt}invariance{\isacharunderscore}{\kern0pt}def\isanewline
\ \ \ \ \ \ \ \ \ \ \ \ \ \ t{\isacharunderscore}{\kern0pt}not{\isacharunderscore}{\kern0pt}satisfied{\isacharunderscore}{\kern0pt}for{\isacharunderscore}{\kern0pt}p\ seq{\isacharunderscore}{\kern0pt}comp{\isacharunderscore}{\kern0pt}def{\isacharunderscore}{\kern0pt}set{\isacharunderscore}{\kern0pt}bounded\isanewline
\ \ \ \ \ \ \ \ \isacommand{by}\isamarkupfalse%
\ {\isacharparenleft}{\kern0pt}smt\ {\isacharparenleft}{\kern0pt}verit{\isacharcomma}{\kern0pt}\ best{\isacharparenright}{\kern0pt}{\isacharparenright}{\kern0pt}\isanewline
\ \ \ \ \ \ \isacommand{thus}\isamarkupfalse%
\ {\isacharquery}{\kern0pt}thesis\isanewline
\ \ \ \ \ \ \ \ \isacommand{using}\isamarkupfalse%
\ card{\isacharunderscore}{\kern0pt}smaller{\isacharunderscore}{\kern0pt}for{\isacharunderscore}{\kern0pt}p\ less{\isachardot}{\kern0pt}hyps\isanewline
\ \ \ \ \ \ \ \ \ \ \ \ \ \ loop{\isacharunderscore}{\kern0pt}comp{\isacharunderscore}{\kern0pt}helper{\isacharunderscore}{\kern0pt}imp{\isacharunderscore}{\kern0pt}no{\isacharunderscore}{\kern0pt}def{\isacharunderscore}{\kern0pt}incr\ monotone{\isacharunderscore}{\kern0pt}m\isanewline
\ \ \ \ \ \ \ \ \ \ \ \ \ \ seq{\isacharunderscore}{\kern0pt}comp{\isacharunderscore}{\kern0pt}presv{\isacharunderscore}{\kern0pt}def{\isacharunderscore}{\kern0pt}lift{\isacharunderscore}{\kern0pt}inv\ f{\isacharunderscore}{\kern0pt}prof\ rec{\isacharunderscore}{\kern0pt}step{\isacharunderscore}{\kern0pt}q\isanewline
\ \ \ \ \ \ \ \ \ \ \ \ \ \ defer{\isacharunderscore}{\kern0pt}lift{\isacharunderscore}{\kern0pt}invariance{\isacharunderscore}{\kern0pt}def\ subsetCE\ subset{\isacharunderscore}{\kern0pt}eq\isanewline
\ \ \ \ \ \ \ \ \isacommand{by}\isamarkupfalse%
\ {\isacharparenleft}{\kern0pt}smt\ {\isacharparenleft}{\kern0pt}verit{\isacharcomma}{\kern0pt}\ ccfv{\isacharunderscore}{\kern0pt}threshold{\isacharparenright}{\kern0pt}{\isacharparenright}{\kern0pt}\isanewline
\ \ \ \ \isacommand{next}\isamarkupfalse%
\isanewline
\ \ \ \ \ \ \isacommand{assume}\isamarkupfalse%
\ t{\isacharunderscore}{\kern0pt}satisfied{\isacharunderscore}{\kern0pt}for{\isacharunderscore}{\kern0pt}p{\isacharcolon}{\kern0pt}\ {\isachardoublequoteopen}{\isasymnot}\ {\isasymnot}t\ {\isacharparenleft}{\kern0pt}acc\ A\ p{\isacharparenright}{\kern0pt}{\isachardoublequoteclose}\isanewline
\ \ \ \ \ \ \isacommand{thus}\isamarkupfalse%
\ {\isacharquery}{\kern0pt}thesis\isanewline
\ \ \ \ \ \ \ \ \isacommand{using}\isamarkupfalse%
\ loop{\isacharunderscore}{\kern0pt}comp{\isacharunderscore}{\kern0pt}helper{\isachardot}{\kern0pt}simps{\isacharparenleft}{\kern0pt}{\isadigit{1}}{\isacharparenright}{\kern0pt}\ defer{\isacharunderscore}{\kern0pt}lift{\isacharunderscore}{\kern0pt}invariance{\isacharunderscore}{\kern0pt}def\isanewline
\ \ \ \ \ \ \ \ \isacommand{by}\isamarkupfalse%
\ metis\isanewline
\ \ \ \ \isacommand{qed}\isamarkupfalse%
\isanewline
\ \ \isacommand{qed}\isamarkupfalse%
\isanewline
\isacommand{qed}\isamarkupfalse%
%
\endisatagproof
{\isafoldproof}%
%
\isadelimproof
\isanewline
%
\endisadelimproof
\isanewline
\isacommand{lemma}\isamarkupfalse%
\ loop{\isacharunderscore}{\kern0pt}comp{\isacharunderscore}{\kern0pt}helper{\isacharunderscore}{\kern0pt}def{\isacharunderscore}{\kern0pt}lift{\isacharunderscore}{\kern0pt}inv{\isacharcolon}{\kern0pt}\isanewline
\ \ \isakeyword{assumes}\isanewline
\ \ \ \ monotone{\isacharunderscore}{\kern0pt}m{\isacharcolon}{\kern0pt}\ {\isachardoublequoteopen}defer{\isacharunderscore}{\kern0pt}lift{\isacharunderscore}{\kern0pt}invariance\ m{\isachardoublequoteclose}\ \isakeyword{and}\isanewline
\ \ \ \ monotone{\isacharunderscore}{\kern0pt}acc{\isacharcolon}{\kern0pt}\ {\isachardoublequoteopen}defer{\isacharunderscore}{\kern0pt}lift{\isacharunderscore}{\kern0pt}invariance\ acc{\isachardoublequoteclose}\ \isakeyword{and}\isanewline
\ \ \ \ profile{\isacharcolon}{\kern0pt}\ {\isachardoublequoteopen}finite{\isacharunderscore}{\kern0pt}profile\ A\ p{\isachardoublequoteclose}\isanewline
\ \ \isakeyword{shows}\isanewline
\ \ \ \ {\isachardoublequoteopen}{\isasymforall}q\ a{\isachardot}{\kern0pt}\ {\isacharparenleft}{\kern0pt}lifted\ A\ p\ q\ a\ {\isasymand}\ a\ {\isasymin}\ {\isacharparenleft}{\kern0pt}defer\ {\isacharparenleft}{\kern0pt}loop{\isacharunderscore}{\kern0pt}comp{\isacharunderscore}{\kern0pt}helper\ acc\ m\ t{\isacharparenright}{\kern0pt}\ A\ p{\isacharparenright}{\kern0pt}{\isacharparenright}{\kern0pt}\ {\isasymlongrightarrow}\isanewline
\ \ \ \ \ \ \ \ {\isacharparenleft}{\kern0pt}loop{\isacharunderscore}{\kern0pt}comp{\isacharunderscore}{\kern0pt}helper\ acc\ m\ t{\isacharparenright}{\kern0pt}\ A\ p\ {\isacharequal}{\kern0pt}\ {\isacharparenleft}{\kern0pt}loop{\isacharunderscore}{\kern0pt}comp{\isacharunderscore}{\kern0pt}helper\ acc\ m\ t{\isacharparenright}{\kern0pt}\ A\ q{\isachardoublequoteclose}\isanewline
%
\isadelimproof
\ \ %
\endisadelimproof
%
\isatagproof
\isacommand{using}\isamarkupfalse%
\ loop{\isacharunderscore}{\kern0pt}comp{\isacharunderscore}{\kern0pt}helper{\isacharunderscore}{\kern0pt}def{\isacharunderscore}{\kern0pt}lift{\isacharunderscore}{\kern0pt}inv{\isacharunderscore}{\kern0pt}helper\isanewline
\ \ \ \ \ \ \ \ monotone{\isacharunderscore}{\kern0pt}m\ monotone{\isacharunderscore}{\kern0pt}acc\ profile\isanewline
\ \ \isacommand{by}\isamarkupfalse%
\ blast%
\endisatagproof
{\isafoldproof}%
%
\isadelimproof
\isanewline
%
\endisadelimproof
\isanewline
\isacommand{lemma}\isamarkupfalse%
\ loop{\isacharunderscore}{\kern0pt}comp{\isacharunderscore}{\kern0pt}helper{\isacharunderscore}{\kern0pt}def{\isacharunderscore}{\kern0pt}lift{\isacharunderscore}{\kern0pt}inv{\isadigit{2}}{\isacharcolon}{\kern0pt}\isanewline
\ \ \isakeyword{assumes}\isanewline
\ \ \ \ monotone{\isacharunderscore}{\kern0pt}m{\isacharcolon}{\kern0pt}\ {\isachardoublequoteopen}defer{\isacharunderscore}{\kern0pt}lift{\isacharunderscore}{\kern0pt}invariance\ m{\isachardoublequoteclose}\ \isakeyword{and}\isanewline
\ \ \ \ monotone{\isacharunderscore}{\kern0pt}acc{\isacharcolon}{\kern0pt}\ {\isachardoublequoteopen}defer{\isacharunderscore}{\kern0pt}lift{\isacharunderscore}{\kern0pt}invariance\ acc{\isachardoublequoteclose}\isanewline
\ \ \isakeyword{shows}\isanewline
\ \ \ \ {\isachardoublequoteopen}{\isasymforall}A\ p\ q\ a{\isachardot}{\kern0pt}\ {\isacharparenleft}{\kern0pt}finite{\isacharunderscore}{\kern0pt}profile\ A\ p\ {\isasymand}\isanewline
\ \ \ \ \ \ \ \ lifted\ A\ p\ q\ a\ {\isasymand}\isanewline
\ \ \ \ \ \ \ \ a\ {\isasymin}\ {\isacharparenleft}{\kern0pt}defer\ {\isacharparenleft}{\kern0pt}loop{\isacharunderscore}{\kern0pt}comp{\isacharunderscore}{\kern0pt}helper\ acc\ m\ t{\isacharparenright}{\kern0pt}\ A\ p{\isacharparenright}{\kern0pt}{\isacharparenright}{\kern0pt}\ {\isasymlongrightarrow}\isanewline
\ \ \ \ \ \ \ \ \ \ \ \ {\isacharparenleft}{\kern0pt}loop{\isacharunderscore}{\kern0pt}comp{\isacharunderscore}{\kern0pt}helper\ acc\ m\ t{\isacharparenright}{\kern0pt}\ A\ p\ {\isacharequal}{\kern0pt}\ {\isacharparenleft}{\kern0pt}loop{\isacharunderscore}{\kern0pt}comp{\isacharunderscore}{\kern0pt}helper\ acc\ m\ t{\isacharparenright}{\kern0pt}\ A\ q{\isachardoublequoteclose}\isanewline
%
\isadelimproof
\ \ %
\endisadelimproof
%
\isatagproof
\isacommand{using}\isamarkupfalse%
\ loop{\isacharunderscore}{\kern0pt}comp{\isacharunderscore}{\kern0pt}helper{\isacharunderscore}{\kern0pt}def{\isacharunderscore}{\kern0pt}lift{\isacharunderscore}{\kern0pt}inv\ monotone{\isacharunderscore}{\kern0pt}acc\ monotone{\isacharunderscore}{\kern0pt}m\isanewline
\ \ \isacommand{by}\isamarkupfalse%
\ blast%
\endisatagproof
{\isafoldproof}%
%
\isadelimproof
\isanewline
%
\endisadelimproof
\isanewline
\isacommand{lemma}\isamarkupfalse%
\ lifted{\isacharunderscore}{\kern0pt}imp{\isacharunderscore}{\kern0pt}fin{\isacharunderscore}{\kern0pt}prof{\isacharcolon}{\kern0pt}\isanewline
\ \ \isakeyword{assumes}\ {\isachardoublequoteopen}lifted\ A\ p\ q\ a{\isachardoublequoteclose}\isanewline
\ \ \isakeyword{shows}\ {\isachardoublequoteopen}finite{\isacharunderscore}{\kern0pt}profile\ A\ p{\isachardoublequoteclose}\isanewline
%
\isadelimproof
\ \ %
\endisadelimproof
%
\isatagproof
\isacommand{using}\isamarkupfalse%
\ assms\ Profile{\isachardot}{\kern0pt}lifted{\isacharunderscore}{\kern0pt}def\isanewline
\ \ \isacommand{by}\isamarkupfalse%
\ fastforce%
\endisatagproof
{\isafoldproof}%
%
\isadelimproof
\isanewline
%
\endisadelimproof
\isanewline
\isacommand{lemma}\isamarkupfalse%
\ loop{\isacharunderscore}{\kern0pt}comp{\isacharunderscore}{\kern0pt}helper{\isacharunderscore}{\kern0pt}presv{\isacharunderscore}{\kern0pt}def{\isacharunderscore}{\kern0pt}lift{\isacharunderscore}{\kern0pt}inv{\isacharcolon}{\kern0pt}\isanewline
\ \ \isakeyword{assumes}\isanewline
\ \ \ \ monotone{\isacharunderscore}{\kern0pt}m{\isacharcolon}{\kern0pt}\ {\isachardoublequoteopen}defer{\isacharunderscore}{\kern0pt}lift{\isacharunderscore}{\kern0pt}invariance\ m{\isachardoublequoteclose}\ \isakeyword{and}\isanewline
\ \ \ \ monotone{\isacharunderscore}{\kern0pt}acc{\isacharcolon}{\kern0pt}\ {\isachardoublequoteopen}defer{\isacharunderscore}{\kern0pt}lift{\isacharunderscore}{\kern0pt}invariance\ acc{\isachardoublequoteclose}\isanewline
\ \ \isakeyword{shows}\ {\isachardoublequoteopen}defer{\isacharunderscore}{\kern0pt}lift{\isacharunderscore}{\kern0pt}invariance\ {\isacharparenleft}{\kern0pt}loop{\isacharunderscore}{\kern0pt}comp{\isacharunderscore}{\kern0pt}helper\ acc\ m\ t{\isacharparenright}{\kern0pt}{\isachardoublequoteclose}\isanewline
%
\isadelimproof
%
\endisadelimproof
%
\isatagproof
\isacommand{proof}\isamarkupfalse%
\ {\isacharminus}{\kern0pt}\isanewline
\ \ \isacommand{have}\isamarkupfalse%
\isanewline
\ \ \ \ {\isachardoublequoteopen}{\isasymforall}f{\isachardot}{\kern0pt}\ {\isacharparenleft}{\kern0pt}defer{\isacharunderscore}{\kern0pt}lift{\isacharunderscore}{\kern0pt}invariance\ f\ {\isasymor}\isanewline
\ \ \ \ \ \ \ \ \ {\isacharparenleft}{\kern0pt}{\isasymexists}A\ rs\ rsa\ a{\isachardot}{\kern0pt}\ f\ A\ rs\ {\isasymnoteq}\ f\ A\ rsa\ {\isasymand}\isanewline
\ \ \ \ \ \ \ \ \ \ \ \ \ \ Profile{\isachardot}{\kern0pt}lifted\ A\ rs\ rsa\ {\isacharparenleft}{\kern0pt}a{\isacharcolon}{\kern0pt}{\isacharcolon}{\kern0pt}{\isacharprime}{\kern0pt}a{\isacharparenright}{\kern0pt}\ {\isasymand}\isanewline
\ \ \ \ \ \ \ \ \ \ \ \ \ \ a\ {\isasymin}\ defer\ f\ A\ rs{\isacharparenright}{\kern0pt}\ {\isasymor}\isanewline
\ \ \ \ \ \ \ \ \ {\isasymnot}\ electoral{\isacharunderscore}{\kern0pt}module\ f{\isacharparenright}{\kern0pt}\ {\isasymand}\isanewline
\ \ \ \ \ \ {\isacharparenleft}{\kern0pt}{\isacharparenleft}{\kern0pt}{\isasymforall}A\ rs\ rsa\ a{\isachardot}{\kern0pt}\ f\ A\ rs\ {\isacharequal}{\kern0pt}\ f\ A\ rsa\ {\isasymor}\ {\isasymnot}\ Profile{\isachardot}{\kern0pt}lifted\ A\ rs\ rsa\ a\ {\isasymor}\isanewline
\ \ \ \ \ \ \ \ \ \ a\ {\isasymnotin}\ defer\ f\ A\ rs{\isacharparenright}{\kern0pt}\ {\isasymand}\isanewline
\ \ \ \ \ \ electoral{\isacharunderscore}{\kern0pt}module\ f\ {\isasymor}\ {\isasymnot}\ defer{\isacharunderscore}{\kern0pt}lift{\isacharunderscore}{\kern0pt}invariance\ f{\isacharparenright}{\kern0pt}{\isachardoublequoteclose}\isanewline
\ \ \ \ \isacommand{using}\isamarkupfalse%
\ defer{\isacharunderscore}{\kern0pt}lift{\isacharunderscore}{\kern0pt}invariance{\isacharunderscore}{\kern0pt}def\isanewline
\ \ \ \ \isacommand{by}\isamarkupfalse%
\ blast\isanewline
\ \ \isacommand{thus}\isamarkupfalse%
\ {\isacharquery}{\kern0pt}thesis\isanewline
\ \ \ \ \isacommand{using}\isamarkupfalse%
\ electoral{\isacharunderscore}{\kern0pt}module{\isacharunderscore}{\kern0pt}def\ lifted{\isacharunderscore}{\kern0pt}imp{\isacharunderscore}{\kern0pt}fin{\isacharunderscore}{\kern0pt}prof\isanewline
\ \ \ \ \ \ \ \ \ \ loop{\isacharunderscore}{\kern0pt}comp{\isacharunderscore}{\kern0pt}helper{\isacharunderscore}{\kern0pt}def{\isacharunderscore}{\kern0pt}lift{\isacharunderscore}{\kern0pt}inv\ loop{\isacharunderscore}{\kern0pt}comp{\isacharunderscore}{\kern0pt}helper{\isacharunderscore}{\kern0pt}imp{\isacharunderscore}{\kern0pt}partit\isanewline
\ \ \ \ \ \ \ \ \ \ monotone{\isacharunderscore}{\kern0pt}acc\ monotone{\isacharunderscore}{\kern0pt}m\isanewline
\ \ \ \ \isacommand{by}\isamarkupfalse%
\ {\isacharparenleft}{\kern0pt}metis\ {\isacharparenleft}{\kern0pt}full{\isacharunderscore}{\kern0pt}types{\isacharparenright}{\kern0pt}{\isacharparenright}{\kern0pt}\isanewline
\isacommand{qed}\isamarkupfalse%
%
\endisatagproof
{\isafoldproof}%
%
\isadelimproof
\isanewline
%
\endisadelimproof
\isanewline
\isanewline
\isacommand{theorem}\isamarkupfalse%
\ loop{\isacharunderscore}{\kern0pt}comp{\isacharunderscore}{\kern0pt}presv{\isacharunderscore}{\kern0pt}def{\isacharunderscore}{\kern0pt}lift{\isacharunderscore}{\kern0pt}inv{\isacharbrackleft}{\kern0pt}simp{\isacharbrackright}{\kern0pt}{\isacharcolon}{\kern0pt}\isanewline
\ \ \isakeyword{assumes}\ monotone{\isacharunderscore}{\kern0pt}m{\isacharcolon}{\kern0pt}\ {\isachardoublequoteopen}defer{\isacharunderscore}{\kern0pt}lift{\isacharunderscore}{\kern0pt}invariance\ m{\isachardoublequoteclose}\isanewline
\ \ \isakeyword{shows}\ {\isachardoublequoteopen}defer{\isacharunderscore}{\kern0pt}lift{\isacharunderscore}{\kern0pt}invariance\ {\isacharparenleft}{\kern0pt}m\ {\isasymcirclearrowleft}\isactrlsub t{\isacharparenright}{\kern0pt}{\isachardoublequoteclose}\isanewline
%
\isadelimproof
%
\endisadelimproof
%
\isatagproof
\isacommand{proof}\isamarkupfalse%
\ {\isacharminus}{\kern0pt}\isanewline
\ \ \isacommand{fix}\isamarkupfalse%
\isanewline
\ \ \ \ A\ {\isacharcolon}{\kern0pt}{\isacharcolon}{\kern0pt}\ {\isachardoublequoteopen}{\isacharprime}{\kern0pt}a\ set{\isachardoublequoteclose}\isanewline
\ \ \isacommand{have}\isamarkupfalse%
\isanewline
\ \ \ \ {\isachardoublequoteopen}{\isasymforall}\ p\ q\ a{\isachardot}{\kern0pt}\ {\isacharparenleft}{\kern0pt}a\ {\isasymin}\ {\isacharparenleft}{\kern0pt}defer\ {\isacharparenleft}{\kern0pt}m\ {\isasymcirclearrowleft}\isactrlsub t{\isacharparenright}{\kern0pt}\ A\ p{\isacharparenright}{\kern0pt}\ {\isasymand}\ lifted\ A\ p\ q\ a{\isacharparenright}{\kern0pt}\ {\isasymlongrightarrow}\isanewline
\ \ \ \ \ \ \ \ {\isacharparenleft}{\kern0pt}m\ {\isasymcirclearrowleft}\isactrlsub t{\isacharparenright}{\kern0pt}\ A\ p\ {\isacharequal}{\kern0pt}\ {\isacharparenleft}{\kern0pt}m\ {\isasymcirclearrowleft}\isactrlsub t{\isacharparenright}{\kern0pt}\ A\ q{\isachardoublequoteclose}\isanewline
\ \ \ \ \isacommand{using}\isamarkupfalse%
\ defer{\isacharunderscore}{\kern0pt}module{\isachardot}{\kern0pt}simps\ monotone{\isacharunderscore}{\kern0pt}m\ lifted{\isacharunderscore}{\kern0pt}imp{\isacharunderscore}{\kern0pt}fin{\isacharunderscore}{\kern0pt}prof\isanewline
\ \ \ \ \ \ \ \ \ \ loop{\isacharunderscore}{\kern0pt}composition{\isachardot}{\kern0pt}simps{\isacharparenleft}{\kern0pt}{\isadigit{1}}{\isacharparenright}{\kern0pt}\ loop{\isacharunderscore}{\kern0pt}composition{\isachardot}{\kern0pt}simps{\isacharparenleft}{\kern0pt}{\isadigit{2}}{\isacharparenright}{\kern0pt}\isanewline
\ \ \ \ \ \ \ \ \ \ loop{\isacharunderscore}{\kern0pt}comp{\isacharunderscore}{\kern0pt}helper{\isacharunderscore}{\kern0pt}def{\isacharunderscore}{\kern0pt}lift{\isacharunderscore}{\kern0pt}inv{\isadigit{2}}\isanewline
\ \ \ \ \isacommand{by}\isamarkupfalse%
\ {\isacharparenleft}{\kern0pt}metis\ {\isacharparenleft}{\kern0pt}full{\isacharunderscore}{\kern0pt}types{\isacharparenright}{\kern0pt}{\isacharparenright}{\kern0pt}\isanewline
\ \ \isacommand{thus}\isamarkupfalse%
\ {\isacharquery}{\kern0pt}thesis\isanewline
\ \ \ \ \isacommand{using}\isamarkupfalse%
\ def{\isacharunderscore}{\kern0pt}mod{\isacharunderscore}{\kern0pt}def{\isacharunderscore}{\kern0pt}lift{\isacharunderscore}{\kern0pt}inv\ monotone{\isacharunderscore}{\kern0pt}m\ loop{\isacharunderscore}{\kern0pt}composition{\isachardot}{\kern0pt}simps{\isacharparenleft}{\kern0pt}{\isadigit{1}}{\isacharparenright}{\kern0pt}\isanewline
\ \ \ \ \ \ \ \ \ \ loop{\isacharunderscore}{\kern0pt}composition{\isachardot}{\kern0pt}simps{\isacharparenleft}{\kern0pt}{\isadigit{2}}{\isacharparenright}{\kern0pt}\ defer{\isacharunderscore}{\kern0pt}lift{\isacharunderscore}{\kern0pt}invariance{\isacharunderscore}{\kern0pt}def\isanewline
\ \ \ \ \ \ \ \ \ \ loop{\isacharunderscore}{\kern0pt}comp{\isacharunderscore}{\kern0pt}sound\ loop{\isacharunderscore}{\kern0pt}comp{\isacharunderscore}{\kern0pt}helper{\isacharunderscore}{\kern0pt}def{\isacharunderscore}{\kern0pt}lift{\isacharunderscore}{\kern0pt}inv{\isadigit{2}}\isanewline
\ \ \ \ \ \ \ \ \ \ lifted{\isacharunderscore}{\kern0pt}imp{\isacharunderscore}{\kern0pt}fin{\isacharunderscore}{\kern0pt}prof\isanewline
\ \ \ \ \isacommand{by}\isamarkupfalse%
\ {\isacharparenleft}{\kern0pt}smt\ {\isacharparenleft}{\kern0pt}verit{\isacharcomma}{\kern0pt}\ best{\isacharparenright}{\kern0pt}{\isacharparenright}{\kern0pt}\isanewline
\isacommand{qed}\isamarkupfalse%
%
\endisatagproof
{\isafoldproof}%
%
\isadelimproof
\isanewline
%
\endisadelimproof
\isanewline
\isanewline
\isacommand{theorem}\isamarkupfalse%
\ rev{\isacharunderscore}{\kern0pt}comp{\isacharunderscore}{\kern0pt}def{\isacharunderscore}{\kern0pt}inv{\isacharunderscore}{\kern0pt}mono{\isacharbrackleft}{\kern0pt}simp{\isacharbrackright}{\kern0pt}{\isacharcolon}{\kern0pt}\isanewline
\ \ \isakeyword{assumes}\ {\isachardoublequoteopen}invariant{\isacharunderscore}{\kern0pt}monotonicity\ m{\isachardoublequoteclose}\isanewline
\ \ \isakeyword{shows}\ {\isachardoublequoteopen}defer{\isacharunderscore}{\kern0pt}invariant{\isacharunderscore}{\kern0pt}monotonicity\ {\isacharparenleft}{\kern0pt}m{\isasymdown}{\isacharparenright}{\kern0pt}{\isachardoublequoteclose}\isanewline
%
\isadelimproof
%
\endisadelimproof
%
\isatagproof
\isacommand{proof}\isamarkupfalse%
\ {\isacharminus}{\kern0pt}\isanewline
\ \ \isacommand{have}\isamarkupfalse%
\ {\isachardoublequoteopen}{\isasymforall}A\ p\ q\ w{\isachardot}{\kern0pt}\ {\isacharparenleft}{\kern0pt}w\ {\isasymin}\ defer\ {\isacharparenleft}{\kern0pt}m{\isasymdown}{\isacharparenright}{\kern0pt}\ A\ p\ {\isasymand}\ lifted\ A\ p\ q\ w{\isacharparenright}{\kern0pt}\ {\isasymlongrightarrow}\isanewline
\ \ \ \ \ \ \ \ \ \ \ \ \ \ \ \ \ \ {\isacharparenleft}{\kern0pt}defer\ {\isacharparenleft}{\kern0pt}m{\isasymdown}{\isacharparenright}{\kern0pt}\ A\ q\ {\isacharequal}{\kern0pt}\ defer\ {\isacharparenleft}{\kern0pt}m{\isasymdown}{\isacharparenright}{\kern0pt}\ A\ p\ {\isasymor}\ defer\ {\isacharparenleft}{\kern0pt}m{\isasymdown}{\isacharparenright}{\kern0pt}\ A\ q\ {\isacharequal}{\kern0pt}\ {\isacharbraceleft}{\kern0pt}w{\isacharbraceright}{\kern0pt}{\isacharparenright}{\kern0pt}{\isachardoublequoteclose}\isanewline
\ \ \ \ \isacommand{using}\isamarkupfalse%
\ assms\isanewline
\ \ \ \ \isacommand{by}\isamarkupfalse%
\ {\isacharparenleft}{\kern0pt}simp\ add{\isacharcolon}{\kern0pt}\ invariant{\isacharunderscore}{\kern0pt}monotonicity{\isacharunderscore}{\kern0pt}def{\isacharparenright}{\kern0pt}\isanewline
\ \ \isacommand{moreover}\isamarkupfalse%
\ \isacommand{have}\isamarkupfalse%
\ {\isachardoublequoteopen}electoral{\isacharunderscore}{\kern0pt}module\ {\isacharparenleft}{\kern0pt}m{\isasymdown}{\isacharparenright}{\kern0pt}{\isachardoublequoteclose}\isanewline
\ \ \ \ \isacommand{using}\isamarkupfalse%
\ assms\ rev{\isacharunderscore}{\kern0pt}comp{\isacharunderscore}{\kern0pt}sound\ invariant{\isacharunderscore}{\kern0pt}monotonicity{\isacharunderscore}{\kern0pt}def\isanewline
\ \ \ \ \isacommand{by}\isamarkupfalse%
\ auto\isanewline
\ \ \isacommand{moreover}\isamarkupfalse%
\ \isacommand{have}\isamarkupfalse%
\ {\isachardoublequoteopen}non{\isacharunderscore}{\kern0pt}electing\ {\isacharparenleft}{\kern0pt}m{\isasymdown}{\isacharparenright}{\kern0pt}{\isachardoublequoteclose}\isanewline
\ \ \ \ \isacommand{using}\isamarkupfalse%
\ assms\ rev{\isacharunderscore}{\kern0pt}comp{\isacharunderscore}{\kern0pt}non{\isacharunderscore}{\kern0pt}electing\ invariant{\isacharunderscore}{\kern0pt}monotonicity{\isacharunderscore}{\kern0pt}def\isanewline
\ \ \ \ \isacommand{by}\isamarkupfalse%
\ auto\isanewline
\ \ \isacommand{ultimately}\isamarkupfalse%
\ \isacommand{have}\isamarkupfalse%
\ {\isachardoublequoteopen}electoral{\isacharunderscore}{\kern0pt}module\ {\isacharparenleft}{\kern0pt}m{\isasymdown}{\isacharparenright}{\kern0pt}\ {\isasymand}\ non{\isacharunderscore}{\kern0pt}electing\ {\isacharparenleft}{\kern0pt}m{\isasymdown}{\isacharparenright}{\kern0pt}\ {\isasymand}\isanewline
\ \ \ \ \ \ {\isacharparenleft}{\kern0pt}{\isasymforall}A\ p\ q\ w{\isachardot}{\kern0pt}\ {\isacharparenleft}{\kern0pt}w\ {\isasymin}\ defer\ {\isacharparenleft}{\kern0pt}m{\isasymdown}{\isacharparenright}{\kern0pt}\ A\ p\ {\isasymand}\ lifted\ A\ p\ q\ w{\isacharparenright}{\kern0pt}\ {\isasymlongrightarrow}\isanewline
\ \ \ \ \ \ \ \ \ \ \ \ \ \ \ \ \ {\isacharparenleft}{\kern0pt}defer\ {\isacharparenleft}{\kern0pt}m{\isasymdown}{\isacharparenright}{\kern0pt}\ A\ q\ {\isacharequal}{\kern0pt}\ defer\ {\isacharparenleft}{\kern0pt}m{\isasymdown}{\isacharparenright}{\kern0pt}\ A\ p\ {\isasymor}\ defer\ {\isacharparenleft}{\kern0pt}m{\isasymdown}{\isacharparenright}{\kern0pt}\ A\ q\ {\isacharequal}{\kern0pt}\ {\isacharbraceleft}{\kern0pt}w{\isacharbraceright}{\kern0pt}{\isacharparenright}{\kern0pt}{\isacharparenright}{\kern0pt}{\isachardoublequoteclose}\isanewline
\ \ \ \ \isacommand{by}\isamarkupfalse%
\ blast\isanewline
\ \ \isacommand{thus}\isamarkupfalse%
\ {\isacharquery}{\kern0pt}thesis\isanewline
\ \ \ \ \isacommand{using}\isamarkupfalse%
\ defer{\isacharunderscore}{\kern0pt}invariant{\isacharunderscore}{\kern0pt}monotonicity{\isacharunderscore}{\kern0pt}def\isanewline
\ \ \ \ \isacommand{by}\isamarkupfalse%
\ {\isacharparenleft}{\kern0pt}simp\ add{\isacharcolon}{\kern0pt}\ defer{\isacharunderscore}{\kern0pt}invariant{\isacharunderscore}{\kern0pt}monotonicity{\isacharunderscore}{\kern0pt}def{\isacharparenright}{\kern0pt}\isanewline
\isacommand{qed}\isamarkupfalse%
%
\endisatagproof
{\isafoldproof}%
%
\isadelimproof
\isanewline
%
\endisadelimproof
\isanewline
\isanewline
\isacommand{theorem}\isamarkupfalse%
\ dl{\isacharunderscore}{\kern0pt}inv{\isacharunderscore}{\kern0pt}imp{\isacharunderscore}{\kern0pt}def{\isacharunderscore}{\kern0pt}mono{\isacharbrackleft}{\kern0pt}simp{\isacharbrackright}{\kern0pt}{\isacharcolon}{\kern0pt}\isanewline
\ \ \isakeyword{assumes}\ {\isachardoublequoteopen}defer{\isacharunderscore}{\kern0pt}lift{\isacharunderscore}{\kern0pt}invariance\ m{\isachardoublequoteclose}\isanewline
\ \ \isakeyword{shows}\ {\isachardoublequoteopen}defer{\isacharunderscore}{\kern0pt}monotonicity\ m{\isachardoublequoteclose}\isanewline
%
\isadelimproof
\ \ %
\endisadelimproof
%
\isatagproof
\isacommand{using}\isamarkupfalse%
\ assms\ defer{\isacharunderscore}{\kern0pt}monotonicity{\isacharunderscore}{\kern0pt}def\ defer{\isacharunderscore}{\kern0pt}lift{\isacharunderscore}{\kern0pt}invariance{\isacharunderscore}{\kern0pt}def\isanewline
\ \ \isacommand{by}\isamarkupfalse%
\ fastforce%
\endisatagproof
{\isafoldproof}%
%
\isadelimproof
\isanewline
%
\endisadelimproof
\isanewline
\isanewline
\isacommand{theorem}\isamarkupfalse%
\ seq{\isacharunderscore}{\kern0pt}comp{\isacharunderscore}{\kern0pt}mono{\isacharbrackleft}{\kern0pt}simp{\isacharbrackright}{\kern0pt}{\isacharcolon}{\kern0pt}\isanewline
\ \ \isakeyword{assumes}\isanewline
\ \ \ \ def{\isacharunderscore}{\kern0pt}monotone{\isacharunderscore}{\kern0pt}m{\isacharcolon}{\kern0pt}\ {\isachardoublequoteopen}defer{\isacharunderscore}{\kern0pt}lift{\isacharunderscore}{\kern0pt}invariance\ m{\isachardoublequoteclose}\ \isakeyword{and}\isanewline
\ \ \ \ non{\isacharunderscore}{\kern0pt}ele{\isacharunderscore}{\kern0pt}m{\isacharcolon}{\kern0pt}\ {\isachardoublequoteopen}non{\isacharunderscore}{\kern0pt}electing\ m{\isachardoublequoteclose}\ \isakeyword{and}\isanewline
\ \ \ \ def{\isacharunderscore}{\kern0pt}one{\isacharunderscore}{\kern0pt}m{\isacharcolon}{\kern0pt}\ {\isachardoublequoteopen}defers\ {\isadigit{1}}\ m{\isachardoublequoteclose}\ \isakeyword{and}\isanewline
\ \ \ \ electing{\isacharunderscore}{\kern0pt}n{\isacharcolon}{\kern0pt}\ {\isachardoublequoteopen}electing\ n{\isachardoublequoteclose}\isanewline
\ \ \isakeyword{shows}\ {\isachardoublequoteopen}monotonicity\ {\isacharparenleft}{\kern0pt}m\ {\isasymtriangleright}\ n{\isacharparenright}{\kern0pt}{\isachardoublequoteclose}\isanewline
%
\isadelimproof
\ \ %
\endisadelimproof
%
\isatagproof
\isacommand{unfolding}\isamarkupfalse%
\ monotonicity{\isacharunderscore}{\kern0pt}def\isanewline
\isacommand{proof}\isamarkupfalse%
\ {\isacharparenleft}{\kern0pt}safe{\isacharparenright}{\kern0pt}\isanewline
\ \ \isacommand{have}\isamarkupfalse%
\ electoral{\isacharunderscore}{\kern0pt}mod{\isacharunderscore}{\kern0pt}m{\isacharcolon}{\kern0pt}\ {\isachardoublequoteopen}electoral{\isacharunderscore}{\kern0pt}module\ m{\isachardoublequoteclose}\isanewline
\ \ \ \ \isacommand{using}\isamarkupfalse%
\ non{\isacharunderscore}{\kern0pt}ele{\isacharunderscore}{\kern0pt}m\isanewline
\ \ \ \ \isacommand{by}\isamarkupfalse%
\ {\isacharparenleft}{\kern0pt}simp\ add{\isacharcolon}{\kern0pt}\ non{\isacharunderscore}{\kern0pt}electing{\isacharunderscore}{\kern0pt}def{\isacharparenright}{\kern0pt}\isanewline
\ \ \isacommand{have}\isamarkupfalse%
\ electoral{\isacharunderscore}{\kern0pt}mod{\isacharunderscore}{\kern0pt}n{\isacharcolon}{\kern0pt}\ {\isachardoublequoteopen}electoral{\isacharunderscore}{\kern0pt}module\ n{\isachardoublequoteclose}\isanewline
\ \ \ \ \isacommand{using}\isamarkupfalse%
\ electing{\isacharunderscore}{\kern0pt}n\isanewline
\ \ \ \ \isacommand{by}\isamarkupfalse%
\ {\isacharparenleft}{\kern0pt}simp\ add{\isacharcolon}{\kern0pt}\ electing{\isacharunderscore}{\kern0pt}def{\isacharparenright}{\kern0pt}\isanewline
\ \ \isacommand{show}\isamarkupfalse%
\ {\isachardoublequoteopen}electoral{\isacharunderscore}{\kern0pt}module\ {\isacharparenleft}{\kern0pt}m\ {\isasymtriangleright}\ n{\isacharparenright}{\kern0pt}{\isachardoublequoteclose}\isanewline
\ \ \ \ \isacommand{using}\isamarkupfalse%
\ electoral{\isacharunderscore}{\kern0pt}mod{\isacharunderscore}{\kern0pt}m\ electoral{\isacharunderscore}{\kern0pt}mod{\isacharunderscore}{\kern0pt}n\isanewline
\ \ \ \ \isacommand{by}\isamarkupfalse%
\ simp\isanewline
\isacommand{next}\isamarkupfalse%
\isanewline
\ \ \isacommand{fix}\isamarkupfalse%
\isanewline
\ \ \ \ A\ {\isacharcolon}{\kern0pt}{\isacharcolon}{\kern0pt}\ {\isachardoublequoteopen}{\isacharprime}{\kern0pt}a\ set{\isachardoublequoteclose}\ \isakeyword{and}\isanewline
\ \ \ \ p\ {\isacharcolon}{\kern0pt}{\isacharcolon}{\kern0pt}\ {\isachardoublequoteopen}{\isacharprime}{\kern0pt}a\ Profile{\isachardoublequoteclose}\ \isakeyword{and}\isanewline
\ \ \ \ q\ {\isacharcolon}{\kern0pt}{\isacharcolon}{\kern0pt}\ {\isachardoublequoteopen}{\isacharprime}{\kern0pt}a\ Profile{\isachardoublequoteclose}\ \isakeyword{and}\isanewline
\ \ \ \ w\ {\isacharcolon}{\kern0pt}{\isacharcolon}{\kern0pt}\ {\isachardoublequoteopen}{\isacharprime}{\kern0pt}a{\isachardoublequoteclose}\isanewline
\ \ \isacommand{assume}\isamarkupfalse%
\isanewline
\ \ \ \ fin{\isacharunderscore}{\kern0pt}A{\isacharcolon}{\kern0pt}\ {\isachardoublequoteopen}finite\ A{\isachardoublequoteclose}\ \isakeyword{and}\isanewline
\ \ \ \ elect{\isacharunderscore}{\kern0pt}w{\isacharunderscore}{\kern0pt}in{\isacharunderscore}{\kern0pt}p{\isacharcolon}{\kern0pt}\ {\isachardoublequoteopen}w\ {\isasymin}\ elect\ {\isacharparenleft}{\kern0pt}m\ {\isasymtriangleright}\ n{\isacharparenright}{\kern0pt}\ A\ p{\isachardoublequoteclose}\ \isakeyword{and}\isanewline
\ \ \ \ lifted{\isacharunderscore}{\kern0pt}w{\isacharcolon}{\kern0pt}\ {\isachardoublequoteopen}Profile{\isachardot}{\kern0pt}lifted\ A\ p\ q\ w{\isachardoublequoteclose}\isanewline
\ \ \isacommand{have}\isamarkupfalse%
\isanewline
\ \ \ \ {\isachardoublequoteopen}finite{\isacharunderscore}{\kern0pt}profile\ A\ p\ {\isasymand}\ finite{\isacharunderscore}{\kern0pt}profile\ A\ q{\isachardoublequoteclose}\isanewline
\ \ \ \ \isacommand{using}\isamarkupfalse%
\ lifted{\isacharunderscore}{\kern0pt}w\ lifted{\isacharunderscore}{\kern0pt}def\isanewline
\ \ \ \ \isacommand{by}\isamarkupfalse%
\ metis\isanewline
\ \ \isacommand{thus}\isamarkupfalse%
\ {\isachardoublequoteopen}w\ {\isasymin}\ elect\ {\isacharparenleft}{\kern0pt}m\ {\isasymtriangleright}\ n{\isacharparenright}{\kern0pt}\ A\ q{\isachardoublequoteclose}\isanewline
\ \ \ \ \isacommand{using}\isamarkupfalse%
\ seq{\isacharunderscore}{\kern0pt}comp{\isacharunderscore}{\kern0pt}def{\isacharunderscore}{\kern0pt}then{\isacharunderscore}{\kern0pt}elect\ defer{\isacharunderscore}{\kern0pt}lift{\isacharunderscore}{\kern0pt}invariance{\isacharunderscore}{\kern0pt}def\isanewline
\ \ \ \ \ \ \ \ \ \ elect{\isacharunderscore}{\kern0pt}w{\isacharunderscore}{\kern0pt}in{\isacharunderscore}{\kern0pt}p\ lifted{\isacharunderscore}{\kern0pt}w\ def{\isacharunderscore}{\kern0pt}monotone{\isacharunderscore}{\kern0pt}m\ non{\isacharunderscore}{\kern0pt}ele{\isacharunderscore}{\kern0pt}m\isanewline
\ \ \ \ \ \ \ \ \ \ def{\isacharunderscore}{\kern0pt}one{\isacharunderscore}{\kern0pt}m\ electing{\isacharunderscore}{\kern0pt}n\isanewline
\ \ \ \ \isacommand{by}\isamarkupfalse%
\ metis\isanewline
\isacommand{qed}\isamarkupfalse%
%
\endisatagproof
{\isafoldproof}%
%
\isadelimproof
\isanewline
%
\endisadelimproof
%
\isadelimtheory
\isanewline
%
\endisadelimtheory
%
\isatagtheory
\isacommand{end}\isamarkupfalse%
%
\endisatagtheory
{\isafoldtheory}%
%
\isadelimtheory
%
\endisadelimtheory
%
\end{isabellebody}%
\endinput
%:%file=~/Documents/Studies/VotingRuleGenerator/virage/src/test/resources/verifiedVotingRuleConstruction/theories/Compositional_Framework/Composition_Rules/Monotonicity_Rules.thy%:%
%:%10=1%:%
%:%11=1%:%
%:%12=2%:%
%:%13=3%:%
%:%14=4%:%
%:%15=5%:%
%:%16=6%:%
%:%17=7%:%
%:%18=8%:%
%:%19=9%:%
%:%20=10%:%
%:%21=11%:%
%:%26=11%:%
%:%29=12%:%
%:%30=17%:%
%:%31=18%:%
%:%32=18%:%
%:%33=19%:%
%:%34=20%:%
%:%35=21%:%
%:%36=22%:%
%:%37=23%:%
%:%38=24%:%
%:%41=25%:%
%:%45=25%:%
%:%46=25%:%
%:%47=26%:%
%:%48=26%:%
%:%49=27%:%
%:%50=27%:%
%:%51=28%:%
%:%52=28%:%
%:%53=29%:%
%:%54=30%:%
%:%55=30%:%
%:%56=31%:%
%:%57=31%:%
%:%58=32%:%
%:%59=32%:%
%:%60=33%:%
%:%61=33%:%
%:%62=34%:%
%:%63=34%:%
%:%64=35%:%
%:%65=35%:%
%:%66=36%:%
%:%67=36%:%
%:%68=37%:%
%:%69=37%:%
%:%70=38%:%
%:%71=38%:%
%:%72=39%:%
%:%73=40%:%
%:%74=41%:%
%:%75=42%:%
%:%76=43%:%
%:%77=43%:%
%:%78=44%:%
%:%79=45%:%
%:%80=46%:%
%:%81=46%:%
%:%82=47%:%
%:%83=47%:%
%:%84=48%:%
%:%85=48%:%
%:%86=49%:%
%:%87=49%:%
%:%88=50%:%
%:%89=50%:%
%:%90=51%:%
%:%91=51%:%
%:%92=52%:%
%:%93=52%:%
%:%94=53%:%
%:%95=53%:%
%:%96=54%:%
%:%97=54%:%
%:%98=55%:%
%:%99=55%:%
%:%100=56%:%
%:%101=56%:%
%:%102=57%:%
%:%103=57%:%
%:%104=58%:%
%:%105=58%:%
%:%106=59%:%
%:%107=59%:%
%:%108=60%:%
%:%109=60%:%
%:%110=61%:%
%:%111=61%:%
%:%112=62%:%
%:%113=62%:%
%:%114=63%:%
%:%115=63%:%
%:%116=64%:%
%:%117=64%:%
%:%118=65%:%
%:%119=65%:%
%:%120=66%:%
%:%121=67%:%
%:%122=68%:%
%:%123=69%:%
%:%124=69%:%
%:%125=70%:%
%:%126=70%:%
%:%127=70%:%
%:%128=71%:%
%:%129=72%:%
%:%130=72%:%
%:%131=73%:%
%:%132=74%:%
%:%133=74%:%
%:%134=75%:%
%:%135=75%:%
%:%136=75%:%
%:%137=76%:%
%:%138=77%:%
%:%139=77%:%
%:%140=78%:%
%:%141=78%:%
%:%142=78%:%
%:%143=79%:%
%:%144=80%:%
%:%145=80%:%
%:%146=81%:%
%:%147=82%:%
%:%148=83%:%
%:%149=84%:%
%:%150=85%:%
%:%151=85%:%
%:%152=86%:%
%:%153=86%:%
%:%154=86%:%
%:%155=87%:%
%:%156=88%:%
%:%157=88%:%
%:%158=89%:%
%:%159=90%:%
%:%160=91%:%
%:%161=91%:%
%:%162=92%:%
%:%163=92%:%
%:%164=92%:%
%:%165=93%:%
%:%166=94%:%
%:%167=94%:%
%:%168=95%:%
%:%169=96%:%
%:%170=96%:%
%:%171=97%:%
%:%172=97%:%
%:%173=97%:%
%:%174=98%:%
%:%175=99%:%
%:%176=99%:%
%:%177=100%:%
%:%178=101%:%
%:%179=102%:%
%:%180=103%:%
%:%181=104%:%
%:%182=105%:%
%:%183=106%:%
%:%184=106%:%
%:%185=107%:%
%:%186=107%:%
%:%187=107%:%
%:%188=108%:%
%:%189=109%:%
%:%190=109%:%
%:%191=110%:%
%:%192=110%:%
%:%193=110%:%
%:%194=111%:%
%:%195=112%:%
%:%196=112%:%
%:%197=113%:%
%:%198=114%:%
%:%199=115%:%
%:%200=116%:%
%:%201=116%:%
%:%202=117%:%
%:%203=117%:%
%:%204=118%:%
%:%205=118%:%
%:%206=119%:%
%:%207=120%:%
%:%208=121%:%
%:%209=122%:%
%:%210=122%:%
%:%211=123%:%
%:%212=123%:%
%:%213=124%:%
%:%214=124%:%
%:%215=125%:%
%:%216=126%:%
%:%217=126%:%
%:%218=127%:%
%:%219=127%:%
%:%220=128%:%
%:%221=128%:%
%:%222=129%:%
%:%223=129%:%
%:%224=129%:%
%:%225=130%:%
%:%226=130%:%
%:%227=131%:%
%:%228=131%:%
%:%229=132%:%
%:%230=132%:%
%:%231=132%:%
%:%232=133%:%
%:%233=134%:%
%:%234=134%:%
%:%235=135%:%
%:%236=135%:%
%:%237=136%:%
%:%238=136%:%
%:%239=137%:%
%:%240=137%:%
%:%241=138%:%
%:%242=138%:%
%:%243=139%:%
%:%247=143%:%
%:%248=144%:%
%:%249=144%:%
%:%250=145%:%
%:%251=146%:%
%:%252=147%:%
%:%253=147%:%
%:%254=148%:%
%:%255=148%:%
%:%256=149%:%
%:%257=149%:%
%:%258=150%:%
%:%259=150%:%
%:%260=151%:%
%:%261=152%:%
%:%262=153%:%
%:%263=153%:%
%:%264=154%:%
%:%265=155%:%
%:%266=156%:%
%:%267=156%:%
%:%268=157%:%
%:%269=157%:%
%:%270=158%:%
%:%271=158%:%
%:%272=159%:%
%:%273=160%:%
%:%274=161%:%
%:%275=161%:%
%:%276=162%:%
%:%277=162%:%
%:%278=162%:%
%:%279=163%:%
%:%280=163%:%
%:%281=164%:%
%:%282=165%:%
%:%283=165%:%
%:%284=166%:%
%:%285=166%:%
%:%286=167%:%
%:%287=167%:%
%:%288=168%:%
%:%289=168%:%
%:%290=169%:%
%:%291=169%:%
%:%292=170%:%
%:%293=170%:%
%:%294=171%:%
%:%295=171%:%
%:%296=172%:%
%:%297=172%:%
%:%298=173%:%
%:%299=174%:%
%:%300=175%:%
%:%301=175%:%
%:%302=176%:%
%:%303=177%:%
%:%304=178%:%
%:%305=178%:%
%:%306=179%:%
%:%307=180%:%
%:%308=181%:%
%:%309=182%:%
%:%310=182%:%
%:%311=183%:%
%:%312=183%:%
%:%313=184%:%
%:%314=185%:%
%:%315=186%:%
%:%316=186%:%
%:%317=187%:%
%:%318=187%:%
%:%319=188%:%
%:%320=188%:%
%:%321=189%:%
%:%322=190%:%
%:%323=191%:%
%:%324=192%:%
%:%325=193%:%
%:%326=193%:%
%:%327=194%:%
%:%328=194%:%
%:%329=195%:%
%:%330=196%:%
%:%331=197%:%
%:%332=197%:%
%:%333=198%:%
%:%334=199%:%
%:%335=200%:%
%:%336=200%:%
%:%337=201%:%
%:%338=201%:%
%:%339=202%:%
%:%340=203%:%
%:%341=203%:%
%:%342=204%:%
%:%343=205%:%
%:%344=205%:%
%:%345=206%:%
%:%346=206%:%
%:%347=207%:%
%:%348=208%:%
%:%349=209%:%
%:%350=209%:%
%:%351=210%:%
%:%352=211%:%
%:%353=212%:%
%:%354=213%:%
%:%355=214%:%
%:%356=214%:%
%:%357=215%:%
%:%358=215%:%
%:%359=216%:%
%:%360=216%:%
%:%361=217%:%
%:%362=217%:%
%:%363=218%:%
%:%364=218%:%
%:%365=219%:%
%:%366=220%:%
%:%367=221%:%
%:%368=221%:%
%:%369=222%:%
%:%370=223%:%
%:%371=223%:%
%:%372=224%:%
%:%373=224%:%
%:%374=225%:%
%:%376=227%:%
%:%377=228%:%
%:%378=228%:%
%:%379=229%:%
%:%380=230%:%
%:%381=230%:%
%:%382=231%:%
%:%383=231%:%
%:%384=232%:%
%:%387=235%:%
%:%388=236%:%
%:%389=236%:%
%:%390=237%:%
%:%391=237%:%
%:%392=238%:%
%:%393=238%:%
%:%394=239%:%
%:%395=239%:%
%:%396=240%:%
%:%397=241%:%
%:%398=242%:%
%:%399=243%:%
%:%400=244%:%
%:%401=244%:%
%:%402=245%:%
%:%403=245%:%
%:%404=246%:%
%:%405=246%:%
%:%406=247%:%
%:%412=247%:%
%:%415=248%:%
%:%416=252%:%
%:%417=253%:%
%:%418=253%:%
%:%419=254%:%
%:%420=255%:%
%:%421=256%:%
%:%422=257%:%
%:%423=258%:%
%:%426=259%:%
%:%430=259%:%
%:%431=259%:%
%:%432=260%:%
%:%433=260%:%
%:%434=261%:%
%:%435=261%:%
%:%436=262%:%
%:%437=262%:%
%:%438=263%:%
%:%439=263%:%
%:%440=264%:%
%:%441=264%:%
%:%442=265%:%
%:%443=265%:%
%:%444=266%:%
%:%445=266%:%
%:%446=267%:%
%:%447=267%:%
%:%448=268%:%
%:%449=268%:%
%:%450=269%:%
%:%451=269%:%
%:%452=270%:%
%:%453=270%:%
%:%454=271%:%
%:%455=271%:%
%:%456=272%:%
%:%457=273%:%
%:%458=274%:%
%:%459=275%:%
%:%460=276%:%
%:%461=276%:%
%:%462=277%:%
%:%463=278%:%
%:%464=279%:%
%:%465=279%:%
%:%466=280%:%
%:%467=280%:%
%:%468=281%:%
%:%469=281%:%
%:%470=281%:%
%:%471=282%:%
%:%474=285%:%
%:%475=286%:%
%:%476=286%:%
%:%477=287%:%
%:%478=287%:%
%:%479=288%:%
%:%480=288%:%
%:%481=289%:%
%:%482=290%:%
%:%483=290%:%
%:%484=291%:%
%:%485=291%:%
%:%486=292%:%
%:%487=292%:%
%:%488=293%:%
%:%489=293%:%
%:%490=294%:%
%:%491=294%:%
%:%492=295%:%
%:%493=295%:%
%:%494=295%:%
%:%495=296%:%
%:%496=296%:%
%:%497=297%:%
%:%498=298%:%
%:%499=298%:%
%:%500=299%:%
%:%501=299%:%
%:%502=300%:%
%:%503=301%:%
%:%504=301%:%
%:%505=302%:%
%:%506=303%:%
%:%507=303%:%
%:%508=304%:%
%:%509=304%:%
%:%510=304%:%
%:%511=305%:%
%:%512=305%:%
%:%513=306%:%
%:%514=307%:%
%:%515=307%:%
%:%516=308%:%
%:%517=308%:%
%:%518=308%:%
%:%519=309%:%
%:%520=310%:%
%:%521=310%:%
%:%522=311%:%
%:%523=312%:%
%:%524=312%:%
%:%525=313%:%
%:%526=313%:%
%:%527=313%:%
%:%528=314%:%
%:%529=315%:%
%:%530=315%:%
%:%531=316%:%
%:%532=316%:%
%:%533=317%:%
%:%534=318%:%
%:%535=318%:%
%:%536=319%:%
%:%537=320%:%
%:%538=320%:%
%:%539=321%:%
%:%540=321%:%
%:%541=321%:%
%:%542=322%:%
%:%543=322%:%
%:%544=323%:%
%:%545=324%:%
%:%546=325%:%
%:%547=325%:%
%:%548=326%:%
%:%549=326%:%
%:%550=326%:%
%:%551=327%:%
%:%552=328%:%
%:%553=328%:%
%:%554=329%:%
%:%555=330%:%
%:%556=330%:%
%:%557=331%:%
%:%558=331%:%
%:%559=331%:%
%:%560=332%:%
%:%561=333%:%
%:%562=333%:%
%:%563=334%:%
%:%564=334%:%
%:%565=335%:%
%:%566=335%:%
%:%567=336%:%
%:%568=336%:%
%:%569=337%:%
%:%570=337%:%
%:%571=338%:%
%:%572=338%:%
%:%573=339%:%
%:%574=339%:%
%:%575=340%:%
%:%576=340%:%
%:%577=341%:%
%:%578=342%:%
%:%579=342%:%
%:%580=343%:%
%:%581=343%:%
%:%582=344%:%
%:%583=344%:%
%:%584=345%:%
%:%585=345%:%
%:%586=346%:%
%:%587=346%:%
%:%588=346%:%
%:%589=347%:%
%:%590=347%:%
%:%591=348%:%
%:%592=349%:%
%:%593=350%:%
%:%594=351%:%
%:%595=352%:%
%:%596=352%:%
%:%597=353%:%
%:%598=353%:%
%:%599=354%:%
%:%600=355%:%
%:%601=355%:%
%:%602=356%:%
%:%603=357%:%
%:%604=357%:%
%:%605=358%:%
%:%606=358%:%
%:%607=358%:%
%:%608=359%:%
%:%609=359%:%
%:%610=360%:%
%:%611=361%:%
%:%612=362%:%
%:%613=362%:%
%:%614=363%:%
%:%615=363%:%
%:%616=363%:%
%:%617=364%:%
%:%618=365%:%
%:%619=365%:%
%:%620=366%:%
%:%621=367%:%
%:%622=367%:%
%:%623=368%:%
%:%624=368%:%
%:%625=368%:%
%:%626=369%:%
%:%627=370%:%
%:%628=370%:%
%:%629=371%:%
%:%630=371%:%
%:%631=372%:%
%:%632=373%:%
%:%633=373%:%
%:%634=374%:%
%:%635=375%:%
%:%636=375%:%
%:%637=376%:%
%:%638=376%:%
%:%639=376%:%
%:%640=377%:%
%:%641=377%:%
%:%642=378%:%
%:%643=379%:%
%:%644=379%:%
%:%645=380%:%
%:%646=380%:%
%:%647=380%:%
%:%648=381%:%
%:%649=382%:%
%:%650=382%:%
%:%651=383%:%
%:%652=384%:%
%:%653=384%:%
%:%654=385%:%
%:%655=385%:%
%:%656=385%:%
%:%657=386%:%
%:%658=387%:%
%:%659=387%:%
%:%660=388%:%
%:%661=388%:%
%:%662=389%:%
%:%663=389%:%
%:%664=390%:%
%:%665=390%:%
%:%666=391%:%
%:%667=391%:%
%:%668=392%:%
%:%669=392%:%
%:%670=393%:%
%:%671=393%:%
%:%672=394%:%
%:%673=394%:%
%:%674=395%:%
%:%675=396%:%
%:%676=396%:%
%:%677=397%:%
%:%683=397%:%
%:%686=398%:%
%:%687=399%:%
%:%688=399%:%
%:%689=400%:%
%:%690=401%:%
%:%691=402%:%
%:%692=403%:%
%:%693=404%:%
%:%700=405%:%
%:%701=405%:%
%:%702=406%:%
%:%703=406%:%
%:%704=407%:%
%:%705=407%:%
%:%706=408%:%
%:%707=408%:%
%:%708=409%:%
%:%709=409%:%
%:%710=410%:%
%:%711=410%:%
%:%712=410%:%
%:%713=411%:%
%:%714=412%:%
%:%715=412%:%
%:%716=413%:%
%:%717=413%:%
%:%718=414%:%
%:%719=414%:%
%:%720=415%:%
%:%721=415%:%
%:%722=416%:%
%:%723=416%:%
%:%724=416%:%
%:%725=417%:%
%:%726=417%:%
%:%727=418%:%
%:%728=418%:%
%:%729=418%:%
%:%730=419%:%
%:%731=419%:%
%:%732=420%:%
%:%733=420%:%
%:%734=421%:%
%:%735=421%:%
%:%736=422%:%
%:%737=422%:%
%:%738=423%:%
%:%739=424%:%
%:%740=424%:%
%:%741=425%:%
%:%742=425%:%
%:%743=426%:%
%:%744=426%:%
%:%745=427%:%
%:%746=427%:%
%:%747=428%:%
%:%748=429%:%
%:%749=429%:%
%:%750=430%:%
%:%751=430%:%
%:%752=431%:%
%:%753=431%:%
%:%754=432%:%
%:%755=433%:%
%:%756=433%:%
%:%757=434%:%
%:%758=434%:%
%:%759=435%:%
%:%760=435%:%
%:%761=436%:%
%:%762=436%:%
%:%763=437%:%
%:%764=438%:%
%:%765=438%:%
%:%766=439%:%
%:%767=439%:%
%:%768=440%:%
%:%769=440%:%
%:%770=441%:%
%:%771=441%:%
%:%772=441%:%
%:%773=442%:%
%:%774=442%:%
%:%775=443%:%
%:%776=443%:%
%:%777=443%:%
%:%778=444%:%
%:%779=445%:%
%:%780=445%:%
%:%781=446%:%
%:%782=446%:%
%:%783=447%:%
%:%784=447%:%
%:%785=447%:%
%:%786=448%:%
%:%787=448%:%
%:%788=449%:%
%:%789=450%:%
%:%790=450%:%
%:%791=451%:%
%:%792=451%:%
%:%793=452%:%
%:%794=452%:%
%:%795=452%:%
%:%796=453%:%
%:%797=453%:%
%:%798=454%:%
%:%799=454%:%
%:%800=455%:%
%:%801=455%:%
%:%802=455%:%
%:%803=455%:%
%:%804=456%:%
%:%805=457%:%
%:%806=457%:%
%:%807=458%:%
%:%808=458%:%
%:%809=458%:%
%:%810=459%:%
%:%814=463%:%
%:%815=464%:%
%:%816=464%:%
%:%817=465%:%
%:%818=465%:%
%:%819=466%:%
%:%820=466%:%
%:%821=466%:%
%:%822=467%:%
%:%823=468%:%
%:%824=469%:%
%:%825=469%:%
%:%826=470%:%
%:%827=470%:%
%:%828=471%:%
%:%829=471%:%
%:%830=471%:%
%:%831=472%:%
%:%833=474%:%
%:%834=475%:%
%:%835=475%:%
%:%836=476%:%
%:%837=477%:%
%:%838=478%:%
%:%839=478%:%
%:%840=479%:%
%:%841=479%:%
%:%842=480%:%
%:%843=480%:%
%:%844=481%:%
%:%845=481%:%
%:%846=482%:%
%:%847=482%:%
%:%848=483%:%
%:%849=483%:%
%:%850=484%:%
%:%851=484%:%
%:%852=485%:%
%:%853=485%:%
%:%854=486%:%
%:%855=486%:%
%:%856=487%:%
%:%857=487%:%
%:%858=488%:%
%:%864=488%:%
%:%867=489%:%
%:%868=490%:%
%:%869=491%:%
%:%870=491%:%
%:%871=492%:%
%:%872=493%:%
%:%873=494%:%
%:%874=495%:%
%:%877=496%:%
%:%881=496%:%
%:%882=496%:%
%:%883=497%:%
%:%884=498%:%
%:%885=498%:%
%:%890=498%:%
%:%893=499%:%
%:%894=500%:%
%:%895=500%:%
%:%896=501%:%
%:%897=502%:%
%:%898=503%:%
%:%899=504%:%
%:%900=505%:%
%:%905=510%:%
%:%912=511%:%
%:%913=511%:%
%:%914=512%:%
%:%915=512%:%
%:%916=513%:%
%:%917=513%:%
%:%918=514%:%
%:%920=516%:%
%:%921=517%:%
%:%922=517%:%
%:%923=518%:%
%:%924=518%:%
%:%925=519%:%
%:%926=519%:%
%:%927=520%:%
%:%929=522%:%
%:%930=523%:%
%:%931=523%:%
%:%932=524%:%
%:%933=524%:%
%:%934=525%:%
%:%936=527%:%
%:%937=528%:%
%:%938=528%:%
%:%939=529%:%
%:%940=529%:%
%:%941=530%:%
%:%942=530%:%
%:%943=531%:%
%:%944=531%:%
%:%945=532%:%
%:%946=532%:%
%:%947=533%:%
%:%948=533%:%
%:%949=534%:%
%:%950=534%:%
%:%951=535%:%
%:%953=537%:%
%:%954=538%:%
%:%955=538%:%
%:%956=539%:%
%:%957=540%:%
%:%958=541%:%
%:%959=542%:%
%:%960=543%:%
%:%961=543%:%
%:%962=544%:%
%:%963=544%:%
%:%964=544%:%
%:%965=544%:%
%:%966=545%:%
%:%967=546%:%
%:%968=546%:%
%:%969=547%:%
%:%970=548%:%
%:%971=548%:%
%:%972=549%:%
%:%973=549%:%
%:%974=549%:%
%:%975=550%:%
%:%979=554%:%
%:%980=555%:%
%:%981=555%:%
%:%982=556%:%
%:%983=556%:%
%:%984=557%:%
%:%985=557%:%
%:%986=558%:%
%:%987=558%:%
%:%988=559%:%
%:%989=559%:%
%:%990=560%:%
%:%991=560%:%
%:%992=561%:%
%:%993=561%:%
%:%994=562%:%
%:%995=563%:%
%:%996=563%:%
%:%997=563%:%
%:%998=564%:%
%:%999=565%:%
%:%1000=566%:%
%:%1001=566%:%
%:%1002=567%:%
%:%1003=568%:%
%:%1004=568%:%
%:%1005=569%:%
%:%1006=569%:%
%:%1007=569%:%
%:%1008=570%:%
%:%1010=572%:%
%:%1011=573%:%
%:%1012=573%:%
%:%1013=574%:%
%:%1014=574%:%
%:%1015=575%:%
%:%1016=575%:%
%:%1017=576%:%
%:%1018=576%:%
%:%1019=577%:%
%:%1020=577%:%
%:%1021=578%:%
%:%1022=578%:%
%:%1023=579%:%
%:%1025=581%:%
%:%1026=582%:%
%:%1027=582%:%
%:%1028=583%:%
%:%1029=583%:%
%:%1030=584%:%
%:%1031=584%:%
%:%1032=585%:%
%:%1033=585%:%
%:%1034=586%:%
%:%1036=588%:%
%:%1037=589%:%
%:%1038=589%:%
%:%1039=590%:%
%:%1040=590%:%
%:%1041=591%:%
%:%1046=596%:%
%:%1047=597%:%
%:%1048=597%:%
%:%1049=598%:%
%:%1050=598%:%
%:%1051=599%:%
%:%1052=599%:%
%:%1053=600%:%
%:%1054=600%:%
%:%1055=601%:%
%:%1056=601%:%
%:%1057=602%:%
%:%1058=602%:%
%:%1059=603%:%
%:%1060=603%:%
%:%1061=604%:%
%:%1063=606%:%
%:%1064=607%:%
%:%1065=607%:%
%:%1066=608%:%
%:%1067=609%:%
%:%1068=610%:%
%:%1069=611%:%
%:%1070=611%:%
%:%1071=612%:%
%:%1072=612%:%
%:%1073=612%:%
%:%1074=613%:%
%:%1077=616%:%
%:%1078=617%:%
%:%1079=617%:%
%:%1080=618%:%
%:%1081=619%:%
%:%1082=620%:%
%:%1083=621%:%
%:%1084=621%:%
%:%1085=622%:%
%:%1086=622%:%
%:%1087=623%:%
%:%1088=624%:%
%:%1089=625%:%
%:%1090=625%:%
%:%1091=626%:%
%:%1092=627%:%
%:%1093=628%:%
%:%1094=629%:%
%:%1095=629%:%
%:%1096=630%:%
%:%1097=630%:%
%:%1098=631%:%
%:%1099=631%:%
%:%1100=632%:%
%:%1101=633%:%
%:%1102=634%:%
%:%1103=635%:%
%:%1104=635%:%
%:%1105=636%:%
%:%1106=636%:%
%:%1107=637%:%
%:%1108=637%:%
%:%1109=638%:%
%:%1110=638%:%
%:%1111=639%:%
%:%1112=639%:%
%:%1113=640%:%
%:%1114=640%:%
%:%1115=641%:%
%:%1116=641%:%
%:%1117=642%:%
%:%1118=642%:%
%:%1119=643%:%
%:%1125=643%:%
%:%1128=644%:%
%:%1129=645%:%
%:%1130=645%:%
%:%1131=646%:%
%:%1132=647%:%
%:%1133=648%:%
%:%1134=649%:%
%:%1135=650%:%
%:%1136=651%:%
%:%1137=652%:%
%:%1140=653%:%
%:%1144=653%:%
%:%1145=653%:%
%:%1146=654%:%
%:%1147=655%:%
%:%1148=655%:%
%:%1153=655%:%
%:%1156=656%:%
%:%1157=657%:%
%:%1158=657%:%
%:%1159=658%:%
%:%1160=659%:%
%:%1161=660%:%
%:%1162=661%:%
%:%1163=662%:%
%:%1166=665%:%
%:%1169=666%:%
%:%1173=666%:%
%:%1174=666%:%
%:%1175=667%:%
%:%1176=667%:%
%:%1181=667%:%
%:%1184=668%:%
%:%1185=669%:%
%:%1186=669%:%
%:%1187=670%:%
%:%1188=671%:%
%:%1191=672%:%
%:%1195=672%:%
%:%1196=672%:%
%:%1197=673%:%
%:%1198=673%:%
%:%1203=673%:%
%:%1206=674%:%
%:%1207=675%:%
%:%1208=675%:%
%:%1209=676%:%
%:%1210=677%:%
%:%1211=678%:%
%:%1212=679%:%
%:%1219=680%:%
%:%1220=680%:%
%:%1221=681%:%
%:%1222=681%:%
%:%1223=682%:%
%:%1230=689%:%
%:%1231=690%:%
%:%1232=690%:%
%:%1233=691%:%
%:%1234=691%:%
%:%1235=692%:%
%:%1236=692%:%
%:%1237=693%:%
%:%1238=693%:%
%:%1239=694%:%
%:%1240=695%:%
%:%1241=696%:%
%:%1242=696%:%
%:%1243=697%:%
%:%1249=697%:%
%:%1252=698%:%
%:%1253=699%:%
%:%1254=700%:%
%:%1255=700%:%
%:%1256=701%:%
%:%1257=702%:%
%:%1264=703%:%
%:%1265=703%:%
%:%1266=704%:%
%:%1267=704%:%
%:%1268=705%:%
%:%1269=706%:%
%:%1270=706%:%
%:%1271=707%:%
%:%1272=708%:%
%:%1273=709%:%
%:%1274=709%:%
%:%1275=710%:%
%:%1276=711%:%
%:%1277=712%:%
%:%1278=712%:%
%:%1279=713%:%
%:%1280=713%:%
%:%1281=714%:%
%:%1282=714%:%
%:%1283=715%:%
%:%1284=716%:%
%:%1285=717%:%
%:%1286=718%:%
%:%1287=718%:%
%:%1288=719%:%
%:%1294=719%:%
%:%1297=720%:%
%:%1298=724%:%
%:%1299=725%:%
%:%1300=725%:%
%:%1301=726%:%
%:%1302=727%:%
%:%1309=728%:%
%:%1310=728%:%
%:%1311=729%:%
%:%1312=729%:%
%:%1313=730%:%
%:%1314=731%:%
%:%1315=731%:%
%:%1316=732%:%
%:%1317=732%:%
%:%1318=733%:%
%:%1319=733%:%
%:%1320=733%:%
%:%1321=734%:%
%:%1322=734%:%
%:%1323=735%:%
%:%1324=735%:%
%:%1325=736%:%
%:%1326=736%:%
%:%1327=736%:%
%:%1328=737%:%
%:%1329=737%:%
%:%1330=738%:%
%:%1331=738%:%
%:%1332=739%:%
%:%1333=739%:%
%:%1334=739%:%
%:%1336=741%:%
%:%1337=742%:%
%:%1338=742%:%
%:%1339=743%:%
%:%1340=743%:%
%:%1341=744%:%
%:%1342=744%:%
%:%1343=745%:%
%:%1344=745%:%
%:%1345=746%:%
%:%1351=746%:%
%:%1354=747%:%
%:%1355=751%:%
%:%1356=752%:%
%:%1357=752%:%
%:%1358=753%:%
%:%1359=754%:%
%:%1362=755%:%
%:%1366=755%:%
%:%1367=755%:%
%:%1368=756%:%
%:%1369=756%:%
%:%1374=756%:%
%:%1377=757%:%
%:%1378=763%:%
%:%1379=764%:%
%:%1380=764%:%
%:%1381=765%:%
%:%1382=766%:%
%:%1383=767%:%
%:%1384=768%:%
%:%1385=769%:%
%:%1386=770%:%
%:%1389=771%:%
%:%1393=771%:%
%:%1394=771%:%
%:%1395=772%:%
%:%1396=772%:%
%:%1397=773%:%
%:%1398=773%:%
%:%1399=774%:%
%:%1400=774%:%
%:%1401=775%:%
%:%1402=775%:%
%:%1403=776%:%
%:%1404=776%:%
%:%1405=777%:%
%:%1406=777%:%
%:%1407=778%:%
%:%1408=778%:%
%:%1409=779%:%
%:%1410=779%:%
%:%1411=780%:%
%:%1412=780%:%
%:%1413=781%:%
%:%1414=781%:%
%:%1415=782%:%
%:%1416=782%:%
%:%1417=783%:%
%:%1418=783%:%
%:%1419=784%:%
%:%1420=785%:%
%:%1421=786%:%
%:%1422=787%:%
%:%1423=788%:%
%:%1424=788%:%
%:%1425=789%:%
%:%1426=790%:%
%:%1427=791%:%
%:%1428=792%:%
%:%1429=792%:%
%:%1430=793%:%
%:%1431=794%:%
%:%1432=794%:%
%:%1433=795%:%
%:%1434=795%:%
%:%1435=796%:%
%:%1436=796%:%
%:%1437=797%:%
%:%1438=797%:%
%:%1439=798%:%
%:%1440=799%:%
%:%1441=800%:%
%:%1442=800%:%
%:%1443=801%:%
%:%1449=801%:%
%:%1454=802%:%
%:%1459=803%:%
%
\begin{isabellebody}%
\setisabellecontext{Disjoint{\isacharunderscore}{\kern0pt}Compatibility{\isacharunderscore}{\kern0pt}Facts}%
%
\isadelimtheory
%
\endisadelimtheory
%
\isatagtheory
\isacommand{theory}\isamarkupfalse%
\ Disjoint{\isacharunderscore}{\kern0pt}Compatibility{\isacharunderscore}{\kern0pt}Facts\isanewline
\ \ \isakeyword{imports}\ {\isachardoublequoteopen}{\isachardot}{\kern0pt}{\isachardot}{\kern0pt}{\isacharslash}{\kern0pt}Properties{\isacharslash}{\kern0pt}Disjoint{\isacharunderscore}{\kern0pt}Compatibility{\isachardoublequoteclose}\isanewline
\ \ \ \ \ \ \ \ \ \ {\isachardoublequoteopen}{\isachardot}{\kern0pt}{\isachardot}{\kern0pt}{\isacharslash}{\kern0pt}Components{\isacharslash}{\kern0pt}Basic{\isacharunderscore}{\kern0pt}Modules{\isacharslash}{\kern0pt}Drop{\isacharunderscore}{\kern0pt}Module{\isachardoublequoteclose}\isanewline
\ \ \ \ \ \ \ \ \ \ {\isachardoublequoteopen}{\isachardot}{\kern0pt}{\isachardot}{\kern0pt}{\isacharslash}{\kern0pt}Components{\isacharslash}{\kern0pt}Basic{\isacharunderscore}{\kern0pt}Modules{\isacharslash}{\kern0pt}Pass{\isacharunderscore}{\kern0pt}Module{\isachardoublequoteclose}\isanewline
\isanewline
\isakeyword{begin}%
\endisatagtheory
{\isafoldtheory}%
%
\isadelimtheory
\isanewline
%
\endisadelimtheory
\isanewline
\isanewline
\isacommand{theorem}\isamarkupfalse%
\ drop{\isacharunderscore}{\kern0pt}pass{\isacharunderscore}{\kern0pt}disj{\isacharunderscore}{\kern0pt}compat{\isacharbrackleft}{\kern0pt}simp{\isacharbrackright}{\kern0pt}{\isacharcolon}{\kern0pt}\isanewline
\ \ \isakeyword{assumes}\ order{\isacharcolon}{\kern0pt}\ {\isachardoublequoteopen}linear{\isacharunderscore}{\kern0pt}order\ r{\isachardoublequoteclose}\isanewline
\ \ \isakeyword{shows}\ {\isachardoublequoteopen}disjoint{\isacharunderscore}{\kern0pt}compatibility\ {\isacharparenleft}{\kern0pt}drop{\isacharunderscore}{\kern0pt}module\ n\ r{\isacharparenright}{\kern0pt}\ {\isacharparenleft}{\kern0pt}pass{\isacharunderscore}{\kern0pt}module\ n\ r{\isacharparenright}{\kern0pt}{\isachardoublequoteclose}\isanewline
%
\isadelimproof
\ \ %
\endisadelimproof
%
\isatagproof
\isacommand{unfolding}\isamarkupfalse%
\ disjoint{\isacharunderscore}{\kern0pt}compatibility{\isacharunderscore}{\kern0pt}def\isanewline
\isacommand{proof}\isamarkupfalse%
\ {\isacharparenleft}{\kern0pt}safe{\isacharparenright}{\kern0pt}\isanewline
\ \ \isacommand{show}\isamarkupfalse%
\ {\isachardoublequoteopen}electoral{\isacharunderscore}{\kern0pt}module\ {\isacharparenleft}{\kern0pt}drop{\isacharunderscore}{\kern0pt}module\ n\ r{\isacharparenright}{\kern0pt}{\isachardoublequoteclose}\isanewline
\ \ \ \ \isacommand{using}\isamarkupfalse%
\ order\isanewline
\ \ \ \ \isacommand{by}\isamarkupfalse%
\ simp\isanewline
\isacommand{next}\isamarkupfalse%
\isanewline
\ \ \isacommand{show}\isamarkupfalse%
\ {\isachardoublequoteopen}electoral{\isacharunderscore}{\kern0pt}module\ {\isacharparenleft}{\kern0pt}pass{\isacharunderscore}{\kern0pt}module\ n\ r{\isacharparenright}{\kern0pt}{\isachardoublequoteclose}\isanewline
\ \ \ \ \isacommand{using}\isamarkupfalse%
\ order\isanewline
\ \ \ \ \isacommand{by}\isamarkupfalse%
\ simp\isanewline
\isacommand{next}\isamarkupfalse%
\isanewline
\ \ \isacommand{fix}\isamarkupfalse%
\isanewline
\ \ \ \ S\ {\isacharcolon}{\kern0pt}{\isacharcolon}{\kern0pt}\ {\isachardoublequoteopen}{\isacharprime}{\kern0pt}a\ set{\isachardoublequoteclose}\isanewline
\ \ \isacommand{assume}\isamarkupfalse%
\isanewline
\ \ \ \ fin{\isacharcolon}{\kern0pt}\ {\isachardoublequoteopen}finite\ S{\isachardoublequoteclose}\isanewline
\ \ \isacommand{obtain}\isamarkupfalse%
\isanewline
\ \ \ \ p\ {\isacharcolon}{\kern0pt}{\isacharcolon}{\kern0pt}\ {\isachardoublequoteopen}{\isacharprime}{\kern0pt}a\ Profile{\isachardoublequoteclose}\isanewline
\ \ \ \ \isakeyword{where}\ {\isachardoublequoteopen}finite{\isacharunderscore}{\kern0pt}profile\ S\ p{\isachardoublequoteclose}\isanewline
\ \ \ \ \isacommand{using}\isamarkupfalse%
\ empty{\isacharunderscore}{\kern0pt}iff\ empty{\isacharunderscore}{\kern0pt}set\ fin\ profile{\isacharunderscore}{\kern0pt}set\isanewline
\ \ \ \ \isacommand{by}\isamarkupfalse%
\ metis\isanewline
\ \ \isacommand{show}\isamarkupfalse%
\isanewline
\ \ \ \ {\isachardoublequoteopen}{\isasymexists}A\ {\isasymsubseteq}\ S{\isachardot}{\kern0pt}\isanewline
\ \ \ \ \ \ {\isacharparenleft}{\kern0pt}{\isasymforall}a\ {\isasymin}\ A{\isachardot}{\kern0pt}\ indep{\isacharunderscore}{\kern0pt}of{\isacharunderscore}{\kern0pt}alt\ {\isacharparenleft}{\kern0pt}drop{\isacharunderscore}{\kern0pt}module\ n\ r{\isacharparenright}{\kern0pt}\ S\ a\ {\isasymand}\isanewline
\ \ \ \ \ \ \ \ {\isacharparenleft}{\kern0pt}{\isasymforall}p{\isachardot}{\kern0pt}\ finite{\isacharunderscore}{\kern0pt}profile\ S\ p\ {\isasymlongrightarrow}\isanewline
\ \ \ \ \ \ \ \ \ \ a\ {\isasymin}\ reject\ {\isacharparenleft}{\kern0pt}drop{\isacharunderscore}{\kern0pt}module\ n\ r{\isacharparenright}{\kern0pt}\ S\ p{\isacharparenright}{\kern0pt}{\isacharparenright}{\kern0pt}\ {\isasymand}\isanewline
\ \ \ \ \ \ {\isacharparenleft}{\kern0pt}{\isasymforall}a\ {\isasymin}\ S{\isacharminus}{\kern0pt}A{\isachardot}{\kern0pt}\ indep{\isacharunderscore}{\kern0pt}of{\isacharunderscore}{\kern0pt}alt\ {\isacharparenleft}{\kern0pt}pass{\isacharunderscore}{\kern0pt}module\ n\ r{\isacharparenright}{\kern0pt}\ S\ a\ {\isasymand}\isanewline
\ \ \ \ \ \ \ \ {\isacharparenleft}{\kern0pt}{\isasymforall}p{\isachardot}{\kern0pt}\ finite{\isacharunderscore}{\kern0pt}profile\ S\ p\ {\isasymlongrightarrow}\isanewline
\ \ \ \ \ \ \ \ \ \ a\ {\isasymin}\ reject\ {\isacharparenleft}{\kern0pt}pass{\isacharunderscore}{\kern0pt}module\ n\ r{\isacharparenright}{\kern0pt}\ S\ p{\isacharparenright}{\kern0pt}{\isacharparenright}{\kern0pt}{\isachardoublequoteclose}\isanewline
\ \ \isacommand{proof}\isamarkupfalse%
\isanewline
\ \ \ \ \isacommand{have}\isamarkupfalse%
\ same{\isacharunderscore}{\kern0pt}A{\isacharcolon}{\kern0pt}\isanewline
\ \ \ \ \ \ {\isachardoublequoteopen}{\isasymforall}p\ q{\isachardot}{\kern0pt}\ {\isacharparenleft}{\kern0pt}finite{\isacharunderscore}{\kern0pt}profile\ S\ p\ {\isasymand}\ finite{\isacharunderscore}{\kern0pt}profile\ S\ q{\isacharparenright}{\kern0pt}\ {\isasymlongrightarrow}\isanewline
\ \ \ \ \ \ \ \ reject\ {\isacharparenleft}{\kern0pt}drop{\isacharunderscore}{\kern0pt}module\ n\ r{\isacharparenright}{\kern0pt}\ S\ p\ {\isacharequal}{\kern0pt}\isanewline
\ \ \ \ \ \ \ \ \ \ reject\ {\isacharparenleft}{\kern0pt}drop{\isacharunderscore}{\kern0pt}module\ n\ r{\isacharparenright}{\kern0pt}\ S\ q{\isachardoublequoteclose}\isanewline
\ \ \ \ \ \ \isacommand{by}\isamarkupfalse%
\ auto\isanewline
\ \ \ \ \isacommand{let}\isamarkupfalse%
\ {\isacharquery}{\kern0pt}A\ {\isacharequal}{\kern0pt}\ {\isachardoublequoteopen}reject\ {\isacharparenleft}{\kern0pt}drop{\isacharunderscore}{\kern0pt}module\ n\ r{\isacharparenright}{\kern0pt}\ S\ p{\isachardoublequoteclose}\isanewline
\ \ \ \ \isacommand{have}\isamarkupfalse%
\ {\isachardoublequoteopen}{\isacharquery}{\kern0pt}A\ {\isasymsubseteq}\ S{\isachardoublequoteclose}\isanewline
\ \ \ \ \ \ \isacommand{by}\isamarkupfalse%
\ auto\isanewline
\ \ \ \ \isacommand{moreover}\isamarkupfalse%
\ \isacommand{have}\isamarkupfalse%
\isanewline
\ \ \ \ \ \ {\isachardoublequoteopen}{\isacharparenleft}{\kern0pt}{\isasymforall}a\ {\isasymin}\ {\isacharquery}{\kern0pt}A{\isachardot}{\kern0pt}\ indep{\isacharunderscore}{\kern0pt}of{\isacharunderscore}{\kern0pt}alt\ {\isacharparenleft}{\kern0pt}drop{\isacharunderscore}{\kern0pt}module\ n\ r{\isacharparenright}{\kern0pt}\ S\ a{\isacharparenright}{\kern0pt}{\isachardoublequoteclose}\isanewline
\ \ \ \ \ \ \isacommand{using}\isamarkupfalse%
\ order\isanewline
\ \ \ \ \ \ \isacommand{by}\isamarkupfalse%
\ {\isacharparenleft}{\kern0pt}simp\ add{\isacharcolon}{\kern0pt}\ indep{\isacharunderscore}{\kern0pt}of{\isacharunderscore}{\kern0pt}alt{\isacharunderscore}{\kern0pt}def{\isacharparenright}{\kern0pt}\isanewline
\ \ \ \ \isacommand{moreover}\isamarkupfalse%
\ \isacommand{have}\isamarkupfalse%
\isanewline
\ \ \ \ \ \ {\isachardoublequoteopen}{\isasymforall}a\ {\isasymin}\ {\isacharquery}{\kern0pt}A{\isachardot}{\kern0pt}\ {\isasymforall}p{\isachardot}{\kern0pt}\ finite{\isacharunderscore}{\kern0pt}profile\ S\ p\ {\isasymlongrightarrow}\isanewline
\ \ \ \ \ \ \ \ a\ {\isasymin}\ reject\ {\isacharparenleft}{\kern0pt}drop{\isacharunderscore}{\kern0pt}module\ n\ r{\isacharparenright}{\kern0pt}\ S\ p{\isachardoublequoteclose}\isanewline
\ \ \ \ \ \ \isacommand{by}\isamarkupfalse%
\ auto\isanewline
\ \ \ \ \isacommand{moreover}\isamarkupfalse%
\ \isacommand{have}\isamarkupfalse%
\isanewline
\ \ \ \ \ \ {\isachardoublequoteopen}{\isacharparenleft}{\kern0pt}{\isasymforall}a\ {\isasymin}\ S{\isacharminus}{\kern0pt}{\isacharquery}{\kern0pt}A{\isachardot}{\kern0pt}\ indep{\isacharunderscore}{\kern0pt}of{\isacharunderscore}{\kern0pt}alt\ {\isacharparenleft}{\kern0pt}pass{\isacharunderscore}{\kern0pt}module\ n\ r{\isacharparenright}{\kern0pt}\ S\ a{\isacharparenright}{\kern0pt}{\isachardoublequoteclose}\isanewline
\ \ \ \ \ \ \isacommand{using}\isamarkupfalse%
\ order\isanewline
\ \ \ \ \ \ \isacommand{by}\isamarkupfalse%
\ {\isacharparenleft}{\kern0pt}simp\ add{\isacharcolon}{\kern0pt}\ indep{\isacharunderscore}{\kern0pt}of{\isacharunderscore}{\kern0pt}alt{\isacharunderscore}{\kern0pt}def{\isacharparenright}{\kern0pt}\isanewline
\ \ \ \ \isacommand{moreover}\isamarkupfalse%
\ \isacommand{have}\isamarkupfalse%
\isanewline
\ \ \ \ \ \ {\isachardoublequoteopen}{\isasymforall}a\ {\isasymin}\ S{\isacharminus}{\kern0pt}{\isacharquery}{\kern0pt}A{\isachardot}{\kern0pt}\ {\isasymforall}p{\isachardot}{\kern0pt}\ finite{\isacharunderscore}{\kern0pt}profile\ S\ p\ {\isasymlongrightarrow}\isanewline
\ \ \ \ \ \ \ \ a\ {\isasymin}\ reject\ {\isacharparenleft}{\kern0pt}pass{\isacharunderscore}{\kern0pt}module\ n\ r{\isacharparenright}{\kern0pt}\ S\ p{\isachardoublequoteclose}\isanewline
\ \ \ \ \ \ \isacommand{by}\isamarkupfalse%
\ auto\isanewline
\ \ \ \ \isacommand{ultimately}\isamarkupfalse%
\ \isacommand{show}\isamarkupfalse%
\isanewline
\ \ \ \ \ \ {\isachardoublequoteopen}{\isacharquery}{\kern0pt}A\ {\isasymsubseteq}\ S\ {\isasymand}\isanewline
\ \ \ \ \ \ \ \ {\isacharparenleft}{\kern0pt}{\isasymforall}a\ {\isasymin}\ {\isacharquery}{\kern0pt}A{\isachardot}{\kern0pt}\ indep{\isacharunderscore}{\kern0pt}of{\isacharunderscore}{\kern0pt}alt\ {\isacharparenleft}{\kern0pt}drop{\isacharunderscore}{\kern0pt}module\ n\ r{\isacharparenright}{\kern0pt}\ S\ a\ {\isasymand}\isanewline
\ \ \ \ \ \ \ \ \ \ {\isacharparenleft}{\kern0pt}{\isasymforall}p{\isachardot}{\kern0pt}\ finite{\isacharunderscore}{\kern0pt}profile\ S\ p\ {\isasymlongrightarrow}\isanewline
\ \ \ \ \ \ \ \ \ \ \ \ a\ {\isasymin}\ reject\ {\isacharparenleft}{\kern0pt}drop{\isacharunderscore}{\kern0pt}module\ n\ r{\isacharparenright}{\kern0pt}\ S\ p{\isacharparenright}{\kern0pt}{\isacharparenright}{\kern0pt}\ {\isasymand}\isanewline
\ \ \ \ \ \ \ \ {\isacharparenleft}{\kern0pt}{\isasymforall}a\ {\isasymin}\ S{\isacharminus}{\kern0pt}{\isacharquery}{\kern0pt}A{\isachardot}{\kern0pt}\ indep{\isacharunderscore}{\kern0pt}of{\isacharunderscore}{\kern0pt}alt\ {\isacharparenleft}{\kern0pt}pass{\isacharunderscore}{\kern0pt}module\ n\ r{\isacharparenright}{\kern0pt}\ S\ a\ {\isasymand}\isanewline
\ \ \ \ \ \ \ \ \ \ {\isacharparenleft}{\kern0pt}{\isasymforall}p{\isachardot}{\kern0pt}\ finite{\isacharunderscore}{\kern0pt}profile\ S\ p\ {\isasymlongrightarrow}\isanewline
\ \ \ \ \ \ \ \ \ \ \ \ a\ {\isasymin}\ reject\ {\isacharparenleft}{\kern0pt}pass{\isacharunderscore}{\kern0pt}module\ n\ r{\isacharparenright}{\kern0pt}\ S\ p{\isacharparenright}{\kern0pt}{\isacharparenright}{\kern0pt}{\isachardoublequoteclose}\isanewline
\ \ \ \ \ \ \isacommand{by}\isamarkupfalse%
\ simp\isanewline
\ \ \isacommand{qed}\isamarkupfalse%
\isanewline
\isacommand{qed}\isamarkupfalse%
%
\endisatagproof
{\isafoldproof}%
%
\isadelimproof
\isanewline
%
\endisadelimproof
%
\isadelimtheory
\isanewline
%
\endisadelimtheory
%
\isatagtheory
\isacommand{end}\isamarkupfalse%
%
\endisatagtheory
{\isafoldtheory}%
%
\isadelimtheory
%
\endisadelimtheory
%
\end{isabellebody}%
\endinput
%:%file=~/Documents/Studies/VotingRuleGenerator/virage/src/test/resources/verifiedVotingRuleConstruction/theories/Compositional_Framework/Composition_Rules/Disjoint_Compatibility_Facts.thy%:%
%:%10=1%:%
%:%11=1%:%
%:%12=2%:%
%:%13=3%:%
%:%14=4%:%
%:%15=5%:%
%:%16=6%:%
%:%21=6%:%
%:%24=7%:%
%:%25=8%:%
%:%26=9%:%
%:%27=9%:%
%:%28=10%:%
%:%29=11%:%
%:%32=12%:%
%:%36=12%:%
%:%37=12%:%
%:%38=13%:%
%:%39=13%:%
%:%40=14%:%
%:%41=14%:%
%:%42=15%:%
%:%43=15%:%
%:%44=16%:%
%:%45=16%:%
%:%46=17%:%
%:%47=17%:%
%:%48=18%:%
%:%49=18%:%
%:%50=19%:%
%:%51=19%:%
%:%52=20%:%
%:%53=20%:%
%:%54=21%:%
%:%55=21%:%
%:%56=22%:%
%:%57=22%:%
%:%58=23%:%
%:%59=24%:%
%:%60=24%:%
%:%61=25%:%
%:%62=26%:%
%:%63=26%:%
%:%64=27%:%
%:%65=28%:%
%:%66=29%:%
%:%67=29%:%
%:%68=30%:%
%:%69=30%:%
%:%70=31%:%
%:%71=31%:%
%:%72=32%:%
%:%78=38%:%
%:%79=39%:%
%:%80=39%:%
%:%81=40%:%
%:%82=40%:%
%:%83=41%:%
%:%85=43%:%
%:%86=44%:%
%:%87=44%:%
%:%88=45%:%
%:%89=45%:%
%:%90=46%:%
%:%91=46%:%
%:%92=47%:%
%:%93=47%:%
%:%94=48%:%
%:%95=48%:%
%:%96=48%:%
%:%97=49%:%
%:%98=50%:%
%:%99=50%:%
%:%100=51%:%
%:%101=51%:%
%:%102=52%:%
%:%103=52%:%
%:%104=52%:%
%:%105=53%:%
%:%106=54%:%
%:%107=55%:%
%:%108=55%:%
%:%109=56%:%
%:%110=56%:%
%:%111=56%:%
%:%112=57%:%
%:%113=58%:%
%:%114=58%:%
%:%115=59%:%
%:%116=59%:%
%:%117=60%:%
%:%118=60%:%
%:%119=60%:%
%:%120=61%:%
%:%121=62%:%
%:%122=63%:%
%:%123=63%:%
%:%124=64%:%
%:%125=64%:%
%:%126=64%:%
%:%127=65%:%
%:%133=71%:%
%:%134=72%:%
%:%135=72%:%
%:%136=73%:%
%:%137=73%:%
%:%138=74%:%
%:%144=74%:%
%:%149=75%:%
%:%154=76%:%
%
\begin{isabellebody}%
\setisabellecontext{Disjoint{\isacharunderscore}{\kern0pt}Compatibility{\isacharunderscore}{\kern0pt}Rules}%
%
\isadelimtheory
%
\endisadelimtheory
%
\isatagtheory
\isacommand{theory}\isamarkupfalse%
\ Disjoint{\isacharunderscore}{\kern0pt}Compatibility{\isacharunderscore}{\kern0pt}Rules\isanewline
\ \ \isakeyword{imports}\ {\isachardoublequoteopen}{\isachardot}{\kern0pt}{\isachardot}{\kern0pt}{\isacharslash}{\kern0pt}Properties{\isacharslash}{\kern0pt}Disjoint{\isacharunderscore}{\kern0pt}Compatibility{\isachardoublequoteclose}\isanewline
\ \ \ \ \ \ \ \ \ \ {\isachardoublequoteopen}{\isachardot}{\kern0pt}{\isachardot}{\kern0pt}{\isacharslash}{\kern0pt}Components{\isacharslash}{\kern0pt}Compositional{\isacharunderscore}{\kern0pt}Structures{\isacharslash}{\kern0pt}Sequential{\isacharunderscore}{\kern0pt}Composition{\isachardoublequoteclose}\isanewline
\isanewline
\isakeyword{begin}%
\endisatagtheory
{\isafoldtheory}%
%
\isadelimtheory
\isanewline
%
\endisadelimtheory
\isanewline
\isanewline
\isacommand{theorem}\isamarkupfalse%
\ disj{\isacharunderscore}{\kern0pt}compat{\isacharunderscore}{\kern0pt}comm{\isacharbrackleft}{\kern0pt}simp{\isacharbrackright}{\kern0pt}{\isacharcolon}{\kern0pt}\isanewline
\ \ \isakeyword{assumes}\ compatible{\isacharcolon}{\kern0pt}\ {\isachardoublequoteopen}disjoint{\isacharunderscore}{\kern0pt}compatibility\ m\ n{\isachardoublequoteclose}\isanewline
\ \ \isakeyword{shows}\ {\isachardoublequoteopen}disjoint{\isacharunderscore}{\kern0pt}compatibility\ n\ m{\isachardoublequoteclose}\isanewline
%
\isadelimproof
%
\endisadelimproof
%
\isatagproof
\isacommand{proof}\isamarkupfalse%
\ {\isacharminus}{\kern0pt}\isanewline
\ \ \isacommand{have}\isamarkupfalse%
\isanewline
\ \ \ \ {\isachardoublequoteopen}{\isasymforall}S{\isachardot}{\kern0pt}\ finite\ S\ {\isasymlongrightarrow}\isanewline
\ \ \ \ \ \ \ \ {\isacharparenleft}{\kern0pt}{\isasymexists}A\ {\isasymsubseteq}\ S{\isachardot}{\kern0pt}\isanewline
\ \ \ \ \ \ \ \ \ \ {\isacharparenleft}{\kern0pt}{\isasymforall}a\ {\isasymin}\ A{\isachardot}{\kern0pt}\ indep{\isacharunderscore}{\kern0pt}of{\isacharunderscore}{\kern0pt}alt\ n\ S\ a\ {\isasymand}\isanewline
\ \ \ \ \ \ \ \ \ \ \ \ {\isacharparenleft}{\kern0pt}{\isasymforall}p{\isachardot}{\kern0pt}\ finite{\isacharunderscore}{\kern0pt}profile\ S\ p\ {\isasymlongrightarrow}\ a\ {\isasymin}\ reject\ n\ S\ p{\isacharparenright}{\kern0pt}{\isacharparenright}{\kern0pt}\ {\isasymand}\isanewline
\ \ \ \ \ \ \ \ \ \ {\isacharparenleft}{\kern0pt}{\isasymforall}a\ {\isasymin}\ S{\isacharminus}{\kern0pt}A{\isachardot}{\kern0pt}\ indep{\isacharunderscore}{\kern0pt}of{\isacharunderscore}{\kern0pt}alt\ m\ S\ a\ {\isasymand}\isanewline
\ \ \ \ \ \ \ \ \ \ \ \ {\isacharparenleft}{\kern0pt}{\isasymforall}p{\isachardot}{\kern0pt}\ finite{\isacharunderscore}{\kern0pt}profile\ S\ p\ {\isasymlongrightarrow}\ a\ {\isasymin}\ reject\ m\ S\ p{\isacharparenright}{\kern0pt}{\isacharparenright}{\kern0pt}{\isacharparenright}{\kern0pt}{\isachardoublequoteclose}\isanewline
\ \ \isacommand{proof}\isamarkupfalse%
\isanewline
\ \ \ \ \isacommand{fix}\isamarkupfalse%
\isanewline
\ \ \ \ \ \ S\ {\isacharcolon}{\kern0pt}{\isacharcolon}{\kern0pt}\ {\isachardoublequoteopen}{\isacharprime}{\kern0pt}a\ set{\isachardoublequoteclose}\isanewline
\ \ \ \ \isacommand{obtain}\isamarkupfalse%
\ A\ \isakeyword{where}\ old{\isacharunderscore}{\kern0pt}A{\isacharcolon}{\kern0pt}\isanewline
\ \ \ \ \ \ {\isachardoublequoteopen}finite\ S\ {\isasymlongrightarrow}\isanewline
\ \ \ \ \ \ \ \ \ \ {\isacharparenleft}{\kern0pt}A\ {\isasymsubseteq}\ S\ {\isasymand}\isanewline
\ \ \ \ \ \ \ \ \ \ \ \ {\isacharparenleft}{\kern0pt}{\isasymforall}a\ {\isasymin}\ A{\isachardot}{\kern0pt}\ indep{\isacharunderscore}{\kern0pt}of{\isacharunderscore}{\kern0pt}alt\ m\ S\ a\ {\isasymand}\isanewline
\ \ \ \ \ \ \ \ \ \ \ \ \ \ {\isacharparenleft}{\kern0pt}{\isasymforall}p{\isachardot}{\kern0pt}\ finite{\isacharunderscore}{\kern0pt}profile\ S\ p\ {\isasymlongrightarrow}\ a\ {\isasymin}\ reject\ m\ S\ p{\isacharparenright}{\kern0pt}{\isacharparenright}{\kern0pt}\ {\isasymand}\isanewline
\ \ \ \ \ \ \ \ \ \ \ \ {\isacharparenleft}{\kern0pt}{\isasymforall}a\ {\isasymin}\ S{\isacharminus}{\kern0pt}A{\isachardot}{\kern0pt}\ indep{\isacharunderscore}{\kern0pt}of{\isacharunderscore}{\kern0pt}alt\ n\ S\ a\ {\isasymand}\isanewline
\ \ \ \ \ \ \ \ \ \ \ \ \ \ {\isacharparenleft}{\kern0pt}{\isasymforall}p{\isachardot}{\kern0pt}\ finite{\isacharunderscore}{\kern0pt}profile\ S\ p\ {\isasymlongrightarrow}\ a\ {\isasymin}\ reject\ n\ S\ p{\isacharparenright}{\kern0pt}{\isacharparenright}{\kern0pt}{\isacharparenright}{\kern0pt}{\isachardoublequoteclose}\isanewline
\ \ \ \ \ \ \isacommand{using}\isamarkupfalse%
\ compatible\ disjoint{\isacharunderscore}{\kern0pt}compatibility{\isacharunderscore}{\kern0pt}def\isanewline
\ \ \ \ \ \ \isacommand{by}\isamarkupfalse%
\ fastforce\isanewline
\ \ \ \ \isacommand{hence}\isamarkupfalse%
\isanewline
\ \ \ \ \ \ {\isachardoublequoteopen}finite\ S\ {\isasymlongrightarrow}\isanewline
\ \ \ \ \ \ \ \ \ \ {\isacharparenleft}{\kern0pt}{\isasymexists}A\ {\isasymsubseteq}\ S{\isachardot}{\kern0pt}\isanewline
\ \ \ \ \ \ \ \ \ \ \ \ {\isacharparenleft}{\kern0pt}{\isasymforall}a\ {\isasymin}\ S{\isacharminus}{\kern0pt}A{\isachardot}{\kern0pt}\ indep{\isacharunderscore}{\kern0pt}of{\isacharunderscore}{\kern0pt}alt\ n\ S\ a\ {\isasymand}\isanewline
\ \ \ \ \ \ \ \ \ \ \ \ \ \ {\isacharparenleft}{\kern0pt}{\isasymforall}p{\isachardot}{\kern0pt}\ finite{\isacharunderscore}{\kern0pt}profile\ S\ p\ {\isasymlongrightarrow}\ a\ {\isasymin}\ reject\ n\ S\ p{\isacharparenright}{\kern0pt}{\isacharparenright}{\kern0pt}\ {\isasymand}\isanewline
\ \ \ \ \ \ \ \ \ \ \ \ {\isacharparenleft}{\kern0pt}{\isasymforall}a\ {\isasymin}\ A{\isachardot}{\kern0pt}\ indep{\isacharunderscore}{\kern0pt}of{\isacharunderscore}{\kern0pt}alt\ m\ S\ a\ {\isasymand}\isanewline
\ \ \ \ \ \ \ \ \ \ \ \ \ \ {\isacharparenleft}{\kern0pt}{\isasymforall}p{\isachardot}{\kern0pt}\ finite{\isacharunderscore}{\kern0pt}profile\ S\ p\ {\isasymlongrightarrow}\ a\ {\isasymin}\ reject\ m\ S\ p{\isacharparenright}{\kern0pt}{\isacharparenright}{\kern0pt}{\isacharparenright}{\kern0pt}{\isachardoublequoteclose}\isanewline
\ \ \ \ \ \ \isacommand{by}\isamarkupfalse%
\ auto\isanewline
\ \ \ \ \isacommand{hence}\isamarkupfalse%
\isanewline
\ \ \ \ \ \ {\isachardoublequoteopen}finite\ S\ {\isasymlongrightarrow}\isanewline
\ \ \ \ \ \ \ \ \ \ {\isacharparenleft}{\kern0pt}{\isasymexists}A\ {\isasymsubseteq}\ S{\isachardot}{\kern0pt}\isanewline
\ \ \ \ \ \ \ \ \ \ \ \ {\isacharparenleft}{\kern0pt}{\isasymforall}a\ {\isasymin}\ S{\isacharminus}{\kern0pt}A{\isachardot}{\kern0pt}\ indep{\isacharunderscore}{\kern0pt}of{\isacharunderscore}{\kern0pt}alt\ n\ S\ a\ {\isasymand}\isanewline
\ \ \ \ \ \ \ \ \ \ \ \ \ \ {\isacharparenleft}{\kern0pt}{\isasymforall}p{\isachardot}{\kern0pt}\ finite{\isacharunderscore}{\kern0pt}profile\ S\ p\ {\isasymlongrightarrow}\ a\ {\isasymin}\ reject\ n\ S\ p{\isacharparenright}{\kern0pt}{\isacharparenright}{\kern0pt}\ {\isasymand}\isanewline
\ \ \ \ \ \ \ \ \ \ \ \ {\isacharparenleft}{\kern0pt}{\isasymforall}a\ {\isasymin}\ S{\isacharminus}{\kern0pt}{\isacharparenleft}{\kern0pt}S{\isacharminus}{\kern0pt}A{\isacharparenright}{\kern0pt}{\isachardot}{\kern0pt}\ indep{\isacharunderscore}{\kern0pt}of{\isacharunderscore}{\kern0pt}alt\ m\ S\ a\ {\isasymand}\isanewline
\ \ \ \ \ \ \ \ \ \ \ \ \ \ {\isacharparenleft}{\kern0pt}{\isasymforall}p{\isachardot}{\kern0pt}\ finite{\isacharunderscore}{\kern0pt}profile\ S\ p\ {\isasymlongrightarrow}\ a\ {\isasymin}\ reject\ m\ S\ p{\isacharparenright}{\kern0pt}{\isacharparenright}{\kern0pt}{\isacharparenright}{\kern0pt}{\isachardoublequoteclose}\isanewline
\ \ \ \ \ \ \isacommand{using}\isamarkupfalse%
\ double{\isacharunderscore}{\kern0pt}diff\ order{\isacharunderscore}{\kern0pt}refl\isanewline
\ \ \ \ \ \ \isacommand{by}\isamarkupfalse%
\ metis\isanewline
\ \ \ \ \isacommand{thus}\isamarkupfalse%
\isanewline
\ \ \ \ \ \ {\isachardoublequoteopen}finite\ S\ {\isasymlongrightarrow}\isanewline
\ \ \ \ \ \ \ \ \ \ {\isacharparenleft}{\kern0pt}{\isasymexists}A\ {\isasymsubseteq}\ S{\isachardot}{\kern0pt}\isanewline
\ \ \ \ \ \ \ \ \ \ \ \ {\isacharparenleft}{\kern0pt}{\isasymforall}a\ {\isasymin}\ A{\isachardot}{\kern0pt}\ indep{\isacharunderscore}{\kern0pt}of{\isacharunderscore}{\kern0pt}alt\ n\ S\ a\ {\isasymand}\isanewline
\ \ \ \ \ \ \ \ \ \ \ \ \ \ {\isacharparenleft}{\kern0pt}{\isasymforall}p{\isachardot}{\kern0pt}\ finite{\isacharunderscore}{\kern0pt}profile\ S\ p\ {\isasymlongrightarrow}\ a\ {\isasymin}\ reject\ n\ S\ p{\isacharparenright}{\kern0pt}{\isacharparenright}{\kern0pt}\ {\isasymand}\isanewline
\ \ \ \ \ \ \ \ \ \ \ \ {\isacharparenleft}{\kern0pt}{\isasymforall}a\ {\isasymin}\ S{\isacharminus}{\kern0pt}A{\isachardot}{\kern0pt}\ indep{\isacharunderscore}{\kern0pt}of{\isacharunderscore}{\kern0pt}alt\ m\ S\ a\ {\isasymand}\isanewline
\ \ \ \ \ \ \ \ \ \ \ \ \ \ {\isacharparenleft}{\kern0pt}{\isasymforall}p{\isachardot}{\kern0pt}\ finite{\isacharunderscore}{\kern0pt}profile\ S\ p\ {\isasymlongrightarrow}\ a\ {\isasymin}\ reject\ m\ S\ p{\isacharparenright}{\kern0pt}{\isacharparenright}{\kern0pt}{\isacharparenright}{\kern0pt}{\isachardoublequoteclose}\isanewline
\ \ \ \ \ \ \isacommand{by}\isamarkupfalse%
\ fastforce\isanewline
\ \ \isacommand{qed}\isamarkupfalse%
\isanewline
\ \ \isacommand{moreover}\isamarkupfalse%
\ \isacommand{have}\isamarkupfalse%
\ {\isachardoublequoteopen}electoral{\isacharunderscore}{\kern0pt}module\ m\ {\isasymand}\ electoral{\isacharunderscore}{\kern0pt}module\ n{\isachardoublequoteclose}\isanewline
\ \ \ \ \isacommand{using}\isamarkupfalse%
\ compatible\ disjoint{\isacharunderscore}{\kern0pt}compatibility{\isacharunderscore}{\kern0pt}def\isanewline
\ \ \ \ \isacommand{by}\isamarkupfalse%
\ auto\isanewline
\ \ \isacommand{ultimately}\isamarkupfalse%
\ \isacommand{show}\isamarkupfalse%
\ {\isacharquery}{\kern0pt}thesis\isanewline
\ \ \ \ \isacommand{by}\isamarkupfalse%
\ {\isacharparenleft}{\kern0pt}simp\ add{\isacharcolon}{\kern0pt}\ disjoint{\isacharunderscore}{\kern0pt}compatibility{\isacharunderscore}{\kern0pt}def{\isacharparenright}{\kern0pt}\isanewline
\isacommand{qed}\isamarkupfalse%
%
\endisatagproof
{\isafoldproof}%
%
\isadelimproof
\isanewline
%
\endisadelimproof
\isanewline
\isanewline
\isacommand{theorem}\isamarkupfalse%
\ disj{\isacharunderscore}{\kern0pt}compat{\isacharunderscore}{\kern0pt}seq{\isacharbrackleft}{\kern0pt}simp{\isacharbrackright}{\kern0pt}{\isacharcolon}{\kern0pt}\isanewline
\ \ \isakeyword{assumes}\isanewline
\ \ \ \ compatible{\isacharcolon}{\kern0pt}\ {\isachardoublequoteopen}disjoint{\isacharunderscore}{\kern0pt}compatibility\ m\ n{\isachardoublequoteclose}\ \isakeyword{and}\isanewline
\ \ \ \ module{\isacharunderscore}{\kern0pt}m{\isadigit{2}}{\isacharcolon}{\kern0pt}\ {\isachardoublequoteopen}electoral{\isacharunderscore}{\kern0pt}module\ m{\isadigit{2}}{\isachardoublequoteclose}\isanewline
\ \ \isakeyword{shows}\ {\isachardoublequoteopen}disjoint{\isacharunderscore}{\kern0pt}compatibility\ {\isacharparenleft}{\kern0pt}m\ {\isasymtriangleright}\ m{\isadigit{2}}{\isacharparenright}{\kern0pt}\ n{\isachardoublequoteclose}\isanewline
%
\isadelimproof
\ \ %
\endisadelimproof
%
\isatagproof
\isacommand{unfolding}\isamarkupfalse%
\ disjoint{\isacharunderscore}{\kern0pt}compatibility{\isacharunderscore}{\kern0pt}def\isanewline
\isacommand{proof}\isamarkupfalse%
\ {\isacharparenleft}{\kern0pt}safe{\isacharparenright}{\kern0pt}\isanewline
\ \ \isacommand{show}\isamarkupfalse%
\ {\isachardoublequoteopen}electoral{\isacharunderscore}{\kern0pt}module\ {\isacharparenleft}{\kern0pt}m\ {\isasymtriangleright}\ m{\isadigit{2}}{\isacharparenright}{\kern0pt}{\isachardoublequoteclose}\isanewline
\ \ \ \ \isacommand{using}\isamarkupfalse%
\ compatible\ disjoint{\isacharunderscore}{\kern0pt}compatibility{\isacharunderscore}{\kern0pt}def\ module{\isacharunderscore}{\kern0pt}m{\isadigit{2}}\ seq{\isacharunderscore}{\kern0pt}comp{\isacharunderscore}{\kern0pt}sound\isanewline
\ \ \ \ \isacommand{by}\isamarkupfalse%
\ metis\isanewline
\isacommand{next}\isamarkupfalse%
\isanewline
\ \ \isacommand{show}\isamarkupfalse%
\ {\isachardoublequoteopen}electoral{\isacharunderscore}{\kern0pt}module\ n{\isachardoublequoteclose}\isanewline
\ \ \ \ \isacommand{using}\isamarkupfalse%
\ compatible\ disjoint{\isacharunderscore}{\kern0pt}compatibility{\isacharunderscore}{\kern0pt}def\isanewline
\ \ \ \ \isacommand{by}\isamarkupfalse%
\ metis\isanewline
\isacommand{next}\isamarkupfalse%
\isanewline
\ \ \isacommand{fix}\isamarkupfalse%
\isanewline
\ \ \ \ S\ {\isacharcolon}{\kern0pt}{\isacharcolon}{\kern0pt}\ {\isachardoublequoteopen}{\isacharprime}{\kern0pt}a\ set{\isachardoublequoteclose}\isanewline
\ \ \isacommand{assume}\isamarkupfalse%
\isanewline
\ \ \ \ fin{\isacharunderscore}{\kern0pt}S{\isacharcolon}{\kern0pt}\ {\isachardoublequoteopen}finite\ S{\isachardoublequoteclose}\isanewline
\ \ \isacommand{have}\isamarkupfalse%
\ modules{\isacharcolon}{\kern0pt}\isanewline
\ \ \ \ {\isachardoublequoteopen}electoral{\isacharunderscore}{\kern0pt}module\ {\isacharparenleft}{\kern0pt}m\ {\isasymtriangleright}\ m{\isadigit{2}}{\isacharparenright}{\kern0pt}\ {\isasymand}\ electoral{\isacharunderscore}{\kern0pt}module\ n{\isachardoublequoteclose}\isanewline
\ \ \ \ \isacommand{using}\isamarkupfalse%
\ compatible\ disjoint{\isacharunderscore}{\kern0pt}compatibility{\isacharunderscore}{\kern0pt}def\ module{\isacharunderscore}{\kern0pt}m{\isadigit{2}}\ seq{\isacharunderscore}{\kern0pt}comp{\isacharunderscore}{\kern0pt}sound\isanewline
\ \ \ \ \isacommand{by}\isamarkupfalse%
\ metis\isanewline
\ \ \isacommand{obtain}\isamarkupfalse%
\ A\ \isakeyword{where}\ A{\isacharcolon}{\kern0pt}\isanewline
\ \ \ \ {\isachardoublequoteopen}A\ {\isasymsubseteq}\ S\ {\isasymand}\isanewline
\ \ \ \ \ \ {\isacharparenleft}{\kern0pt}{\isasymforall}a\ {\isasymin}\ A{\isachardot}{\kern0pt}\ indep{\isacharunderscore}{\kern0pt}of{\isacharunderscore}{\kern0pt}alt\ m\ S\ a\ {\isasymand}\isanewline
\ \ \ \ \ \ \ \ {\isacharparenleft}{\kern0pt}{\isasymforall}p{\isachardot}{\kern0pt}\ finite{\isacharunderscore}{\kern0pt}profile\ S\ p\ {\isasymlongrightarrow}\ a\ {\isasymin}\ reject\ m\ S\ p{\isacharparenright}{\kern0pt}{\isacharparenright}{\kern0pt}\ {\isasymand}\isanewline
\ \ \ \ \ \ {\isacharparenleft}{\kern0pt}{\isasymforall}a\ {\isasymin}\ S{\isacharminus}{\kern0pt}A{\isachardot}{\kern0pt}\ indep{\isacharunderscore}{\kern0pt}of{\isacharunderscore}{\kern0pt}alt\ n\ S\ a\ {\isasymand}\isanewline
\ \ \ \ \ \ \ \ {\isacharparenleft}{\kern0pt}{\isasymforall}p{\isachardot}{\kern0pt}\ finite{\isacharunderscore}{\kern0pt}profile\ S\ p\ {\isasymlongrightarrow}\ a\ {\isasymin}\ reject\ n\ S\ p{\isacharparenright}{\kern0pt}{\isacharparenright}{\kern0pt}{\isachardoublequoteclose}\isanewline
\ \ \ \ \isacommand{using}\isamarkupfalse%
\ compatible\ disjoint{\isacharunderscore}{\kern0pt}compatibility{\isacharunderscore}{\kern0pt}def\ fin{\isacharunderscore}{\kern0pt}S\isanewline
\ \ \ \ \isacommand{by}\isamarkupfalse%
\ {\isacharparenleft}{\kern0pt}metis\ {\isacharparenleft}{\kern0pt}no{\isacharunderscore}{\kern0pt}types{\isacharcomma}{\kern0pt}\ lifting{\isacharparenright}{\kern0pt}{\isacharparenright}{\kern0pt}\isanewline
\ \ \isacommand{show}\isamarkupfalse%
\isanewline
\ \ \ \ {\isachardoublequoteopen}{\isasymexists}A\ {\isasymsubseteq}\ S{\isachardot}{\kern0pt}\isanewline
\ \ \ \ \ \ {\isacharparenleft}{\kern0pt}{\isasymforall}a\ {\isasymin}\ A{\isachardot}{\kern0pt}\ indep{\isacharunderscore}{\kern0pt}of{\isacharunderscore}{\kern0pt}alt\ {\isacharparenleft}{\kern0pt}m\ {\isasymtriangleright}\ m{\isadigit{2}}{\isacharparenright}{\kern0pt}\ S\ a\ {\isasymand}\isanewline
\ \ \ \ \ \ \ \ {\isacharparenleft}{\kern0pt}{\isasymforall}p{\isachardot}{\kern0pt}\ finite{\isacharunderscore}{\kern0pt}profile\ S\ p\ {\isasymlongrightarrow}\ a\ {\isasymin}\ reject\ {\isacharparenleft}{\kern0pt}m\ {\isasymtriangleright}\ m{\isadigit{2}}{\isacharparenright}{\kern0pt}\ S\ p{\isacharparenright}{\kern0pt}{\isacharparenright}{\kern0pt}\ {\isasymand}\isanewline
\ \ \ \ \ \ {\isacharparenleft}{\kern0pt}{\isasymforall}a\ {\isasymin}\ S{\isacharminus}{\kern0pt}A{\isachardot}{\kern0pt}\ indep{\isacharunderscore}{\kern0pt}of{\isacharunderscore}{\kern0pt}alt\ n\ S\ a\ {\isasymand}\isanewline
\ \ \ \ \ \ \ \ {\isacharparenleft}{\kern0pt}{\isasymforall}p{\isachardot}{\kern0pt}\ finite{\isacharunderscore}{\kern0pt}profile\ S\ p\ {\isasymlongrightarrow}\ a\ {\isasymin}\ reject\ n\ S\ p{\isacharparenright}{\kern0pt}{\isacharparenright}{\kern0pt}{\isachardoublequoteclose}\isanewline
\ \ \isacommand{proof}\isamarkupfalse%
\isanewline
\ \ \ \ \isacommand{have}\isamarkupfalse%
\isanewline
\ \ \ \ \ \ {\isachardoublequoteopen}{\isasymforall}a\ p\ q{\isachardot}{\kern0pt}\isanewline
\ \ \ \ \ \ \ \ a\ {\isasymin}\ A\ {\isasymand}\ equiv{\isacharunderscore}{\kern0pt}prof{\isacharunderscore}{\kern0pt}except{\isacharunderscore}{\kern0pt}a\ S\ p\ q\ a\ {\isasymlongrightarrow}\isanewline
\ \ \ \ \ \ \ \ \ \ {\isacharparenleft}{\kern0pt}m\ {\isasymtriangleright}\ m{\isadigit{2}}{\isacharparenright}{\kern0pt}\ S\ p\ {\isacharequal}{\kern0pt}\ {\isacharparenleft}{\kern0pt}m\ {\isasymtriangleright}\ m{\isadigit{2}}{\isacharparenright}{\kern0pt}\ S\ q{\isachardoublequoteclose}\isanewline
\ \ \ \ \isacommand{proof}\isamarkupfalse%
\ {\isacharparenleft}{\kern0pt}safe{\isacharparenright}{\kern0pt}\isanewline
\ \ \ \ \ \ \isacommand{fix}\isamarkupfalse%
\isanewline
\ \ \ \ \ \ \ \ a\ {\isacharcolon}{\kern0pt}{\isacharcolon}{\kern0pt}\ {\isachardoublequoteopen}{\isacharprime}{\kern0pt}a{\isachardoublequoteclose}\ \isakeyword{and}\isanewline
\ \ \ \ \ \ \ \ p\ {\isacharcolon}{\kern0pt}{\isacharcolon}{\kern0pt}\ {\isachardoublequoteopen}{\isacharprime}{\kern0pt}a\ Profile{\isachardoublequoteclose}\ \isakeyword{and}\isanewline
\ \ \ \ \ \ \ \ q\ {\isacharcolon}{\kern0pt}{\isacharcolon}{\kern0pt}\ {\isachardoublequoteopen}{\isacharprime}{\kern0pt}a\ Profile{\isachardoublequoteclose}\isanewline
\ \ \ \ \ \ \isacommand{assume}\isamarkupfalse%
\isanewline
\ \ \ \ \ \ \ \ a{\isacharcolon}{\kern0pt}\ {\isachardoublequoteopen}a\ {\isasymin}\ A{\isachardoublequoteclose}\ \isakeyword{and}\isanewline
\ \ \ \ \ \ \ \ b{\isacharcolon}{\kern0pt}\ {\isachardoublequoteopen}equiv{\isacharunderscore}{\kern0pt}prof{\isacharunderscore}{\kern0pt}except{\isacharunderscore}{\kern0pt}a\ S\ p\ q\ a{\isachardoublequoteclose}\isanewline
\ \ \ \ \ \ \isacommand{have}\isamarkupfalse%
\ eq{\isacharunderscore}{\kern0pt}def{\isacharcolon}{\kern0pt}\isanewline
\ \ \ \ \ \ \ \ {\isachardoublequoteopen}defer\ m\ S\ p\ {\isacharequal}{\kern0pt}\ defer\ m\ S\ q{\isachardoublequoteclose}\isanewline
\ \ \ \ \ \ \ \ \isacommand{using}\isamarkupfalse%
\ A\ a\ b\ indep{\isacharunderscore}{\kern0pt}of{\isacharunderscore}{\kern0pt}alt{\isacharunderscore}{\kern0pt}def\isanewline
\ \ \ \ \ \ \ \ \isacommand{by}\isamarkupfalse%
\ metis\isanewline
\ \ \ \ \ \ \isacommand{from}\isamarkupfalse%
\ a\ b\ \isacommand{have}\isamarkupfalse%
\ profiles{\isacharcolon}{\kern0pt}\isanewline
\ \ \ \ \ \ \ \ {\isachardoublequoteopen}finite{\isacharunderscore}{\kern0pt}profile\ S\ p\ {\isasymand}\ finite{\isacharunderscore}{\kern0pt}profile\ S\ q{\isachardoublequoteclose}\isanewline
\ \ \ \ \ \ \ \ \isacommand{using}\isamarkupfalse%
\ equiv{\isacharunderscore}{\kern0pt}prof{\isacharunderscore}{\kern0pt}except{\isacharunderscore}{\kern0pt}a{\isacharunderscore}{\kern0pt}def\isanewline
\ \ \ \ \ \ \ \ \isacommand{by}\isamarkupfalse%
\ fastforce\isanewline
\ \ \ \ \ \ \isacommand{hence}\isamarkupfalse%
\ {\isachardoublequoteopen}{\isacharparenleft}{\kern0pt}defer\ m\ S\ p{\isacharparenright}{\kern0pt}\ {\isasymsubseteq}\ S{\isachardoublequoteclose}\isanewline
\ \ \ \ \ \ \ \ \isacommand{using}\isamarkupfalse%
\ compatible\ defer{\isacharunderscore}{\kern0pt}in{\isacharunderscore}{\kern0pt}alts\ disjoint{\isacharunderscore}{\kern0pt}compatibility{\isacharunderscore}{\kern0pt}def\isanewline
\ \ \ \ \ \ \ \ \isacommand{by}\isamarkupfalse%
\ blast\isanewline
\ \ \ \ \ \ \isacommand{hence}\isamarkupfalse%
\isanewline
\ \ \ \ \ \ \ \ {\isachardoublequoteopen}limit{\isacharunderscore}{\kern0pt}profile\ {\isacharparenleft}{\kern0pt}defer\ m\ S\ p{\isacharparenright}{\kern0pt}\ p\ {\isacharequal}{\kern0pt}\isanewline
\ \ \ \ \ \ \ \ \ \ limit{\isacharunderscore}{\kern0pt}profile\ {\isacharparenleft}{\kern0pt}defer\ m\ S\ q{\isacharparenright}{\kern0pt}\ q{\isachardoublequoteclose}\isanewline
\ \ \ \ \ \ \ \ \isacommand{using}\isamarkupfalse%
\ A\ DiffD{\isadigit{2}}\ a\ b\ compatible\ defer{\isacharunderscore}{\kern0pt}not{\isacharunderscore}{\kern0pt}elec{\isacharunderscore}{\kern0pt}or{\isacharunderscore}{\kern0pt}rej\isanewline
\ \ \ \ \ \ \ \ \ \ \ \ \ \ disjoint{\isacharunderscore}{\kern0pt}compatibility{\isacharunderscore}{\kern0pt}def\ eq{\isacharunderscore}{\kern0pt}def\ profiles\isanewline
\ \ \ \ \ \ \ \ \ \ \ \ \ \ negl{\isacharunderscore}{\kern0pt}diff{\isacharunderscore}{\kern0pt}imp{\isacharunderscore}{\kern0pt}eq{\isacharunderscore}{\kern0pt}limit{\isacharunderscore}{\kern0pt}prof\isanewline
\ \ \ \ \ \ \ \ \isacommand{by}\isamarkupfalse%
\ {\isacharparenleft}{\kern0pt}metis\ {\isacharparenleft}{\kern0pt}no{\isacharunderscore}{\kern0pt}types{\isacharcomma}{\kern0pt}\ lifting{\isacharparenright}{\kern0pt}{\isacharparenright}{\kern0pt}\isanewline
\ \ \ \ \ \ \isacommand{with}\isamarkupfalse%
\ eq{\isacharunderscore}{\kern0pt}def\ \isacommand{have}\isamarkupfalse%
\isanewline
\ \ \ \ \ \ \ \ {\isachardoublequoteopen}m{\isadigit{2}}\ {\isacharparenleft}{\kern0pt}defer\ m\ S\ p{\isacharparenright}{\kern0pt}\ {\isacharparenleft}{\kern0pt}limit{\isacharunderscore}{\kern0pt}profile\ {\isacharparenleft}{\kern0pt}defer\ m\ S\ p{\isacharparenright}{\kern0pt}\ p{\isacharparenright}{\kern0pt}\ {\isacharequal}{\kern0pt}\isanewline
\ \ \ \ \ \ \ \ \ \ m{\isadigit{2}}\ {\isacharparenleft}{\kern0pt}defer\ m\ S\ q{\isacharparenright}{\kern0pt}\ {\isacharparenleft}{\kern0pt}limit{\isacharunderscore}{\kern0pt}profile\ {\isacharparenleft}{\kern0pt}defer\ m\ S\ q{\isacharparenright}{\kern0pt}\ q{\isacharparenright}{\kern0pt}{\isachardoublequoteclose}\isanewline
\ \ \ \ \ \ \ \ \isacommand{by}\isamarkupfalse%
\ simp\isanewline
\ \ \ \ \ \ \isacommand{moreover}\isamarkupfalse%
\ \isacommand{have}\isamarkupfalse%
\ {\isachardoublequoteopen}m\ S\ p\ {\isacharequal}{\kern0pt}\ m\ S\ q{\isachardoublequoteclose}\isanewline
\ \ \ \ \ \ \ \ \isacommand{using}\isamarkupfalse%
\ A\ a\ b\ indep{\isacharunderscore}{\kern0pt}of{\isacharunderscore}{\kern0pt}alt{\isacharunderscore}{\kern0pt}def\isanewline
\ \ \ \ \ \ \ \ \isacommand{by}\isamarkupfalse%
\ metis\isanewline
\ \ \ \ \ \ \isacommand{ultimately}\isamarkupfalse%
\ \isacommand{show}\isamarkupfalse%
\isanewline
\ \ \ \ \ \ \ \ {\isachardoublequoteopen}{\isacharparenleft}{\kern0pt}m\ {\isasymtriangleright}\ m{\isadigit{2}}{\isacharparenright}{\kern0pt}\ S\ p\ {\isacharequal}{\kern0pt}\ {\isacharparenleft}{\kern0pt}m\ {\isasymtriangleright}\ m{\isadigit{2}}{\isacharparenright}{\kern0pt}\ S\ q{\isachardoublequoteclose}\isanewline
\ \ \ \ \ \ \ \ \isacommand{using}\isamarkupfalse%
\ sequential{\isacharunderscore}{\kern0pt}composition{\isachardot}{\kern0pt}simps\isanewline
\ \ \ \ \ \ \ \ \isacommand{by}\isamarkupfalse%
\ {\isacharparenleft}{\kern0pt}metis\ {\isacharparenleft}{\kern0pt}full{\isacharunderscore}{\kern0pt}types{\isacharparenright}{\kern0pt}{\isacharparenright}{\kern0pt}\isanewline
\ \ \ \ \isacommand{qed}\isamarkupfalse%
\isanewline
\ \ \ \ \isacommand{moreover}\isamarkupfalse%
\ \isacommand{have}\isamarkupfalse%
\isanewline
\ \ \ \ \ \ {\isachardoublequoteopen}{\isasymforall}a\ {\isasymin}\ A{\isachardot}{\kern0pt}\ {\isasymforall}p{\isachardot}{\kern0pt}\ finite{\isacharunderscore}{\kern0pt}profile\ S\ p\ {\isasymlongrightarrow}\ a\ {\isasymin}\ reject\ {\isacharparenleft}{\kern0pt}m\ {\isasymtriangleright}\ m{\isadigit{2}}{\isacharparenright}{\kern0pt}\ S\ p{\isachardoublequoteclose}\isanewline
\ \ \ \ \ \ \isacommand{using}\isamarkupfalse%
\ A\ UnI{\isadigit{1}}\ prod{\isachardot}{\kern0pt}sel\ sequential{\isacharunderscore}{\kern0pt}composition{\isachardot}{\kern0pt}simps\isanewline
\ \ \ \ \ \ \isacommand{by}\isamarkupfalse%
\ metis\isanewline
\ \ \ \ \isacommand{ultimately}\isamarkupfalse%
\ \isacommand{show}\isamarkupfalse%
\isanewline
\ \ \ \ \ \ {\isachardoublequoteopen}A\ {\isasymsubseteq}\ S\ {\isasymand}\isanewline
\ \ \ \ \ \ \ \ {\isacharparenleft}{\kern0pt}{\isasymforall}a\ {\isasymin}\ A{\isachardot}{\kern0pt}\ indep{\isacharunderscore}{\kern0pt}of{\isacharunderscore}{\kern0pt}alt\ {\isacharparenleft}{\kern0pt}m\ {\isasymtriangleright}\ m{\isadigit{2}}{\isacharparenright}{\kern0pt}\ S\ a\ {\isasymand}\isanewline
\ \ \ \ \ \ \ \ \ \ {\isacharparenleft}{\kern0pt}{\isasymforall}p{\isachardot}{\kern0pt}\ finite{\isacharunderscore}{\kern0pt}profile\ S\ p\ {\isasymlongrightarrow}\ a\ {\isasymin}\ reject\ {\isacharparenleft}{\kern0pt}m\ {\isasymtriangleright}\ m{\isadigit{2}}{\isacharparenright}{\kern0pt}\ S\ p{\isacharparenright}{\kern0pt}{\isacharparenright}{\kern0pt}\ {\isasymand}\isanewline
\ \ \ \ \ \ \ \ {\isacharparenleft}{\kern0pt}{\isasymforall}a\ {\isasymin}\ S{\isacharminus}{\kern0pt}A{\isachardot}{\kern0pt}\ indep{\isacharunderscore}{\kern0pt}of{\isacharunderscore}{\kern0pt}alt\ n\ S\ a\ {\isasymand}\isanewline
\ \ \ \ \ \ \ \ \ \ {\isacharparenleft}{\kern0pt}{\isasymforall}p{\isachardot}{\kern0pt}\ finite{\isacharunderscore}{\kern0pt}profile\ S\ p\ {\isasymlongrightarrow}\ a\ {\isasymin}\ reject\ n\ S\ p{\isacharparenright}{\kern0pt}{\isacharparenright}{\kern0pt}{\isachardoublequoteclose}\isanewline
\ \ \ \ \ \ \isacommand{using}\isamarkupfalse%
\ A\ indep{\isacharunderscore}{\kern0pt}of{\isacharunderscore}{\kern0pt}alt{\isacharunderscore}{\kern0pt}def\ modules\isanewline
\ \ \ \ \ \ \isacommand{by}\isamarkupfalse%
\ {\isacharparenleft}{\kern0pt}metis\ {\isacharparenleft}{\kern0pt}mono{\isacharunderscore}{\kern0pt}tags{\isacharcomma}{\kern0pt}\ lifting{\isacharparenright}{\kern0pt}{\isacharparenright}{\kern0pt}\isanewline
\ \ \isacommand{qed}\isamarkupfalse%
\isanewline
\isacommand{qed}\isamarkupfalse%
%
\endisatagproof
{\isafoldproof}%
%
\isadelimproof
\isanewline
%
\endisadelimproof
%
\isadelimtheory
\isanewline
%
\endisadelimtheory
%
\isatagtheory
\isacommand{end}\isamarkupfalse%
%
\endisatagtheory
{\isafoldtheory}%
%
\isadelimtheory
%
\endisadelimtheory
%
\end{isabellebody}%
\endinput
%:%file=~/Documents/Studies/VotingRuleGenerator/virage/src/test/resources/verifiedVotingRuleConstruction/theories/Compositional_Framework/Composition_Rules/Disjoint_Compatibility_Rules.thy%:%
%:%10=1%:%
%:%11=1%:%
%:%12=2%:%
%:%13=3%:%
%:%14=4%:%
%:%15=5%:%
%:%20=5%:%
%:%23=6%:%
%:%24=7%:%
%:%25=8%:%
%:%26=8%:%
%:%27=9%:%
%:%28=10%:%
%:%35=11%:%
%:%36=11%:%
%:%37=12%:%
%:%38=12%:%
%:%39=13%:%
%:%44=18%:%
%:%45=19%:%
%:%46=19%:%
%:%47=20%:%
%:%48=20%:%
%:%49=21%:%
%:%50=22%:%
%:%51=22%:%
%:%52=23%:%
%:%57=28%:%
%:%58=29%:%
%:%59=29%:%
%:%60=30%:%
%:%61=30%:%
%:%62=31%:%
%:%63=31%:%
%:%64=32%:%
%:%69=37%:%
%:%70=38%:%
%:%71=38%:%
%:%72=39%:%
%:%73=39%:%
%:%74=40%:%
%:%79=45%:%
%:%80=46%:%
%:%81=46%:%
%:%82=47%:%
%:%83=47%:%
%:%84=48%:%
%:%85=48%:%
%:%86=49%:%
%:%91=54%:%
%:%92=55%:%
%:%93=55%:%
%:%94=56%:%
%:%95=56%:%
%:%96=57%:%
%:%97=57%:%
%:%98=57%:%
%:%99=58%:%
%:%100=58%:%
%:%101=59%:%
%:%102=59%:%
%:%103=60%:%
%:%104=60%:%
%:%105=60%:%
%:%106=61%:%
%:%107=61%:%
%:%108=62%:%
%:%114=62%:%
%:%117=63%:%
%:%118=67%:%
%:%119=68%:%
%:%120=68%:%
%:%121=69%:%
%:%122=70%:%
%:%123=71%:%
%:%124=72%:%
%:%127=73%:%
%:%131=73%:%
%:%132=73%:%
%:%133=74%:%
%:%134=74%:%
%:%135=75%:%
%:%136=75%:%
%:%137=76%:%
%:%138=76%:%
%:%139=77%:%
%:%140=77%:%
%:%141=78%:%
%:%142=78%:%
%:%143=79%:%
%:%144=79%:%
%:%145=80%:%
%:%146=80%:%
%:%147=81%:%
%:%148=81%:%
%:%149=82%:%
%:%150=82%:%
%:%151=83%:%
%:%152=83%:%
%:%153=84%:%
%:%154=85%:%
%:%155=85%:%
%:%156=86%:%
%:%157=87%:%
%:%158=87%:%
%:%159=88%:%
%:%160=89%:%
%:%161=89%:%
%:%162=90%:%
%:%163=90%:%
%:%164=91%:%
%:%165=91%:%
%:%166=92%:%
%:%170=96%:%
%:%171=97%:%
%:%172=97%:%
%:%173=98%:%
%:%174=98%:%
%:%175=99%:%
%:%176=99%:%
%:%177=100%:%
%:%181=104%:%
%:%182=105%:%
%:%183=105%:%
%:%184=106%:%
%:%185=106%:%
%:%186=107%:%
%:%188=109%:%
%:%189=110%:%
%:%190=110%:%
%:%191=111%:%
%:%192=111%:%
%:%193=112%:%
%:%194=113%:%
%:%195=114%:%
%:%196=115%:%
%:%197=115%:%
%:%198=116%:%
%:%199=117%:%
%:%200=118%:%
%:%201=118%:%
%:%202=119%:%
%:%203=120%:%
%:%204=120%:%
%:%205=121%:%
%:%206=121%:%
%:%207=122%:%
%:%208=122%:%
%:%209=122%:%
%:%210=123%:%
%:%211=124%:%
%:%212=124%:%
%:%213=125%:%
%:%214=125%:%
%:%215=126%:%
%:%216=126%:%
%:%217=127%:%
%:%218=127%:%
%:%219=128%:%
%:%220=128%:%
%:%221=129%:%
%:%222=129%:%
%:%223=130%:%
%:%224=131%:%
%:%225=132%:%
%:%226=132%:%
%:%227=133%:%
%:%228=134%:%
%:%229=135%:%
%:%230=135%:%
%:%231=136%:%
%:%232=136%:%
%:%233=136%:%
%:%234=137%:%
%:%235=138%:%
%:%236=139%:%
%:%237=139%:%
%:%238=140%:%
%:%239=140%:%
%:%240=140%:%
%:%241=141%:%
%:%242=141%:%
%:%243=142%:%
%:%244=142%:%
%:%245=143%:%
%:%246=143%:%
%:%247=143%:%
%:%248=144%:%
%:%249=145%:%
%:%250=145%:%
%:%251=146%:%
%:%252=146%:%
%:%253=147%:%
%:%254=147%:%
%:%255=148%:%
%:%256=148%:%
%:%257=148%:%
%:%258=149%:%
%:%259=150%:%
%:%260=150%:%
%:%261=151%:%
%:%262=151%:%
%:%263=152%:%
%:%264=152%:%
%:%265=152%:%
%:%266=153%:%
%:%270=157%:%
%:%271=158%:%
%:%272=158%:%
%:%273=159%:%
%:%274=159%:%
%:%275=160%:%
%:%276=160%:%
%:%277=161%:%
%:%283=161%:%
%:%288=162%:%
%:%293=163%:%
%
\begin{isabellebody}%
\setisabellecontext{Sequential{\isacharunderscore}{\kern0pt}Majority{\isacharunderscore}{\kern0pt}Comparison}%
%
\isadelimdocument
\isanewline
%
\endisadelimdocument
%
\isatagdocument
\isanewline
\isanewline
%
\isamarkupsection{Sequential Majority Comparison%
}
\isamarkuptrue%
%
\endisatagdocument
{\isafolddocument}%
%
\isadelimdocument
%
\endisadelimdocument
%
\isadelimtheory
%
\endisadelimtheory
%
\isatagtheory
\isacommand{theory}\isamarkupfalse%
\ Sequential{\isacharunderscore}{\kern0pt}Majority{\isacharunderscore}{\kern0pt}Comparison\isanewline
\ \ \isakeyword{imports}\ {\isachardoublequoteopen}{\isachardot}{\kern0pt}{\isachardot}{\kern0pt}{\isacharslash}{\kern0pt}Compositional{\isacharunderscore}{\kern0pt}Framework{\isacharslash}{\kern0pt}Components{\isacharslash}{\kern0pt}Basic{\isacharunderscore}{\kern0pt}Modules{\isacharslash}{\kern0pt}Plurality{\isacharunderscore}{\kern0pt}Module{\isachardoublequoteclose}\isanewline
\ \ \ \ \ \ \ \ \ \ {\isachardoublequoteopen}{\isachardot}{\kern0pt}{\isachardot}{\kern0pt}{\isacharslash}{\kern0pt}Compositional{\isacharunderscore}{\kern0pt}Framework{\isacharslash}{\kern0pt}Components{\isacharslash}{\kern0pt}Basic{\isacharunderscore}{\kern0pt}Modules{\isacharslash}{\kern0pt}Pass{\isacharunderscore}{\kern0pt}Module{\isachardoublequoteclose}\isanewline
\ \ \ \ \ \ \ \ \ \ {\isachardoublequoteopen}{\isachardot}{\kern0pt}{\isachardot}{\kern0pt}{\isacharslash}{\kern0pt}Compositional{\isacharunderscore}{\kern0pt}Framework{\isacharslash}{\kern0pt}Components{\isacharslash}{\kern0pt}Basic{\isacharunderscore}{\kern0pt}Modules{\isacharslash}{\kern0pt}Drop{\isacharunderscore}{\kern0pt}Module{\isachardoublequoteclose}\isanewline
\ \ \ \ \ \ \ \ \ \ {\isachardoublequoteopen}{\isachardot}{\kern0pt}{\isachardot}{\kern0pt}{\isacharslash}{\kern0pt}Compositional{\isacharunderscore}{\kern0pt}Framework{\isacharslash}{\kern0pt}Components{\isacharslash}{\kern0pt}Compositional{\isacharunderscore}{\kern0pt}Structures{\isacharslash}{\kern0pt}Revision{\isacharunderscore}{\kern0pt}Composition{\isachardoublequoteclose}\isanewline
\ \ \ \ \ \ \ \ \ \ {\isachardoublequoteopen}{\isachardot}{\kern0pt}{\isachardot}{\kern0pt}{\isacharslash}{\kern0pt}Compositional{\isacharunderscore}{\kern0pt}Framework{\isacharslash}{\kern0pt}Components{\isacharslash}{\kern0pt}Composites{\isacharslash}{\kern0pt}Composite{\isacharunderscore}{\kern0pt}Structures{\isachardoublequoteclose}\isanewline
\ \ \ \ \ \ \ \ \ \ {\isachardoublequoteopen}{\isachardot}{\kern0pt}{\isachardot}{\kern0pt}{\isacharslash}{\kern0pt}Compositional{\isacharunderscore}{\kern0pt}Framework{\isacharslash}{\kern0pt}Composition{\isacharunderscore}{\kern0pt}Rules{\isacharslash}{\kern0pt}Monotonicity{\isacharunderscore}{\kern0pt}Rules{\isachardoublequoteclose}\isanewline
\ \ \ \ \ \ \ \ \ \ {\isachardoublequoteopen}{\isachardot}{\kern0pt}{\isachardot}{\kern0pt}{\isacharslash}{\kern0pt}Compositional{\isacharunderscore}{\kern0pt}Framework{\isacharslash}{\kern0pt}Composition{\isacharunderscore}{\kern0pt}Rules{\isacharslash}{\kern0pt}Result{\isacharunderscore}{\kern0pt}Rules{\isachardoublequoteclose}\isanewline
\ \ \ \ \ \ \ \ \ \ {\isachardoublequoteopen}{\isachardot}{\kern0pt}{\isachardot}{\kern0pt}{\isacharslash}{\kern0pt}Compositional{\isacharunderscore}{\kern0pt}Framework{\isacharslash}{\kern0pt}Composition{\isacharunderscore}{\kern0pt}Rules{\isacharslash}{\kern0pt}Disjoint{\isacharunderscore}{\kern0pt}Compatibility{\isacharunderscore}{\kern0pt}Facts{\isachardoublequoteclose}\isanewline
\ \ \ \ \ \ \ \ \ \ {\isachardoublequoteopen}{\isachardot}{\kern0pt}{\isachardot}{\kern0pt}{\isacharslash}{\kern0pt}Compositional{\isacharunderscore}{\kern0pt}Framework{\isacharslash}{\kern0pt}Composition{\isacharunderscore}{\kern0pt}Rules{\isacharslash}{\kern0pt}Disjoint{\isacharunderscore}{\kern0pt}Compatibility{\isacharunderscore}{\kern0pt}Rules{\isachardoublequoteclose}\isanewline
\isanewline
\isakeyword{begin}%
\endisatagtheory
{\isafoldtheory}%
%
\isadelimtheory
%
\endisadelimtheory
%
\begin{isamarkuptext}%
Sequential majority comparison compares two alternatives by plurality voting.
The loser gets rejected, and the winner is compared to the next alternative.
This process is repeated until only a single alternative is left, which is
then elected.%
\end{isamarkuptext}\isamarkuptrue%
%
\isadelimdocument
%
\endisadelimdocument
%
\isatagdocument
%
\isamarkupsubsection{Definition%
}
\isamarkuptrue%
%
\endisatagdocument
{\isafolddocument}%
%
\isadelimdocument
%
\endisadelimdocument
\isacommand{fun}\isamarkupfalse%
\ smc\ {\isacharcolon}{\kern0pt}{\isacharcolon}{\kern0pt}\ {\isachardoublequoteopen}{\isacharprime}{\kern0pt}a\ Preference{\isacharunderscore}{\kern0pt}Relation\ {\isasymRightarrow}\ {\isacharprime}{\kern0pt}a\ Electoral{\isacharunderscore}{\kern0pt}Module{\isachardoublequoteclose}\ \isakeyword{where}\isanewline
\ \ {\isachardoublequoteopen}smc\ x\ A\ p\ {\isacharequal}{\kern0pt}\isanewline
\ \ \ \ \ \ {\isacharparenleft}{\kern0pt}{\isacharparenleft}{\kern0pt}{\isacharparenleft}{\kern0pt}{\isacharparenleft}{\kern0pt}{\isacharparenleft}{\kern0pt}{\isacharparenleft}{\kern0pt}pass{\isacharunderscore}{\kern0pt}module\ {\isadigit{2}}\ x{\isacharparenright}{\kern0pt}\ {\isasymtriangleright}\ {\isacharparenleft}{\kern0pt}{\isacharparenleft}{\kern0pt}plurality{\isasymdown}{\isacharparenright}{\kern0pt}\ {\isasymtriangleright}\ {\isacharparenleft}{\kern0pt}pass{\isacharunderscore}{\kern0pt}module\ {\isadigit{1}}\ x{\isacharparenright}{\kern0pt}{\isacharparenright}{\kern0pt}{\isacharparenright}{\kern0pt}\ {\isasymparallel}\isactrlsub {\isasymup}\isanewline
\ \ \ \ \ \ {\isacharparenleft}{\kern0pt}drop{\isacharunderscore}{\kern0pt}module\ {\isadigit{2}}\ x{\isacharparenright}{\kern0pt}{\isacharparenright}{\kern0pt}\ {\isasymcirclearrowleft}\isactrlsub {\isasymexists}\isactrlsub {\isacharbang}{\kern0pt}\isactrlsub d{\isacharparenright}{\kern0pt}\ {\isasymtriangleright}\ elect{\isacharunderscore}{\kern0pt}module{\isacharparenright}{\kern0pt}\ A\ p{\isacharparenright}{\kern0pt}{\isachardoublequoteclose}%
\isadelimdocument
%
\endisadelimdocument
%
\isatagdocument
%
\isamarkupsubsection{Soundness%
}
\isamarkuptrue%
%
\endisatagdocument
{\isafolddocument}%
%
\isadelimdocument
%
\endisadelimdocument
\isacommand{theorem}\isamarkupfalse%
\ smc{\isacharunderscore}{\kern0pt}sound{\isacharcolon}{\kern0pt}\isanewline
\ \ \isakeyword{assumes}\ order{\isacharcolon}{\kern0pt}\ {\isachardoublequoteopen}linear{\isacharunderscore}{\kern0pt}order\ x{\isachardoublequoteclose}\isanewline
\ \ \isakeyword{shows}\ {\isachardoublequoteopen}electoral{\isacharunderscore}{\kern0pt}module\ {\isacharparenleft}{\kern0pt}smc\ x{\isacharparenright}{\kern0pt}{\isachardoublequoteclose}\isanewline
%
\isadelimproof
\ \ %
\endisadelimproof
%
\isatagproof
\isacommand{unfolding}\isamarkupfalse%
\ electoral{\isacharunderscore}{\kern0pt}module{\isacharunderscore}{\kern0pt}def\isanewline
\isacommand{proof}\isamarkupfalse%
\ {\isacharparenleft}{\kern0pt}simp{\isacharcomma}{\kern0pt}\ safe{\isacharcomma}{\kern0pt}\ simp{\isacharunderscore}{\kern0pt}all{\isacharparenright}{\kern0pt}\isanewline
\ \ \isacommand{fix}\isamarkupfalse%
\isanewline
\ \ \ \ A\ {\isacharcolon}{\kern0pt}{\isacharcolon}{\kern0pt}\ {\isachardoublequoteopen}{\isacharprime}{\kern0pt}a\ set{\isachardoublequoteclose}\ \isakeyword{and}\isanewline
\ \ \ \ p\ {\isacharcolon}{\kern0pt}{\isacharcolon}{\kern0pt}\ {\isachardoublequoteopen}{\isacharprime}{\kern0pt}a\ Profile{\isachardoublequoteclose}\ \isakeyword{and}\isanewline
\ \ \ \ xa\ {\isacharcolon}{\kern0pt}{\isacharcolon}{\kern0pt}\ {\isachardoublequoteopen}{\isacharprime}{\kern0pt}a{\isachardoublequoteclose}\isanewline
\ \ \isacommand{let}\isamarkupfalse%
\ {\isacharquery}{\kern0pt}a\ {\isacharequal}{\kern0pt}\ {\isachardoublequoteopen}max{\isacharunderscore}{\kern0pt}aggregator{\isachardoublequoteclose}\isanewline
\ \ \isacommand{let}\isamarkupfalse%
\ {\isacharquery}{\kern0pt}t\ {\isacharequal}{\kern0pt}\ {\isachardoublequoteopen}defer{\isacharunderscore}{\kern0pt}equal{\isacharunderscore}{\kern0pt}condition{\isachardoublequoteclose}\isanewline
\ \ \isacommand{let}\isamarkupfalse%
\ {\isacharquery}{\kern0pt}smc\ {\isacharequal}{\kern0pt}\isanewline
\ \ \ \ {\isachardoublequoteopen}pass{\isacharunderscore}{\kern0pt}module\ {\isadigit{2}}\ x\ {\isasymtriangleright}\isanewline
\ \ \ \ \ \ \ {\isacharparenleft}{\kern0pt}{\isacharparenleft}{\kern0pt}plurality{\isasymdown}{\isacharparenright}{\kern0pt}\ {\isasymtriangleright}\ pass{\isacharunderscore}{\kern0pt}module\ {\isacharparenleft}{\kern0pt}Suc\ {\isadigit{0}}{\isacharparenright}{\kern0pt}\ x{\isacharparenright}{\kern0pt}\ {\isasymparallel}\isactrlsub {\isacharquery}{\kern0pt}a\isanewline
\ \ \ \ \ \ \ \ \ drop{\isacharunderscore}{\kern0pt}module\ {\isadigit{2}}\ x\ {\isasymcirclearrowleft}\isactrlsub {\isacharquery}{\kern0pt}t\ {\isacharparenleft}{\kern0pt}Suc\ {\isadigit{0}}{\isacharparenright}{\kern0pt}{\isachardoublequoteclose}\isanewline
\ \ \isacommand{assume}\isamarkupfalse%
\isanewline
\ \ \ \ fin{\isacharunderscore}{\kern0pt}A{\isacharcolon}{\kern0pt}\ {\isachardoublequoteopen}finite\ A{\isachardoublequoteclose}\ \isakeyword{and}\isanewline
\ \ \ \ prof{\isacharunderscore}{\kern0pt}A{\isacharcolon}{\kern0pt}\ {\isachardoublequoteopen}profile\ A\ p{\isachardoublequoteclose}\ \isakeyword{and}\isanewline
\ \ \ \ reject{\isacharunderscore}{\kern0pt}xa{\isacharcolon}{\kern0pt}\isanewline
\ \ \ \ \ \ {\isachardoublequoteopen}xa\ {\isasymin}\ reject\ {\isacharparenleft}{\kern0pt}{\isacharquery}{\kern0pt}smc{\isacharparenright}{\kern0pt}\ A\ p{\isachardoublequoteclose}\ \isakeyword{and}\isanewline
\ \ \ \ elect{\isacharunderscore}{\kern0pt}xa{\isacharcolon}{\kern0pt}\isanewline
\ \ \ \ \ \ {\isachardoublequoteopen}xa\ {\isasymin}\ elect\ {\isacharparenleft}{\kern0pt}{\isacharquery}{\kern0pt}smc{\isacharparenright}{\kern0pt}\ A\ p{\isachardoublequoteclose}\isanewline
\ \ \isacommand{show}\isamarkupfalse%
\ {\isachardoublequoteopen}False{\isachardoublequoteclose}\isanewline
\ \ \ \ \isacommand{using}\isamarkupfalse%
\ IntI\ drop{\isacharunderscore}{\kern0pt}mod{\isacharunderscore}{\kern0pt}sound\ elect{\isacharunderscore}{\kern0pt}xa\ emptyE\ fin{\isacharunderscore}{\kern0pt}A\isanewline
\ \ \ \ \ \ \ \ \ \ loop{\isacharunderscore}{\kern0pt}comp{\isacharunderscore}{\kern0pt}sound\ max{\isacharunderscore}{\kern0pt}agg{\isacharunderscore}{\kern0pt}sound\ order\ prof{\isacharunderscore}{\kern0pt}A\isanewline
\ \ \ \ \ \ \ \ \ \ par{\isacharunderscore}{\kern0pt}comp{\isacharunderscore}{\kern0pt}sound\ pass{\isacharunderscore}{\kern0pt}mod{\isacharunderscore}{\kern0pt}sound\ reject{\isacharunderscore}{\kern0pt}xa\isanewline
\ \ \ \ \ \ \ \ \ \ plurality{\isacharunderscore}{\kern0pt}sound\ result{\isacharunderscore}{\kern0pt}disj\ rev{\isacharunderscore}{\kern0pt}comp{\isacharunderscore}{\kern0pt}sound\isanewline
\ \ \ \ \ \ \ \ \ \ seq{\isacharunderscore}{\kern0pt}comp{\isacharunderscore}{\kern0pt}sound\isanewline
\ \ \ \ \isacommand{by}\isamarkupfalse%
\ metis\isanewline
\isacommand{next}\isamarkupfalse%
\isanewline
\ \ \isacommand{fix}\isamarkupfalse%
\isanewline
\ \ \ \ A\ {\isacharcolon}{\kern0pt}{\isacharcolon}{\kern0pt}\ {\isachardoublequoteopen}{\isacharprime}{\kern0pt}a\ set{\isachardoublequoteclose}\ \isakeyword{and}\isanewline
\ \ \ \ p\ {\isacharcolon}{\kern0pt}{\isacharcolon}{\kern0pt}\ {\isachardoublequoteopen}{\isacharprime}{\kern0pt}a\ Profile{\isachardoublequoteclose}\ \isakeyword{and}\isanewline
\ \ \ \ xa\ {\isacharcolon}{\kern0pt}{\isacharcolon}{\kern0pt}\ {\isachardoublequoteopen}{\isacharprime}{\kern0pt}a{\isachardoublequoteclose}\isanewline
\ \ \isacommand{let}\isamarkupfalse%
\ {\isacharquery}{\kern0pt}a\ {\isacharequal}{\kern0pt}\ {\isachardoublequoteopen}max{\isacharunderscore}{\kern0pt}aggregator{\isachardoublequoteclose}\isanewline
\ \ \isacommand{let}\isamarkupfalse%
\ {\isacharquery}{\kern0pt}t\ {\isacharequal}{\kern0pt}\ {\isachardoublequoteopen}defer{\isacharunderscore}{\kern0pt}equal{\isacharunderscore}{\kern0pt}condition{\isachardoublequoteclose}\isanewline
\ \ \isacommand{let}\isamarkupfalse%
\ {\isacharquery}{\kern0pt}smc\ {\isacharequal}{\kern0pt}\isanewline
\ \ \ \ {\isachardoublequoteopen}pass{\isacharunderscore}{\kern0pt}module\ {\isadigit{2}}\ x\ {\isasymtriangleright}\isanewline
\ \ \ \ \ \ \ {\isacharparenleft}{\kern0pt}{\isacharparenleft}{\kern0pt}plurality{\isasymdown}{\isacharparenright}{\kern0pt}\ {\isasymtriangleright}\ pass{\isacharunderscore}{\kern0pt}module\ {\isacharparenleft}{\kern0pt}Suc\ {\isadigit{0}}{\isacharparenright}{\kern0pt}\ x{\isacharparenright}{\kern0pt}\ {\isasymparallel}\isactrlsub {\isacharquery}{\kern0pt}a\isanewline
\ \ \ \ \ \ \ \ \ drop{\isacharunderscore}{\kern0pt}module\ {\isadigit{2}}\ x\ {\isasymcirclearrowleft}\isactrlsub {\isacharquery}{\kern0pt}t\ {\isacharparenleft}{\kern0pt}Suc\ {\isadigit{0}}{\isacharparenright}{\kern0pt}{\isachardoublequoteclose}\isanewline
\ \ \isacommand{assume}\isamarkupfalse%
\isanewline
\ \ \ \ fin{\isacharunderscore}{\kern0pt}A{\isacharcolon}{\kern0pt}\ {\isachardoublequoteopen}finite\ A{\isachardoublequoteclose}\ \isakeyword{and}\isanewline
\ \ \ \ prof{\isacharunderscore}{\kern0pt}A{\isacharcolon}{\kern0pt}\ {\isachardoublequoteopen}profile\ A\ p{\isachardoublequoteclose}\ \isakeyword{and}\isanewline
\ \ \ \ reject{\isacharunderscore}{\kern0pt}xa{\isacharcolon}{\kern0pt}\isanewline
\ \ \ \ \ \ {\isachardoublequoteopen}xa\ {\isasymin}\ reject\ {\isacharparenleft}{\kern0pt}{\isacharquery}{\kern0pt}smc{\isacharparenright}{\kern0pt}\ A\ p{\isachardoublequoteclose}\ \isakeyword{and}\isanewline
\ \ \ \ defer{\isacharunderscore}{\kern0pt}xa{\isacharcolon}{\kern0pt}\isanewline
\ \ \ \ \ \ {\isachardoublequoteopen}xa\ {\isasymin}\ defer\ {\isacharparenleft}{\kern0pt}{\isacharquery}{\kern0pt}smc{\isacharparenright}{\kern0pt}\ A\ p{\isachardoublequoteclose}\isanewline
\ \ \isacommand{show}\isamarkupfalse%
\ {\isachardoublequoteopen}False{\isachardoublequoteclose}\isanewline
\ \ \ \ \isacommand{using}\isamarkupfalse%
\ IntI\ drop{\isacharunderscore}{\kern0pt}mod{\isacharunderscore}{\kern0pt}sound\ defer{\isacharunderscore}{\kern0pt}xa\ emptyE\ fin{\isacharunderscore}{\kern0pt}A\isanewline
\ \ \ \ \ \ \ \ \ \ loop{\isacharunderscore}{\kern0pt}comp{\isacharunderscore}{\kern0pt}sound\ max{\isacharunderscore}{\kern0pt}agg{\isacharunderscore}{\kern0pt}sound\ order\ prof{\isacharunderscore}{\kern0pt}A\isanewline
\ \ \ \ \ \ \ \ \ \ par{\isacharunderscore}{\kern0pt}comp{\isacharunderscore}{\kern0pt}sound\ pass{\isacharunderscore}{\kern0pt}mod{\isacharunderscore}{\kern0pt}sound\ reject{\isacharunderscore}{\kern0pt}xa\isanewline
\ \ \ \ \ \ \ \ \ \ plurality{\isacharunderscore}{\kern0pt}sound\ result{\isacharunderscore}{\kern0pt}disj\ rev{\isacharunderscore}{\kern0pt}comp{\isacharunderscore}{\kern0pt}sound\isanewline
\ \ \ \ \ \ \ \ \ \ seq{\isacharunderscore}{\kern0pt}comp{\isacharunderscore}{\kern0pt}sound\isanewline
\ \ \ \ \isacommand{by}\isamarkupfalse%
\ metis\isanewline
\isacommand{next}\isamarkupfalse%
\isanewline
\ \ \isacommand{fix}\isamarkupfalse%
\isanewline
\ \ \ \ A\ {\isacharcolon}{\kern0pt}{\isacharcolon}{\kern0pt}\ {\isachardoublequoteopen}{\isacharprime}{\kern0pt}a\ set{\isachardoublequoteclose}\ \isakeyword{and}\isanewline
\ \ \ \ p\ {\isacharcolon}{\kern0pt}{\isacharcolon}{\kern0pt}\ {\isachardoublequoteopen}{\isacharprime}{\kern0pt}a\ Profile{\isachardoublequoteclose}\ \isakeyword{and}\isanewline
\ \ \ \ xa\ {\isacharcolon}{\kern0pt}{\isacharcolon}{\kern0pt}\ {\isachardoublequoteopen}{\isacharprime}{\kern0pt}a{\isachardoublequoteclose}\isanewline
\ \ \isacommand{let}\isamarkupfalse%
\ {\isacharquery}{\kern0pt}a\ {\isacharequal}{\kern0pt}\ {\isachardoublequoteopen}max{\isacharunderscore}{\kern0pt}aggregator{\isachardoublequoteclose}\isanewline
\ \ \isacommand{let}\isamarkupfalse%
\ {\isacharquery}{\kern0pt}t\ {\isacharequal}{\kern0pt}\ {\isachardoublequoteopen}defer{\isacharunderscore}{\kern0pt}equal{\isacharunderscore}{\kern0pt}condition{\isachardoublequoteclose}\isanewline
\ \ \isacommand{let}\isamarkupfalse%
\ {\isacharquery}{\kern0pt}smc\ {\isacharequal}{\kern0pt}\isanewline
\ \ \ \ {\isachardoublequoteopen}pass{\isacharunderscore}{\kern0pt}module\ {\isadigit{2}}\ x\ {\isasymtriangleright}\isanewline
\ \ \ \ \ \ \ {\isacharparenleft}{\kern0pt}{\isacharparenleft}{\kern0pt}plurality{\isasymdown}{\isacharparenright}{\kern0pt}\ {\isasymtriangleright}\ pass{\isacharunderscore}{\kern0pt}module\ {\isacharparenleft}{\kern0pt}Suc\ {\isadigit{0}}{\isacharparenright}{\kern0pt}\ x{\isacharparenright}{\kern0pt}\ {\isasymparallel}\isactrlsub {\isacharquery}{\kern0pt}a\isanewline
\ \ \ \ \ \ \ \ \ drop{\isacharunderscore}{\kern0pt}module\ {\isadigit{2}}\ x\ {\isasymcirclearrowleft}\isactrlsub {\isacharquery}{\kern0pt}t\ {\isacharparenleft}{\kern0pt}Suc\ {\isadigit{0}}{\isacharparenright}{\kern0pt}{\isachardoublequoteclose}\isanewline
\ \ \isacommand{assume}\isamarkupfalse%
\isanewline
\ \ \ \ fin{\isacharunderscore}{\kern0pt}A{\isacharcolon}{\kern0pt}\ {\isachardoublequoteopen}finite\ A{\isachardoublequoteclose}\ \isakeyword{and}\isanewline
\ \ \ \ prof{\isacharunderscore}{\kern0pt}A{\isacharcolon}{\kern0pt}\ {\isachardoublequoteopen}profile\ A\ p{\isachardoublequoteclose}\ \isakeyword{and}\isanewline
\ \ \ \ elect{\isacharunderscore}{\kern0pt}xa{\isacharcolon}{\kern0pt}\isanewline
\ \ \ \ \ \ {\isachardoublequoteopen}xa\ {\isasymin}\ elect\ {\isacharparenleft}{\kern0pt}{\isacharquery}{\kern0pt}smc{\isacharparenright}{\kern0pt}\ A\ p{\isachardoublequoteclose}\isanewline
\ \ \isacommand{show}\isamarkupfalse%
\ {\isachardoublequoteopen}xa\ {\isasymin}\ A{\isachardoublequoteclose}\isanewline
\ \ \ \ \isacommand{using}\isamarkupfalse%
\ drop{\isacharunderscore}{\kern0pt}mod{\isacharunderscore}{\kern0pt}sound\ elect{\isacharunderscore}{\kern0pt}in{\isacharunderscore}{\kern0pt}alts\ elect{\isacharunderscore}{\kern0pt}xa\ fin{\isacharunderscore}{\kern0pt}A\isanewline
\ \ \ \ \ \ \ \ \ \ in{\isacharunderscore}{\kern0pt}mono\ loop{\isacharunderscore}{\kern0pt}comp{\isacharunderscore}{\kern0pt}sound\ max{\isacharunderscore}{\kern0pt}agg{\isacharunderscore}{\kern0pt}sound\ order\isanewline
\ \ \ \ \ \ \ \ \ \ par{\isacharunderscore}{\kern0pt}comp{\isacharunderscore}{\kern0pt}sound\ pass{\isacharunderscore}{\kern0pt}mod{\isacharunderscore}{\kern0pt}sound\ plurality{\isacharunderscore}{\kern0pt}sound\isanewline
\ \ \ \ \ \ \ \ \ \ prof{\isacharunderscore}{\kern0pt}A\ rev{\isacharunderscore}{\kern0pt}comp{\isacharunderscore}{\kern0pt}sound\ seq{\isacharunderscore}{\kern0pt}comp{\isacharunderscore}{\kern0pt}sound\isanewline
\ \ \ \ \isacommand{by}\isamarkupfalse%
\ metis\isanewline
\isacommand{next}\isamarkupfalse%
\isanewline
\ \ \isacommand{fix}\isamarkupfalse%
\isanewline
\ \ \ \ A\ {\isacharcolon}{\kern0pt}{\isacharcolon}{\kern0pt}\ {\isachardoublequoteopen}{\isacharprime}{\kern0pt}a\ set{\isachardoublequoteclose}\ \isakeyword{and}\isanewline
\ \ \ \ p\ {\isacharcolon}{\kern0pt}{\isacharcolon}{\kern0pt}\ {\isachardoublequoteopen}{\isacharprime}{\kern0pt}a\ Profile{\isachardoublequoteclose}\ \isakeyword{and}\isanewline
\ \ \ \ xa\ {\isacharcolon}{\kern0pt}{\isacharcolon}{\kern0pt}\ {\isachardoublequoteopen}{\isacharprime}{\kern0pt}a{\isachardoublequoteclose}\isanewline
\ \ \isacommand{let}\isamarkupfalse%
\ {\isacharquery}{\kern0pt}a\ {\isacharequal}{\kern0pt}\ {\isachardoublequoteopen}max{\isacharunderscore}{\kern0pt}aggregator{\isachardoublequoteclose}\isanewline
\ \ \isacommand{let}\isamarkupfalse%
\ {\isacharquery}{\kern0pt}t\ {\isacharequal}{\kern0pt}\ {\isachardoublequoteopen}defer{\isacharunderscore}{\kern0pt}equal{\isacharunderscore}{\kern0pt}condition{\isachardoublequoteclose}\isanewline
\ \ \isacommand{let}\isamarkupfalse%
\ {\isacharquery}{\kern0pt}smc\ {\isacharequal}{\kern0pt}\isanewline
\ \ \ \ {\isachardoublequoteopen}pass{\isacharunderscore}{\kern0pt}module\ {\isadigit{2}}\ x\ {\isasymtriangleright}\isanewline
\ \ \ \ \ \ \ {\isacharparenleft}{\kern0pt}{\isacharparenleft}{\kern0pt}plurality{\isasymdown}{\isacharparenright}{\kern0pt}\ {\isasymtriangleright}\ pass{\isacharunderscore}{\kern0pt}module\ {\isacharparenleft}{\kern0pt}Suc\ {\isadigit{0}}{\isacharparenright}{\kern0pt}\ x{\isacharparenright}{\kern0pt}\ {\isasymparallel}\isactrlsub {\isacharquery}{\kern0pt}a\isanewline
\ \ \ \ \ \ \ \ \ drop{\isacharunderscore}{\kern0pt}module\ {\isadigit{2}}\ x\ {\isasymcirclearrowleft}\isactrlsub {\isacharquery}{\kern0pt}t\ {\isacharparenleft}{\kern0pt}Suc\ {\isadigit{0}}{\isacharparenright}{\kern0pt}{\isachardoublequoteclose}\isanewline
\ \ \isacommand{assume}\isamarkupfalse%
\isanewline
\ \ \ \ fin{\isacharunderscore}{\kern0pt}A{\isacharcolon}{\kern0pt}\ {\isachardoublequoteopen}finite\ A{\isachardoublequoteclose}\ \isakeyword{and}\isanewline
\ \ \ \ prof{\isacharunderscore}{\kern0pt}A{\isacharcolon}{\kern0pt}\ {\isachardoublequoteopen}profile\ A\ p{\isachardoublequoteclose}\ \isakeyword{and}\isanewline
\ \ \ \ defer{\isacharunderscore}{\kern0pt}xa{\isacharcolon}{\kern0pt}\isanewline
\ \ \ \ \ \ {\isachardoublequoteopen}xa\ {\isasymin}\ defer\ {\isacharparenleft}{\kern0pt}{\isacharquery}{\kern0pt}smc{\isacharparenright}{\kern0pt}\ A\ p{\isachardoublequoteclose}\isanewline
\ \ \isacommand{show}\isamarkupfalse%
\ {\isachardoublequoteopen}xa\ {\isasymin}\ A{\isachardoublequoteclose}\isanewline
\ \ \ \ \isacommand{using}\isamarkupfalse%
\ drop{\isacharunderscore}{\kern0pt}mod{\isacharunderscore}{\kern0pt}sound\ defer{\isacharunderscore}{\kern0pt}in{\isacharunderscore}{\kern0pt}alts\ defer{\isacharunderscore}{\kern0pt}xa\ fin{\isacharunderscore}{\kern0pt}A\isanewline
\ \ \ \ \ \ \ \ \ \ in{\isacharunderscore}{\kern0pt}mono\ loop{\isacharunderscore}{\kern0pt}comp{\isacharunderscore}{\kern0pt}sound\ max{\isacharunderscore}{\kern0pt}agg{\isacharunderscore}{\kern0pt}sound\ order\isanewline
\ \ \ \ \ \ \ \ \ \ par{\isacharunderscore}{\kern0pt}comp{\isacharunderscore}{\kern0pt}sound\ pass{\isacharunderscore}{\kern0pt}mod{\isacharunderscore}{\kern0pt}sound\ plurality{\isacharunderscore}{\kern0pt}sound\isanewline
\ \ \ \ \ \ \ \ \ \ prof{\isacharunderscore}{\kern0pt}A\ rev{\isacharunderscore}{\kern0pt}comp{\isacharunderscore}{\kern0pt}sound\ seq{\isacharunderscore}{\kern0pt}comp{\isacharunderscore}{\kern0pt}sound\isanewline
\ \ \ \ \isacommand{by}\isamarkupfalse%
\ {\isacharparenleft}{\kern0pt}metis\ {\isacharparenleft}{\kern0pt}no{\isacharunderscore}{\kern0pt}types{\isacharcomma}{\kern0pt}\ lifting{\isacharparenright}{\kern0pt}{\isacharparenright}{\kern0pt}\isanewline
\isacommand{next}\isamarkupfalse%
\isanewline
\ \ \isacommand{fix}\isamarkupfalse%
\isanewline
\ \ \ \ A\ {\isacharcolon}{\kern0pt}{\isacharcolon}{\kern0pt}\ {\isachardoublequoteopen}{\isacharprime}{\kern0pt}a\ set{\isachardoublequoteclose}\ \isakeyword{and}\isanewline
\ \ \ \ p\ {\isacharcolon}{\kern0pt}{\isacharcolon}{\kern0pt}\ {\isachardoublequoteopen}{\isacharprime}{\kern0pt}a\ Profile{\isachardoublequoteclose}\ \isakeyword{and}\isanewline
\ \ \ \ xa\ {\isacharcolon}{\kern0pt}{\isacharcolon}{\kern0pt}\ {\isachardoublequoteopen}{\isacharprime}{\kern0pt}a{\isachardoublequoteclose}\isanewline
\ \ \isacommand{let}\isamarkupfalse%
\ {\isacharquery}{\kern0pt}a\ {\isacharequal}{\kern0pt}\ {\isachardoublequoteopen}max{\isacharunderscore}{\kern0pt}aggregator{\isachardoublequoteclose}\isanewline
\ \ \isacommand{let}\isamarkupfalse%
\ {\isacharquery}{\kern0pt}t\ {\isacharequal}{\kern0pt}\ {\isachardoublequoteopen}defer{\isacharunderscore}{\kern0pt}equal{\isacharunderscore}{\kern0pt}condition{\isachardoublequoteclose}\isanewline
\ \ \isacommand{let}\isamarkupfalse%
\ {\isacharquery}{\kern0pt}smc\ {\isacharequal}{\kern0pt}\isanewline
\ \ \ \ {\isachardoublequoteopen}pass{\isacharunderscore}{\kern0pt}module\ {\isadigit{2}}\ x\ {\isasymtriangleright}\isanewline
\ \ \ \ \ \ \ {\isacharparenleft}{\kern0pt}{\isacharparenleft}{\kern0pt}plurality{\isasymdown}{\isacharparenright}{\kern0pt}\ {\isasymtriangleright}\ pass{\isacharunderscore}{\kern0pt}module\ {\isacharparenleft}{\kern0pt}Suc\ {\isadigit{0}}{\isacharparenright}{\kern0pt}\ x{\isacharparenright}{\kern0pt}\ {\isasymparallel}\isactrlsub {\isacharquery}{\kern0pt}a\isanewline
\ \ \ \ \ \ \ \ \ drop{\isacharunderscore}{\kern0pt}module\ {\isadigit{2}}\ x\ {\isasymcirclearrowleft}\isactrlsub {\isacharquery}{\kern0pt}t\ {\isacharparenleft}{\kern0pt}Suc\ {\isadigit{0}}{\isacharparenright}{\kern0pt}{\isachardoublequoteclose}\isanewline
\ \ \isacommand{assume}\isamarkupfalse%
\isanewline
\ \ \ \ fin{\isacharunderscore}{\kern0pt}A{\isacharcolon}{\kern0pt}\ {\isachardoublequoteopen}finite\ A{\isachardoublequoteclose}\ \isakeyword{and}\isanewline
\ \ \ \ prof{\isacharunderscore}{\kern0pt}A{\isacharcolon}{\kern0pt}\ {\isachardoublequoteopen}profile\ A\ p{\isachardoublequoteclose}\ \isakeyword{and}\isanewline
\ \ \ \ reject{\isacharunderscore}{\kern0pt}xa{\isacharcolon}{\kern0pt}\isanewline
\ \ \ \ \ \ {\isachardoublequoteopen}xa\ {\isasymin}\ reject\ {\isacharparenleft}{\kern0pt}{\isacharquery}{\kern0pt}smc{\isacharparenright}{\kern0pt}\ A\ p{\isachardoublequoteclose}\isanewline
\ \ \isacommand{have}\isamarkupfalse%
\ plurality{\isacharunderscore}{\kern0pt}rev{\isacharunderscore}{\kern0pt}sound{\isacharcolon}{\kern0pt}\isanewline
\ \ \ \ {\isachardoublequoteopen}electoral{\isacharunderscore}{\kern0pt}module\isanewline
\ \ \ \ \ \ {\isacharparenleft}{\kern0pt}plurality{\isacharcolon}{\kern0pt}{\isacharcolon}{\kern0pt}{\isacharprime}{\kern0pt}a\ set\ {\isasymRightarrow}\ {\isacharparenleft}{\kern0pt}{\isacharunderscore}{\kern0pt}\ {\isasymtimes}\ {\isacharunderscore}{\kern0pt}{\isacharparenright}{\kern0pt}\ set\ list\ {\isasymRightarrow}\ {\isacharunderscore}{\kern0pt}\ set\ {\isasymtimes}\ {\isacharunderscore}{\kern0pt}\ set\ {\isasymtimes}\ {\isacharunderscore}{\kern0pt}\ set{\isasymdown}{\isacharparenright}{\kern0pt}{\isachardoublequoteclose}\isanewline
\ \ \ \ \isacommand{by}\isamarkupfalse%
\ simp\isanewline
\ \ \isacommand{have}\isamarkupfalse%
\ par{\isadigit{1}}{\isacharunderscore}{\kern0pt}sound{\isacharcolon}{\kern0pt}\isanewline
\ \ \ \ {\isachardoublequoteopen}electoral{\isacharunderscore}{\kern0pt}module\ {\isacharparenleft}{\kern0pt}pass{\isacharunderscore}{\kern0pt}module\ {\isadigit{2}}\ x\ {\isasymtriangleright}\ {\isacharparenleft}{\kern0pt}{\isacharparenleft}{\kern0pt}plurality{\isasymdown}{\isacharparenright}{\kern0pt}\ {\isasymtriangleright}\ pass{\isacharunderscore}{\kern0pt}module\ {\isadigit{1}}\ x{\isacharparenright}{\kern0pt}{\isacharparenright}{\kern0pt}{\isachardoublequoteclose}\isanewline
\ \ \ \ \isacommand{using}\isamarkupfalse%
\ order\isanewline
\ \ \ \ \isacommand{by}\isamarkupfalse%
\ simp\isanewline
\ \ \isacommand{also}\isamarkupfalse%
\ \isacommand{have}\isamarkupfalse%
\ par{\isadigit{2}}{\isacharunderscore}{\kern0pt}sound{\isacharcolon}{\kern0pt}\isanewline
\ \ \ \ \ \ {\isachardoublequoteopen}electoral{\isacharunderscore}{\kern0pt}module\ {\isacharparenleft}{\kern0pt}drop{\isacharunderscore}{\kern0pt}module\ {\isadigit{2}}\ x{\isacharparenright}{\kern0pt}{\isachardoublequoteclose}\isanewline
\ \ \ \ \isacommand{using}\isamarkupfalse%
\ order\isanewline
\ \ \ \ \isacommand{by}\isamarkupfalse%
\ simp\isanewline
\ \ \isacommand{show}\isamarkupfalse%
\ {\isachardoublequoteopen}xa\ {\isasymin}\ A{\isachardoublequoteclose}\isanewline
\ \ \ \ \isacommand{using}\isamarkupfalse%
\ reject{\isacharunderscore}{\kern0pt}in{\isacharunderscore}{\kern0pt}alts\ reject{\isacharunderscore}{\kern0pt}xa\ fin{\isacharunderscore}{\kern0pt}A\ in{\isacharunderscore}{\kern0pt}mono\isanewline
\ \ \ \ \ \ \ \ \ \ loop{\isacharunderscore}{\kern0pt}comp{\isacharunderscore}{\kern0pt}sound\ max{\isacharunderscore}{\kern0pt}agg{\isacharunderscore}{\kern0pt}sound\ order\isanewline
\ \ \ \ \ \ \ \ \ \ par{\isacharunderscore}{\kern0pt}comp{\isacharunderscore}{\kern0pt}sound\ pass{\isacharunderscore}{\kern0pt}mod{\isacharunderscore}{\kern0pt}sound\ prof{\isacharunderscore}{\kern0pt}A\isanewline
\ \ \ \ \ \ \ \ \ \ seq{\isacharunderscore}{\kern0pt}comp{\isacharunderscore}{\kern0pt}sound\ pass{\isacharunderscore}{\kern0pt}mod{\isacharunderscore}{\kern0pt}sound\ par{\isadigit{1}}{\isacharunderscore}{\kern0pt}sound\isanewline
\ \ \ \ \ \ \ \ \ \ par{\isadigit{2}}{\isacharunderscore}{\kern0pt}sound\ plurality{\isacharunderscore}{\kern0pt}rev{\isacharunderscore}{\kern0pt}sound\isanewline
\ \ \ \ \isacommand{by}\isamarkupfalse%
\ {\isacharparenleft}{\kern0pt}metis\ {\isacharparenleft}{\kern0pt}no{\isacharunderscore}{\kern0pt}types{\isacharparenright}{\kern0pt}{\isacharparenright}{\kern0pt}\isanewline
\isacommand{next}\isamarkupfalse%
\isanewline
\ \ \isacommand{fix}\isamarkupfalse%
\isanewline
\ \ \ \ A\ {\isacharcolon}{\kern0pt}{\isacharcolon}{\kern0pt}\ {\isachardoublequoteopen}{\isacharprime}{\kern0pt}a\ set{\isachardoublequoteclose}\ \isakeyword{and}\isanewline
\ \ \ \ p\ {\isacharcolon}{\kern0pt}{\isacharcolon}{\kern0pt}\ {\isachardoublequoteopen}{\isacharprime}{\kern0pt}a\ Profile{\isachardoublequoteclose}\ \isakeyword{and}\isanewline
\ \ \ \ xa\ {\isacharcolon}{\kern0pt}{\isacharcolon}{\kern0pt}\ {\isachardoublequoteopen}{\isacharprime}{\kern0pt}a{\isachardoublequoteclose}\isanewline
\ \ \isacommand{let}\isamarkupfalse%
\ {\isacharquery}{\kern0pt}a\ {\isacharequal}{\kern0pt}\ {\isachardoublequoteopen}max{\isacharunderscore}{\kern0pt}aggregator{\isachardoublequoteclose}\isanewline
\ \ \isacommand{let}\isamarkupfalse%
\ {\isacharquery}{\kern0pt}t\ {\isacharequal}{\kern0pt}\ {\isachardoublequoteopen}defer{\isacharunderscore}{\kern0pt}equal{\isacharunderscore}{\kern0pt}condition{\isachardoublequoteclose}\isanewline
\ \ \isacommand{let}\isamarkupfalse%
\ {\isacharquery}{\kern0pt}smc\ {\isacharequal}{\kern0pt}\isanewline
\ \ \ \ {\isachardoublequoteopen}pass{\isacharunderscore}{\kern0pt}module\ {\isadigit{2}}\ x\ {\isasymtriangleright}\isanewline
\ \ \ \ \ \ \ {\isacharparenleft}{\kern0pt}{\isacharparenleft}{\kern0pt}plurality{\isasymdown}{\isacharparenright}{\kern0pt}\ {\isasymtriangleright}\ pass{\isacharunderscore}{\kern0pt}module\ {\isacharparenleft}{\kern0pt}Suc\ {\isadigit{0}}{\isacharparenright}{\kern0pt}\ x{\isacharparenright}{\kern0pt}\ {\isasymparallel}\isactrlsub {\isacharquery}{\kern0pt}a\isanewline
\ \ \ \ \ \ \ \ \ drop{\isacharunderscore}{\kern0pt}module\ {\isadigit{2}}\ x\ {\isasymcirclearrowleft}\isactrlsub {\isacharquery}{\kern0pt}t\ {\isacharparenleft}{\kern0pt}Suc\ {\isadigit{0}}{\isacharparenright}{\kern0pt}{\isachardoublequoteclose}\isanewline
\ \ \isacommand{assume}\isamarkupfalse%
\isanewline
\ \ \ \ fin{\isacharunderscore}{\kern0pt}A{\isacharcolon}{\kern0pt}\ {\isachardoublequoteopen}finite\ A{\isachardoublequoteclose}\ \isakeyword{and}\isanewline
\ \ \ \ prof{\isacharunderscore}{\kern0pt}A{\isacharcolon}{\kern0pt}\ {\isachardoublequoteopen}profile\ A\ p{\isachardoublequoteclose}\ \isakeyword{and}\isanewline
\ \ \ \ xa{\isacharunderscore}{\kern0pt}in{\isacharunderscore}{\kern0pt}A{\isacharcolon}{\kern0pt}\ {\isachardoublequoteopen}xa\ {\isasymin}\ A{\isachardoublequoteclose}\ \isakeyword{and}\isanewline
\ \ \ \ not{\isacharunderscore}{\kern0pt}defer{\isacharunderscore}{\kern0pt}xa{\isacharcolon}{\kern0pt}\isanewline
\ \ \ \ \ \ {\isachardoublequoteopen}xa\ {\isasymnotin}\ defer\ {\isacharparenleft}{\kern0pt}{\isacharquery}{\kern0pt}smc{\isacharparenright}{\kern0pt}\ A\ p{\isachardoublequoteclose}\ \isakeyword{and}\isanewline
\ \ \ \ not{\isacharunderscore}{\kern0pt}reject{\isacharunderscore}{\kern0pt}xa{\isacharcolon}{\kern0pt}\isanewline
\ \ \ \ \ \ {\isachardoublequoteopen}xa\ {\isasymnotin}\ reject\ {\isacharparenleft}{\kern0pt}{\isacharquery}{\kern0pt}smc{\isacharparenright}{\kern0pt}\ A\ p{\isachardoublequoteclose}\isanewline
\ \ \isacommand{show}\isamarkupfalse%
\ {\isachardoublequoteopen}xa\ {\isasymin}\ elect\ {\isacharparenleft}{\kern0pt}{\isacharquery}{\kern0pt}smc{\isacharparenright}{\kern0pt}\ A\ p{\isachardoublequoteclose}\isanewline
\ \ \ \ \isacommand{using}\isamarkupfalse%
\ drop{\isacharunderscore}{\kern0pt}mod{\isacharunderscore}{\kern0pt}sound\ loop{\isacharunderscore}{\kern0pt}comp{\isacharunderscore}{\kern0pt}sound\ max{\isacharunderscore}{\kern0pt}agg{\isacharunderscore}{\kern0pt}sound\isanewline
\ \ \ \ \ \ \ \ \ \ order\ par{\isacharunderscore}{\kern0pt}comp{\isacharunderscore}{\kern0pt}sound\ pass{\isacharunderscore}{\kern0pt}mod{\isacharunderscore}{\kern0pt}sound\ xa{\isacharunderscore}{\kern0pt}in{\isacharunderscore}{\kern0pt}A\isanewline
\ \ \ \ \ \ \ \ \ \ plurality{\isacharunderscore}{\kern0pt}sound\ rev{\isacharunderscore}{\kern0pt}comp{\isacharunderscore}{\kern0pt}sound\ seq{\isacharunderscore}{\kern0pt}comp{\isacharunderscore}{\kern0pt}sound\isanewline
\ \ \ \ \ \ \ \ \ \ electoral{\isacharunderscore}{\kern0pt}mod{\isacharunderscore}{\kern0pt}defer{\isacharunderscore}{\kern0pt}elem\ fin{\isacharunderscore}{\kern0pt}A\ not{\isacharunderscore}{\kern0pt}defer{\isacharunderscore}{\kern0pt}xa\isanewline
\ \ \ \ \ \ \ \ \ \ not{\isacharunderscore}{\kern0pt}reject{\isacharunderscore}{\kern0pt}xa\ prof{\isacharunderscore}{\kern0pt}A\isanewline
\ \ \ \ \isacommand{by}\isamarkupfalse%
\ metis\isanewline
\isacommand{qed}\isamarkupfalse%
%
\endisatagproof
{\isafoldproof}%
%
\isadelimproof
%
\endisadelimproof
%
\isadelimdocument
%
\endisadelimdocument
%
\isatagdocument
%
\isamarkupsubsection{Electing%
}
\isamarkuptrue%
%
\endisatagdocument
{\isafolddocument}%
%
\isadelimdocument
%
\endisadelimdocument
\isacommand{theorem}\isamarkupfalse%
\ smc{\isacharunderscore}{\kern0pt}electing{\isacharcolon}{\kern0pt}\isanewline
\ \ \isakeyword{assumes}\ order{\isacharcolon}{\kern0pt}\ {\isachardoublequoteopen}linear{\isacharunderscore}{\kern0pt}order\ x{\isachardoublequoteclose}\isanewline
\ \ \isakeyword{shows}\ {\isachardoublequoteopen}electing\ {\isacharparenleft}{\kern0pt}smc\ x{\isacharparenright}{\kern0pt}{\isachardoublequoteclose}\isanewline
%
\isadelimproof
%
\endisadelimproof
%
\isatagproof
\isacommand{proof}\isamarkupfalse%
\ {\isacharminus}{\kern0pt}\isanewline
\ \ \isacommand{let}\isamarkupfalse%
\ {\isacharquery}{\kern0pt}pass{\isadigit{2}}\ {\isacharequal}{\kern0pt}\ {\isachardoublequoteopen}pass{\isacharunderscore}{\kern0pt}module\ {\isadigit{2}}\ x{\isachardoublequoteclose}\isanewline
\ \ \isacommand{let}\isamarkupfalse%
\ {\isacharquery}{\kern0pt}tie{\isacharunderscore}{\kern0pt}breaker\ {\isacharequal}{\kern0pt}\ {\isachardoublequoteopen}{\isacharparenleft}{\kern0pt}pass{\isacharunderscore}{\kern0pt}module\ {\isadigit{1}}\ x{\isacharparenright}{\kern0pt}{\isachardoublequoteclose}\isanewline
\ \ \isacommand{let}\isamarkupfalse%
\ {\isacharquery}{\kern0pt}plurality{\isacharunderscore}{\kern0pt}defer\ {\isacharequal}{\kern0pt}\ {\isachardoublequoteopen}{\isacharparenleft}{\kern0pt}plurality{\isasymdown}{\isacharparenright}{\kern0pt}\ {\isasymtriangleright}\ {\isacharquery}{\kern0pt}tie{\isacharunderscore}{\kern0pt}breaker{\isachardoublequoteclose}\isanewline
\ \ \isacommand{let}\isamarkupfalse%
\ {\isacharquery}{\kern0pt}compare{\isacharunderscore}{\kern0pt}two\ {\isacharequal}{\kern0pt}\ {\isachardoublequoteopen}{\isacharquery}{\kern0pt}pass{\isadigit{2}}\ {\isasymtriangleright}\ {\isacharquery}{\kern0pt}plurality{\isacharunderscore}{\kern0pt}defer{\isachardoublequoteclose}\isanewline
\ \ \isacommand{let}\isamarkupfalse%
\ {\isacharquery}{\kern0pt}drop{\isadigit{2}}\ {\isacharequal}{\kern0pt}\ {\isachardoublequoteopen}drop{\isacharunderscore}{\kern0pt}module\ {\isadigit{2}}\ x{\isachardoublequoteclose}\isanewline
\ \ \isacommand{let}\isamarkupfalse%
\ {\isacharquery}{\kern0pt}eliminator\ {\isacharequal}{\kern0pt}\ {\isachardoublequoteopen}{\isacharquery}{\kern0pt}compare{\isacharunderscore}{\kern0pt}two\ {\isasymparallel}\isactrlsub {\isasymup}\ {\isacharquery}{\kern0pt}drop{\isadigit{2}}{\isachardoublequoteclose}\isanewline
\ \ \isacommand{let}\isamarkupfalse%
\ {\isacharquery}{\kern0pt}loop\ {\isacharequal}{\kern0pt}\isanewline
\ \ \ \ {\isachardoublequoteopen}let\ t\ {\isacharequal}{\kern0pt}\ defer{\isacharunderscore}{\kern0pt}equal{\isacharunderscore}{\kern0pt}condition\ {\isadigit{1}}\ in\ {\isacharparenleft}{\kern0pt}{\isacharquery}{\kern0pt}eliminator\ {\isasymcirclearrowleft}\isactrlsub t{\isacharparenright}{\kern0pt}{\isachardoublequoteclose}\isanewline
\isanewline
\ \ \isacommand{have}\isamarkupfalse%
\ {\isadigit{0}}{\isadigit{0}}{\isadigit{0}}{\isadigit{1}}{\isadigit{1}}{\isacharcolon}{\kern0pt}\ {\isachardoublequoteopen}non{\isacharunderscore}{\kern0pt}electing\ {\isacharparenleft}{\kern0pt}plurality{\isasymdown}{\isacharparenright}{\kern0pt}{\isachardoublequoteclose}\isanewline
\ \ \ \ \isacommand{by}\isamarkupfalse%
\ simp\isanewline
\ \ \isacommand{have}\isamarkupfalse%
\ {\isadigit{0}}{\isadigit{0}}{\isadigit{0}}{\isadigit{1}}{\isadigit{2}}{\isacharcolon}{\kern0pt}\ {\isachardoublequoteopen}non{\isacharunderscore}{\kern0pt}electing\ {\isacharquery}{\kern0pt}tie{\isacharunderscore}{\kern0pt}breaker{\isachardoublequoteclose}\isanewline
\ \ \ \ \isacommand{using}\isamarkupfalse%
\ order\isanewline
\ \ \ \ \isacommand{by}\isamarkupfalse%
\ simp\isanewline
\ \ \isacommand{have}\isamarkupfalse%
\ {\isadigit{0}}{\isadigit{0}}{\isadigit{0}}{\isadigit{1}}{\isadigit{3}}{\isacharcolon}{\kern0pt}\ {\isachardoublequoteopen}defers\ {\isadigit{1}}\ {\isacharquery}{\kern0pt}tie{\isacharunderscore}{\kern0pt}breaker{\isachardoublequoteclose}\isanewline
\ \ \ \ \isacommand{using}\isamarkupfalse%
\ order\ pass{\isacharunderscore}{\kern0pt}one{\isacharunderscore}{\kern0pt}mod{\isacharunderscore}{\kern0pt}def{\isacharunderscore}{\kern0pt}one\isanewline
\ \ \ \ \isacommand{by}\isamarkupfalse%
\ simp\isanewline
\ \ \isacommand{have}\isamarkupfalse%
\ {\isadigit{2}}{\isadigit{0}}{\isadigit{0}}{\isadigit{0}}{\isadigit{0}}{\isacharcolon}{\kern0pt}\ {\isachardoublequoteopen}non{\isacharunderscore}{\kern0pt}blocking\ {\isacharparenleft}{\kern0pt}plurality{\isasymdown}{\isacharparenright}{\kern0pt}{\isachardoublequoteclose}\isanewline
\ \ \ \ \isacommand{by}\isamarkupfalse%
\ simp\isanewline
\isanewline
\ \ \isacommand{have}\isamarkupfalse%
\ {\isadigit{0}}{\isadigit{0}}{\isadigit{2}}{\isadigit{0}}{\isacharcolon}{\kern0pt}\ {\isachardoublequoteopen}disjoint{\isacharunderscore}{\kern0pt}compatibility\ {\isacharquery}{\kern0pt}pass{\isadigit{2}}\ {\isacharquery}{\kern0pt}drop{\isadigit{2}}{\isachardoublequoteclose}\isanewline
\ \ \ \ \isacommand{using}\isamarkupfalse%
\ order\isanewline
\ \ \ \ \isacommand{by}\isamarkupfalse%
\ simp\ \isanewline
\ \ \isacommand{have}\isamarkupfalse%
\ {\isadigit{1}}{\isadigit{0}}{\isadigit{0}}{\isadigit{0}}{\isacharcolon}{\kern0pt}\ {\isachardoublequoteopen}non{\isacharunderscore}{\kern0pt}electing\ {\isacharquery}{\kern0pt}pass{\isadigit{2}}{\isachardoublequoteclose}\isanewline
\ \ \ \ \isacommand{using}\isamarkupfalse%
\ order\isanewline
\ \ \ \ \isacommand{by}\isamarkupfalse%
\ simp\isanewline
\ \ \isacommand{have}\isamarkupfalse%
\ {\isadigit{1}}{\isadigit{0}}{\isadigit{0}}{\isadigit{1}}{\isacharcolon}{\kern0pt}\ {\isachardoublequoteopen}non{\isacharunderscore}{\kern0pt}electing\ {\isacharquery}{\kern0pt}plurality{\isacharunderscore}{\kern0pt}defer{\isachardoublequoteclose}\isanewline
\ \ \ \ \isacommand{using}\isamarkupfalse%
\ {\isadigit{0}}{\isadigit{0}}{\isadigit{0}}{\isadigit{1}}{\isadigit{1}}\ {\isadigit{0}}{\isadigit{0}}{\isadigit{0}}{\isadigit{1}}{\isadigit{2}}\isanewline
\ \ \ \ \isacommand{by}\isamarkupfalse%
\ simp\isanewline
\ \ \isacommand{have}\isamarkupfalse%
\ {\isadigit{2}}{\isadigit{0}}{\isadigit{0}}{\isadigit{0}}{\isacharcolon}{\kern0pt}\ {\isachardoublequoteopen}non{\isacharunderscore}{\kern0pt}blocking\ {\isacharquery}{\kern0pt}pass{\isadigit{2}}{\isachardoublequoteclose}\isanewline
\ \ \ \ \isacommand{using}\isamarkupfalse%
\ order\isanewline
\ \ \ \ \isacommand{by}\isamarkupfalse%
\ simp\isanewline
\ \ \isacommand{have}\isamarkupfalse%
\ {\isadigit{2}}{\isadigit{0}}{\isadigit{0}}{\isadigit{1}}{\isacharcolon}{\kern0pt}\ {\isachardoublequoteopen}defers\ {\isadigit{1}}\ {\isacharquery}{\kern0pt}plurality{\isacharunderscore}{\kern0pt}defer{\isachardoublequoteclose}\isanewline
\ \ \ \ \isacommand{using}\isamarkupfalse%
\ {\isadigit{2}}{\isadigit{0}}{\isadigit{0}}{\isadigit{0}}{\isadigit{0}}\ {\isadigit{0}}{\isadigit{0}}{\isadigit{0}}{\isadigit{1}}{\isadigit{1}}\ {\isadigit{0}}{\isadigit{0}}{\isadigit{0}}{\isadigit{1}}{\isadigit{3}}\ seq{\isacharunderscore}{\kern0pt}comp{\isacharunderscore}{\kern0pt}def{\isacharunderscore}{\kern0pt}one\isanewline
\ \ \ \ \isacommand{by}\isamarkupfalse%
\ blast\isanewline
\isanewline
\ \ \isacommand{have}\isamarkupfalse%
\ {\isadigit{0}}{\isadigit{0}}{\isadigit{2}}{\isacharcolon}{\kern0pt}\ {\isachardoublequoteopen}disjoint{\isacharunderscore}{\kern0pt}compatibility\ {\isacharquery}{\kern0pt}compare{\isacharunderscore}{\kern0pt}two\ {\isacharquery}{\kern0pt}drop{\isadigit{2}}{\isachardoublequoteclose}\isanewline
\ \ \ \ \isacommand{using}\isamarkupfalse%
\ order\ {\isadigit{0}}{\isadigit{0}}{\isadigit{2}}{\isadigit{0}}\isanewline
\ \ \ \ \isacommand{by}\isamarkupfalse%
\ simp\isanewline
\ \ \isacommand{have}\isamarkupfalse%
\ {\isadigit{1}}{\isadigit{0}}{\isadigit{0}}{\isacharcolon}{\kern0pt}\ {\isachardoublequoteopen}non{\isacharunderscore}{\kern0pt}electing\ {\isacharquery}{\kern0pt}compare{\isacharunderscore}{\kern0pt}two{\isachardoublequoteclose}\isanewline
\ \ \ \ \isacommand{using}\isamarkupfalse%
\ {\isadigit{1}}{\isadigit{0}}{\isadigit{0}}{\isadigit{0}}\ {\isadigit{1}}{\isadigit{0}}{\isadigit{0}}{\isadigit{1}}\isanewline
\ \ \ \ \isacommand{by}\isamarkupfalse%
\ simp\isanewline
\ \ \isacommand{have}\isamarkupfalse%
\ {\isadigit{1}}{\isadigit{0}}{\isadigit{1}}{\isacharcolon}{\kern0pt}\ {\isachardoublequoteopen}non{\isacharunderscore}{\kern0pt}electing\ {\isacharquery}{\kern0pt}drop{\isadigit{2}}{\isachardoublequoteclose}\isanewline
\ \ \ \ \isacommand{using}\isamarkupfalse%
\ order\isanewline
\ \ \ \ \isacommand{by}\isamarkupfalse%
\ simp\isanewline
\ \ \isacommand{have}\isamarkupfalse%
\ {\isadigit{1}}{\isadigit{0}}{\isadigit{2}}{\isacharcolon}{\kern0pt}\ {\isachardoublequoteopen}agg{\isacharunderscore}{\kern0pt}conservative\ max{\isacharunderscore}{\kern0pt}aggregator{\isachardoublequoteclose}\isanewline
\ \ \ \ \isacommand{by}\isamarkupfalse%
\ simp\isanewline
\ \ \isacommand{have}\isamarkupfalse%
\ {\isadigit{2}}{\isadigit{0}}{\isadigit{0}}{\isacharcolon}{\kern0pt}\ {\isachardoublequoteopen}defers\ {\isadigit{1}}\ {\isacharquery}{\kern0pt}compare{\isacharunderscore}{\kern0pt}two{\isachardoublequoteclose}\isanewline
\ \ \ \ \isacommand{using}\isamarkupfalse%
\ {\isadigit{2}}{\isadigit{0}}{\isadigit{0}}{\isadigit{0}}\ {\isadigit{1}}{\isadigit{0}}{\isadigit{0}}{\isadigit{0}}\ {\isadigit{2}}{\isadigit{0}}{\isadigit{0}}{\isadigit{1}}\ seq{\isacharunderscore}{\kern0pt}comp{\isacharunderscore}{\kern0pt}def{\isacharunderscore}{\kern0pt}one\isanewline
\ \ \ \ \isacommand{by}\isamarkupfalse%
\ auto\isanewline
\ \ \isacommand{have}\isamarkupfalse%
\ {\isadigit{2}}{\isadigit{0}}{\isadigit{1}}{\isacharcolon}{\kern0pt}\ {\isachardoublequoteopen}rejects\ {\isadigit{2}}\ {\isacharquery}{\kern0pt}drop{\isadigit{2}}{\isachardoublequoteclose}\isanewline
\ \ \ \ \isacommand{using}\isamarkupfalse%
\ order\isanewline
\ \ \ \ \isacommand{by}\isamarkupfalse%
\ simp\isanewline
\isanewline
\ \ \isacommand{have}\isamarkupfalse%
\ {\isadigit{1}}{\isadigit{0}}{\isacharcolon}{\kern0pt}\ {\isachardoublequoteopen}non{\isacharunderscore}{\kern0pt}electing\ {\isacharquery}{\kern0pt}eliminator{\isachardoublequoteclose}\isanewline
\ \ \ \ \isacommand{using}\isamarkupfalse%
\ {\isadigit{1}}{\isadigit{0}}{\isadigit{0}}\ {\isadigit{1}}{\isadigit{0}}{\isadigit{1}}\ {\isadigit{1}}{\isadigit{0}}{\isadigit{2}}\isanewline
\ \ \ \ \isacommand{by}\isamarkupfalse%
\ simp\isanewline
\ \ \isacommand{have}\isamarkupfalse%
\ {\isadigit{2}}{\isadigit{0}}{\isacharcolon}{\kern0pt}\ {\isachardoublequoteopen}eliminates\ {\isadigit{1}}\ {\isacharquery}{\kern0pt}eliminator{\isachardoublequoteclose}\isanewline
\ \ \ \ \isacommand{using}\isamarkupfalse%
\ {\isadigit{2}}{\isadigit{0}}{\isadigit{0}}\ {\isadigit{1}}{\isadigit{0}}{\isadigit{0}}\ {\isadigit{2}}{\isadigit{0}}{\isadigit{1}}\ {\isadigit{0}}{\isadigit{0}}{\isadigit{2}}\ par{\isacharunderscore}{\kern0pt}comp{\isacharunderscore}{\kern0pt}elim{\isacharunderscore}{\kern0pt}one\isanewline
\ \ \ \ \isacommand{by}\isamarkupfalse%
\ metis\isanewline
\isanewline
\ \ \isacommand{have}\isamarkupfalse%
\ {\isadigit{2}}{\isacharcolon}{\kern0pt}\ {\isachardoublequoteopen}defers\ {\isadigit{1}}\ {\isacharquery}{\kern0pt}loop{\isachardoublequoteclose}\isanewline
\ \ \ \ \isacommand{using}\isamarkupfalse%
\ {\isadigit{1}}{\isadigit{0}}\ {\isadigit{2}}{\isadigit{0}}\isanewline
\ \ \ \ \isacommand{by}\isamarkupfalse%
\ simp\isanewline
\ \ \isacommand{have}\isamarkupfalse%
\ {\isadigit{3}}{\isacharcolon}{\kern0pt}\ {\isachardoublequoteopen}electing\ elect{\isacharunderscore}{\kern0pt}module{\isachardoublequoteclose}\isanewline
\ \ \ \ \isacommand{by}\isamarkupfalse%
\ simp\isanewline
\isanewline
\ \ \isacommand{show}\isamarkupfalse%
\ {\isacharquery}{\kern0pt}thesis\isanewline
\ \ \ \ \isacommand{using}\isamarkupfalse%
\ {\isadigit{2}}\ {\isadigit{3}}\ smc{\isacharunderscore}{\kern0pt}sound\ smc{\isachardot}{\kern0pt}simps\ electing{\isacharunderscore}{\kern0pt}def\isanewline
\ \ \ \ \ \ \ \ \ \ iter{\isachardot}{\kern0pt}simps\isanewline
\ \ \ \ \ \ \ \ \ \ order\ seq{\isacharunderscore}{\kern0pt}comp{\isacharunderscore}{\kern0pt}electing\isanewline
\ \ \ \ \isacommand{by}\isamarkupfalse%
\ metis\isanewline
\isacommand{qed}\isamarkupfalse%
%
\endisatagproof
{\isafoldproof}%
%
\isadelimproof
%
\endisadelimproof
%
\isadelimdocument
%
\endisadelimdocument
%
\isatagdocument
%
\isamarkupsubsection{(Weak) Monotonicity Property%
}
\isamarkuptrue%
%
\endisatagdocument
{\isafolddocument}%
%
\isadelimdocument
%
\endisadelimdocument
\isacommand{theorem}\isamarkupfalse%
\ smc{\isacharunderscore}{\kern0pt}monotone{\isacharcolon}{\kern0pt}\isanewline
\ \ \isakeyword{assumes}\ order{\isacharcolon}{\kern0pt}\ {\isachardoublequoteopen}linear{\isacharunderscore}{\kern0pt}order\ x{\isachardoublequoteclose}\isanewline
\ \ \isakeyword{shows}\ {\isachardoublequoteopen}monotonicity\ {\isacharparenleft}{\kern0pt}smc\ x{\isacharparenright}{\kern0pt}{\isachardoublequoteclose}\isanewline
%
\isadelimproof
%
\endisadelimproof
%
\isatagproof
\isacommand{proof}\isamarkupfalse%
\ {\isacharminus}{\kern0pt}\isanewline
\isanewline
\ \ \isacommand{let}\isamarkupfalse%
\ {\isacharquery}{\kern0pt}pass{\isadigit{2}}\ {\isacharequal}{\kern0pt}\ {\isachardoublequoteopen}pass{\isacharunderscore}{\kern0pt}module\ {\isadigit{2}}\ x{\isachardoublequoteclose}\isanewline
\ \ \isacommand{let}\isamarkupfalse%
\ {\isacharquery}{\kern0pt}tie{\isacharunderscore}{\kern0pt}breaker\ {\isacharequal}{\kern0pt}\ {\isachardoublequoteopen}{\isacharparenleft}{\kern0pt}pass{\isacharunderscore}{\kern0pt}module\ {\isadigit{1}}\ x{\isacharparenright}{\kern0pt}{\isachardoublequoteclose}\isanewline
\ \ \isacommand{let}\isamarkupfalse%
\ {\isacharquery}{\kern0pt}plurality{\isacharunderscore}{\kern0pt}defer\ {\isacharequal}{\kern0pt}\ {\isachardoublequoteopen}{\isacharparenleft}{\kern0pt}plurality{\isasymdown}{\isacharparenright}{\kern0pt}\ {\isasymtriangleright}\ {\isacharquery}{\kern0pt}tie{\isacharunderscore}{\kern0pt}breaker{\isachardoublequoteclose}\isanewline
\ \ \isacommand{let}\isamarkupfalse%
\ {\isacharquery}{\kern0pt}compare{\isacharunderscore}{\kern0pt}two\ {\isacharequal}{\kern0pt}\ {\isachardoublequoteopen}{\isacharquery}{\kern0pt}pass{\isadigit{2}}\ {\isasymtriangleright}\ {\isacharquery}{\kern0pt}plurality{\isacharunderscore}{\kern0pt}defer{\isachardoublequoteclose}\isanewline
\ \ \isacommand{let}\isamarkupfalse%
\ {\isacharquery}{\kern0pt}drop{\isadigit{2}}\ {\isacharequal}{\kern0pt}\ {\isachardoublequoteopen}drop{\isacharunderscore}{\kern0pt}module\ {\isadigit{2}}\ x{\isachardoublequoteclose}\isanewline
\ \ \isacommand{let}\isamarkupfalse%
\ {\isacharquery}{\kern0pt}eliminator\ {\isacharequal}{\kern0pt}\ {\isachardoublequoteopen}{\isacharquery}{\kern0pt}compare{\isacharunderscore}{\kern0pt}two\ {\isasymparallel}\isactrlsub {\isasymup}\ {\isacharquery}{\kern0pt}drop{\isadigit{2}}{\isachardoublequoteclose}\isanewline
\ \ \isacommand{let}\isamarkupfalse%
\ {\isacharquery}{\kern0pt}loop\ {\isacharequal}{\kern0pt}\isanewline
\ \ \ \ {\isachardoublequoteopen}let\ t\ {\isacharequal}{\kern0pt}\ defer{\isacharunderscore}{\kern0pt}equal{\isacharunderscore}{\kern0pt}condition\ {\isadigit{1}}\ in\ {\isacharparenleft}{\kern0pt}{\isacharquery}{\kern0pt}eliminator\ {\isasymcirclearrowleft}\isactrlsub t{\isacharparenright}{\kern0pt}{\isachardoublequoteclose}\isanewline
\isanewline
\ \ \isacommand{have}\isamarkupfalse%
\ {\isadigit{0}}{\isadigit{0}}{\isadigit{0}}{\isadigit{1}}{\isadigit{0}}{\isacharcolon}{\kern0pt}\ {\isachardoublequoteopen}defer{\isacharunderscore}{\kern0pt}invariant{\isacharunderscore}{\kern0pt}monotonicity\ {\isacharparenleft}{\kern0pt}plurality{\isasymdown}{\isacharparenright}{\kern0pt}{\isachardoublequoteclose}\isanewline
\ \ \ \ \isacommand{by}\isamarkupfalse%
\ simp\ \isanewline
\ \ \isacommand{have}\isamarkupfalse%
\ {\isadigit{0}}{\isadigit{0}}{\isadigit{0}}{\isadigit{1}}{\isadigit{1}}{\isacharcolon}{\kern0pt}\ {\isachardoublequoteopen}non{\isacharunderscore}{\kern0pt}electing\ {\isacharparenleft}{\kern0pt}plurality{\isasymdown}{\isacharparenright}{\kern0pt}{\isachardoublequoteclose}\isanewline
\ \ \ \ \isacommand{by}\isamarkupfalse%
\ simp\ \isanewline
\ \ \isacommand{have}\isamarkupfalse%
\ {\isadigit{0}}{\isadigit{0}}{\isadigit{0}}{\isadigit{1}}{\isadigit{2}}{\isacharcolon}{\kern0pt}\ {\isachardoublequoteopen}non{\isacharunderscore}{\kern0pt}electing\ {\isacharquery}{\kern0pt}tie{\isacharunderscore}{\kern0pt}breaker{\isachardoublequoteclose}\isanewline
\ \ \ \ \isacommand{using}\isamarkupfalse%
\ order\isanewline
\ \ \ \ \isacommand{by}\isamarkupfalse%
\ simp\ \isanewline
\ \ \isacommand{have}\isamarkupfalse%
\ {\isadigit{0}}{\isadigit{0}}{\isadigit{0}}{\isadigit{1}}{\isadigit{3}}{\isacharcolon}{\kern0pt}\ {\isachardoublequoteopen}defers\ {\isadigit{1}}\ {\isacharquery}{\kern0pt}tie{\isacharunderscore}{\kern0pt}breaker{\isachardoublequoteclose}\isanewline
\ \ \ \ \isacommand{using}\isamarkupfalse%
\ order\ pass{\isacharunderscore}{\kern0pt}one{\isacharunderscore}{\kern0pt}mod{\isacharunderscore}{\kern0pt}def{\isacharunderscore}{\kern0pt}one\isanewline
\ \ \ \ \isacommand{by}\isamarkupfalse%
\ simp\isanewline
\ \ \isacommand{have}\isamarkupfalse%
\ {\isadigit{0}}{\isadigit{0}}{\isadigit{0}}{\isadigit{1}}{\isadigit{4}}{\isacharcolon}{\kern0pt}\ {\isachardoublequoteopen}defer{\isacharunderscore}{\kern0pt}monotonicity\ {\isacharquery}{\kern0pt}tie{\isacharunderscore}{\kern0pt}breaker{\isachardoublequoteclose}\isanewline
\ \ \ \ \isacommand{using}\isamarkupfalse%
\ order\isanewline
\ \ \ \ \isacommand{by}\isamarkupfalse%
\ simp\ \isanewline
\ \ \isacommand{have}\isamarkupfalse%
\ {\isadigit{2}}{\isadigit{0}}{\isadigit{0}}{\isadigit{0}}{\isadigit{0}}{\isacharcolon}{\kern0pt}\ {\isachardoublequoteopen}non{\isacharunderscore}{\kern0pt}blocking\ {\isacharparenleft}{\kern0pt}plurality{\isasymdown}{\isacharparenright}{\kern0pt}{\isachardoublequoteclose}\isanewline
\ \ \ \ \isacommand{by}\isamarkupfalse%
\ simp\ \isanewline
\isanewline
\ \ \isacommand{have}\isamarkupfalse%
\ {\isadigit{0}}{\isadigit{0}}{\isadigit{0}}{\isadigit{0}}{\isacharcolon}{\kern0pt}\ {\isachardoublequoteopen}defer{\isacharunderscore}{\kern0pt}lift{\isacharunderscore}{\kern0pt}invariance\ {\isacharquery}{\kern0pt}pass{\isadigit{2}}{\isachardoublequoteclose}\isanewline
\ \ \ \ \isacommand{using}\isamarkupfalse%
\ order\isanewline
\ \ \ \ \isacommand{by}\isamarkupfalse%
\ simp\ \isanewline
\ \ \isacommand{have}\isamarkupfalse%
\ {\isadigit{0}}{\isadigit{0}}{\isadigit{0}}{\isadigit{1}}{\isacharcolon}{\kern0pt}\ {\isachardoublequoteopen}defer{\isacharunderscore}{\kern0pt}lift{\isacharunderscore}{\kern0pt}invariance\ {\isacharquery}{\kern0pt}plurality{\isacharunderscore}{\kern0pt}defer{\isachardoublequoteclose}\isanewline
\ \ \ \ \isacommand{using}\isamarkupfalse%
\ {\isadigit{0}}{\isadigit{0}}{\isadigit{0}}{\isadigit{1}}{\isadigit{0}}\ {\isadigit{0}}{\isadigit{0}}{\isadigit{0}}{\isadigit{1}}{\isadigit{1}}\ {\isadigit{0}}{\isadigit{0}}{\isadigit{0}}{\isadigit{1}}{\isadigit{2}}\ {\isadigit{0}}{\isadigit{0}}{\isadigit{0}}{\isadigit{1}}{\isadigit{3}}\ {\isadigit{0}}{\isadigit{0}}{\isadigit{0}}{\isadigit{1}}{\isadigit{4}}\isanewline
\ \ \ \ \isacommand{by}\isamarkupfalse%
\ simp\ \isanewline
\ \ \isacommand{have}\isamarkupfalse%
\ {\isadigit{0}}{\isadigit{0}}{\isadigit{2}}{\isadigit{0}}{\isacharcolon}{\kern0pt}\ {\isachardoublequoteopen}disjoint{\isacharunderscore}{\kern0pt}compatibility\ {\isacharquery}{\kern0pt}pass{\isadigit{2}}\ {\isacharquery}{\kern0pt}drop{\isadigit{2}}{\isachardoublequoteclose}\isanewline
\ \ \ \ \isacommand{using}\isamarkupfalse%
\ order\isanewline
\ \ \ \ \isacommand{by}\isamarkupfalse%
\ simp\ \isanewline
\ \ \isacommand{have}\isamarkupfalse%
\ {\isadigit{1}}{\isadigit{0}}{\isadigit{0}}{\isadigit{0}}{\isacharcolon}{\kern0pt}\ {\isachardoublequoteopen}non{\isacharunderscore}{\kern0pt}electing\ {\isacharquery}{\kern0pt}pass{\isadigit{2}}{\isachardoublequoteclose}\isanewline
\ \ \ \ \isacommand{using}\isamarkupfalse%
\ order\isanewline
\ \ \ \ \isacommand{by}\isamarkupfalse%
\ simp\ \isanewline
\ \ \isacommand{have}\isamarkupfalse%
\ {\isadigit{1}}{\isadigit{0}}{\isadigit{0}}{\isadigit{1}}{\isacharcolon}{\kern0pt}\ {\isachardoublequoteopen}non{\isacharunderscore}{\kern0pt}electing\ {\isacharquery}{\kern0pt}plurality{\isacharunderscore}{\kern0pt}defer{\isachardoublequoteclose}\isanewline
\ \ \ \ \isacommand{using}\isamarkupfalse%
\ {\isadigit{0}}{\isadigit{0}}{\isadigit{0}}{\isadigit{1}}{\isadigit{1}}\ {\isadigit{0}}{\isadigit{0}}{\isadigit{0}}{\isadigit{1}}{\isadigit{2}}\isanewline
\ \ \ \ \isacommand{by}\isamarkupfalse%
\ simp\ \isanewline
\ \ \isacommand{have}\isamarkupfalse%
\ {\isadigit{2}}{\isadigit{0}}{\isadigit{0}}{\isadigit{0}}{\isacharcolon}{\kern0pt}\ {\isachardoublequoteopen}non{\isacharunderscore}{\kern0pt}blocking\ {\isacharquery}{\kern0pt}pass{\isadigit{2}}{\isachardoublequoteclose}\isanewline
\ \ \ \ \isacommand{using}\isamarkupfalse%
\ order\isanewline
\ \ \ \ \isacommand{by}\isamarkupfalse%
\ simp\ \isanewline
\ \ \isacommand{have}\isamarkupfalse%
\ {\isadigit{2}}{\isadigit{0}}{\isadigit{0}}{\isadigit{1}}{\isacharcolon}{\kern0pt}\ {\isachardoublequoteopen}defers\ {\isadigit{1}}\ {\isacharquery}{\kern0pt}plurality{\isacharunderscore}{\kern0pt}defer{\isachardoublequoteclose}\isanewline
\ \ \ \ \isacommand{using}\isamarkupfalse%
\ {\isadigit{2}}{\isadigit{0}}{\isadigit{0}}{\isadigit{0}}{\isadigit{0}}\ {\isadigit{0}}{\isadigit{0}}{\isadigit{0}}{\isadigit{1}}{\isadigit{1}}\ {\isadigit{0}}{\isadigit{0}}{\isadigit{0}}{\isadigit{1}}{\isadigit{3}}\ seq{\isacharunderscore}{\kern0pt}comp{\isacharunderscore}{\kern0pt}def{\isacharunderscore}{\kern0pt}one\isanewline
\ \ \ \ \isacommand{by}\isamarkupfalse%
\ blast\isanewline
\isanewline
\ \ \isacommand{have}\isamarkupfalse%
\ {\isadigit{0}}{\isadigit{0}}{\isadigit{0}}{\isacharcolon}{\kern0pt}\ {\isachardoublequoteopen}defer{\isacharunderscore}{\kern0pt}lift{\isacharunderscore}{\kern0pt}invariance\ {\isacharquery}{\kern0pt}compare{\isacharunderscore}{\kern0pt}two{\isachardoublequoteclose}\isanewline
\ \ \ \ \isacommand{using}\isamarkupfalse%
\ {\isadigit{0}}{\isadigit{0}}{\isadigit{0}}{\isadigit{0}}\ {\isadigit{0}}{\isadigit{0}}{\isadigit{0}}{\isadigit{1}}\isanewline
\ \ \ \ \isacommand{by}\isamarkupfalse%
\ simp\ \isanewline
\ \ \isacommand{have}\isamarkupfalse%
\ {\isadigit{0}}{\isadigit{0}}{\isadigit{1}}{\isacharcolon}{\kern0pt}\ {\isachardoublequoteopen}defer{\isacharunderscore}{\kern0pt}lift{\isacharunderscore}{\kern0pt}invariance\ {\isacharquery}{\kern0pt}drop{\isadigit{2}}{\isachardoublequoteclose}\isanewline
\ \ \ \ \isacommand{using}\isamarkupfalse%
\ order\isanewline
\ \ \ \ \isacommand{by}\isamarkupfalse%
\ simp\ \isanewline
\ \ \isacommand{have}\isamarkupfalse%
\ {\isadigit{0}}{\isadigit{0}}{\isadigit{2}}{\isacharcolon}{\kern0pt}\ {\isachardoublequoteopen}disjoint{\isacharunderscore}{\kern0pt}compatibility\ {\isacharquery}{\kern0pt}compare{\isacharunderscore}{\kern0pt}two\ {\isacharquery}{\kern0pt}drop{\isadigit{2}}{\isachardoublequoteclose}\isanewline
\ \ \ \ \isacommand{using}\isamarkupfalse%
\ order\ {\isadigit{0}}{\isadigit{0}}{\isadigit{2}}{\isadigit{0}}\isanewline
\ \ \ \ \isacommand{by}\isamarkupfalse%
\ simp\isanewline
\ \ \ \ \ \ \isanewline
\ \ \isacommand{have}\isamarkupfalse%
\ {\isadigit{1}}{\isadigit{0}}{\isadigit{0}}{\isacharcolon}{\kern0pt}\ {\isachardoublequoteopen}non{\isacharunderscore}{\kern0pt}electing\ {\isacharquery}{\kern0pt}compare{\isacharunderscore}{\kern0pt}two{\isachardoublequoteclose}\isanewline
\ \ \ \ \isacommand{using}\isamarkupfalse%
\ {\isadigit{1}}{\isadigit{0}}{\isadigit{0}}{\isadigit{0}}\ {\isadigit{1}}{\isadigit{0}}{\isadigit{0}}{\isadigit{1}}\isanewline
\ \ \ \ \isacommand{by}\isamarkupfalse%
\ simp\ \isanewline
\ \ \isacommand{have}\isamarkupfalse%
\ {\isadigit{1}}{\isadigit{0}}{\isadigit{1}}{\isacharcolon}{\kern0pt}\ {\isachardoublequoteopen}non{\isacharunderscore}{\kern0pt}electing\ {\isacharquery}{\kern0pt}drop{\isadigit{2}}{\isachardoublequoteclose}\isanewline
\ \ \ \ \isacommand{using}\isamarkupfalse%
\ order\isanewline
\ \ \ \ \isacommand{by}\isamarkupfalse%
\ simp\ \isanewline
\ \ \isacommand{have}\isamarkupfalse%
\ {\isadigit{1}}{\isadigit{0}}{\isadigit{2}}{\isacharcolon}{\kern0pt}\ {\isachardoublequoteopen}agg{\isacharunderscore}{\kern0pt}conservative\ max{\isacharunderscore}{\kern0pt}aggregator{\isachardoublequoteclose}\isanewline
\ \ \ \ \isacommand{by}\isamarkupfalse%
\ simp\ \isanewline
\ \ \isacommand{have}\isamarkupfalse%
\ {\isadigit{2}}{\isadigit{0}}{\isadigit{0}}{\isacharcolon}{\kern0pt}\ {\isachardoublequoteopen}defers\ {\isadigit{1}}\ {\isacharquery}{\kern0pt}compare{\isacharunderscore}{\kern0pt}two{\isachardoublequoteclose}\isanewline
\ \ \ \ \isacommand{using}\isamarkupfalse%
\ {\isadigit{2}}{\isadigit{0}}{\isadigit{0}}{\isadigit{0}}\ {\isadigit{1}}{\isadigit{0}}{\isadigit{0}}{\isadigit{0}}\ {\isadigit{2}}{\isadigit{0}}{\isadigit{0}}{\isadigit{1}}\ seq{\isacharunderscore}{\kern0pt}comp{\isacharunderscore}{\kern0pt}def{\isacharunderscore}{\kern0pt}one\isanewline
\ \ \ \ \isacommand{by}\isamarkupfalse%
\ auto\isanewline
\ \ \isacommand{have}\isamarkupfalse%
\ {\isadigit{2}}{\isadigit{0}}{\isadigit{1}}{\isacharcolon}{\kern0pt}\ {\isachardoublequoteopen}rejects\ {\isadigit{2}}\ {\isacharquery}{\kern0pt}drop{\isadigit{2}}{\isachardoublequoteclose}\isanewline
\ \ \ \ \isacommand{using}\isamarkupfalse%
\ order\isanewline
\ \ \ \ \isacommand{by}\isamarkupfalse%
\ simp\ \isanewline
\isanewline
\ \ \isacommand{have}\isamarkupfalse%
\ {\isadigit{0}}{\isadigit{0}}{\isacharcolon}{\kern0pt}\ {\isachardoublequoteopen}defer{\isacharunderscore}{\kern0pt}lift{\isacharunderscore}{\kern0pt}invariance\ {\isacharquery}{\kern0pt}eliminator{\isachardoublequoteclose}\isanewline
\ \ \ \ \isacommand{using}\isamarkupfalse%
\ {\isadigit{0}}{\isadigit{0}}{\isadigit{0}}\ {\isadigit{0}}{\isadigit{0}}{\isadigit{1}}\ {\isadigit{0}}{\isadigit{0}}{\isadigit{2}}\ par{\isacharunderscore}{\kern0pt}comp{\isacharunderscore}{\kern0pt}def{\isacharunderscore}{\kern0pt}lift{\isacharunderscore}{\kern0pt}inv\isanewline
\ \ \ \ \isacommand{by}\isamarkupfalse%
\ simp\ \isanewline
\ \ \isacommand{have}\isamarkupfalse%
\ {\isadigit{1}}{\isadigit{0}}{\isacharcolon}{\kern0pt}\ {\isachardoublequoteopen}non{\isacharunderscore}{\kern0pt}electing\ {\isacharquery}{\kern0pt}eliminator{\isachardoublequoteclose}\isanewline
\ \ \ \ \isacommand{using}\isamarkupfalse%
\ {\isadigit{1}}{\isadigit{0}}{\isadigit{0}}\ {\isadigit{1}}{\isadigit{0}}{\isadigit{1}}\ {\isadigit{1}}{\isadigit{0}}{\isadigit{2}}\isanewline
\ \ \ \ \isacommand{by}\isamarkupfalse%
\ simp\ \isanewline
\ \ \isacommand{have}\isamarkupfalse%
\ {\isadigit{2}}{\isadigit{0}}{\isacharcolon}{\kern0pt}\ {\isachardoublequoteopen}eliminates\ {\isadigit{1}}\ {\isacharquery}{\kern0pt}eliminator{\isachardoublequoteclose}\isanewline
\ \ \ \ \isacommand{using}\isamarkupfalse%
\ {\isadigit{2}}{\isadigit{0}}{\isadigit{0}}\ {\isadigit{1}}{\isadigit{0}}{\isadigit{0}}\ {\isadigit{2}}{\isadigit{0}}{\isadigit{1}}\ {\isadigit{0}}{\isadigit{0}}{\isadigit{2}}\ par{\isacharunderscore}{\kern0pt}comp{\isacharunderscore}{\kern0pt}elim{\isacharunderscore}{\kern0pt}one\isanewline
\ \ \ \ \isacommand{by}\isamarkupfalse%
\ simp\isanewline
\isanewline
\ \ \isacommand{have}\isamarkupfalse%
\ {\isadigit{0}}{\isacharcolon}{\kern0pt}\ {\isachardoublequoteopen}defer{\isacharunderscore}{\kern0pt}lift{\isacharunderscore}{\kern0pt}invariance\ {\isacharquery}{\kern0pt}loop{\isachardoublequoteclose}\isanewline
\ \ \ \ \isacommand{using}\isamarkupfalse%
\ {\isadigit{0}}{\isadigit{0}}\isanewline
\ \ \ \ \isacommand{by}\isamarkupfalse%
\ simp\ \isanewline
\ \ \isacommand{have}\isamarkupfalse%
\ {\isadigit{1}}{\isacharcolon}{\kern0pt}\ {\isachardoublequoteopen}non{\isacharunderscore}{\kern0pt}electing\ {\isacharquery}{\kern0pt}loop{\isachardoublequoteclose}\isanewline
\ \ \ \ \isacommand{using}\isamarkupfalse%
\ {\isadigit{1}}{\isadigit{0}}\isanewline
\ \ \ \ \isacommand{by}\isamarkupfalse%
\ simp\ \isanewline
\ \ \isacommand{have}\isamarkupfalse%
\ {\isadigit{2}}{\isacharcolon}{\kern0pt}\ {\isachardoublequoteopen}defers\ {\isadigit{1}}\ {\isacharquery}{\kern0pt}loop{\isachardoublequoteclose}\isanewline
\ \ \ \ \isacommand{using}\isamarkupfalse%
\ {\isadigit{1}}{\isadigit{0}}\ {\isadigit{2}}{\isadigit{0}}\isanewline
\ \ \ \ \isacommand{by}\isamarkupfalse%
\ simp\ \isanewline
\ \ \isacommand{have}\isamarkupfalse%
\ {\isadigit{3}}{\isacharcolon}{\kern0pt}\ {\isachardoublequoteopen}electing\ elect{\isacharunderscore}{\kern0pt}module{\isachardoublequoteclose}\isanewline
\ \ \ \ \isacommand{by}\isamarkupfalse%
\ simp\ \isanewline
\isanewline
\ \ \isacommand{show}\isamarkupfalse%
\ {\isacharquery}{\kern0pt}thesis\isanewline
\ \ \ \ \isacommand{using}\isamarkupfalse%
\ {\isadigit{0}}\ {\isadigit{1}}\ {\isadigit{2}}\ {\isadigit{3}}\isanewline
\ \ \ \ \ \ \ \ \ \ monotonicity{\isacharunderscore}{\kern0pt}def\isanewline
\ \ \ \ \ \ \ \ \ \ iter{\isachardot}{\kern0pt}simps\isanewline
\ \ \ \ \ \ \ \ \ \ smc{\isacharunderscore}{\kern0pt}sound\ smc{\isachardot}{\kern0pt}simps\ order\ seq{\isacharunderscore}{\kern0pt}comp{\isacharunderscore}{\kern0pt}mono\isanewline
\ \ \ \ \isacommand{by}\isamarkupfalse%
\ {\isacharparenleft}{\kern0pt}metis\ {\isacharparenleft}{\kern0pt}full{\isacharunderscore}{\kern0pt}types{\isacharparenright}{\kern0pt}{\isacharparenright}{\kern0pt}\isanewline
\isacommand{qed}\isamarkupfalse%
%
\endisatagproof
{\isafoldproof}%
%
\isadelimproof
\isanewline
%
\endisadelimproof
%
\isadelimtheory
\isanewline
%
\endisadelimtheory
%
\isatagtheory
\isacommand{end}\isamarkupfalse%
%
\endisatagtheory
{\isafoldtheory}%
%
\isadelimtheory
%
\endisadelimtheory
%
\end{isabellebody}%
\endinput
%:%file=~/Documents/Studies/VotingRuleGenerator/virage/src/test/resources/verifiedVotingRuleConstruction/theories/Voting_Rules/Sequential_Majority_Comparison.thy%:%
%:%6=3%:%
%:%11=4%:%
%:%12=5%:%
%:%14=8%:%
%:%30=10%:%
%:%31=10%:%
%:%32=11%:%
%:%33=12%:%
%:%34=13%:%
%:%35=14%:%
%:%36=15%:%
%:%37=16%:%
%:%38=17%:%
%:%39=18%:%
%:%40=19%:%
%:%41=20%:%
%:%42=21%:%
%:%51=24%:%
%:%52=25%:%
%:%53=26%:%
%:%54=27%:%
%:%63=29%:%
%:%73=31%:%
%:%74=31%:%
%:%75=32%:%
%:%84=36%:%
%:%94=43%:%
%:%95=43%:%
%:%96=44%:%
%:%97=45%:%
%:%100=46%:%
%:%104=46%:%
%:%105=46%:%
%:%106=47%:%
%:%107=47%:%
%:%108=48%:%
%:%109=48%:%
%:%110=49%:%
%:%111=50%:%
%:%112=51%:%
%:%113=52%:%
%:%114=52%:%
%:%115=53%:%
%:%116=53%:%
%:%117=54%:%
%:%118=54%:%
%:%119=55%:%
%:%121=57%:%
%:%122=58%:%
%:%123=58%:%
%:%124=59%:%
%:%125=60%:%
%:%126=61%:%
%:%127=62%:%
%:%128=63%:%
%:%129=64%:%
%:%130=65%:%
%:%131=65%:%
%:%132=66%:%
%:%133=66%:%
%:%134=67%:%
%:%135=68%:%
%:%136=69%:%
%:%137=70%:%
%:%138=71%:%
%:%139=71%:%
%:%140=72%:%
%:%141=72%:%
%:%142=73%:%
%:%143=73%:%
%:%144=74%:%
%:%145=75%:%
%:%146=76%:%
%:%147=77%:%
%:%148=77%:%
%:%149=78%:%
%:%150=78%:%
%:%151=79%:%
%:%152=79%:%
%:%153=80%:%
%:%155=82%:%
%:%156=83%:%
%:%157=83%:%
%:%158=84%:%
%:%159=85%:%
%:%160=86%:%
%:%161=87%:%
%:%162=88%:%
%:%163=89%:%
%:%164=90%:%
%:%165=90%:%
%:%166=91%:%
%:%167=91%:%
%:%168=92%:%
%:%169=93%:%
%:%170=94%:%
%:%171=95%:%
%:%172=96%:%
%:%173=96%:%
%:%174=97%:%
%:%175=97%:%
%:%176=98%:%
%:%177=98%:%
%:%178=99%:%
%:%179=100%:%
%:%180=101%:%
%:%181=102%:%
%:%182=102%:%
%:%183=103%:%
%:%184=103%:%
%:%185=104%:%
%:%186=104%:%
%:%187=105%:%
%:%189=107%:%
%:%190=108%:%
%:%191=108%:%
%:%192=109%:%
%:%193=110%:%
%:%194=111%:%
%:%195=112%:%
%:%196=113%:%
%:%197=113%:%
%:%198=114%:%
%:%199=114%:%
%:%200=115%:%
%:%201=116%:%
%:%202=117%:%
%:%203=118%:%
%:%204=118%:%
%:%205=119%:%
%:%206=119%:%
%:%207=120%:%
%:%208=120%:%
%:%209=121%:%
%:%210=122%:%
%:%211=123%:%
%:%212=124%:%
%:%213=124%:%
%:%214=125%:%
%:%215=125%:%
%:%216=126%:%
%:%217=126%:%
%:%218=127%:%
%:%220=129%:%
%:%221=130%:%
%:%222=130%:%
%:%223=131%:%
%:%224=132%:%
%:%225=133%:%
%:%226=134%:%
%:%227=135%:%
%:%228=135%:%
%:%229=136%:%
%:%230=136%:%
%:%231=137%:%
%:%232=138%:%
%:%233=139%:%
%:%234=140%:%
%:%235=140%:%
%:%236=141%:%
%:%237=141%:%
%:%238=142%:%
%:%239=142%:%
%:%240=143%:%
%:%241=144%:%
%:%242=145%:%
%:%243=146%:%
%:%244=146%:%
%:%245=147%:%
%:%246=147%:%
%:%247=148%:%
%:%248=148%:%
%:%249=149%:%
%:%251=151%:%
%:%252=152%:%
%:%253=152%:%
%:%254=153%:%
%:%255=154%:%
%:%256=155%:%
%:%257=156%:%
%:%258=157%:%
%:%259=157%:%
%:%260=158%:%
%:%261=159%:%
%:%262=160%:%
%:%263=160%:%
%:%264=161%:%
%:%265=161%:%
%:%266=162%:%
%:%267=163%:%
%:%268=163%:%
%:%269=164%:%
%:%270=164%:%
%:%271=165%:%
%:%272=165%:%
%:%273=165%:%
%:%274=166%:%
%:%275=167%:%
%:%276=167%:%
%:%277=168%:%
%:%278=168%:%
%:%279=169%:%
%:%280=169%:%
%:%281=170%:%
%:%282=170%:%
%:%283=171%:%
%:%284=172%:%
%:%285=173%:%
%:%286=174%:%
%:%287=175%:%
%:%288=175%:%
%:%289=176%:%
%:%290=176%:%
%:%291=177%:%
%:%292=177%:%
%:%293=178%:%
%:%294=179%:%
%:%295=180%:%
%:%296=181%:%
%:%297=181%:%
%:%298=182%:%
%:%299=182%:%
%:%300=183%:%
%:%301=183%:%
%:%302=184%:%
%:%304=186%:%
%:%305=187%:%
%:%306=187%:%
%:%307=188%:%
%:%308=189%:%
%:%309=190%:%
%:%310=191%:%
%:%311=192%:%
%:%312=193%:%
%:%313=194%:%
%:%314=195%:%
%:%315=195%:%
%:%316=196%:%
%:%317=196%:%
%:%318=197%:%
%:%319=198%:%
%:%320=199%:%
%:%321=200%:%
%:%322=201%:%
%:%323=201%:%
%:%324=202%:%
%:%339=205%:%
%:%349=213%:%
%:%350=213%:%
%:%351=214%:%
%:%352=215%:%
%:%359=216%:%
%:%360=216%:%
%:%361=217%:%
%:%362=217%:%
%:%363=218%:%
%:%364=218%:%
%:%365=219%:%
%:%366=219%:%
%:%367=220%:%
%:%368=220%:%
%:%369=221%:%
%:%370=221%:%
%:%371=222%:%
%:%372=222%:%
%:%373=223%:%
%:%374=223%:%
%:%375=224%:%
%:%376=225%:%
%:%377=226%:%
%:%378=226%:%
%:%379=227%:%
%:%380=227%:%
%:%381=228%:%
%:%382=228%:%
%:%383=229%:%
%:%384=229%:%
%:%385=230%:%
%:%386=230%:%
%:%387=231%:%
%:%388=231%:%
%:%389=232%:%
%:%390=232%:%
%:%391=233%:%
%:%392=233%:%
%:%393=234%:%
%:%394=234%:%
%:%395=235%:%
%:%396=235%:%
%:%397=236%:%
%:%398=237%:%
%:%399=237%:%
%:%400=238%:%
%:%401=238%:%
%:%402=239%:%
%:%403=239%:%
%:%404=240%:%
%:%405=240%:%
%:%406=241%:%
%:%407=241%:%
%:%408=242%:%
%:%409=242%:%
%:%410=243%:%
%:%411=243%:%
%:%412=244%:%
%:%413=244%:%
%:%414=245%:%
%:%415=245%:%
%:%416=246%:%
%:%417=246%:%
%:%418=247%:%
%:%419=247%:%
%:%420=248%:%
%:%421=248%:%
%:%422=249%:%
%:%423=249%:%
%:%424=250%:%
%:%425=250%:%
%:%426=251%:%
%:%427=251%:%
%:%428=252%:%
%:%429=253%:%
%:%430=253%:%
%:%431=254%:%
%:%432=254%:%
%:%433=255%:%
%:%434=255%:%
%:%435=256%:%
%:%436=256%:%
%:%437=257%:%
%:%438=257%:%
%:%439=258%:%
%:%440=258%:%
%:%441=259%:%
%:%442=259%:%
%:%443=260%:%
%:%444=260%:%
%:%445=261%:%
%:%446=261%:%
%:%447=262%:%
%:%448=262%:%
%:%449=263%:%
%:%450=263%:%
%:%451=264%:%
%:%452=264%:%
%:%453=265%:%
%:%454=265%:%
%:%455=266%:%
%:%456=266%:%
%:%457=267%:%
%:%458=267%:%
%:%459=268%:%
%:%460=268%:%
%:%461=269%:%
%:%462=269%:%
%:%463=270%:%
%:%464=271%:%
%:%465=271%:%
%:%466=272%:%
%:%467=272%:%
%:%468=273%:%
%:%469=273%:%
%:%470=274%:%
%:%471=274%:%
%:%472=275%:%
%:%473=275%:%
%:%474=276%:%
%:%475=276%:%
%:%476=277%:%
%:%477=278%:%
%:%478=278%:%
%:%479=279%:%
%:%480=279%:%
%:%481=280%:%
%:%482=280%:%
%:%483=281%:%
%:%484=281%:%
%:%485=282%:%
%:%486=282%:%
%:%487=283%:%
%:%488=284%:%
%:%489=284%:%
%:%490=285%:%
%:%491=285%:%
%:%492=286%:%
%:%493=287%:%
%:%494=288%:%
%:%495=288%:%
%:%496=289%:%
%:%511=291%:%
%:%521=297%:%
%:%522=297%:%
%:%523=298%:%
%:%524=299%:%
%:%531=300%:%
%:%532=300%:%
%:%533=301%:%
%:%534=302%:%
%:%535=302%:%
%:%536=303%:%
%:%537=303%:%
%:%538=304%:%
%:%539=304%:%
%:%540=305%:%
%:%541=305%:%
%:%542=306%:%
%:%543=306%:%
%:%544=307%:%
%:%545=307%:%
%:%546=308%:%
%:%547=308%:%
%:%548=309%:%
%:%549=310%:%
%:%550=311%:%
%:%551=311%:%
%:%552=312%:%
%:%553=312%:%
%:%554=313%:%
%:%555=313%:%
%:%556=314%:%
%:%557=314%:%
%:%558=315%:%
%:%559=315%:%
%:%560=316%:%
%:%561=316%:%
%:%562=317%:%
%:%563=317%:%
%:%564=318%:%
%:%565=318%:%
%:%566=319%:%
%:%567=319%:%
%:%568=320%:%
%:%569=320%:%
%:%570=321%:%
%:%571=321%:%
%:%572=322%:%
%:%573=322%:%
%:%574=323%:%
%:%575=323%:%
%:%576=324%:%
%:%577=324%:%
%:%578=325%:%
%:%579=325%:%
%:%580=326%:%
%:%581=327%:%
%:%582=327%:%
%:%583=328%:%
%:%584=328%:%
%:%585=329%:%
%:%586=329%:%
%:%587=330%:%
%:%588=330%:%
%:%589=331%:%
%:%590=331%:%
%:%591=332%:%
%:%592=332%:%
%:%593=333%:%
%:%594=333%:%
%:%595=334%:%
%:%596=334%:%
%:%597=335%:%
%:%598=335%:%
%:%599=336%:%
%:%600=336%:%
%:%601=337%:%
%:%602=337%:%
%:%603=338%:%
%:%604=338%:%
%:%605=339%:%
%:%606=339%:%
%:%607=340%:%
%:%608=340%:%
%:%609=341%:%
%:%610=341%:%
%:%611=342%:%
%:%612=342%:%
%:%613=343%:%
%:%614=343%:%
%:%615=344%:%
%:%616=344%:%
%:%617=345%:%
%:%618=345%:%
%:%619=346%:%
%:%620=346%:%
%:%621=347%:%
%:%622=347%:%
%:%623=348%:%
%:%624=349%:%
%:%625=349%:%
%:%626=350%:%
%:%627=350%:%
%:%628=351%:%
%:%629=351%:%
%:%630=352%:%
%:%631=352%:%
%:%632=353%:%
%:%633=353%:%
%:%634=354%:%
%:%635=354%:%
%:%636=355%:%
%:%637=355%:%
%:%638=356%:%
%:%639=356%:%
%:%640=357%:%
%:%641=357%:%
%:%642=358%:%
%:%642=359%:%
%:%643=360%:%
%:%644=360%:%
%:%645=361%:%
%:%646=361%:%
%:%647=362%:%
%:%648=362%:%
%:%649=363%:%
%:%650=363%:%
%:%651=364%:%
%:%652=364%:%
%:%653=365%:%
%:%654=365%:%
%:%655=366%:%
%:%656=366%:%
%:%657=367%:%
%:%658=367%:%
%:%659=368%:%
%:%660=368%:%
%:%661=369%:%
%:%662=369%:%
%:%663=370%:%
%:%664=370%:%
%:%665=371%:%
%:%666=371%:%
%:%667=372%:%
%:%668=372%:%
%:%669=373%:%
%:%670=373%:%
%:%671=374%:%
%:%672=375%:%
%:%673=375%:%
%:%674=376%:%
%:%675=376%:%
%:%676=377%:%
%:%677=377%:%
%:%678=378%:%
%:%679=378%:%
%:%680=379%:%
%:%681=379%:%
%:%682=380%:%
%:%683=380%:%
%:%684=381%:%
%:%685=381%:%
%:%686=382%:%
%:%687=382%:%
%:%688=383%:%
%:%689=383%:%
%:%690=384%:%
%:%691=385%:%
%:%692=385%:%
%:%693=386%:%
%:%694=386%:%
%:%695=387%:%
%:%696=387%:%
%:%697=388%:%
%:%698=388%:%
%:%699=389%:%
%:%700=389%:%
%:%701=390%:%
%:%702=390%:%
%:%703=391%:%
%:%704=391%:%
%:%705=392%:%
%:%706=392%:%
%:%707=393%:%
%:%708=393%:%
%:%709=394%:%
%:%710=394%:%
%:%711=395%:%
%:%712=395%:%
%:%713=396%:%
%:%714=397%:%
%:%715=397%:%
%:%716=398%:%
%:%717=398%:%
%:%718=399%:%
%:%719=400%:%
%:%720=401%:%
%:%721=402%:%
%:%722=402%:%
%:%723=403%:%
%:%729=403%:%
%:%734=404%:%
%:%739=405%:%
%
\begin{isabellebody}%
\setisabellecontext{Homogeneity}%
%
\isadelimtheory
%
\endisadelimtheory
%
\isatagtheory
\isacommand{theory}\isamarkupfalse%
\ Homogeneity\isanewline
\ \ \isakeyword{imports}\ {\isachardoublequoteopen}{\isachardot}{\kern0pt}{\isachardot}{\kern0pt}{\isacharslash}{\kern0pt}Compositional{\isacharunderscore}{\kern0pt}Framework{\isacharslash}{\kern0pt}Components{\isacharslash}{\kern0pt}Electoral{\isacharunderscore}{\kern0pt}Module{\isachardoublequoteclose}\isanewline
\isanewline
\isakeyword{begin}%
\endisatagtheory
{\isafoldtheory}%
%
\isadelimtheory
\ \isanewline
%
\endisadelimtheory
\isanewline
\isacommand{fun}\isamarkupfalse%
\ times\ {\isacharcolon}{\kern0pt}{\isacharcolon}{\kern0pt}\ {\isachardoublequoteopen}nat\ {\isasymRightarrow}\ {\isacharprime}{\kern0pt}a\ list\ {\isasymRightarrow}\ {\isacharprime}{\kern0pt}a\ list{\isachardoublequoteclose}\ \isakeyword{where}\isanewline
\ \ {\isachardoublequoteopen}times\ n\ l\ {\isacharequal}{\kern0pt}\ concat\ {\isacharparenleft}{\kern0pt}replicate\ n\ l{\isacharparenright}{\kern0pt}{\isachardoublequoteclose}\isanewline
\isanewline
\isacommand{definition}\isamarkupfalse%
\ homogeneity\ {\isacharcolon}{\kern0pt}{\isacharcolon}{\kern0pt}\ {\isachardoublequoteopen}{\isacharprime}{\kern0pt}a\ Electoral{\isacharunderscore}{\kern0pt}Module\ {\isasymRightarrow}\ bool{\isachardoublequoteclose}\ \isakeyword{where}\isanewline
{\isachardoublequoteopen}homogeneity\ m\ {\isasymequiv}\isanewline
\ \ electoral{\isacharunderscore}{\kern0pt}module\ m\ {\isasymand}\isanewline
\ \ \ \ {\isacharparenleft}{\kern0pt}{\isasymforall}\ A\ p\ n\ {\isachardot}{\kern0pt}\isanewline
\ \ \ \ \ \ {\isacharparenleft}{\kern0pt}finite{\isacharunderscore}{\kern0pt}profile\ A\ p\ {\isasymand}\ n\ {\isachargreater}{\kern0pt}\ {\isadigit{0}}\ {\isasymlongrightarrow}\isanewline
\ \ \ \ \ \ \ \ \ \ {\isacharparenleft}{\kern0pt}m\ A\ p\ {\isacharequal}{\kern0pt}\ m\ A\ {\isacharparenleft}{\kern0pt}times\ n\ p{\isacharparenright}{\kern0pt}{\isacharparenright}{\kern0pt}{\isacharparenright}{\kern0pt}{\isacharparenright}{\kern0pt}{\isachardoublequoteclose}\isanewline
%
\isadelimtheory
\isanewline
%
\endisadelimtheory
%
\isatagtheory
\isacommand{end}\isamarkupfalse%
%
\endisatagtheory
{\isafoldtheory}%
%
\isadelimtheory
%
\endisadelimtheory
%
\end{isabellebody}%
\endinput
%:%file=~/Documents/Studies/VotingRuleGenerator/virage/src/test/resources/verifiedVotingRuleConstruction/theories/Social_Choice_Properties/Homogeneity.thy%:%
%:%10=1%:%
%:%11=1%:%
%:%12=2%:%
%:%13=3%:%
%:%14=4%:%
%:%19=4%:%
%:%22=5%:%
%:%23=6%:%
%:%24=6%:%
%:%25=7%:%
%:%26=8%:%
%:%27=9%:%
%:%28=9%:%
%:%29=10%:%
%:%33=14%:%
%:%36=15%:%
%:%41=16%:%


% optional bibliography
\bibliographystyle{abbrv}
\bibliography{root}

\end{document}
