%
\begin{isabellebody}%
\setisabellecontext{Aggregator}%
%
\isadelimdocument
\isanewline
%
\endisadelimdocument
%
\isatagdocument
\isanewline
\isanewline
\isanewline
%
\isamarkupsection{Aggregator%
}
\isamarkuptrue%
%
\endisatagdocument
{\isafolddocument}%
%
\isadelimdocument
%
\endisadelimdocument
%
\isadelimtheory
%
\endisadelimtheory
%
\isatagtheory
\isacommand{theory}\isamarkupfalse%
\ Aggregator\isanewline
\ \ \isakeyword{imports}\ {\isachardoublequoteopen}{\isachardot}{\kern0pt}{\isachardot}{\kern0pt}{\isacharslash}{\kern0pt}{\isachardot}{\kern0pt}{\isachardot}{\kern0pt}{\isacharslash}{\kern0pt}Social{\isacharunderscore}{\kern0pt}Choice{\isacharunderscore}{\kern0pt}Types{\isacharslash}{\kern0pt}Result{\isachardoublequoteclose}\isanewline
\isakeyword{begin}%
\endisatagtheory
{\isafoldtheory}%
%
\isadelimtheory
%
\endisadelimtheory
%
\begin{isamarkuptext}%
An aggregator gets two partitions (results of electoral modules) as input and
output another partition. They are used to aggregate results of parallel
composed electoral modules.
They are commutative, i.e., the order of the aggregated modules does not affect
the resulting aggregation. Moreover, they are conservative in the sense that
the resulting decisions are subsets of the two given partitions' decisions.%
\end{isamarkuptext}\isamarkuptrue%
%
\isadelimdocument
%
\endisadelimdocument
%
\isatagdocument
%
\isamarkupsubsection{Definition%
}
\isamarkuptrue%
%
\endisatagdocument
{\isafolddocument}%
%
\isadelimdocument
%
\endisadelimdocument
\isacommand{type{\isacharunderscore}{\kern0pt}synonym}\isamarkupfalse%
\ {\isacharprime}{\kern0pt}a\ Aggregator\ {\isacharequal}{\kern0pt}\ {\isachardoublequoteopen}{\isacharprime}{\kern0pt}a\ set\ {\isasymRightarrow}\ {\isacharprime}{\kern0pt}a\ Result\ {\isasymRightarrow}\ {\isacharprime}{\kern0pt}a\ Result\ {\isasymRightarrow}\ {\isacharprime}{\kern0pt}a\ Result{\isachardoublequoteclose}\isanewline
\isanewline
\isacommand{definition}\isamarkupfalse%
\ aggregator\ {\isacharcolon}{\kern0pt}{\isacharcolon}{\kern0pt}\ {\isachardoublequoteopen}{\isacharprime}{\kern0pt}a\ Aggregator\ {\isasymRightarrow}\ bool{\isachardoublequoteclose}\ \isakeyword{where}\isanewline
\ \ {\isachardoublequoteopen}aggregator\ agg\ {\isasymequiv}\isanewline
\ \ \ \ {\isasymforall}A\ e{\isadigit{1}}\ e{\isadigit{2}}\ d{\isadigit{1}}\ d{\isadigit{2}}\ r{\isadigit{1}}\ r{\isadigit{2}}{\isachardot}{\kern0pt}\isanewline
\ \ \ \ \ \ {\isacharparenleft}{\kern0pt}well{\isacharunderscore}{\kern0pt}formed\ A\ {\isacharparenleft}{\kern0pt}e{\isadigit{1}}{\isacharcomma}{\kern0pt}\ r{\isadigit{1}}{\isacharcomma}{\kern0pt}\ d{\isadigit{1}}{\isacharparenright}{\kern0pt}\ {\isasymand}\ well{\isacharunderscore}{\kern0pt}formed\ A\ {\isacharparenleft}{\kern0pt}e{\isadigit{2}}{\isacharcomma}{\kern0pt}\ r{\isadigit{2}}{\isacharcomma}{\kern0pt}\ d{\isadigit{2}}{\isacharparenright}{\kern0pt}{\isacharparenright}{\kern0pt}\ {\isasymlongrightarrow}\isanewline
\ \ \ \ \ \ well{\isacharunderscore}{\kern0pt}formed\ A\ {\isacharparenleft}{\kern0pt}agg\ A\ {\isacharparenleft}{\kern0pt}e{\isadigit{1}}{\isacharcomma}{\kern0pt}\ r{\isadigit{1}}{\isacharcomma}{\kern0pt}\ d{\isadigit{1}}{\isacharparenright}{\kern0pt}\ {\isacharparenleft}{\kern0pt}e{\isadigit{2}}{\isacharcomma}{\kern0pt}\ r{\isadigit{2}}{\isacharcomma}{\kern0pt}\ d{\isadigit{2}}{\isacharparenright}{\kern0pt}{\isacharparenright}{\kern0pt}{\isachardoublequoteclose}%
\isadelimdocument
%
\endisadelimdocument
%
\isatagdocument
%
\isamarkupsubsection{Properties%
}
\isamarkuptrue%
%
\endisatagdocument
{\isafolddocument}%
%
\isadelimdocument
%
\endisadelimdocument
%
\isadelimtheory
%
\endisadelimtheory
%
\isatagtheory
\isacommand{end}\isamarkupfalse%
%
\endisatagtheory
{\isafoldtheory}%
%
\isadelimtheory
%
\endisadelimtheory
%
\end{isabellebody}%
\endinput
%:%file=~/Documents/Studies/VotingRuleGenerator/virage/src/test/resources/verifiedVotingRuleConstruction/theories/Compositional_Framework/Components/Aggregator.thy%:%
%:%6=3%:%
%:%11=4%:%
%:%12=5%:%
%:%13=6%:%
%:%15=9%:%
%:%31=11%:%
%:%32=11%:%
%:%33=12%:%
%:%34=13%:%
%:%43=16%:%
%:%44=17%:%
%:%45=18%:%
%:%46=19%:%
%:%47=20%:%
%:%48=21%:%
%:%57=23%:%
%:%67=25%:%
%:%68=25%:%
%:%69=26%:%
%:%70=27%:%
%:%71=27%:%
%:%72=28%:%
%:%82=33%:%
%:%98=36%:%