%
\begin{isabellebody}%
\setisabellecontext{Elect{\isacharunderscore}{\kern0pt}Module}%
%
\isadelimdocument
\isanewline
%
\endisadelimdocument
%
\isatagdocument
\isanewline
\isanewline
%
\isamarkupsection{Elect Module%
}
\isamarkuptrue%
%
\endisatagdocument
{\isafolddocument}%
%
\isadelimdocument
%
\endisadelimdocument
%
\isadelimtheory
%
\endisadelimtheory
%
\isatagtheory
\isacommand{theory}\isamarkupfalse%
\ Elect{\isacharunderscore}{\kern0pt}Module\isanewline
\ \ \isakeyword{imports}\ {\isachardoublequoteopen}{\isachardot}{\kern0pt}{\isachardot}{\kern0pt}{\isacharslash}{\kern0pt}Electoral{\isacharunderscore}{\kern0pt}Module{\isachardoublequoteclose}\isanewline
\isakeyword{begin}%
\endisatagtheory
{\isafoldtheory}%
%
\isadelimtheory
%
\endisadelimtheory
%
\begin{isamarkuptext}%
The elect module is not concerned about the voter's ballots, and
just elects all alternatives. It is primarily used in sequence after
an electoral module that only defers alternatives to finalize the decision,
thereby inducing a proper voting rule in the social choice sense.%
\end{isamarkuptext}\isamarkuptrue%
%
\isadelimdocument
%
\endisadelimdocument
%
\isatagdocument
%
\isamarkupsubsection{Definition%
}
\isamarkuptrue%
%
\endisatagdocument
{\isafolddocument}%
%
\isadelimdocument
%
\endisadelimdocument
\isacommand{fun}\isamarkupfalse%
\ elect{\isacharunderscore}{\kern0pt}module\ {\isacharcolon}{\kern0pt}{\isacharcolon}{\kern0pt}\ {\isachardoublequoteopen}{\isacharprime}{\kern0pt}a\ Electoral{\isacharunderscore}{\kern0pt}Module{\isachardoublequoteclose}\ \isakeyword{where}\isanewline
\ \ {\isachardoublequoteopen}elect{\isacharunderscore}{\kern0pt}module\ A\ p\ {\isacharequal}{\kern0pt}\ {\isacharparenleft}{\kern0pt}A{\isacharcomma}{\kern0pt}\ {\isacharbraceleft}{\kern0pt}{\isacharbraceright}{\kern0pt}{\isacharcomma}{\kern0pt}\ {\isacharbraceleft}{\kern0pt}{\isacharbraceright}{\kern0pt}{\isacharparenright}{\kern0pt}{\isachardoublequoteclose}%
\isadelimdocument
%
\endisadelimdocument
%
\isatagdocument
%
\isamarkupsubsection{Soundness%
}
\isamarkuptrue%
%
\endisatagdocument
{\isafolddocument}%
%
\isadelimdocument
%
\endisadelimdocument
\isacommand{theorem}\isamarkupfalse%
\ elect{\isacharunderscore}{\kern0pt}mod{\isacharunderscore}{\kern0pt}sound{\isacharbrackleft}{\kern0pt}simp{\isacharbrackright}{\kern0pt}{\isacharcolon}{\kern0pt}\ {\isachardoublequoteopen}electoral{\isacharunderscore}{\kern0pt}module\ elect{\isacharunderscore}{\kern0pt}module{\isachardoublequoteclose}\isanewline
%
\isadelimproof
\ \ %
\endisadelimproof
%
\isatagproof
\isacommand{unfolding}\isamarkupfalse%
\ electoral{\isacharunderscore}{\kern0pt}module{\isacharunderscore}{\kern0pt}def\isanewline
\ \ \isacommand{by}\isamarkupfalse%
\ simp%
\endisatagproof
{\isafoldproof}%
%
\isadelimproof
\isanewline
%
\endisadelimproof
%
\isadelimtheory
\isanewline
%
\endisadelimtheory
%
\isatagtheory
\isacommand{end}\isamarkupfalse%
%
\endisatagtheory
{\isafoldtheory}%
%
\isadelimtheory
%
\endisadelimtheory
%
\end{isabellebody}%
\endinput
%:%file=~/Documents/Studies/VotingRuleGenerator/virage/src/test/resources/verifiedVotingRuleConstruction/theories/Compositional_Framework/Components/Basic_Modules/Elect_Module.thy%:%
%:%6=3%:%
%:%11=4%:%
%:%12=5%:%
%:%14=8%:%
%:%30=10%:%
%:%31=10%:%
%:%32=11%:%
%:%33=12%:%
%:%42=15%:%
%:%43=16%:%
%:%44=17%:%
%:%45=18%:%
%:%54=20%:%
%:%64=22%:%
%:%65=22%:%
%:%66=23%:%
%:%73=25%:%
%:%83=27%:%
%:%84=27%:%
%:%87=28%:%
%:%91=28%:%
%:%92=28%:%
%:%93=29%:%
%:%94=29%:%
%:%99=29%:%
%:%104=30%:%
%:%109=31%:%